\section{C Examples Book LaTeX/book.tex}
\section{C Examples Book LaTeX/book/snippets.tex}
\section{Combined Examples Book LaTeX/book/snippets/chapters.tex}
\section{Examples Book LaTeX/book/snippets/chapters/outlines.tex}
\title{sitandr.github.io/typst-examples-book/book/snippets/chapters/outlines}

\section{\texorpdfstring{\hyperref[outlines]{Outlines}}{Outlines}}\label{outlines}

\section{\texorpdfstring{\hyperref[outlines-1]{Outlines}}{Outlines}}\label{outlines-1}

\begin{quote}
Lots of outlines examples are already available in
\href{https://typst.app/docs/reference/meta/outline/}{official
reference}
\end{quote}

\subsection{\texorpdfstring{\hyperref[table-of-contents]{Table of
contents}}{Table of contents}}\label{table-of-contents}

\begin{verbatim}
#outline()

= Introduction
#lorem(5)

= Prior work
#lorem(10)
\end{verbatim}

\pandocbounded{\includesvg[keepaspectratio]{typst-img/77dbcfc8072afc53714fea404efaa1f60692fee68a19821e77feb8bafdd5bd46-1.svg}}

\subsection{\texorpdfstring{\hyperref[outline-of-figures]{Outline of
figures}}{Outline of figures}}\label{outline-of-figures}

\begin{verbatim}
#outline(
  title: [List of Figures],
  target: figure.where(kind: table),
)

#figure(
  table(
    columns: 4,
    [t], [1], [2], [3],
    [y], [0.3], [0.7], [0.5],
  ),
  caption: [Experiment results],
)
\end{verbatim}

\pandocbounded{\includesvg[keepaspectratio]{typst-img/a898ed56e04bc374a8b88580ae203c7695d92445179cffad2642d1a08a8f7ef1-1.svg}}

You can use arbitrary selector there, so you can do any crazy things.

\subsection{\texorpdfstring{\hyperref[ignore-low-level-headings]{Ignore
low-level
headings}}{Ignore low-level headings}}\label{ignore-low-level-headings}

\begin{verbatim}
#set heading(numbering: "1.")
#outline(depth: 2)

= Yes
Top-level section.

== Still
Subsection.

=== Nope
Not included.
\end{verbatim}

\pandocbounded{\includesvg[keepaspectratio]{typst-img/c6947016b324ba83db8aea6da98d4658877618b4ee650edabdb2360782fd520c-1.svg}}

\subsection{\texorpdfstring{\hyperref[set-indentation]{Set
indentation}}{Set indentation}}\label{set-indentation}

\begin{verbatim}
#set heading(numbering: "1.a.")

#outline(
  title: [Contents (Automatic)],
  indent: auto,
)

#outline(
  title: [Contents (Length)],
  indent: 2em,
)

#outline(
  title: [Contents (Function)],
  indent: n => [→ ] * n,
)

= About ACME Corp.
== History
=== Origins
#lorem(10)

== Products
#lorem(10)
\end{verbatim}

\pandocbounded{\includesvg[keepaspectratio]{typst-img/59dc3acb28c16d258b93278079848404454449450103de6f456454aef50a8e55-1.svg}}

\subsection{\texorpdfstring{\hyperref[replace-default-dots]{Replace
default dots}}{Replace default dots}}\label{replace-default-dots}

\begin{verbatim}
#outline(fill: line(length: 100%), indent: 2em)

= First level
== Second level
\end{verbatim}

\pandocbounded{\includesvg[keepaspectratio]{typst-img/81c9f957fb8944561295980a7dfd1ee2b1fbd58f77d90e7b904aa8b99b3bbf9e-1.svg}}

\subsection{\texorpdfstring{\hyperref[make-different-outline-levels-look-different]{Make
different outline levels look
different}}{Make different outline levels look different}}\label{make-different-outline-levels-look-different}

\begin{verbatim}
#set heading(numbering: "1.")

#show outline.entry.where(
  level: 1
): it => {
  v(12pt, weak: true)
  strong(it)
}

#outline(indent: auto)

= Introduction
= Background
== History
== State of the Art
= Analysis
== Setup
\end{verbatim}

\pandocbounded{\includesvg[keepaspectratio]{typst-img/e620be5254ab48d4bb3f5e1b8bc928e1f8c43d0ba0929b6dc858522539ff4e0c-1.svg}}

\subsection{\texorpdfstring{\hyperref[long-and-short-captions-for-the-outline]{Long
and short captions for the
outline}}{Long and short captions for the outline}}\label{long-and-short-captions-for-the-outline}

\begin{verbatim}
// author: laurmaedje
// Put this somewhere in your template or at the start of your document.
#let in-outline = state("in-outline", false)
#show outline: it => {
  in-outline.update(true)
  it
  in-outline.update(false)
}

#let flex-caption(long, short) = context if in-outline.get() { short } else { long }

// And this is in the document.
#outline(title: [Figures], target: figure)

#figure(
  rect(),
  caption: flex-caption(
    [This is my long caption text in the document.],
    [This is short],
  )
)
\end{verbatim}

\pandocbounded{\includesvg[keepaspectratio]{typst-img/fc4dc1c50f173f2ee6d73ee8868e6a8cd3d4a666165d2d05d21cfaa410361e31-1.svg}}

\subsection{\texorpdfstring{\hyperref[ignore-citations-and-footnotes]{Ignore
citations and
footnotes}}{Ignore citations and footnotes}}\label{ignore-citations-and-footnotes}

That\textquotesingle s a workaround a problem that if you make a
footnote a heading, its number will be displayed in outline:

\begin{verbatim}
= Heading #footnote[A footnote]

Text

#outline() // bad :(

#pagebreak()
#{
  set footnote.entry(
    separator: none
  )
  show footnote.entry: hide
  show ref: none
  show footnote: none

  outline()
}
\end{verbatim}

\pandocbounded{\includesvg[keepaspectratio]{typst-img/baac264bb9ab5bfcf402ee4508cb3c18a8e747b17fefa125c9d2998df0f5a283-1.svg}}

\pandocbounded{\includesvg[keepaspectratio]{typst-img/baac264bb9ab5bfcf402ee4508cb3c18a8e747b17fefa125c9d2998df0f5a283-2.svg}}


\section{Examples Book LaTeX/book/snippets/chapters/page-numbering.tex}
\title{sitandr.github.io/typst-examples-book/book/snippets/chapters/page-numbering}

\section{\texorpdfstring{\hyperref[page-numbering]{Page
numbering}}{Page numbering}}\label{page-numbering}

\subsection{\texorpdfstring{\hyperref[separate-page-numbering-for-each-chapter]{Separate
page numbering for each
chapter}}{Separate page numbering for each chapter}}\label{separate-page-numbering-for-each-chapter}

\begin{verbatim}
/// author: tinger

// unnumbered title page if needed
// ...

// front-matter
#set page(numbering: "I")
#counter(page).update(1)
#lorem(50)
// ...

// page counter anchor
#metadata(()) <front-matter>

// main document body
#set page(numbering: "1")
#lorem(50)
#counter(page).update(1)
// ...

// back-matter
#set page(numbering: "I")
// must take page breaks into account, may need to be offset by +1 or -1
#context counter(page).update(counter(page).at(<front-matter>).first())
#lorem(50)
// ...
\end{verbatim}

\pandocbounded{\includesvg[keepaspectratio]{typst-img/0cd153b35bf7532971dbbb220095812665f44b0ab9cca9d7a8c6c000f83e3e30-1.svg}}

\pandocbounded{\includesvg[keepaspectratio]{typst-img/0cd153b35bf7532971dbbb220095812665f44b0ab9cca9d7a8c6c000f83e3e30-2.svg}}

\pandocbounded{\includesvg[keepaspectratio]{typst-img/0cd153b35bf7532971dbbb220095812665f44b0ab9cca9d7a8c6c000f83e3e30-3.svg}}




\section{Combined Examples Book LaTeX/book/snippets/snippets.tex}
\section{Examples Book LaTeX/book/snippets/index.tex}
\title{sitandr.github.io/typst-examples-book/book/snippets/index}

\section{\texorpdfstring{\hyperref[typst-snippets]{Typst
Snippets}}{Typst Snippets}}\label{typst-snippets}

Useful snippets for common (and not) tasks.


\section{Examples Book LaTeX/book/snippets/numbering.tex}
\title{sitandr.github.io/typst-examples-book/book/snippets/numbering}

\section{\texorpdfstring{\hyperref[numbering]{Numbering}}{Numbering}}\label{numbering}

\subsection{\texorpdfstring{\hyperref[individual-heading-without-numbering]{Individual
heading without
numbering}}{Individual heading without numbering}}\label{individual-heading-without-numbering}

\begin{verbatim}
#let numless(it) = {set heading(numbering: none); it }

= Heading
#numless[=No numbering heading]
\end{verbatim}

\pandocbounded{\includesvg[keepaspectratio]{typst-img/e04f844b270049702ac72dff7bfadf5963cdb2bc8a541e81b685124fbb61c48e-1.svg}}

\subsection{\texorpdfstring{\hyperref[clean-numbering]{"Clean"
numbering}}{"Clean" numbering}}\label{clean-numbering}

\begin{verbatim}
// original author: tromboneher

// Number sections according to a number of schemes, omitting previous leading elements.
// For example, where the numbering pattern "A.I.1." would produce:
//
// A. A part of the story
//   A.I. A chapter
//   A.II. Another chapter
//     A.II.1. A section
//       A.II.1.a. A subsection
//       A.II.1.b. Another subsection
//     A.II.2. Another section
// B. Another part of the story
//   B.I. A chapter in the second part
//   B.II. Another chapter in the second part
//
// clean_numbering("A.", "I.", "1.a.") would produce:
//
// A. A part of the story
//   I. A chapter
//   II. Another chapter
//     1. A section
//       1.a. A subsection
//       1.b. Another subsection
//     2. Another section
// B. Another part of the story
//   I. A chapter in the second part
//   II. Another chapter in the second part
//
#let clean_numbering(..schemes) = {
  (..nums) => {
    let (section, ..subsections) = nums.pos()
    let (section_scheme, ..subschemes) = schemes.pos()

    if subsections.len() == 0 {
      numbering(section_scheme, section)
    } else if subschemes.len() == 0 {
      numbering(section_scheme, ..nums.pos())
    }
    else {
      clean_numbering(..subschemes)(..subsections)
    }
  }
}

#set heading(numbering: clean_numbering("A.", "I.", "1.a."))

= Part
== Chapter
== Another chapter
=== Section
==== Subsection
==== Another subsection
= Another part of the story
== A chapter in the second part
== Another chapter in the second part
\end{verbatim}

\pandocbounded{\includesvg[keepaspectratio]{typst-img/4e29319442704545bf58d12448745836598c12f59162d3199aaad21c752e4483-1.svg}}

\subsection{\texorpdfstring{\hyperref[math-numbering]{Math
numbering}}{Math numbering}}\label{math-numbering}

See \href{./math/numbering.html}{there} .

\subsection{\texorpdfstring{\hyperref[numbering-each-paragraph]{Numbering
each
paragraph}}{Numbering each paragraph}}\label{numbering-each-paragraph}

By the 0.12 version of Typst, this should be replaced with good native
solution.

\begin{verbatim}
// original author: roehlichA
// Legal formatting of enumeration
#show enum: it => context {
  // Retrieve the last heading so we know what level to step at
  let headings = query(selector(heading).before(here()))
  let last = headings.at(-1)

  // Combine the output items
  let output = ()
  for item in it.children {
    output.push([
      #context{
        counter(heading).step(level: last.level + 1)
      }
      #context {
        counter(heading).display() 
      }
    ])
    output.push([
      #text(item.body)
      #parbreak()
    ])
  }

  // Display in a grid
  grid(
    columns: (auto, 1fr),
    column-gutter: 1em,
    row-gutter: 1em,
    ..output
  )

}

#set heading(numbering: "1.")

= Some heading
+ Paragraph
= Other
+ Paragraphs here are preceded with a number so they can be referenced directly.
+ _#lorem(100)_
+ _#lorem(100)_

== A subheading
+ Paragraphs are also numbered correctly in subheadings.
+ _#lorem(50)_
+ _#lorem(50)_
\end{verbatim}

\pandocbounded{\includesvg[keepaspectratio]{typst-img/8d5603f93334c1d0fd7391811f90b161d4ff8c7eb81100dc152caac5c6d13daf-1.svg}}


\section{Examples Book LaTeX/book/snippets/demos.tex}
\title{sitandr.github.io/typst-examples-book/book/snippets/demos}

\section{\texorpdfstring{\hyperref[demos]{Demos}}{Demos}}\label{demos}

\subsection{\texorpdfstring{\hyperref[resume-using-template]{Resume
(using
template)}}{Resume (using template)}}\label{resume-using-template}

\begin{verbatim}
#import "@preview/modern-cv:0.1.0": *

#show: resume.with(
  author: (
      firstname: "John", 
      lastname: "Smith",
      email: "js@example.com", 
      phone: "(+1) 111-111-1111",
      github: "DeveloperPaul123",
      linkedin: "Example",
      address: "111 Example St. Example City, EX 11111",
      positions: (
        "Software Engineer",
        "Software Architect"
      )
  ),
  date: datetime.today().display()
)

= Education

#resume-entry(
  title: "Example University",
  location: "B.S. in Computer Science",
  date: "August 2014 - May 2019",
  description: "Example"
)

#resume-item[
  - #lorem(20)
  - #lorem(15)
  - #lorem(25)
]
\end{verbatim}

\pandocbounded{\includesvg[keepaspectratio]{typst-img/fc69693c49a6cf8021751980642ed7649c9d905056f510fb8e4a994937faeaa2-1.svg}}

\subsection{\texorpdfstring{\hyperref[book-cover]{Book
cover}}{Book cover}}\label{book-cover}

\begin{verbatim}
// author: bamdone
#let accent  = rgb("#00A98F")
#let accent1 = rgb("#98FFB3")
#let accent2 = rgb("#D1FF94")
#let accent3 = rgb("#D3D3D3")
#let accent4 = rgb("#ADD8E6")
#let accent5 = rgb("#FFFFCC")
#let accent6 = rgb("#F5F5DC")

#set page(paper: "a4",margin: 0.0in, fill: accent)

#set rect(stroke: 4pt)
#move(
  dx: -6cm, dy: 1.0cm,
  rotate(-45deg,
    rect(
      width: 100cm,
      height: 2cm,
      radius: 50%,
      stroke: 0pt,
      fill:accent1,
)))

#set rect(stroke: 4pt)
#move(
  dx: -2cm, dy: -1.0cm,
  rotate(-45deg,
    rect(
      width: 100cm,
      height: 2cm,
      radius: 50%,
      stroke: 0pt,
      fill:accent2,
)))

#set rect(stroke: 4pt)
#move(
  dx: 8cm, dy: -10cm,
  rotate(-45deg,
    rect(
      width: 100cm,
      height: 1cm,
      radius: 50%,
      stroke: 0pt,
      fill:accent3,
)))

#set rect(stroke: 4pt)
#move(
  dx: 7cm, dy: -8cm,
  rotate(-45deg,
    rect(
      width: 1000cm,
      height: 2cm,
      radius: 50%,
      stroke: 0pt,
      fill:accent4,
)))

#set rect(stroke: 4pt)
#move(
  dx: 0cm, dy: -0cm,
  rotate(-45deg,
    rect(
      width: 1000cm,
      height: 2cm,
      radius: 50%,
      stroke: 0pt,
      fill:accent1,
)))

#set rect(stroke: 4pt)
#move(
  dx: 9cm, dy: -7cm,
  rotate(-45deg,
    rect(
      width: 1000cm,
      height: 1.5cm,
      radius: 50%,
      stroke: 0pt,
      fill:accent6,
)))

#set rect(stroke: 4pt)
#move(
  dx: 16cm, dy: -13cm,
  rotate(-45deg,
    rect(
      width: 1000cm,
      height: 1cm,
      radius: 50%,
      stroke: 0pt,
      fill:accent2,
)))

#align(center)[
  #rect(width: 30%,
    fill: accent4,
    stroke:none,
    [#align(center)[
      #text(size: 60pt,[Title])
    ]
    ])
]

#align(center)[
  #rect(width: 30%,
    fill: accent4,
    stroke:none,
    [#align(center)[
      #text(size: 20pt,[author])
    ]
    ])
]
\end{verbatim}

\pandocbounded{\includesvg[keepaspectratio]{typst-img/7c2e798dacec8ac970ac2b328c60f8145441d059f16f7bd193f389d78d121981-1.svg}}


\section{Examples Book LaTeX/book/snippets/pretty.tex}
\title{sitandr.github.io/typst-examples-book/book/snippets/pretty}

\section{\texorpdfstring{\hyperref[pretty-things]{Pretty
things}}{Pretty things}}\label{pretty-things}

\subsection{\texorpdfstring{\hyperref[set-bar-to-the-texts-left]{Set bar
to the text\textquotesingle s
left}}{Set bar to the text\textquotesingle s left}}\label{set-bar-to-the-texts-left}

(also known as quote formatting)

\begin{verbatim}
#let line-block = rect.with(fill: luma(240), stroke: (left: 0.25em))

+ #lorem(10) \
  #line-block[
    *Solution:* #lorem(10)

    $ a_(n+1)x^n = 2... $
  ]
\end{verbatim}

\pandocbounded{\includesvg[keepaspectratio]{typst-img/fcddd92f117eeeb99d7b422dfc0c20a254e163e09fc5b80251a088771792ff5a-1.svg}}

\subsection{\texorpdfstring{\hyperref[text-on-box-top]{Text on box
top}}{Text on box top}}\label{text-on-box-top}

\begin{verbatim}
// author: gaiajack
#let todo(body) = block(
  above: 2em, stroke: 0.5pt + red,
  width: 100%, inset: 14pt
)[
  #set text(font: "Noto Sans", fill: red)
  #place(
    top + left,
    dy: -6pt - 14pt, // Account for inset of block
    dx: 6pt - 14pt,
    block(fill: white, inset: 2pt)[*DRAFT*]
  )
  #body
]

#todo(lorem(100))
\end{verbatim}

\pandocbounded{\includesvg[keepaspectratio]{typst-img/7a5d79c63f3a0b28ec6bdec78da80d81252ff1975b883162c84b813f938c94c0-1.svg}}

\subsection{\texorpdfstring{\hyperref[book-ornament]{Book
Ornament}}{Book Ornament}}\label{book-ornament}

\begin{verbatim}
// author: thevec

#let parSepOrnament = [\ \ #h(1fr) $#line(start:(0em,-.15em), end:(12em,-.15em), stroke: (cap: "round", paint:gradient.linear(white,black,white))) #move(dx:.5em,dy:0em,"🙠")#text(15pt)[🙣] #h(0.4em) #move(dy:-0.25em,text(12pt)[✢]) #h(0.4em) #text(15pt)[🙡]#move(dx:-.5em,dy:0em,"🙢") #line(start:(0em,-.15em), end:(12em,-.15em), stroke: (cap: "round", paint:gradient.linear(white,black,white)))$ #h(1fr)\ \ ];

#lorem(30)
#parSepOrnament
#lorem(30)
\end{verbatim}

\pandocbounded{\includesvg[keepaspectratio]{typst-img/ad56a859952fab3706dcb76434e492a9c14057bff1ee897ae2bfe3672fe17e18-1.svg}}


\section{Examples Book LaTeX/book/snippets/external.tex}
\title{sitandr.github.io/typst-examples-book/book/snippets/external}

\section{\texorpdfstring{\hyperref[use-with-external-tools]{Use with
external
tools}}{Use with external tools}}\label{use-with-external-tools}

Currently the best ways to communicate is using

\begin{enumerate}
\tightlist
\item
  Preprocessing. The tool should generate Typst file
\item
  Typst Query (CLI). See the docs
  \href{https://typst.app/docs/reference/meta/query\#command-line-queries}{there}
  .
\item
  WebAssembly plugins. See the docs
  \href{https://typst.app/docs/reference/foundations/plugin/}{there} .
\end{enumerate}

In some time there will be examples of successful usage of first two
methods. For the third one, see \href{../packages/index.html}{packages}
.


\section{Examples Book LaTeX/book/snippets/code.tex}
\title{sitandr.github.io/typst-examples-book/book/snippets/code}

\section{\texorpdfstring{\hyperref[code-formatting]{Code
formatting}}{Code formatting}}\label{code-formatting}

\subsection{\texorpdfstring{\hyperref[inline-highlighting]{Inline
highlighting}}{Inline highlighting}}\label{inline-highlighting}

\begin{verbatim}
#let r = raw.with(lang: "r")

This can then be used like: #r("x <- c(10, 42)")
\end{verbatim}

\pandocbounded{\includesvg[keepaspectratio]{typst-img/dadb41acb1c458d9af5b909d657de5f46dca019f1a81cc17b75b9863d60fa9eb-1.svg}}

\subsection{\texorpdfstring{\hyperref[tab-size]{Tab
size}}{Tab size}}\label{tab-size}

\begin{verbatim}
#set raw(tab-size: 8)
```tsv
Year  Month   Day
2000  2   3
2001  2   1
2002  3   10
```
\end{verbatim}

\pandocbounded{\includesvg[keepaspectratio]{typst-img/1c3900a79521f6b9cd852a68a3dea627ddbd1b8fc6062d3ca344e4259a30d212-1.svg}}

\subsection{\texorpdfstring{\hyperref[theme]{Theme}}{Theme}}\label{theme}

See
\href{https://typst.app/docs/reference/text/raw/\#parameters-theme}{reference}

\subsection{\texorpdfstring{\hyperref[enable-ligatures-for-code]{Enable
ligatures for
code}}{Enable ligatures for code}}\label{enable-ligatures-for-code}

\begin{verbatim}
#show raw: set text(ligatures: true, font: "Cascadia Code")

Then the code becomes `x <- a`
\end{verbatim}

\pandocbounded{\includesvg[keepaspectratio]{typst-img/3513eee2f5ca33825d09149f4ad9169abf95d3b8ad02cc5a7bf91cc9b96517d0-1.svg}}

\subsection{\texorpdfstring{\hyperref[advanced-formatting]{Advanced
formatting}}{Advanced formatting}}\label{advanced-formatting}

See \href{../packages/code.html}{packages} section.


\section{Examples Book LaTeX/book/snippets/labels.tex}
\title{sitandr.github.io/typst-examples-book/book/snippets/labels}

\section{\texorpdfstring{\hyperref[labels]{Labels}}{Labels}}\label{labels}

\subsection{\texorpdfstring{\hyperref[get-chapter-of-label]{Get chapter
of label}}{Get chapter of label}}\label{get-chapter-of-label}

\begin{verbatim}
#let ref-heading(label) = context {
  let elems = query(label)
  if elems.len() != 1 {
    panic("found multiple elements")
  }
  let element = elems.first()
  if element.func() != heading {
    panic("label must target heading")
  }
  link(label, element.body)
}

= Design <design>
#lorem(20)

= Implementation
In #ref-heading(<design>), we discussed...
\end{verbatim}

\pandocbounded{\includesvg[keepaspectratio]{typst-img/7a4a9436d9aa0cbf0d3212b45d54bd90a896181c30b036326d99dee9f58eb117-1.svg}}

\subsection{\texorpdfstring{\hyperref[allow-missing-references]{Allow
missing
references}}{Allow missing references}}\label{allow-missing-references}

\begin{verbatim}
// author: Enivex
#set heading(numbering: "1.")

#let myref(label) = context {
    if query(label).len() != 0 {
        ref(label)
    } else {
        // missing reference
        text(fill: red)[???]
    }
}

= Second <test2>

#myref(<test>)

#myref(<test2>)
\end{verbatim}

\pandocbounded{\includesvg[keepaspectratio]{typst-img/cd5f1f34e81c117063da3bb176c1dda726bbc18ac981121f75555d5834b08058-1.svg}}


\section{Examples Book LaTeX/book/snippets/gradients.tex}
\title{sitandr.github.io/typst-examples-book/book/snippets/gradients}

\section{\texorpdfstring{\hyperref[color--gradients]{Color \&
Gradients}}{Color \& Gradients}}\label{color--gradients}

\subsection{\texorpdfstring{\hyperref[gradients]{Gradients}}{Gradients}}\label{gradients}

Gradients may be very cool for presentations or just a pretty look.

\begin{verbatim}
/// author: frozolotl
#set page(paper: "presentation-16-9", margin: 0pt)
#set text(fill: white, font: "Inter")

#let grad = gradient.linear(rgb("#953afa"), rgb("#c61a22"), angle: 135deg)

#place(horizon + left, image(width: 60%, "../img/landscape.png"))

#place(top, polygon(
  (0%, 0%),
  (70%, 0%),
  (70%, 25%),
  (0%, 29%),
  fill: white,
))
#place(bottom, polygon(
  (0%, 100%),
  (100%, 100%),
  (100%, 30%),
  (60%, 30% + 60% * 4%),
  (60%, 60%),
  (0%, 64%),
  fill: grad,
))

#place(top + right, block(inset: 7pt, image(height: 19%, "../img/tub.png")))

#place(bottom, block(inset: 40pt)[
  #text(size: 30pt)[
    Presentation Title
  ]

  #text(size: 16pt)[#lorem(20) | #datetime.today().display()]
])
\end{verbatim}

\pandocbounded{\includesvg[keepaspectratio]{typst-img/19558fb55cdd5c8f8193724d92b54b1258ca8ee3d4d4c9e7077fc50d11e1a79d-1.svg}}


\section{Examples Book LaTeX/book/snippets/grids.tex}
\title{sitandr.github.io/typst-examples-book/book/snippets/grids}

\subsection{\texorpdfstring{\hyperref[fractional-grids]{Fractional
grids}}{Fractional grids}}\label{fractional-grids}

For tables with lines of changing length, you can try using \emph{grids
in grids} .

Don\textquotesingle t use this where
\href{https://typst.app/docs/reference/model/table/\#definitions-cell-colspan}{cell.colspan
and rowspan} will do.

\begin{verbatim}
// author: jimpjorps

#grid(
  columns: (1fr,),
  grid(
    columns: (1fr,)*2, inset: 5pt, stroke: 1pt, [hello], [world]
  ),
  grid(
    columns: (1fr,)*3, inset: 5pt, stroke: 1pt, [foo], [bar], [baz]
  ),
  grid.cell(inset: 5pt, stroke: 1pt)[abcxyz]
)
\end{verbatim}

\pandocbounded{\includesvg[keepaspectratio]{typst-img/5b2869a2b2efca1af57cb7ed6fab90ad0c83a35b76c05258a1ae64096d5a8173-1.svg}}

\subsection{\texorpdfstring{\hyperref[automerge-adjacent-cells-with-same-values]{Automerge
adjacent cells with same
values}}{Automerge adjacent cells with same values}}\label{automerge-adjacent-cells-with-same-values}

This example works for adjacent cells horizontally, but
it\textquotesingle s not hard to upgrade it to columns too.

\begin{verbatim}
// author: tebine
#let merge(children, n-cols) = {
  let rows = children.chunks(n-cols)
  let new-children = ()
  for r in rows {
    // First group starts at index 0
    let i = 0 
    // Search next group
    while i < r.len() {
      // Group starts with one cell
      let c = r.at(i).body
      let n = 1
      for j in range(i+1, r.len()) {
        let c-next = r.at(j).body
        if c-next == c {
          // Add cell to group
          n += 1
        } else {
          break
        }
      }
      // Group is finished
      new-children.push(table.cell(colspan: n, c))
      i += n
    }
  }
  return new-children
}
#show table: it => {
  let merged = merge(it.children, it.columns.len())
  if it.children.len() == merged.len() { // trick to avoid recursion
    return it
  }
  table(columns: it.columns.len(), ..merged)
}
#table(columns: 2,
  [1], [2],
  [3], [3],
  [4], [5],
)
\end{verbatim}

\pandocbounded{\includesvg[keepaspectratio]{typst-img/5bf649017afba6f1af8a5ae7e6a1e8b614def90749a092f92e5886a58b351205-1.svg}}

\subsection{\texorpdfstring{\hyperref[slanted-column-headers-with-slanted-borders]{Slanted
column headers with slanted
borders}}{Slanted column headers with slanted borders}}\label{slanted-column-headers-with-slanted-borders}

\begin{verbatim}
// author: tebine
#let slanted(it, alpha: 45deg, len: 2.5cm) = layout(size => {
  let width = size.width
  let b = box(inset: 5pt, rotate(-alpha, reflow: true, it))
  let b-size = measure(b)
  let l = line(angle: -alpha, length: len)
  let l-width = len * calc.cos(alpha)
  let l-height = len * calc.sin(alpha)
  place(bottom+left, l)
  place(bottom+left, l, dx: width)
  place(bottom+left, line(length: width), dx: l-width, dy: -l-height)
  place(bottom+left, dx: width/2, b)
  box(height: l-height) // invisible box to set the height
})

#table(
  columns: 2,
  align: center,
  table.header(
    table.cell(stroke: none, inset: 0pt, slanted[*AAA*]),
    table.cell(stroke: none, inset: 0pt, slanted[*BBBBBB*]),
  ),
  [aaaaa], [bbbbbb], [c], [d],
)
\end{verbatim}

\pandocbounded{\includesvg[keepaspectratio]{typst-img/d5f49858e9acc4bad217904e87abb368aa5e38652bdcba27a971b3ddd10f0361-1.svg}}


\section{Examples Book LaTeX/book/snippets/logos.tex}
\title{sitandr.github.io/typst-examples-book/book/snippets/logos}

\section{\texorpdfstring{\hyperref[logos--figures]{Logos \&
Figures}}{Logos \& Figures}}\label{logos--figures}

Using SVG-s images is totally fine. Totally. But if you are lazy and
don\textquotesingle t want to search for images, here are some logos you
can just copy-paste in your document.

\textbf{Important} : \emph{Typst in text doesn\textquotesingle t need a
special writing (unlike LaTeX)} . Just write "Typst", maybe "
\textbf{Typst} ", and it is okay.

\subsection{\texorpdfstring{\hyperref[tex-and-latex]{TeX and
LaTeX}}{TeX and LaTeX}}\label{tex-and-latex}

\begin{verbatim}
#let TeX = {
  set text(font: "New Computer Modern", weight: "regular")
  box(width: 1.7em, {
    [T]
    place(top, dx: 0.56em, dy: 0.22em)[E]
    place(top, dx: 1.1em)[X]
  })
}

#let LaTeX = {
  set text(font: "New Computer Modern", weight: "regular")
  box(width: 2.55em, {
    [L]
    place(top, dx: 0.3em, text(size: 0.7em)[A])
    place(top, dx: 0.7em)[#TeX]
  })
}

Typst is not that hard to learn when you know #TeX and #LaTeX.
\end{verbatim}

\pandocbounded{\includesvg[keepaspectratio]{typst-img/9432efecd4502f681e3582d8581d0c325e0a89729d57b6d4bea732c2b9f476ec-1.svg}}

\subsection{\texorpdfstring{\hyperref[typst-guy]{Typst
guy}}{Typst guy}}\label{typst-guy}

\begin{verbatim}
// author: fenjalien
#import "@preview/cetz:0.1.2": *

#set page(width: auto, height: auto)

#canvas(length: 1pt, {
  import draw: *
  let color = rgb("239DAD")
  scale((y: -1))
  set-style(fill: color, stroke: none,)

  // body
  merge-path({
    bezier(
      (112.847, 134.007),
      (114.835, 143.178),
      (112.847, 138.562),
      (113.509, 141.619),
      name: "b"
    )
    bezier(
      "b.end",
      (122.063, 145.515),
      (116.16, 144.736),
      (118.569, 145.515),
      name: "b"
    )
    bezier(
      "b.end",
      (135.977, 140.121),
      (125.677, 145.515),
      (130.315, 143.717)
    )
    bezier(
      (139.591, 146.055),
      (113.389, 159.182),
      (128.99, 154.806),
      (120.256, 159.182),
      name: "b"
    )
    bezier(
      "b.end",
      (97.1258, 154.327),
      (106.522, 159.182),
      (101.101, 157.563),
      name: "b"
    )
    bezier(
      "b.end",
      (91.1626, 136.704),
      (93.1503, 150.97),
      (91.1626, 145.096),
      name: "b"
    )
    line(
      (rel: (0, -47.1126), to: "b.end"),
      (rel: (-9.0352, 0)),
      (80.6818, 82.9381),
      (91.1626, 79.7013),
      (rel: (0, -8.8112)),
      (112.847, 61),
      (rel: (0, 19.7802)),
      (134.17, 79.1618),
      (132.182, 90.8501),
      (112.847, 90.1309)
    )
  })

  // left pupil
  merge-path({
    bezier(
      (70.4667, 65.6833),
      (71.9727, 70.5068),
      (71.4946, 66.9075),
      (71.899, 69.4091)
    )
    bezier(
      (71.9727, 70.5068),
      (75.9104, 64.5912),
      (72.9675, 69.6715),
      (75.1477, 67.319)
    )
    bezier(
      (75.9104, 64.5912),
      (72.0556, 60.0005),
      (76.8638, 61.1815),
      (74.4045, 59.7677)
    )
    bezier(
      (72.0556, 60.0005),
      (66.833, 64.3859),
      (70.1766, 60.1867),
      (67.7909, 63.0017)
    )
    bezier(
      (66.833, 64.3859),
      (70.4667, 65.6833),
      (67.6159, 64.3083),
      (69.4388, 64.4591)
    )
  })

  // right pupil
  merge-path({
    bezier(
      (132.37, 61.668),
      (133.948, 66.7212),
      (133.447, 62.9505),
      (133.87, 65.5712)
    )
    bezier(
      (133.948, 66.7212),
      (138.073, 60.5239),
      (134.99, 65.8461),
      (137.274, 63.3815)
    )
    bezier(
      (138.073, 60.5239),
      (134.034, 55.7145),
      (139.066, 56.9513),
      (136.495, 55.4706)
    )
    bezier(
      (134.034, 55.7145),
      (128.563, 60.3087),
      (132.066, 55.9066),
      (129.567, 58.8586),
    )
    bezier(
      (128.563, 60.3087),
      (132.37, 61.668),
      (129.383, 60.2274),
      (131.293, 60.3855),
    )
  })

  set-style(
    stroke: (paint: rgb("239DAD"), thickness: 6pt, cap: "round"),
    fill: none,
  )

  // left eye
  merge-path({
    bezier(
      (58.5, 64.7273),
      (73.6136, 52),
      (58.5, 58.3636),
      (64.0682, 52.7955),
      name: "b"
    )
    bezier(
      "b.end",
      (84.75, 64.7273),
      (81.5682, 52),
      (84.75, 57.5682),
      name: "b"
    )
    bezier(
      "b.end",
      (71.2273, 76.6591),
      (84.75, 71.8864),
      (79.1818, 76.6591),
      name: "b"
    )
    bezier(
      "b.end",
      (58.5, 64.7273),
      (63.2727, 76.6591),
      (58.5, 71.0909)
    )
  })
  // eye lash
  line(
    (62.5, 55),
    (59.5, 52),
  )

  merge-path({
    bezier(
      (146.5, 61.043),
      (136.234, 49),
      (146.5, 52.7634),
      (141.367, 49)
    )
    bezier(
      (136.234, 49),
      (121.569, 62.5484),
      (125.969, 49),
      (120.836, 54.2688)
    )
    bezier(
      (121.569, 62.5484),
      (134.034, 72.3333),
      (122.302, 70.8279),
      (128.168, 72.3333)
    )
    bezier(
      (134.034, 72.3333),
      (146.5, 61.043),
      (139.901, 72.3333),
      (146.5, 69.3225)
    )
  })

  set-style(stroke: (thickness: 4pt))

  // right arm
  merge-path({
    bezier(
      (109.523, 115.614),
      (127.679, 110.918),
      (115.413, 115.3675),
      (122.283, 113.112)
    )
    bezier(
      (127.679, 110.918),
      (137, 106.591),
      (130.378, 109.821),
      (132.708, 108.739)
    )
  })

  // right first finger
  bezier(
    (137, 106.591),
    (140.5, 98.0908),
    (137.385, 102.891),
    (138.562, 99.817)
  )

  // right second finger
  bezier(
    (137, 106.591),
    (146, 101.591),
    (139.21, 103.799),
    (142.425, 101.713)
  )

  // right third finger
  line(
    (137, 106.591),
    (148, 106.591)
  )

  //right forth finger
  bezier(
    (137, 106.591),
    (146, 111.091),
    (140.243, 109.552),
    (143.119, 110.812)
  )

  // left arm
  bezier(
    (95.365, 116.979),
    (73.5, 107.591),
    (88.691, 115.549),
    (80.587, 112.887)
  )

  // left first finger
  line(
    (73.5, 107.591),
    (rel: (0, -9.5))
  )
  // left second finger
  line(
    (73.5, 107.591),
    (65.396, 100.824)
  )
  // left third finger
  line(
    (73.5, 107.591),
    (63.012, 105.839)
  )
  // left fourth finger
  bezier(
    (73.5, 107.591),
    (63.012, 111.04),
    (70.783, 109.121),
    (67.214, 111.255)
  )
})
\end{verbatim}

\pandocbounded{\includesvg[keepaspectratio]{typst-img/4a142b60394d5730a373a7ee2229a3a42a8af8f31b314c70b5bd192210982b09-1.svg}}




\section{Combined Examples Book LaTeX/book/snippets/scripting.tex}
\section{Examples Book LaTeX/book/snippets/scripting/index.tex}
\title{sitandr.github.io/typst-examples-book/book/snippets/scripting/index}

\section{\texorpdfstring{\hyperref[scripting]{Scripting}}{Scripting}}\label{scripting}

\subsection{\texorpdfstring{\hyperref[unflatten-arrays]{Unflatten
arrays}}{Unflatten arrays}}\label{unflatten-arrays}

\begin{verbatim}
// author: PgSuper
#let unflatten(arr, n) = {
  let columns = range(0, n).map(_ => ())
  for (i, x) in arr.enumerate() {
    columns.at(calc.rem(i, n)).push(x)
  }
  array.zip(..columns)
}

#unflatten((1, 2, 3, 4, 5, 6), 2)
#unflatten((1, 2, 3, 4, 5, 6), 3)
\end{verbatim}

\pandocbounded{\includesvg[keepaspectratio]{typst-img/98271a255f0fb10f31ba1d8199ba5a91ebb6f647cdd570220f95e1b88d193ca0-1.svg}}

\subsection{\texorpdfstring{\hyperref[create-an-abbreviation]{Create an
abbreviation}}{Create an abbreviation}}\label{create-an-abbreviation}

\begin{verbatim}
#let full-name = "Federal University of Ceará"

#let letts = {
  full-name
    .split()
    .map(word => word.at(0)) // filter only capital letters
    .filter(l => upper(l) == l)
    .join()
}
#letts
\end{verbatim}

\pandocbounded{\includesvg[keepaspectratio]{typst-img/e95b77243a1305a47517cb128577d1c7633d858561de0ef797ff551f35be40de-1.svg}}

\subsection{\texorpdfstring{\hyperref[split-the-string-retrieving-separators]{Split
the string retrieving
separators}}{Split the string retrieving separators}}\label{split-the-string-retrieving-separators}

\begin{verbatim}
#",this, is a a a a; a. test? string!".matches(regex("(\b[\P{Punct}\s]+\b|\p{Punct})")).map(x => x.captures).join()
\end{verbatim}

\pandocbounded{\includesvg[keepaspectratio]{typst-img/c5d183e45097449e4f52b07f82185847092ad28bcad3b9474093d341c4b07c4a-1.svg}}

\subsection{\texorpdfstring{\hyperref[create-selector-matching-any-values-in-an-array]{Create
selector matching any values in an
array}}{Create selector matching any values in an array}}\label{create-selector-matching-any-values-in-an-array}

This snippet creates a selector (that is then used in a show rule) that
matches any of the values inside the array. Here, it is used to
highlight a few raw lines, but it can be easily adapted to any kind of
selector.

\begin{verbatim}
// author: Blokyk
#let lines = (2, 3, 5)
#let lines-selectors = lines.map(lineno => raw.line.where(number: lineno))
#let lines-combined-selector = lines-selectors.fold(
  // start with the first selector by default
  // you can also use a selector that wouldn't ever match anything, if possible
  lines-selectors.at(0),
  selector.or // create an OR of all selectors (alternatively: (acc, sel) => acc.or(sel))
)

#show lines-combined-selector: highlight

```py
def foo(x, y):
  if x == y:
    return False
  z = x + y
  return z * x - z * y >= z
```
\end{verbatim}

\pandocbounded{\includesvg[keepaspectratio]{typst-img/085d1ae3a0672ba278edcde3ebb229a34a40ab5166d0b6d5b469d838b9262a51-1.svg}}

\subsection{\texorpdfstring{\hyperref[synthesize-show-or-show-set-rules-from-dictionnary]{Synthesize
show (or show-set) rules from
dictionnary}}{Synthesize show (or show-set) rules from dictionnary}}\label{synthesize-show-or-show-set-rules-from-dictionnary}

This snippet applies a show-set rule to any element inside a dictionary,
by using the key as the selector and the value as the parameter to set.
In this example, it\textquotesingle s used to give custom supplements to
custom figure kinds, based on a dictionnary of correspondances.

\begin{verbatim}
// author: laurmaedje
#let kind_supp_dict = (
  algo: "Pseudo-code",
  ex: "Example",
  prob: "Problem",
)

// apply this rule to the whole (rest of the) document
#show: it => {
  kind_supp_dict
    .pairs() // get an array of key-value pairs
    .fold( // we're going to stack show-set rules before the document
      it, // start with the default document
      (acc, (kind, supp)) => {
        // add the curent kind-supp combination on top of the rest
        show figure.where(kind: kind): set figure(supplement: supp)
        acc
      }
    )
}
#figure(
    kind: "algo",
    caption: [My code],
    ```Algorithm there```
)
\end{verbatim}

\pandocbounded{\includesvg[keepaspectratio]{typst-img/9de9b5f4bb801735b13ffafe54d35ebcfc78f1df78a34a8ab90f8a6c350b986e-1.svg}}

Additonnaly, as this is applied at the position where you write it,
these show-set rules will appear as if they were added in the same place
where you wrote this rule. This means that you can override them later,
just like any other show-set rules.




\section{Combined Examples Book LaTeX/book/snippets/text.tex}
\section{Examples Book LaTeX/book/snippets/text/individual_lang_fonts.tex}
\title{sitandr.github.io/typst-examples-book/book/snippets/text/individual_lang_fonts}

\section{\texorpdfstring{\hyperref[individual-language-fonts]{Individual
language
fonts}}{Individual language fonts}}\label{individual-language-fonts}

\begin{verbatim}
A cat แปลว่า แมว

#show regex("\p{Thai}+"): text.with(font: "Noto Serif Thai")

A cat แปลว่า แมว
\end{verbatim}

\pandocbounded{\includesvg[keepaspectratio]{typst-img/612267fd94fab114a3e0b75bdb3785b818c0f83427071db0dce086d1b0a6a54a-1.svg}}


\section{Examples Book LaTeX/book/snippets/text/text_shadows.tex}
\title{sitandr.github.io/typst-examples-book/book/snippets/text/text_shadows}

\section{\texorpdfstring{\hyperref[fake-italic--text-shadows]{Fake
italic \& Text
shadows}}{Fake italic \& Text shadows}}\label{fake-italic--text-shadows}

\subsection{\texorpdfstring{\hyperref[skew]{Skew}}{Skew}}\label{skew}

\begin{verbatim}
// author: Enivex
#set page(width: 21cm, height: 3cm)
#set text(size:25pt)
#let skew(angle,vscale: 1,body) = {
  let (a,b,c,d)= (1,vscale*calc.tan(angle),0,vscale)
  let E = (a + d)/2
  let F = (a - d)/2
  let G = (b + c)/2
  let H = (c - b)/2
  let Q = calc.sqrt(E*E + H*H)
  let R = calc.sqrt(F*F + G*G)
  let sx = Q + R
  let sy = Q - R
  let a1 = calc.atan2(F,G)
  let a2 = calc.atan2(E,H)
  let theta = (a2 - a1) /2
  let phi = (a2 + a1)/2

  set rotate(origin: bottom+center)
  set scale(origin: bottom+center)

  rotate(phi,scale(x: sx*100%, y: sy*100%,rotate(theta,body)))
}

#let fake-italic(body) = skew(-12deg,body)
#fake-italic[This is fake italic text]

#let shadowed(body) = box(place(skew(-50deg, vscale: 0.8, text(fill:luma(200),body)))+place(body))
#shadowed[This is some fancy text with a shadow]
\end{verbatim}

\pandocbounded{\includesvg[keepaspectratio]{typst-img/1c00de41a0643ecf254de80601efa4a043302c1e76aedfbf2458a9e30f1c7fd3-1.svg}}




\section{Combined Examples Book LaTeX/book/snippets/layout.tex}
\section{Examples Book LaTeX/book/snippets/layout/multiline_detect.tex}
\title{sitandr.github.io/typst-examples-book/book/snippets/layout/multiline_detect}

\section{\texorpdfstring{\hyperref[multiline-detection]{Multiline
detection}}{Multiline detection}}\label{multiline-detection}

Detects if figure caption (may be any other element) \emph{has more than
one line} .

If the caption is multiline, it makes it left-aligned.

Breaks on manual linebreaks.

\begin{verbatim}
#show figure.caption: it => {
  layout(size => context [
    #let text-size = measure(
      ..size,
      it.supplement + it.separator + it.body,
    )

    #let my-align

    #if text-size.width < size.width {
      my-align = center
    } else {
      my-align = left
    }

    #align(my-align, it)
  ])
}

#figure(caption: lorem(6))[
    ```rust
    pub fn main() {
        println!("Hello, world!");
    }
    ```
]

#figure(caption: lorem(20))[
    ```rust
    pub fn main() {
        println!("Hello, world!");
    }
    ```
]
\end{verbatim}

\pandocbounded{\includesvg[keepaspectratio]{typst-img/8e2a1d9e2e66f654938733a2ed1d9a0dcc771165a60d89c4410f4d970054121c-1.svg}}


\section{Examples Book LaTeX/book/snippets/layout/hiding.tex}
\title{sitandr.github.io/typst-examples-book/book/snippets/layout/hiding}

\section{\texorpdfstring{\hyperref[hiding-things]{Hiding
things}}{Hiding things}}\label{hiding-things}

\begin{verbatim}
// author: GeorgeMuscat
#let redact(text, fill: black, height: 1em) = {
  box(rect(fill: fill, height: height)[#hide(text)])
}

Example:
  - Unredacted text
  - Redacted #redact("text")
\end{verbatim}

\pandocbounded{\includesvg[keepaspectratio]{typst-img/6b85fdf4b9ba387543271058b6acb27e202dab93b01c2cd7ac93187c1e8b643c-1.svg}}


\section{Examples Book LaTeX/book/snippets/layout/duplicate.tex}
\title{sitandr.github.io/typst-examples-book/book/snippets/layout/duplicate}

\section{\texorpdfstring{\hyperref[duplicate-content]{Duplicate
content}}{Duplicate content}}\label{duplicate-content}

Notice that this implementation will mess up with labels and similar
things. For complex cases see one below.

```typ \#set page(paper: "a4", flipped: true) \#show: body
=\textgreater{} grid( columns: (1fr, 1fr), column-gutter: 1cm, body,
body, ) \#lorem(200) ```

\subsection{\texorpdfstring{\hyperref[advanced]{Advanced}}{Advanced}}\label{advanced}

\begin{verbatim}
/// author: frozolotl
#set page(paper: "a4", flipped: true)
#set heading(numbering: "1.1")
#show ref: it => {
  if it.element != none {
    it
  } else {
    let targets = query(it.target, it.location())
    if targets.len() == 2 {
      let target = targets.first()
      if target.func() == heading {
        let num = numbering(target.numbering, ..counter(heading).at(target.location()))
        [#target.supplement #num]
      } else if target.func() == figure {
        let num = numbering(target.numbering, ..target.counter.at(target.location()))
        [#target.supplement #num]
      } else {
        it
      }
    } else {
      it
    }
  }
}
#show link: it => context {
  let dest = query(it.dest)
  if dest.len() == 2 {
    link(dest.first().location(), it.body)
  } else {
    it
  }
}
#show: body => context grid(
  columns: (1fr, 1fr),
  column-gutter: 1cm,
  body,
  {
    let reset-counter(kind) = counter(kind).update(counter(kind).get())
    reset-counter(heading)
    reset-counter(figure.where(kind: image))
    reset-counter(figure.where(kind: raw))
    set heading(outlined: false)
    set figure(outlined: false)
    body
  },
)

#outline()

= Foo <foo>
See @foo and @foobar.

#figure(rect[This is an image], caption: [Foobar], kind: raw) <foobar>

== Bar
== Baz
#link(<foo>)[Click to visit Foo]
\end{verbatim}

\pandocbounded{\includesvg[keepaspectratio]{typst-img/2fdcc2778a936608ed868521793f59311ac54d43e226639db3ab14c6ca37c75f-1.svg}}


\section{Examples Book LaTeX/book/snippets/layout/shapes.tex}
\title{sitandr.github.io/typst-examples-book/book/snippets/layout/shapes}

\section{\texorpdfstring{\hyperref[shaped-boxes-with-text]{Shaped boxes
with text}}{Shaped boxes with text}}\label{shaped-boxes-with-text}

(I guess that will make a package eventually, but let it be a snippet
for now)

\begin{verbatim}
/// author: JustForFun88
#import "@preview/oxifmt:0.2.0": strfmt

#let shadow_svg_path = `
<svg
    width="{canvas-width}"
    height="{canvas-height}"
    viewBox="{viewbox}"
    version="1.1"
    xmlns="http://www.w3.org/2000/svg"
    xmlns:svg="http://www.w3.org/2000/svg">
    <!-- Definitions for reusable components -->
    <defs>
        <filter id="shadowing" >
            <feGaussianBlur in="SourceGraphic" stdDeviation="{blur}" />
        </filter>
    </defs>

    <!-- Drawing the rectangle with a fill and feGaussianBlur effect -->
    <path
        style="fill: {flood-color}; opacity: {flood-opacity}; filter:url(#shadowing)"
        d="{vertices} Z" />
</svg>
`.text

#let parallelogram(width: 20mm, height: 5mm, angle: 30deg) = {
  let δ = height * calc.tan(angle)
  (
    (      + δ,     0pt   ),
    (width + δ * 2, 0pt   ),
    (width + δ,     height),
    (0pt,           height),
  )
}

#let hexagon(width: 100pt, height: 30pt, angle: 30deg) = {
  let dy = height / 2;
  let δ = dy * calc.tan(angle)
  (
    (0pt,           dy    ),
    (      + δ,     0pt   ),
    (width + δ,     0pt   ),
    (width + δ * 2, dy    ),
    (width + δ,     height),
    (      + δ,     height),
  )
}

#let shape_size(vertices) = {
    let x_vertices = vertices.map(array.first);
    let y_vertices = vertices.map(array.last);

    (
      calc.max(..x_vertices) - calc.min(..x_vertices),
      calc.max(..y_vertices) - calc.min(..y_vertices)
    )
}

#let shadowed_shape(shape: hexagon, fill: none,
  stroke: auto, angle: 30deg, shadow_fill: black, alpha: 0.5, 
  blur: 1.5, blur_margin: 5, dx: 0pt, dy: 0pt, ..args, content
) = layout(size => context {
    let named = args.named()
    for key in ("width", "height") {
      if key in named and type(named.at(key)) == ratio {
        named.insert(key, size.at(key) * named.at(key))
      }
    }

    let opts = (blur: blur, flood-color: shadow_fill.to-hex())
       
    let content = box(content, ..named)
    let size = measure(content)

    let vertices = shape(..size, angle: angle)
    let (shape_width, shape_height) = shape_size(vertices)
    let margin = opts.blur * blur_margin * 1pt

    opts += (
      canvas-width:  shape_width  + margin,
      canvas-height: shape_height + margin,
      flood-opacity: alpha
    )

    opts.viewbox = (0, 0, opts.canvas-width.pt(), opts.canvas-height.pt()).map(str).join(",")

    opts.vertices = "";
    let d = margin / 2;
    for (i, p) in vertices.enumerate() {
        let prefix = if i == 0 { "M " } else { " L " };
        opts.vertices += prefix + p.map(x => str((x + d).pt())).join(", ");
    }

    let svg-shadow = image.decode(strfmt(shadow_svg_path, ..opts))
    place(dx: dx, dy: dy, svg-shadow)
    place(path(..vertices, fill: fill, stroke: stroke, closed: true))
    box(h((shape_width - size.width) / 2) + content, width: shape_width)
})

#set text(3em);

#shadowed_shape(shape: hexagon,
    inset: 1em, fill: teal,
    stroke: 1.5pt + teal.darken(50%),
    shadow_fill: red,
    dx: 0.5em, dy: 0.35em, blur: 3)[Hello there!]
#shadowed_shape(shape: parallelogram,
    inset: 1em, fill: teal,
    stroke: 1.5pt + teal.darken(50%),
    shadow_fill: red,
    dx: 0.5em, dy: 0.35em, blur: 3)[Hello there!]
\end{verbatim}

\pandocbounded{\includesvg[keepaspectratio]{typst-img/f40acb7d6d2753b0845c9dd1fb26979c29dd0850448cf585f0c7f1b20acde7ea-1.svg}}


\section{Examples Book LaTeX/book/snippets/layout/insert_lines.tex}
\title{sitandr.github.io/typst-examples-book/book/snippets/layout/insert_lines}

\section{\texorpdfstring{\hyperref[lines-between-list-items]{Lines
between list
items}}{Lines between list items}}\label{lines-between-list-items}

\begin{verbatim}
/// author: frozolotl
#show enum.where(tight: false): it => {
  it.children
    .enumerate()
    .map(((n, item)) => block(below: .6em, above: .6em)[#numbering("1.", n + 1) #item.body])
    .join(line(length: 100%))
}

+ Item 1

+ Item 2

+ Item 3
\end{verbatim}

\pandocbounded{\includesvg[keepaspectratio]{typst-img/b1660863fded6fc3d870f8a92f364040d5ba9beaaf5bbd4a114b5384abe3db4c-1.svg}}

The same approach may be easily adapted to style the enums as you want.


\section{Examples Book LaTeX/book/snippets/layout/page_setup.tex}
\title{sitandr.github.io/typst-examples-book/book/snippets/layout/page_setup}

\section{\texorpdfstring{\hyperref[page-setup]{Page
setup}}{Page setup}}\label{page-setup}

\begin{quote}
See \href{https://typst.app/docs/guides/page-setup-guide/}{Official Page
Setup guide}
\end{quote}

\begin{verbatim}
#set page(
  width: 3cm,
  margin: (x: 0cm),
)

#for i in range(3) {
  box(square(width: 1cm))
}
\end{verbatim}

\pandocbounded{\includesvg[keepaspectratio]{typst-img/6a1e9261d0b0bcd09b578e8361c939100328fbccfd8289402ad62f768b55a0c1-1.svg}}

\begin{verbatim}
#set page(columns: 2, height: 4.8cm)
Climate change is one of the most
pressing issues of our time, with
the potential to devastate
communities, ecosystems, and
economies around the world. It's
clear that we need to take urgent
action to reduce our carbon
emissions and mitigate the impacts
of a rapidly changing climate.
\end{verbatim}

\pandocbounded{\includesvg[keepaspectratio]{typst-img/2b0351806e86c3410f445beb2a51887aebd3f73649e2fe638ba45a39026284dd-1.svg}}

\begin{verbatim}
#set page(fill: rgb("444352"))
#set text(fill: rgb("fdfdfd"))
*Dark mode enabled.*
\end{verbatim}

\pandocbounded{\includesvg[keepaspectratio]{typst-img/340892f7237f4bc864f9ca9dc5fd956fe4032a157a373e0bb4b7358200daa72e-1.svg}}

\begin{verbatim}
#set par(justify: true)
#set page(
  margin: (top: 32pt, bottom: 20pt),
  header: [
    #set text(8pt)
    #smallcaps[Typst Academcy]
    #h(1fr) _Exercise Sheet 3_
  ],
)

#lorem(19)
\end{verbatim}

\pandocbounded{\includesvg[keepaspectratio]{typst-img/bfb28329922a1eb129dd2c7d7003dcfa30ebdc119265f19f8190b69d3e40ff68-1.svg}}

\begin{verbatim}
#set page(foreground: text(24pt)[🥸])

Reviewer 2 has marked our paper
"Weak Reject" because they did
not understand our approach...
\end{verbatim}

\pandocbounded{\includesvg[keepaspectratio]{typst-img/b88eae1fcb87b110e66ee6493c60c2c3e0d0c9a7f1c288e739bf1bb8e09c8d70-1.svg}}




\section{Combined Examples Book LaTeX/book/snippets/math.tex}
\section{Examples Book LaTeX/book/snippets/math/calligraphic.tex}
\title{sitandr.github.io/typst-examples-book/book/snippets/math/calligraphic}

\section{\texorpdfstring{\hyperref[calligraphic-letters]{Calligraphic
letters}}{Calligraphic letters}}\label{calligraphic-letters}

\begin{verbatim}
#let scr(it) = math.class("normal",
  text(font: "", stylistic-set: 1, $cal(it)$) + h(0em)
)

$ scr(A) scr(B) + scr(C), -scr(D) $
\end{verbatim}

\pandocbounded{\includesvg[keepaspectratio]{typst-img/6ee9ca10515c1b6158d8d7bddd4418a713313052c0114fe851be455fc09b2c92-1.svg}}

Unfortunately, currently just
\texttt{\ }{\texttt{\ stylistic-set\ }}\texttt{\ } for math creates bad
spacing. Math engine detects if the letter should be correctly spaced by
whether it is the default font. However, just making it "normal"
isn\textquotesingle t enough, because than it can be reduced.
That\textquotesingle s way the snippet is as hacky as it is (probably
should be located in Typstonomicon, but it\textquotesingle s not large
enough).


\section{Examples Book LaTeX/book/snippets/math/numbering.tex}
\title{sitandr.github.io/typst-examples-book/book/snippets/math/numbering}

\section{\texorpdfstring{\hyperref[math-numbering]{Math
Numbering}}{Math Numbering}}\label{math-numbering}

\subsection{\texorpdfstring{\hyperref[number-by-current-heading]{Number
by current
heading}}{Number by current heading}}\label{number-by-current-heading}

\begin{quote}
See also built-in numbering in
\href{../../packages/math.html\#theorems}{math package section}
\end{quote}

\begin{verbatim}
/// original author: laurmaedje
#set heading(numbering: "1.")

// reset counter at each chapter
// if you want to change the number of displayed 
// section numbers, change the level there
#show heading.where(level:1): it => {
  counter(math.equation).update(0)
  it
}

#set math.equation(numbering: n => {
  numbering("(1.1)", counter(heading).get().first(), n)
  // if you want change the number of number of displayed
  // section numbers, modify it this way:
  /*
  let count = counter(heading).get()
  let h1 = count.first()
  let h2 = count.at(1, default: 0)
  numbering("(1.1.1)", h1, h2, n)
  */
})


= Section
== Subsection

$ 5 + 3 = 8 $ <a>
$ 5 + 3 = 8 $

= New Section
== Subsection
$ 5 + 3 = 8 $
== Subsection
$ 5 + 3 = 8 $ <b>

Mentioning @a and @b.
\end{verbatim}

\pandocbounded{\includesvg[keepaspectratio]{typst-img/9662902bb463e350d7a9bdf94e143bbaab8245da34eee4a426d2263d44511d1f-1.svg}}

\subsection{\texorpdfstring{\hyperref[number-only-labeled-equations]{Number
only labeled
equations}}{Number only labeled equations}}\label{number-only-labeled-equations}

\subsubsection{\texorpdfstring{\hyperref[simple-code]{Simple
code}}{Simple code}}\label{simple-code}

\begin{verbatim}
// author: shampoohere
#show math.equation:it => {
  if it.fields().keys().contains("label"){
    math.equation(block: true, numbering: "(1)", it)
    // Don't forget to change your numbering style in `numbering`
    // to the one you actually want to use.
    //
    // Note that you don't need to #set the numbering now.
  } else {
    it
  }
}

$ sum_x^2 $
$ dif/(dif x)(A(t)+B(x))=dif/(dif x)A(t)+dif/(dif x)B(t) $ <ep-2>
$ sum_x^2 $
$ dif/(dif x)(A(t)+B(x))=dif/(dif x)A(t)+dif/(dif x)B(t) $ <ep-3>
\end{verbatim}

\pandocbounded{\includesvg[keepaspectratio]{typst-img/84052f83d0e2e2c330ef041c254dfb7c735526fc7f47cdb14ecc46961f66fee3-1.svg}}

\subsubsection{\texorpdfstring{\hyperref[make-the-hacked-references-clickable-again]{Make
the hacked references clickable
again}}{Make the hacked references clickable again}}\label{make-the-hacked-references-clickable-again}

\begin{verbatim}
// author: gijsleb
#show math.equation:it => {
  if it.has("label") {
    // Don't forget to change your numbering style in `numbering`
    // to the one you actually want to use.
    math.equation(block: true, numbering: "(1)", it)
  } else {
    it
  }
}

#show ref: it => {
  let el = it.element
  if el != none and el.func() == math.equation {
    link(el.location(), numbering(
      // don't forget to change the numbering according to the one
      // you are actually using (e.g. section numbering)
      "(1)",
      counter(math.equation).at(el.location()).at(0) + 1
    ))
  } else {
    it
  }
}

$ sum_x^2 $
$ dif/(dif x)(A(t)+B(x))=dif/(dif x)A(t)+dif/(dif x)B(t) $ <ep-2>
$ sum_x^2 $
$ dif/(dif x)(A(t)+B(x))=dif/(dif x)A(t)+dif/(dif x)B(t) $ <ep-3>
In @ep-2 and @ep-3 we see equations
\end{verbatim}

\pandocbounded{\includesvg[keepaspectratio]{typst-img/c17ccdb8d65b5aa0d0e58b1fba75c67bfc162400ba86ad64f37aa038ad6d8887-1.svg}}


\section{Examples Book LaTeX/book/snippets/math/scripts.tex}
\title{sitandr.github.io/typst-examples-book/book/snippets/math/scripts}

\section{\texorpdfstring{\hyperref[scripts]{Scripts}}{Scripts}}\label{scripts}

\begin{quote}
To set scripts and limits see \href{../../basics/math/limits.html}{Typst
Basics section}
\end{quote}

\subsection{\texorpdfstring{\hyperref[make-every-character-upright-when-used-in-subscript]{Make
every character upright when used in
subscript}}{Make every character upright when used in subscript}}\label{make-every-character-upright-when-used-in-subscript}

\begin{verbatim}
// author: emilyyyylime

$f_a, f_b, f^a, f_italic("word")$
#show math.attach: it => {
  import math: *
  if it.b != none and it.b.func() != upright[].func() and it.b.has("text") and it.b.text.len() == 1 {
    let args = it.fields()
    let _ = args.remove("base")
    let _ = args.remove("b")
    attach(it.base, b: upright(it.b), ..args)
  } else {
    it
  }
}

$f_a, f_b, f^a, f_italic("word")$
\end{verbatim}

\pandocbounded{\includesvg[keepaspectratio]{typst-img/40b68a1d7f5aeb1c498431996be1b140b3d217ba5b0230b6b73e6fefe64d45cd-1.svg}}


\section{Examples Book LaTeX/book/snippets/math/vecs.tex}
\title{sitandr.github.io/typst-examples-book/book/snippets/math/vecs}

\section{\texorpdfstring{\hyperref[vectors--matrices]{Vectors \&
Matrices}}{Vectors \& Matrices}}\label{vectors--matrices}

You can easily note that the gap isn\textquotesingle t necessarily even
or the same in different vectors and matrices:

\begin{verbatim}
$
mat(0, 1, -1; -1, 0, 1; 1, -1, 0) vec(a/b, a/b, a/b) = vec(c, d, e)
$
\end{verbatim}

\pandocbounded{\includesvg[keepaspectratio]{typst-img/6a28529f5b38a17bcb660981691cdcc2bc4b6ddfbdb103327ae9e42b1365458e-1.svg}}

That happens because \texttt{\ }{\texttt{\ gap\ }}\texttt{\ } refers to
\emph{spacing between} elements, not the distance between their centers.

To fix this, you can use this snippet:

\begin{verbatim}
// Fixed height vector
#let fvec(..children, delim: "(", gap: 1.5em) = { // change default gap there
  context math.vec(
      delim: delim,
      gap: 0em,
      ..for el in children.pos() {
        ({
          box(
            width: measure(el).width,
            height: gap, place(horizon, el)
          )
        },) // this is an array
        // `for` merges all these arrays, then we pass it to arguments
      }
    )
}

// fixed hight matrix
// accepts also row-gap, column-gap and gap
#let fmat(..rows, delim: "(", augment: none) = {
  let args = rows.named()
  let (gap, row-gap, column-gap) = (none,)*3;

  if "gap" in args {
    gap = args.at("gap")
    row-gap = args.at("row-gap", default: gap)
    column-gap = args.at("row-gap", default: gap)
  }
  else {
    // change default vertical there
    row-gap = args.at("row-gap", default: 1.5em) 
    // and horizontal there
    column-gap = rows.named().at("column-gap", default: 0.5em)
  }

  context math.mat(
      delim: delim,
      row-gap: 0em,
      column-gap: column-gap,
      ..for row in rows.pos() {
        (for el in row {
          ({
          box(
            width: measure(el).width,
            height: row-gap, place(horizon, el)
          )
        },)
        }, )
      }
    )
}

$
"Before:"& vec(((a/b))/c, a/b, c) = vec(1, 1, 1)\
"After:"& fvec(((a/b))/c, a/b, c) = fvec(1, 1, 1)\

"Before:"& mat(a, b; c, d) vec(e, dot) = vec(c/d, e/f)\
"After:"& fmat(a, b; c, d) fvec(e, dot) = fvec(c/d, e/f)
$
\end{verbatim}

\pandocbounded{\includesvg[keepaspectratio]{typst-img/98195a6d9cfb93fdc5dca4db04dde22c00b969129e2962c8f7cba9012cd2bd0d-1.svg}}


\section{Examples Book LaTeX/book/snippets/math/operations.tex}
\title{sitandr.github.io/typst-examples-book/book/snippets/math/operations}

\section{\texorpdfstring{\hyperref[operations]{Operations}}{Operations}}\label{operations}

\subsection{\texorpdfstring{\hyperref[fractions]{Fractions}}{Fractions}}\label{fractions}

\begin{verbatim}
$
p/q, p slash q, p\/q
$
\end{verbatim}

\pandocbounded{\includesvg[keepaspectratio]{typst-img/7e6b189e7b1c1329caebb4d4c6ea718c897ef64f51383889c65e62e308c73478-1.svg}}

\subsubsection{\texorpdfstring{\hyperref[slightly-moved]{Slightly
moved:}}{Slightly moved:}}\label{slightly-moved}

\begin{verbatim}
#let mfrac(a, b) = move(a, dy: -0.2em) + "/" + move(b, dy: 0.2em, dx: -0.1em)
$A\/B, #mfrac($A$, $B$)$,
\end{verbatim}

\pandocbounded{\includesvg[keepaspectratio]{typst-img/002c9e0e934a98cfb5e93a407d130841a5e1a493d361c368ae605acdfd6f64bc-1.svg}}

\subsubsection{\texorpdfstring{\hyperref[large-fractions]{Large
fractions}}{Large fractions}}\label{large-fractions}

\begin{verbatim}
#let dfrac(a, b) = $display(frac(#a, #b))$

$(x + y)/(1/x + 1/y) quad (x + y)/(dfrac(1,x) + dfrac(1, y))$
\end{verbatim}

\pandocbounded{\includesvg[keepaspectratio]{typst-img/36454aba32957127c97710e4fc1db3e6d8c9a558e886b7103915d501004bad76-1.svg}}


\section{Examples Book LaTeX/book/snippets/math/fonts.tex}
\title{sitandr.github.io/typst-examples-book/book/snippets/math/fonts}

\section{\texorpdfstring{\hyperref[fonts]{Fonts}}{Fonts}}\label{fonts}

\subsection{\texorpdfstring{\hyperref[set-math-font]{Set math
font}}{Set math font}}\label{set-math-font}

\textbf{Important:} The font you want to set for math should
\emph{contain} necessary math symbols. That should be a special font
with math. If it isn\textquotesingle t, you are very likely to get
\emph{an error} (remember to set
\texttt{\ }{\texttt{\ fallback:\ false\ }}\texttt{\ } and check
\texttt{\ }{\texttt{\ typst\ fonts\ }}\texttt{\ } to debug the fonts).

\begin{verbatim}
#show math.equation: set text(font: "Fira Math", fallback: false)

$
emptyset \

integral_a^b sum (A + B)/C dif x \
$
\end{verbatim}

\pandocbounded{\includesvg[keepaspectratio]{typst-img/f1cc8c8afe82aeb14cb0898ff5a83292a7c6a73f8b84cac8dbffa19af0d9370f-1.svg}}




\section{Combined Examples Book LaTeX/book/snippets/special.tex}
\section{Examples Book LaTeX/book/snippets/special/index.tex}
\title{sitandr.github.io/typst-examples-book/book/snippets/special/index}

\section{\texorpdfstring{\hyperref[special-documents]{Special
documents}}{Special documents}}\label{special-documents}

\subsection{\texorpdfstring{\hyperref[signature-places]{Signature
places}}{Signature places}}\label{signature-places}

\begin{verbatim}
#block(width: 150pt)[
  #line(length: 100%)
  #align(center)[Signature]
]
\end{verbatim}

\pandocbounded{\includesvg[keepaspectratio]{typst-img/04e318a822e6a90fbae23cce4c1b829e03e4d283051bb5dd613be2d7fe5933a2-1.svg}}

\subsection{\texorpdfstring{\hyperref[presentations]{Presentations}}{Presentations}}\label{presentations}

See \href{../../packages/}{polylux} .

\subsection{\texorpdfstring{\hyperref[forms]{Forms}}{Forms}}\label{forms}

\subsubsection{\texorpdfstring{\hyperref[form-with-placeholder]{Form
with placeholder}}{Form with placeholder}}\label{form-with-placeholder}

\begin{verbatim}
#grid(
  columns: 2,
  rows: 4,
  gutter: 1em,

  [Student:],
  [#block()#align(bottom)[#line(length: 10em, stroke: 0.5pt)]],
  [Teacher:],
  [#block()#align(bottom)[#line(length: 10em, stroke: 0.5pt)]],
  [ID:],
  [#block()#align(bottom)[#line(length: 10em, stroke: 0.5pt)]],
  [School:],
  [#block()#align(bottom)[#line(length: 10em, stroke: 0.5pt)]],
)
\end{verbatim}

\pandocbounded{\includesvg[keepaspectratio]{typst-img/d921f4df08e8dab19c9440b3aca6b065cb83ba11f015240b2115a63351ce64ce-1.svg}}

\subsubsection{\texorpdfstring{\hyperref[interactive]{Interactive}}{Interactive}}\label{interactive}

\begin{quote}
Presentation interactive forms are coming! They are currently under
heavy work by @tinger.
\end{quote}






\section{C Examples Book LaTeX/book/book.tex}
\section{Combined Examples Book LaTeX/book/snippets.tex}
\section{Combined Examples Book LaTeX/book/snippets/chapters.tex}
\section{Examples Book LaTeX/book/snippets/chapters/outlines.tex}
\title{sitandr.github.io/typst-examples-book/book/snippets/chapters/outlines}

\section{\texorpdfstring{\hyperref[outlines]{Outlines}}{Outlines}}\label{outlines}

\section{\texorpdfstring{\hyperref[outlines-1]{Outlines}}{Outlines}}\label{outlines-1}

\begin{quote}
Lots of outlines examples are already available in
\href{https://typst.app/docs/reference/meta/outline/}{official
reference}
\end{quote}

\subsection{\texorpdfstring{\hyperref[table-of-contents]{Table of
contents}}{Table of contents}}\label{table-of-contents}

\begin{verbatim}
#outline()

= Introduction
#lorem(5)

= Prior work
#lorem(10)
\end{verbatim}

\pandocbounded{\includesvg[keepaspectratio]{typst-img/77dbcfc8072afc53714fea404efaa1f60692fee68a19821e77feb8bafdd5bd46-1.svg}}

\subsection{\texorpdfstring{\hyperref[outline-of-figures]{Outline of
figures}}{Outline of figures}}\label{outline-of-figures}

\begin{verbatim}
#outline(
  title: [List of Figures],
  target: figure.where(kind: table),
)

#figure(
  table(
    columns: 4,
    [t], [1], [2], [3],
    [y], [0.3], [0.7], [0.5],
  ),
  caption: [Experiment results],
)
\end{verbatim}

\pandocbounded{\includesvg[keepaspectratio]{typst-img/a898ed56e04bc374a8b88580ae203c7695d92445179cffad2642d1a08a8f7ef1-1.svg}}

You can use arbitrary selector there, so you can do any crazy things.

\subsection{\texorpdfstring{\hyperref[ignore-low-level-headings]{Ignore
low-level
headings}}{Ignore low-level headings}}\label{ignore-low-level-headings}

\begin{verbatim}
#set heading(numbering: "1.")
#outline(depth: 2)

= Yes
Top-level section.

== Still
Subsection.

=== Nope
Not included.
\end{verbatim}

\pandocbounded{\includesvg[keepaspectratio]{typst-img/c6947016b324ba83db8aea6da98d4658877618b4ee650edabdb2360782fd520c-1.svg}}

\subsection{\texorpdfstring{\hyperref[set-indentation]{Set
indentation}}{Set indentation}}\label{set-indentation}

\begin{verbatim}
#set heading(numbering: "1.a.")

#outline(
  title: [Contents (Automatic)],
  indent: auto,
)

#outline(
  title: [Contents (Length)],
  indent: 2em,
)

#outline(
  title: [Contents (Function)],
  indent: n => [→ ] * n,
)

= About ACME Corp.
== History
=== Origins
#lorem(10)

== Products
#lorem(10)
\end{verbatim}

\pandocbounded{\includesvg[keepaspectratio]{typst-img/59dc3acb28c16d258b93278079848404454449450103de6f456454aef50a8e55-1.svg}}

\subsection{\texorpdfstring{\hyperref[replace-default-dots]{Replace
default dots}}{Replace default dots}}\label{replace-default-dots}

\begin{verbatim}
#outline(fill: line(length: 100%), indent: 2em)

= First level
== Second level
\end{verbatim}

\pandocbounded{\includesvg[keepaspectratio]{typst-img/81c9f957fb8944561295980a7dfd1ee2b1fbd58f77d90e7b904aa8b99b3bbf9e-1.svg}}

\subsection{\texorpdfstring{\hyperref[make-different-outline-levels-look-different]{Make
different outline levels look
different}}{Make different outline levels look different}}\label{make-different-outline-levels-look-different}

\begin{verbatim}
#set heading(numbering: "1.")

#show outline.entry.where(
  level: 1
): it => {
  v(12pt, weak: true)
  strong(it)
}

#outline(indent: auto)

= Introduction
= Background
== History
== State of the Art
= Analysis
== Setup
\end{verbatim}

\pandocbounded{\includesvg[keepaspectratio]{typst-img/e620be5254ab48d4bb3f5e1b8bc928e1f8c43d0ba0929b6dc858522539ff4e0c-1.svg}}

\subsection{\texorpdfstring{\hyperref[long-and-short-captions-for-the-outline]{Long
and short captions for the
outline}}{Long and short captions for the outline}}\label{long-and-short-captions-for-the-outline}

\begin{verbatim}
// author: laurmaedje
// Put this somewhere in your template or at the start of your document.
#let in-outline = state("in-outline", false)
#show outline: it => {
  in-outline.update(true)
  it
  in-outline.update(false)
}

#let flex-caption(long, short) = context if in-outline.get() { short } else { long }

// And this is in the document.
#outline(title: [Figures], target: figure)

#figure(
  rect(),
  caption: flex-caption(
    [This is my long caption text in the document.],
    [This is short],
  )
)
\end{verbatim}

\pandocbounded{\includesvg[keepaspectratio]{typst-img/fc4dc1c50f173f2ee6d73ee8868e6a8cd3d4a666165d2d05d21cfaa410361e31-1.svg}}

\subsection{\texorpdfstring{\hyperref[ignore-citations-and-footnotes]{Ignore
citations and
footnotes}}{Ignore citations and footnotes}}\label{ignore-citations-and-footnotes}

That\textquotesingle s a workaround a problem that if you make a
footnote a heading, its number will be displayed in outline:

\begin{verbatim}
= Heading #footnote[A footnote]

Text

#outline() // bad :(

#pagebreak()
#{
  set footnote.entry(
    separator: none
  )
  show footnote.entry: hide
  show ref: none
  show footnote: none

  outline()
}
\end{verbatim}

\pandocbounded{\includesvg[keepaspectratio]{typst-img/baac264bb9ab5bfcf402ee4508cb3c18a8e747b17fefa125c9d2998df0f5a283-1.svg}}

\pandocbounded{\includesvg[keepaspectratio]{typst-img/baac264bb9ab5bfcf402ee4508cb3c18a8e747b17fefa125c9d2998df0f5a283-2.svg}}


\section{Examples Book LaTeX/book/snippets/chapters/page-numbering.tex}
\title{sitandr.github.io/typst-examples-book/book/snippets/chapters/page-numbering}

\section{\texorpdfstring{\hyperref[page-numbering]{Page
numbering}}{Page numbering}}\label{page-numbering}

\subsection{\texorpdfstring{\hyperref[separate-page-numbering-for-each-chapter]{Separate
page numbering for each
chapter}}{Separate page numbering for each chapter}}\label{separate-page-numbering-for-each-chapter}

\begin{verbatim}
/// author: tinger

// unnumbered title page if needed
// ...

// front-matter
#set page(numbering: "I")
#counter(page).update(1)
#lorem(50)
// ...

// page counter anchor
#metadata(()) <front-matter>

// main document body
#set page(numbering: "1")
#lorem(50)
#counter(page).update(1)
// ...

// back-matter
#set page(numbering: "I")
// must take page breaks into account, may need to be offset by +1 or -1
#context counter(page).update(counter(page).at(<front-matter>).first())
#lorem(50)
// ...
\end{verbatim}

\pandocbounded{\includesvg[keepaspectratio]{typst-img/0cd153b35bf7532971dbbb220095812665f44b0ab9cca9d7a8c6c000f83e3e30-1.svg}}

\pandocbounded{\includesvg[keepaspectratio]{typst-img/0cd153b35bf7532971dbbb220095812665f44b0ab9cca9d7a8c6c000f83e3e30-2.svg}}

\pandocbounded{\includesvg[keepaspectratio]{typst-img/0cd153b35bf7532971dbbb220095812665f44b0ab9cca9d7a8c6c000f83e3e30-3.svg}}




\section{Combined Examples Book LaTeX/book/snippets/snippets.tex}
\section{Examples Book LaTeX/book/snippets/index.tex}
\title{sitandr.github.io/typst-examples-book/book/snippets/index}

\section{\texorpdfstring{\hyperref[typst-snippets]{Typst
Snippets}}{Typst Snippets}}\label{typst-snippets}

Useful snippets for common (and not) tasks.


\section{Examples Book LaTeX/book/snippets/numbering.tex}
\title{sitandr.github.io/typst-examples-book/book/snippets/numbering}

\section{\texorpdfstring{\hyperref[numbering]{Numbering}}{Numbering}}\label{numbering}

\subsection{\texorpdfstring{\hyperref[individual-heading-without-numbering]{Individual
heading without
numbering}}{Individual heading without numbering}}\label{individual-heading-without-numbering}

\begin{verbatim}
#let numless(it) = {set heading(numbering: none); it }

= Heading
#numless[=No numbering heading]
\end{verbatim}

\pandocbounded{\includesvg[keepaspectratio]{typst-img/e04f844b270049702ac72dff7bfadf5963cdb2bc8a541e81b685124fbb61c48e-1.svg}}

\subsection{\texorpdfstring{\hyperref[clean-numbering]{"Clean"
numbering}}{"Clean" numbering}}\label{clean-numbering}

\begin{verbatim}
// original author: tromboneher

// Number sections according to a number of schemes, omitting previous leading elements.
// For example, where the numbering pattern "A.I.1." would produce:
//
// A. A part of the story
//   A.I. A chapter
//   A.II. Another chapter
//     A.II.1. A section
//       A.II.1.a. A subsection
//       A.II.1.b. Another subsection
//     A.II.2. Another section
// B. Another part of the story
//   B.I. A chapter in the second part
//   B.II. Another chapter in the second part
//
// clean_numbering("A.", "I.", "1.a.") would produce:
//
// A. A part of the story
//   I. A chapter
//   II. Another chapter
//     1. A section
//       1.a. A subsection
//       1.b. Another subsection
//     2. Another section
// B. Another part of the story
//   I. A chapter in the second part
//   II. Another chapter in the second part
//
#let clean_numbering(..schemes) = {
  (..nums) => {
    let (section, ..subsections) = nums.pos()
    let (section_scheme, ..subschemes) = schemes.pos()

    if subsections.len() == 0 {
      numbering(section_scheme, section)
    } else if subschemes.len() == 0 {
      numbering(section_scheme, ..nums.pos())
    }
    else {
      clean_numbering(..subschemes)(..subsections)
    }
  }
}

#set heading(numbering: clean_numbering("A.", "I.", "1.a."))

= Part
== Chapter
== Another chapter
=== Section
==== Subsection
==== Another subsection
= Another part of the story
== A chapter in the second part
== Another chapter in the second part
\end{verbatim}

\pandocbounded{\includesvg[keepaspectratio]{typst-img/4e29319442704545bf58d12448745836598c12f59162d3199aaad21c752e4483-1.svg}}

\subsection{\texorpdfstring{\hyperref[math-numbering]{Math
numbering}}{Math numbering}}\label{math-numbering}

See \href{./math/numbering.html}{there} .

\subsection{\texorpdfstring{\hyperref[numbering-each-paragraph]{Numbering
each
paragraph}}{Numbering each paragraph}}\label{numbering-each-paragraph}

By the 0.12 version of Typst, this should be replaced with good native
solution.

\begin{verbatim}
// original author: roehlichA
// Legal formatting of enumeration
#show enum: it => context {
  // Retrieve the last heading so we know what level to step at
  let headings = query(selector(heading).before(here()))
  let last = headings.at(-1)

  // Combine the output items
  let output = ()
  for item in it.children {
    output.push([
      #context{
        counter(heading).step(level: last.level + 1)
      }
      #context {
        counter(heading).display() 
      }
    ])
    output.push([
      #text(item.body)
      #parbreak()
    ])
  }

  // Display in a grid
  grid(
    columns: (auto, 1fr),
    column-gutter: 1em,
    row-gutter: 1em,
    ..output
  )

}

#set heading(numbering: "1.")

= Some heading
+ Paragraph
= Other
+ Paragraphs here are preceded with a number so they can be referenced directly.
+ _#lorem(100)_
+ _#lorem(100)_

== A subheading
+ Paragraphs are also numbered correctly in subheadings.
+ _#lorem(50)_
+ _#lorem(50)_
\end{verbatim}

\pandocbounded{\includesvg[keepaspectratio]{typst-img/8d5603f93334c1d0fd7391811f90b161d4ff8c7eb81100dc152caac5c6d13daf-1.svg}}


\section{Examples Book LaTeX/book/snippets/demos.tex}
\title{sitandr.github.io/typst-examples-book/book/snippets/demos}

\section{\texorpdfstring{\hyperref[demos]{Demos}}{Demos}}\label{demos}

\subsection{\texorpdfstring{\hyperref[resume-using-template]{Resume
(using
template)}}{Resume (using template)}}\label{resume-using-template}

\begin{verbatim}
#import "@preview/modern-cv:0.1.0": *

#show: resume.with(
  author: (
      firstname: "John", 
      lastname: "Smith",
      email: "js@example.com", 
      phone: "(+1) 111-111-1111",
      github: "DeveloperPaul123",
      linkedin: "Example",
      address: "111 Example St. Example City, EX 11111",
      positions: (
        "Software Engineer",
        "Software Architect"
      )
  ),
  date: datetime.today().display()
)

= Education

#resume-entry(
  title: "Example University",
  location: "B.S. in Computer Science",
  date: "August 2014 - May 2019",
  description: "Example"
)

#resume-item[
  - #lorem(20)
  - #lorem(15)
  - #lorem(25)
]
\end{verbatim}

\pandocbounded{\includesvg[keepaspectratio]{typst-img/fc69693c49a6cf8021751980642ed7649c9d905056f510fb8e4a994937faeaa2-1.svg}}

\subsection{\texorpdfstring{\hyperref[book-cover]{Book
cover}}{Book cover}}\label{book-cover}

\begin{verbatim}
// author: bamdone
#let accent  = rgb("#00A98F")
#let accent1 = rgb("#98FFB3")
#let accent2 = rgb("#D1FF94")
#let accent3 = rgb("#D3D3D3")
#let accent4 = rgb("#ADD8E6")
#let accent5 = rgb("#FFFFCC")
#let accent6 = rgb("#F5F5DC")

#set page(paper: "a4",margin: 0.0in, fill: accent)

#set rect(stroke: 4pt)
#move(
  dx: -6cm, dy: 1.0cm,
  rotate(-45deg,
    rect(
      width: 100cm,
      height: 2cm,
      radius: 50%,
      stroke: 0pt,
      fill:accent1,
)))

#set rect(stroke: 4pt)
#move(
  dx: -2cm, dy: -1.0cm,
  rotate(-45deg,
    rect(
      width: 100cm,
      height: 2cm,
      radius: 50%,
      stroke: 0pt,
      fill:accent2,
)))

#set rect(stroke: 4pt)
#move(
  dx: 8cm, dy: -10cm,
  rotate(-45deg,
    rect(
      width: 100cm,
      height: 1cm,
      radius: 50%,
      stroke: 0pt,
      fill:accent3,
)))

#set rect(stroke: 4pt)
#move(
  dx: 7cm, dy: -8cm,
  rotate(-45deg,
    rect(
      width: 1000cm,
      height: 2cm,
      radius: 50%,
      stroke: 0pt,
      fill:accent4,
)))

#set rect(stroke: 4pt)
#move(
  dx: 0cm, dy: -0cm,
  rotate(-45deg,
    rect(
      width: 1000cm,
      height: 2cm,
      radius: 50%,
      stroke: 0pt,
      fill:accent1,
)))

#set rect(stroke: 4pt)
#move(
  dx: 9cm, dy: -7cm,
  rotate(-45deg,
    rect(
      width: 1000cm,
      height: 1.5cm,
      radius: 50%,
      stroke: 0pt,
      fill:accent6,
)))

#set rect(stroke: 4pt)
#move(
  dx: 16cm, dy: -13cm,
  rotate(-45deg,
    rect(
      width: 1000cm,
      height: 1cm,
      radius: 50%,
      stroke: 0pt,
      fill:accent2,
)))

#align(center)[
  #rect(width: 30%,
    fill: accent4,
    stroke:none,
    [#align(center)[
      #text(size: 60pt,[Title])
    ]
    ])
]

#align(center)[
  #rect(width: 30%,
    fill: accent4,
    stroke:none,
    [#align(center)[
      #text(size: 20pt,[author])
    ]
    ])
]
\end{verbatim}

\pandocbounded{\includesvg[keepaspectratio]{typst-img/7c2e798dacec8ac970ac2b328c60f8145441d059f16f7bd193f389d78d121981-1.svg}}


\section{Examples Book LaTeX/book/snippets/pretty.tex}
\title{sitandr.github.io/typst-examples-book/book/snippets/pretty}

\section{\texorpdfstring{\hyperref[pretty-things]{Pretty
things}}{Pretty things}}\label{pretty-things}

\subsection{\texorpdfstring{\hyperref[set-bar-to-the-texts-left]{Set bar
to the text\textquotesingle s
left}}{Set bar to the text\textquotesingle s left}}\label{set-bar-to-the-texts-left}

(also known as quote formatting)

\begin{verbatim}
#let line-block = rect.with(fill: luma(240), stroke: (left: 0.25em))

+ #lorem(10) \
  #line-block[
    *Solution:* #lorem(10)

    $ a_(n+1)x^n = 2... $
  ]
\end{verbatim}

\pandocbounded{\includesvg[keepaspectratio]{typst-img/fcddd92f117eeeb99d7b422dfc0c20a254e163e09fc5b80251a088771792ff5a-1.svg}}

\subsection{\texorpdfstring{\hyperref[text-on-box-top]{Text on box
top}}{Text on box top}}\label{text-on-box-top}

\begin{verbatim}
// author: gaiajack
#let todo(body) = block(
  above: 2em, stroke: 0.5pt + red,
  width: 100%, inset: 14pt
)[
  #set text(font: "Noto Sans", fill: red)
  #place(
    top + left,
    dy: -6pt - 14pt, // Account for inset of block
    dx: 6pt - 14pt,
    block(fill: white, inset: 2pt)[*DRAFT*]
  )
  #body
]

#todo(lorem(100))
\end{verbatim}

\pandocbounded{\includesvg[keepaspectratio]{typst-img/7a5d79c63f3a0b28ec6bdec78da80d81252ff1975b883162c84b813f938c94c0-1.svg}}

\subsection{\texorpdfstring{\hyperref[book-ornament]{Book
Ornament}}{Book Ornament}}\label{book-ornament}

\begin{verbatim}
// author: thevec

#let parSepOrnament = [\ \ #h(1fr) $#line(start:(0em,-.15em), end:(12em,-.15em), stroke: (cap: "round", paint:gradient.linear(white,black,white))) #move(dx:.5em,dy:0em,"🙠")#text(15pt)[🙣] #h(0.4em) #move(dy:-0.25em,text(12pt)[✢]) #h(0.4em) #text(15pt)[🙡]#move(dx:-.5em,dy:0em,"🙢") #line(start:(0em,-.15em), end:(12em,-.15em), stroke: (cap: "round", paint:gradient.linear(white,black,white)))$ #h(1fr)\ \ ];

#lorem(30)
#parSepOrnament
#lorem(30)
\end{verbatim}

\pandocbounded{\includesvg[keepaspectratio]{typst-img/ad56a859952fab3706dcb76434e492a9c14057bff1ee897ae2bfe3672fe17e18-1.svg}}


\section{Examples Book LaTeX/book/snippets/external.tex}
\title{sitandr.github.io/typst-examples-book/book/snippets/external}

\section{\texorpdfstring{\hyperref[use-with-external-tools]{Use with
external
tools}}{Use with external tools}}\label{use-with-external-tools}

Currently the best ways to communicate is using

\begin{enumerate}
\tightlist
\item
  Preprocessing. The tool should generate Typst file
\item
  Typst Query (CLI). See the docs
  \href{https://typst.app/docs/reference/meta/query\#command-line-queries}{there}
  .
\item
  WebAssembly plugins. See the docs
  \href{https://typst.app/docs/reference/foundations/plugin/}{there} .
\end{enumerate}

In some time there will be examples of successful usage of first two
methods. For the third one, see \href{../packages/index.html}{packages}
.


\section{Examples Book LaTeX/book/snippets/code.tex}
\title{sitandr.github.io/typst-examples-book/book/snippets/code}

\section{\texorpdfstring{\hyperref[code-formatting]{Code
formatting}}{Code formatting}}\label{code-formatting}

\subsection{\texorpdfstring{\hyperref[inline-highlighting]{Inline
highlighting}}{Inline highlighting}}\label{inline-highlighting}

\begin{verbatim}
#let r = raw.with(lang: "r")

This can then be used like: #r("x <- c(10, 42)")
\end{verbatim}

\pandocbounded{\includesvg[keepaspectratio]{typst-img/dadb41acb1c458d9af5b909d657de5f46dca019f1a81cc17b75b9863d60fa9eb-1.svg}}

\subsection{\texorpdfstring{\hyperref[tab-size]{Tab
size}}{Tab size}}\label{tab-size}

\begin{verbatim}
#set raw(tab-size: 8)
```tsv
Year  Month   Day
2000  2   3
2001  2   1
2002  3   10
```
\end{verbatim}

\pandocbounded{\includesvg[keepaspectratio]{typst-img/1c3900a79521f6b9cd852a68a3dea627ddbd1b8fc6062d3ca344e4259a30d212-1.svg}}

\subsection{\texorpdfstring{\hyperref[theme]{Theme}}{Theme}}\label{theme}

See
\href{https://typst.app/docs/reference/text/raw/\#parameters-theme}{reference}

\subsection{\texorpdfstring{\hyperref[enable-ligatures-for-code]{Enable
ligatures for
code}}{Enable ligatures for code}}\label{enable-ligatures-for-code}

\begin{verbatim}
#show raw: set text(ligatures: true, font: "Cascadia Code")

Then the code becomes `x <- a`
\end{verbatim}

\pandocbounded{\includesvg[keepaspectratio]{typst-img/3513eee2f5ca33825d09149f4ad9169abf95d3b8ad02cc5a7bf91cc9b96517d0-1.svg}}

\subsection{\texorpdfstring{\hyperref[advanced-formatting]{Advanced
formatting}}{Advanced formatting}}\label{advanced-formatting}

See \href{../packages/code.html}{packages} section.


\section{Examples Book LaTeX/book/snippets/labels.tex}
\title{sitandr.github.io/typst-examples-book/book/snippets/labels}

\section{\texorpdfstring{\hyperref[labels]{Labels}}{Labels}}\label{labels}

\subsection{\texorpdfstring{\hyperref[get-chapter-of-label]{Get chapter
of label}}{Get chapter of label}}\label{get-chapter-of-label}

\begin{verbatim}
#let ref-heading(label) = context {
  let elems = query(label)
  if elems.len() != 1 {
    panic("found multiple elements")
  }
  let element = elems.first()
  if element.func() != heading {
    panic("label must target heading")
  }
  link(label, element.body)
}

= Design <design>
#lorem(20)

= Implementation
In #ref-heading(<design>), we discussed...
\end{verbatim}

\pandocbounded{\includesvg[keepaspectratio]{typst-img/7a4a9436d9aa0cbf0d3212b45d54bd90a896181c30b036326d99dee9f58eb117-1.svg}}

\subsection{\texorpdfstring{\hyperref[allow-missing-references]{Allow
missing
references}}{Allow missing references}}\label{allow-missing-references}

\begin{verbatim}
// author: Enivex
#set heading(numbering: "1.")

#let myref(label) = context {
    if query(label).len() != 0 {
        ref(label)
    } else {
        // missing reference
        text(fill: red)[???]
    }
}

= Second <test2>

#myref(<test>)

#myref(<test2>)
\end{verbatim}

\pandocbounded{\includesvg[keepaspectratio]{typst-img/cd5f1f34e81c117063da3bb176c1dda726bbc18ac981121f75555d5834b08058-1.svg}}


\section{Examples Book LaTeX/book/snippets/gradients.tex}
\title{sitandr.github.io/typst-examples-book/book/snippets/gradients}

\section{\texorpdfstring{\hyperref[color--gradients]{Color \&
Gradients}}{Color \& Gradients}}\label{color--gradients}

\subsection{\texorpdfstring{\hyperref[gradients]{Gradients}}{Gradients}}\label{gradients}

Gradients may be very cool for presentations or just a pretty look.

\begin{verbatim}
/// author: frozolotl
#set page(paper: "presentation-16-9", margin: 0pt)
#set text(fill: white, font: "Inter")

#let grad = gradient.linear(rgb("#953afa"), rgb("#c61a22"), angle: 135deg)

#place(horizon + left, image(width: 60%, "../img/landscape.png"))

#place(top, polygon(
  (0%, 0%),
  (70%, 0%),
  (70%, 25%),
  (0%, 29%),
  fill: white,
))
#place(bottom, polygon(
  (0%, 100%),
  (100%, 100%),
  (100%, 30%),
  (60%, 30% + 60% * 4%),
  (60%, 60%),
  (0%, 64%),
  fill: grad,
))

#place(top + right, block(inset: 7pt, image(height: 19%, "../img/tub.png")))

#place(bottom, block(inset: 40pt)[
  #text(size: 30pt)[
    Presentation Title
  ]

  #text(size: 16pt)[#lorem(20) | #datetime.today().display()]
])
\end{verbatim}

\pandocbounded{\includesvg[keepaspectratio]{typst-img/19558fb55cdd5c8f8193724d92b54b1258ca8ee3d4d4c9e7077fc50d11e1a79d-1.svg}}


\section{Examples Book LaTeX/book/snippets/grids.tex}
\title{sitandr.github.io/typst-examples-book/book/snippets/grids}

\subsection{\texorpdfstring{\hyperref[fractional-grids]{Fractional
grids}}{Fractional grids}}\label{fractional-grids}

For tables with lines of changing length, you can try using \emph{grids
in grids} .

Don\textquotesingle t use this where
\href{https://typst.app/docs/reference/model/table/\#definitions-cell-colspan}{cell.colspan
and rowspan} will do.

\begin{verbatim}
// author: jimpjorps

#grid(
  columns: (1fr,),
  grid(
    columns: (1fr,)*2, inset: 5pt, stroke: 1pt, [hello], [world]
  ),
  grid(
    columns: (1fr,)*3, inset: 5pt, stroke: 1pt, [foo], [bar], [baz]
  ),
  grid.cell(inset: 5pt, stroke: 1pt)[abcxyz]
)
\end{verbatim}

\pandocbounded{\includesvg[keepaspectratio]{typst-img/5b2869a2b2efca1af57cb7ed6fab90ad0c83a35b76c05258a1ae64096d5a8173-1.svg}}

\subsection{\texorpdfstring{\hyperref[automerge-adjacent-cells-with-same-values]{Automerge
adjacent cells with same
values}}{Automerge adjacent cells with same values}}\label{automerge-adjacent-cells-with-same-values}

This example works for adjacent cells horizontally, but
it\textquotesingle s not hard to upgrade it to columns too.

\begin{verbatim}
// author: tebine
#let merge(children, n-cols) = {
  let rows = children.chunks(n-cols)
  let new-children = ()
  for r in rows {
    // First group starts at index 0
    let i = 0 
    // Search next group
    while i < r.len() {
      // Group starts with one cell
      let c = r.at(i).body
      let n = 1
      for j in range(i+1, r.len()) {
        let c-next = r.at(j).body
        if c-next == c {
          // Add cell to group
          n += 1
        } else {
          break
        }
      }
      // Group is finished
      new-children.push(table.cell(colspan: n, c))
      i += n
    }
  }
  return new-children
}
#show table: it => {
  let merged = merge(it.children, it.columns.len())
  if it.children.len() == merged.len() { // trick to avoid recursion
    return it
  }
  table(columns: it.columns.len(), ..merged)
}
#table(columns: 2,
  [1], [2],
  [3], [3],
  [4], [5],
)
\end{verbatim}

\pandocbounded{\includesvg[keepaspectratio]{typst-img/5bf649017afba6f1af8a5ae7e6a1e8b614def90749a092f92e5886a58b351205-1.svg}}

\subsection{\texorpdfstring{\hyperref[slanted-column-headers-with-slanted-borders]{Slanted
column headers with slanted
borders}}{Slanted column headers with slanted borders}}\label{slanted-column-headers-with-slanted-borders}

\begin{verbatim}
// author: tebine
#let slanted(it, alpha: 45deg, len: 2.5cm) = layout(size => {
  let width = size.width
  let b = box(inset: 5pt, rotate(-alpha, reflow: true, it))
  let b-size = measure(b)
  let l = line(angle: -alpha, length: len)
  let l-width = len * calc.cos(alpha)
  let l-height = len * calc.sin(alpha)
  place(bottom+left, l)
  place(bottom+left, l, dx: width)
  place(bottom+left, line(length: width), dx: l-width, dy: -l-height)
  place(bottom+left, dx: width/2, b)
  box(height: l-height) // invisible box to set the height
})

#table(
  columns: 2,
  align: center,
  table.header(
    table.cell(stroke: none, inset: 0pt, slanted[*AAA*]),
    table.cell(stroke: none, inset: 0pt, slanted[*BBBBBB*]),
  ),
  [aaaaa], [bbbbbb], [c], [d],
)
\end{verbatim}

\pandocbounded{\includesvg[keepaspectratio]{typst-img/d5f49858e9acc4bad217904e87abb368aa5e38652bdcba27a971b3ddd10f0361-1.svg}}


\section{Examples Book LaTeX/book/snippets/logos.tex}
\title{sitandr.github.io/typst-examples-book/book/snippets/logos}

\section{\texorpdfstring{\hyperref[logos--figures]{Logos \&
Figures}}{Logos \& Figures}}\label{logos--figures}

Using SVG-s images is totally fine. Totally. But if you are lazy and
don\textquotesingle t want to search for images, here are some logos you
can just copy-paste in your document.

\textbf{Important} : \emph{Typst in text doesn\textquotesingle t need a
special writing (unlike LaTeX)} . Just write "Typst", maybe "
\textbf{Typst} ", and it is okay.

\subsection{\texorpdfstring{\hyperref[tex-and-latex]{TeX and
LaTeX}}{TeX and LaTeX}}\label{tex-and-latex}

\begin{verbatim}
#let TeX = {
  set text(font: "New Computer Modern", weight: "regular")
  box(width: 1.7em, {
    [T]
    place(top, dx: 0.56em, dy: 0.22em)[E]
    place(top, dx: 1.1em)[X]
  })
}

#let LaTeX = {
  set text(font: "New Computer Modern", weight: "regular")
  box(width: 2.55em, {
    [L]
    place(top, dx: 0.3em, text(size: 0.7em)[A])
    place(top, dx: 0.7em)[#TeX]
  })
}

Typst is not that hard to learn when you know #TeX and #LaTeX.
\end{verbatim}

\pandocbounded{\includesvg[keepaspectratio]{typst-img/9432efecd4502f681e3582d8581d0c325e0a89729d57b6d4bea732c2b9f476ec-1.svg}}

\subsection{\texorpdfstring{\hyperref[typst-guy]{Typst
guy}}{Typst guy}}\label{typst-guy}

\begin{verbatim}
// author: fenjalien
#import "@preview/cetz:0.1.2": *

#set page(width: auto, height: auto)

#canvas(length: 1pt, {
  import draw: *
  let color = rgb("239DAD")
  scale((y: -1))
  set-style(fill: color, stroke: none,)

  // body
  merge-path({
    bezier(
      (112.847, 134.007),
      (114.835, 143.178),
      (112.847, 138.562),
      (113.509, 141.619),
      name: "b"
    )
    bezier(
      "b.end",
      (122.063, 145.515),
      (116.16, 144.736),
      (118.569, 145.515),
      name: "b"
    )
    bezier(
      "b.end",
      (135.977, 140.121),
      (125.677, 145.515),
      (130.315, 143.717)
    )
    bezier(
      (139.591, 146.055),
      (113.389, 159.182),
      (128.99, 154.806),
      (120.256, 159.182),
      name: "b"
    )
    bezier(
      "b.end",
      (97.1258, 154.327),
      (106.522, 159.182),
      (101.101, 157.563),
      name: "b"
    )
    bezier(
      "b.end",
      (91.1626, 136.704),
      (93.1503, 150.97),
      (91.1626, 145.096),
      name: "b"
    )
    line(
      (rel: (0, -47.1126), to: "b.end"),
      (rel: (-9.0352, 0)),
      (80.6818, 82.9381),
      (91.1626, 79.7013),
      (rel: (0, -8.8112)),
      (112.847, 61),
      (rel: (0, 19.7802)),
      (134.17, 79.1618),
      (132.182, 90.8501),
      (112.847, 90.1309)
    )
  })

  // left pupil
  merge-path({
    bezier(
      (70.4667, 65.6833),
      (71.9727, 70.5068),
      (71.4946, 66.9075),
      (71.899, 69.4091)
    )
    bezier(
      (71.9727, 70.5068),
      (75.9104, 64.5912),
      (72.9675, 69.6715),
      (75.1477, 67.319)
    )
    bezier(
      (75.9104, 64.5912),
      (72.0556, 60.0005),
      (76.8638, 61.1815),
      (74.4045, 59.7677)
    )
    bezier(
      (72.0556, 60.0005),
      (66.833, 64.3859),
      (70.1766, 60.1867),
      (67.7909, 63.0017)
    )
    bezier(
      (66.833, 64.3859),
      (70.4667, 65.6833),
      (67.6159, 64.3083),
      (69.4388, 64.4591)
    )
  })

  // right pupil
  merge-path({
    bezier(
      (132.37, 61.668),
      (133.948, 66.7212),
      (133.447, 62.9505),
      (133.87, 65.5712)
    )
    bezier(
      (133.948, 66.7212),
      (138.073, 60.5239),
      (134.99, 65.8461),
      (137.274, 63.3815)
    )
    bezier(
      (138.073, 60.5239),
      (134.034, 55.7145),
      (139.066, 56.9513),
      (136.495, 55.4706)
    )
    bezier(
      (134.034, 55.7145),
      (128.563, 60.3087),
      (132.066, 55.9066),
      (129.567, 58.8586),
    )
    bezier(
      (128.563, 60.3087),
      (132.37, 61.668),
      (129.383, 60.2274),
      (131.293, 60.3855),
    )
  })

  set-style(
    stroke: (paint: rgb("239DAD"), thickness: 6pt, cap: "round"),
    fill: none,
  )

  // left eye
  merge-path({
    bezier(
      (58.5, 64.7273),
      (73.6136, 52),
      (58.5, 58.3636),
      (64.0682, 52.7955),
      name: "b"
    )
    bezier(
      "b.end",
      (84.75, 64.7273),
      (81.5682, 52),
      (84.75, 57.5682),
      name: "b"
    )
    bezier(
      "b.end",
      (71.2273, 76.6591),
      (84.75, 71.8864),
      (79.1818, 76.6591),
      name: "b"
    )
    bezier(
      "b.end",
      (58.5, 64.7273),
      (63.2727, 76.6591),
      (58.5, 71.0909)
    )
  })
  // eye lash
  line(
    (62.5, 55),
    (59.5, 52),
  )

  merge-path({
    bezier(
      (146.5, 61.043),
      (136.234, 49),
      (146.5, 52.7634),
      (141.367, 49)
    )
    bezier(
      (136.234, 49),
      (121.569, 62.5484),
      (125.969, 49),
      (120.836, 54.2688)
    )
    bezier(
      (121.569, 62.5484),
      (134.034, 72.3333),
      (122.302, 70.8279),
      (128.168, 72.3333)
    )
    bezier(
      (134.034, 72.3333),
      (146.5, 61.043),
      (139.901, 72.3333),
      (146.5, 69.3225)
    )
  })

  set-style(stroke: (thickness: 4pt))

  // right arm
  merge-path({
    bezier(
      (109.523, 115.614),
      (127.679, 110.918),
      (115.413, 115.3675),
      (122.283, 113.112)
    )
    bezier(
      (127.679, 110.918),
      (137, 106.591),
      (130.378, 109.821),
      (132.708, 108.739)
    )
  })

  // right first finger
  bezier(
    (137, 106.591),
    (140.5, 98.0908),
    (137.385, 102.891),
    (138.562, 99.817)
  )

  // right second finger
  bezier(
    (137, 106.591),
    (146, 101.591),
    (139.21, 103.799),
    (142.425, 101.713)
  )

  // right third finger
  line(
    (137, 106.591),
    (148, 106.591)
  )

  //right forth finger
  bezier(
    (137, 106.591),
    (146, 111.091),
    (140.243, 109.552),
    (143.119, 110.812)
  )

  // left arm
  bezier(
    (95.365, 116.979),
    (73.5, 107.591),
    (88.691, 115.549),
    (80.587, 112.887)
  )

  // left first finger
  line(
    (73.5, 107.591),
    (rel: (0, -9.5))
  )
  // left second finger
  line(
    (73.5, 107.591),
    (65.396, 100.824)
  )
  // left third finger
  line(
    (73.5, 107.591),
    (63.012, 105.839)
  )
  // left fourth finger
  bezier(
    (73.5, 107.591),
    (63.012, 111.04),
    (70.783, 109.121),
    (67.214, 111.255)
  )
})
\end{verbatim}

\pandocbounded{\includesvg[keepaspectratio]{typst-img/4a142b60394d5730a373a7ee2229a3a42a8af8f31b314c70b5bd192210982b09-1.svg}}




\section{Combined Examples Book LaTeX/book/snippets/scripting.tex}
\section{Examples Book LaTeX/book/snippets/scripting/index.tex}
\title{sitandr.github.io/typst-examples-book/book/snippets/scripting/index}

\section{\texorpdfstring{\hyperref[scripting]{Scripting}}{Scripting}}\label{scripting}

\subsection{\texorpdfstring{\hyperref[unflatten-arrays]{Unflatten
arrays}}{Unflatten arrays}}\label{unflatten-arrays}

\begin{verbatim}
// author: PgSuper
#let unflatten(arr, n) = {
  let columns = range(0, n).map(_ => ())
  for (i, x) in arr.enumerate() {
    columns.at(calc.rem(i, n)).push(x)
  }
  array.zip(..columns)
}

#unflatten((1, 2, 3, 4, 5, 6), 2)
#unflatten((1, 2, 3, 4, 5, 6), 3)
\end{verbatim}

\pandocbounded{\includesvg[keepaspectratio]{typst-img/98271a255f0fb10f31ba1d8199ba5a91ebb6f647cdd570220f95e1b88d193ca0-1.svg}}

\subsection{\texorpdfstring{\hyperref[create-an-abbreviation]{Create an
abbreviation}}{Create an abbreviation}}\label{create-an-abbreviation}

\begin{verbatim}
#let full-name = "Federal University of Ceará"

#let letts = {
  full-name
    .split()
    .map(word => word.at(0)) // filter only capital letters
    .filter(l => upper(l) == l)
    .join()
}
#letts
\end{verbatim}

\pandocbounded{\includesvg[keepaspectratio]{typst-img/e95b77243a1305a47517cb128577d1c7633d858561de0ef797ff551f35be40de-1.svg}}

\subsection{\texorpdfstring{\hyperref[split-the-string-retrieving-separators]{Split
the string retrieving
separators}}{Split the string retrieving separators}}\label{split-the-string-retrieving-separators}

\begin{verbatim}
#",this, is a a a a; a. test? string!".matches(regex("(\b[\P{Punct}\s]+\b|\p{Punct})")).map(x => x.captures).join()
\end{verbatim}

\pandocbounded{\includesvg[keepaspectratio]{typst-img/c5d183e45097449e4f52b07f82185847092ad28bcad3b9474093d341c4b07c4a-1.svg}}

\subsection{\texorpdfstring{\hyperref[create-selector-matching-any-values-in-an-array]{Create
selector matching any values in an
array}}{Create selector matching any values in an array}}\label{create-selector-matching-any-values-in-an-array}

This snippet creates a selector (that is then used in a show rule) that
matches any of the values inside the array. Here, it is used to
highlight a few raw lines, but it can be easily adapted to any kind of
selector.

\begin{verbatim}
// author: Blokyk
#let lines = (2, 3, 5)
#let lines-selectors = lines.map(lineno => raw.line.where(number: lineno))
#let lines-combined-selector = lines-selectors.fold(
  // start with the first selector by default
  // you can also use a selector that wouldn't ever match anything, if possible
  lines-selectors.at(0),
  selector.or // create an OR of all selectors (alternatively: (acc, sel) => acc.or(sel))
)

#show lines-combined-selector: highlight

```py
def foo(x, y):
  if x == y:
    return False
  z = x + y
  return z * x - z * y >= z
```
\end{verbatim}

\pandocbounded{\includesvg[keepaspectratio]{typst-img/085d1ae3a0672ba278edcde3ebb229a34a40ab5166d0b6d5b469d838b9262a51-1.svg}}

\subsection{\texorpdfstring{\hyperref[synthesize-show-or-show-set-rules-from-dictionnary]{Synthesize
show (or show-set) rules from
dictionnary}}{Synthesize show (or show-set) rules from dictionnary}}\label{synthesize-show-or-show-set-rules-from-dictionnary}

This snippet applies a show-set rule to any element inside a dictionary,
by using the key as the selector and the value as the parameter to set.
In this example, it\textquotesingle s used to give custom supplements to
custom figure kinds, based on a dictionnary of correspondances.

\begin{verbatim}
// author: laurmaedje
#let kind_supp_dict = (
  algo: "Pseudo-code",
  ex: "Example",
  prob: "Problem",
)

// apply this rule to the whole (rest of the) document
#show: it => {
  kind_supp_dict
    .pairs() // get an array of key-value pairs
    .fold( // we're going to stack show-set rules before the document
      it, // start with the default document
      (acc, (kind, supp)) => {
        // add the curent kind-supp combination on top of the rest
        show figure.where(kind: kind): set figure(supplement: supp)
        acc
      }
    )
}
#figure(
    kind: "algo",
    caption: [My code],
    ```Algorithm there```
)
\end{verbatim}

\pandocbounded{\includesvg[keepaspectratio]{typst-img/9de9b5f4bb801735b13ffafe54d35ebcfc78f1df78a34a8ab90f8a6c350b986e-1.svg}}

Additonnaly, as this is applied at the position where you write it,
these show-set rules will appear as if they were added in the same place
where you wrote this rule. This means that you can override them later,
just like any other show-set rules.




\section{Combined Examples Book LaTeX/book/snippets/text.tex}
\section{Examples Book LaTeX/book/snippets/text/individual_lang_fonts.tex}
\title{sitandr.github.io/typst-examples-book/book/snippets/text/individual_lang_fonts}

\section{\texorpdfstring{\hyperref[individual-language-fonts]{Individual
language
fonts}}{Individual language fonts}}\label{individual-language-fonts}

\begin{verbatim}
A cat แปลว่า แมว

#show regex("\p{Thai}+"): text.with(font: "Noto Serif Thai")

A cat แปลว่า แมว
\end{verbatim}

\pandocbounded{\includesvg[keepaspectratio]{typst-img/612267fd94fab114a3e0b75bdb3785b818c0f83427071db0dce086d1b0a6a54a-1.svg}}


\section{Examples Book LaTeX/book/snippets/text/text_shadows.tex}
\title{sitandr.github.io/typst-examples-book/book/snippets/text/text_shadows}

\section{\texorpdfstring{\hyperref[fake-italic--text-shadows]{Fake
italic \& Text
shadows}}{Fake italic \& Text shadows}}\label{fake-italic--text-shadows}

\subsection{\texorpdfstring{\hyperref[skew]{Skew}}{Skew}}\label{skew}

\begin{verbatim}
// author: Enivex
#set page(width: 21cm, height: 3cm)
#set text(size:25pt)
#let skew(angle,vscale: 1,body) = {
  let (a,b,c,d)= (1,vscale*calc.tan(angle),0,vscale)
  let E = (a + d)/2
  let F = (a - d)/2
  let G = (b + c)/2
  let H = (c - b)/2
  let Q = calc.sqrt(E*E + H*H)
  let R = calc.sqrt(F*F + G*G)
  let sx = Q + R
  let sy = Q - R
  let a1 = calc.atan2(F,G)
  let a2 = calc.atan2(E,H)
  let theta = (a2 - a1) /2
  let phi = (a2 + a1)/2

  set rotate(origin: bottom+center)
  set scale(origin: bottom+center)

  rotate(phi,scale(x: sx*100%, y: sy*100%,rotate(theta,body)))
}

#let fake-italic(body) = skew(-12deg,body)
#fake-italic[This is fake italic text]

#let shadowed(body) = box(place(skew(-50deg, vscale: 0.8, text(fill:luma(200),body)))+place(body))
#shadowed[This is some fancy text with a shadow]
\end{verbatim}

\pandocbounded{\includesvg[keepaspectratio]{typst-img/1c00de41a0643ecf254de80601efa4a043302c1e76aedfbf2458a9e30f1c7fd3-1.svg}}




\section{Combined Examples Book LaTeX/book/snippets/layout.tex}
\section{Examples Book LaTeX/book/snippets/layout/multiline_detect.tex}
\title{sitandr.github.io/typst-examples-book/book/snippets/layout/multiline_detect}

\section{\texorpdfstring{\hyperref[multiline-detection]{Multiline
detection}}{Multiline detection}}\label{multiline-detection}

Detects if figure caption (may be any other element) \emph{has more than
one line} .

If the caption is multiline, it makes it left-aligned.

Breaks on manual linebreaks.

\begin{verbatim}
#show figure.caption: it => {
  layout(size => context [
    #let text-size = measure(
      ..size,
      it.supplement + it.separator + it.body,
    )

    #let my-align

    #if text-size.width < size.width {
      my-align = center
    } else {
      my-align = left
    }

    #align(my-align, it)
  ])
}

#figure(caption: lorem(6))[
    ```rust
    pub fn main() {
        println!("Hello, world!");
    }
    ```
]

#figure(caption: lorem(20))[
    ```rust
    pub fn main() {
        println!("Hello, world!");
    }
    ```
]
\end{verbatim}

\pandocbounded{\includesvg[keepaspectratio]{typst-img/8e2a1d9e2e66f654938733a2ed1d9a0dcc771165a60d89c4410f4d970054121c-1.svg}}


\section{Examples Book LaTeX/book/snippets/layout/hiding.tex}
\title{sitandr.github.io/typst-examples-book/book/snippets/layout/hiding}

\section{\texorpdfstring{\hyperref[hiding-things]{Hiding
things}}{Hiding things}}\label{hiding-things}

\begin{verbatim}
// author: GeorgeMuscat
#let redact(text, fill: black, height: 1em) = {
  box(rect(fill: fill, height: height)[#hide(text)])
}

Example:
  - Unredacted text
  - Redacted #redact("text")
\end{verbatim}

\pandocbounded{\includesvg[keepaspectratio]{typst-img/6b85fdf4b9ba387543271058b6acb27e202dab93b01c2cd7ac93187c1e8b643c-1.svg}}


\section{Examples Book LaTeX/book/snippets/layout/duplicate.tex}
\title{sitandr.github.io/typst-examples-book/book/snippets/layout/duplicate}

\section{\texorpdfstring{\hyperref[duplicate-content]{Duplicate
content}}{Duplicate content}}\label{duplicate-content}

Notice that this implementation will mess up with labels and similar
things. For complex cases see one below.

```typ \#set page(paper: "a4", flipped: true) \#show: body
=\textgreater{} grid( columns: (1fr, 1fr), column-gutter: 1cm, body,
body, ) \#lorem(200) ```

\subsection{\texorpdfstring{\hyperref[advanced]{Advanced}}{Advanced}}\label{advanced}

\begin{verbatim}
/// author: frozolotl
#set page(paper: "a4", flipped: true)
#set heading(numbering: "1.1")
#show ref: it => {
  if it.element != none {
    it
  } else {
    let targets = query(it.target, it.location())
    if targets.len() == 2 {
      let target = targets.first()
      if target.func() == heading {
        let num = numbering(target.numbering, ..counter(heading).at(target.location()))
        [#target.supplement #num]
      } else if target.func() == figure {
        let num = numbering(target.numbering, ..target.counter.at(target.location()))
        [#target.supplement #num]
      } else {
        it
      }
    } else {
      it
    }
  }
}
#show link: it => context {
  let dest = query(it.dest)
  if dest.len() == 2 {
    link(dest.first().location(), it.body)
  } else {
    it
  }
}
#show: body => context grid(
  columns: (1fr, 1fr),
  column-gutter: 1cm,
  body,
  {
    let reset-counter(kind) = counter(kind).update(counter(kind).get())
    reset-counter(heading)
    reset-counter(figure.where(kind: image))
    reset-counter(figure.where(kind: raw))
    set heading(outlined: false)
    set figure(outlined: false)
    body
  },
)

#outline()

= Foo <foo>
See @foo and @foobar.

#figure(rect[This is an image], caption: [Foobar], kind: raw) <foobar>

== Bar
== Baz
#link(<foo>)[Click to visit Foo]
\end{verbatim}

\pandocbounded{\includesvg[keepaspectratio]{typst-img/2fdcc2778a936608ed868521793f59311ac54d43e226639db3ab14c6ca37c75f-1.svg}}


\section{Examples Book LaTeX/book/snippets/layout/shapes.tex}
\title{sitandr.github.io/typst-examples-book/book/snippets/layout/shapes}

\section{\texorpdfstring{\hyperref[shaped-boxes-with-text]{Shaped boxes
with text}}{Shaped boxes with text}}\label{shaped-boxes-with-text}

(I guess that will make a package eventually, but let it be a snippet
for now)

\begin{verbatim}
/// author: JustForFun88
#import "@preview/oxifmt:0.2.0": strfmt

#let shadow_svg_path = `
<svg
    width="{canvas-width}"
    height="{canvas-height}"
    viewBox="{viewbox}"
    version="1.1"
    xmlns="http://www.w3.org/2000/svg"
    xmlns:svg="http://www.w3.org/2000/svg">
    <!-- Definitions for reusable components -->
    <defs>
        <filter id="shadowing" >
            <feGaussianBlur in="SourceGraphic" stdDeviation="{blur}" />
        </filter>
    </defs>

    <!-- Drawing the rectangle with a fill and feGaussianBlur effect -->
    <path
        style="fill: {flood-color}; opacity: {flood-opacity}; filter:url(#shadowing)"
        d="{vertices} Z" />
</svg>
`.text

#let parallelogram(width: 20mm, height: 5mm, angle: 30deg) = {
  let δ = height * calc.tan(angle)
  (
    (      + δ,     0pt   ),
    (width + δ * 2, 0pt   ),
    (width + δ,     height),
    (0pt,           height),
  )
}

#let hexagon(width: 100pt, height: 30pt, angle: 30deg) = {
  let dy = height / 2;
  let δ = dy * calc.tan(angle)
  (
    (0pt,           dy    ),
    (      + δ,     0pt   ),
    (width + δ,     0pt   ),
    (width + δ * 2, dy    ),
    (width + δ,     height),
    (      + δ,     height),
  )
}

#let shape_size(vertices) = {
    let x_vertices = vertices.map(array.first);
    let y_vertices = vertices.map(array.last);

    (
      calc.max(..x_vertices) - calc.min(..x_vertices),
      calc.max(..y_vertices) - calc.min(..y_vertices)
    )
}

#let shadowed_shape(shape: hexagon, fill: none,
  stroke: auto, angle: 30deg, shadow_fill: black, alpha: 0.5, 
  blur: 1.5, blur_margin: 5, dx: 0pt, dy: 0pt, ..args, content
) = layout(size => context {
    let named = args.named()
    for key in ("width", "height") {
      if key in named and type(named.at(key)) == ratio {
        named.insert(key, size.at(key) * named.at(key))
      }
    }

    let opts = (blur: blur, flood-color: shadow_fill.to-hex())
       
    let content = box(content, ..named)
    let size = measure(content)

    let vertices = shape(..size, angle: angle)
    let (shape_width, shape_height) = shape_size(vertices)
    let margin = opts.blur * blur_margin * 1pt

    opts += (
      canvas-width:  shape_width  + margin,
      canvas-height: shape_height + margin,
      flood-opacity: alpha
    )

    opts.viewbox = (0, 0, opts.canvas-width.pt(), opts.canvas-height.pt()).map(str).join(",")

    opts.vertices = "";
    let d = margin / 2;
    for (i, p) in vertices.enumerate() {
        let prefix = if i == 0 { "M " } else { " L " };
        opts.vertices += prefix + p.map(x => str((x + d).pt())).join(", ");
    }

    let svg-shadow = image.decode(strfmt(shadow_svg_path, ..opts))
    place(dx: dx, dy: dy, svg-shadow)
    place(path(..vertices, fill: fill, stroke: stroke, closed: true))
    box(h((shape_width - size.width) / 2) + content, width: shape_width)
})

#set text(3em);

#shadowed_shape(shape: hexagon,
    inset: 1em, fill: teal,
    stroke: 1.5pt + teal.darken(50%),
    shadow_fill: red,
    dx: 0.5em, dy: 0.35em, blur: 3)[Hello there!]
#shadowed_shape(shape: parallelogram,
    inset: 1em, fill: teal,
    stroke: 1.5pt + teal.darken(50%),
    shadow_fill: red,
    dx: 0.5em, dy: 0.35em, blur: 3)[Hello there!]
\end{verbatim}

\pandocbounded{\includesvg[keepaspectratio]{typst-img/f40acb7d6d2753b0845c9dd1fb26979c29dd0850448cf585f0c7f1b20acde7ea-1.svg}}


\section{Examples Book LaTeX/book/snippets/layout/insert_lines.tex}
\title{sitandr.github.io/typst-examples-book/book/snippets/layout/insert_lines}

\section{\texorpdfstring{\hyperref[lines-between-list-items]{Lines
between list
items}}{Lines between list items}}\label{lines-between-list-items}

\begin{verbatim}
/// author: frozolotl
#show enum.where(tight: false): it => {
  it.children
    .enumerate()
    .map(((n, item)) => block(below: .6em, above: .6em)[#numbering("1.", n + 1) #item.body])
    .join(line(length: 100%))
}

+ Item 1

+ Item 2

+ Item 3
\end{verbatim}

\pandocbounded{\includesvg[keepaspectratio]{typst-img/b1660863fded6fc3d870f8a92f364040d5ba9beaaf5bbd4a114b5384abe3db4c-1.svg}}

The same approach may be easily adapted to style the enums as you want.


\section{Examples Book LaTeX/book/snippets/layout/page_setup.tex}
\title{sitandr.github.io/typst-examples-book/book/snippets/layout/page_setup}

\section{\texorpdfstring{\hyperref[page-setup]{Page
setup}}{Page setup}}\label{page-setup}

\begin{quote}
See \href{https://typst.app/docs/guides/page-setup-guide/}{Official Page
Setup guide}
\end{quote}

\begin{verbatim}
#set page(
  width: 3cm,
  margin: (x: 0cm),
)

#for i in range(3) {
  box(square(width: 1cm))
}
\end{verbatim}

\pandocbounded{\includesvg[keepaspectratio]{typst-img/6a1e9261d0b0bcd09b578e8361c939100328fbccfd8289402ad62f768b55a0c1-1.svg}}

\begin{verbatim}
#set page(columns: 2, height: 4.8cm)
Climate change is one of the most
pressing issues of our time, with
the potential to devastate
communities, ecosystems, and
economies around the world. It's
clear that we need to take urgent
action to reduce our carbon
emissions and mitigate the impacts
of a rapidly changing climate.
\end{verbatim}

\pandocbounded{\includesvg[keepaspectratio]{typst-img/2b0351806e86c3410f445beb2a51887aebd3f73649e2fe638ba45a39026284dd-1.svg}}

\begin{verbatim}
#set page(fill: rgb("444352"))
#set text(fill: rgb("fdfdfd"))
*Dark mode enabled.*
\end{verbatim}

\pandocbounded{\includesvg[keepaspectratio]{typst-img/340892f7237f4bc864f9ca9dc5fd956fe4032a157a373e0bb4b7358200daa72e-1.svg}}

\begin{verbatim}
#set par(justify: true)
#set page(
  margin: (top: 32pt, bottom: 20pt),
  header: [
    #set text(8pt)
    #smallcaps[Typst Academcy]
    #h(1fr) _Exercise Sheet 3_
  ],
)

#lorem(19)
\end{verbatim}

\pandocbounded{\includesvg[keepaspectratio]{typst-img/bfb28329922a1eb129dd2c7d7003dcfa30ebdc119265f19f8190b69d3e40ff68-1.svg}}

\begin{verbatim}
#set page(foreground: text(24pt)[🥸])

Reviewer 2 has marked our paper
"Weak Reject" because they did
not understand our approach...
\end{verbatim}

\pandocbounded{\includesvg[keepaspectratio]{typst-img/b88eae1fcb87b110e66ee6493c60c2c3e0d0c9a7f1c288e739bf1bb8e09c8d70-1.svg}}




\section{Combined Examples Book LaTeX/book/snippets/math.tex}
\section{Examples Book LaTeX/book/snippets/math/calligraphic.tex}
\title{sitandr.github.io/typst-examples-book/book/snippets/math/calligraphic}

\section{\texorpdfstring{\hyperref[calligraphic-letters]{Calligraphic
letters}}{Calligraphic letters}}\label{calligraphic-letters}

\begin{verbatim}
#let scr(it) = math.class("normal",
  text(font: "", stylistic-set: 1, $cal(it)$) + h(0em)
)

$ scr(A) scr(B) + scr(C), -scr(D) $
\end{verbatim}

\pandocbounded{\includesvg[keepaspectratio]{typst-img/6ee9ca10515c1b6158d8d7bddd4418a713313052c0114fe851be455fc09b2c92-1.svg}}

Unfortunately, currently just
\texttt{\ }{\texttt{\ stylistic-set\ }}\texttt{\ } for math creates bad
spacing. Math engine detects if the letter should be correctly spaced by
whether it is the default font. However, just making it "normal"
isn\textquotesingle t enough, because than it can be reduced.
That\textquotesingle s way the snippet is as hacky as it is (probably
should be located in Typstonomicon, but it\textquotesingle s not large
enough).


\section{Examples Book LaTeX/book/snippets/math/numbering.tex}
\title{sitandr.github.io/typst-examples-book/book/snippets/math/numbering}

\section{\texorpdfstring{\hyperref[math-numbering]{Math
Numbering}}{Math Numbering}}\label{math-numbering}

\subsection{\texorpdfstring{\hyperref[number-by-current-heading]{Number
by current
heading}}{Number by current heading}}\label{number-by-current-heading}

\begin{quote}
See also built-in numbering in
\href{../../packages/math.html\#theorems}{math package section}
\end{quote}

\begin{verbatim}
/// original author: laurmaedje
#set heading(numbering: "1.")

// reset counter at each chapter
// if you want to change the number of displayed 
// section numbers, change the level there
#show heading.where(level:1): it => {
  counter(math.equation).update(0)
  it
}

#set math.equation(numbering: n => {
  numbering("(1.1)", counter(heading).get().first(), n)
  // if you want change the number of number of displayed
  // section numbers, modify it this way:
  /*
  let count = counter(heading).get()
  let h1 = count.first()
  let h2 = count.at(1, default: 0)
  numbering("(1.1.1)", h1, h2, n)
  */
})


= Section
== Subsection

$ 5 + 3 = 8 $ <a>
$ 5 + 3 = 8 $

= New Section
== Subsection
$ 5 + 3 = 8 $
== Subsection
$ 5 + 3 = 8 $ <b>

Mentioning @a and @b.
\end{verbatim}

\pandocbounded{\includesvg[keepaspectratio]{typst-img/9662902bb463e350d7a9bdf94e143bbaab8245da34eee4a426d2263d44511d1f-1.svg}}

\subsection{\texorpdfstring{\hyperref[number-only-labeled-equations]{Number
only labeled
equations}}{Number only labeled equations}}\label{number-only-labeled-equations}

\subsubsection{\texorpdfstring{\hyperref[simple-code]{Simple
code}}{Simple code}}\label{simple-code}

\begin{verbatim}
// author: shampoohere
#show math.equation:it => {
  if it.fields().keys().contains("label"){
    math.equation(block: true, numbering: "(1)", it)
    // Don't forget to change your numbering style in `numbering`
    // to the one you actually want to use.
    //
    // Note that you don't need to #set the numbering now.
  } else {
    it
  }
}

$ sum_x^2 $
$ dif/(dif x)(A(t)+B(x))=dif/(dif x)A(t)+dif/(dif x)B(t) $ <ep-2>
$ sum_x^2 $
$ dif/(dif x)(A(t)+B(x))=dif/(dif x)A(t)+dif/(dif x)B(t) $ <ep-3>
\end{verbatim}

\pandocbounded{\includesvg[keepaspectratio]{typst-img/84052f83d0e2e2c330ef041c254dfb7c735526fc7f47cdb14ecc46961f66fee3-1.svg}}

\subsubsection{\texorpdfstring{\hyperref[make-the-hacked-references-clickable-again]{Make
the hacked references clickable
again}}{Make the hacked references clickable again}}\label{make-the-hacked-references-clickable-again}

\begin{verbatim}
// author: gijsleb
#show math.equation:it => {
  if it.has("label") {
    // Don't forget to change your numbering style in `numbering`
    // to the one you actually want to use.
    math.equation(block: true, numbering: "(1)", it)
  } else {
    it
  }
}

#show ref: it => {
  let el = it.element
  if el != none and el.func() == math.equation {
    link(el.location(), numbering(
      // don't forget to change the numbering according to the one
      // you are actually using (e.g. section numbering)
      "(1)",
      counter(math.equation).at(el.location()).at(0) + 1
    ))
  } else {
    it
  }
}

$ sum_x^2 $
$ dif/(dif x)(A(t)+B(x))=dif/(dif x)A(t)+dif/(dif x)B(t) $ <ep-2>
$ sum_x^2 $
$ dif/(dif x)(A(t)+B(x))=dif/(dif x)A(t)+dif/(dif x)B(t) $ <ep-3>
In @ep-2 and @ep-3 we see equations
\end{verbatim}

\pandocbounded{\includesvg[keepaspectratio]{typst-img/c17ccdb8d65b5aa0d0e58b1fba75c67bfc162400ba86ad64f37aa038ad6d8887-1.svg}}


\section{Examples Book LaTeX/book/snippets/math/scripts.tex}
\title{sitandr.github.io/typst-examples-book/book/snippets/math/scripts}

\section{\texorpdfstring{\hyperref[scripts]{Scripts}}{Scripts}}\label{scripts}

\begin{quote}
To set scripts and limits see \href{../../basics/math/limits.html}{Typst
Basics section}
\end{quote}

\subsection{\texorpdfstring{\hyperref[make-every-character-upright-when-used-in-subscript]{Make
every character upright when used in
subscript}}{Make every character upright when used in subscript}}\label{make-every-character-upright-when-used-in-subscript}

\begin{verbatim}
// author: emilyyyylime

$f_a, f_b, f^a, f_italic("word")$
#show math.attach: it => {
  import math: *
  if it.b != none and it.b.func() != upright[].func() and it.b.has("text") and it.b.text.len() == 1 {
    let args = it.fields()
    let _ = args.remove("base")
    let _ = args.remove("b")
    attach(it.base, b: upright(it.b), ..args)
  } else {
    it
  }
}

$f_a, f_b, f^a, f_italic("word")$
\end{verbatim}

\pandocbounded{\includesvg[keepaspectratio]{typst-img/40b68a1d7f5aeb1c498431996be1b140b3d217ba5b0230b6b73e6fefe64d45cd-1.svg}}


\section{Examples Book LaTeX/book/snippets/math/vecs.tex}
\title{sitandr.github.io/typst-examples-book/book/snippets/math/vecs}

\section{\texorpdfstring{\hyperref[vectors--matrices]{Vectors \&
Matrices}}{Vectors \& Matrices}}\label{vectors--matrices}

You can easily note that the gap isn\textquotesingle t necessarily even
or the same in different vectors and matrices:

\begin{verbatim}
$
mat(0, 1, -1; -1, 0, 1; 1, -1, 0) vec(a/b, a/b, a/b) = vec(c, d, e)
$
\end{verbatim}

\pandocbounded{\includesvg[keepaspectratio]{typst-img/6a28529f5b38a17bcb660981691cdcc2bc4b6ddfbdb103327ae9e42b1365458e-1.svg}}

That happens because \texttt{\ }{\texttt{\ gap\ }}\texttt{\ } refers to
\emph{spacing between} elements, not the distance between their centers.

To fix this, you can use this snippet:

\begin{verbatim}
// Fixed height vector
#let fvec(..children, delim: "(", gap: 1.5em) = { // change default gap there
  context math.vec(
      delim: delim,
      gap: 0em,
      ..for el in children.pos() {
        ({
          box(
            width: measure(el).width,
            height: gap, place(horizon, el)
          )
        },) // this is an array
        // `for` merges all these arrays, then we pass it to arguments
      }
    )
}

// fixed hight matrix
// accepts also row-gap, column-gap and gap
#let fmat(..rows, delim: "(", augment: none) = {
  let args = rows.named()
  let (gap, row-gap, column-gap) = (none,)*3;

  if "gap" in args {
    gap = args.at("gap")
    row-gap = args.at("row-gap", default: gap)
    column-gap = args.at("row-gap", default: gap)
  }
  else {
    // change default vertical there
    row-gap = args.at("row-gap", default: 1.5em) 
    // and horizontal there
    column-gap = rows.named().at("column-gap", default: 0.5em)
  }

  context math.mat(
      delim: delim,
      row-gap: 0em,
      column-gap: column-gap,
      ..for row in rows.pos() {
        (for el in row {
          ({
          box(
            width: measure(el).width,
            height: row-gap, place(horizon, el)
          )
        },)
        }, )
      }
    )
}

$
"Before:"& vec(((a/b))/c, a/b, c) = vec(1, 1, 1)\
"After:"& fvec(((a/b))/c, a/b, c) = fvec(1, 1, 1)\

"Before:"& mat(a, b; c, d) vec(e, dot) = vec(c/d, e/f)\
"After:"& fmat(a, b; c, d) fvec(e, dot) = fvec(c/d, e/f)
$
\end{verbatim}

\pandocbounded{\includesvg[keepaspectratio]{typst-img/98195a6d9cfb93fdc5dca4db04dde22c00b969129e2962c8f7cba9012cd2bd0d-1.svg}}


\section{Examples Book LaTeX/book/snippets/math/operations.tex}
\title{sitandr.github.io/typst-examples-book/book/snippets/math/operations}

\section{\texorpdfstring{\hyperref[operations]{Operations}}{Operations}}\label{operations}

\subsection{\texorpdfstring{\hyperref[fractions]{Fractions}}{Fractions}}\label{fractions}

\begin{verbatim}
$
p/q, p slash q, p\/q
$
\end{verbatim}

\pandocbounded{\includesvg[keepaspectratio]{typst-img/7e6b189e7b1c1329caebb4d4c6ea718c897ef64f51383889c65e62e308c73478-1.svg}}

\subsubsection{\texorpdfstring{\hyperref[slightly-moved]{Slightly
moved:}}{Slightly moved:}}\label{slightly-moved}

\begin{verbatim}
#let mfrac(a, b) = move(a, dy: -0.2em) + "/" + move(b, dy: 0.2em, dx: -0.1em)
$A\/B, #mfrac($A$, $B$)$,
\end{verbatim}

\pandocbounded{\includesvg[keepaspectratio]{typst-img/002c9e0e934a98cfb5e93a407d130841a5e1a493d361c368ae605acdfd6f64bc-1.svg}}

\subsubsection{\texorpdfstring{\hyperref[large-fractions]{Large
fractions}}{Large fractions}}\label{large-fractions}

\begin{verbatim}
#let dfrac(a, b) = $display(frac(#a, #b))$

$(x + y)/(1/x + 1/y) quad (x + y)/(dfrac(1,x) + dfrac(1, y))$
\end{verbatim}

\pandocbounded{\includesvg[keepaspectratio]{typst-img/36454aba32957127c97710e4fc1db3e6d8c9a558e886b7103915d501004bad76-1.svg}}


\section{Examples Book LaTeX/book/snippets/math/fonts.tex}
\title{sitandr.github.io/typst-examples-book/book/snippets/math/fonts}

\section{\texorpdfstring{\hyperref[fonts]{Fonts}}{Fonts}}\label{fonts}

\subsection{\texorpdfstring{\hyperref[set-math-font]{Set math
font}}{Set math font}}\label{set-math-font}

\textbf{Important:} The font you want to set for math should
\emph{contain} necessary math symbols. That should be a special font
with math. If it isn\textquotesingle t, you are very likely to get
\emph{an error} (remember to set
\texttt{\ }{\texttt{\ fallback:\ false\ }}\texttt{\ } and check
\texttt{\ }{\texttt{\ typst\ fonts\ }}\texttt{\ } to debug the fonts).

\begin{verbatim}
#show math.equation: set text(font: "Fira Math", fallback: false)

$
emptyset \

integral_a^b sum (A + B)/C dif x \
$
\end{verbatim}

\pandocbounded{\includesvg[keepaspectratio]{typst-img/f1cc8c8afe82aeb14cb0898ff5a83292a7c6a73f8b84cac8dbffa19af0d9370f-1.svg}}




\section{Combined Examples Book LaTeX/book/snippets/special.tex}
\section{Examples Book LaTeX/book/snippets/special/index.tex}
\title{sitandr.github.io/typst-examples-book/book/snippets/special/index}

\section{\texorpdfstring{\hyperref[special-documents]{Special
documents}}{Special documents}}\label{special-documents}

\subsection{\texorpdfstring{\hyperref[signature-places]{Signature
places}}{Signature places}}\label{signature-places}

\begin{verbatim}
#block(width: 150pt)[
  #line(length: 100%)
  #align(center)[Signature]
]
\end{verbatim}

\pandocbounded{\includesvg[keepaspectratio]{typst-img/04e318a822e6a90fbae23cce4c1b829e03e4d283051bb5dd613be2d7fe5933a2-1.svg}}

\subsection{\texorpdfstring{\hyperref[presentations]{Presentations}}{Presentations}}\label{presentations}

See \href{../../packages/}{polylux} .

\subsection{\texorpdfstring{\hyperref[forms]{Forms}}{Forms}}\label{forms}

\subsubsection{\texorpdfstring{\hyperref[form-with-placeholder]{Form
with placeholder}}{Form with placeholder}}\label{form-with-placeholder}

\begin{verbatim}
#grid(
  columns: 2,
  rows: 4,
  gutter: 1em,

  [Student:],
  [#block()#align(bottom)[#line(length: 10em, stroke: 0.5pt)]],
  [Teacher:],
  [#block()#align(bottom)[#line(length: 10em, stroke: 0.5pt)]],
  [ID:],
  [#block()#align(bottom)[#line(length: 10em, stroke: 0.5pt)]],
  [School:],
  [#block()#align(bottom)[#line(length: 10em, stroke: 0.5pt)]],
)
\end{verbatim}

\pandocbounded{\includesvg[keepaspectratio]{typst-img/d921f4df08e8dab19c9440b3aca6b065cb83ba11f015240b2115a63351ce64ce-1.svg}}

\subsubsection{\texorpdfstring{\hyperref[interactive]{Interactive}}{Interactive}}\label{interactive}

\begin{quote}
Presentation interactive forms are coming! They are currently under
heavy work by @tinger.
\end{quote}






\section{Combined Examples Book LaTeX/book/book.tex}
\section{Examples Book LaTeX/book/about.tex}
\title{sitandr.github.io/typst-examples-book/book/about}

\section{\texorpdfstring{\hyperref[typst-examples-book]{Typst Examples
Book}}{Typst Examples Book}}\label{typst-examples-book}

This book provides an extended \emph{tutorial} and lots of
\href{https://github.com/typst/typst}{Typst} snippets that can help you
to write better Typst code.

This is an unofficial book. Some snippets \& suggestions here may be
outdated or useless (please let me know if you find some).

However, \emph{all of them should compile on last version of Typst
\textsuperscript{\hyperref[1]{1}}} .

\textbf{CAUTION:} the book is (probably forever) a \textbf{WIP} , so
don\textquotesingle t rely on it.

If you like it, consider
\href{https://github.com/sitandr/typst-examples-book}{giving a star on
GitHub} !

This will help me to stay motivated and continue working on this book.

\subsection{\texorpdfstring{\hyperref[navigation]{Navigation}}{Navigation}}\label{navigation}

The book consists of several chapters, each with its own goal:

\begin{enumerate}
\tightlist
\item
  \href{./basics/index.html}{Typst Basics}
\item
  \href{./snippets/index.html}{Typst Snippets}
\item
  \href{./packages/index.html}{Typst Packages}
\item
  \href{./typstonomicon/index.html}{Typstonomicon}
\end{enumerate}

\subsection{\texorpdfstring{\hyperref[contributions]{Contributions}}{Contributions}}\label{contributions}

Any contributions are very welcome! If you have a good code snippet that
you want to share, feel free to submit an issue with snippet or make a
PR in the
\href{https://github.com/sitandr/typst-examples-book}{repository} .

I will especially appreciate submissions of active community members and
compiler contributors!

However, I will also really appreciate feedback from beginners to make
the book as comprehensible as possible!

\subsection{\texorpdfstring{\hyperref[acknowledgements]{Acknowledgements}}{Acknowledgements}}\label{acknowledgements}

Thanks to everyone in the community who published their code snippets!

If someone doesn\textquotesingle t like their code and/or name being
published, please contact me.

\phantomsection\label{1}
\textsuperscript{1}

When a new version launches, it may take some time to update the book,
feel free to tag me to speed up the process.


\section{Examples Book LaTeX/book/print.tex}
\title{sitandr.github.io/typst-examples-book/book/print}

\section{\texorpdfstring{\hyperref[typst-examples-book]{Typst Examples
Book}}{Typst Examples Book}}\label{typst-examples-book}

This book provides an extended \emph{tutorial} and lots of
\href{https://github.com/typst/typst}{Typst} snippets that can help you
to write better Typst code.

This is an unofficial book. Some snippets \& suggestions here may be
outdated or useless (please let me know if you find some).

However, \emph{all of them should compile on last version of Typst
\textsuperscript{\hyperref[1]{1}}} .

\textbf{CAUTION:} the book is (probably forever) a \textbf{WIP} , so
don\textquotesingle t rely on it.

If you like it, consider
\href{https://github.com/sitandr/typst-examples-book}{giving a star on
GitHub} !

This will help me to stay motivated and continue working on this book.

\subsection{\texorpdfstring{\hyperref[navigation]{Navigation}}{Navigation}}\label{navigation}

The book consists of several chapters, each with its own goal:

\begin{enumerate}
\tightlist
\item
  \href{./basics/index.html}{Typst Basics}
\item
  \href{./snippets/index.html}{Typst Snippets}
\item
  \href{./packages/index.html}{Typst Packages}
\item
  \href{./typstonomicon/index.html}{Typstonomicon}
\end{enumerate}

\subsection{\texorpdfstring{\hyperref[contributions]{Contributions}}{Contributions}}\label{contributions}

Any contributions are very welcome! If you have a good code snippet that
you want to share, feel free to submit an issue with snippet or make a
PR in the
\href{https://github.com/sitandr/typst-examples-book}{repository} .

I will especially appreciate submissions of active community members and
compiler contributors!

However, I will also really appreciate feedback from beginners to make
the book as comprehensible as possible!

\subsection{\texorpdfstring{\hyperref[acknowledgements]{Acknowledgements}}{Acknowledgements}}\label{acknowledgements}

Thanks to everyone in the community who published their code snippets!

If someone doesn\textquotesingle t like their code and/or name being
published, please contact me.

\phantomsection\label{1}
\textsuperscript{1}

When a new version launches, it may take some time to update the book,
feel free to tag me to speed up the process.

\section{\texorpdfstring{\hyperref[typst-basics]{Typst
Basics}}{Typst Basics}}\label{typst-basics}

This is a chapter that consistently introduces you to the most things
you need to know when writing with Typst.

It show much more things than official tutorial, so maybe it will be
interesting to read for some of the experienced users too.

Some examples are taken from
\href{https://typst.app/docs/tutorial/}{Official Tutorial} and
\href{https://typst.app/docs/reference/}{Official Reference} . Most are
created and edited specially for this book.

\begin{quote}
\emph{Important:} in most cases there will be used "clipped" examples of
your rendered documents (no margins, smaller width and so on).

To set up the spacing as you want, see
\href{https://typst.app/docs/guides/page-setup-guide/}{Official Page
Setup Guide} .
\end{quote}

\section{\texorpdfstring{\hyperref[tutorial-by-examples]{Tutorial by
Examples}}{Tutorial by Examples}}\label{tutorial-by-examples}

The first section of Typst Basics is very similar to
\href{https://typst.app/docs/tutorial/}{Official Tutorial} , with more
specialized examples and less words. It is \emph{highly recommended to
read the official tutorial anyway} .

\section{\texorpdfstring{\hyperref[markup-language]{Markup
language}}{Markup language}}\label{markup-language}

\subsection{\texorpdfstring{\hyperref[starting]{Starting}}{Starting}}\label{starting}

\begin{verbatim}
Starting typing in Typst is easy.
You don't need packages or other weird things for most of things.

Blank line will move text to a new paragraph.

Btw, you can use any language and unicode symbols
without any problems as long as the font supports it: ßçœ̃ɛ̃ø∀αβёыა😆…
\end{verbatim}

\pandocbounded{\includesvg[keepaspectratio]{basics/tutorial/typst-img/ee9f64251c99c7aeaaf6fa1d5bc7e907c2d51a34aa38126544d515ca197ca2a8-1.svg}}

\subsection{\texorpdfstring{\hyperref[markup]{Markup}}{Markup}}\label{markup}

\begin{verbatim}
= Markup

This was a heading. Number of `=` in front of name corresponds to heading level.

== Second-level heading

Okay, let's move to _emphasis_ and *bold* text.

Markup syntax is generally similar to
`AsciiDoc` (this was `raw` for monospace text!)
\end{verbatim}

\pandocbounded{\includesvg[keepaspectratio]{basics/tutorial/typst-img/fa8b95f9b15083387a29c11d17efca9873b8e778643b1b5079aa137891d01c8d-1.svg}}

\subsection{\texorpdfstring{\hyperref[new-lines--escaping]{New lines \&
Escaping}}{New lines \& Escaping}}\label{new-lines--escaping}

\begin{verbatim}
You can break \
line anywhere you \
want using "\\" symbol.

Also you can use that symbol to
escape \_all the symbols you want\_,
if you don't want it to be interpreted as markup
or other special symbols.
\end{verbatim}

\pandocbounded{\includesvg[keepaspectratio]{basics/tutorial/typst-img/4dabdee2a61e7d10773d51772dba3665271a09d4d5df4a8f66dd80589f0bcd7a-1.svg}}

\subsection{\texorpdfstring{\hyperref[comments--codeblocks]{Comments \&
codeblocks}}{Comments \& codeblocks}}\label{comments--codeblocks}

\begin{verbatim}
You can write comments with `//` and `/* comment */`:
// Like this
/* Or even like
this */

```typ
Just in case you didn't read source,
this is how it is written:

// Like this
/* Or even like
this */

By the way, I'm writing it all in a _fenced code block_ with *syntax highlighting*!
```
\end{verbatim}

\pandocbounded{\includesvg[keepaspectratio]{basics/tutorial/typst-img/a481d12b3ed0bbe2d9db6cc4b4a1237cba9936de83333254dfce8702832db125-1.svg}}

\subsection{\texorpdfstring{\hyperref[smart-quotes]{Smart
quotes}}{Smart quotes}}\label{smart-quotes}

\begin{verbatim}
== What else?

There are not much things in basic "markup" syntax,
but we will see much more interesting things very soon!
I hope you noticed auto-matched "smart quotes" there.
\end{verbatim}

\pandocbounded{\includesvg[keepaspectratio]{basics/tutorial/typst-img/89114a6e9af45c2eb9db2ef44d0e5ba41e31bf816e72803bd1a9a02120e69fc3-1.svg}}

\subsection{\texorpdfstring{\hyperref[lists]{Lists}}{Lists}}\label{lists}

\begin{verbatim}
- Writing lists in a simple way is great.
- Nothing complex, start your points with `-`
  and this will become a list.
  - Indented lists are created via indentation.

+ Numbered lists start with `+` instead of `-`.
+ There is no alternative markup syntax for lists
+ So just remember `-` and `+`, all other symbols
  wouldn't work in an unintended way.
  + That is a general property of Typst's markup.
  + Unlike Markdown, there is only one way
    to write something with it.
\end{verbatim}

\pandocbounded{\includesvg[keepaspectratio]{basics/tutorial/typst-img/ad4e424e067a4362e9f145c0c4ba4b7c1b65e17e7d0e7631b6836841607ef85e-1.svg}}

\textbf{Notice:}

\begin{verbatim}
Typst numbered lists differ from markdown-like syntax for lists. If you write them by hand, numbering is preserved:

1. Apple
1. Orange
1. Peach
\end{verbatim}

\pandocbounded{\includesvg[keepaspectratio]{basics/tutorial/typst-img/477695c86becc136dceb144e90c0acd2b75faa2a49743f8673d09974b71da324-1.svg}}

\subsection{\texorpdfstring{\hyperref[math]{Math}}{Math}}\label{math}

\begin{verbatim}
I will just mention math ($a + b/c = sum_i x^i$)
is possible and quite pretty there:

$
7.32 beta +
  sum_(i=0)^nabla
    (Q_i (a_i - epsilon)) / 2
$

To learn more about math, see corresponding chapter.
\end{verbatim}

\pandocbounded{\includesvg[keepaspectratio]{basics/tutorial/typst-img/12cc318c8438cd8e91706013bbd53fee5ee004620a63348cfe2d7dcc3b8a19d4-1.svg}}

\section{\texorpdfstring{\hyperref[functions]{Functions}}{Functions}}\label{functions}

\subsection{\texorpdfstring{\hyperref[functions-1]{Functions}}{Functions}}\label{functions-1}

\begin{verbatim}
Okay, let's now move to more complex things.

First of all, there are *lots of magic* in Typst.
And it major part of it is called "scripting".

To go to scripting mode, type `#` and *some function name*
after that. We will start with _something dull_:

#lorem(50)

_That *function* just generated 50 "Lorem Ipsum" words!_
\end{verbatim}

\pandocbounded{\includesvg[keepaspectratio]{basics/tutorial/typst-img/036fce36d10e06e8e41be8e77d7d5672f5dfc82c57e7c3ba9b8060d0822ca115-1.svg}}

\subsection{\texorpdfstring{\hyperref[more-functions]{More
functions}}{More functions}}\label{more-functions}

\begin{verbatim}
#underline[functions can do everything!]

#text(orange)[L]ike #text(size: 0.8em)[Really] #sub[E]verything!

#figure(
  caption: [
    This is a screenshot from one of first theses written in Typst. \
    _All these things are written with #text(blue)[custom functions] too._
  ],
  image("../boxes.png", width: 80%)
)

In fact, you can #strong[forget] about markup
and #emph[just write] functions everywhere!

#list[
  All that markup is just a #emph[syntax sugar] over functions!
]
\end{verbatim}

\pandocbounded{\includesvg[keepaspectratio]{basics/tutorial/typst-img/455e15e83c25259f932178d68517cc012432cb17d072e60c659169470fe191ce-1.svg}}

\subsection{\texorpdfstring{\hyperref[how-to-call-functions]{How to call
functions}}{How to call functions}}\label{how-to-call-functions}

\begin{verbatim}
First, start with `#`. Then write the name.
Finally, write some parentheses and maybe something inside.

You can navigate lots of built-in functions
in #link("https://typst.app/docs/reference/")[Official Reference].

#quote(block: true, attribution: "Typst Examples Book")[
  That's right, links, quotes and lots of
  other document elements are created with functions.
]
\end{verbatim}

\pandocbounded{\includesvg[keepaspectratio]{basics/tutorial/typst-img/4c63fde73bb1ad0afe1332ab68c5b540ec786c6352a76860f4398fec32034cf0-1.svg}}

\subsection{\texorpdfstring{\hyperref[function-arguments]{Function
arguments}}{Function arguments}}\label{function-arguments}

\begin{verbatim}
There are _two types_ of function arguments:

+ *Positional.* Like `50` in `lorem(50)`.
  Just write them in parentheses and it will be okay.
  If you have many, use commas.
+ *Named.* Like in `#quote(attribution: "Whoever")`.
  Write the value after a name and a colon.

If argument is named, it has some _default value_.
To find out what it is, see
#link("https://typst.app/docs/reference/")[Official Typst Reference].
\end{verbatim}

\pandocbounded{\includesvg[keepaspectratio]{basics/tutorial/typst-img/d66fb474260490595a207f06c687efcc85808701c39c2a6e8b686bc22ffde279-1.svg}}

\subsection{\texorpdfstring{\hyperref[content]{Content}}{Content}}\label{content}

\begin{verbatim}
The most "universal" type in Typst language is *content*.
Everything you write in the document becomes content.

#[
  But you can explicitly create it with
  _scripting mode_ and *square brackets*.

  There, in square brackets, you can use any markup
  functions or whatever you want.
]
\end{verbatim}

\pandocbounded{\includesvg[keepaspectratio]{basics/tutorial/typst-img/faf9d7cddd55e68f84d212013a52a724c2ad763f18d83221a99bbd380410d7d1-1.svg}}

\subsection{\texorpdfstring{\hyperref[markup-and-code-modes]{Markup and
code modes}}{Markup and code modes}}\label{markup-and-code-modes}

\begin{verbatim}
When you use `#`, you are "switching" to code mode.
When you use `[]`, you turn back:

// +-- going from markup (the default mode) to scripting for that function
// |                 +-- scripting mode: calling `text`, the last argument is markup
// |     first arg   |
// v     vvvvvvvvv   vvvv
   #rect(width: 5cm, text(red)[hello *world*])
//  ^^^^                       ^^^^^^^^^^^^^ just a markup argument for `text`
//  |
//  +-- calling `rect` in scripting mode, with two arguments: width and other content
\end{verbatim}

\pandocbounded{\includesvg[keepaspectratio]{basics/tutorial/typst-img/0cabe3da1eb49f805535fb1d7e34a0d6eb1a6c49227b0be98634c6965e892185-1.svg}}

\subsection{\texorpdfstring{\hyperref[passing-content-into-functions]{Passing
content into
functions}}{Passing content into functions}}\label{passing-content-into-functions}

\begin{verbatim}
So what are these square brackets after functions?

If you *write content right after
function, it will be passed as positional argument there*.

#quote(block: true)[
  So #text(red)[_that_] allows me to write
  _literally anything in things
  I pass to #underline[functions]!_
]
\end{verbatim}

\pandocbounded{\includesvg[keepaspectratio]{basics/tutorial/typst-img/686d2b2a361a60244452ce53bd37ebef0699e92cf962c477bfb62bafdc0f7241-1.svg}}

\subsection{\texorpdfstring{\hyperref[passing-content-part-ii]{Passing
content, part
II}}{Passing content, part II}}\label{passing-content-part-ii}

\begin{verbatim}
So, just to make it clear, when I write

```typ
- #text(red)[red text]
- #text([red text], red)
- #text("red text", red)
//      ^        ^
// Quotes there mean a plain string, not a content!
// This is just text.
```

It all will result in a #text([red text], red).
\end{verbatim}

\pandocbounded{\includesvg[keepaspectratio]{basics/tutorial/typst-img/4686939b6d0932f1ebebac4111d8f02919dbc16446def7855c521d8dbf293689-1.svg}}

\section{\texorpdfstring{\hyperref[basic-styling]{Basic
styling}}{Basic styling}}\label{basic-styling}

\subsection{\texorpdfstring{\hyperref[set-rule]{\texttt{\ }{\texttt{\ Set\ }}\texttt{\ }
rule}}{  Set   rule}}\label{set-rule}

\begin{verbatim}
#set page(width: 15cm, margin: (left: 4cm, right: 4cm))

That was great, but using functions everywhere, especially
with many arguments every time is awfully cumbersome.

That's why Typst has _rules_. No, not for you, for the document.

#set par(justify: true)

And the first rule we will consider there is `set` rule.
As you see, I've just used it on `par` (which is short from paragraph)
and now all paragraphs became _justified_.

It will apply to all paragraphs after the rule,
but will work only in it's _scope_ (we will discuss them later).

#par(justify: false)[
  Of course, you can override a `set` rule.
  This rule just sets the _default value_
  of an argument of an element.
]

By the way, at first line of this snippet
I've reduced page size to make justifying more visible,
also increasing margins to add blank space on left and right.
\end{verbatim}

\pandocbounded{\includesvg[keepaspectratio]{basics/tutorial/typst-img/cee42a8b1274afa36891438d4b1611eb55b2cd8bb4546df47128a7d3eb66653b-1.svg}}

\subsection{\texorpdfstring{\hyperref[a-bit-about-length-units]{A bit
about length
units}}{A bit about length units}}\label{a-bit-about-length-units}

\begin{verbatim}
Before we continue with rules, we should talk about length. There are several absolute length units in Typst:

#set rect(height: 1em)

#table(
  columns: 2,
  [Points], rect(width: 72pt),
  [Millimeters], rect(width: 25.4mm),
  [Centimeters], rect(width: 2.54cm),
  [Inches], rect(width: 1in),
  [Relative to font size], rect(width: 6.5em)
)

`1 em` = current font size. \
It is a very convenient unit,
so we are going to use it a lot
\end{verbatim}

\pandocbounded{\includesvg[keepaspectratio]{basics/tutorial/typst-img/5f8abc94a3d9df0e16f78c258e7f487d5698b4c96491300b3a48ad8e685534bc-1.svg}}

\subsection{\texorpdfstring{\hyperref[setting-something-else]{Setting
something else}}{Setting something else}}\label{setting-something-else}

Of course, you can use \texttt{\ }{\texttt{\ set\ }}\texttt{\ } rule
with all built-in functions and all their named arguments to make some
argument "default".

For example, let\textquotesingle s make all quotes in this snippet
authored by the book:

\begin{verbatim}
#set quote(block: true, attribution: [Typst Examples Book])

#quote[
  Typst is great!
]

#quote[
  The problem with quotes on the internet is
  that it is hard to verify their authenticity.
]
\end{verbatim}

\pandocbounded{\includesvg[keepaspectratio]{basics/tutorial/typst-img/c34c25cad05b7c20b6e0f146002886a1de65b61f48666cfec3d3494bd694a641-1.svg}}

\subsection{\texorpdfstring{\hyperref[opinionated-defaults]{Opinionated
defaults}}{Opinionated defaults}}\label{opinionated-defaults}

That allows you to set Typst default styling as you want it:

\begin{verbatim}
#set par(justify: true)
#set list(indent: 1em)
#set enum(indent: 1em)
#set page(numbering: "1")

- List item
- List item

+ Enum item
+ Enum item
\end{verbatim}

\pandocbounded{\includesvg[keepaspectratio]{basics/tutorial/typst-img/773d68bc55eb89f119ad07b882eae5fd31868d8a1bb3d4963573ec80fb1c7466-1.svg}}

Don\textquotesingle t complain about bad defaults!
\texttt{\ }{\texttt{\ Set\ }}\texttt{\ } your own.

\subsection{\texorpdfstring{\hyperref[numbering]{Numbering}}{Numbering}}\label{numbering}

\begin{verbatim}
= Numbering

Some of elements have a property called "numbering".
They accept so-called "numbering patterns" and
are very useful with set rules. Let's see what I mean.

#set heading(numbering: "I.1:")

= This is first level
= Another first
== Second
== Another second
=== Now third
== And second again
= Now returning to first
= These are actual romanian numerals
\end{verbatim}

\pandocbounded{\includesvg[keepaspectratio]{basics/tutorial/typst-img/39fb958032888b1e41da849152fed716b6f590eed3ea975b051ab786fac4ce5c-1.svg}}

Of course, there are lots of other cool properties that can be
\emph{set} , so feel free to dive into
\href{https://typst.app/docs/reference/}{Official Reference} and explore
them!

And now we are moving into something much more interesting\ldots{}

\section{\texorpdfstring{\hyperref[advanced-styling]{Advanced
styling}}{Advanced styling}}\label{advanced-styling}

\subsection{\texorpdfstring{\hyperref[the-show-rule]{The
\texttt{\ }{\texttt{\ show\ }}\texttt{\ }
rule}}{The   show   rule}}\label{the-show-rule}

\begin{verbatim}
Advanced styling comes with another rule. The _`show` rule_.

Now please compare the source code and the output.

#show "Be careful": strong[Play]

This is a very powerful thing, sometimes even too powerful.
Be careful with it.

#show "it is holding me hostage": text(green)[I'm fine]

Wait, what? I told you "Be careful!", not "Play!".

Help, it is holding me hostage.
\end{verbatim}

\pandocbounded{\includesvg[keepaspectratio]{basics/tutorial/typst-img/8a9ac38769d4ac7b42a2755047d0cd5a6404ad26e9e7f5b72b6984fa67abadf9-1.svg}}

\subsection{\texorpdfstring{\hyperref[now-a-bit-more-serious]{Now a bit
more serious}}{Now a bit more serious}}\label{now-a-bit-more-serious}

\begin{verbatim}
Show rule is a powerful thing that takes a _selector_
and what to apply to it. After that it will apply to
all elements it can find.

It may be extremely useful like that:

#show emph: set text(blue)

Now if I want to _emphasize_ something,
it will be both _emphasized_ and _blue_.
Isn't that cool?
\end{verbatim}

\pandocbounded{\includesvg[keepaspectratio]{basics/tutorial/typst-img/657acaf5c4ca684408bbc6fe0dec4c74b9fa58d24805ec975be1382aa7bf959c-1.svg}}

\subsection{\texorpdfstring{\hyperref[about-syntax]{About
syntax}}{About syntax}}\label{about-syntax}

\begin{verbatim}
Sometimes show rules may be confusing. They may seem very diverse, but in fact they all are quite the same! So

// actually, this is the same as
// redify = text.with(red)
// `with` creates a new function with this argument already set
#let redify(string) = text(red, string)

// and this is the same as
// framify = rect.with(stroke: orange)
#let framify(object) = rect(object, stroke: orange)

// set default color of text blue for all following text
#show: set text(blue)

Blue text.

// wrap everything into a frame
#show: framify

Framed text.

// it's the same, just creating new function that calls framify
#show: a => framify(a)

Double-framed.

// apply function to `the`
#show "the": redify
// set text color for all the headings
#show heading: set text(purple)

= Conclusion

All these rules do basically the same!
\end{verbatim}

\pandocbounded{\includesvg[keepaspectratio]{basics/tutorial/typst-img/2dfcde68345d3fa276b99a1f308343118c6eeae09fd106389a8fc488d7244ebb-1.svg}}

\subsection{\texorpdfstring{\hyperref[blocks]{Blocks}}{Blocks}}\label{blocks}

One of the most important usages is that you can set up all spacing
using blocks. Like every element with text contains text that can be set
up, every \emph{block element} contains blocks:

\begin{verbatim}
Text before
= Heading
Text after

#show heading: set block(spacing: 0.5em)

Text before
= Heading
Text after
\end{verbatim}

\pandocbounded{\includesvg[keepaspectratio]{basics/tutorial/typst-img/7891207932d0918c88b5804b3a7ee051ce5dda93081f8999eb0f7ebaee48400a-1.svg}}

\subsection{\texorpdfstring{\hyperref[selector]{Selector}}{Selector}}\label{selector}

\begin{verbatim}
So show rule can accept _selectors_.

There are lots of different selector types,
for example

- element functions
- strings
- regular expressions
- field filters

Let's see example of the latter:

#show heading.where(level: 1): set align(center)

= Title
== Small title

Of course, you can set align by hand,
no need to use show rules
(but they are very handy!):

#align(center)[== Centered small title]
\end{verbatim}

\pandocbounded{\includesvg[keepaspectratio]{basics/tutorial/typst-img/f41f337dd75b55211dd8d16e2682132c1ffb1ef19f774ba6cafc94cae090ec75-1.svg}}

\subsection{\texorpdfstring{\hyperref[custom-formatting]{Custom
formatting}}{Custom formatting}}\label{custom-formatting}

\begin{verbatim}
Let's try now writing custom functions.
It is very easy, see yourself:

// "it" is a heading, we take it and output things in braces
#show heading: it => {
  // center it
  set align(center)
  // set size and weight
  set text(12pt, weight: "regular")
  // see more about blocks and boxes
  // in corresponding chapter
  block(smallcaps(it.body))
}

= Smallcaps heading
\end{verbatim}

\pandocbounded{\includesvg[keepaspectratio]{basics/tutorial/typst-img/a5c37bce3cf9a077a4eb62a4d95f89584b5ef8acee279b81de6019d0e5768ba0-1.svg}}

\subsection{\texorpdfstring{\hyperref[setting-spacing]{Setting
spacing}}{Setting spacing}}\label{setting-spacing}

TODO: explain block spacing for common elements

\subsection{\texorpdfstring{\hyperref[formatting-to-get-an-article-look]{Formatting
to get an "article
look"}}{Formatting to get an "article look"}}\label{formatting-to-get-an-article-look}

\begin{verbatim}
#set page(
  // Header is that small thing on top
  header: align(
    right + horizon,
    [Some header there]
  ),
  height: 12cm
)

#align(center, text(17pt)[
  *Important title*
])

#grid(
  columns: (1fr, 1fr),
  align(center)[
    Some author \
    Some Institute \
    #link("mailto:some@mail.edu")
  ],
  align(center)[
    Another author \
    Another Institute \
    #link("mailto:another@mail.edu")
  ]
)

Now let's split text into two columns:

#show: rest => columns(2, rest)

#show heading.where(
  level: 1
): it => block(width: 100%)[
  #set align(center)
  #set text(12pt, weight: "regular")
  #smallcaps(it.body)
]

#show heading.where(
  level: 2
): it => text(
  size: 11pt,
  weight: "regular",
  style: "italic",
  it.body + [.],
)

// Now let's fill it with words:

= Heading
== Small heading
#lorem(10)
== Second subchapter
#lorem(10)
= Second heading
#lorem(40)

== Second subchapter
#lorem(40)
\end{verbatim}

\pandocbounded{\includesvg[keepaspectratio]{basics/tutorial/typst-img/76ee0cca809299df178ec9d94371c01031d1808a700b39deac5245dd6b83157f-1.svg}}

\section{\texorpdfstring{\hyperref[templates]{Templates}}{Templates}}\label{templates}

\subsection{\texorpdfstring{\hyperref[templates-1]{Templates}}{Templates}}\label{templates-1}

If you want to reuse styling in other files, you can use the
\emph{template} idiom. Because \texttt{\ }{\texttt{\ set\ }}\texttt{\ }
and \texttt{\ }{\texttt{\ show\ }}\texttt{\ } rules are only active in
their current scope, they will not affect content in a file you imported
your file into. But functions can circumvent this in a predictable way:

\begin{verbatim}
// define a function that:
// - takes content
// - applies styling to it
// - returns the styled content
#let apply-template(body) = [
  #show heading.where(level: 1): emph
  #set heading(numbering: "1.1")
  // ...
  #body
]
\end{verbatim}

This is equivalent to:

\begin{verbatim}
// we can reduce the number of hashes needed here by using scripting mode
// same as above but we exchanged `[...]` for `{...}` to switch from markup
// into scripting mode
#let apply-template(body) = {
  show heading.where(level: 1): emph
  set heading(numbering: "1.1")
  // ...
  body
}
\end{verbatim}

Then in your main file:

\begin{verbatim}
#import "template.typ": apply-template
#show: apply-template
\end{verbatim}

\emph{This will apply a "template" function to the rest of your
document!}

\subsubsection{\texorpdfstring{\hyperref[passing-arguments]{Passing
arguments}}{Passing arguments}}\label{passing-arguments}

\begin{verbatim}
// add optional named arguments
#let apply-template(body, name: "My document") = {
  show heading.where(level: 1): emph
  set heading(numbering: "1.1")

  align(center, text(name, size: 2em))

  body
}
\end{verbatim}

Then, in template file:

\begin{verbatim}
#import "template.typ": apply-template

// `func.with(..)` applies the arguments to the function and returns the new
// function with those defaults applied
#show: apply-template.with(name: "Report")

// it is functionally the same as this
#let new-template(..args) = apply-template(name: "Report", ..args)
#show: new-template
\end{verbatim}

Writing templates is fairly easy if you understand
\href{basics/tutorial/../scripting/index.html}{scripting} .

See more information about writing templates in
\href{https://typst.app/docs/tutorial/making-a-template/}{Official
Tutorial} .

There is no official repository for templates yet, but there are a
plenty community ones in
\href{https://github.com/qjcg/awesome-typst?ysclid=lj8pur1am7431908794\#general}{awesome-typst}
.

\section{\texorpdfstring{\hyperref[must-know]{Must-know}}{Must-know}}\label{must-know}

This section contains things, that are not general enough to be part of
"tutorial", but still are very important to know for proper typesetting.

Feel free to skip through things you are sure you will not use.

\section{\texorpdfstring{\hyperref[boxing--blocking]{Boxing \&
Blocking}}{Boxing \& Blocking}}\label{boxing--blocking}

\begin{verbatim}
You can use boxes to wrap anything
into text: #box(image("../tiger.jpg", height: 2em)).

Blocks will always be "separate paragraphs".
They will not fit into a text: #block(image("../tiger.jpg", height: 2em))
\end{verbatim}

\pandocbounded{\includesvg[keepaspectratio]{basics/must_know/typst-img/8e3bd89485b00259666bd636cf28586f92db9c3c3922f0adcdad765ee66a06b1-1.svg}}

Both have similar useful properties:

\begin{verbatim}
#box(stroke: red, inset: 1em)[Box text]
#block(stroke: red, inset: 1em)[Block text]
\end{verbatim}

\pandocbounded{\includesvg[keepaspectratio]{basics/must_know/typst-img/9e3562619cb8a31b3d2311f53c3815a214f081e033a564e63dc003dfbc50d68d-1.svg}}

\subsection{\texorpdfstring{\hyperref[rect]{\texttt{\ }{\texttt{\ rect\ }}\texttt{\ }}}{  rect  }}\label{rect}

There is also \texttt{\ }{\texttt{\ rect\ }}\texttt{\ } that works like
\texttt{\ }{\texttt{\ block\ }}\texttt{\ } , but has useful default
inset and stroke:

\begin{verbatim}
#rect[Block text]
\end{verbatim}

\pandocbounded{\includesvg[keepaspectratio]{basics/must_know/typst-img/c778d1e94a3663a4f258985368c02e294a1333554c550b6cfe0465275a2eef0f-1.svg}}

\subsection{\texorpdfstring{\hyperref[figures]{Figures}}{Figures}}\label{figures}

For the purposes of adding a \emph{figure} to your document, use
\texttt{\ }{\texttt{\ figure\ }}\texttt{\ } function.
Don\textquotesingle t try to use boxes or blocks there.

Figures are that things like centered images (probably with captions),
tables, even code.

\begin{verbatim}
@tiger shows a tiger. Tigers
are animals.

#figure(
  image("../tiger.jpg", width: 80%),
  caption: [A tiger.],
) <tiger>
\end{verbatim}

\pandocbounded{\includesvg[keepaspectratio]{basics/must_know/typst-img/09a8b5b3c3bfffd81be7f34c31cc93ca5f8341b2594d022b2b92ac285aeb959d-1.svg}}

In fact, you can put there anything you want:

\begin{verbatim}
They told me to write a letter to you. Here it is:

#figure(
  text(size: 5em)[I],
  caption: [I'm cool, right?],
) 
\end{verbatim}

\pandocbounded{\includesvg[keepaspectratio]{basics/must_know/typst-img/e009534c4572064346490dfac659ff94a5a11d7f46af7a2b46c2136d206088c6-1.svg}}

\section{\texorpdfstring{\hyperref[using-spacing]{Using
spacing}}{Using spacing}}\label{using-spacing}

Most time you will pass spacing into functions. There are special
function fields that take only \emph{size} . They are usually called
like
\texttt{\ }{\texttt{\ width,\ length,\ in(out)set,\ spacing\ }}\texttt{\ }
and so on.

Like in CSS, one of the ways to set up spacing in Typst is setting
margins and padding of elements. However, you can also insert spacing
directly using functions \texttt{\ }{\texttt{\ h\ }}\texttt{\ }
(horizontal spacing) and \texttt{\ }{\texttt{\ v\ }}\texttt{\ }
(vertical spacing).

\begin{quote}
Links to reference: \href{https://typst.app/docs/reference/layout/h/}{h}
, \href{https://typst.app/docs/reference/layout/v/}{v} .
\end{quote}

\begin{verbatim}
Horizontal #h(1cm) spacing.
#v(1cm)
And some vertical too!
\end{verbatim}

\pandocbounded{\includesvg[keepaspectratio]{basics/must_know/typst-img/47b3ea7d16575780e489790177df9a624ad3c6c669594baa4127c1db516ebc94-1.svg}}

\section{\texorpdfstring{\hyperref[absolute-length-units]{Absolute
length units}}{Absolute length units}}\label{absolute-length-units}

\begin{quote}
Link to
\href{https://typst.app/docs/reference/layout/length/}{reference}
\end{quote}

Absolute length (aka just "length") units are not affected by outer
content and size of parent.

\begin{verbatim}
#set rect(height: 1em)
#table(
  columns: 2,
  [Points], rect(width: 72pt),
  [Millimeters], rect(width: 25.4mm),
  [Centimeters], rect(width: 2.54cm),
  [Inches], rect(width: 1in),
)
\end{verbatim}

\pandocbounded{\includesvg[keepaspectratio]{basics/must_know/typst-img/073ad26fe313743ab62dca82f30208dbf2d57ff354d5c37f0b6d4c063dc37d76-1.svg}}

\subsection{\texorpdfstring{\hyperref[relative-to-current-font-size]{Relative
to current font
size}}{Relative to current font size}}\label{relative-to-current-font-size}

\texttt{\ }{\texttt{\ 1em\ =\ 1\ current\ font\ size\ }}\texttt{\ } :

\begin{verbatim}
#set rect(height: 1em)
#table(
  columns: 2,
  [Centimeters], rect(width: 2.54cm),
  [Relative to font size], rect(width: 6.5em)
)

Double font size: #box(stroke: red, baseline: 40%, height: 2em, width: 2em)
\end{verbatim}

\pandocbounded{\includesvg[keepaspectratio]{basics/must_know/typst-img/7d62c9e2540f8bce40d8a3fc65a2779b161eb6b5b5682cf87247fee7f14145c2-1.svg}}

It is a very convenient unit, so it is used a lot in Typst.

\subsection{\texorpdfstring{\hyperref[combined]{Combined}}{Combined}}\label{combined}

\begin{verbatim}
Combined: #box(rect(height: 5pt + 1em))

#(5pt + 1em).abs
#(5pt + 1em).em
\end{verbatim}

\pandocbounded{\includesvg[keepaspectratio]{basics/must_know/typst-img/c8a0cae6047f35c85c41ac44ff2a6b0d28a28d0e097ca61b367202f9a361136e-1.svg}}

\section{\texorpdfstring{\hyperref[ratio-length]{Ratio
length}}{Ratio length}}\label{ratio-length}

\begin{quote}
Link to \href{https://typst.app/docs/reference/layout/ratio/}{reference}
\end{quote}

\texttt{\ }{\texttt{\ 1\%\ =\ 1\%\ from\ parent\ size\ in\ that\ dimension\ }}\texttt{\ }

\begin{verbatim}
This line width is 50% of available page size (without margins):

#line(length: 50%)

This line width is 50% of the box width: #box(stroke: red, width: 4em, inset: (y: 0.5em), line(length: 50%))
\end{verbatim}

\pandocbounded{\includesvg[keepaspectratio]{basics/must_know/typst-img/d478cb8be0a049380479b634cae709dc1e1ed406d323ecb1edbca1e582d7eafe-1.svg}}

\section{\texorpdfstring{\hyperref[relative-length]{Relative
length}}{Relative length}}\label{relative-length}

\begin{quote}
Link to
\href{https://typst.app/docs/reference/layout/relative/}{reference}
\end{quote}

You can \emph{combine} absolute and ratio lengths into \emph{relative
length} :

\begin{verbatim}
#rect(width: 100% - 50pt)

#(100% - 50pt).length \
#(100% - 50pt).ratio
\end{verbatim}

\pandocbounded{\includesvg[keepaspectratio]{basics/must_know/typst-img/6b72620a1972e758e55ef1ecf49d3e843095037399ed4dd2dfcd262ebbbe803f-1.svg}}

\section{\texorpdfstring{\hyperref[fractional-length]{Fractional
length}}{Fractional length}}\label{fractional-length}

\begin{quote}
Link to
\href{https://typst.app/docs/reference/layout/fraction/}{reference}
\end{quote}

Single fraction length just takes \emph{maximum size possible} to fill
the parent:

\begin{verbatim}
Left #h(1fr) Right

#rect(height: 1em)[
  #h(1fr)
]
\end{verbatim}

\pandocbounded{\includesvg[keepaspectratio]{basics/must_know/typst-img/b9c91f53b684699fff70c6889c8a47fccc57c5c540d7629b93c51a797eb2ef3c-1.svg}}

There are not many places you can use fractions, mainly those are
\texttt{\ }{\texttt{\ h\ }}\texttt{\ } and
\texttt{\ }{\texttt{\ v\ }}\texttt{\ } .

\subsection{\texorpdfstring{\hyperref[several-fractions]{Several
fractions}}{Several fractions}}\label{several-fractions}

If you use several fractions inside one parent, they will take all
remaining space \emph{proportional to their number} :

\begin{verbatim}
Left #h(1fr) Left-ish #h(2fr) Right
\end{verbatim}

\pandocbounded{\includesvg[keepaspectratio]{basics/must_know/typst-img/45182cbcecf395256d133af78fccacd9d48e29073672317744cb17340d0bafd8-1.svg}}

\subsection{\texorpdfstring{\hyperref[nested-layout]{Nested
layout}}{Nested layout}}\label{nested-layout}

Remember that fractions work in parent only, don\textquotesingle t
\emph{rely on them in nested layout} :

\begin{verbatim}
Word: #h(1fr) #box(height: 1em, stroke: red)[
  #h(2fr)
]
\end{verbatim}

\pandocbounded{\includesvg[keepaspectratio]{basics/must_know/typst-img/0c7ed8b25ea7e39a0907b1105b82027a0fb8b921b28978f30692f6c693bea5f7-1.svg}}

\section{\texorpdfstring{\hyperref[placing-moving-scale--hide]{Placing,
Moving, Scale \&
Hide}}{Placing, Moving, Scale \& Hide}}\label{placing-moving-scale--hide}

This is \textbf{a very important section} if you want to do arbitrary
things with layout, create custom elements and hacking a way around
current Typst limitations.

TODO: WIP, add text and better examples

\section{\texorpdfstring{\hyperref[place]{Place}}{Place}}\label{place}

\emph{Ignore layout} , just put some object somehow relative to parent
and current position. The placed object \emph{will not} affect layouting

\begin{quote}
Link to \href{https://typst.app/docs/reference/layout/place/}{reference}
\end{quote}

\begin{verbatim}
#set page(height: 60pt)
Hello, world!

#place(
  top + right, // place at the page right and top
  square(
    width: 20pt,
    stroke: 2pt + blue
  ),
)
\end{verbatim}

\pandocbounded{\includesvg[keepaspectratio]{basics/must_know/typst-img/e0d4c250d0f288e1a110ebddcb06149e0acd11b626a0ccb0ca9feb1c1d7be359-1.svg}}

\subsubsection{\texorpdfstring{\hyperref[basic-floating-with-place]{Basic
floating with
place}}{Basic floating with place}}\label{basic-floating-with-place}

\begin{verbatim}
#set page(height: 150pt)
#let note(where, body) = place(
  center + where,
  float: true,
  clearance: 6pt,
  rect(body),
)

#lorem(10)
#note(bottom)[Bottom 1]
#note(bottom)[Bottom 2]
#lorem(40)
#note(top)[Top]
#lorem(10)
\end{verbatim}

\pandocbounded{\includesvg[keepaspectratio]{basics/must_know/typst-img/b770cfef024690b5fc7ab82458797d6cfab0c5cc8f52078ecf2d61be17c13acc-1.svg}}

\pandocbounded{\includesvg[keepaspectratio]{basics/must_know/typst-img/b770cfef024690b5fc7ab82458797d6cfab0c5cc8f52078ecf2d61be17c13acc-2.svg}}

\subsubsection{\texorpdfstring{\hyperref[dx-dy]{dx,
dy}}{dx, dy}}\label{dx-dy}

Manually change position by
\texttt{\ }{\texttt{\ (dx,\ dy)\ }}\texttt{\ } relative to intended.

\begin{verbatim}
#set page(height: 100pt)
#for i in range(16) {
  let amount = i * 4pt
  place(center, dx: amount - 32pt, dy: amount)[A]
}
\end{verbatim}

\pandocbounded{\includesvg[keepaspectratio]{basics/must_know/typst-img/12464f1a2cfe81fb04623033345f3f88ff598af5dc77de378b9d7cf88fc1d5b3-1.svg}}

\section{\texorpdfstring{\hyperref[move]{Move}}{Move}}\label{move}

\begin{quote}
Link to \href{https://typst.app/docs/reference/layout/move/}{reference}
\end{quote}

\begin{verbatim}
#rect(inset: 0pt, move(
  dx: 6pt, dy: 6pt,
  rect(
    inset: 8pt,
    fill: white,
    stroke: black,
    [Abra cadabra]
  )
))
\end{verbatim}

\pandocbounded{\includesvg[keepaspectratio]{basics/must_know/typst-img/3292aebf7b633a2d9574027f50867d723d80850e046a101b9df5ab5143eb8a8d-1.svg}}

\section{\texorpdfstring{\hyperref[scale]{Scale}}{Scale}}\label{scale}

Scale content \emph{without affecting the layout} .

\begin{quote}
Link to \href{https://typst.app/docs/reference/layout/scale/}{reference}
\end{quote}

\begin{verbatim}
#scale(x: -100%)[This is mirrored.]
\end{verbatim}

\pandocbounded{\includesvg[keepaspectratio]{basics/must_know/typst-img/401c8cd6f306771a3b12432c3c51e097a3ec1d12656c131c0043a12c4c1c3a0e-1.svg}}

\begin{verbatim}
A#box(scale(75%)[A])A \
B#box(scale(75%, origin: bottom + left)[B])B
\end{verbatim}

\pandocbounded{\includesvg[keepaspectratio]{basics/must_know/typst-img/204b55690645eb6cc623c8d2d74b5521d72e4ba38d58ea40ea5e2d4354a01836-1.svg}}

\section{\texorpdfstring{\hyperref[hide]{Hide}}{Hide}}\label{hide}

Don\textquotesingle t show content, but leave empty space there.

\begin{quote}
Link to \href{https://typst.app/docs/reference/layout/hide/}{reference}
\end{quote}

\begin{verbatim}
Hello Jane \
#hide[Hello] Joe
\end{verbatim}

\pandocbounded{\includesvg[keepaspectratio]{basics/must_know/typst-img/610672d5e43baa3ce94fe61f8d6dd0307e405c785639359c6a9e84bdd66884ad-1.svg}}

\section{\texorpdfstring{\hyperref[tables-and-grids]{Tables and
grids}}{Tables and grids}}\label{tables-and-grids}

While tables are not that necessary to know if you don\textquotesingle t
plan to use them in your documents, grids may be very useful for
\emph{document layout} . We will use both of them them in the book
later.

Let\textquotesingle s not bother with copying examples from official
documentation. Just make sure to skim through it, okay?

\subsection{\texorpdfstring{\hyperref[basic-snippets]{Basic
snippets}}{Basic snippets}}\label{basic-snippets}

\subsubsection{\texorpdfstring{\hyperref[spreading]{Spreading}}{Spreading}}\label{spreading}

Spreading operators (see
\href{basics/must_know/../scripting/arguments.html}{there} ) may be
especially useful for the tables:

\begin{verbatim}
#set text(size: 9pt)

#let yield_cells(n) = {
  for i in range(0, n + 1) {
    for j in range(0, n + 1) {
      let product = if i * j != 0 {
        // math is used for the better look 
        if j <= i { $#{ j * i }$ } 
        else {
          // upper part of the table
          text(gray.darken(50%), str(i * j))
        }
      } else {
        if i == j {
          // the top right corner 
          $times$
        } else {
          // on of them is zero, we are at top/left
          $#{i + j}$
        }
      }
      // this is an array, for loops merge them together
      // into one large array of cells
      (
        table.cell(
          fill: if i == j and j == 0 { orange } // top right corner
          else if i == j { yellow } // the diagonal
          else if i * j == 0 { blue.lighten(50%) }, // multipliers
          product,),
      )
    }
  }
}

#let n = 10
#table(
  columns: (0.6cm,) * (n + 1), rows: (0.6cm,) * (n + 1), align: center + horizon, inset: 3pt, ..yield_cells(n),
)
\end{verbatim}

\pandocbounded{\includesvg[keepaspectratio]{basics/must_know/typst-img/0640c1d0e5f79bdcb5e60f7675ff1b1eb18810078f5bbbdfaf1c5648b987706e-1.svg}}

\subsubsection{\texorpdfstring{\hyperref[highlighting-table-row]{Highlighting
table row}}{Highlighting table row}}\label{highlighting-table-row}

\begin{verbatim}
#table(
  columns: 2,
  fill: (x, y) => if y == 2 { highlight.fill },
  [A], [B],
  [C], [D],
  [E], [F],
  [G], [H],
)
\end{verbatim}

\pandocbounded{\includesvg[keepaspectratio]{basics/must_know/typst-img/4ff8cbb75f85dbab08a336be31115bcb4cb8ca505799641534d937d444e88082-1.svg}}

For individual cells, use

\begin{verbatim}
#table(
  columns: 2,
  [A], [B],
  table.cell(fill: yellow)[C], table.cell(fill: yellow)[D],
  [E], [F],
  [G], [H],
)
\end{verbatim}

\pandocbounded{\includesvg[keepaspectratio]{basics/must_know/typst-img/07676a86d4643ff83988c0907aa17995b3d1f8fa7b5be4f11959551afd674bc9-1.svg}}

\subsubsection{\texorpdfstring{\hyperref[splitting-tables]{Splitting
tables}}{Splitting tables}}\label{splitting-tables}

Tables are split between pages automatically.

\begin{verbatim}
#set page(height: 8em)
#(
table(
  columns: 5,
  [Aligner], [publication], [Indexing], [Pairwise alignment], [Max. read length  (bp)],
  [BWA], [2009], [BWT-FM], [Semi-Global], [125],
  [Bowtie], [2009], [BWT-FM], [HD], [76],
  [CloudBurst], [2009], [Hashing], [Landau-Vishkin], [36],
  [GNUMAP], [2009], [Hashing], [NW], [36]
  )
)
\end{verbatim}

\pandocbounded{\includesvg[keepaspectratio]{basics/must_know/typst-img/34794c27fefc5c307a1dfdc9ad7958c1dcca0ff8fb64962047051c6a216e0ff7-1.svg}}

\pandocbounded{\includesvg[keepaspectratio]{basics/must_know/typst-img/34794c27fefc5c307a1dfdc9ad7958c1dcca0ff8fb64962047051c6a216e0ff7-2.svg}}

However, if you want to make it breakable inside other element,
you\textquotesingle ll have to make that element breakable too:

\begin{verbatim}
#set page(height: 8em)
// Without this, the table fails to split upon several pages
#show figure: set block(breakable: true)
#figure(
table(
  columns: 5,
  [Aligner], [publication], [Indexing], [Pairwise alignment], [Max. read length  (bp)],
  [BWA], [2009], [BWT-FM], [Semi-Global], [125],
  [Bowtie], [2009], [BWT-FM], [HD], [76],
  [CloudBurst], [2009], [Hashing], [Landau-Vishkin], [36],
  [GNUMAP], [2009], [Hashing], [NW], [36]
  )
)
\end{verbatim}

\pandocbounded{\includesvg[keepaspectratio]{basics/must_know/typst-img/5be04bf8770a33256599791fb50751bcb24fa5108c13d0e5e2807b675fed00fb-1.svg}}

\pandocbounded{\includesvg[keepaspectratio]{basics/must_know/typst-img/5be04bf8770a33256599791fb50751bcb24fa5108c13d0e5e2807b675fed00fb-2.svg}}

\section{\texorpdfstring{\hyperref[project-structure]{Project
structure}}{Project structure}}\label{project-structure}

\subsection{\texorpdfstring{\hyperref[large-document]{Large
document}}{Large document}}\label{large-document}

Once the document becomes large enough, it becomes harder to navigate
it. If you haven\textquotesingle t reached that size yet, you can ignore
that section.

For managing that I would recommend splitting your document into
\emph{chapters} . It is just a way to work with this, but once you
understand how it works, you can do anything you want.

Let\textquotesingle s say you have two chapters, then the recommended
structure will look like this:

\begin{verbatim}
#import "@preview/treet:0.1.1": *

#show list: tree-list
#set par(leading: 0.8em)
#show list: set text(font: "DejaVu Sans Mono", size: 0.8em)
- chapters/
  - chapter_1.typ
  - chapter_2.typ
- main.typ 👁 #text(gray)[← document entry point]
- template.typ
\end{verbatim}

\pandocbounded{\includesvg[keepaspectratio]{basics/must_know/typst-img/291489e71b40beea77872ad05adb609349872e9a11fc3a9c3f2008c88e37c9d5-1.svg}}

The exact file names are up to you.

Let\textquotesingle s see what to put in each of these files.

\subsubsection{\texorpdfstring{\hyperref[template]{Template}}{Template}}\label{template}

In the "template" file goes \emph{all useful functions and variables}
you will use across the chapters. If you have your own template or want
to write one, you can write it there.

\begin{verbatim}
// template.typ

#let template = doc => {
    set page(header: "My super document")
    show "physics": "magic"
    doc
}

#let info-block = block.with(stroke: blue, fill: blue.lighten(70%))
#let author = "@sitandr"
\end{verbatim}

\subsubsection{\texorpdfstring{\hyperref[main]{Main}}{Main}}\label{main}

\textbf{This file should be compiled} to get the whole compiled
document.

\begin{verbatim}
// main.typ

#import "template.typ": *
// if you have a template
#show: template

= This is the document title

// some additional formatting

#show emph: set text(blue)

// but don't define functions or variables there!
// chapters will not see it

// Now the chapters themselves as some Typst content
#include("chapters/chapter_1.typ")
#include("chapters/chapter_1.typ")
\end{verbatim}

\subsubsection{\texorpdfstring{\hyperref[chapter]{Chapter}}{Chapter}}\label{chapter}

\begin{verbatim}
// chapter_1.typ

#import "../template.typ": *

That's just content with _styling_ and blocks:

#infoblock[Some information].

// just any content you want to include in the document
\end{verbatim}

\subsection{\texorpdfstring{\hyperref[notes]{Notes}}{Notes}}\label{notes}

Note that modules in Typst can see only what they created themselves or
imported. Anything else is invisible for them. That\textquotesingle s
why you need \texttt{\ }{\texttt{\ template.typ\ }}\texttt{\ } file to
define all functions within.

That means chapters \emph{don\textquotesingle t see each other either} ,
only what is in the template.

\subsection{\texorpdfstring{\hyperref[cyclic-imports]{Cyclic
imports}}{Cyclic imports}}\label{cyclic-imports}

\textbf{Important:} Typst \emph{forbids} cyclic imports. That means you
can\textquotesingle t import
\texttt{\ }{\texttt{\ chapter\_1\ }}\texttt{\ } from
\texttt{\ }{\texttt{\ chapter\_2\ }}\texttt{\ } and
\texttt{\ }{\texttt{\ chapter\_2\ }}\texttt{\ } from
\texttt{\ }{\texttt{\ chapter\_1\ }}\texttt{\ } at the same time!

But the good news is that you can always create some other file to
import variable from.

\section{\texorpdfstring{\hyperref[scripting]{Scripting}}{Scripting}}\label{scripting}

\textbf{Typst} has a complete interpreted language inside. One of key
aspects of working with your document in a nicer way

\section{\texorpdfstring{\hyperref[basics]{Basics}}{Basics}}\label{basics}

\subsection{\texorpdfstring{\hyperref[variables-i]{Variables
I}}{Variables I}}\label{variables-i}

Let\textquotesingle s start with \emph{variables} .

The concept is very simple, just some value you can reuse:

\begin{verbatim}
#let author = "John Doe"

This is a book by #author. #author is a great guy.

#quote(block: true, attribution: author)[
  \<Some quote\>
]
\end{verbatim}

\pandocbounded{\includesvg[keepaspectratio]{basics/scripting/typst-img/c311c1612cafa802f16f0d4ca2d6f1ecca59f545ed1f6ee99d3c4ae06ee2bff4-1.svg}}

\subsection{\texorpdfstring{\hyperref[variables-ii]{Variables
II}}{Variables II}}\label{variables-ii}

You can store \emph{any} Typst value in variable:

\begin{verbatim}
#let block_text = block(stroke: red, inset: 1em)[Text]

#block_text

#figure(caption: "The block", block_text)
\end{verbatim}

\pandocbounded{\includesvg[keepaspectratio]{basics/scripting/typst-img/c6290389652d1771d5149c9393af8eb32bd37e4b2bfb2c11764f9f22c294f84b-1.svg}}

\subsection{\texorpdfstring{\hyperref[functions-2]{Functions}}{Functions}}\label{functions-2}

We have already seen some "custom" functions in
\href{basics/scripting/../tutorial/advanced_styling.html}{Advanced
Styling} chapter.

Functions are values that take some values and output some values:

\begin{verbatim}
// This is a syntax that we have seen earlier
#let f = (name) => "Hello, " + name

#f("world!")
\end{verbatim}

\pandocbounded{\includesvg[keepaspectratio]{basics/scripting/typst-img/23fba8e9081a8b32b16d7deb54018bb73a8ac910adbfb1a0ca577eb3520a73b4-1.svg}}

\subsubsection{\texorpdfstring{\hyperref[alternative-syntax]{Alternative
syntax}}{Alternative syntax}}\label{alternative-syntax}

You can write the same shorter:

\begin{verbatim}
// The following syntaxes are equivalent
#let f = (name) => "Hello, " + name
#let f(name) = "Hello, " + name

#f("world!")
\end{verbatim}

\pandocbounded{\includesvg[keepaspectratio]{basics/scripting/typst-img/e6e4bd179a38f1b3af96f3e7c6308be6f9494f41f43daa26ebabf7a77fc54780-1.svg}}

\section{\texorpdfstring{\hyperref[braces-brackets-and-default]{Braces,
brackets and
default}}{Braces, brackets and default}}\label{braces-brackets-and-default}

\subsection{\texorpdfstring{\hyperref[square-brackets]{Square
brackets}}{Square brackets}}\label{square-brackets}

You may remember that square brackets convert everything inside to
\emph{content} .

\begin{verbatim}
#let v = [Some text, _markup_ and other #strong[functions]]
#v
\end{verbatim}

\pandocbounded{\includesvg[keepaspectratio]{basics/scripting/typst-img/5ba617daa8d4c166d96a0abbba02d6502fe7fde1ded460afa78682993295142d-1.svg}}

We may use same for functions bodies:

\begin{verbatim}
#let f(name) = [Hello, #name]
#f[World] // also don't forget we can use it to pass content!
\end{verbatim}

\pandocbounded{\includesvg[keepaspectratio]{basics/scripting/typst-img/4545deeee45655ee6666feb4773416cd075fe7522cbfd80d0847c615c6c5f30a-1.svg}}

\textbf{Important:} It is very hard to convert \emph{content} to
\emph{plain text} , as \emph{content} may contain \emph{anything} ! Sp
be careful when passing and storing content in variables.

\subsection{\texorpdfstring{\hyperref[braces]{Braces}}{Braces}}\label{braces}

However, we often want to use code inside functions.
That\textquotesingle s when we use
\texttt{\ }{\texttt{\ \{\}\ }}\texttt{\ } :

\begin{verbatim}
#let f(name) = {
  // this is code mode

  // First part of our output
  "Hello, "

  // we check if name is empty, and if it is,
  // insert placeholder
  if name == "" {
      "anonym"
  } else {
      name
  }

  // finish sentence
  "!"
}

#f("")
#f("Joe")
#f("world")
\end{verbatim}

\pandocbounded{\includesvg[keepaspectratio]{basics/scripting/typst-img/f2bc6aebef06f213c9a8e740266a96e424318d953c09cffba6c5811375e91395-1.svg}}

\subsection{\texorpdfstring{\hyperref[scopes]{Scopes}}{Scopes}}\label{scopes}

\textbf{This is a very important thing to remember} .

\emph{You can\textquotesingle t use variables outside of scopes they are
defined (unless it is file root, then you can import them)} . \emph{Set
and show rules affect things in their scope only.}

\begin{verbatim}
#{
  let a = 3;
}
// can't use "a" there.

#[
  #show "true": "false"

  This is true.
]

This is true.
\end{verbatim}

\pandocbounded{\includesvg[keepaspectratio]{basics/scripting/typst-img/c25d356831eeea19bb243b87c0f32d062c7086a55b4ee432e41b388d626f875b-1.svg}}

\subsection{\texorpdfstring{\hyperref[return]{Return}}{Return}}\label{return}

\textbf{Important} : by default braces return anything that "returns"
into them. For example,

\begin{verbatim}
#let change_world() = {
  // some code there changing everything in the world
  str(4e7)
  // another code changing the world
}

#let g() = {
  "Hahaha, I will change the world now! "
  change_world()
  " So here is my long evil monologue..."
}

#g()
\end{verbatim}

\pandocbounded{\includesvg[keepaspectratio]{basics/scripting/typst-img/160d9672bd7abc64ea61943d1bfcbd1b06dc70f87be5e5cf9c411fe4ee6d2a44-1.svg}}

To avoid returning everything, return only what you want explicitly,
otherwise everything will be joined:

\begin{verbatim}
#let f() = {
  "Some long text"
  // Crazy numbers
  "2e7"
  return none
}

// Returns nothing
#f()
\end{verbatim}

\pandocbounded{\includesvg[keepaspectratio]{basics/scripting/typst-img/14c935733a8c91165ee4ebe8246efb841207feeaa0309e36a1cde2888acffb10-1.svg}}

\subsection{\texorpdfstring{\hyperref[default-values]{Default
values}}{Default values}}\label{default-values}

What we made just now was inventing "default values".

They are very common in styling, so there is a special syntax for them:

\begin{verbatim}
#let f(name: "anonym") = [Hello, #name!]

#f()
#f(name: "Joe")
#f(name: "world")
\end{verbatim}

\pandocbounded{\includesvg[keepaspectratio]{basics/scripting/typst-img/e9730d0d1f30ec9f2404179560ae4a4b19dd788b1afc2f6b956fb32268439cb6-1.svg}}

You may have noticed that the argument became \emph{named} now. In
Typst, named argument is an argument \emph{that has default value} .

\section{\texorpdfstring{\hyperref[types-part-i]{Types, part
I}}{Types, part I}}\label{types-part-i}

Each value in Typst has a type. You don\textquotesingle t have to
specify it, but it is important.

\subsection{\texorpdfstring{\hyperref[content-content]{Content (
\texttt{\ }{\texttt{\ content\ }}\texttt{\ }
)}}{Content (   content   )}}\label{content-content}

\begin{quote}
\href{https://typst.app/docs/reference/foundations/content/}{Link to
Reference} .
\end{quote}

We have already seen it. A type that represents what is displayed in
document.

\begin{verbatim}
#let c = [It is _content_!]

// Check type of c
#(type(c) == content)

#c

// repr gives an "inner representation" of value
#repr(c)
\end{verbatim}

\pandocbounded{\includesvg[keepaspectratio]{basics/scripting/typst-img/21fd80460de8e8a377a9ef2046a27232ad88924070509ccf8647c9135c9c2fe3-1.svg}}

\textbf{Important:} It is very hard to convert \emph{content} to
\emph{plain text} , as \emph{content} may contain \emph{anything} ! So
be careful when passing and storing content in variables.

\subsection{\texorpdfstring{\hyperref[none-none]{None (
\texttt{\ }{\texttt{\ none\ }}\texttt{\ }
)}}{None (   none   )}}\label{none-none}

Nothing. Also known as \texttt{\ }{\texttt{\ null\ }}\texttt{\ } in
other languages. It isn\textquotesingle t displayed, converts to empty
content.

\begin{verbatim}
#none
#repr(none)
\end{verbatim}

\pandocbounded{\includesvg[keepaspectratio]{basics/scripting/typst-img/c4100c1d1df8fc0a51bd99945d9bac3c5aa67de19b8f872fd33fd9068bb2507b-1.svg}}

\subsection{\texorpdfstring{\hyperref[string-str]{String (
\texttt{\ }{\texttt{\ str\ }}\texttt{\ }
)}}{String (   str   )}}\label{string-str}

\begin{quote}
\href{https://typst.app/docs/reference/foundations/str/}{Link to
Reference} .
\end{quote}

String contains only plain text and no formatting. Just some chars. That
allows us to work with chars:

\begin{verbatim}
#let s = "Some large string. There could be escape sentences: \n,
 line breaks, and even unicode codes: \u{1251}"
#s \
#type(s) \
`repr`: #repr(s)

#let s = "another small string"
#s.replace("a", sym.alpha) \
#s.split(" ") // split by space
\end{verbatim}

\pandocbounded{\includesvg[keepaspectratio]{basics/scripting/typst-img/b797f9c4a540fcf1429bec801d0b334e7d88dc9ccd10e3b7b859f451e269f30f-1.svg}}

You can convert other types to their string representation using this
type\textquotesingle s constructor (e.g. convert number to string):

\begin{verbatim}
#str(5) // string, can be worked with as string
\end{verbatim}

\pandocbounded{\includesvg[keepaspectratio]{basics/scripting/typst-img/ab4d4a5d93533525f7f9b2cc8378b79f1561904f3c5d5f6d2ec4bdc448669cb5-1.svg}}

\subsection{\texorpdfstring{\hyperref[boolean-bool]{Boolean (
\texttt{\ }{\texttt{\ bool\ }}\texttt{\ }
)}}{Boolean (   bool   )}}\label{boolean-bool}

\begin{quote}
\href{https://typst.app/docs/reference/foundations/bool/}{Link to
Reference} .
\end{quote}

true/false. Used in \texttt{\ }{\texttt{\ if\ }}\texttt{\ } and many
others

\begin{verbatim}
#let b = false
#b \
#repr(b) \
#(true and not true or true) = #((true and (not true)) or true) \
#if (4 > 3) {
  "4 is more than 3"
}
\end{verbatim}

\pandocbounded{\includesvg[keepaspectratio]{basics/scripting/typst-img/e848d78e220ca8cf3b6c323a99d5d963e529aad36857f0e6753c56c02984a616-1.svg}}

\subsection{\texorpdfstring{\hyperref[integer-int]{Integer (
\texttt{\ }{\texttt{\ int\ }}\texttt{\ }
)}}{Integer (   int   )}}\label{integer-int}

\begin{quote}
\href{https://typst.app/docs/reference/foundations/int/}{Link to
Reference} .
\end{quote}

A whole number.

The number can also be specified as hexadecimal, octal, or binary by
starting it with a zero followed by either x, o, or b.

\begin{verbatim}
#let n = 5
#n \
#(n += 1) \
#n \
#calc.pow(2, n) \
#type(n) \
#repr(n)
\end{verbatim}

\pandocbounded{\includesvg[keepaspectratio]{basics/scripting/typst-img/6f1c9e02393e14aa23add33d0e6dc2b596ee97a0d425cd3edb3e2b912c6ef6b0-1.svg}}

\begin{verbatim}
#(1 + 2) \
#(2 - 5) \
#(3 + 4 < 8)
\end{verbatim}

\pandocbounded{\includesvg[keepaspectratio]{basics/scripting/typst-img/e610f15659cb6b64c3516be48740b54e6caf3d933919004157ba64b757389ba5-1.svg}}

\begin{verbatim}
#0xff \
#0o10 \
#0b1001
\end{verbatim}

\pandocbounded{\includesvg[keepaspectratio]{basics/scripting/typst-img/1446dba05ee6f8006884c280ff32e31ede8425d4847445e97cae5dfcde1efe7f-1.svg}}

You can convert a value to an integer with this type\textquotesingle s
constructor (e.g. convert string to int).

\begin{verbatim}
#int(false) \
#int(true) \
#int(2.7) \
#(int("27") + int("4"))
\end{verbatim}

\pandocbounded{\includesvg[keepaspectratio]{basics/scripting/typst-img/b44779a87fd984d317ec4d1aed732c0ebdc6220fd4764e407f77fedd139c0d8c-1.svg}}

\subsection{\texorpdfstring{\hyperref[float-float]{Float (
\texttt{\ }{\texttt{\ float\ }}\texttt{\ }
)}}{Float (   float   )}}\label{float-float}

\begin{quote}
\href{https://typst.app/docs/reference/foundations/float/}{Link to
Reference} .
\end{quote}

Works the same way as integer, but can store floating point numbers.
However, precision may be lost.

\begin{verbatim}
#let n = 5.0

// You can mix floats and integers, 
// they will be implicitly converted
#(n += 1) \
#calc.pow(2, n) \
#(0.2 + 0.1) \
#type(n) 
\end{verbatim}

\pandocbounded{\includesvg[keepaspectratio]{basics/scripting/typst-img/21cafe751ec803dd9598c871b283a29bc3c6b2e302f0f9bd78edc17330b45616-1.svg}}

\begin{verbatim}
#3.14 \
#1e4 \
#(10 / 4)
\end{verbatim}

\pandocbounded{\includesvg[keepaspectratio]{basics/scripting/typst-img/05bd400096c1df5a954fda0897f3c1756c9f99f73503d32d992b3222667a45cd-1.svg}}

You can convert a value to a float with this type\textquotesingle s
constructor (e.g. convert string to float).

\begin{verbatim}
#float(40%) \
#float("2.7") \
#float("1e5")
\end{verbatim}

\pandocbounded{\includesvg[keepaspectratio]{basics/scripting/typst-img/f50a22cbea42fded97ab8340f0939e786e5c1cdb5ea531cd4b35b1f732947b7f-1.svg}}

\section{\texorpdfstring{\hyperref[types-part-ii]{Types, part
II}}{Types, part II}}\label{types-part-ii}

In Typst, most of things are \textbf{immutable} . You
can\textquotesingle t change content, you can just create new using this
one (for example, using addition).

Immutability is very important for Typst since it tries to be \emph{as
pure language as possible} . Functions do nothing outside of returning
some value.

However, purity is partly "broken" by these types. They are
\emph{super-useful} and not adding them would make Typst much pain.

However, using them adds complexity.

\subsection{\texorpdfstring{\hyperref[arrays-array]{Arrays (
\texttt{\ }{\texttt{\ array\ }}\texttt{\ }
)}}{Arrays (   array   )}}\label{arrays-array}

\begin{quote}
\href{https://typst.app/docs/reference/foundations/array/}{Link to
Reference} .
\end{quote}

Mutable object that stores data with their indices.

\subsubsection{\texorpdfstring{\hyperref[working-with-indices]{Working
with indices}}{Working with indices}}\label{working-with-indices}

\begin{verbatim}
#let values = (1, 7, 4, -3, 2)

// take value at index 0
#values.at(0) \
// set value at 0 to 3
#(values.at(0) = 3)
// negative index => start from the back
#values.at(-1) \
// add index of something that is even
#values.find(calc.even)
\end{verbatim}

\pandocbounded{\includesvg[keepaspectratio]{basics/scripting/typst-img/0374c20b28fbf2b2d15bc32e5428f7f5121ea9d673d96de3274a0c6d988d5fb5-1.svg}}

\subsubsection{\texorpdfstring{\hyperref[iterating-methods]{Iterating
methods}}{Iterating methods}}\label{iterating-methods}

\begin{verbatim}
#let values = (1, 7, 4, -3, 2)

// leave only what is odd
#values.filter(calc.odd) \
// create new list of absolute values of list values
#values.map(calc.abs) \
// reverse
#values.rev() \
// convert array of arrays to flat array
#(1, (2, 3)).flatten() \
// join array of string to string
#(("A", "B", "C")
 .join(", ", last: " and "))
\end{verbatim}

\pandocbounded{\includesvg[keepaspectratio]{basics/scripting/typst-img/684400186916f8f16a2d7edb151b7f5023c7e4c010b23a2c6566f0bd7a224061-1.svg}}

\subsubsection{\texorpdfstring{\hyperref[list-operations]{List
operations}}{List operations}}\label{list-operations}

\begin{verbatim}
// sum of lists:
#((1, 2, 3) + (4, 5, 6))

// list product:
#((1, 2, 3) * 4)
\end{verbatim}

\pandocbounded{\includesvg[keepaspectratio]{basics/scripting/typst-img/abe2d311638b351e0938be0e432f10265ca81a69a9ed7d2e6f88f656c60dfc65-1.svg}}

\subsubsection{\texorpdfstring{\hyperref[empty-list]{Empty
list}}{Empty list}}\label{empty-list}

\begin{verbatim}
#() \ // this is an empty list
#(1,) \  // this is a list with one element
BAD: #(1) // this is just an element, not a list!
\end{verbatim}

\pandocbounded{\includesvg[keepaspectratio]{basics/scripting/typst-img/da4f77f8784462ca5c4f73862e58420695916064d56921e4adef7a7e37d5a532-1.svg}}

\subsection{\texorpdfstring{\hyperref[dictionaries-dict]{Dictionaries (
\texttt{\ }{\texttt{\ dict\ }}\texttt{\ }
)}}{Dictionaries (   dict   )}}\label{dictionaries-dict}

\begin{quote}
\href{https://typst.app/docs/reference/foundations/dictionary/}{Link to
Reference} .
\end{quote}

Dictionaries are objects that store a string "key" and a value,
associated with that key.

\begin{verbatim}
#let dict = (
  name: "Typst",
  born: 2019,
)

#dict.name \
#(dict.launch = 20)
#dict.len() \
#dict.keys() \
#dict.values() \
#dict.at("born") \
#dict.insert("city", "Berlin ")
#("name" in dict)
\end{verbatim}

\pandocbounded{\includesvg[keepaspectratio]{basics/scripting/typst-img/638ada64eb36af0b1891def1b2c0a2cc97a14d87987df8c16f5f3872244553d6-1.svg}}

\subsubsection{\texorpdfstring{\hyperref[empty-dictionary]{Empty
dictionary}}{Empty dictionary}}\label{empty-dictionary}

\begin{verbatim}
This is an empty list: #() \
This is an empty dict: #(:)
\end{verbatim}

\pandocbounded{\includesvg[keepaspectratio]{basics/scripting/typst-img/6ef41801d46f0b7256bb6913482fde054c811a1850ecae3a446660eb6d1c8850-1.svg}}

\section{\texorpdfstring{\hyperref[conditions--loops]{Conditions \&
loops}}{Conditions \& loops}}\label{conditions--loops}

\subsection{\texorpdfstring{\hyperref[conditions]{Conditions}}{Conditions}}\label{conditions}

\begin{quote}
See
\href{https://typst.app/docs/reference/scripting/\#conditionals}{official
documentation} .
\end{quote}

In Typst, you can use \texttt{\ }{\texttt{\ if-else\ }}\texttt{\ }
statements. This is especially useful inside function bodies to vary
behavior depending on arguments types or many other things.

\begin{verbatim}
#if 1 < 2 [
  This is shown
] else [
  This is not.
]
\end{verbatim}

\pandocbounded{\includesvg[keepaspectratio]{basics/scripting/typst-img/2e914defa3353d6fd42ed58c37a97aedcc2237cfe20228f0cc0d223dfff4619a-1.svg}}

Of course, \texttt{\ }{\texttt{\ else\ }}\texttt{\ } is unnecessary:

\begin{verbatim}
#let a = 3

#if a < 4 {
  a = 5
}

#a
\end{verbatim}

\pandocbounded{\includesvg[keepaspectratio]{basics/scripting/typst-img/a7264774be154606a44d829d31edae18bf686262ccea66de9ed97fa20c720bd8-1.svg}}

You can also use \texttt{\ }{\texttt{\ else\ if\ }}\texttt{\ } statement
(known as \texttt{\ }{\texttt{\ elif\ }}\texttt{\ } in Python):

\begin{verbatim}
#let a = 5

#if a < 4 {
  a = 5
} else if a < 6 {
  a = -3
}

#a
\end{verbatim}

\pandocbounded{\includesvg[keepaspectratio]{basics/scripting/typst-img/9f65678fc26af2d197d979e1b0a5295ed64037ee00c30fa28c9c417a2c7dc308-1.svg}}

\subsubsection{\texorpdfstring{\hyperref[booleans]{Booleans}}{Booleans}}\label{booleans}

\texttt{\ }{\texttt{\ if,\ else\ if,\ else\ }}\texttt{\ } accept
\emph{only boolean} values as a switch. You can combine booleans as
described in \href{basics/scripting/./types.html\#boolean-bool}{types
section} :

\begin{verbatim}
#let a = 5

#if (a > 1 and a <= 4) or a == 5 [
    `a` matches the condition
]
\end{verbatim}

\pandocbounded{\includesvg[keepaspectratio]{basics/scripting/typst-img/21d3a48404d4e0c59bc0fccb114fdeac7384189db0020247796f44b0e9a7c362-1.svg}}

\subsection{\texorpdfstring{\hyperref[loops]{Loops}}{Loops}}\label{loops}

\begin{quote}
See \href{https://typst.app/docs/reference/scripting/\#loops}{official
documentation} .
\end{quote}

There are two kinds of loops: \texttt{\ }{\texttt{\ while\ }}\texttt{\ }
and \texttt{\ }{\texttt{\ for\ }}\texttt{\ } . While repeats body while
the condition is met:

\begin{verbatim}
#let a = 3

#while a < 100 {
    a *= 2
    str(a)
    " "
}
\end{verbatim}

\pandocbounded{\includesvg[keepaspectratio]{basics/scripting/typst-img/ece06c012663616cac05b0f365bd02ea5607dcddfaa0249963088ceff797c100-1.svg}}

\texttt{\ }{\texttt{\ for\ }}\texttt{\ } iterates over all elements of
sequence. The sequence may be an
\texttt{\ }{\texttt{\ array\ }}\texttt{\ } ,
\texttt{\ }{\texttt{\ string\ }}\texttt{\ } or
\texttt{\ }{\texttt{\ dictionary\ }}\texttt{\ } (
\texttt{\ }{\texttt{\ for\ }}\texttt{\ } iterates over its
\emph{key-value pairs} ).

\begin{verbatim}
#for c in "ABC" [
  #c is a letter.
]
\end{verbatim}

\pandocbounded{\includesvg[keepaspectratio]{basics/scripting/typst-img/9e70091e4c1f276d548f8200329298bf6b98946c331ca4630fec8313d5a91eff-1.svg}}

To iterate to all numbers from \texttt{\ }{\texttt{\ a\ }}\texttt{\ } to
\texttt{\ }{\texttt{\ b\ }}\texttt{\ } , use
\texttt{\ }{\texttt{\ range(a,\ b+1)\ }}\texttt{\ } :

\begin{verbatim}
#let s = 0

#for i in range(3, 6) {
    s += i
    [Number #i is added to sum. Now sum is #s.]
}
\end{verbatim}

\pandocbounded{\includesvg[keepaspectratio]{basics/scripting/typst-img/1e3d95ee79d7bc6989e40ff1e27c0ef6e3b152a1e5f8a0df5b2819621e0e299f-1.svg}}

Because range is end-exclusive this is equal to

\begin{verbatim}
#let s = 0

#for i in (3, 4, 5) {
    s += i
    [Number #i is added to sum. Now sum is #s.]
}
\end{verbatim}

\pandocbounded{\includesvg[keepaspectratio]{basics/scripting/typst-img/6158d29261339f8f285d592deff8992ca129ce32264abcdcf6734ac44cf558a4-1.svg}}

\begin{verbatim}
#let people = (Alice: 3, Bob: 5)

#for (name, value) in people [
    #name has #value apples.
]
\end{verbatim}

\pandocbounded{\includesvg[keepaspectratio]{basics/scripting/typst-img/50ff0963afe8c9ec5a0562d518431b63d5dd3810525f55f084f812452b11eb21-1.svg}}

\subsubsection{\texorpdfstring{\hyperref[break-and-continue]{Break and
continue}}{Break and continue}}\label{break-and-continue}

Inside loops can be used \texttt{\ }{\texttt{\ break\ }}\texttt{\ } and
\texttt{\ }{\texttt{\ continue\ }}\texttt{\ } commands.
\texttt{\ }{\texttt{\ break\ }}\texttt{\ } breaks loop, jumping outside.
\texttt{\ }{\texttt{\ continue\ }}\texttt{\ } jumps to next loop
iteration.

See the difference on these examples:

\begin{verbatim}
#for letter in "abc nope" {
  if letter == " " {
    // stop when there is space
    break
  }

  letter
}
\end{verbatim}

\pandocbounded{\includesvg[keepaspectratio]{basics/scripting/typst-img/a744551cab635d3ab70d9bf4258bb5fc26fe384f8e9f487ad0b8eee986ffe581-1.svg}}

\begin{verbatim}
#for letter in "abc nope" {
  if letter == " " {
    // skip the space
    continue
  }

  letter
}
\end{verbatim}

\pandocbounded{\includesvg[keepaspectratio]{basics/scripting/typst-img/bbb719820f986e52fbf64306536766ecbfd7264d29429a5c62d1bd648a4754c5-1.svg}}

\section{\texorpdfstring{\hyperref[advanced-arguments]{Advanced
arguments}}{Advanced arguments}}\label{advanced-arguments}

\subsection{\texorpdfstring{\hyperref[spreading-arguments-from-list]{Spreading
arguments from
list}}{Spreading arguments from list}}\label{spreading-arguments-from-list}

Spreading operator allows you to "unpack" the list of values into
arguments of function:

\begin{verbatim}
#let func(a, b, c, d, e) = [#a #b #c #d #e]
#func(..(([hi],) * 5))
\end{verbatim}

\pandocbounded{\includesvg[keepaspectratio]{basics/scripting/typst-img/0586f1f7eb73effd507824b57f7282f12fe2612119d64413f72e6518aba01513-1.svg}}

This may be super useful in tables:

\begin{verbatim}
#let a = ("hi", "b", "c")

#table(columns: 3,
  [test], [x], [hello],
  ..a
)
\end{verbatim}

\pandocbounded{\includesvg[keepaspectratio]{basics/scripting/typst-img/eb669f70df63815adcbe764fdb8635eecab33651c7eef55ea4de6cd63c96d9de-1.svg}}

\subsection{\texorpdfstring{\hyperref[key-arguments]{Key
arguments}}{Key arguments}}\label{key-arguments}

The same idea works with key arguments:

\begin{verbatim}
#let text-params = (fill: blue, size: 0.8em)

Some #text(..text-params)[text].
\end{verbatim}

\pandocbounded{\includesvg[keepaspectratio]{basics/scripting/typst-img/e56483e8f4285f8fed8cd6867e720b9a1c9f62ef0bffea28d124159f8a61648d-1.svg}}

\section{\texorpdfstring{\hyperref[managing-arbitrary-arguments]{Managing
arbitrary
arguments}}{Managing arbitrary arguments}}\label{managing-arbitrary-arguments}

Typst allows taking as many arbitrary positional and key arguments as
you want.

In that case function is given special
\texttt{\ }{\texttt{\ arguments\ }}\texttt{\ } object that stores in it
positional and named arguments.

\begin{quote}
Link to
\href{https://typst.app/docs/reference/foundations/arguments/}{reference}
\end{quote}

\begin{verbatim}
#let f(..args) = [
  #args.pos()\
  #args.named()
]

#f(1, "a", width: 50%, block: false)
\end{verbatim}

\pandocbounded{\includesvg[keepaspectratio]{basics/scripting/typst-img/2fc64c8521734ea689368ec83fe54025eb94b016a8ed1f6d6a9880ac6c94edf5-1.svg}}

You can combine them with other arguments. Spreading operator will "eat"
all remaining arguments:

\begin{verbatim}
#let format(title, ..authors) = {
  let by = authors
    .pos()
    .join(", ", last: " and ")

  [*#title* \ _Written by #by;_]
}

#format("ArtosFlow", "Jane", "Joe")
\end{verbatim}

\pandocbounded{\includesvg[keepaspectratio]{basics/scripting/typst-img/4ba76c5176e0b93c6c2b03c38d55f88702546a5183717ed8c3567865c0d1bf5d-1.svg}}

\subsection{\texorpdfstring{\hyperref[optional-argument]{Optional
argument}}{Optional argument}}\label{optional-argument}

\emph{Currently the only way in Typst to create optional positional
arguments is using \texttt{\ }{\texttt{\ arguments\ }}\texttt{\ }
object:}

TODO

\section{\texorpdfstring{\hyperref[tips]{Tips}}{Tips}}\label{tips}

There are lots of elements in Typst scripting that are not obvious, but
important. All the book is designated to show them, but some of them

\subsection{\texorpdfstring{\hyperref[equality]{Equality}}{Equality}}\label{equality}

Equality doesn\textquotesingle t mean objects are really the same, like
in many other objects:

\begin{verbatim}
#let a = 7
#let b = 7.0
#(a == b)
#(type(a) == type(b))
\end{verbatim}

\pandocbounded{\includesvg[keepaspectratio]{basics/scripting/typst-img/3632e0202f7aae6ed6e2958b7bc6360a6cba31aa3d1aaf169a133ef987c839de-1.svg}}

That may be less obvious for dictionaries. In dictionaries \textbf{the
order may matter} , so equality doesn\textquotesingle t mean they behave
exactly the same way:

\begin{verbatim}
#let a = (x: 1, y: 2)
#let b = (y: 2, x: 1)
#(a == b)
#(a.pairs() == b.pairs())
\end{verbatim}

\pandocbounded{\includesvg[keepaspectratio]{basics/scripting/typst-img/f7277d7cc170d7cc2ae1de5436b534fb113cda82d8e7829a0fc92e950b78238f-1.svg}}

\subsection{\texorpdfstring{\hyperref[check-key-is-in-dictionary]{Check
key is in
dictionary}}{Check key is in dictionary}}\label{check-key-is-in-dictionary}

Use the keyword \texttt{\ }{\texttt{\ in\ }}\texttt{\ } , like in
\texttt{\ }{\texttt{\ Python\ }}\texttt{\ } :

\begin{verbatim}
#let dict = (a: 1, b: 2)

#("a" in dict)
// gives the same as
#(dict.keys().contains("a"))
\end{verbatim}

\pandocbounded{\includesvg[keepaspectratio]{basics/scripting/typst-img/c4ae77418e54911af371f203d2bd3d5badb7269496bb8f07a2e3010e15f18922-1.svg}}

Note it works for lists too:

\begin{verbatim}
#("a" in ("b", "c", "a"))
#(("b", "c", "a").contains("a"))
\end{verbatim}

\pandocbounded{\includesvg[keepaspectratio]{basics/scripting/typst-img/0fc3ff7d44bbb5bcacd38e921f199699d2ea43ce0a142e79f67314d4f24386a7-1.svg}}

\section{\texorpdfstring{\hyperref[states--query]{States \&
Query}}{States \& Query}}\label{states--query}

This section is outdated. It may be still useful, but it is strongly
recommended to study new context system (using the reference).

Typst tries to be a \emph{pure language} as much as possible.

That means, a function can\textquotesingle t change anything outside of
it. That also means, if you call function, the result should be always
the same.

Unfortunately, our world (and therefore our documents)
isn\textquotesingle t pure. If you create a heading №2, you want the
next number to be three.

That section will guide you to using impure Typst. Don\textquotesingle t
overuse it, as this knowledge comes close to the Dark Arts of Typst!

\section{\texorpdfstring{\hyperref[states]{States}}{States}}\label{states}

This section is outdated. It may be still useful, but it is strongly
recommended to study new context system (using the reference).

Before we start something practical, it is important to understand
states in general.

Here is a good explanation of why do we \emph{need} them:
\href{https://typst.app/docs/reference/meta/state/}{Official Reference
about states} . It is highly recommended to read it first.

So instead of

\begin{verbatim}
#let x = 0
#let compute(expr) = {
  // eval evaluates string as Typst code
  // to calculate new x value
  x = eval(
    expr.replace("x", str(x))
  )
  [New value is #x.]
}

#compute("10") \
#compute("x + 3") \
#compute("x * 2") \
#compute("x - 5")
\end{verbatim}

\textbf{THIS DOES NOT COMPILE:} Variables from outside the function are
read-only and cannot be modified

Instead, you should write

\begin{verbatim}
#let s = state("x", 0)
#let compute(expr) = [
  // updates x current state with this function
  #s.update(x =>
    eval(expr.replace("x", str(x)))
  )
  // and displays it
  New value is #context s.get().
]

#compute("10") \
#compute("x + 3") \
#compute("x * 2") \
#compute("x - 5")

The computations will be made _in order_ they are _located_ in the document. So if you create computations first, but put them in the document later... See yourself:

#let more = [
  #compute("x * 2") \
  #compute("x - 5")
]

#compute("10") \
#compute("x + 3") \
#more
\end{verbatim}

\pandocbounded{\includesvg[keepaspectratio]{basics/states/typst-img/9a88397d1a9b5a44b1a3a218894595121bd4c5ec875df2b960638f2925060334-1.svg}}

\subsection{\texorpdfstring{\hyperref[context-magic]{Context
magic}}{Context magic}}\label{context-magic}

So what does this magic
\texttt{\ }{\texttt{\ context\ s.get()\ }}\texttt{\ } mean?

\begin{quote}
\href{https://typst.app/docs/reference/context/}{Context in Reference}
\end{quote}

In short, it specifies what part of your code (or markup) can
\emph{depend on states outside} . This context-expression is packed then
as one object, and it is evaluated on layout stage.

That means it is impossible to look from "normal" code at whatever is
inside the \texttt{\ }{\texttt{\ context\ }}\texttt{\ } . This is a
black box that would be known \emph{only after putting it into the
document} .

We will discuss \texttt{\ }{\texttt{\ context\ }}\texttt{\ } features
later.

\subsection{\texorpdfstring{\hyperref[operations-with-states]{Operations
with states}}{Operations with states}}\label{operations-with-states}

\subsubsection{\texorpdfstring{\hyperref[creating-new-state]{Creating
new state}}{Creating new state}}\label{creating-new-state}

\begin{verbatim}
#let x = state("state-id")
#let y = state("state-id", 2)

#x, #y

State is #context x.get() \ // the same as
#context [State is #y.get()] \ // the same as
#context {"State is" + str(y.get())}
\end{verbatim}

\pandocbounded{\includesvg[keepaspectratio]{basics/states/typst-img/4a52375bdeea2b7ca31dc51740563d01b3678f817dd6bc8c349d0714c2ac503f-1.svg}}

\subsubsection{\texorpdfstring{\hyperref[update]{Update}}{Update}}\label{update}

Updating is \emph{a content} that is an instruction. That instruction
tells compiler that in this place of document the state \emph{should be
updated} .

\begin{verbatim}
#let x = state("x", 0)
#context x.get() \
#let _ = x.update(3)
// nothing happens, we don't put `update` into the document flow
#context x.get()

#repr(x.update(3)) // this is how that content looks \

#context x.update(3)
#context x.get() // Finally!
\end{verbatim}

\pandocbounded{\includesvg[keepaspectratio]{basics/states/typst-img/3732a9c7bca8c4faedf9b024e09e647a65222c8244e9f3235a6057dfebc0a511-1.svg}}

Here we can see one of \emph{important
\texttt{\ }{\texttt{\ context\ }}\texttt{\ } traits} : it "sees" states
from outside, but can\textquotesingle t see how they change inside it:

\begin{verbatim}
#let x = state("x", 0)

#context {
  x.update(3)
  str(x.get())
}
\end{verbatim}

\pandocbounded{\includesvg[keepaspectratio]{basics/states/typst-img/78e500b80cb85e086a81302e2ce3dad88cb4304d4685b88e3f59111bc71f6748-1.svg}}

\subsubsection{\texorpdfstring{\hyperref[id-collision]{ID
collision}}{ID collision}}\label{id-collision}

\emph{TLDR; \textbf{Never allow colliding states.}}

States are described by their id-s, if they are the same, the code will
break.

So, if you write functions or loops that are used several times,
\emph{be careful} !

\begin{verbatim}
#let f(x) = {
  // return new state…
  // …but their id-s are the same!
  // so it will always be the same state!
  let y = state("x", 0)
  y.update(y => y + x)
  context y.get()
}

#let a = f(2)
#let b = f(3)

#a, #b \
#raw(repr(a) + "\n" + repr(b))
\end{verbatim}

\pandocbounded{\includesvg[keepaspectratio]{basics/states/typst-img/31a3e88747ed09ae6078bd3caf986f0e6ba744e055d0889d92bfa23941e7e451-1.svg}}

However, this \emph{may seem} okay:

\begin{verbatim}
// locations in code are different!
#let x = state("state-id")
#let y = state("state-id", 2)

#x, #y
\end{verbatim}

\pandocbounded{\includesvg[keepaspectratio]{basics/states/typst-img/1901e1449942d821c66f53bd6bc5fda10d63591aa45346fdf88bcbc3f2ab3425-1.svg}}

But in fact, it \emph{isn\textquotesingle t} :

\begin{verbatim}
#let x = state("state-id")
#let y = state("state-id", 2)

#context [#x.get(); #y.get()]

#x.update(3)

#context [#x.get(); #y.get()]
\end{verbatim}

\pandocbounded{\includesvg[keepaspectratio]{basics/states/typst-img/9185a298f9bcf8c519fa85481b9272e6ef3a00c117a0904d0509920a6abef8b2-1.svg}}

\section{\texorpdfstring{\hyperref[counters]{Counters}}{Counters}}\label{counters}

This section is outdated. It may be still useful, but it is strongly
recommended to study new context system (using the reference).

Counters are special states that \emph{count} elements of some type. As
with states, you can create your own with identifier strings.

\emph{Important:} to initiate counters of elements, you need to
\emph{set numbering for them} .

\subsection{\texorpdfstring{\hyperref[states-methods]{States
methods}}{States methods}}\label{states-methods}

Counters are states, so they can do all things states can do.

\begin{verbatim}
#set heading(numbering: "1.")

= Background
#counter(heading).update(3)
#counter(heading).update(n => n * 2)

== Analysis
Current heading number: #counter(heading).display().
\end{verbatim}

\pandocbounded{\includesvg[keepaspectratio]{basics/states/typst-img/c57c9907a5f238f0b5eee74f8c23c57a5e2d5b0c9cbf7ebd1befdfcbd33289df-1.svg}}

\begin{verbatim}
#let mine = counter("mycounter")
#mine.display()

#mine.step()
#mine.display()

#mine.update(c => c * 3)
#mine.display()
\end{verbatim}

\pandocbounded{\includesvg[keepaspectratio]{basics/states/typst-img/876103777c9564f0bb524f83a988a6d444c4e889baed31ee960548d90f3233e2-1.svg}}

\subsection{\texorpdfstring{\hyperref[displaying-counters]{Displaying
counters}}{Displaying counters}}\label{displaying-counters}

\begin{verbatim}
#set heading(numbering: "1.")

= Introduction
Some text here.

= Background
The current value is:
#counter(heading).display()

Or in roman numerals:
#counter(heading).display("I")
\end{verbatim}

\pandocbounded{\includesvg[keepaspectratio]{basics/states/typst-img/1ac65f4be42131b3cca1d7c56c6c60c3932a703e5e499c1c5cb874458028abea-1.svg}}

Counters also support displaying \emph{both current and final values}
out-of-box:

\begin{verbatim}
#set heading(numbering: "1.")

= Introduction
Some text here.

#counter(heading).display(both: true) \
#counter(heading).display("1 of 1", both: true) \
#counter(heading).display(
  (num, max) => [#num of #max],
   both: true
)

= Background
The current value is: #counter(heading).display()
\end{verbatim}

\pandocbounded{\includesvg[keepaspectratio]{basics/states/typst-img/af9d0da905bbb2215461b07b39653ef3890ff11a364afe018dae4ce4216f4961-1.svg}}

\subsection{\texorpdfstring{\hyperref[step]{Step}}{Step}}\label{step}

That\textquotesingle s quite easy, for counters you can increment value
using \texttt{\ }{\texttt{\ step\ }}\texttt{\ } . It works the same way
as \texttt{\ }{\texttt{\ update\ }}\texttt{\ } .

\begin{verbatim}
#set heading(numbering: "1.")

= Introduction
#counter(heading).step()

= Analysis
Let's skip 3.1.
#counter(heading).step(level: 2)

== Analysis
At #counter(heading).display().
\end{verbatim}

\pandocbounded{\includesvg[keepaspectratio]{basics/states/typst-img/12446a2258e9862d8df8b6b250ff14efbb9c35da165a2a04e8c4aa12c9b68cdf-1.svg}}

\subsection{\texorpdfstring{\hyperref[you-can-use-counters-in-your-functions]{You
can use counters in your
functions:}}{You can use counters in your functions:}}\label{you-can-use-counters-in-your-functions}

\begin{verbatim}
#let c = counter("theorem")
#let theorem(it) = block[
  #c.step()
  *Theorem #c.display():*
  #it
]

#theorem[$1 = 1$]
#theorem[$2 < 3$]
\end{verbatim}

\pandocbounded{\includesvg[keepaspectratio]{basics/states/typst-img/0f178f614e49a7400d646926705364a92ca3d4d888423b2693f071f83ce09e7d-1.svg}}

\section{\texorpdfstring{\hyperref[measure-layout]{Measure,
Layout}}{Measure, Layout}}\label{measure-layout}

This section is outdated. It may be still useful, but it is strongly
recommended to study new context system (using the reference).

\subsection{\texorpdfstring{\hyperref[style--measure]{Style \&
Measure}}{Style \& Measure}}\label{style--measure}

\begin{quote}
Style
\href{https://typst.app/docs/reference/foundations/style/}{documentation}
.
\end{quote}

\begin{quote}
Measure
\href{https://typst.app/docs/reference/layout/measure/}{documentation} .
\end{quote}

\texttt{\ }{\texttt{\ measure\ }}\texttt{\ } returns \emph{the element
size} . This command is extremely helpful when doing custom layout with
\texttt{\ }{\texttt{\ place\ }}\texttt{\ } .

However, there is a catch. Element size depends on styles, applied to
this element.

\begin{verbatim}
#let content = [Hello!]
#content
#set text(14pt)
#content
\end{verbatim}

\pandocbounded{\includesvg[keepaspectratio]{basics/typst-img/00a6cbbc650947c03f34564786b0645eee60396f288d26333c591ff9059cc369-1.svg}}

So if we will set the big text size for some part of our text, to
measure the element\textquotesingle s size, we have to know \emph{where
the element is located} . Without knowing it, we can\textquotesingle t
tell what styles should be applied.

So we need a scheme similar to
\texttt{\ }{\texttt{\ locate\ }}\texttt{\ } .

This is what \texttt{\ }{\texttt{\ styles\ }}\texttt{\ } function is
used for. It is \emph{a content} , which, when located in document,
calls a function inside on \emph{current styles} .

Now, when we got fixed \texttt{\ }{\texttt{\ styles\ }}\texttt{\ } , we
can get the element\textquotesingle s size using
\texttt{\ }{\texttt{\ measure\ }}\texttt{\ } :

\begin{verbatim}
#let thing(body) = style(styles => {
  let size = measure(body, styles)
  [Width of "#body" is #size.width]
})

#thing[Hey] \
#thing[Welcome]
\end{verbatim}

\pandocbounded{\includesvg[keepaspectratio]{basics/typst-img/5afe1855072b4ee8e343e5b5aa79affae5b17bc89738ffbe93dac245576cdd04-1.svg}}

\section{\texorpdfstring{\hyperref[layout]{Layout}}{Layout}}\label{layout}

Layout is similar to \texttt{\ }{\texttt{\ measure\ }}\texttt{\ } , but
it returns current scope \textbf{parent size} .

If you are putting elements in block, that will be
block\textquotesingle s size. If you are just putting right on the page,
that will be page\textquotesingle s size.

As parent\textquotesingle s size depends on it\textquotesingle s place
in document, it uses the similar scheme to
\texttt{\ }{\texttt{\ locate\ }}\texttt{\ } and
\texttt{\ }{\texttt{\ style\ }}\texttt{\ } :

\begin{verbatim}
#layout(size => {
  let half = 50% * size.width
  [Half a page is #half wide.]
})
\end{verbatim}

\pandocbounded{\includesvg[keepaspectratio]{basics/typst-img/c68a166f6e6b1b3229fd56478ae302dbeb39c882e229c69d4c6ebb6c9c528985-1.svg}}

It may be extremely useful to combine
\texttt{\ }{\texttt{\ layout\ }}\texttt{\ } with
\texttt{\ }{\texttt{\ measure\ }}\texttt{\ } , to get width of things
that depend on parent\textquotesingle s size:

\begin{verbatim}
#let text = lorem(30)
#layout(size => style(styles => [
  #let (height,) = measure(
    block(width: size.width, text),
    styles,
  )
  This text is #height high with
  the current page width: \
  #text
]))
\end{verbatim}

\pandocbounded{\includesvg[keepaspectratio]{basics/typst-img/93167dc0b22b02fe27aa92c6b03c2281665b4352624364a19c63f61a488aa75a-1.svg}}

\section{\texorpdfstring{\hyperref[query]{Query}}{Query}}\label{query}

This section is outdated. It may be still useful, but it is strongly
recommended to study new context system (using the reference).

\begin{quote}
Link \href{https://typst.app/docs/reference/meta/query/}{there}
\end{quote}

Query is a thing that allows you getting location by \emph{selector}
(this is the same thing we used in show rules).

That enables "time travel", getting information about document from its
parts and so on. \emph{That is a way to violate Typst\textquotesingle s
purity.}

It is currently one of the \emph{the darkest magics currently existing
in Typst} . It gives you great powers, but with great power comes great
responsibility.

\subsection{\texorpdfstring{\hyperref[time-travel]{Time
travel}}{Time travel}}\label{time-travel}

\begin{verbatim}
#let s = state("x", 0)
#let compute(expr) = [
  #s.update(x =>
    eval(expr.replace("x", str(x)))
  )
  New value is #s.display().
]

Value at `<here>` is
#context s.at(
  query(<here>)
    .first()
    .location()
)

#compute("10") \
#compute("x + 3") \
*Here.* <here> \
#compute("x * 2") \
#compute("x - 5")
\end{verbatim}

\pandocbounded{\includesvg[keepaspectratio]{basics/states/typst-img/130940aa5ae2ceb3364ef655c84cf8e7d2178210851b8fb20e6c0c3345c3ace7-1.svg}}

\subsection{\texorpdfstring{\hyperref[getting-nearest-chapter]{Getting
nearest
chapter}}{Getting nearest chapter}}\label{getting-nearest-chapter}

\begin{verbatim}
#set page(header: context {
  let elems = query(
    selector(heading).before(here()),
    here(),
  )
  let academy = smallcaps[
    Typst Academy
  ]
  if elems == () {
    align(right, academy)
  } else {
    let body = elems.last().body
    academy + h(1fr) + emph(body)
  }
})

= Introduction
#lorem(23)

= Background
#lorem(30)

= Analysis
#lorem(15)
\end{verbatim}

\pandocbounded{\includesvg[keepaspectratio]{basics/states/typst-img/b4d0562911dd308b0d9cbc36ad20ba6ed91fc3c3da5b6259eb6721f3a53a18e3-1.svg}}

\section{\texorpdfstring{\hyperref[metadata]{Metadata}}{Metadata}}\label{metadata}

Metadata is invisible content that can be extracted using query or other
content. This may be very useful with
\texttt{\ }{\texttt{\ typst\ query\ }}\texttt{\ } to pass values to
external tools.

\begin{verbatim}
// Put metadata somewhere.
#metadata("This is a note") <note>

// And find it from anywhere else.
#context {
  query(<note>).first().value
}
\end{verbatim}

\pandocbounded{\includesvg[keepaspectratio]{basics/states/typst-img/ef1c7d9faf74901f6c5266d48ae006167003a22754408a70ae9f9d1088b5fe24-1.svg}}

\section{\texorpdfstring{\hyperref[math-1]{Math}}{Math}}\label{math-1}

Math is a special environment that has special features related to...
math.

\subsection{\texorpdfstring{\hyperref[syntax]{Syntax}}{Syntax}}\label{syntax}

To start math environment, \texttt{\ }{\texttt{\ \$\ }}\texttt{\ } . The
spacing around \texttt{\ }{\texttt{\ \$\ }}\texttt{\ } will make it
either \emph{inline} math (smaller, used in text) or \emph{display} math
(used on math equations on their own).

\begin{verbatim}
// This is inline math
Let $a$, $b$, and $c$ be the side
lengths of right-angled triangle.
Then, we know that:

// This is display math
$ a^2 + b^2 = c^2 $

Prove by induction:

// You can use new lines as spacing too!
$
sum_(k=1)^n k = (n(n+1)) / 2
$
\end{verbatim}

\pandocbounded{\includesvg[keepaspectratio]{basics/math/typst-img/068db3a521a38c3acede771ebb6342807cca4fd98baf5b2b508184a6854ea8ff-1.svg}}

\subsection{\texorpdfstring{\hyperref[mathequation]{Math.equation}}{Math.equation}}\label{mathequation}

The element that math is displayed in is called
\texttt{\ }{\texttt{\ math.equation\ }}\texttt{\ } . You can use it for
set/show rules:

\begin{verbatim}
#show math.equation: set text(red)

$
integral_0^oo (f(t) + g(t))/2
$
\end{verbatim}

\pandocbounded{\includesvg[keepaspectratio]{basics/math/typst-img/94e0532dd7224d08e966cb82834283efd8889d7f117b04116e721a788bfcc16c-1.svg}}

Any symbol/command that is available in math, \emph{is also available}
in code mode using \texttt{\ }{\texttt{\ math.command\ }}\texttt{\ } :

\begin{verbatim}
#math.integral, #math.underbrace([a + b], [c])
\end{verbatim}

\pandocbounded{\includesvg[keepaspectratio]{basics/math/typst-img/b4ca12d7f34ed342f3cb3fba2ee1f5b58faa8fceecb74671baacd9166fcbb5aa-1.svg}}

\subsection{\texorpdfstring{\hyperref[letters-and-commands]{Letters and
commands}}{Letters and commands}}\label{letters-and-commands}

Typst aims to have as simple and effective syntax for math as possible.
That means no special symbols, just using commands.

To make it short, Typst uses several simple rules:

\begin{itemize}
\item
  All single-letter words \emph{turn into variables} . That includes any
  \emph{unicode symbols} too!
\item
  All multi-letter words \emph{turn into commands} . They may be
  built-in commands (available with math.something outside of math
  environment). Or they \textbf{may be user-defined variables/functions}
  . If the command \textbf{isn\textquotesingle t defined} , there will
  be \textbf{compilation error} .

  If you use kebab-case or snake\_case for variables you want to use in
  math, you will have to refer to them as \#snake-case-variable.
\item
  To write simple text, use quotes:

\begin{verbatim}
$a "equals to" 2$
\end{verbatim}

  \pandocbounded{\includesvg[keepaspectratio]{basics/math/typst-img/811f30ede68d08bec254f184c1be319958c3e11f9f9d58c40b2f460bba037e3d-1.svg}}

  Spacing matters there!

\begin{verbatim}
$a "is" 2$, $a"is"2$
\end{verbatim}

  \pandocbounded{\includesvg[keepaspectratio]{basics/math/typst-img/9cc2d263c76646c623e1e6b73756e1fe1e2c56d7fe0324ee945652107e6456ba-1.svg}}
\item
  You can turn it into multi-letter variables using
  \texttt{\ }{\texttt{\ italic\ }}\texttt{\ } :

\begin{verbatim}
$(italic("mass") v^2)/2$
\end{verbatim}

  \pandocbounded{\includesvg[keepaspectratio]{basics/math/typst-img/141d8a3b9beb3559387411170f7378078c80cb2ff80d8d5f5345c3231f55df9c-1.svg}}
\end{itemize}

Commands see
\href{https://typst.app/docs/reference/math/\#definitions}{there} (go to
the links to see the commands).

All symbols see
\href{https://typst.app/docs/reference/symbols/sym/}{there} .

\subsection{\texorpdfstring{\hyperref[multiline-equations]{Multiline
equations}}{Multiline equations}}\label{multiline-equations}

To create multiline \emph{display equation} , use the same symbol as in
markup mode: \texttt{\ }{\texttt{\ \textbackslash{}\ }}\texttt{\ } :

\begin{verbatim}
$
a = b\
a = c
$
\end{verbatim}

\pandocbounded{\includesvg[keepaspectratio]{basics/math/typst-img/2f16d9e64e38ff22ca27a09b0d8eaef1b020e4eccd7d2ce1380e10a0efcea163-1.svg}}

\subsection{\texorpdfstring{\hyperref[escaping]{Escaping}}{Escaping}}\label{escaping}

Any symbol that is used may be escaped with
\texttt{\ }{\texttt{\ \textbackslash{}\ }}\texttt{\ } , like in markup
mode. For example, you can disable fraction:

\begin{verbatim}
$
a  / b \
a \/ b
$
\end{verbatim}

\pandocbounded{\includesvg[keepaspectratio]{basics/math/typst-img/e7931e55a2772ee992446af8506d8d25b96167e3bb71d5c63ed8ca156530f2d9-1.svg}}

The same way it works with any other syntax.

\subsection{\texorpdfstring{\hyperref[wrapping-inline-math]{Wrapping
inline math}}{Wrapping inline math}}\label{wrapping-inline-math}

Sometimes, when you write large math, it may be too close to text
(especially for some long letter tails).

\begin{verbatim}
#lorem(17) $display(1)/display(1+x^n)$ #lorem(20)
\end{verbatim}

\pandocbounded{\includesvg[keepaspectratio]{basics/math/typst-img/a9cce2b851a01939a0abfc02e8cd994d20c465d2800cf64c5c6051ead5bc4e9a-1.svg}}

You may easily increase the distance it by wrapping into box:

\begin{verbatim}
#lorem(17) #box($display(1)/display(1+x^n)$, inset: 0.2em) #lorem(20)
\end{verbatim}

\pandocbounded{\includesvg[keepaspectratio]{basics/math/typst-img/ee9fc5a3ec529a9f3e811a70724c1585c294d82454c22ee9343235556f572792-1.svg}}

\section{\texorpdfstring{\hyperref[symbols]{Symbols}}{Symbols}}\label{symbols}

Multiletter words in math refer either to local variables, functions,
text operators, spacing or \emph{special symbols} . The latter are very
important for advanced math.

\begin{verbatim}
$
forall v, w in V, alpha in KK: alpha dot (v + w) = alpha v + alpha w
$
\end{verbatim}

\pandocbounded{\includesvg[keepaspectratio]{basics/math/typst-img/60a6e3e08582c87ec082b6714a45a90a914dd1299f788e2bb21b0cc5adc80e6a-1.svg}}

You can write the same with unicode:

\begin{verbatim}
$
∀ v, w ∈ V, α ∈ 𝕂: α ⋅ (v + w) = α v + α w
$
\end{verbatim}

\pandocbounded{\includesvg[keepaspectratio]{basics/math/typst-img/d37776c21d5c4d692e4ebbe7e5ce7e7cdf5e2c0777a88a47abe0c0c5992cf41a-1.svg}}

\subsection{\texorpdfstring{\hyperref[symbols-naming]{Symbols
naming}}{Symbols naming}}\label{symbols-naming}

\begin{quote}
See all available symbols list
\href{https://typst.app/docs/reference/symbols/sym/}{there} .
\end{quote}

\subsubsection{\texorpdfstring{\hyperref[general-idea]{General
idea}}{General idea}}\label{general-idea}

Typst wants to define some "basic" symbols with small easy-to-remember
words, and build complex ones using combinations. For example,

\begin{verbatim}
$
// cont — contour
integral, integral.cont, integral.double, integral.square, sum.integral\

// lt — less than, gt — greater than
lt, lt.circle, lt.eq, lt.not, lt.eq.not, lt.tri, lt.tri.eq, lt.tri.eq.not, gt, lt.gt.eq, lt.gt.not
$
\end{verbatim}

\pandocbounded{\includesvg[keepaspectratio]{basics/math/typst-img/a0ee196d2bf305ca6c2d812008f9955e5ae526de0b0ac0b83ca016a66bdc00f1-1.svg}}

I highly recommend using WebApp/Typst LSP when writing math with lots of
complex symbols. That helps you to quickly choose the right symbol
within all combinations.

Sometimes the names are not obvious, for example, sometimes it is used
prefix \texttt{\ }{\texttt{\ n-\ }}\texttt{\ } instead of
\texttt{\ }{\texttt{\ not\ }}\texttt{\ } :

\begin{verbatim}
$
gt.nequiv, gt.napprox, gt.ntilde, gt.tilde.not
$
\end{verbatim}

\pandocbounded{\includesvg[keepaspectratio]{basics/math/typst-img/e4d0ef024efaf9f4334ebf04a2ac4e015fc5ec76617be8b6d7aad2f4429e3317-1.svg}}

\subsubsection{\texorpdfstring{\hyperref[common-modifiers]{Common
modifiers}}{Common modifiers}}\label{common-modifiers}

\begin{itemize}
\item
  \texttt{\ }{\texttt{\ .b,\ .t,\ .l,\ .r\ }}\texttt{\ } : bottom, top,
  left, right. Change direction of symbol.

\begin{verbatim}
$arrow.b, triangle.r, angle.l$
\end{verbatim}

  \pandocbounded{\includesvg[keepaspectratio]{basics/math/typst-img/8ab0fa590b7a39023b1467e7a376a4810f997f720dd5d221ad83d7e741943b55-1.svg}}
\end{itemize}


\section{Examples Book LaTeX/book/packages.tex}
\title{sitandr.github.io/typst-examples-book/book/packages}

\section{\texorpdfstring{\hyperref[packages]{Packages}}{Packages}}\label{packages}

Once the \href{https://typst.app/universe}{Typst Universe} was launched,
this chapter has become almost redundant. The Universe is actually a
very cool place to look for packages.

However, there are still some cool examples of interesting package
usage.

\subsection{\texorpdfstring{\hyperref[general]{General}}{General}}\label{general}

Typst has packages, but, unlike LaTeX, you need to remember:

\begin{itemize}
\tightlist
\item
  You need them only for some specialized tasks, basic formatting
  \emph{can be totally done without them} .
\item
  Packages are much lighter and much easier "installed" than LaTeX ones.
\item
  Packages are just plain Typst files (and sometimes plugins), so you
  can easily write your own!
\end{itemize}

To use mighty package, just write, like this:

\begin{verbatim}
#import "@preview/cetz:0.1.2": canvas, plot

#canvas(length: 1cm, {
  plot.plot(size: (8, 6),
    x-tick-step: none,
    x-ticks: ((-calc.pi, $-pi$), (0, $0$), (calc.pi, $pi$)),
    y-tick-step: 1,
    {
      plot.add(
        style: plot.palette.blue,
        domain: (-calc.pi, calc.pi), x => calc.sin(x * 1rad))
      plot.add(
        hypograph: true,
        style: plot.palette.blue,
        domain: (-calc.pi, calc.pi), x => calc.cos(x * 1rad))
      plot.add(
        hypograph: true,
        style: plot.palette.blue,
        domain: (-calc.pi, calc.pi), x => calc.cos((x + calc.pi) * 1rad))
    })
})
\end{verbatim}

\pandocbounded{\includesvg[keepaspectratio]{typst-img/29d7015ed96122fa3fb663929c1ac58d25340995423c82456ab8815811373979-1.svg}}

\subsection{\texorpdfstring{\hyperref[contributing]{Contributing}}{Contributing}}\label{contributing}

If you are author of a package or just want to make a fair overview,
feel free to make issues/PR-s!




\section{Combined Examples Book LaTeX/book/basics.tex}
\section{Combined Examples Book LaTeX/book/basics/tutorial.tex}
\section{Examples Book LaTeX/book/basics/tutorial/markup.tex}
\title{sitandr.github.io/typst-examples-book/book/basics/tutorial/markup}

\section{\texorpdfstring{\hyperref[markup-language]{Markup
language}}{Markup language}}\label{markup-language}

\subsection{\texorpdfstring{\hyperref[starting]{Starting}}{Starting}}\label{starting}

\begin{verbatim}
Starting typing in Typst is easy.
You don't need packages or other weird things for most of things.

Blank line will move text to a new paragraph.

Btw, you can use any language and unicode symbols
without any problems as long as the font supports it: ßçœ̃ɛ̃ø∀αβёыა😆…
\end{verbatim}

\pandocbounded{\includesvg[keepaspectratio]{typst-img/ee9f64251c99c7aeaaf6fa1d5bc7e907c2d51a34aa38126544d515ca197ca2a8-1.svg}}

\subsection{\texorpdfstring{\hyperref[markup]{Markup}}{Markup}}\label{markup}

\begin{verbatim}
= Markup

This was a heading. Number of `=` in front of name corresponds to heading level.

== Second-level heading

Okay, let's move to _emphasis_ and *bold* text.

Markup syntax is generally similar to
`AsciiDoc` (this was `raw` for monospace text!)
\end{verbatim}

\pandocbounded{\includesvg[keepaspectratio]{typst-img/fa8b95f9b15083387a29c11d17efca9873b8e778643b1b5079aa137891d01c8d-1.svg}}

\subsection{\texorpdfstring{\hyperref[new-lines--escaping]{New lines \&
Escaping}}{New lines \& Escaping}}\label{new-lines--escaping}

\begin{verbatim}
You can break \
line anywhere you \
want using "\\" symbol.

Also you can use that symbol to
escape \_all the symbols you want\_,
if you don't want it to be interpreted as markup
or other special symbols.
\end{verbatim}

\pandocbounded{\includesvg[keepaspectratio]{typst-img/4dabdee2a61e7d10773d51772dba3665271a09d4d5df4a8f66dd80589f0bcd7a-1.svg}}

\subsection{\texorpdfstring{\hyperref[comments--codeblocks]{Comments \&
codeblocks}}{Comments \& codeblocks}}\label{comments--codeblocks}

\begin{verbatim}
You can write comments with `//` and `/* comment */`:
// Like this
/* Or even like
this */

```typ
Just in case you didn't read source,
this is how it is written:

// Like this
/* Or even like
this */

By the way, I'm writing it all in a _fenced code block_ with *syntax highlighting*!
```
\end{verbatim}

\pandocbounded{\includesvg[keepaspectratio]{typst-img/a481d12b3ed0bbe2d9db6cc4b4a1237cba9936de83333254dfce8702832db125-1.svg}}

\subsection{\texorpdfstring{\hyperref[smart-quotes]{Smart
quotes}}{Smart quotes}}\label{smart-quotes}

\begin{verbatim}
== What else?

There are not much things in basic "markup" syntax,
but we will see much more interesting things very soon!
I hope you noticed auto-matched "smart quotes" there.
\end{verbatim}

\pandocbounded{\includesvg[keepaspectratio]{typst-img/89114a6e9af45c2eb9db2ef44d0e5ba41e31bf816e72803bd1a9a02120e69fc3-1.svg}}

\subsection{\texorpdfstring{\hyperref[lists]{Lists}}{Lists}}\label{lists}

\begin{verbatim}
- Writing lists in a simple way is great.
- Nothing complex, start your points with `-`
  and this will become a list.
  - Indented lists are created via indentation.

+ Numbered lists start with `+` instead of `-`.
+ There is no alternative markup syntax for lists
+ So just remember `-` and `+`, all other symbols
  wouldn't work in an unintended way.
  + That is a general property of Typst's markup.
  + Unlike Markdown, there is only one way
    to write something with it.
\end{verbatim}

\pandocbounded{\includesvg[keepaspectratio]{typst-img/ad4e424e067a4362e9f145c0c4ba4b7c1b65e17e7d0e7631b6836841607ef85e-1.svg}}

\textbf{Notice:}

\begin{verbatim}
Typst numbered lists differ from markdown-like syntax for lists. If you write them by hand, numbering is preserved:

1. Apple
1. Orange
1. Peach
\end{verbatim}

\pandocbounded{\includesvg[keepaspectratio]{typst-img/477695c86becc136dceb144e90c0acd2b75faa2a49743f8673d09974b71da324-1.svg}}

\subsection{\texorpdfstring{\hyperref[math]{Math}}{Math}}\label{math}

\begin{verbatim}
I will just mention math ($a + b/c = sum_i x^i$)
is possible and quite pretty there:

$
7.32 beta +
  sum_(i=0)^nabla
    (Q_i (a_i - epsilon)) / 2
$

To learn more about math, see corresponding chapter.
\end{verbatim}

\pandocbounded{\includesvg[keepaspectratio]{typst-img/12cc318c8438cd8e91706013bbd53fee5ee004620a63348cfe2d7dcc3b8a19d4-1.svg}}


\section{Examples Book LaTeX/book/basics/tutorial/templates.tex}
\title{sitandr.github.io/typst-examples-book/book/basics/tutorial/templates}

\section{\texorpdfstring{\hyperref[templates]{Templates}}{Templates}}\label{templates}

\subsection{\texorpdfstring{\hyperref[templates-1]{Templates}}{Templates}}\label{templates-1}

If you want to reuse styling in other files, you can use the
\emph{template} idiom. Because \texttt{\ }{\texttt{\ set\ }}\texttt{\ }
and \texttt{\ }{\texttt{\ show\ }}\texttt{\ } rules are only active in
their current scope, they will not affect content in a file you imported
your file into. But functions can circumvent this in a predictable way:

\begin{verbatim}
// define a function that:
// - takes content
// - applies styling to it
// - returns the styled content
#let apply-template(body) = [
  #show heading.where(level: 1): emph
  #set heading(numbering: "1.1")
  // ...
  #body
]
\end{verbatim}

This is equivalent to:

\begin{verbatim}
// we can reduce the number of hashes needed here by using scripting mode
// same as above but we exchanged `[...]` for `{...}` to switch from markup
// into scripting mode
#let apply-template(body) = {
  show heading.where(level: 1): emph
  set heading(numbering: "1.1")
  // ...
  body
}
\end{verbatim}

Then in your main file:

\begin{verbatim}
#import "template.typ": apply-template
#show: apply-template
\end{verbatim}

\emph{This will apply a "template" function to the rest of your
document!}

\subsubsection{\texorpdfstring{\hyperref[passing-arguments]{Passing
arguments}}{Passing arguments}}\label{passing-arguments}

\begin{verbatim}
// add optional named arguments
#let apply-template(body, name: "My document") = {
  show heading.where(level: 1): emph
  set heading(numbering: "1.1")

  align(center, text(name, size: 2em))

  body
}
\end{verbatim}

Then, in template file:

\begin{verbatim}
#import "template.typ": apply-template

// `func.with(..)` applies the arguments to the function and returns the new
// function with those defaults applied
#show: apply-template.with(name: "Report")

// it is functionally the same as this
#let new-template(..args) = apply-template(name: "Report", ..args)
#show: new-template
\end{verbatim}

Writing templates is fairly easy if you understand
\href{../scripting/index.html}{scripting} .

See more information about writing templates in
\href{https://typst.app/docs/tutorial/making-a-template/}{Official
Tutorial} .

There is no official repository for templates yet, but there are a
plenty community ones in
\href{https://github.com/qjcg/awesome-typst?ysclid=lj8pur1am7431908794\#general}{awesome-typst}
.


\section{Examples Book LaTeX/book/basics/tutorial/index.tex}
\title{sitandr.github.io/typst-examples-book/book/basics/tutorial/index}

\section{\texorpdfstring{\hyperref[tutorial-by-examples]{Tutorial by
Examples}}{Tutorial by Examples}}\label{tutorial-by-examples}

The first section of Typst Basics is very similar to
\href{https://typst.app/docs/tutorial/}{Official Tutorial} , with more
specialized examples and less words. It is \emph{highly recommended to
read the official tutorial anyway} .


\section{Examples Book LaTeX/book/basics/tutorial/functions.tex}
\title{sitandr.github.io/typst-examples-book/book/basics/tutorial/functions}

\section{\texorpdfstring{\hyperref[functions]{Functions}}{Functions}}\label{functions}

\subsection{\texorpdfstring{\hyperref[functions-1]{Functions}}{Functions}}\label{functions-1}

\begin{verbatim}
Okay, let's now move to more complex things.

First of all, there are *lots of magic* in Typst.
And it major part of it is called "scripting".

To go to scripting mode, type `#` and *some function name*
after that. We will start with _something dull_:

#lorem(50)

_That *function* just generated 50 "Lorem Ipsum" words!_
\end{verbatim}

\pandocbounded{\includesvg[keepaspectratio]{typst-img/036fce36d10e06e8e41be8e77d7d5672f5dfc82c57e7c3ba9b8060d0822ca115-1.svg}}

\subsection{\texorpdfstring{\hyperref[more-functions]{More
functions}}{More functions}}\label{more-functions}

\begin{verbatim}
#underline[functions can do everything!]

#text(orange)[L]ike #text(size: 0.8em)[Really] #sub[E]verything!

#figure(
  caption: [
    This is a screenshot from one of first theses written in Typst. \
    _All these things are written with #text(blue)[custom functions] too._
  ],
  image("../boxes.png", width: 80%)
)

In fact, you can #strong[forget] about markup
and #emph[just write] functions everywhere!

#list[
  All that markup is just a #emph[syntax sugar] over functions!
]
\end{verbatim}

\pandocbounded{\includesvg[keepaspectratio]{typst-img/455e15e83c25259f932178d68517cc012432cb17d072e60c659169470fe191ce-1.svg}}

\subsection{\texorpdfstring{\hyperref[how-to-call-functions]{How to call
functions}}{How to call functions}}\label{how-to-call-functions}

\begin{verbatim}
First, start with `#`. Then write the name.
Finally, write some parentheses and maybe something inside.

You can navigate lots of built-in functions
in #link("https://typst.app/docs/reference/")[Official Reference].

#quote(block: true, attribution: "Typst Examples Book")[
  That's right, links, quotes and lots of
  other document elements are created with functions.
]
\end{verbatim}

\pandocbounded{\includesvg[keepaspectratio]{typst-img/4c63fde73bb1ad0afe1332ab68c5b540ec786c6352a76860f4398fec32034cf0-1.svg}}

\subsection{\texorpdfstring{\hyperref[function-arguments]{Function
arguments}}{Function arguments}}\label{function-arguments}

\begin{verbatim}
There are _two types_ of function arguments:

+ *Positional.* Like `50` in `lorem(50)`.
  Just write them in parentheses and it will be okay.
  If you have many, use commas.
+ *Named.* Like in `#quote(attribution: "Whoever")`.
  Write the value after a name and a colon.

If argument is named, it has some _default value_.
To find out what it is, see
#link("https://typst.app/docs/reference/")[Official Typst Reference].
\end{verbatim}

\pandocbounded{\includesvg[keepaspectratio]{typst-img/d66fb474260490595a207f06c687efcc85808701c39c2a6e8b686bc22ffde279-1.svg}}

\subsection{\texorpdfstring{\hyperref[content]{Content}}{Content}}\label{content}

\begin{verbatim}
The most "universal" type in Typst language is *content*.
Everything you write in the document becomes content.

#[
  But you can explicitly create it with
  _scripting mode_ and *square brackets*.

  There, in square brackets, you can use any markup
  functions or whatever you want.
]
\end{verbatim}

\pandocbounded{\includesvg[keepaspectratio]{typst-img/faf9d7cddd55e68f84d212013a52a724c2ad763f18d83221a99bbd380410d7d1-1.svg}}

\subsection{\texorpdfstring{\hyperref[markup-and-code-modes]{Markup and
code modes}}{Markup and code modes}}\label{markup-and-code-modes}

\begin{verbatim}
When you use `#`, you are "switching" to code mode.
When you use `[]`, you turn back:

// +-- going from markup (the default mode) to scripting for that function
// |                 +-- scripting mode: calling `text`, the last argument is markup
// |     first arg   |
// v     vvvvvvvvv   vvvv
   #rect(width: 5cm, text(red)[hello *world*])
//  ^^^^                       ^^^^^^^^^^^^^ just a markup argument for `text`
//  |
//  +-- calling `rect` in scripting mode, with two arguments: width and other content
\end{verbatim}

\pandocbounded{\includesvg[keepaspectratio]{typst-img/0cabe3da1eb49f805535fb1d7e34a0d6eb1a6c49227b0be98634c6965e892185-1.svg}}

\subsection{\texorpdfstring{\hyperref[passing-content-into-functions]{Passing
content into
functions}}{Passing content into functions}}\label{passing-content-into-functions}

\begin{verbatim}
So what are these square brackets after functions?

If you *write content right after
function, it will be passed as positional argument there*.

#quote(block: true)[
  So #text(red)[_that_] allows me to write
  _literally anything in things
  I pass to #underline[functions]!_
]
\end{verbatim}

\pandocbounded{\includesvg[keepaspectratio]{typst-img/686d2b2a361a60244452ce53bd37ebef0699e92cf962c477bfb62bafdc0f7241-1.svg}}

\subsection{\texorpdfstring{\hyperref[passing-content-part-ii]{Passing
content, part
II}}{Passing content, part II}}\label{passing-content-part-ii}

\begin{verbatim}
So, just to make it clear, when I write

```typ
- #text(red)[red text]
- #text([red text], red)
- #text("red text", red)
//      ^        ^
// Quotes there mean a plain string, not a content!
// This is just text.
```

It all will result in a #text([red text], red).
\end{verbatim}

\pandocbounded{\includesvg[keepaspectratio]{typst-img/4686939b6d0932f1ebebac4111d8f02919dbc16446def7855c521d8dbf293689-1.svg}}


\section{Examples Book LaTeX/book/basics/tutorial/basic_styling.tex}
\title{sitandr.github.io/typst-examples-book/book/basics/tutorial/basic_styling}

\section{\texorpdfstring{\hyperref[basic-styling]{Basic
styling}}{Basic styling}}\label{basic-styling}

\subsection{\texorpdfstring{\hyperref[set-rule]{\texttt{\ }{\texttt{\ Set\ }}\texttt{\ }
rule}}{  Set   rule}}\label{set-rule}

\begin{verbatim}
#set page(width: 15cm, margin: (left: 4cm, right: 4cm))

That was great, but using functions everywhere, especially
with many arguments every time is awfully cumbersome.

That's why Typst has _rules_. No, not for you, for the document.

#set par(justify: true)

And the first rule we will consider there is `set` rule.
As you see, I've just used it on `par` (which is short from paragraph)
and now all paragraphs became _justified_.

It will apply to all paragraphs after the rule,
but will work only in it's _scope_ (we will discuss them later).

#par(justify: false)[
  Of course, you can override a `set` rule.
  This rule just sets the _default value_
  of an argument of an element.
]

By the way, at first line of this snippet
I've reduced page size to make justifying more visible,
also increasing margins to add blank space on left and right.
\end{verbatim}

\pandocbounded{\includesvg[keepaspectratio]{typst-img/cee42a8b1274afa36891438d4b1611eb55b2cd8bb4546df47128a7d3eb66653b-1.svg}}

\subsection{\texorpdfstring{\hyperref[a-bit-about-length-units]{A bit
about length
units}}{A bit about length units}}\label{a-bit-about-length-units}

\begin{verbatim}
Before we continue with rules, we should talk about length. There are several absolute length units in Typst:

#set rect(height: 1em)

#table(
  columns: 2,
  [Points], rect(width: 72pt),
  [Millimeters], rect(width: 25.4mm),
  [Centimeters], rect(width: 2.54cm),
  [Inches], rect(width: 1in),
  [Relative to font size], rect(width: 6.5em)
)

`1 em` = current font size. \
It is a very convenient unit,
so we are going to use it a lot
\end{verbatim}

\pandocbounded{\includesvg[keepaspectratio]{typst-img/5f8abc94a3d9df0e16f78c258e7f487d5698b4c96491300b3a48ad8e685534bc-1.svg}}

\subsection{\texorpdfstring{\hyperref[setting-something-else]{Setting
something else}}{Setting something else}}\label{setting-something-else}

Of course, you can use \texttt{\ }{\texttt{\ set\ }}\texttt{\ } rule
with all built-in functions and all their named arguments to make some
argument "default".

For example, let\textquotesingle s make all quotes in this snippet
authored by the book:

\begin{verbatim}
#set quote(block: true, attribution: [Typst Examples Book])

#quote[
  Typst is great!
]

#quote[
  The problem with quotes on the internet is
  that it is hard to verify their authenticity.
]
\end{verbatim}

\pandocbounded{\includesvg[keepaspectratio]{typst-img/c34c25cad05b7c20b6e0f146002886a1de65b61f48666cfec3d3494bd694a641-1.svg}}

\subsection{\texorpdfstring{\hyperref[opinionated-defaults]{Opinionated
defaults}}{Opinionated defaults}}\label{opinionated-defaults}

That allows you to set Typst default styling as you want it:

\begin{verbatim}
#set par(justify: true)
#set list(indent: 1em)
#set enum(indent: 1em)
#set page(numbering: "1")

- List item
- List item

+ Enum item
+ Enum item
\end{verbatim}

\pandocbounded{\includesvg[keepaspectratio]{typst-img/773d68bc55eb89f119ad07b882eae5fd31868d8a1bb3d4963573ec80fb1c7466-1.svg}}

Don\textquotesingle t complain about bad defaults!
\texttt{\ }{\texttt{\ Set\ }}\texttt{\ } your own.

\subsection{\texorpdfstring{\hyperref[numbering]{Numbering}}{Numbering}}\label{numbering}

\begin{verbatim}
= Numbering

Some of elements have a property called "numbering".
They accept so-called "numbering patterns" and
are very useful with set rules. Let's see what I mean.

#set heading(numbering: "I.1:")

= This is first level
= Another first
== Second
== Another second
=== Now third
== And second again
= Now returning to first
= These are actual romanian numerals
\end{verbatim}

\pandocbounded{\includesvg[keepaspectratio]{typst-img/39fb958032888b1e41da849152fed716b6f590eed3ea975b051ab786fac4ce5c-1.svg}}

Of course, there are lots of other cool properties that can be
\emph{set} , so feel free to dive into
\href{https://typst.app/docs/reference/}{Official Reference} and explore
them!

And now we are moving into something much more interesting\ldots{}


\section{Examples Book LaTeX/book/basics/tutorial/advanced_styling.tex}
\title{sitandr.github.io/typst-examples-book/book/basics/tutorial/advanced_styling}

\section{\texorpdfstring{\hyperref[advanced-styling]{Advanced
styling}}{Advanced styling}}\label{advanced-styling}

\subsection{\texorpdfstring{\hyperref[the-show-rule]{The
\texttt{\ }{\texttt{\ show\ }}\texttt{\ }
rule}}{The   show   rule}}\label{the-show-rule}

\begin{verbatim}
Advanced styling comes with another rule. The _`show` rule_.

Now please compare the source code and the output.

#show "Be careful": strong[Play]

This is a very powerful thing, sometimes even too powerful.
Be careful with it.

#show "it is holding me hostage": text(green)[I'm fine]

Wait, what? I told you "Be careful!", not "Play!".

Help, it is holding me hostage.
\end{verbatim}

\pandocbounded{\includesvg[keepaspectratio]{typst-img/8a9ac38769d4ac7b42a2755047d0cd5a6404ad26e9e7f5b72b6984fa67abadf9-1.svg}}

\subsection{\texorpdfstring{\hyperref[now-a-bit-more-serious]{Now a bit
more serious}}{Now a bit more serious}}\label{now-a-bit-more-serious}

\begin{verbatim}
Show rule is a powerful thing that takes a _selector_
and what to apply to it. After that it will apply to
all elements it can find.

It may be extremely useful like that:

#show emph: set text(blue)

Now if I want to _emphasize_ something,
it will be both _emphasized_ and _blue_.
Isn't that cool?
\end{verbatim}

\pandocbounded{\includesvg[keepaspectratio]{typst-img/657acaf5c4ca684408bbc6fe0dec4c74b9fa58d24805ec975be1382aa7bf959c-1.svg}}

\subsection{\texorpdfstring{\hyperref[about-syntax]{About
syntax}}{About syntax}}\label{about-syntax}

\begin{verbatim}
Sometimes show rules may be confusing. They may seem very diverse, but in fact they all are quite the same! So

// actually, this is the same as
// redify = text.with(red)
// `with` creates a new function with this argument already set
#let redify(string) = text(red, string)

// and this is the same as
// framify = rect.with(stroke: orange)
#let framify(object) = rect(object, stroke: orange)

// set default color of text blue for all following text
#show: set text(blue)

Blue text.

// wrap everything into a frame
#show: framify

Framed text.

// it's the same, just creating new function that calls framify
#show: a => framify(a)

Double-framed.

// apply function to `the`
#show "the": redify
// set text color for all the headings
#show heading: set text(purple)

= Conclusion

All these rules do basically the same!
\end{verbatim}

\pandocbounded{\includesvg[keepaspectratio]{typst-img/2dfcde68345d3fa276b99a1f308343118c6eeae09fd106389a8fc488d7244ebb-1.svg}}

\subsection{\texorpdfstring{\hyperref[blocks]{Blocks}}{Blocks}}\label{blocks}

One of the most important usages is that you can set up all spacing
using blocks. Like every element with text contains text that can be set
up, every \emph{block element} contains blocks:

\begin{verbatim}
Text before
= Heading
Text after

#show heading: set block(spacing: 0.5em)

Text before
= Heading
Text after
\end{verbatim}

\pandocbounded{\includesvg[keepaspectratio]{typst-img/7891207932d0918c88b5804b3a7ee051ce5dda93081f8999eb0f7ebaee48400a-1.svg}}

\subsection{\texorpdfstring{\hyperref[selector]{Selector}}{Selector}}\label{selector}

\begin{verbatim}
So show rule can accept _selectors_.

There are lots of different selector types,
for example

- element functions
- strings
- regular expressions
- field filters

Let's see example of the latter:

#show heading.where(level: 1): set align(center)

= Title
== Small title

Of course, you can set align by hand,
no need to use show rules
(but they are very handy!):

#align(center)[== Centered small title]
\end{verbatim}

\pandocbounded{\includesvg[keepaspectratio]{typst-img/f41f337dd75b55211dd8d16e2682132c1ffb1ef19f774ba6cafc94cae090ec75-1.svg}}

\subsection{\texorpdfstring{\hyperref[custom-formatting]{Custom
formatting}}{Custom formatting}}\label{custom-formatting}

\begin{verbatim}
Let's try now writing custom functions.
It is very easy, see yourself:

// "it" is a heading, we take it and output things in braces
#show heading: it => {
  // center it
  set align(center)
  // set size and weight
  set text(12pt, weight: "regular")
  // see more about blocks and boxes
  // in corresponding chapter
  block(smallcaps(it.body))
}

= Smallcaps heading
\end{verbatim}

\pandocbounded{\includesvg[keepaspectratio]{typst-img/a5c37bce3cf9a077a4eb62a4d95f89584b5ef8acee279b81de6019d0e5768ba0-1.svg}}

\subsection{\texorpdfstring{\hyperref[setting-spacing]{Setting
spacing}}{Setting spacing}}\label{setting-spacing}

TODO: explain block spacing for common elements

\subsection{\texorpdfstring{\hyperref[formatting-to-get-an-article-look]{Formatting
to get an "article
look"}}{Formatting to get an "article look"}}\label{formatting-to-get-an-article-look}

\begin{verbatim}
#set page(
  // Header is that small thing on top
  header: align(
    right + horizon,
    [Some header there]
  ),
  height: 12cm
)

#align(center, text(17pt)[
  *Important title*
])

#grid(
  columns: (1fr, 1fr),
  align(center)[
    Some author \
    Some Institute \
    #link("mailto:some@mail.edu")
  ],
  align(center)[
    Another author \
    Another Institute \
    #link("mailto:another@mail.edu")
  ]
)

Now let's split text into two columns:

#show: rest => columns(2, rest)

#show heading.where(
  level: 1
): it => block(width: 100%)[
  #set align(center)
  #set text(12pt, weight: "regular")
  #smallcaps(it.body)
]

#show heading.where(
  level: 2
): it => text(
  size: 11pt,
  weight: "regular",
  style: "italic",
  it.body + [.],
)

// Now let's fill it with words:

= Heading
== Small heading
#lorem(10)
== Second subchapter
#lorem(10)
= Second heading
#lorem(40)

== Second subchapter
#lorem(40)
\end{verbatim}

\pandocbounded{\includesvg[keepaspectratio]{typst-img/76ee0cca809299df178ec9d94371c01031d1808a700b39deac5245dd6b83157f-1.svg}}




\section{Combined Examples Book LaTeX/book/basics/must_know.tex}
\section{Examples Book LaTeX/book/basics/must_know/spacing.tex}
\title{sitandr.github.io/typst-examples-book/book/basics/must_know/spacing}

\section{\texorpdfstring{\hyperref[using-spacing]{Using
spacing}}{Using spacing}}\label{using-spacing}

Most time you will pass spacing into functions. There are special
function fields that take only \emph{size} . They are usually called
like
\texttt{\ }{\texttt{\ width,\ length,\ in(out)set,\ spacing\ }}\texttt{\ }
and so on.

Like in CSS, one of the ways to set up spacing in Typst is setting
margins and padding of elements. However, you can also insert spacing
directly using functions \texttt{\ }{\texttt{\ h\ }}\texttt{\ }
(horizontal spacing) and \texttt{\ }{\texttt{\ v\ }}\texttt{\ }
(vertical spacing).

\begin{quote}
Links to reference: \href{https://typst.app/docs/reference/layout/h/}{h}
, \href{https://typst.app/docs/reference/layout/v/}{v} .
\end{quote}

\begin{verbatim}
Horizontal #h(1cm) spacing.
#v(1cm)
And some vertical too!
\end{verbatim}

\pandocbounded{\includesvg[keepaspectratio]{typst-img/47b3ea7d16575780e489790177df9a624ad3c6c669594baa4127c1db516ebc94-1.svg}}

\section{\texorpdfstring{\hyperref[absolute-length-units]{Absolute
length units}}{Absolute length units}}\label{absolute-length-units}

\begin{quote}
Link to
\href{https://typst.app/docs/reference/layout/length/}{reference}
\end{quote}

Absolute length (aka just "length") units are not affected by outer
content and size of parent.

\begin{verbatim}
#set rect(height: 1em)
#table(
  columns: 2,
  [Points], rect(width: 72pt),
  [Millimeters], rect(width: 25.4mm),
  [Centimeters], rect(width: 2.54cm),
  [Inches], rect(width: 1in),
)
\end{verbatim}

\pandocbounded{\includesvg[keepaspectratio]{typst-img/073ad26fe313743ab62dca82f30208dbf2d57ff354d5c37f0b6d4c063dc37d76-1.svg}}

\subsection{\texorpdfstring{\hyperref[relative-to-current-font-size]{Relative
to current font
size}}{Relative to current font size}}\label{relative-to-current-font-size}

\texttt{\ }{\texttt{\ 1em\ =\ 1\ current\ font\ size\ }}\texttt{\ } :

\begin{verbatim}
#set rect(height: 1em)
#table(
  columns: 2,
  [Centimeters], rect(width: 2.54cm),
  [Relative to font size], rect(width: 6.5em)
)

Double font size: #box(stroke: red, baseline: 40%, height: 2em, width: 2em)
\end{verbatim}

\pandocbounded{\includesvg[keepaspectratio]{typst-img/7d62c9e2540f8bce40d8a3fc65a2779b161eb6b5b5682cf87247fee7f14145c2-1.svg}}

It is a very convenient unit, so it is used a lot in Typst.

\subsection{\texorpdfstring{\hyperref[combined]{Combined}}{Combined}}\label{combined}

\begin{verbatim}
Combined: #box(rect(height: 5pt + 1em))

#(5pt + 1em).abs
#(5pt + 1em).em
\end{verbatim}

\pandocbounded{\includesvg[keepaspectratio]{typst-img/c8a0cae6047f35c85c41ac44ff2a6b0d28a28d0e097ca61b367202f9a361136e-1.svg}}

\section{\texorpdfstring{\hyperref[ratio-length]{Ratio
length}}{Ratio length}}\label{ratio-length}

\begin{quote}
Link to \href{https://typst.app/docs/reference/layout/ratio/}{reference}
\end{quote}

\texttt{\ }{\texttt{\ 1\%\ =\ 1\%\ from\ parent\ size\ in\ that\ dimension\ }}\texttt{\ }

\begin{verbatim}
This line width is 50% of available page size (without margins):

#line(length: 50%)

This line width is 50% of the box width: #box(stroke: red, width: 4em, inset: (y: 0.5em), line(length: 50%))
\end{verbatim}

\pandocbounded{\includesvg[keepaspectratio]{typst-img/d478cb8be0a049380479b634cae709dc1e1ed406d323ecb1edbca1e582d7eafe-1.svg}}

\section{\texorpdfstring{\hyperref[relative-length]{Relative
length}}{Relative length}}\label{relative-length}

\begin{quote}
Link to
\href{https://typst.app/docs/reference/layout/relative/}{reference}
\end{quote}

You can \emph{combine} absolute and ratio lengths into \emph{relative
length} :

\begin{verbatim}
#rect(width: 100% - 50pt)

#(100% - 50pt).length \
#(100% - 50pt).ratio
\end{verbatim}

\pandocbounded{\includesvg[keepaspectratio]{typst-img/6b72620a1972e758e55ef1ecf49d3e843095037399ed4dd2dfcd262ebbbe803f-1.svg}}

\section{\texorpdfstring{\hyperref[fractional-length]{Fractional
length}}{Fractional length}}\label{fractional-length}

\begin{quote}
Link to
\href{https://typst.app/docs/reference/layout/fraction/}{reference}
\end{quote}

Single fraction length just takes \emph{maximum size possible} to fill
the parent:

\begin{verbatim}
Left #h(1fr) Right

#rect(height: 1em)[
  #h(1fr)
]
\end{verbatim}

\pandocbounded{\includesvg[keepaspectratio]{typst-img/b9c91f53b684699fff70c6889c8a47fccc57c5c540d7629b93c51a797eb2ef3c-1.svg}}

There are not many places you can use fractions, mainly those are
\texttt{\ }{\texttt{\ h\ }}\texttt{\ } and
\texttt{\ }{\texttt{\ v\ }}\texttt{\ } .

\subsection{\texorpdfstring{\hyperref[several-fractions]{Several
fractions}}{Several fractions}}\label{several-fractions}

If you use several fractions inside one parent, they will take all
remaining space \emph{proportional to their number} :

\begin{verbatim}
Left #h(1fr) Left-ish #h(2fr) Right
\end{verbatim}

\pandocbounded{\includesvg[keepaspectratio]{typst-img/45182cbcecf395256d133af78fccacd9d48e29073672317744cb17340d0bafd8-1.svg}}

\subsection{\texorpdfstring{\hyperref[nested-layout]{Nested
layout}}{Nested layout}}\label{nested-layout}

Remember that fractions work in parent only, don\textquotesingle t
\emph{rely on them in nested layout} :

\begin{verbatim}
Word: #h(1fr) #box(height: 1em, stroke: red)[
  #h(2fr)
]
\end{verbatim}

\pandocbounded{\includesvg[keepaspectratio]{typst-img/0c7ed8b25ea7e39a0907b1105b82027a0fb8b921b28978f30692f6c693bea5f7-1.svg}}


\section{Examples Book LaTeX/book/basics/must_know/box_block.tex}
\title{sitandr.github.io/typst-examples-book/book/basics/must_know/box_block}

\section{\texorpdfstring{\hyperref[boxing--blocking]{Boxing \&
Blocking}}{Boxing \& Blocking}}\label{boxing--blocking}

\begin{verbatim}
You can use boxes to wrap anything
into text: #box(image("../tiger.jpg", height: 2em)).

Blocks will always be "separate paragraphs".
They will not fit into a text: #block(image("../tiger.jpg", height: 2em))
\end{verbatim}

\pandocbounded{\includesvg[keepaspectratio]{typst-img/8e3bd89485b00259666bd636cf28586f92db9c3c3922f0adcdad765ee66a06b1-1.svg}}

Both have similar useful properties:

\begin{verbatim}
#box(stroke: red, inset: 1em)[Box text]
#block(stroke: red, inset: 1em)[Block text]
\end{verbatim}

\pandocbounded{\includesvg[keepaspectratio]{typst-img/9e3562619cb8a31b3d2311f53c3815a214f081e033a564e63dc003dfbc50d68d-1.svg}}

\subsection{\texorpdfstring{\hyperref[rect]{\texttt{\ }{\texttt{\ rect\ }}\texttt{\ }}}{  rect  }}\label{rect}

There is also \texttt{\ }{\texttt{\ rect\ }}\texttt{\ } that works like
\texttt{\ }{\texttt{\ block\ }}\texttt{\ } , but has useful default
inset and stroke:

\begin{verbatim}
#rect[Block text]
\end{verbatim}

\pandocbounded{\includesvg[keepaspectratio]{typst-img/c778d1e94a3663a4f258985368c02e294a1333554c550b6cfe0465275a2eef0f-1.svg}}

\subsection{\texorpdfstring{\hyperref[figures]{Figures}}{Figures}}\label{figures}

For the purposes of adding a \emph{figure} to your document, use
\texttt{\ }{\texttt{\ figure\ }}\texttt{\ } function.
Don\textquotesingle t try to use boxes or blocks there.

Figures are that things like centered images (probably with captions),
tables, even code.

\begin{verbatim}
@tiger shows a tiger. Tigers
are animals.

#figure(
  image("../tiger.jpg", width: 80%),
  caption: [A tiger.],
) <tiger>
\end{verbatim}

\pandocbounded{\includesvg[keepaspectratio]{typst-img/09a8b5b3c3bfffd81be7f34c31cc93ca5f8341b2594d022b2b92ac285aeb959d-1.svg}}

In fact, you can put there anything you want:

\begin{verbatim}
They told me to write a letter to you. Here it is:

#figure(
  text(size: 5em)[I],
  caption: [I'm cool, right?],
)
\end{verbatim}

\pandocbounded{\includesvg[keepaspectratio]{typst-img/e009534c4572064346490dfac659ff94a5a11d7f46af7a2b46c2136d206088c6-1.svg}}


\section{Examples Book LaTeX/book/basics/must_know/index.tex}
\title{sitandr.github.io/typst-examples-book/book/basics/must_know/index}

\section{\texorpdfstring{\hyperref[must-know]{Must-know}}{Must-know}}\label{must-know}

This section contains things, that are not general enough to be part of
"tutorial", but still are very important to know for proper typesetting.

Feel free to skip through things you are sure you will not use.


\section{Examples Book LaTeX/book/basics/must_know/place.tex}
\title{sitandr.github.io/typst-examples-book/book/basics/must_know/place}

\section{\texorpdfstring{\hyperref[placing-moving-scale--hide]{Placing,
Moving, Scale \&
Hide}}{Placing, Moving, Scale \& Hide}}\label{placing-moving-scale--hide}

This is \textbf{a very important section} if you want to do arbitrary
things with layout, create custom elements and hacking a way around
current Typst limitations.

TODO: WIP, add text and better examples

\section{\texorpdfstring{\hyperref[place]{Place}}{Place}}\label{place}

\emph{Ignore layout} , just put some object somehow relative to parent
and current position. The placed object \emph{will not} affect layouting

\begin{quote}
Link to \href{https://typst.app/docs/reference/layout/place/}{reference}
\end{quote}

\begin{verbatim}
#set page(height: 60pt)
Hello, world!

#place(
  top + right, // place at the page right and top
  square(
    width: 20pt,
    stroke: 2pt + blue
  ),
)
\end{verbatim}

\pandocbounded{\includesvg[keepaspectratio]{typst-img/e0d4c250d0f288e1a110ebddcb06149e0acd11b626a0ccb0ca9feb1c1d7be359-1.svg}}

\subsubsection{\texorpdfstring{\hyperref[basic-floating-with-place]{Basic
floating with
place}}{Basic floating with place}}\label{basic-floating-with-place}

\begin{verbatim}
#set page(height: 150pt)
#let note(where, body) = place(
  center + where,
  float: true,
  clearance: 6pt,
  rect(body),
)

#lorem(10)
#note(bottom)[Bottom 1]
#note(bottom)[Bottom 2]
#lorem(40)
#note(top)[Top]
#lorem(10)
\end{verbatim}

\pandocbounded{\includesvg[keepaspectratio]{typst-img/b770cfef024690b5fc7ab82458797d6cfab0c5cc8f52078ecf2d61be17c13acc-1.svg}}

\pandocbounded{\includesvg[keepaspectratio]{typst-img/b770cfef024690b5fc7ab82458797d6cfab0c5cc8f52078ecf2d61be17c13acc-2.svg}}

\subsubsection{\texorpdfstring{\hyperref[dx-dy]{dx,
dy}}{dx, dy}}\label{dx-dy}

Manually change position by
\texttt{\ }{\texttt{\ (dx,\ dy)\ }}\texttt{\ } relative to intended.

\begin{verbatim}
#set page(height: 100pt)
#for i in range(16) {
  let amount = i * 4pt
  place(center, dx: amount - 32pt, dy: amount)[A]
}
\end{verbatim}

\pandocbounded{\includesvg[keepaspectratio]{typst-img/12464f1a2cfe81fb04623033345f3f88ff598af5dc77de378b9d7cf88fc1d5b3-1.svg}}

\section{\texorpdfstring{\hyperref[move]{Move}}{Move}}\label{move}

\begin{quote}
Link to \href{https://typst.app/docs/reference/layout/move/}{reference}
\end{quote}

\begin{verbatim}
#rect(inset: 0pt, move(
  dx: 6pt, dy: 6pt,
  rect(
    inset: 8pt,
    fill: white,
    stroke: black,
    [Abra cadabra]
  )
))
\end{verbatim}

\pandocbounded{\includesvg[keepaspectratio]{typst-img/3292aebf7b633a2d9574027f50867d723d80850e046a101b9df5ab5143eb8a8d-1.svg}}

\section{\texorpdfstring{\hyperref[scale]{Scale}}{Scale}}\label{scale}

Scale content \emph{without affecting the layout} .

\begin{quote}
Link to \href{https://typst.app/docs/reference/layout/scale/}{reference}
\end{quote}

\begin{verbatim}
#scale(x: -100%)[This is mirrored.]
\end{verbatim}

\pandocbounded{\includesvg[keepaspectratio]{typst-img/401c8cd6f306771a3b12432c3c51e097a3ec1d12656c131c0043a12c4c1c3a0e-1.svg}}

\begin{verbatim}
A#box(scale(75%)[A])A \
B#box(scale(75%, origin: bottom + left)[B])B
\end{verbatim}

\pandocbounded{\includesvg[keepaspectratio]{typst-img/204b55690645eb6cc623c8d2d74b5521d72e4ba38d58ea40ea5e2d4354a01836-1.svg}}

\section{\texorpdfstring{\hyperref[hide]{Hide}}{Hide}}\label{hide}

Don\textquotesingle t show content, but leave empty space there.

\begin{quote}
Link to \href{https://typst.app/docs/reference/layout/hide/}{reference}
\end{quote}

\begin{verbatim}
Hello Jane \
#hide[Hello] Joe
\end{verbatim}

\pandocbounded{\includesvg[keepaspectratio]{typst-img/610672d5e43baa3ce94fe61f8d6dd0307e405c785639359c6a9e84bdd66884ad-1.svg}}


\section{Examples Book LaTeX/book/basics/must_know/project_struct.tex}
\title{sitandr.github.io/typst-examples-book/book/basics/must_know/project_struct}

\section{\texorpdfstring{\hyperref[project-structure]{Project
structure}}{Project structure}}\label{project-structure}

\subsection{\texorpdfstring{\hyperref[large-document]{Large
document}}{Large document}}\label{large-document}

Once the document becomes large enough, it becomes harder to navigate
it. If you haven\textquotesingle t reached that size yet, you can ignore
that section.

For managing that I would recommend splitting your document into
\emph{chapters} . It is just a way to work with this, but once you
understand how it works, you can do anything you want.

Let\textquotesingle s say you have two chapters, then the recommended
structure will look like this:

\begin{verbatim}
#import "@preview/treet:0.1.1": *

#show list: tree-list
#set par(leading: 0.8em)
#show list: set text(font: "DejaVu Sans Mono", size: 0.8em)
- chapters/
  - chapter_1.typ
  - chapter_2.typ
- main.typ 👁 #text(gray)[← document entry point]
- template.typ
\end{verbatim}

\pandocbounded{\includesvg[keepaspectratio]{typst-img/291489e71b40beea77872ad05adb609349872e9a11fc3a9c3f2008c88e37c9d5-1.svg}}

The exact file names are up to you.

Let\textquotesingle s see what to put in each of these files.

\subsubsection{\texorpdfstring{\hyperref[template]{Template}}{Template}}\label{template}

In the "template" file goes \emph{all useful functions and variables}
you will use across the chapters. If you have your own template or want
to write one, you can write it there.

\begin{verbatim}
// template.typ

#let template = doc => {
    set page(header: "My super document")
    show "physics": "magic"
    doc
}

#let info-block = block.with(stroke: blue, fill: blue.lighten(70%))
#let author = "@sitandr"
\end{verbatim}

\subsubsection{\texorpdfstring{\hyperref[main]{Main}}{Main}}\label{main}

\textbf{This file should be compiled} to get the whole compiled
document.

\begin{verbatim}
// main.typ

#import "template.typ": *
// if you have a template
#show: template

= This is the document title

// some additional formatting

#show emph: set text(blue)

// but don't define functions or variables there!
// chapters will not see it

// Now the chapters themselves as some Typst content
#include("chapters/chapter_1.typ")
#include("chapters/chapter_1.typ")
\end{verbatim}

\subsubsection{\texorpdfstring{\hyperref[chapter]{Chapter}}{Chapter}}\label{chapter}

\begin{verbatim}
// chapter_1.typ

#import "../template.typ": *

That's just content with _styling_ and blocks:

#infoblock[Some information].

// just any content you want to include in the document
\end{verbatim}

\subsection{\texorpdfstring{\hyperref[notes]{Notes}}{Notes}}\label{notes}

Note that modules in Typst can see only what they created themselves or
imported. Anything else is invisible for them. That\textquotesingle s
why you need \texttt{\ }{\texttt{\ template.typ\ }}\texttt{\ } file to
define all functions within.

That means chapters \emph{don\textquotesingle t see each other either} ,
only what is in the template.

\subsection{\texorpdfstring{\hyperref[cyclic-imports]{Cyclic
imports}}{Cyclic imports}}\label{cyclic-imports}

\textbf{Important:} Typst \emph{forbids} cyclic imports. That means you
can\textquotesingle t import
\texttt{\ }{\texttt{\ chapter\_1\ }}\texttt{\ } from
\texttt{\ }{\texttt{\ chapter\_2\ }}\texttt{\ } and
\texttt{\ }{\texttt{\ chapter\_2\ }}\texttt{\ } from
\texttt{\ }{\texttt{\ chapter\_1\ }}\texttt{\ } at the same time!

But the good news is that you can always create some other file to
import variable from.


\section{Examples Book LaTeX/book/basics/must_know/tables.tex}
\title{sitandr.github.io/typst-examples-book/book/basics/must_know/tables}

\section{\texorpdfstring{\hyperref[tables-and-grids]{Tables and
grids}}{Tables and grids}}\label{tables-and-grids}

While tables are not that necessary to know if you don\textquotesingle t
plan to use them in your documents, grids may be very useful for
\emph{document layout} . We will use both of them them in the book
later.

Let\textquotesingle s not bother with copying examples from official
documentation. Just make sure to skim through it, okay?

\subsection{\texorpdfstring{\hyperref[basic-snippets]{Basic
snippets}}{Basic snippets}}\label{basic-snippets}

\subsubsection{\texorpdfstring{\hyperref[spreading]{Spreading}}{Spreading}}\label{spreading}

Spreading operators (see \href{../scripting/arguments.html}{there} ) may
be especially useful for the tables:

\begin{verbatim}
#set text(size: 9pt)

#let yield_cells(n) = {
  for i in range(0, n + 1) {
    for j in range(0, n + 1) {
      let product = if i * j != 0 {
        // math is used for the better look
        if j <= i { $#{ j * i }$ }
        else {
          // upper part of the table
          text(gray.darken(50%), str(i * j))
        }
      } else {
        if i == j {
          // the top right corner
          $times$
        } else {
          // on of them is zero, we are at top/left
          $#{i + j}$
        }
      }
      // this is an array, for loops merge them together
      // into one large array of cells
      (
        table.cell(
          fill: if i == j and j == 0 { orange } // top right corner
          else if i == j { yellow } // the diagonal
          else if i * j == 0 { blue.lighten(50%) }, // multipliers
          product,),
      )
    }
  }
}

#let n = 10
#table(
  columns: (0.6cm,) * (n + 1), rows: (0.6cm,) * (n + 1), align: center + horizon, inset: 3pt, ..yield_cells(n),
)
\end{verbatim}

\pandocbounded{\includesvg[keepaspectratio]{typst-img/0640c1d0e5f79bdcb5e60f7675ff1b1eb18810078f5bbbdfaf1c5648b987706e-1.svg}}

\subsubsection{\texorpdfstring{\hyperref[highlighting-table-row]{Highlighting
table row}}{Highlighting table row}}\label{highlighting-table-row}

\begin{verbatim}
#table(
  columns: 2,
  fill: (x, y) => if y == 2 { highlight.fill },
  [A], [B],
  [C], [D],
  [E], [F],
  [G], [H],
)
\end{verbatim}

\pandocbounded{\includesvg[keepaspectratio]{typst-img/4ff8cbb75f85dbab08a336be31115bcb4cb8ca505799641534d937d444e88082-1.svg}}

For individual cells, use

\begin{verbatim}
#table(
  columns: 2,
  [A], [B],
  table.cell(fill: yellow)[C], table.cell(fill: yellow)[D],
  [E], [F],
  [G], [H],
)
\end{verbatim}

\pandocbounded{\includesvg[keepaspectratio]{typst-img/07676a86d4643ff83988c0907aa17995b3d1f8fa7b5be4f11959551afd674bc9-1.svg}}

\subsubsection{\texorpdfstring{\hyperref[splitting-tables]{Splitting
tables}}{Splitting tables}}\label{splitting-tables}

Tables are split between pages automatically.

\begin{verbatim}
#set page(height: 8em)
#(
table(
  columns: 5,
  [Aligner], [publication], [Indexing], [Pairwise alignment], [Max. read length  (bp)],
  [BWA], [2009], [BWT-FM], [Semi-Global], [125],
  [Bowtie], [2009], [BWT-FM], [HD], [76],
  [CloudBurst], [2009], [Hashing], [Landau-Vishkin], [36],
  [GNUMAP], [2009], [Hashing], [NW], [36]
  )
)
\end{verbatim}

\pandocbounded{\includesvg[keepaspectratio]{typst-img/34794c27fefc5c307a1dfdc9ad7958c1dcca0ff8fb64962047051c6a216e0ff7-1.svg}}

\pandocbounded{\includesvg[keepaspectratio]{typst-img/34794c27fefc5c307a1dfdc9ad7958c1dcca0ff8fb64962047051c6a216e0ff7-2.svg}}

However, if you want to make it breakable inside other element,
you\textquotesingle ll have to make that element breakable too:

\begin{verbatim}
#set page(height: 8em)
// Without this, the table fails to split upon several pages
#show figure: set block(breakable: true)
#figure(
table(
  columns: 5,
  [Aligner], [publication], [Indexing], [Pairwise alignment], [Max. read length  (bp)],
  [BWA], [2009], [BWT-FM], [Semi-Global], [125],
  [Bowtie], [2009], [BWT-FM], [HD], [76],
  [CloudBurst], [2009], [Hashing], [Landau-Vishkin], [36],
  [GNUMAP], [2009], [Hashing], [NW], [36]
  )
)
\end{verbatim}

\pandocbounded{\includesvg[keepaspectratio]{typst-img/5be04bf8770a33256599791fb50751bcb24fa5108c13d0e5e2807b675fed00fb-1.svg}}

\pandocbounded{\includesvg[keepaspectratio]{typst-img/5be04bf8770a33256599791fb50751bcb24fa5108c13d0e5e2807b675fed00fb-2.svg}}




\section{Combined Examples Book LaTeX/book/basics/states.tex}
\section{Examples Book LaTeX/book/basics/states/counters.tex}
\title{sitandr.github.io/typst-examples-book/book/basics/states/counters}

\section{\texorpdfstring{\hyperref[counters]{Counters}}{Counters}}\label{counters}

This section is outdated. It may be still useful, but it is strongly
recommended to study new context system (using the reference).

Counters are special states that \emph{count} elements of some type. As
with states, you can create your own with identifier strings.

\emph{Important:} to initiate counters of elements, you need to
\emph{set numbering for them} .

\subsection{\texorpdfstring{\hyperref[states-methods]{States
methods}}{States methods}}\label{states-methods}

Counters are states, so they can do all things states can do.

\begin{verbatim}
#set heading(numbering: "1.")

= Background
#counter(heading).update(3)
#counter(heading).update(n => n * 2)

== Analysis
Current heading number: #counter(heading).display().
\end{verbatim}

\pandocbounded{\includesvg[keepaspectratio]{typst-img/c57c9907a5f238f0b5eee74f8c23c57a5e2d5b0c9cbf7ebd1befdfcbd33289df-1.svg}}

\begin{verbatim}
#let mine = counter("mycounter")
#mine.display()

#mine.step()
#mine.display()

#mine.update(c => c * 3)
#mine.display()
\end{verbatim}

\pandocbounded{\includesvg[keepaspectratio]{typst-img/876103777c9564f0bb524f83a988a6d444c4e889baed31ee960548d90f3233e2-1.svg}}

\subsection{\texorpdfstring{\hyperref[displaying-counters]{Displaying
counters}}{Displaying counters}}\label{displaying-counters}

\begin{verbatim}
#set heading(numbering: "1.")

= Introduction
Some text here.

= Background
The current value is:
#counter(heading).display()

Or in roman numerals:
#counter(heading).display("I")
\end{verbatim}

\pandocbounded{\includesvg[keepaspectratio]{typst-img/1ac65f4be42131b3cca1d7c56c6c60c3932a703e5e499c1c5cb874458028abea-1.svg}}

Counters also support displaying \emph{both current and final values}
out-of-box:

\begin{verbatim}
#set heading(numbering: "1.")

= Introduction
Some text here.

#counter(heading).display(both: true) \
#counter(heading).display("1 of 1", both: true) \
#counter(heading).display(
  (num, max) => [#num of #max],
   both: true
)

= Background
The current value is: #counter(heading).display()
\end{verbatim}

\pandocbounded{\includesvg[keepaspectratio]{typst-img/af9d0da905bbb2215461b07b39653ef3890ff11a364afe018dae4ce4216f4961-1.svg}}

\subsection{\texorpdfstring{\hyperref[step]{Step}}{Step}}\label{step}

That\textquotesingle s quite easy, for counters you can increment value
using \texttt{\ }{\texttt{\ step\ }}\texttt{\ } . It works the same way
as \texttt{\ }{\texttt{\ update\ }}\texttt{\ } .

\begin{verbatim}
#set heading(numbering: "1.")

= Introduction
#counter(heading).step()

= Analysis
Let's skip 3.1.
#counter(heading).step(level: 2)

== Analysis
At #counter(heading).display().
\end{verbatim}

\pandocbounded{\includesvg[keepaspectratio]{typst-img/12446a2258e9862d8df8b6b250ff14efbb9c35da165a2a04e8c4aa12c9b68cdf-1.svg}}

\subsection{\texorpdfstring{\hyperref[you-can-use-counters-in-your-functions]{You
can use counters in your
functions:}}{You can use counters in your functions:}}\label{you-can-use-counters-in-your-functions}

\begin{verbatim}
#let c = counter("theorem")
#let theorem(it) = block[
  #c.step()
  *Theorem #c.display():*
  #it
]

#theorem[$1 = 1$]
#theorem[$2 < 3$]
\end{verbatim}

\pandocbounded{\includesvg[keepaspectratio]{typst-img/0f178f614e49a7400d646926705364a92ca3d4d888423b2693f071f83ce09e7d-1.svg}}


\section{Examples Book LaTeX/book/basics/states/metadata.tex}
\title{sitandr.github.io/typst-examples-book/book/basics/states/metadata}

\section{\texorpdfstring{\hyperref[metadata]{Metadata}}{Metadata}}\label{metadata}

Metadata is invisible content that can be extracted using query or other
content. This may be very useful with
\texttt{\ }{\texttt{\ typst\ query\ }}\texttt{\ } to pass values to
external tools.

\begin{verbatim}
// Put metadata somewhere.
#metadata("This is a note") <note>

// And find it from anywhere else.
#context {
  query(<note>).first().value
}
\end{verbatim}

\pandocbounded{\includesvg[keepaspectratio]{typst-img/ef1c7d9faf74901f6c5266d48ae006167003a22754408a70ae9f9d1088b5fe24-1.svg}}


\section{Examples Book LaTeX/book/basics/states/index.tex}
\title{sitandr.github.io/typst-examples-book/book/basics/states/index}

\section{\texorpdfstring{\hyperref[states--query]{States \&
Query}}{States \& Query}}\label{states--query}

This section is outdated. It may be still useful, but it is strongly
recommended to study new context system (using the reference).

Typst tries to be a \emph{pure language} as much as possible.

That means, a function can\textquotesingle t change anything outside of
it. That also means, if you call function, the result should be always
the same.

Unfortunately, our world (and therefore our documents)
isn\textquotesingle t pure. If you create a heading №2, you want the
next number to be three.

That section will guide you to using impure Typst. Don\textquotesingle t
overuse it, as this knowledge comes close to the Dark Arts of Typst!


\section{Examples Book LaTeX/book/basics/states/states.tex}
\title{sitandr.github.io/typst-examples-book/book/basics/states/states}

\section{\texorpdfstring{\hyperref[states]{States}}{States}}\label{states}

This section is outdated. It may be still useful, but it is strongly
recommended to study new context system (using the reference).

Before we start something practical, it is important to understand
states in general.

Here is a good explanation of why do we \emph{need} them:
\href{https://typst.app/docs/reference/meta/state/}{Official Reference
about states} . It is highly recommended to read it first.

So instead of

\begin{verbatim}
#let x = 0
#let compute(expr) = {
  // eval evaluates string as Typst code
  // to calculate new x value
  x = eval(
    expr.replace("x", str(x))
  )
  [New value is #x.]
}

#compute("10") \
#compute("x + 3") \
#compute("x * 2") \
#compute("x - 5")
\end{verbatim}

\textbf{THIS DOES NOT COMPILE:} Variables from outside the function are
read-only and cannot be modified

Instead, you should write

\begin{verbatim}
#let s = state("x", 0)
#let compute(expr) = [
  // updates x current state with this function
  #s.update(x =>
    eval(expr.replace("x", str(x)))
  )
  // and displays it
  New value is #context s.get().
]

#compute("10") \
#compute("x + 3") \
#compute("x * 2") \
#compute("x - 5")

The computations will be made _in order_ they are _located_ in the document. So if you create computations first, but put them in the document later... See yourself:

#let more = [
  #compute("x * 2") \
  #compute("x - 5")
]

#compute("10") \
#compute("x + 3") \
#more
\end{verbatim}

\pandocbounded{\includesvg[keepaspectratio]{typst-img/9a88397d1a9b5a44b1a3a218894595121bd4c5ec875df2b960638f2925060334-1.svg}}

\subsection{\texorpdfstring{\hyperref[context-magic]{Context
magic}}{Context magic}}\label{context-magic}

So what does this magic
\texttt{\ }{\texttt{\ context\ s.get()\ }}\texttt{\ } mean?

\begin{quote}
\href{https://typst.app/docs/reference/context/}{Context in Reference}
\end{quote}

In short, it specifies what part of your code (or markup) can
\emph{depend on states outside} . This context-expression is packed then
as one object, and it is evaluated on layout stage.

That means it is impossible to look from "normal" code at whatever is
inside the \texttt{\ }{\texttt{\ context\ }}\texttt{\ } . This is a
black box that would be known \emph{only after putting it into the
document} .

We will discuss \texttt{\ }{\texttt{\ context\ }}\texttt{\ } features
later.

\subsection{\texorpdfstring{\hyperref[operations-with-states]{Operations
with states}}{Operations with states}}\label{operations-with-states}

\subsubsection{\texorpdfstring{\hyperref[creating-new-state]{Creating
new state}}{Creating new state}}\label{creating-new-state}

\begin{verbatim}
#let x = state("state-id")
#let y = state("state-id", 2)

#x, #y

State is #context x.get() \ // the same as
#context [State is #y.get()] \ // the same as
#context {"State is" + str(y.get())}
\end{verbatim}

\pandocbounded{\includesvg[keepaspectratio]{typst-img/4a52375bdeea2b7ca31dc51740563d01b3678f817dd6bc8c349d0714c2ac503f-1.svg}}

\subsubsection{\texorpdfstring{\hyperref[update]{Update}}{Update}}\label{update}

Updating is \emph{a content} that is an instruction. That instruction
tells compiler that in this place of document the state \emph{should be
updated} .

\begin{verbatim}
#let x = state("x", 0)
#context x.get() \
#let _ = x.update(3)
// nothing happens, we don't put `update` into the document flow
#context x.get()

#repr(x.update(3)) // this is how that content looks \

#context x.update(3)
#context x.get() // Finally!
\end{verbatim}

\pandocbounded{\includesvg[keepaspectratio]{typst-img/3732a9c7bca8c4faedf9b024e09e647a65222c8244e9f3235a6057dfebc0a511-1.svg}}

Here we can see one of \emph{important
\texttt{\ }{\texttt{\ context\ }}\texttt{\ } traits} : it "sees" states
from outside, but can\textquotesingle t see how they change inside it:

\begin{verbatim}
#let x = state("x", 0)

#context {
  x.update(3)
  str(x.get())
}
\end{verbatim}

\pandocbounded{\includesvg[keepaspectratio]{typst-img/78e500b80cb85e086a81302e2ce3dad88cb4304d4685b88e3f59111bc71f6748-1.svg}}

\subsubsection{\texorpdfstring{\hyperref[id-collision]{ID
collision}}{ID collision}}\label{id-collision}

\emph{TLDR; \textbf{Never allow colliding states.}}

States are described by their id-s, if they are the same, the code will
break.

So, if you write functions or loops that are used several times,
\emph{be careful} !

\begin{verbatim}
#let f(x) = {
  // return new state…
  // …but their id-s are the same!
  // so it will always be the same state!
  let y = state("x", 0)
  y.update(y => y + x)
  context y.get()
}

#let a = f(2)
#let b = f(3)

#a, #b \
#raw(repr(a) + "\n" + repr(b))
\end{verbatim}

\pandocbounded{\includesvg[keepaspectratio]{typst-img/31a3e88747ed09ae6078bd3caf986f0e6ba744e055d0889d92bfa23941e7e451-1.svg}}

However, this \emph{may seem} okay:

\begin{verbatim}
// locations in code are different!
#let x = state("state-id")
#let y = state("state-id", 2)

#x, #y
\end{verbatim}

\pandocbounded{\includesvg[keepaspectratio]{typst-img/1901e1449942d821c66f53bd6bc5fda10d63591aa45346fdf88bcbc3f2ab3425-1.svg}}

But in fact, it \emph{isn\textquotesingle t} :

\begin{verbatim}
#let x = state("state-id")
#let y = state("state-id", 2)

#context [#x.get(); #y.get()]

#x.update(3)

#context [#x.get(); #y.get()]
\end{verbatim}

\pandocbounded{\includesvg[keepaspectratio]{typst-img/9185a298f9bcf8c519fa85481b9272e6ef3a00c117a0904d0509920a6abef8b2-1.svg}}


\section{Examples Book LaTeX/book/basics/states/query.tex}
\title{sitandr.github.io/typst-examples-book/book/basics/states/query}

\section{\texorpdfstring{\hyperref[query]{Query}}{Query}}\label{query}

This section is outdated. It may be still useful, but it is strongly
recommended to study new context system (using the reference).

\begin{quote}
Link \href{https://typst.app/docs/reference/meta/query/}{there}
\end{quote}

Query is a thing that allows you getting location by \emph{selector}
(this is the same thing we used in show rules).

That enables "time travel", getting information about document from its
parts and so on. \emph{That is a way to violate Typst\textquotesingle s
purity.}

It is currently one of the \emph{the darkest magics currently existing
in Typst} . It gives you great powers, but with great power comes great
responsibility.

\subsection{\texorpdfstring{\hyperref[time-travel]{Time
travel}}{Time travel}}\label{time-travel}

\begin{verbatim}
#let s = state("x", 0)
#let compute(expr) = [
  #s.update(x =>
    eval(expr.replace("x", str(x)))
  )
  New value is #s.display().
]

Value at `<here>` is
#context s.at(
  query(<here>)
    .first()
    .location()
)

#compute("10") \
#compute("x + 3") \
*Here.* <here> \
#compute("x * 2") \
#compute("x - 5")
\end{verbatim}

\pandocbounded{\includesvg[keepaspectratio]{typst-img/130940aa5ae2ceb3364ef655c84cf8e7d2178210851b8fb20e6c0c3345c3ace7-1.svg}}

\subsection{\texorpdfstring{\hyperref[getting-nearest-chapter]{Getting
nearest
chapter}}{Getting nearest chapter}}\label{getting-nearest-chapter}

\begin{verbatim}
#set page(header: context {
  let elems = query(
    selector(heading).before(here()),
    here(),
  )
  let academy = smallcaps[
    Typst Academy
  ]
  if elems == () {
    align(right, academy)
  } else {
    let body = elems.last().body
    academy + h(1fr) + emph(body)
  }
})

= Introduction
#lorem(23)

= Background
#lorem(30)

= Analysis
#lorem(15)
\end{verbatim}

\pandocbounded{\includesvg[keepaspectratio]{typst-img/b4d0562911dd308b0d9cbc36ad20ba6ed91fc3c3da5b6259eb6721f3a53a18e3-1.svg}}




\section{Combined Examples Book LaTeX/book/basics/scripting.tex}
\section{Examples Book LaTeX/book/basics/scripting/conditions.tex}
\title{sitandr.github.io/typst-examples-book/book/basics/scripting/conditions}

\section{\texorpdfstring{\hyperref[conditions--loops]{Conditions \&
loops}}{Conditions \& loops}}\label{conditions--loops}

\subsection{\texorpdfstring{\hyperref[conditions]{Conditions}}{Conditions}}\label{conditions}

\begin{quote}
See
\href{https://typst.app/docs/reference/scripting/\#conditionals}{official
documentation} .
\end{quote}

In Typst, you can use \texttt{\ }{\texttt{\ if-else\ }}\texttt{\ }
statements. This is especially useful inside function bodies to vary
behavior depending on arguments types or many other things.

\begin{verbatim}
#if 1 < 2 [
  This is shown
] else [
  This is not.
]
\end{verbatim}

\pandocbounded{\includesvg[keepaspectratio]{typst-img/2e914defa3353d6fd42ed58c37a97aedcc2237cfe20228f0cc0d223dfff4619a-1.svg}}

Of course, \texttt{\ }{\texttt{\ else\ }}\texttt{\ } is unnecessary:

\begin{verbatim}
#let a = 3

#if a < 4 {
  a = 5
}

#a
\end{verbatim}

\pandocbounded{\includesvg[keepaspectratio]{typst-img/a7264774be154606a44d829d31edae18bf686262ccea66de9ed97fa20c720bd8-1.svg}}

You can also use \texttt{\ }{\texttt{\ else\ if\ }}\texttt{\ } statement
(known as \texttt{\ }{\texttt{\ elif\ }}\texttt{\ } in Python):

\begin{verbatim}
#let a = 5

#if a < 4 {
  a = 5
} else if a < 6 {
  a = -3
}

#a
\end{verbatim}

\pandocbounded{\includesvg[keepaspectratio]{typst-img/9f65678fc26af2d197d979e1b0a5295ed64037ee00c30fa28c9c417a2c7dc308-1.svg}}

\subsubsection{\texorpdfstring{\hyperref[booleans]{Booleans}}{Booleans}}\label{booleans}

\texttt{\ }{\texttt{\ if,\ else\ if,\ else\ }}\texttt{\ } accept
\emph{only boolean} values as a switch. You can combine booleans as
described in \href{./types.html\#boolean-bool}{types section} :

\begin{verbatim}
#let a = 5

#if (a > 1 and a <= 4) or a == 5 [
    `a` matches the condition
]
\end{verbatim}

\pandocbounded{\includesvg[keepaspectratio]{typst-img/21d3a48404d4e0c59bc0fccb114fdeac7384189db0020247796f44b0e9a7c362-1.svg}}

\subsection{\texorpdfstring{\hyperref[loops]{Loops}}{Loops}}\label{loops}

\begin{quote}
See \href{https://typst.app/docs/reference/scripting/\#loops}{official
documentation} .
\end{quote}

There are two kinds of loops: \texttt{\ }{\texttt{\ while\ }}\texttt{\ }
and \texttt{\ }{\texttt{\ for\ }}\texttt{\ } . While repeats body while
the condition is met:

\begin{verbatim}
#let a = 3

#while a < 100 {
    a *= 2
    str(a)
    " "
}
\end{verbatim}

\pandocbounded{\includesvg[keepaspectratio]{typst-img/ece06c012663616cac05b0f365bd02ea5607dcddfaa0249963088ceff797c100-1.svg}}

\texttt{\ }{\texttt{\ for\ }}\texttt{\ } iterates over all elements of
sequence. The sequence may be an
\texttt{\ }{\texttt{\ array\ }}\texttt{\ } ,
\texttt{\ }{\texttt{\ string\ }}\texttt{\ } or
\texttt{\ }{\texttt{\ dictionary\ }}\texttt{\ } (
\texttt{\ }{\texttt{\ for\ }}\texttt{\ } iterates over its
\emph{key-value pairs} ).

\begin{verbatim}
#for c in "ABC" [
  #c is a letter.
]
\end{verbatim}

\pandocbounded{\includesvg[keepaspectratio]{typst-img/9e70091e4c1f276d548f8200329298bf6b98946c331ca4630fec8313d5a91eff-1.svg}}

To iterate to all numbers from \texttt{\ }{\texttt{\ a\ }}\texttt{\ } to
\texttt{\ }{\texttt{\ b\ }}\texttt{\ } , use
\texttt{\ }{\texttt{\ range(a,\ b+1)\ }}\texttt{\ } :

\begin{verbatim}
#let s = 0

#for i in range(3, 6) {
    s += i
    [Number #i is added to sum. Now sum is #s.]
}
\end{verbatim}

\pandocbounded{\includesvg[keepaspectratio]{typst-img/1e3d95ee79d7bc6989e40ff1e27c0ef6e3b152a1e5f8a0df5b2819621e0e299f-1.svg}}

Because range is end-exclusive this is equal to

\begin{verbatim}
#let s = 0

#for i in (3, 4, 5) {
    s += i
    [Number #i is added to sum. Now sum is #s.]
}
\end{verbatim}

\pandocbounded{\includesvg[keepaspectratio]{typst-img/6158d29261339f8f285d592deff8992ca129ce32264abcdcf6734ac44cf558a4-1.svg}}

\begin{verbatim}
#let people = (Alice: 3, Bob: 5)

#for (name, value) in people [
    #name has #value apples.
]
\end{verbatim}

\pandocbounded{\includesvg[keepaspectratio]{typst-img/50ff0963afe8c9ec5a0562d518431b63d5dd3810525f55f084f812452b11eb21-1.svg}}

\subsubsection{\texorpdfstring{\hyperref[break-and-continue]{Break and
continue}}{Break and continue}}\label{break-and-continue}

Inside loops can be used \texttt{\ }{\texttt{\ break\ }}\texttt{\ } and
\texttt{\ }{\texttt{\ continue\ }}\texttt{\ } commands.
\texttt{\ }{\texttt{\ break\ }}\texttt{\ } breaks loop, jumping outside.
\texttt{\ }{\texttt{\ continue\ }}\texttt{\ } jumps to next loop
iteration.

See the difference on these examples:

\begin{verbatim}
#for letter in "abc nope" {
  if letter == " " {
    // stop when there is space
    break
  }

  letter
}
\end{verbatim}

\pandocbounded{\includesvg[keepaspectratio]{typst-img/a744551cab635d3ab70d9bf4258bb5fc26fe384f8e9f487ad0b8eee986ffe581-1.svg}}

\begin{verbatim}
#for letter in "abc nope" {
  if letter == " " {
    // skip the space
    continue
  }

  letter
}
\end{verbatim}

\pandocbounded{\includesvg[keepaspectratio]{typst-img/bbb719820f986e52fbf64306536766ecbfd7264d29429a5c62d1bd648a4754c5-1.svg}}


\section{Examples Book LaTeX/book/basics/scripting/index.tex}
\title{sitandr.github.io/typst-examples-book/book/basics/scripting/index}

\section{\texorpdfstring{\hyperref[scripting]{Scripting}}{Scripting}}\label{scripting}

\textbf{Typst} has a complete interpreted language inside. One of key
aspects of working with your document in a nicer way


\section{Examples Book LaTeX/book/basics/scripting/types_2.tex}
\title{sitandr.github.io/typst-examples-book/book/basics/scripting/types_2}

\section{\texorpdfstring{\hyperref[types-part-ii]{Types, part
II}}{Types, part II}}\label{types-part-ii}

In Typst, most of things are \textbf{immutable} . You
can\textquotesingle t change content, you can just create new using this
one (for example, using addition).

Immutability is very important for Typst since it tries to be \emph{as
pure language as possible} . Functions do nothing outside of returning
some value.

However, purity is partly "broken" by these types. They are
\emph{super-useful} and not adding them would make Typst much pain.

However, using them adds complexity.

\subsection{\texorpdfstring{\hyperref[arrays-array]{Arrays (
\texttt{\ }{\texttt{\ array\ }}\texttt{\ }
)}}{Arrays (   array   )}}\label{arrays-array}

\begin{quote}
\href{https://typst.app/docs/reference/foundations/array/}{Link to
Reference} .
\end{quote}

Mutable object that stores data with their indices.

\subsubsection{\texorpdfstring{\hyperref[working-with-indices]{Working
with indices}}{Working with indices}}\label{working-with-indices}

\begin{verbatim}
#let values = (1, 7, 4, -3, 2)

// take value at index 0
#values.at(0) \
// set value at 0 to 3
#(values.at(0) = 3)
// negative index => start from the back
#values.at(-1) \
// add index of something that is even
#values.find(calc.even)
\end{verbatim}

\pandocbounded{\includesvg[keepaspectratio]{typst-img/0374c20b28fbf2b2d15bc32e5428f7f5121ea9d673d96de3274a0c6d988d5fb5-1.svg}}

\subsubsection{\texorpdfstring{\hyperref[iterating-methods]{Iterating
methods}}{Iterating methods}}\label{iterating-methods}

\begin{verbatim}
#let values = (1, 7, 4, -3, 2)

// leave only what is odd
#values.filter(calc.odd) \
// create new list of absolute values of list values
#values.map(calc.abs) \
// reverse
#values.rev() \
// convert array of arrays to flat array
#(1, (2, 3)).flatten() \
// join array of string to string
#(("A", "B", "C")
 .join(", ", last: " and "))
\end{verbatim}

\pandocbounded{\includesvg[keepaspectratio]{typst-img/684400186916f8f16a2d7edb151b7f5023c7e4c010b23a2c6566f0bd7a224061-1.svg}}

\subsubsection{\texorpdfstring{\hyperref[list-operations]{List
operations}}{List operations}}\label{list-operations}

\begin{verbatim}
// sum of lists:
#((1, 2, 3) + (4, 5, 6))

// list product:
#((1, 2, 3) * 4)
\end{verbatim}

\pandocbounded{\includesvg[keepaspectratio]{typst-img/abe2d311638b351e0938be0e432f10265ca81a69a9ed7d2e6f88f656c60dfc65-1.svg}}

\subsubsection{\texorpdfstring{\hyperref[empty-list]{Empty
list}}{Empty list}}\label{empty-list}

\begin{verbatim}
#() \ // this is an empty list
#(1,) \  // this is a list with one element
BAD: #(1) // this is just an element, not a list!
\end{verbatim}

\pandocbounded{\includesvg[keepaspectratio]{typst-img/da4f77f8784462ca5c4f73862e58420695916064d56921e4adef7a7e37d5a532-1.svg}}

\subsection{\texorpdfstring{\hyperref[dictionaries-dict]{Dictionaries (
\texttt{\ }{\texttt{\ dict\ }}\texttt{\ }
)}}{Dictionaries (   dict   )}}\label{dictionaries-dict}

\begin{quote}
\href{https://typst.app/docs/reference/foundations/dictionary/}{Link to
Reference} .
\end{quote}

Dictionaries are objects that store a string "key" and a value,
associated with that key.

\begin{verbatim}
#let dict = (
  name: "Typst",
  born: 2019,
)

#dict.name \
#(dict.launch = 20)
#dict.len() \
#dict.keys() \
#dict.values() \
#dict.at("born") \
#dict.insert("city", "Berlin ")
#("name" in dict)
\end{verbatim}

\pandocbounded{\includesvg[keepaspectratio]{typst-img/638ada64eb36af0b1891def1b2c0a2cc97a14d87987df8c16f5f3872244553d6-1.svg}}

\subsubsection{\texorpdfstring{\hyperref[empty-dictionary]{Empty
dictionary}}{Empty dictionary}}\label{empty-dictionary}

\begin{verbatim}
This is an empty list: #() \
This is an empty dict: #(:)
\end{verbatim}

\pandocbounded{\includesvg[keepaspectratio]{typst-img/6ef41801d46f0b7256bb6913482fde054c811a1850ecae3a446660eb6d1c8850-1.svg}}


\section{Examples Book LaTeX/book/basics/scripting/basics.tex}
\title{sitandr.github.io/typst-examples-book/book/basics/scripting/basics}

\section{\texorpdfstring{\hyperref[basics]{Basics}}{Basics}}\label{basics}

\subsection{\texorpdfstring{\hyperref[variables-i]{Variables
I}}{Variables I}}\label{variables-i}

Let\textquotesingle s start with \emph{variables} .

The concept is very simple, just some value you can reuse:

\begin{verbatim}
#let author = "John Doe"

This is a book by #author. #author is a great guy.

#quote(block: true, attribution: author)[
  \<Some quote\>
]
\end{verbatim}

\pandocbounded{\includesvg[keepaspectratio]{typst-img/c311c1612cafa802f16f0d4ca2d6f1ecca59f545ed1f6ee99d3c4ae06ee2bff4-1.svg}}

\subsection{\texorpdfstring{\hyperref[variables-ii]{Variables
II}}{Variables II}}\label{variables-ii}

You can store \emph{any} Typst value in variable:

\begin{verbatim}
#let block_text = block(stroke: red, inset: 1em)[Text]

#block_text

#figure(caption: "The block", block_text)
\end{verbatim}

\pandocbounded{\includesvg[keepaspectratio]{typst-img/c6290389652d1771d5149c9393af8eb32bd37e4b2bfb2c11764f9f22c294f84b-1.svg}}

\subsection{\texorpdfstring{\hyperref[functions]{Functions}}{Functions}}\label{functions}

We have already seen some "custom" functions in
\href{../tutorial/advanced_styling.html}{Advanced Styling} chapter.

Functions are values that take some values and output some values:

\begin{verbatim}
// This is a syntax that we have seen earlier
#let f = (name) => "Hello, " + name

#f("world!")
\end{verbatim}

\pandocbounded{\includesvg[keepaspectratio]{typst-img/23fba8e9081a8b32b16d7deb54018bb73a8ac910adbfb1a0ca577eb3520a73b4-1.svg}}

\subsubsection{\texorpdfstring{\hyperref[alternative-syntax]{Alternative
syntax}}{Alternative syntax}}\label{alternative-syntax}

You can write the same shorter:

\begin{verbatim}
// The following syntaxes are equivalent
#let f = (name) => "Hello, " + name
#let f(name) = "Hello, " + name

#f("world!")
\end{verbatim}

\pandocbounded{\includesvg[keepaspectratio]{typst-img/e6e4bd179a38f1b3af96f3e7c6308be6f9494f41f43daa26ebabf7a77fc54780-1.svg}}


\section{Examples Book LaTeX/book/basics/scripting/arguments.tex}
\title{sitandr.github.io/typst-examples-book/book/basics/scripting/arguments}

\section{\texorpdfstring{\hyperref[advanced-arguments]{Advanced
arguments}}{Advanced arguments}}\label{advanced-arguments}

\subsection{\texorpdfstring{\hyperref[spreading-arguments-from-list]{Spreading
arguments from
list}}{Spreading arguments from list}}\label{spreading-arguments-from-list}

Spreading operator allows you to "unpack" the list of values into
arguments of function:

\begin{verbatim}
#let func(a, b, c, d, e) = [#a #b #c #d #e]
#func(..(([hi],) * 5))
\end{verbatim}

\pandocbounded{\includesvg[keepaspectratio]{typst-img/0586f1f7eb73effd507824b57f7282f12fe2612119d64413f72e6518aba01513-1.svg}}

This may be super useful in tables:

\begin{verbatim}
#let a = ("hi", "b", "c")

#table(columns: 3,
  [test], [x], [hello],
  ..a
)
\end{verbatim}

\pandocbounded{\includesvg[keepaspectratio]{typst-img/eb669f70df63815adcbe764fdb8635eecab33651c7eef55ea4de6cd63c96d9de-1.svg}}

\subsection{\texorpdfstring{\hyperref[key-arguments]{Key
arguments}}{Key arguments}}\label{key-arguments}

The same idea works with key arguments:

\begin{verbatim}
#let text-params = (fill: blue, size: 0.8em)

Some #text(..text-params)[text].
\end{verbatim}

\pandocbounded{\includesvg[keepaspectratio]{typst-img/e56483e8f4285f8fed8cd6867e720b9a1c9f62ef0bffea28d124159f8a61648d-1.svg}}

\section{\texorpdfstring{\hyperref[managing-arbitrary-arguments]{Managing
arbitrary
arguments}}{Managing arbitrary arguments}}\label{managing-arbitrary-arguments}

Typst allows taking as many arbitrary positional and key arguments as
you want.

In that case function is given special
\texttt{\ }{\texttt{\ arguments\ }}\texttt{\ } object that stores in it
positional and named arguments.

\begin{quote}
Link to
\href{https://typst.app/docs/reference/foundations/arguments/}{reference}
\end{quote}

\begin{verbatim}
#let f(..args) = [
  #args.pos()\
  #args.named()
]

#f(1, "a", width: 50%, block: false)
\end{verbatim}

\pandocbounded{\includesvg[keepaspectratio]{typst-img/2fc64c8521734ea689368ec83fe54025eb94b016a8ed1f6d6a9880ac6c94edf5-1.svg}}

You can combine them with other arguments. Spreading operator will "eat"
all remaining arguments:

\begin{verbatim}
#let format(title, ..authors) = {
  let by = authors
    .pos()
    .join(", ", last: " and ")

  [*#title* \ _Written by #by;_]
}

#format("ArtosFlow", "Jane", "Joe")
\end{verbatim}

\pandocbounded{\includesvg[keepaspectratio]{typst-img/4ba76c5176e0b93c6c2b03c38d55f88702546a5183717ed8c3567865c0d1bf5d-1.svg}}

\subsection{\texorpdfstring{\hyperref[optional-argument]{Optional
argument}}{Optional argument}}\label{optional-argument}

\emph{Currently the only way in Typst to create optional positional
arguments is using \texttt{\ }{\texttt{\ arguments\ }}\texttt{\ }
object:}

TODO


\section{Examples Book LaTeX/book/basics/scripting/types.tex}
\title{sitandr.github.io/typst-examples-book/book/basics/scripting/types}

\section{\texorpdfstring{\hyperref[types-part-i]{Types, part
I}}{Types, part I}}\label{types-part-i}

Each value in Typst has a type. You don\textquotesingle t have to
specify it, but it is important.

\subsection{\texorpdfstring{\hyperref[content-content]{Content (
\texttt{\ }{\texttt{\ content\ }}\texttt{\ }
)}}{Content (   content   )}}\label{content-content}

\begin{quote}
\href{https://typst.app/docs/reference/foundations/content/}{Link to
Reference} .
\end{quote}

We have already seen it. A type that represents what is displayed in
document.

\begin{verbatim}
#let c = [It is _content_!]

// Check type of c
#(type(c) == content)

#c

// repr gives an "inner representation" of value
#repr(c)
\end{verbatim}

\pandocbounded{\includesvg[keepaspectratio]{typst-img/21fd80460de8e8a377a9ef2046a27232ad88924070509ccf8647c9135c9c2fe3-1.svg}}

\textbf{Important:} It is very hard to convert \emph{content} to
\emph{plain text} , as \emph{content} may contain \emph{anything} ! So
be careful when passing and storing content in variables.

\subsection{\texorpdfstring{\hyperref[none-none]{None (
\texttt{\ }{\texttt{\ none\ }}\texttt{\ }
)}}{None (   none   )}}\label{none-none}

Nothing. Also known as \texttt{\ }{\texttt{\ null\ }}\texttt{\ } in
other languages. It isn\textquotesingle t displayed, converts to empty
content.

\begin{verbatim}
#none
#repr(none)
\end{verbatim}

\pandocbounded{\includesvg[keepaspectratio]{typst-img/c4100c1d1df8fc0a51bd99945d9bac3c5aa67de19b8f872fd33fd9068bb2507b-1.svg}}

\subsection{\texorpdfstring{\hyperref[string-str]{String (
\texttt{\ }{\texttt{\ str\ }}\texttt{\ }
)}}{String (   str   )}}\label{string-str}

\begin{quote}
\href{https://typst.app/docs/reference/foundations/str/}{Link to
Reference} .
\end{quote}

String contains only plain text and no formatting. Just some chars. That
allows us to work with chars:

\begin{verbatim}
#let s = "Some large string. There could be escape sentences: \n,
 line breaks, and even unicode codes: \u{1251}"
#s \
#type(s) \
`repr`: #repr(s)

#let s = "another small string"
#s.replace("a", sym.alpha) \
#s.split(" ") // split by space
\end{verbatim}

\pandocbounded{\includesvg[keepaspectratio]{typst-img/b797f9c4a540fcf1429bec801d0b334e7d88dc9ccd10e3b7b859f451e269f30f-1.svg}}

You can convert other types to their string representation using this
type\textquotesingle s constructor (e.g. convert number to string):

\begin{verbatim}
#str(5) // string, can be worked with as string
\end{verbatim}

\pandocbounded{\includesvg[keepaspectratio]{typst-img/ab4d4a5d93533525f7f9b2cc8378b79f1561904f3c5d5f6d2ec4bdc448669cb5-1.svg}}

\subsection{\texorpdfstring{\hyperref[boolean-bool]{Boolean (
\texttt{\ }{\texttt{\ bool\ }}\texttt{\ }
)}}{Boolean (   bool   )}}\label{boolean-bool}

\begin{quote}
\href{https://typst.app/docs/reference/foundations/bool/}{Link to
Reference} .
\end{quote}

true/false. Used in \texttt{\ }{\texttt{\ if\ }}\texttt{\ } and many
others

\begin{verbatim}
#let b = false
#b \
#repr(b) \
#(true and not true or true) = #((true and (not true)) or true) \
#if (4 > 3) {
  "4 is more than 3"
}
\end{verbatim}

\pandocbounded{\includesvg[keepaspectratio]{typst-img/e848d78e220ca8cf3b6c323a99d5d963e529aad36857f0e6753c56c02984a616-1.svg}}

\subsection{\texorpdfstring{\hyperref[integer-int]{Integer (
\texttt{\ }{\texttt{\ int\ }}\texttt{\ }
)}}{Integer (   int   )}}\label{integer-int}

\begin{quote}
\href{https://typst.app/docs/reference/foundations/int/}{Link to
Reference} .
\end{quote}

A whole number.

The number can also be specified as hexadecimal, octal, or binary by
starting it with a zero followed by either x, o, or b.

\begin{verbatim}
#let n = 5
#n \
#(n += 1) \
#n \
#calc.pow(2, n) \
#type(n) \
#repr(n)
\end{verbatim}

\pandocbounded{\includesvg[keepaspectratio]{typst-img/6f1c9e02393e14aa23add33d0e6dc2b596ee97a0d425cd3edb3e2b912c6ef6b0-1.svg}}

\begin{verbatim}
#(1 + 2) \
#(2 - 5) \
#(3 + 4 < 8)
\end{verbatim}

\pandocbounded{\includesvg[keepaspectratio]{typst-img/e610f15659cb6b64c3516be48740b54e6caf3d933919004157ba64b757389ba5-1.svg}}

\begin{verbatim}
#0xff \
#0o10 \
#0b1001
\end{verbatim}

\pandocbounded{\includesvg[keepaspectratio]{typst-img/1446dba05ee6f8006884c280ff32e31ede8425d4847445e97cae5dfcde1efe7f-1.svg}}

You can convert a value to an integer with this type\textquotesingle s
constructor (e.g. convert string to int).

\begin{verbatim}
#int(false) \
#int(true) \
#int(2.7) \
#(int("27") + int("4"))
\end{verbatim}

\pandocbounded{\includesvg[keepaspectratio]{typst-img/b44779a87fd984d317ec4d1aed732c0ebdc6220fd4764e407f77fedd139c0d8c-1.svg}}

\subsection{\texorpdfstring{\hyperref[float-float]{Float (
\texttt{\ }{\texttt{\ float\ }}\texttt{\ }
)}}{Float (   float   )}}\label{float-float}

\begin{quote}
\href{https://typst.app/docs/reference/foundations/float/}{Link to
Reference} .
\end{quote}

Works the same way as integer, but can store floating point numbers.
However, precision may be lost.

\begin{verbatim}
#let n = 5.0

// You can mix floats and integers,
// they will be implicitly converted
#(n += 1) \
#calc.pow(2, n) \
#(0.2 + 0.1) \
#type(n)
\end{verbatim}

\pandocbounded{\includesvg[keepaspectratio]{typst-img/21cafe751ec803dd9598c871b283a29bc3c6b2e302f0f9bd78edc17330b45616-1.svg}}

\begin{verbatim}
#3.14 \
#1e4 \
#(10 / 4)
\end{verbatim}

\pandocbounded{\includesvg[keepaspectratio]{typst-img/05bd400096c1df5a954fda0897f3c1756c9f99f73503d32d992b3222667a45cd-1.svg}}

You can convert a value to a float with this type\textquotesingle s
constructor (e.g. convert string to float).

\begin{verbatim}
#float(40%) \
#float("2.7") \
#float("1e5")
\end{verbatim}

\pandocbounded{\includesvg[keepaspectratio]{typst-img/f50a22cbea42fded97ab8340f0939e786e5c1cdb5ea531cd4b35b1f732947b7f-1.svg}}


\section{Examples Book LaTeX/book/basics/scripting/tips.tex}
\title{sitandr.github.io/typst-examples-book/book/basics/scripting/tips}

\section{\texorpdfstring{\hyperref[tips]{Tips}}{Tips}}\label{tips}

There are lots of elements in Typst scripting that are not obvious, but
important. All the book is designated to show them, but some of them

\subsection{\texorpdfstring{\hyperref[equality]{Equality}}{Equality}}\label{equality}

Equality doesn\textquotesingle t mean objects are really the same, like
in many other objects:

\begin{verbatim}
#let a = 7
#let b = 7.0
#(a == b)
#(type(a) == type(b))
\end{verbatim}

\pandocbounded{\includesvg[keepaspectratio]{typst-img/3632e0202f7aae6ed6e2958b7bc6360a6cba31aa3d1aaf169a133ef987c839de-1.svg}}

That may be less obvious for dictionaries. In dictionaries \textbf{the
order may matter} , so equality doesn\textquotesingle t mean they behave
exactly the same way:

\begin{verbatim}
#let a = (x: 1, y: 2)
#let b = (y: 2, x: 1)
#(a == b)
#(a.pairs() == b.pairs())
\end{verbatim}

\pandocbounded{\includesvg[keepaspectratio]{typst-img/f7277d7cc170d7cc2ae1de5436b534fb113cda82d8e7829a0fc92e950b78238f-1.svg}}

\subsection{\texorpdfstring{\hyperref[check-key-is-in-dictionary]{Check
key is in
dictionary}}{Check key is in dictionary}}\label{check-key-is-in-dictionary}

Use the keyword \texttt{\ }{\texttt{\ in\ }}\texttt{\ } , like in
\texttt{\ }{\texttt{\ Python\ }}\texttt{\ } :

\begin{verbatim}
#let dict = (a: 1, b: 2)

#("a" in dict)
// gives the same as
#(dict.keys().contains("a"))
\end{verbatim}

\pandocbounded{\includesvg[keepaspectratio]{typst-img/c4ae77418e54911af371f203d2bd3d5badb7269496bb8f07a2e3010e15f18922-1.svg}}

Note it works for lists too:

\begin{verbatim}
#("a" in ("b", "c", "a"))
#(("b", "c", "a").contains("a"))
\end{verbatim}

\pandocbounded{\includesvg[keepaspectratio]{typst-img/0fc3ff7d44bbb5bcacd38e921f199699d2ea43ce0a142e79f67314d4f24386a7-1.svg}}


\section{Examples Book LaTeX/book/basics/scripting/braces.tex}
\title{sitandr.github.io/typst-examples-book/book/basics/scripting/braces}

\section{\texorpdfstring{\hyperref[braces-brackets-and-default]{Braces,
brackets and
default}}{Braces, brackets and default}}\label{braces-brackets-and-default}

\subsection{\texorpdfstring{\hyperref[square-brackets]{Square
brackets}}{Square brackets}}\label{square-brackets}

You may remember that square brackets convert everything inside to
\emph{content} .

\begin{verbatim}
#let v = [Some text, _markup_ and other #strong[functions]]
#v
\end{verbatim}

\pandocbounded{\includesvg[keepaspectratio]{typst-img/5ba617daa8d4c166d96a0abbba02d6502fe7fde1ded460afa78682993295142d-1.svg}}

We may use same for functions bodies:

\begin{verbatim}
#let f(name) = [Hello, #name]
#f[World] // also don't forget we can use it to pass content!
\end{verbatim}

\pandocbounded{\includesvg[keepaspectratio]{typst-img/4545deeee45655ee6666feb4773416cd075fe7522cbfd80d0847c615c6c5f30a-1.svg}}

\textbf{Important:} It is very hard to convert \emph{content} to
\emph{plain text} , as \emph{content} may contain \emph{anything} ! Sp
be careful when passing and storing content in variables.

\subsection{\texorpdfstring{\hyperref[braces]{Braces}}{Braces}}\label{braces}

However, we often want to use code inside functions.
That\textquotesingle s when we use
\texttt{\ }{\texttt{\ \{\}\ }}\texttt{\ } :

\begin{verbatim}
#let f(name) = {
  // this is code mode

  // First part of our output
  "Hello, "

  // we check if name is empty, and if it is,
  // insert placeholder
  if name == "" {
      "anonym"
  } else {
      name
  }

  // finish sentence
  "!"
}

#f("")
#f("Joe")
#f("world")
\end{verbatim}

\pandocbounded{\includesvg[keepaspectratio]{typst-img/f2bc6aebef06f213c9a8e740266a96e424318d953c09cffba6c5811375e91395-1.svg}}

\subsection{\texorpdfstring{\hyperref[scopes]{Scopes}}{Scopes}}\label{scopes}

\textbf{This is a very important thing to remember} .

\emph{You can\textquotesingle t use variables outside of scopes they are
defined (unless it is file root, then you can import them)} . \emph{Set
and show rules affect things in their scope only.}

\begin{verbatim}
#{
  let a = 3;
}
// can't use "a" there.

#[
  #show "true": "false"

  This is true.
]

This is true.
\end{verbatim}

\pandocbounded{\includesvg[keepaspectratio]{typst-img/c25d356831eeea19bb243b87c0f32d062c7086a55b4ee432e41b388d626f875b-1.svg}}

\subsection{\texorpdfstring{\hyperref[return]{Return}}{Return}}\label{return}

\textbf{Important} : by default braces return anything that "returns"
into them. For example,

\begin{verbatim}
#let change_world() = {
  // some code there changing everything in the world
  str(4e7)
  // another code changing the world
}

#let g() = {
  "Hahaha, I will change the world now! "
  change_world()
  " So here is my long evil monologue..."
}

#g()
\end{verbatim}

\pandocbounded{\includesvg[keepaspectratio]{typst-img/160d9672bd7abc64ea61943d1bfcbd1b06dc70f87be5e5cf9c411fe4ee6d2a44-1.svg}}

To avoid returning everything, return only what you want explicitly,
otherwise everything will be joined:

\begin{verbatim}
#let f() = {
  "Some long text"
  // Crazy numbers
  "2e7"
  return none
}

// Returns nothing
#f()
\end{verbatim}

\pandocbounded{\includesvg[keepaspectratio]{typst-img/14c935733a8c91165ee4ebe8246efb841207feeaa0309e36a1cde2888acffb10-1.svg}}

\subsection{\texorpdfstring{\hyperref[default-values]{Default
values}}{Default values}}\label{default-values}

What we made just now was inventing "default values".

They are very common in styling, so there is a special syntax for them:

\begin{verbatim}
#let f(name: "anonym") = [Hello, #name!]

#f()
#f(name: "Joe")
#f(name: "world")
\end{verbatim}

\pandocbounded{\includesvg[keepaspectratio]{typst-img/e9730d0d1f30ec9f2404179560ae4a4b19dd788b1afc2f6b956fb32268439cb6-1.svg}}

You may have noticed that the argument became \emph{named} now. In
Typst, named argument is an argument \emph{that has default value} .




\section{Combined Examples Book LaTeX/book/basics/basics.tex}
\section{Examples Book LaTeX/book/basics/extra.tex}
\title{sitandr.github.io/typst-examples-book/book/basics/extra}

\section{\texorpdfstring{\hyperref[extra]{Extra}}{Extra}}\label{extra}

\subsection{\texorpdfstring{\hyperref[bibliography]{Bibliography}}{Bibliography}}\label{bibliography}

Typst supports bibliography using BibLaTex
\texttt{\ }{\texttt{\ .bib\ }}\texttt{\ } file or its own Hayagriva
\texttt{\ }{\texttt{\ .yml\ }}\texttt{\ } format.

BibLaTex is wider supported, but Hayagriva is easier to work with.

\begin{quote}
Link to Hayagriva
\href{https://github.com/typst/hayagriva/blob/main/docs/file-format.md}{documentation}
and some
\href{https://github.com/typst/hayagriva/blob/main/tests/data/basic.yml}{examples}
.
\end{quote}

\subsubsection{\texorpdfstring{\hyperref[citation-style]{Citation
Style}}{Citation Style}}\label{citation-style}

The style can be customized via CSL, citation style language, with more
than 10 000 styles available online. See
\href{https://github.com/citation-style-language/styles}{official
repository} .


\section{Examples Book LaTeX/book/basics/index.tex}
\title{sitandr.github.io/typst-examples-book/book/basics/index}

\section{\texorpdfstring{\hyperref[typst-basics]{Typst
Basics}}{Typst Basics}}\label{typst-basics}

This is a chapter that consistently introduces you to the most things
you need to know when writing with Typst.

It show much more things than official tutorial, so maybe it will be
interesting to read for some of the experienced users too.

Some examples are taken from
\href{https://typst.app/docs/tutorial/}{Official Tutorial} and
\href{https://typst.app/docs/reference/}{Official Reference} . Most are
created and edited specially for this book.

\begin{quote}
\emph{Important:} in most cases there will be used "clipped" examples of
your rendered documents (no margins, smaller width and so on).

To set up the spacing as you want, see
\href{https://typst.app/docs/guides/page-setup-guide/}{Official Page
Setup Guide} .
\end{quote}


\section{Examples Book LaTeX/book/basics/special_symbols.tex}
\title{sitandr.github.io/typst-examples-book/book/basics/special_symbols}

\section{\texorpdfstring{\hyperref[special-symbols]{Special
symbols}}{Special symbols}}\label{special-symbols}

\begin{quote}
\emph{Important:} I\textquotesingle m not great with special symbols, so
I would additionally appreciate additions and corrections.
\end{quote}

Typst has a great support of \emph{unicode} . That also means it
supports \emph{special symbols} . They may be very useful for
typesetting.

In most cases, you shouldn\textquotesingle t use these symbols directly
often. If possible, use them with show rules (for example, replace all
\texttt{\ }{\texttt{\ -th\ }}\texttt{\ } with
\texttt{\ }{\texttt{\ \textbackslash{}u\ }}\texttt{\ }{\texttt{\ \{2011\}th\ }}\texttt{\ }
, a non-breaking hyphen).

\subsection{\texorpdfstring{\hyperref[non-breaking-symbols]{Non-breaking
symbols}}{Non-breaking symbols}}\label{non-breaking-symbols}

Non-breaking symbols can make sure the word/phrase will not be
separated. Typst will try to put them as a whole.

\subsubsection{\texorpdfstring{\hyperref[non-breaking-space]{Non-breaking
space}}{Non-breaking space}}\label{non-breaking-space}

\begin{quote}
\emph{Important:} As it is spacing symbols, copy-pasting it will not
help. Typst will see it as just a usual spacing symbol you used for your
source code to look nicer in your editor. Again, it will interpret it
\emph{as a basic space} .
\end{quote}

This is a symbol you should\textquotesingle t use often (use Typst boxes
instead), but it is a good demonstration of how non-breaking symbol
work:

\begin{verbatim}
#set page(width: 9em)

// Cruel and world are separated.
// Imagine this is a phrase that can't be split, what to do then?
Hello cruel world

// Let's connect them with a special space!

// No usual spacing is allowed, so either use semicolumn...
Hello cruel#sym.space.nobreak;world

// ...parentheses...
Hello cruel#(sym.space.nobreak)world

// ...or unicode code
Hello cruel\u{00a0}world

// Well, to achieve the same effect I recommend using box:
Hello #box[cruel world]
\end{verbatim}

\pandocbounded{\includesvg[keepaspectratio]{typst-img/be9e5cddfdd58a5f21a2b17e32227ac0c96e2d6eeffe764ef2809257aa416c59-1.svg}}

\subsubsection{\texorpdfstring{\hyperref[non-breaking-hyphen]{Non-breaking
hyphen}}{Non-breaking hyphen}}\label{non-breaking-hyphen}

\begin{verbatim}
#set page(width: 8em)

This is an $i$-th element.

This is an $i$\u{2011}th element.

// the best way would be
#show "-th": "\u{2011}th"

This is an $i$-th element.
\end{verbatim}

\pandocbounded{\includesvg[keepaspectratio]{typst-img/02baa9a61778ef23389d4ceb2fae4d2ac699d72b127b447ca6f25037096d2df9-1.svg}}

\subsection{\texorpdfstring{\hyperref[connectors-and-separators]{Connectors
and
separators}}{Connectors and separators}}\label{connectors-and-separators}

\subsubsection{\texorpdfstring{\hyperref[world-joiner]{World
joiner}}{World joiner}}\label{world-joiner}

Initially, world joiner indicates that no line break should occur at
this position. It is also a zero-width symbol (invisible), so it can be
used as a space removing thing:

\begin{verbatim}
#set page(width: 9em)
#set text(hyphenate: true)

Thisisawordthathastobreak

// Be careful, there is no line break at all now!
Thisi#sym.wj;sawordthathastobreak

// code from `physica` package
// word joiner here is used to avoid extra spacing
#let just-hbar = move(dy: -0.08em, strike(offset: -0.55em, extent: -0.05em, sym.planck))
#let hbar = (sym.wj, just-hbar, sym.wj).join()

$ a #just-hbar b, a hbar b$
\end{verbatim}

\pandocbounded{\includesvg[keepaspectratio]{typst-img/7df9031646c932030adb0fc5a97446e7560ca7d353ef935d4034dc0a4b8be5c1-1.svg}}

\subsubsection{\texorpdfstring{\hyperref[zero-width-space]{Zero width
space}}{Zero width space}}\label{zero-width-space}

Similar to word-joiner, but this is a \emph{space} . It
doesn\textquotesingle t prevent word break. On the contrary, it breaks
it without any hyphen at all!

\begin{verbatim}
#set page(width: 9em)
#set text(hyphenate: true)

// There is a space inside!
Thisisa#sym.zws;word

// Be careful, there is no hyphen at all now!
Thisisawo#sym.zws;rdthathastobreak
\end{verbatim}

\pandocbounded{\includesvg[keepaspectratio]{typst-img/7fd917d4e0422bc1bb72d451b6da6e38fb9fe28cd28152ab60bdfb7ad5d1cab1-1.svg}}


\section{Examples Book LaTeX/book/basics/measure.tex}
\title{sitandr.github.io/typst-examples-book/book/basics/measure}

\section{\texorpdfstring{\hyperref[measure-layout]{Measure,
Layout}}{Measure, Layout}}\label{measure-layout}

This section is outdated. It may be still useful, but it is strongly
recommended to study new context system (using the reference).

\subsection{\texorpdfstring{\hyperref[style--measure]{Style \&
Measure}}{Style \& Measure}}\label{style--measure}

\begin{quote}
Style
\href{https://typst.app/docs/reference/foundations/style/}{documentation}
.
\end{quote}

\begin{quote}
Measure
\href{https://typst.app/docs/reference/layout/measure/}{documentation} .
\end{quote}

\texttt{\ }{\texttt{\ measure\ }}\texttt{\ } returns \emph{the element
size} . This command is extremely helpful when doing custom layout with
\texttt{\ }{\texttt{\ place\ }}\texttt{\ } .

However, there is a catch. Element size depends on styles, applied to
this element.

\begin{verbatim}
#let content = [Hello!]
#content
#set text(14pt)
#content
\end{verbatim}

\pandocbounded{\includesvg[keepaspectratio]{typst-img/00a6cbbc650947c03f34564786b0645eee60396f288d26333c591ff9059cc369-1.svg}}

So if we will set the big text size for some part of our text, to
measure the element\textquotesingle s size, we have to know \emph{where
the element is located} . Without knowing it, we can\textquotesingle t
tell what styles should be applied.

So we need a scheme similar to
\texttt{\ }{\texttt{\ locate\ }}\texttt{\ } .

This is what \texttt{\ }{\texttt{\ styles\ }}\texttt{\ } function is
used for. It is \emph{a content} , which, when located in document,
calls a function inside on \emph{current styles} .

Now, when we got fixed \texttt{\ }{\texttt{\ styles\ }}\texttt{\ } , we
can get the element\textquotesingle s size using
\texttt{\ }{\texttt{\ measure\ }}\texttt{\ } :

\begin{verbatim}
#let thing(body) = style(styles => {
  let size = measure(body, styles)
  [Width of "#body" is #size.width]
})

#thing[Hey] \
#thing[Welcome]
\end{verbatim}

\pandocbounded{\includesvg[keepaspectratio]{typst-img/5afe1855072b4ee8e343e5b5aa79affae5b17bc89738ffbe93dac245576cdd04-1.svg}}

\section{\texorpdfstring{\hyperref[layout]{Layout}}{Layout}}\label{layout}

Layout is similar to \texttt{\ }{\texttt{\ measure\ }}\texttt{\ } , but
it returns current scope \textbf{parent size} .

If you are putting elements in block, that will be
block\textquotesingle s size. If you are just putting right on the page,
that will be page\textquotesingle s size.

As parent\textquotesingle s size depends on it\textquotesingle s place
in document, it uses the similar scheme to
\texttt{\ }{\texttt{\ locate\ }}\texttt{\ } and
\texttt{\ }{\texttt{\ style\ }}\texttt{\ } :

\begin{verbatim}
#layout(size => {
  let half = 50% * size.width
  [Half a page is #half wide.]
})
\end{verbatim}

\pandocbounded{\includesvg[keepaspectratio]{typst-img/c68a166f6e6b1b3229fd56478ae302dbeb39c882e229c69d4c6ebb6c9c528985-1.svg}}

It may be extremely useful to combine
\texttt{\ }{\texttt{\ layout\ }}\texttt{\ } with
\texttt{\ }{\texttt{\ measure\ }}\texttt{\ } , to get width of things
that depend on parent\textquotesingle s size:

\begin{verbatim}
#let text = lorem(30)
#layout(size => style(styles => [
  #let (height,) = measure(
    block(width: size.width, text),
    styles,
  )
  This text is #height high with
  the current page width: \
  #text
]))
\end{verbatim}

\pandocbounded{\includesvg[keepaspectratio]{typst-img/93167dc0b22b02fe27aa92c6b03c2281665b4352624364a19c63f61a488aa75a-1.svg}}




\section{Combined Examples Book LaTeX/book/basics/math.tex}
\section{Examples Book LaTeX/book/basics/math/symbols.tex}
\title{sitandr.github.io/typst-examples-book/book/basics/math/symbols}

\section{\texorpdfstring{\hyperref[symbols]{Symbols}}{Symbols}}\label{symbols}

Multiletter words in math refer either to local variables, functions,
text operators, spacing or \emph{special symbols} . The latter are very
important for advanced math.

\begin{verbatim}
$
forall v, w in V, alpha in KK: alpha dot (v + w) = alpha v + alpha w
$
\end{verbatim}

\pandocbounded{\includesvg[keepaspectratio]{typst-img/60a6e3e08582c87ec082b6714a45a90a914dd1299f788e2bb21b0cc5adc80e6a-1.svg}}

You can write the same with unicode:

\begin{verbatim}
$
∀ v, w ∈ V, α ∈ 𝕂: α ⋅ (v + w) = α v + α w
$
\end{verbatim}

\pandocbounded{\includesvg[keepaspectratio]{typst-img/d37776c21d5c4d692e4ebbe7e5ce7e7cdf5e2c0777a88a47abe0c0c5992cf41a-1.svg}}

\subsection{\texorpdfstring{\hyperref[symbols-naming]{Symbols
naming}}{Symbols naming}}\label{symbols-naming}

\begin{quote}
See all available symbols list
\href{https://typst.app/docs/reference/symbols/sym/}{there} .
\end{quote}

\subsubsection{\texorpdfstring{\hyperref[general-idea]{General
idea}}{General idea}}\label{general-idea}

Typst wants to define some "basic" symbols with small easy-to-remember
words, and build complex ones using combinations. For example,

\begin{verbatim}
$
// cont — contour
integral, integral.cont, integral.double, integral.square, sum.integral\

// lt — less than, gt — greater than
lt, lt.circle, lt.eq, lt.not, lt.eq.not, lt.tri, lt.tri.eq, lt.tri.eq.not, gt, lt.gt.eq, lt.gt.not
$
\end{verbatim}

\pandocbounded{\includesvg[keepaspectratio]{typst-img/a0ee196d2bf305ca6c2d812008f9955e5ae526de0b0ac0b83ca016a66bdc00f1-1.svg}}

I highly recommend using WebApp/Typst LSP when writing math with lots of
complex symbols. That helps you to quickly choose the right symbol
within all combinations.

Sometimes the names are not obvious, for example, sometimes it is used
prefix \texttt{\ }{\texttt{\ n-\ }}\texttt{\ } instead of
\texttt{\ }{\texttt{\ not\ }}\texttt{\ } :

\begin{verbatim}
$
gt.nequiv, gt.napprox, gt.ntilde, gt.tilde.not
$
\end{verbatim}

\pandocbounded{\includesvg[keepaspectratio]{typst-img/e4d0ef024efaf9f4334ebf04a2ac4e015fc5ec76617be8b6d7aad2f4429e3317-1.svg}}

\subsubsection{\texorpdfstring{\hyperref[common-modifiers]{Common
modifiers}}{Common modifiers}}\label{common-modifiers}

\begin{itemize}
\item
  \texttt{\ }{\texttt{\ .b,\ .t,\ .l,\ .r\ }}\texttt{\ } : bottom, top,
  left, right. Change direction of symbol.

\begin{verbatim}
$arrow.b, triangle.r, angle.l$
\end{verbatim}

  \pandocbounded{\includesvg[keepaspectratio]{typst-img/8ab0fa590b7a39023b1467e7a376a4810f997f720dd5d221ad83d7e741943b55-1.svg}}
\end{itemize}


\section{Examples Book LaTeX/book/basics/math/grouping.tex}
\title{sitandr.github.io/typst-examples-book/book/basics/math/grouping}

\section{\texorpdfstring{\hyperref[grouping]{Grouping}}{Grouping}}\label{grouping}

Every grouping can be (currently) done by parenthesis. So the
parenthesis may be both "real" parenthesis and grouping ones.

For example, these parentheses specify nominator of the fraction:

\begin{verbatim}
$ (a^2 + b^2)/2 $
\end{verbatim}

\pandocbounded{\includesvg[keepaspectratio]{typst-img/6f4767b2aee69b5c3a22df5f394105df9f19c9762678d02b297c4d4f8d1cf6ad-1.svg}}

\subsection{\texorpdfstring{\hyperref[left-right]{Left-right}}{Left-right}}\label{left-right}

\begin{quote}
See \href{https://typst.app/docs/reference/math/lr}{official
documentation} .
\end{quote}

If there are two matching braces of any kind, they will be wrapped as
\texttt{\ }{\texttt{\ lr\ }}\texttt{\ } (left-right) group.

\begin{verbatim}
$
{[((a + b)/2) + 1]_0}
$
\end{verbatim}

\pandocbounded{\includesvg[keepaspectratio]{typst-img/a4137ff5d1f577cc816776cb4279cce4cd964601c20eb244d12e170deecd5d6a-1.svg}}

You can disable it by escaping.

You can also match braces of any kind by using
\texttt{\ }{\texttt{\ lr\ }}\texttt{\ } directly:

\begin{verbatim}
$
lr([a/2, b)) \
lr([a/2, b), size: #150%)
$
\end{verbatim}

\pandocbounded{\includesvg[keepaspectratio]{typst-img/fb81420a901d8b570ef03d1f50c83f7b8c483c9366222156ea193ac2976b63ed-1.svg}}

\subsection{\texorpdfstring{\hyperref[fences]{Fences}}{Fences}}\label{fences}

Fences \emph{are not matched automatically} because of large amount of
false-positives.

You can use \texttt{\ }{\texttt{\ abs\ }}\texttt{\ } or
\texttt{\ }{\texttt{\ norm\ }}\texttt{\ } to match them:

\begin{verbatim}
$
abs(a + b), norm(a + b), floor(a + b), ceil(a + b), round(a + b)
$
\end{verbatim}

\pandocbounded{\includesvg[keepaspectratio]{typst-img/fd8454b2a97d649525827367f459f3163d830b5db9181178304d5fd2b44fcca1-1.svg}}


\section{Examples Book LaTeX/book/basics/math/classes.tex}
\title{sitandr.github.io/typst-examples-book/book/basics/math/classes}

\section{\texorpdfstring{\hyperref[classes]{Classes}}{Classes}}\label{classes}

\begin{quote}
See \href{https://typst.app/docs/reference/math/class/}{official
documentation}
\end{quote}

Each math symbol has its own "class", the way it behaves.
That\textquotesingle s one of the main reasons why they are layouted
differently.

\subsection{\texorpdfstring{\hyperref[classes-1]{Classes}}{Classes}}\label{classes-1}

\begin{verbatim}
$
a b c\
a class("normal", b) c\
a class("punctuation", b) c\
a class("opening", b) c\
a lr(b c]) c\
a lr(class("opening", b) c ]) c // notice it is moved vertically \
a class("closing", b) c\
a class("fence", b) c\
a class("large", b) c\
a class("relation", b) c\
a class("unary", b) c\
a class("binary", b) c\
a class("vary", b) c\
$
\end{verbatim}

\pandocbounded{\includesvg[keepaspectratio]{typst-img/5d4604274229b2f53ee04b88ff0e73d9aa8365643c5e60052fcca1298d4f5a23-1.svg}}

\subsection{\texorpdfstring{\hyperref[setting-class-for-symbol]{Setting
class for
symbol}}{Setting class for symbol}}\label{setting-class-for-symbol}

\begin{verbatim}
Default:

$square circle square$

With `#h(0)`:

$square #h(0pt) circle #h(0pt) square$

With `math.class`:

#show math.circle: math.class.with("normal")
$square circle square$
\end{verbatim}

\pandocbounded{\includesvg[keepaspectratio]{typst-img/86a709c6189649b79005752253a842631eed4722b350e4197116e0be19094035-1.svg}}


\section{Examples Book LaTeX/book/basics/math/index.tex}
\title{sitandr.github.io/typst-examples-book/book/basics/math/index}

\section{\texorpdfstring{\hyperref[math]{Math}}{Math}}\label{math}

Math is a special environment that has special features related to...
math.

\subsection{\texorpdfstring{\hyperref[syntax]{Syntax}}{Syntax}}\label{syntax}

To start math environment, \texttt{\ }{\texttt{\ \$\ }}\texttt{\ } . The
spacing around \texttt{\ }{\texttt{\ \$\ }}\texttt{\ } will make it
either \emph{inline} math (smaller, used in text) or \emph{display} math
(used on math equations on their own).

\begin{verbatim}
// This is inline math
Let $a$, $b$, and $c$ be the side
lengths of right-angled triangle.
Then, we know that:

// This is display math
$ a^2 + b^2 = c^2 $

Prove by induction:

// You can use new lines as spacing too!
$
sum_(k=1)^n k = (n(n+1)) / 2
$
\end{verbatim}

\pandocbounded{\includesvg[keepaspectratio]{typst-img/068db3a521a38c3acede771ebb6342807cca4fd98baf5b2b508184a6854ea8ff-1.svg}}

\subsection{\texorpdfstring{\hyperref[mathequation]{Math.equation}}{Math.equation}}\label{mathequation}

The element that math is displayed in is called
\texttt{\ }{\texttt{\ math.equation\ }}\texttt{\ } . You can use it for
set/show rules:

\begin{verbatim}
#show math.equation: set text(red)

$
integral_0^oo (f(t) + g(t))/2
$
\end{verbatim}

\pandocbounded{\includesvg[keepaspectratio]{typst-img/94e0532dd7224d08e966cb82834283efd8889d7f117b04116e721a788bfcc16c-1.svg}}

Any symbol/command that is available in math, \emph{is also available}
in code mode using \texttt{\ }{\texttt{\ math.command\ }}\texttt{\ } :

\begin{verbatim}
#math.integral, #math.underbrace([a + b], [c])
\end{verbatim}

\pandocbounded{\includesvg[keepaspectratio]{typst-img/b4ca12d7f34ed342f3cb3fba2ee1f5b58faa8fceecb74671baacd9166fcbb5aa-1.svg}}

\subsection{\texorpdfstring{\hyperref[letters-and-commands]{Letters and
commands}}{Letters and commands}}\label{letters-and-commands}

Typst aims to have as simple and effective syntax for math as possible.
That means no special symbols, just using commands.

To make it short, Typst uses several simple rules:

\begin{itemize}
\item
  All single-letter words \emph{turn into variables} . That includes any
  \emph{unicode symbols} too!
\item
  All multi-letter words \emph{turn into commands} . They may be
  built-in commands (available with math.something outside of math
  environment). Or they \textbf{may be user-defined variables/functions}
  . If the command \textbf{isn\textquotesingle t defined} , there will
  be \textbf{compilation error} .

  If you use kebab-case or snake\_case for variables you want to use in
  math, you will have to refer to them as \#snake-case-variable.
\item
  To write simple text, use quotes:

\begin{verbatim}
$a "equals to" 2$
\end{verbatim}

  \pandocbounded{\includesvg[keepaspectratio]{typst-img/811f30ede68d08bec254f184c1be319958c3e11f9f9d58c40b2f460bba037e3d-1.svg}}

  Spacing matters there!

\begin{verbatim}
$a "is" 2$, $a"is"2$
\end{verbatim}

  \pandocbounded{\includesvg[keepaspectratio]{typst-img/9cc2d263c76646c623e1e6b73756e1fe1e2c56d7fe0324ee945652107e6456ba-1.svg}}
\item
  You can turn it into multi-letter variables using
  \texttt{\ }{\texttt{\ italic\ }}\texttt{\ } :

\begin{verbatim}
$(italic("mass") v^2)/2$
\end{verbatim}

  \pandocbounded{\includesvg[keepaspectratio]{typst-img/141d8a3b9beb3559387411170f7378078c80cb2ff80d8d5f5345c3231f55df9c-1.svg}}
\end{itemize}

Commands see
\href{https://typst.app/docs/reference/math/\#definitions}{there} (go to
the links to see the commands).

All symbols see
\href{https://typst.app/docs/reference/symbols/sym/}{there} .

\subsection{\texorpdfstring{\hyperref[multiline-equations]{Multiline
equations}}{Multiline equations}}\label{multiline-equations}

To create multiline \emph{display equation} , use the same symbol as in
markup mode: \texttt{\ }{\texttt{\ \textbackslash{}\ }}\texttt{\ } :

\begin{verbatim}
$
a = b\
a = c
$
\end{verbatim}

\pandocbounded{\includesvg[keepaspectratio]{typst-img/2f16d9e64e38ff22ca27a09b0d8eaef1b020e4eccd7d2ce1380e10a0efcea163-1.svg}}

\subsection{\texorpdfstring{\hyperref[escaping]{Escaping}}{Escaping}}\label{escaping}

Any symbol that is used may be escaped with
\texttt{\ }{\texttt{\ \textbackslash{}\ }}\texttt{\ } , like in markup
mode. For example, you can disable fraction:

\begin{verbatim}
$
a  / b \
a \/ b
$
\end{verbatim}

\pandocbounded{\includesvg[keepaspectratio]{typst-img/e7931e55a2772ee992446af8506d8d25b96167e3bb71d5c63ed8ca156530f2d9-1.svg}}

The same way it works with any other syntax.

\subsection{\texorpdfstring{\hyperref[wrapping-inline-math]{Wrapping
inline math}}{Wrapping inline math}}\label{wrapping-inline-math}

Sometimes, when you write large math, it may be too close to text
(especially for some long letter tails).

\begin{verbatim}
#lorem(17) $display(1)/display(1+x^n)$ #lorem(20)
\end{verbatim}

\pandocbounded{\includesvg[keepaspectratio]{typst-img/a9cce2b851a01939a0abfc02e8cd994d20c465d2800cf64c5c6051ead5bc4e9a-1.svg}}

You may easily increase the distance it by wrapping into box:

\begin{verbatim}
#lorem(17) #box($display(1)/display(1+x^n)$, inset: 0.2em) #lorem(20)
\end{verbatim}

\pandocbounded{\includesvg[keepaspectratio]{typst-img/ee9fc5a3ec529a9f3e811a70724c1585c294d82454c22ee9343235556f572792-1.svg}}


\section{Examples Book LaTeX/book/basics/math/alignment.tex}
\title{sitandr.github.io/typst-examples-book/book/basics/math/alignment}

\section{\texorpdfstring{\hyperref[alignment]{Alignment}}{Alignment}}\label{alignment}

\subsection{\texorpdfstring{\hyperref[general-alignment]{General
alignment}}{General alignment}}\label{general-alignment}

By default display math is center-aligned, but that can be set up with
\texttt{\ }{\texttt{\ show\ }}\texttt{\ } rule:

\begin{verbatim}
#show math.equation: set align(right)

$
(a + b)/2
$
\end{verbatim}

\pandocbounded{\includesvg[keepaspectratio]{typst-img/bcd19808066d4eee09c984bf17077653b1c1bf25115c10a155611056a30e2cb6-1.svg}}

Or using \texttt{\ }{\texttt{\ align\ }}\texttt{\ } element:

\begin{verbatim}
#align(left, block($ x = 5 $))
\end{verbatim}

\pandocbounded{\includesvg[keepaspectratio]{typst-img/4545bd54c4090d4c9599e639aa441b68eb214011861d9949652df140843af042-1.svg}}

\subsection{\texorpdfstring{\hyperref[alignment-points]{Alignment
points}}{Alignment points}}\label{alignment-points}

When equations include multiple alignment points (\&), this creates
blocks of alternatingly \emph{right-} and \emph{left-} aligned columns.

In the example below, the expression
\texttt{\ }{\texttt{\ (3x\ +\ y)\ /\ 7\ }}\texttt{\ } is
\emph{right-aligned} and
\texttt{\ }{\texttt{\ =\ }}\texttt{\ }{\texttt{\ 9\ }}\texttt{\ } is
\emph{left-aligned} .

\begin{verbatim}
$ (3x + y) / 7 &= 9 && "given" \
  3x + y &= 63 & "multiply by 7" \
  3x &= 63 - y && "subtract y" \
  x &= 21 - y/3 & "divide by 3" $
\end{verbatim}

\pandocbounded{\includesvg[keepaspectratio]{typst-img/bfb7a5df8873923079f45d12fa92204afeddecb15ec31d6b8624ac4610d29677-1.svg}}

The word "given" is also left-aligned because
\texttt{\ }{\texttt{\ \&\&\ }}\texttt{\ } creates two alignment points
in a row, \emph{alternating the alignment twice} .

\texttt{\ }{\texttt{\ \&\ \&\ }}\texttt{\ } and
\texttt{\ }{\texttt{\ \&\&\ }}\texttt{\ } behave exactly the same way.
Meanwhile, "multiply by 7" is left-aligned because just one
\texttt{\ }{\texttt{\ \&\ }}\texttt{\ } precedes it.

\textbf{Each alignment point simply alternates between
right-aligned/left-aligned.}


\section{Examples Book LaTeX/book/basics/math/sizes.tex}
\title{sitandr.github.io/typst-examples-book/book/basics/math/sizes}

\section{\texorpdfstring{\hyperref[location-and-sizes]{Location and
sizes}}{Location and sizes}}\label{location-and-sizes}

We talked already about display and inline math. They differ not only by
aligning and spacing, but also by size and style:

\begin{verbatim}
Inline: $a/(b + 1/c), sum_(n=0)^3 x_n$

$
a/(b + 1/c), sum_(n=0)^3 x_n
$
\end{verbatim}

\pandocbounded{\includesvg[keepaspectratio]{typst-img/7de20fcaee4fb6ea523b34bfe9b2be6b91cc6e6a5b46fab0eebe7f0155689f8e-1.svg}}

The size and style of current environment is described by Math Size, see
\href{https://typst.app/docs/reference/math/sizes}{reference} .

There are for sizes:

\begin{itemize}
\tightlist
\item
  Display math size ( \texttt{\ }{\texttt{\ display\ }}\texttt{\ } )
\item
  Inline math size ( \texttt{\ }{\texttt{\ inline\ }}\texttt{\ } )
\item
  Script math size ( \texttt{\ }{\texttt{\ script\ }}\texttt{\ } )
\item
  Sub/super script math size (
  \texttt{\ }{\texttt{\ sscript\ }}\texttt{\ } )
\end{itemize}

Each time thing is used in fraction, script or exponent, it is moved
several "levels lowers", becoming smaller and more "crapping".
\texttt{\ }{\texttt{\ sscript\ }}\texttt{\ } isn\textquotesingle t
reduced father:

\begin{verbatim}
$
"display:" 1/("inline:" a + 1/("script:" b + 1/("sscript:" c + 1/("sscript:" d + 1/("sscript:" e + 1/f)))))
$
\end{verbatim}

\pandocbounded{\includesvg[keepaspectratio]{typst-img/9c8cbc46da7dc8eb9436c561107cbb97a836aaa7b120a9bc3f044dd648d702e1-1.svg}}

\subsection{\texorpdfstring{\hyperref[setting-sizes-manually]{Setting
sizes manually}}{Setting sizes manually}}\label{setting-sizes-manually}

Just use the corresponding command:

\begin{verbatim}
Inine: $sum_0^oo e^x^a$\
Inline with limits: $limits(sum)_0^oo e^x^a$\
Inline, but like true display: $display(sum_0^oo e^x^a)$
\end{verbatim}

\pandocbounded{\includesvg[keepaspectratio]{typst-img/0d16a9d157c9689f4b3cce434ebf89d9a18d67b4916ac0ebfbce3daccb94e709-1.svg}}


\section{Examples Book LaTeX/book/basics/math/vec.tex}
\title{sitandr.github.io/typst-examples-book/book/basics/math/vec}

\section{\texorpdfstring{\hyperref[vectors-matrices-semicolumn-syntax]{Vectors,
matrices, semicolumn
syntax}}{Vectors, matrices, semicolumn syntax}}\label{vectors-matrices-semicolumn-syntax}

\subsection{\texorpdfstring{\hyperref[vectors]{Vectors}}{Vectors}}\label{vectors}

\begin{quote}
By vector we mean a column there.\\
To write arrow notations for letters, use
\texttt{\ }{\texttt{\ \$\ }}\texttt{\ }{\texttt{\ arrow\ }}\texttt{\ }{\texttt{\ (\ }}\texttt{\ }{\texttt{\ v\ }}\texttt{\ }{\texttt{\ )\ }}\texttt{\ }{\texttt{\ \$\ }}\texttt{\ }\\
I recommend to create shortcut for this, like
\texttt{\ }{\texttt{\ \#let\ }}\texttt{\ }{\texttt{\ arr\ }}\texttt{\ }{\texttt{\ =\ }}\texttt{\ }{\texttt{\ math.arrow\ }}\texttt{\ }
\end{quote}

To write columns, use \texttt{\ }{\texttt{\ vec\ }}\texttt{\ } command:

\begin{verbatim}
$
vec(a, b, c) + vec(1, 2, 3) = vec(a + 1, b + 2, c + 3)
$
\end{verbatim}

\pandocbounded{\includesvg[keepaspectratio]{typst-img/92aa72b3d4f797123f550cc8630b34e09176956c4b116cc0a4fe48d457e1ee0a-1.svg}}

\subsubsection{\texorpdfstring{\hyperref[delimiter]{Delimiter}}{Delimiter}}\label{delimiter}

You can change parentheses around the column or even remove them:

\begin{verbatim}
$
vec(1, 2, 3, delim: "{") \
vec(1, 2, 3, delim: bar.double) \
vec(1, 2, 3, delim: #none)
$
\end{verbatim}

\pandocbounded{\includesvg[keepaspectratio]{typst-img/efd7cc6c6abb317c316b746f7a286ab2f8b2a023fe19bf77c15638db9c6bed8f-1.svg}}

\subsubsection{\texorpdfstring{\hyperref[gap]{Gap}}{Gap}}\label{gap}

You can change the size of gap between rows:

\begin{verbatim}
$
vec(a, b, c)
vec(a, b, c, gap:#0em)
vec(a, b, c, gap:#1em)
$
\end{verbatim}

\pandocbounded{\includesvg[keepaspectratio]{typst-img/8977ff36f1f7a4b78c2fdbaef8764fec4b2cb42092f63b07176cca13707c0407-1.svg}}

\subsubsection{\texorpdfstring{\hyperref[making-gap-even]{Making gap
even}}{Making gap even}}\label{making-gap-even}

You can easily note that the gap isn\textquotesingle t necessarily even
or the same in different vectors:

\begin{verbatim}
$
vec(a/b, a/b, a/b) = vec(1, 1, 1)
$
\end{verbatim}

\pandocbounded{\includesvg[keepaspectratio]{typst-img/c3141fb95a4280df589e5e9fc0d605d59b16a8da6b4a01be532fab0bf04f6a00-1.svg}}

That happens because \texttt{\ }{\texttt{\ gap\ }}\texttt{\ } refers to
\emph{spacing between} elements, not the distance between their centers.

To fix this, you can use \href{../../snippets/math/vecs.html}{this
snippet} .

\subsection{\texorpdfstring{\hyperref[matrix]{Matrix}}{Matrix}}\label{matrix}

\begin{quote}
See \href{https://typst.app/docs/reference/math/mat/}{official
reference}
\end{quote}

Matrix is very similar to \texttt{\ }{\texttt{\ vec\ }}\texttt{\ } , but
accepts rows, separated by \texttt{\ }{\texttt{\ ;\ }}\texttt{\ } :

\begin{verbatim}
$
mat(
    1, 2, ..., 10;
    2, 2, ..., 10;
    dots.v, dots.v, dots.down, dots.v;
    10, 10, ..., 10; // `;` in the end is optional
)
$
\end{verbatim}

\pandocbounded{\includesvg[keepaspectratio]{typst-img/ca1e7bdfe61f2ae541843aeff854d40882487bed8fd5b1e094852cf662a759f8-1.svg}}

\subsubsection{\texorpdfstring{\hyperref[delimiters-and-gaps]{Delimiters
and gaps}}{Delimiters and gaps}}\label{delimiters-and-gaps}

You can specify them the same way as for vectors.

Specify the arguments either before the content, or \textbf{after the
semicolon} . The code will panic if there is no semicolon!

\begin{verbatim}
$
mat(
    delim: "|",
    1, 2, ..., 10;
    2, 2, ..., 10;
    dots.v, dots.v, dots.down, dots.v;
    10, 10, ..., 10;
    gap: #0.3em
)
$
\end{verbatim}

\pandocbounded{\includesvg[keepaspectratio]{typst-img/8fd5effce0cef589ea8f7e7388cf221f1c8d7f0ac6c76d8d7d2fb14c4840bef7-1.svg}}

\subsection{\texorpdfstring{\hyperref[semicolon-syntax]{Semicolon
syntax}}{Semicolon syntax}}\label{semicolon-syntax}

When you use semicolons, the arguments \emph{between the semicolons} are
merged into arrays. See yourself:

\begin{verbatim}
#let fun(..args) = {
    args.pos()
}

$
fun(1, 2;3, 4; 6, ; 8)
$
\end{verbatim}

\pandocbounded{\includesvg[keepaspectratio]{typst-img/a589a9f51ffa925d9dce1da521c4d15373e236fd19db49317091d681c2fface0-1.svg}}

If you miss some of elements, they will be replaced by
\texttt{\ }{\texttt{\ none\ }}\texttt{\ } -s.

You can mix semicolon syntax and named arguments, but be careful!

\begin{verbatim}
#let fun(..args) = {
    repr(args.pos())
    repr(args.named())
}

$
fun(1, 2; gap: #3em, 4)
$
\end{verbatim}

\pandocbounded{\includesvg[keepaspectratio]{typst-img/7a3c90212650f7f7df0cb42177753236eddae675ac3220fbabd0f40e4af8b842-1.svg}}

For example, this will not work:

\begin{verbatim}
$
//         ↓ there is no `;`, so it tries to add (gap:) to array
mat(1, 2; 4, gap: #3em)
$
\end{verbatim}


\section{Examples Book LaTeX/book/basics/math/operators.tex}
\title{sitandr.github.io/typst-examples-book/book/basics/math/operators}

\section{\texorpdfstring{\hyperref[operators]{Operators}}{Operators}}\label{operators}

\begin{quote}
See \href{https://typst.app/docs/reference/math/op/}{reference} .
\end{quote}

There are lots of built-in "text operators" in Typst math. This is a
symbol that behaves very close to plain text. Nevertheless, it is
different:

\begin{verbatim}
$
lim x_n, "lim" x_n, "lim"x_n
$
\end{verbatim}

\pandocbounded{\includesvg[keepaspectratio]{typst-img/b195783135218e8117ac954790e7a108297d7a3e532136d851e2c397358509f0-1.svg}}

\subsection{\texorpdfstring{\hyperref[predefined-operators]{Predefined
operators}}{Predefined operators}}\label{predefined-operators}

Here are all text operators Typst has built-in:

\begin{verbatim}
$
arccos, arcsin, arctan, arg, cos, cosh, cot, coth, csc,\
csch, ctg, deg, det, dim, exp, gcd, hom, id, im, inf, ker,\
lg, lim, liminf, limsup, ln, log, max, min, mod, Pr, sec,\
sech, sin, sinc, sinh, sup, tan, tanh, tg "and" tr
$
\end{verbatim}

\pandocbounded{\includesvg[keepaspectratio]{typst-img/8a14bfdd8bd657d613ccbcd3f77d68f31e6d73e509ba85dd8e6f5207d5c8c7e4-1.svg}}

\subsection{\texorpdfstring{\hyperref[creating-custom-operator]{Creating
custom
operator}}{Creating custom operator}}\label{creating-custom-operator}

Of course, there always will be some text operators you will need that
are not in the list.

But don\textquotesingle t worry, it is very easy to add your own:

\begin{verbatim}
#let arcsinh = math.op("arcsinh")

$
arcsinh x
$
\end{verbatim}

\pandocbounded{\includesvg[keepaspectratio]{typst-img/e4f5a9aa5dfd03914d26ad85ed73eff426d21badca21ea5a6e8de5032b2f29bb-1.svg}}

\subsubsection{\texorpdfstring{\hyperref[limits-for-operators]{Limits
for operators}}{Limits for operators}}\label{limits-for-operators}

When creating operators (upright text with proper spacing), you can set
limits for \emph{display mode} at the same time:

\begin{verbatim}
$
op("liminf")_a, op("liminf", limits: #true)_a
$
\end{verbatim}

\pandocbounded{\includesvg[keepaspectratio]{typst-img/9c3593b91bf3810a593b622e4972c5a87d637696f35850422f9232c74802a394-1.svg}}

This is roughly equivalent to

\begin{verbatim}
$
limits(op("liminf"))_a
$
\end{verbatim}

\pandocbounded{\includesvg[keepaspectratio]{typst-img/7aaabb25d8e73d54504aa3e99b9c8b341759f165923439447f4990871ec3943f-1.svg}}

Everything can be combined to create new operators:

\begin{verbatim}
#let liminf = math.op(math.underline(math.lim), limits: true)
#let limsup = math.op(math.overline(math.lim), limits: true)
#let integrate = math.op($integral dif x$)

$
liminf_(x->oo)\
limsup_(x->oo)\
integrate x^2
$
\end{verbatim}

\pandocbounded{\includesvg[keepaspectratio]{typst-img/adf6ee9659a71ecefb64d09f5f27f01acdd193bc79c792abf95fc56821bca4cb-1.svg}}


\section{Examples Book LaTeX/book/basics/math/limits.tex}
\title{sitandr.github.io/typst-examples-book/book/basics/math/limits}

\section{\texorpdfstring{\hyperref[setting-limits]{Setting
limits}}{Setting limits}}\label{setting-limits}

Sometimes we want to change how the default attaching should work.

\subsection{\texorpdfstring{\hyperref[limits]{Limits}}{Limits}}\label{limits}

For example, in many countries it is common to write definite integrals
with limits below and above. To set this, use
\texttt{\ }{\texttt{\ limits\ }}\texttt{\ } function:

\begin{verbatim}
$
integral_a^b\
limits(integral)_a^b
$
\end{verbatim}

\pandocbounded{\includesvg[keepaspectratio]{typst-img/ade8f85a6178d42d58769da477afa5349a3db9df3075a3d5f8e4a6b546c3d43e-1.svg}}

You can set this by default using
\texttt{\ }{\texttt{\ show\ }}\texttt{\ } rule:

\begin{verbatim}
#show math.integral: math.limits

$
integral_a^b
$

This is inline equation: $integral_a^b$
\end{verbatim}

\pandocbounded{\includesvg[keepaspectratio]{typst-img/e0011edccf76468c3d77a7502ce1dc001c82bfd9d590b258d8c8453d056bc966-1.svg}}

\subsection{\texorpdfstring{\hyperref[only-display-mode]{Only display
mode}}{Only display mode}}\label{only-display-mode}

Notice that this will also affect inline equations. To enable limits for
display math only, use
\texttt{\ }{\texttt{\ limits(inline:\ false)\ }}\texttt{\ } :

\begin{verbatim}
#show math.integral: math.limits.with(inline: false)

$
integral_a^b
$

This is inline equation: $integral_a^b$.
\end{verbatim}

\pandocbounded{\includesvg[keepaspectratio]{typst-img/d37f1132cdf338670e131079a57ae724a7dfcb102f3125dad712173fbf115bcd-1.svg}}

Of course, it is possible to move them back as bottom attachments:

\begin{verbatim}
$
sum_a^b, scripts(sum)_a^b
$
\end{verbatim}

\pandocbounded{\includesvg[keepaspectratio]{typst-img/7134a72120f7217b1f11438e166fa7e53f3a9287fa4c9079019181a6e16affb8-1.svg}}

\subsection{\texorpdfstring{\hyperref[operations]{Operations}}{Operations}}\label{operations}

The same scheme works for operations. By default, they are attached to
the bottom and top:

\begin{verbatim}
$a =_"By lemme 1" b, a scripts(=)_+ b$
\end{verbatim}

\pandocbounded{\includesvg[keepaspectratio]{typst-img/98d790005c43aa666b392f8a35f1e9564ff315aaf9881ceb309e53bd5db542b1-1.svg}}






\section{Combined Examples Book LaTeX/book/typstonomicon.tex}
\section{Examples Book LaTeX/book/typstonomicon/try_catch.tex}
\title{sitandr.github.io/typst-examples-book/book/typstonomicon/try_catch}

\section{\texorpdfstring{\hyperref[try--catch]{Try \&
Catch}}{Try \& Catch}}\label{try--catch}

\begin{verbatim}
// author: laurmaedje
// Renders an image or a placeholder if it doesn't exist.
// Don’t try this at home, kids!
#let maybe-image(path, ..args) = context {
  let path-label = label(path)
   let first-time = query((context {}).func()).len() == 0
   if first-time or query(path-label).len() > 0 {
    [#image(path, ..args)#path-label]
  } else {
    rect(width: 50%, height: 5em, fill: luma(235), stroke: 1pt)[
      #set align(center + horizon)
      Could not find #raw(path)
    ]
  }
}

#maybe-image("../tiger.jpg")
#maybe-image("../tiger1.jpg")
\end{verbatim}

\pandocbounded{\includesvg[keepaspectratio]{typst-img/ee71afd2e954c4ab04385fb359baa63b3c6852718ae7b0d63948cf9180d50e89-1.svg}}


\section{Examples Book LaTeX/book/typstonomicon/multiple-show.tex}
\title{sitandr.github.io/typst-examples-book/book/typstonomicon/multiple-show}

\subsection{\texorpdfstring{\hyperref[multiple-show-rules]{Multiple show
rules}}{Multiple show rules}}\label{multiple-show-rules}

Sometimes there is a need to apply several rules that look very similar.
Or generate them from code. One of the ways to deal with this, the most
cursed one, is this:

\begin{verbatim}
#let rules = (math.sum, math.product, math.root)

#let apply-rules(rules, it) = {
  if rules.len() == 0 {
    return it
  }
  show rules.pop(): math.display
  apply-rules(rules, it)
}

$product/sum root(3, x)/2$

#show: apply-rules.with(rules)

$product/sum root(3, x)/2$
\end{verbatim}

\pandocbounded{\includesvg[keepaspectratio]{typst-img/3f8166b0ca4ea7bdcf8017e914da7036f5b5ac804c34535f36b2a67bba3d995b-1.svg}}

The recursion problem may be avoided with the power of
\texttt{\ }{\texttt{\ fold\ }}\texttt{\ } , with basically the same
idea:

\begin{verbatim}
// author: Eric
#let kind_supp = (code: "Listing", algo: "Algorithme")
#show: it => kind_supp.pairs().fold(it, (acc, (kind, supp)) => {
  show figure.where(kind: kind): set figure(supplement: supp)
  acc
})
\end{verbatim}

\pandocbounded{\includesvg[keepaspectratio]{typst-img/e2ee1949cb74ef6dc8109f082f424dcb30765452043f5e93ccdd8a4fc30029b3-1.svg}}

Note that just in case of symbols (if you don\textquotesingle t need
element functions), one can use regular expressions. That is a more
robust way:

\begin{verbatim}
#show regex("[" + math.product + math.sum + "]"): math.display

$product/sum root(3, x)/2$
\end{verbatim}

\pandocbounded{\includesvg[keepaspectratio]{typst-img/b0f3afcb048a141cbfc9404f17ab9f91c701528560eb09810ce0bbaae66adbaa-1.svg}}


\section{Examples Book LaTeX/book/typstonomicon/chapters.tex}
\title{sitandr.github.io/typst-examples-book/book/typstonomicon/chapters}

\section{\texorpdfstring{\hyperref[create-zero-level-chapters]{Create
zero-level
chapters}}{Create zero-level chapters}}\label{create-zero-level-chapters}

\begin{verbatim}
// author: tinger

#let chapter = figure.with(
  kind: "chapter",
  // same as heading
  numbering: none,
  // this cannot use auto to translate this automatically as headings can, auto also means something different for figures
  supplement: "Chapter",
  // empty caption required to be included in outline
  caption: [],
)

// emulate element function by creating show rule
#show figure.where(kind: "chapter"): it => {
  set text(22pt)
  counter(heading).update(0)
  if it.numbering != none { strong(it.counter.display(it.numbering)) } + [ ] + strong(it.body)
}

// no access to element in outline(indent: it => ...), so we must do indentation in here instead of outline
#show outline.entry: it => {
  if it.element.func() == figure {
    // we're configuring chapter printing here, effectively recreating the default show impl with slight tweaks
    let res = link(it.element.location(), 
      // we must recreate part of the show rule from above once again
      if it.element.numbering != none {
        numbering(it.element.numbering, ..it.element.counter.at(it.element.location()))
      } + [ ] + it.element.body
    )

    if it.fill != none {
      res += [ ] + box(width: 1fr, it.fill) + [ ] 
    } else {
      res += h(1fr)
    }

    res += link(it.element.location(), it.page)
    strong(res)
  } else {
    // we're doing indenting here
    h(1em * it.level) + it
  }
}

// new target selector for default outline
#let chapters-and-headings = figure.where(kind: "chapter", outlined: true).or(heading.where(outlined: true))

//
// start of actual doc prelude
//

#set heading(numbering: "1.")

// can't use set, so we reassign with default args
#let chapter = chapter.with(numbering: "I")

// an example of a "show rule" for a chapter
// can't use chapter because it's not an element after using .with() anymore
#show figure.where(kind: "chapter"): set text(red)

//
// start of actual doc
//

// as you can see these are not elements like headings, which makes the setup a bit harder
// because the chapters are not headings, the numbering does not include their chapter, but could using a show rule for headings

#outline(target: chapters-and-headings)

#chapter[Chapter]
= Chap Heading
== Sub Heading

#chapter[Chapter again]
= Chap Heading
= Chap Heading
== Sub Heading
=== Sub Sub Heading
== Sub Heading

#chapter[Chapter yet again]
\end{verbatim}

\pandocbounded{\includesvg[keepaspectratio]{typst-img/0ad0c265008f81fca8694b44d2d9407815cd64a7bb4b10a631aac3370a9d52e1-1.svg}}


\section{Examples Book LaTeX/book/typstonomicon/index.tex}
\title{sitandr.github.io/typst-examples-book/book/typstonomicon/index}

\section{\texorpdfstring{\hyperref[typstonomicon-or-the-code-you-should-not-write]{Typstonomicon,
or The Code You Should Not
Write}}{Typstonomicon, or The Code You Should Not Write}}\label{typstonomicon-or-the-code-you-should-not-write}

Totally cursed examples with lots of quires, measure and other things to
hack around current Typst limitations. Generally you should use this
code only if you really need it.

Code in this chapter may break in lots of circumstances and debugging it
will be very painful. You are warned.

I think that this chapter will slowly die as Typst matures.


\section{Examples Book LaTeX/book/typstonomicon/inline_with.tex}
\title{sitandr.github.io/typst-examples-book/book/typstonomicon/inline_with}

\section{\texorpdfstring{\hyperref[horizontally-align-something-with-something]{Horizontally
align something with
something}}{Horizontally align something with something}}\label{horizontally-align-something-with-something}

\begin{verbatim}
// author: tabiasgeehuman
#let inline-with(select, content) = context {
  let target = query(
    selector(select)
  ).last().location().position().x
  let current = here().position().x

  box(inset: (x: target - current + 0.3em), content)
}

#let inline-label(name) = [#line(length: 0%) #name]

#inline-with(selector(<start-c>))[= Common values]
#align(left, box[$
    #inline-label(<start-c>) "Circles"(0) =& 0 \
    lim_(x -> 1) "Circles"(0) =& 0
$])
\end{verbatim}

\pandocbounded{\includesvg[keepaspectratio]{typst-img/6aedec57e7a1480b4eeb9ad00c327661943c9144d64eaaffe73de22872386739-1.svg}}


\section{Examples Book LaTeX/book/typstonomicon/math_display.tex}
\title{sitandr.github.io/typst-examples-book/book/typstonomicon/math_display}

\section{\texorpdfstring{\hyperref[make-all-math-display-math]{Make all
math display
math}}{Make all math display math}}\label{make-all-math-display-math}

May slightly interfere with math blocks.

\begin{verbatim}
// author: eric1102
#show math.equation: it => {
  if it.body.fields().at("size", default: none) != "display" {
    return math.display(it)
  }
  it
}

Inline math: $sum_(n=0)^oo e^(x^2 - n/x^2)$\
Some other text on new line.


$
sum_(n=0)^oo e^(x^2 - n/x^2)
$
\end{verbatim}

\pandocbounded{\includesvg[keepaspectratio]{typst-img/e339a54461b130913bf4f724b29b763aec7dffa0662827935aeb7b258538523b-1.svg}}


\section{Examples Book LaTeX/book/typstonomicon/totally-empty.tex}
\title{sitandr.github.io/typst-examples-book/book/typstonomicon/totally-empty}

\section{\texorpdfstring{\hyperref[empty-pages-without-numbering]{Empty
pages without
numbering}}{Empty pages without numbering}}\label{empty-pages-without-numbering}

\subsection{\texorpdfstring{\hyperref[empty-pages-before-chapters-starting-at-odd-pages]{Empty
pages before chapters starting at odd
pages}}{Empty pages before chapters starting at odd pages}}\label{empty-pages-before-chapters-starting-at-odd-pages}

This snippet has been broken on 0.12.0. If someone will help fixing it,
this would be cool.

\begin{verbatim}
// author: janekfleper

#set page(height: 20em)

#let find-labels(name) = {
  return query(name).map(label => label.location().page())
}

#let page-header = context {
  let empty-pages = find-labels(<empty-page>)
  let new-chapters = find-labels(<new-chapter>)
  if new-chapters.len() > 0 {
    if new-chapters.contains(here().page()) [
      _a new chapter starts on this page_
      #return
    ]

    // get the index of the next <new-chapter> label
    let new-chapter-index = new-chapters.position(page => page > here().page())
    if new-chapter-index != none {
      let empty-page = empty-pages.at(new-chapter-index)
      if empty-page < here().page() [
        _this is an empty page to make the next chapter start on an odd page_
        #return
      ]
    }
  }

  [and this would be a regular header]
  line(length: 100%)
}

#let page-footer = context {
  // since the page breaks in chapter-heading() are inserted after the <empty-page> label,
  // the selector has to look "before" the current page to find the relevant label
  let empty-page-labels = query(selector(<empty-page>).before(here()))
  if empty-page-labels.len() > 0 {
    let empty-page = empty-page-labels.last().location().page()
    // look back at the most recent <new-chapter> label
    let new-chapter = query(selector(<new-chapter>).before(here())).last().location().page()
    // check that there is no <new-chapter> label on the current page
    if (new-chapter != here().page()) and (empty-page + 1 == here().page()) [
      _this is an empty page where the page number should be omitted_
      #return
    ]
  }

  let page-display = counter(page).display(here().page-numbering())
  h(1fr) + page-display + h(1fr)
}

#show heading.where(level: 1): it => [
  #[] <empty-page>
  #pagebreak(to: "even", weak: true)
  #[] <new-chapter>
  #pagebreak(to: "odd", weak: true)
  #it.body
  #v(2em)
]


#show outline.entry.where(level: 1): it => {
  // reverse the results of the label queries to find the last <empty-page> label for the targeted page
  // the method array.position() will always return the first one...
  let empty-pages = find-labels(<empty-page>).rev()
  let new-chapters = query(<new-chapter>).rev()
  let empty-page-index = empty-pages.position(page => page == int(it.page.text))
  let new-chapter = new-chapters.at(empty-page-index)
  link(new-chapter.location())[#it.body #box(width: 1fr)[#it.fill] #new-chapter.location().page()]
}

#set page(header: page-header, footer: page-footer, numbering: "1")

#outline()

= The explanation

```
These queries reveal where the corresponding tags are found. The actual empty page is always at the location of the label <empty-page> + 1. If an empty page is actually inserted by the pagebreaks, the two labels will cover the page of the heading and one page before that. If no empty page was inserted, both labels will point to the same page which is not an issue either. And even then we can check for the <new-chapter> label first to give it a higher priority.

The first <empty-page> label is always on page 1 and can just be ignored since it points to the (non-existing) empty page before the first chapter.

pages with the label <empty-page>: #context find-labels(<empty-page>)
pages with the label <new-chapter>: #context find-labels(<new-chapter>)
```

= A heading
#lorem(190)

= Another heading
#lorem(100)

= The last heading
#lorem(400)
\end{verbatim}


\section{Examples Book LaTeX/book/typstonomicon/block_break.tex}
\title{sitandr.github.io/typst-examples-book/book/typstonomicon/block_break}

\section{\texorpdfstring{\hyperref[breakpoints-on-broken-blocks]{Breakpoints
on broken
blocks}}{Breakpoints on broken blocks}}\label{breakpoints-on-broken-blocks}

\subsubsection{\texorpdfstring{\hyperref[implementation-with-table-headers--footers]{Implementation
with table headers \&
footers}}{Implementation with table headers \& footers}}\label{implementation-with-table-headers--footers}

See a demo project (more comments, I stripped some of them)
\href{https://typst.app/project/r-yQHF952iFnPme9BWbRu3}{there} .

\begin{verbatim}
/// author: wrzian

// Underlying counter and zig-zag functions
#let counter-family(id) = {
  let parent = counter(id)
  let parent-step() = parent.step()
  let get-child() = counter(id + str(parent.get().at(0)))
  return (parent-step, get-child)
}

// A fun zig-zag line!
#let zig-zag(fill: black, rough-width: 6pt, height: 4pt, thick: 1pt, angle: 0deg) = {
  layout((size) => {
    // Use layout to get the size and measure our horizontal distance
    // Then get the per-zigzag width with some maths.
    let count = int(calc.round(size.width / rough-width))
    // Need to add extra thickness since we join with `h(-thick)`
    let width = thick + (size.width - thick) / count
    // One zig and one zag:
    let zig-and-zag = {
      let line-stroke = stroke(thickness: thick, cap: "round", paint: fill)
      let top-left = (thick/2, thick/2)
      let bottom-mid = (width/2, height - thick/2)
      let top-right = (width - thick/2, thick/2)
      let zig = line(stroke: line-stroke, start: top-left, end: bottom-mid)
      let zag = line(stroke: line-stroke, start: bottom-mid, end: top-right)
      box(place(zig) + place(zag), width: width, height: height, clip: true)
    }
    let zig-zags = ((zig-and-zag,) * count).join(h(-thick))
    rotate(zig-zags, angle)
  })
}

// ---- Define split-box ---- //

// Customizable options for a split-box border:
#let default-border = (
  // The starting and ending lines
  above: line(length: 100%),
  below: line(length: 100%),
  // Lines to put between the box over multiple pages
  btwn-above: line(length: 100%, stroke: (dash:"dotted")),
  btwn-below: line(length: 100%, stroke: (dash:"dotted")),
  // Left/right lines
  // These *must* use `grid.vline()`, otherwise you will get an error.
  // To remove the lines, set them to: `grid.vline(stroke: none)`.
  // You could probably configure this better with a rowspan, but I'm lazy.
  left: grid.vline(),
  right: grid.vline(),
)

// Create a box for content which spans multiple pages/columns and
// has custom borders above and below the column-break.
#let split-box(
  // Set the border dictionary, see `default-border` above for options
  border: default-border,
  // The cell to place content in, this should resolve to a `grid.cell`
  cell: grid.cell.with(inset: 5pt),
  // The last positional arg or args are your actual content
  // Any extra named args will be sent to the underlying grid when called
  // This is useful for fill, align, etc.
  ..args
) = {
  // See `utils.typ` for more info.
  let (parent-step, get-child) = counter-family("split-box-unique-counter-string")
  parent-step() // Place the parent counter once.
  // Keep track of each time the header is placed on a page.
  // Then check if we're at the first placement (for header) or the last (footer)
  // If not, we'll use the 'between' forms of the  border lines.
  let border-above = context {
    let header-count = get-child()
    header-count.step()
    context if header-count.get() == (1,) { border.above } else { border.btwn-above }
  }
  let border-below = context {
    let header-count = get-child()
    if header-count.get() == header-count.final() { border.below } else { border.btwn-below }
  }
  // Place the grid!
  grid(
    ..args.named(),
    columns: 3,
    border.left,
    grid.header(border-above , repeat: true),
    ..args.pos().map(cell),
    grid.footer(border-below, repeat: true),
    border.right,
  )
}

// ---- Examples ---- //

#set page(width: 7.2in, height: 3in, columns: 6)

// Tada!
#split-box[
  #lorem(20)
]

// And here's a fun example:

#let fun-border = (
  // gradients!
  above: line(length: 100%, stroke: 2pt + gradient.linear(..color.map.rainbow)),
  below: line(length: 100%, stroke: 2pt + gradient.linear(..color.map.rainbow, angle: 180deg)),
  // zig-zags!
  btwn-above: move(dy: +2pt, zig-zag(fill: blue, angle: 3deg)),
  btwn-below: move(dy: -2pt, zig-zag(fill: orange, angle: 177deg)),
  left: grid.vline(stroke: (cap: "round", paint: purple)),
  right: grid.vline(stroke: (cap: "round", paint: purple)),
)

#split-box(border: fun-border)[
  #lorem(25)
]

// And some more tame friends:

#split-box(border: (
  above: move(dy: -0.5pt, line(length: 100%)),
  below: move(dy: +0.5pt, line(length: 100%)),
  // zig-zags!
  btwn-above: move(dy: -1.1pt, zig-zag()),
  btwn-below: move(dy: +1.1pt, zig-zag(angle: 180deg)),
  left: grid.vline(stroke: (cap: "round")),
  right: grid.vline(stroke: (cap: "round")),
))[
  #lorem(10)
]

#split-box(
  border: (
    above: line(length: 100%, stroke: luma(50%)),
    below: line(length: 100%, stroke: luma(50%)),
    btwn-above: line(length: 100%, stroke: (dash: "dashed", paint: luma(50%))),
    btwn-below: line(length: 100%, stroke: (dash: "dashed", paint: luma(50%))),
    left: grid.vline(stroke: none),
    right: grid.vline(stroke: none),
  ),
  cell: grid.cell.with(inset: 5pt, fill: color.yellow.saturate(-85%))
)[
  #lorem(20)
]
\end{verbatim}

\pandocbounded{\includesvg[keepaspectratio]{typst-img/52bd37f3e860317c6a162bb4a1ea8275ac73dede79a9e50c5201b4d1fd59c323-1.svg}}

\subsubsection{\texorpdfstring{\hyperref[implementation-via-headers-footers-and-stated]{Implementation
via headers, footers and
stated}}{Implementation via headers, footers and stated}}\label{implementation-via-headers-footers-and-stated}

Limitations: \textbf{works only with one-column layout and one break} .

\begin{verbatim}
#let countBoundaries(loc, fromHeader) = {
  let startSelector = selector(label("boundary-start"))
  let endSelector = selector(label("boundary-end"))

  if fromHeader {
    // Count down from the top of the page
    startSelector = startSelector.after(loc)
    endSelector = endSelector.after(loc)
  } else {
    // Count up from the bottom of the page
    startSelector = startSelector.before(loc)
    endSelector = endSelector.before(loc)
  }

  let startMarkers = query(startSelector)
  let endMarkers = query(endSelector)
  let currentPage = loc.position().page

  let pageStartMarkers = startMarkers.filter(elem =>
    elem.location().position().page == currentPage)

  let pageEndMarkers = endMarkers.filter(elem =>
    elem.location().position().page == currentPage)

  (start: pageStartMarkers.len(), end: pageEndMarkers.len())
}

#set page(
  margin: 2em,
  // ... other page setup here ...
  header: context {
    let boundaryCount = countBoundaries(here(), true)

    if boundaryCount.end > boundaryCount.start {
      // Decorate this header with an opening decoration
      [Block break top: $-->$]
    }
  },
  footer: context {
    let boundaryCount = countBoundaries(here(), false)

    if boundaryCount.start > boundaryCount.end {
      // Decorate this footer with a closing decoration
      [Block break end: $<--$]
    }
  }
)

#let breakable-block(body) = block({
  [
    #metadata("boundary") <boundary-start>
  ]
  stack(
    // Breakable list content goes here
    body
  )
  [
    #metadata("boundary") <boundary-end>
  ]
})

#set page(height: 10em)

#breakable-block[
    #([Something \ ]*10)
]
\end{verbatim}

\pandocbounded{\includesvg[keepaspectratio]{typst-img/b5c4a13157c5e42b879173a5b11ec49526bdaec107c979e90572aa38aadb424f-1.svg}}

\pandocbounded{\includesvg[keepaspectratio]{typst-img/b5c4a13157c5e42b879173a5b11ec49526bdaec107c979e90572aa38aadb424f-2.svg}}


\section{Examples Book LaTeX/book/typstonomicon/word_count.tex}
\title{sitandr.github.io/typst-examples-book/book/typstonomicon/word_count}

\section{\texorpdfstring{\hyperref[word-count]{Word
count}}{Word count}}\label{word-count}

This chapter is deprecated now. It will be removed soon.

\subsection{\texorpdfstring{\hyperref[recommended-solution]{Recommended
solution}}{Recommended solution}}\label{recommended-solution}

Use \texttt{\ }{\texttt{\ wordometr\ }}\texttt{\ }
\href{https://github.com/Jollywatt/typst-wordometer}{package} :

\begin{verbatim}
#import "@preview/wordometer:0.1.0": word-count, total-words

#show: word-count

In this document, there are #total-words words all up.

#word-count(total => [
  The number of words in this block is #total.words
  and there are #total.characters letters.
])
\end{verbatim}

\pandocbounded{\includesvg[keepaspectratio]{typst-img/a36d12209002f93aeaf23d4b21fcd4dcb1f9326f6ad358ad01558f09dede39c2-1.svg}}

\subsection{\texorpdfstring{\hyperref[just-count-all-words-in-document]{Just
count \emph{all} words in
document}}{Just count all words in document}}\label{just-count-all-words-in-document}

\begin{verbatim}
// original author: laurmaedje
#let words = counter("words")
#show regex("\p{L}+"): it => it + words.step()

== A heading
#lorem(50)

=== Strong chapter
#strong(lorem(25))

// it is ignoring comments

#align(right)[(#words.display() words)]
\end{verbatim}

\pandocbounded{\includesvg[keepaspectratio]{typst-img/b32a6f39c86a7719a156fb53625f8ec0d8a5f559e85367107b07b52cc7172e3a-1.svg}}

\subsection{\texorpdfstring{\hyperref[count-only-some-elements-ignore-others]{Count
only some elements, ignore
others}}{Count only some elements, ignore others}}\label{count-only-some-elements-ignore-others}

\begin{verbatim}
// original author: jollywatt
#let count-words(it) = {
    let fn = repr(it.func())
    if fn == "sequence" { it.children.map(count-words).sum() }
    else if fn == "text" { it.text.split().len() }
    else if fn in ("styled") { count-words(it.child) }
    else if fn in ("highlight", "item", "strong", "link") { count-words(it.body) }
    else if fn in ("footnote", "heading", "equation") { 0 }
    else { 0 }
}

#show: rest => {
    let n = count-words(rest)
    rest + align(right, [(#n words)])
}

== A heading (shouldn't be counted)
#lorem(50)

=== Strong chapter
#strong(lorem(25)) // counted too!
\end{verbatim}

\pandocbounded{\includesvg[keepaspectratio]{typst-img/0ba529d013270ae2cb21618241d5c3562ce4743815a68146fb4d5617dc1c4b22-1.svg}}


\section{Examples Book LaTeX/book/typstonomicon/extract_plain_text.tex}
\title{sitandr.github.io/typst-examples-book/book/typstonomicon/extract_plain_text}

\section{\texorpdfstring{\hyperref[extracting-plain-text]{Extracting
plain text}}{Extracting plain text}}\label{extracting-plain-text}

\begin{verbatim}
// original author: ntjess
#let stringify-by-func(it) = {
  let func = it.func()
  return if func in (parbreak, pagebreak, linebreak) {
    "\n"
  } else if func == smartquote {
    if it.double { "\"" } else { "'" } // "
  } else if it.fields() == (:) {
    // a fieldless element is either specially represented (and caught earlier) or doesn't have text
    ""
  } else {
    panic("Not sure how to handle type `" + repr(func) + "`")
  }
}

#let plain-text(it) = {
  return if type(it) == str {
    it
  } else if it == [ ] {
    " "
  } else if it.has("children") {
    it.children.map(plain-text).join()
  } else if it.has("body") {
    plain-text(it.body)
  } else if it.has("text") {
    if type(it.text) == "string" {
      it.text
    } else {
      plain-text(it.text)
    }
  } else {
    // remove this to ignore all other non-text elements
    stringify-by-func(it)
  }
}

#plain-text(`raw inline text`)

#plain-text(highlight[Highlighted text])

#plain-text[List
  - With
  - Some
  - Elements

  + And
  + Enumerated
  + Too
]

#plain-text(underline[Underlined])

#plain-text($sin(x + y)$)

#for el in (
  circle,
  rect,
  ellipse,
  block,
  box,
  par,
  raw.with(block: true),
  raw.with(block: false),
  heading,
) {
  plain-text(el(repr(el)))
  linebreak()
}

// Some empty elements
#plain-text(circle())
#plain-text(line())

#for spacer in (linebreak, pagebreak, parbreak) {
  plain-text(spacer())
}
\end{verbatim}

\pandocbounded{\includesvg[keepaspectratio]{typst-img/bcf07a5cddbcf3f046484609e01f9a05df81807d05391d590bf8c8e96b324d1b-1.svg}}


\section{Examples Book LaTeX/book/typstonomicon/original_image.tex}
\title{sitandr.github.io/typst-examples-book/book/typstonomicon/original_image}

\section{\texorpdfstring{\hyperref[image-with-original-size]{Image with
original
size}}{Image with original size}}\label{image-with-original-size}

This function renders image with the size it "naturally" has.

\textbf{Note: starting from v0.11} , Typst tries using default image
size when width and height are \texttt{\ }{\texttt{\ auto\ }}\texttt{\ }
. It only uses container\textquotesingle s size if the image
doesn\textquotesingle t fit. So this code is more like a legacy, but
still may be useful.

This works because measure conceptually places the image onto a page
with infinite size and then the image defaults to 1pt per pixel instead
of becoming infinitely larger itself.

\begin{verbatim}
// author: laurmaedje
#let natural-image(..args) = style(styles => {
  let (width, height) = measure(image(..args), styles)
  image(..args, width: width, height: height)
})

#image("../tiger.jpg")
#natural-image("../tiger.jpg")
\end{verbatim}

\pandocbounded{\includesvg[keepaspectratio]{typst-img/59503efa7e4aa0d37418ed3d0cb2c0c123268fae37cdcd54f8f7eb06b556e05d-1.svg}}


\section{Examples Book LaTeX/book/typstonomicon/remove-indent-nested.tex}
\title{sitandr.github.io/typst-examples-book/book/typstonomicon/remove-indent-nested}

\section{\texorpdfstring{\hyperref[remove-indent-from-nested-lists]{Remove
indent from nested
lists}}{Remove indent from nested lists}}\label{remove-indent-from-nested-lists}

\begin{verbatim}
// author: fenjalien
#show enum.item: it => {
  if repr(it.body.func()) == "sequence" {
    let children = it.body.children
    let index = children.position(x => x.func() == enum.item)
    if index != none {
      enum.item({
        children.slice(0, index).join()
        set enum(indent: -1.2em) // Note that this stops an infinitly recursive show rule
        children.slice(index).join()
      })
    } else {
      it
    }
  } else {
    it
  }
}

arst
+ A
+ b
+ c
  + d
+ e
  + f
+ g
+ h
+ i
+ 
\end{verbatim}

\pandocbounded{\includesvg[keepaspectratio]{typst-img/39725eefebf4a24de8f643e32c454fc7dff8f4f594ba29c6ca84c098b8983860-1.svg}}




\section{Combined Examples Book LaTeX/book/packages.tex}
\section{Examples Book LaTeX/book/packages/boxes.tex}
\title{sitandr.github.io/typst-examples-book/book/packages/boxes}

\section{\texorpdfstring{\hyperref[custom-boxes]{Custom
boxes}}{Custom boxes}}\label{custom-boxes}

\subsection{\texorpdfstring{\hyperref[showbox]{Showbox}}{Showbox}}\label{showbox}

\begin{verbatim}
#import "@preview/showybox:2.0.1": showybox

#showybox(
  [Hello world!]
)
\end{verbatim}

\pandocbounded{\includesvg[keepaspectratio]{typst-img/5b1a31dde61cee643fe9c8550a396d2cad3d27bcaf56412fb1b1a1a2563c462e-1.svg}}

\begin{verbatim}
#import "@preview/showybox:2.0.1": showybox

// First showybox
#showybox(
  frame: (
    border-color: red.darken(50%),
    title-color: red.lighten(60%),
    body-color: red.lighten(80%)
  ),
  title-style: (
    color: black,
    weight: "regular",
    align: center
  ),
  shadow: (
    offset: 3pt,
  ),
  title: "Red-ish showybox with separated sections!",
  lorem(20),
  lorem(12)
)

// Second showybox
#showybox(
  frame: (
    dash: "dashed",
    border-color: red.darken(40%)
  ),
  body-style: (
    align: center
  ),
  sep: (
    dash: "dashed"
  ),
  shadow: (
    offset: (x: 2pt, y: 3pt),
    color: yellow.lighten(70%)
  ),
  [This is an important message!],
  [Be careful outside. There are dangerous bananas!]
)
\end{verbatim}

\pandocbounded{\includesvg[keepaspectratio]{typst-img/71353a03ef746508398e53dc16ea676041d953dadb029a8e186fd9c317085510-1.svg}}

\begin{verbatim}
#import "@preview/showybox:2.0.1": showybox

#showybox(
  title: "Stokes' theorem",
  frame: (
    border-color: blue,
    title-color: blue.lighten(30%),
    body-color: blue.lighten(95%),
    footer-color: blue.lighten(80%)
  ),
  footer: "Information extracted from a well-known public encyclopedia"
)[
  Let $Sigma$ be a smooth oriented surface in $RR^3$ with boundary $diff Sigma equiv Gamma$. If a vector field $bold(F)(x,y,z)=(F_x (x,y,z), F_y (x,y,z), F_z (x,y,z))$ is defined and has continuous first order partial derivatives in a region containing $Sigma$, then

  $ integral.double_Sigma (bold(nabla) times bold(F)) dot bold(Sigma) = integral.cont_(diff Sigma) bold(F) dot dif bold(Gamma) $
]
\end{verbatim}

\pandocbounded{\includesvg[keepaspectratio]{typst-img/9e5c363090d9b928ee6c998876dd9e15a388ab6f6ae793f8a86ad688d2a62f2c-1.svg}}

\begin{verbatim}
#import "@preview/showybox:2.0.1": showybox

#showybox(
  title-style: (
    weight: 900,
    color: red.darken(40%),
    sep-thickness: 0pt,
    align: center
  ),
  frame: (
    title-color: red.lighten(80%),
    border-color: red.darken(40%),
    thickness: (left: 1pt),
    radius: 0pt
  ),
  title: "Carnot cycle's efficiency"
)[
  Inside a Carnot cycle, the efficiency $eta$ is defined to be:

  $ eta = W/Q_H = frac(Q_H + Q_C, Q_H) = 1 - T_C/T_H $
]
\end{verbatim}

\pandocbounded{\includesvg[keepaspectratio]{typst-img/3ce2b6bf5cd66f8aaa6c960c8f18902b63518eb4c6ee3f41337c1857e31128e9-1.svg}}

\begin{verbatim}
#import "@preview/showybox:2.0.1": showybox

#showybox(
  footer-style: (
    sep-thickness: 0pt,
    align: right,
    color: black
  ),
  title: "Divergence theorem",
  footer: [
    In the case of $n=3$, $V$ represents a volume in three-dimensional space, and $diff V = S$ its surface
  ]
)[
  Suppose $V$ is a subset of $RR^n$ which is compact and has a piecewise smooth boundary $S$ (also indicated with $diff V = S$). If $bold(F)$ is a continuously differentiable vector field defined on a neighborhood of $V$, then:

  $ integral.triple_V (bold(nabla) dot bold(F)) dif V = integral.surf_S (bold(F) dot bold(hat(n))) dif S $
]
\end{verbatim}

\pandocbounded{\includesvg[keepaspectratio]{typst-img/9abf5c05795f94a0b36b0e0fe84bb13ae210e6c234ad306606ed9bf52bd5e481-1.svg}}

\begin{verbatim}
#import "@preview/showybox:2.0.1": showybox

#showybox(
  frame: (
    border-color: red.darken(30%),
    title-color: red.darken(30%),
    radius: 0pt,
    thickness: 2pt,
    body-inset: 2em,
    dash: "densely-dash-dotted"
  ),
  title: "Gauss's Law"
)[
  The net electric flux through any hypothetical closed surface is equal to $1/epsilon_0$ times the net electric charge enclosed within that closed surface. The closed surface is also referred to as Gaussian surface. In its integral form:

  $ Phi_E = integral.surf_S bold(E) dot dif bold(A) = Q/epsilon_0 $
]
\end{verbatim}

\pandocbounded{\includesvg[keepaspectratio]{typst-img/9ae97a9b51a35a54fab7e017b1f500b5062b7e644928fa132a4cd1b218e8aad8-1.svg}}

\subsection{\texorpdfstring{\hyperref[colorful-boxes]{Colorful
boxes}}{Colorful boxes}}\label{colorful-boxes}

\begin{verbatim}
#import "@preview/colorful-boxes:1.2.0": colorbox, slantedColorbox, outlinebox, stickybox

#colorbox(
  title: lorem(5),
  color: "blue",
  radius: 2pt,
  width: auto
)[
  #lorem(50)
]

#slantedColorbox(
  title: lorem(5),
  color: "red",
  radius: 0pt,
  width: auto
)[
  #lorem(50)
]

#outlinebox(
  title: lorem(5),
  color: none,
  width: auto,
  radius: 2pt,
  centering: false
)[
  #lorem(50)
]

#outlinebox(
  title: lorem(5),
  color: "green",
  width: auto,
  radius: 2pt,
  centering: true
)[
  #lorem(50)
]

#stickybox(
  rotation: 3deg,
  width: 7cm
)[
  #lorem(20)
]
\end{verbatim}

\pandocbounded{\includesvg[keepaspectratio]{typst-img/a8efee5212da42450ccb46cedda2280b5e876e22cc08ab656a73d379754c8661-1.svg}}

\subsection{\texorpdfstring{\hyperref[theorems]{Theorems}}{Theorems}}\label{theorems}

See \href{./math.html}{math}


\section{Examples Book LaTeX/book/packages/physics.tex}
\title{sitandr.github.io/typst-examples-book/book/packages/physics}

\section{\texorpdfstring{\hyperref[physics]{Physics}}{Physics}}\label{physics}

\subsection{\texorpdfstring{\hyperref[physica]{\texttt{\ }{\texttt{\ physica\ }}\texttt{\ }}}{  physica  }}\label{physica}

\begin{quote}
Physica (Latin for \emph{natural sciences} ) provides utilities that
simplify otherwise complex and repetitive mathematical expressions in
natural sciences.
\end{quote}

\begin{quote}
Its
\href{https://github.com/Leedehai/typst-physics/blob/master/physica-manual.pdf}{manual}
provides a full set of demonstrations of how the package could be
helpful.
\end{quote}

\subsubsection{\texorpdfstring{\hyperref[mathematical-physics]{Mathematical
physics}}{Mathematical physics}}\label{mathematical-physics}

The \href{./math.html\#common-notations}{packages/math.md} page has more
examples on its math capabilities. Below is a preview that may be of
particular interest in the domain of physics:

\begin{itemize}
\tightlist
\item
  Calculus: differential, ordinary and partial derivatives

  \begin{itemize}
  \tightlist
  \item
    Optional function name,
  \item
    Optional order number or array of order numbers,
  \item
    Customizable "d" symbol and product joiner (say, exterior product),
  \item
    Overridable total order calculation,
  \end{itemize}
\item
  Vectors and vector fields: div, grad, curl,
\item
  Taylor expansion,
\item
  Dirac braket notations,
\item
  Tensors with abstract index notations,
\item
  Matrix transpose and dagger (conjugate transpose).
\item
  Special matrices: determinant, (anti-)diagonal, identity, zero,
  Jacobian, Hessian, etc.
\end{itemize}

A partial glimpse:

\begin{verbatim}
#import "@preview/physica:0.9.1": *
#show: super-T-as-transpose // put in a #[...] to limit its scope...
#show: super-plus-as-dagger // ...or use scripts() to manually override

$ dd(x,y,2) quad dv(f,x,d:Delta)      quad pdv(,x,y,[2i+1,2+i]) quad
  vb(a) va(a) vu(a_i)  quad mat(laplacian, div; grad, curl)     quad
  tensor(T,+a,-b,+c)   quad ket(phi)  quad A^+ e^scripts(+) A^T integral^T $
\end{verbatim}

\pandocbounded{\includesvg[keepaspectratio]{typst-img/fa8a12d2904a08958d4f83d69dda6bb38308b431055a25790d286a250e364c6c-1.svg}}

\subsubsection{\texorpdfstring{\hyperref[isotopes]{Isotopes}}{Isotopes}}\label{isotopes}

\begin{verbatim}
#import "@preview/physica:0.9.1": isotope

// a: mass number A
// z: the atomic number Z
$
isotope(I, a:127), quad isotope("Fe", z:26), quad
isotope("Tl",a:207,z:81) --> isotope("Pb",a:207,z:82) + isotope(e,a:0,z:-1)
$
\end{verbatim}

\pandocbounded{\includesvg[keepaspectratio]{typst-img/b290d801c6760a41e50520401d9e72cb63a8691aa136308cbad87349e7e436f0-1.svg}}

\subsubsection{\texorpdfstring{\hyperref[reduced-planck-constant-hbar]{Reduced
Planck constant
(hbar)}}{Reduced Planck constant (hbar)}}\label{reduced-planck-constant-hbar}

In the default font, the Typst built-in symbol
\texttt{\ }{\texttt{\ planck.reduce\ }}\texttt{\ } looks a bit off: on
letter "h" there is a slash instead of a horizontal bar, contrary to the
symbol\textquotesingle s colloquial name "h-bar". This package offers
\texttt{\ }{\texttt{\ hbar\ }}\texttt{\ } to render the symbol in the
familiar form⁠. Contrast:

\begin{verbatim}
#import "@preview/physica:0.9.1": hbar

$ E = planck.reduce omega => E = hbar omega, wide
  frac(planck.reduce^2, 2m) => frac(hbar^2, 2m), wide
  (pi G^2) / (planck.reduce c^4) => (pi G^2) / (hbar c^4), wide
  e^(frac(i(p x - E t), planck.reduce)) => e^(frac(i(p x - E t), hbar)) $
\end{verbatim}

\pandocbounded{\includesvg[keepaspectratio]{typst-img/efab3b0486d1cddc3388248c4100e1cc919088cdb93f3e072001547c40005f01-1.svg}}

\subsection{\texorpdfstring{\hyperref[quill-quantum-diagrams]{\texttt{\ }{\texttt{\ quill\ }}\texttt{\ }
: quantum
diagrams}}{  quill   : quantum diagrams}}\label{quill-quantum-diagrams}

\begin{quote}
See \href{https://github.com/Mc-Zen/quill/tree/main}{documentation} .
\end{quote}

\begin{verbatim}
#import "@preview/quill:0.2.0": *
#quantum-circuit(
  lstick($|0〉$), gate($H$), ctrl(1), rstick($(|00〉+|11〉)/√2$, n: 2), [\ ],
  lstick($|0〉$), 1, targ(), 1
)
\end{verbatim}

\pandocbounded{\includesvg[keepaspectratio]{typst-img/bd14c65cd60e1efc4d15ae7234e364c6d5740a168e2cb275743ed1fbcc9483eb-1.svg}}

\begin{verbatim}
#import "@preview/quill:0.2.0": *

#let ancillas = (setwire(0), 5, lstick($|0〉$), setwire(1), targ(), 2, [\ ],
setwire(0), 5, lstick($|0〉$), setwire(1), 1, targ(), 1)

#quantum-circuit(
  scale-factor: 80%,
  lstick($|ψ〉$), 1, 10pt, ctrl(3), ctrl(6), $H$, 1, 15pt, 
    ctrl(1), ctrl(2), 1, [\ ],
  ..ancillas, [\ ],
  lstick($|0〉$), 1, targ(), 1, $H$, 1, ctrl(1), ctrl(2), 
    1, [\ ],
  ..ancillas, [\ ],
  lstick($|0〉$), 2, targ(),  $H$, 1, ctrl(1), ctrl(2), 
    1, [\ ],
  ..ancillas
)
\end{verbatim}

\pandocbounded{\includesvg[keepaspectratio]{typst-img/597640923e31369199c6e7158de9094a2c94f2c5dae6ced72c6b83b1067fa8e4-1.svg}}

\begin{verbatim}
#import "@preview/quill:0.2.0": *

#quantum-circuit(
  lstick($|psi〉$),  ctrl(1), gate($H$), 1, ctrl(2), meter(), [\ ],
  lstick($|beta_00〉$, n: 2), targ(), 1, ctrl(1), 1, meter(), [\ ],
  3, gate($X$), gate($Z$),  midstick($|psi〉$)
)
\end{verbatim}

\pandocbounded{\includesvg[keepaspectratio]{typst-img/cc71bc052c7a80c702289f780ee42a168c1491076dd5934408373895ca95c35e-1.svg}}


\section{Examples Book LaTeX/book/packages/presentation.tex}
\title{sitandr.github.io/typst-examples-book/book/packages/presentation}

\section{\texorpdfstring{\hyperref[presentations]{Presentations}}{Presentations}}\label{presentations}

\subsection{\texorpdfstring{\hyperref[polylux]{Polylux}}{Polylux}}\label{polylux}

\begin{quote}
See \href{https://polylux.dev/book/}{polylux book}
\end{quote}

\begin{verbatim}
// Get Polylux from the official package repository
#import "@preview/polylux:0.3.1": *

// Make the paper dimensions fit for a presentation and the text larger
#set page(paper: "presentation-16-9")
#set text(size: 25pt)

// Use #polylux-slide to create a slide and style it using your favourite Typst functions
#polylux-slide[
  #align(horizon + center)[
    = Very minimalist slides

    A lazy author

    July 23, 2023
  ]
]

#polylux-slide[
  == First slide

  Some static text on this slide.
]

#polylux-slide[
  == This slide changes!

  You can always see this.
  // Make use of features like #uncover, #only, and others to create dynamic content
  #uncover(2)[But this appears later!]
]
\end{verbatim}

\pandocbounded{\includesvg[keepaspectratio]{typst-img/f46993d445b33c112929c1b2e3308a9a2b27297acc2eb470701fbe6b8707f710-1.svg}}

\pandocbounded{\includesvg[keepaspectratio]{typst-img/f46993d445b33c112929c1b2e3308a9a2b27297acc2eb470701fbe6b8707f710-2.svg}}

\pandocbounded{\includesvg[keepaspectratio]{typst-img/f46993d445b33c112929c1b2e3308a9a2b27297acc2eb470701fbe6b8707f710-3.svg}}

\pandocbounded{\includesvg[keepaspectratio]{typst-img/f46993d445b33c112929c1b2e3308a9a2b27297acc2eb470701fbe6b8707f710-4.svg}}

\subsection{\texorpdfstring{\hyperref[slydst]{Slydst}}{Slydst}}\label{slydst}

\begin{quote}
See the documentation
\href{https://github.com/glambrechts/slydst?ysclid=lr2gszrkck541184604}{there}
.
\end{quote}

Much more simpler and less powerful than polulyx:

\begin{verbatim}
#import "@preview/slydst:0.1.0": *

#show: slides.with(
  title: "Insert your title here", // Required
  subtitle: none,
  date: none,
  authors: (),
  layout: "medium",
  title-color: none,
)

== Outline

#outline()

= First section

== First slide

#figure(rect(width: 60%), caption: "Caption")

#v(1fr)

#lorem(20)

#definition(title: "An interesting definition")[
  #lorem(20)
]
\end{verbatim}

\pandocbounded{\includesvg[keepaspectratio]{typst-img/9d718fb02239fe71227dce959f0f468c0520df208e9b55e518dcf43f554bbd28-1.svg}}

\pandocbounded{\includesvg[keepaspectratio]{typst-img/9d718fb02239fe71227dce959f0f468c0520df208e9b55e518dcf43f554bbd28-2.svg}}

\pandocbounded{\includesvg[keepaspectratio]{typst-img/9d718fb02239fe71227dce959f0f468c0520df208e9b55e518dcf43f554bbd28-3.svg}}

\pandocbounded{\includesvg[keepaspectratio]{typst-img/9d718fb02239fe71227dce959f0f468c0520df208e9b55e518dcf43f554bbd28-4.svg}}


\section{Examples Book LaTeX/book/packages/drawing.tex}
\title{sitandr.github.io/typst-examples-book/book/packages/drawing}

\section{\texorpdfstring{\hyperref[drawing]{Drawing}}{Drawing}}\label{drawing}

\subsection{\texorpdfstring{\hyperref[cetz]{\texttt{\ }{\texttt{\ cetz\ }}\texttt{\ }}}{  cetz  }}\label{cetz}

Cetz is an analogue of LaTeX\textquotesingle s
\texttt{\ }{\texttt{\ tikz\ }}\texttt{\ } . Maybe it is not as powerful
yet, but certainly easier to learn and use.

It is the best choice in most of cases you want to draw something in
Typst.

\begin{verbatim}
#import "@preview/cetz:0.1.2"

#cetz.canvas(length: 1cm, {
  import cetz.draw: *
  import cetz.angle: angle
  let (a, b, c) = ((0,0), (-1,1), (1.5,0))
  line(a, b)
  line(a, c)
  set-style(angle: (radius: 1, label-radius: .5), stroke: blue)
  angle(a, c, b, label: $alpha$, mark: (end: ">"), stroke: blue)
  set-style(stroke: red)
  angle(a, b, c, label: n => $#{n/1deg} degree$,
    mark: (end: ">"), stroke: red, inner: false)
})
\end{verbatim}

\pandocbounded{\includesvg[keepaspectratio]{typst-img/d3b5277dd18dffb6da9a8f41486cb85a5044597821e80867652f062724ed8dd4-1.svg}}

\begin{verbatim}
#import "@preview/cetz:0.1.2": canvas, draw

#canvas(length: 1cm, {
  import draw: *
  intersections(name: "demo", {
    circle((0, 0))
    bezier((0,0), (3,0), (1,-1), (2,1))
    line((0,-1), (0,1))
    rect((1.5,-1),(2.5,1))
  })
  for-each-anchor("demo", (name) => {
    circle("demo." + name, radius: .1, fill: black)
  })
})
\end{verbatim}

\pandocbounded{\includesvg[keepaspectratio]{typst-img/05a1dbe2a2d17e5e81991406bed640775db6ab4ce2d585bc5a0d1175def43ea1-1.svg}}

\begin{verbatim}
#import "@preview/cetz:0.1.2": canvas, draw

#canvas(length: 1cm, {
  import draw: *
  let (a, b, c) = ((0, 0), (1, 1), (2, -1))
  line(a, b, c, stroke: gray)
  bezier-through(a, b, c, name: "b")
  // Show calculated control points
  line(a, "b.ctrl-1", "b.ctrl-2", c, stroke: gray)
})
\end{verbatim}

\pandocbounded{\includesvg[keepaspectratio]{typst-img/8e7d39d73212ebf8f230a0bd54a7fb7e58607a99f327e29809c4021b9e797345-1.svg}}

\begin{verbatim}
#import "@preview/cetz:0.1.2": canvas, draw

#canvas(length: 1cm, {
  import draw: *
  group(name: "g", {
    rotate(45deg)
    rect((0,0), (1,1), name: "r")
    copy-anchors("r")
  })
  circle("g.top", radius: .1, fill: black)
})
\end{verbatim}

\pandocbounded{\includesvg[keepaspectratio]{typst-img/b3d0b37a84cddb77a1508333743f851509e2250930abdcbda7ec4675e00077c3-1.svg}}

\begin{verbatim}
// author: LDemetrios
#import "@preview/cetz:0.2.2"

#cetz.canvas({
  let left = (a:2, b:1, d:-1, e:-2)
  let right = (p:2.7, q: 1.8, r: 0.9, s: -.3, t: -1.5, u: -2.4)
  let edges = "as,bq,dq,et".split(",")

  let ell-width = 1.5
  let ell-height = 3
  let dist = 5
  let dot-radius = 0.1
  let dot-clr = blue

  import cetz.draw: *
  circle((-dist/2, 0), radius:(ell-width ,  ell-height))
  circle((+dist/2, 0), radius:(ell-width ,  ell-height))

  for (name, y) in left {
    circle((-dist/2, y), radius:dot-radius, fill:dot-clr, name:name)
    content(name, anchor:"east", pad(right:.7em, text(fill:dot-clr, name)))
  }

  for (name, y) in right {
    circle((dist/2, y), radius:dot-radius, fill:dot-clr, name:name)
    content(name, anchor:"west", pad(left:.7em, text(fill:dot-clr, name)))
  }

  for edge in edges {
    let from = edge.at(0)
    let to = edge.at(1)
    line(from, to)
    mark(from, to, symbol: ">",  fill: black)
  }

  content((0, - ell-height), text(fill:blue)[APPLICATION], anchor:"south")
})
\end{verbatim}

\pandocbounded{\includesvg[keepaspectratio]{typst-img/7a4a9224b76305ecd694fd4505b3fdee8c706ccea76ac0e59fd13d469c343dd4-1.svg}}


\section{Examples Book LaTeX/book/packages/index.tex}
\title{sitandr.github.io/typst-examples-book/book/packages/index}

\section{\texorpdfstring{\hyperref[packages]{Packages}}{Packages}}\label{packages}

Once the \href{https://typst.app/universe}{Typst Universe} was launched,
this chapter has become almost redundant. The Universe is actually a
very cool place to look for packages.

However, there are still some cool examples of interesting package
usage.

\subsection{\texorpdfstring{\hyperref[general]{General}}{General}}\label{general}

Typst has packages, but, unlike LaTeX, you need to remember:

\begin{itemize}
\tightlist
\item
  You need them only for some specialized tasks, basic formatting
  \emph{can be totally done without them} .
\item
  Packages are much lighter and much easier "installed" than LaTeX ones.
\item
  Packages are just plain Typst files (and sometimes plugins), so you
  can easily write your own!
\end{itemize}

To use mighty package, just write, like this:

\begin{verbatim}
#import "@preview/cetz:0.1.2": canvas, plot

#canvas(length: 1cm, {
  plot.plot(size: (8, 6),
    x-tick-step: none,
    x-ticks: ((-calc.pi, $-pi$), (0, $0$), (calc.pi, $pi$)),
    y-tick-step: 1,
    {
      plot.add(
        style: plot.palette.blue,
        domain: (-calc.pi, calc.pi), x => calc.sin(x * 1rad))
      plot.add(
        hypograph: true,
        style: plot.palette.blue,
        domain: (-calc.pi, calc.pi), x => calc.cos(x * 1rad))
      plot.add(
        hypograph: true,
        style: plot.palette.blue,
        domain: (-calc.pi, calc.pi), x => calc.cos((x + calc.pi) * 1rad))
    })
})
\end{verbatim}

\pandocbounded{\includesvg[keepaspectratio]{typst-img/29d7015ed96122fa3fb663929c1ac58d25340995423c82456ab8815811373979-1.svg}}

\subsection{\texorpdfstring{\hyperref[contributing]{Contributing}}{Contributing}}\label{contributing}

If you are author of a package or just want to make a fair overview,
feel free to make issues/PR-s!


\section{Examples Book LaTeX/book/packages/headers.tex}
\title{sitandr.github.io/typst-examples-book/book/packages/headers}

\section{\texorpdfstring{\hyperref[headers]{Headers}}{Headers}}\label{headers}

\subsection{\texorpdfstring{\hyperref[hydra-contextual-headers]{\texttt{\ }{\texttt{\ hydra\ }}\texttt{\ }
: Contextual
headers}}{  hydra   : Contextual headers}}\label{hydra-contextual-headers}

We have discussed in \texttt{\ }{\texttt{\ Typst\ Basics\ }}\texttt{\ }
how to get current heading with
\texttt{\ }{\texttt{\ query(selector(heading).before(here()))\ }}\texttt{\ }
for headers. However, this works badly for nested headings with
numbering and similar things. For these cases there is
\texttt{\ }{\texttt{\ hydra\ }}\texttt{\ } :

\begin{verbatim}
#import "@preview/hydra:0.5.1": hydra

#set page(height: 10 * 20pt, margin: (y: 4em), numbering: "1", header: context {
  if calc.odd(here().page()) {
    align(right, emph(hydra(1)))
  } else {
    align(left, emph(hydra(2)))
  }
  line(length: 100%)
})
#set heading(numbering: "1.1")
#show heading.where(level: 1): it => pagebreak(weak: true) + it

= Introduction
#lorem(50)

= Content
== First Section
#lorem(50)
== Second Section
#lorem(100)
\end{verbatim}

\pandocbounded{\includesvg[keepaspectratio]{typst-img/1a1e2d4655c80e3b0cd9cd7db25c191054aac7ff69aa9cf7cda6935041b614ae-1.svg}}

\pandocbounded{\includesvg[keepaspectratio]{typst-img/1a1e2d4655c80e3b0cd9cd7db25c191054aac7ff69aa9cf7cda6935041b614ae-2.svg}}

\pandocbounded{\includesvg[keepaspectratio]{typst-img/1a1e2d4655c80e3b0cd9cd7db25c191054aac7ff69aa9cf7cda6935041b614ae-3.svg}}

\pandocbounded{\includesvg[keepaspectratio]{typst-img/1a1e2d4655c80e3b0cd9cd7db25c191054aac7ff69aa9cf7cda6935041b614ae-4.svg}}


\section{Examples Book LaTeX/book/packages/external.tex}
\title{sitandr.github.io/typst-examples-book/book/packages/external}

\section{\texorpdfstring{\hyperref[external]{External}}{External}}\label{external}

These are not official packages. Maybe once they will become one.

However, they may be very useful.

\subsection{\texorpdfstring{\hyperref[treemap-display]{Treemap
display}}{Treemap display}}\label{treemap-display}

\href{https://gist.github.com/taylorh140/9e353fdf737f1ef51aacb332efdd9516}{Code
Link}

\pandocbounded{\includegraphics[keepaspectratio]{img/treemap.png}}


\section{Examples Book LaTeX/book/packages/code.tex}
\title{sitandr.github.io/typst-examples-book/book/packages/code}

\section{\texorpdfstring{\hyperref[code]{Code}}{Code}}\label{code}

\subsection{\texorpdfstring{\hyperref[codly]{\texttt{\ }{\texttt{\ codly\ }}\texttt{\ }}}{  codly  }}\label{codly}

\begin{quote}
See docs \href{https://github.com/Dherse/codly}{there}
\end{quote}

\begin{verbatim}
#import "@preview/codly:0.1.0": codly-init, codly, disable-codly
#show: codly-init.with()

#codly(languages: (
        typst: (name: "Typst", color: rgb("#41A241"), icon: none),
    ),
    breakable: false
)

```typst
#import "@preview/codly:0.1.0": codly-init
#show: codly-init.with()
```

// Still formatted!
```rust
pub fn main() {
    println!("Hello, world!");
}
```

#disable-codly()
\end{verbatim}

\pandocbounded{\includesvg[keepaspectratio]{typst-img/eaa07afd21b4783a4be0a9726e714a8a4644421e5a93383e7deaeffaf4765105-1.svg}}

\subsection{\texorpdfstring{\hyperref[codelst]{Codelst}}{Codelst}}\label{codelst}

\begin{verbatim}
#import "@preview/codelst:2.0.0": sourcecode

#sourcecode[```typ
#show "ArtosFlow": name => box[
  #box(image(
    "logo.svg",
    height: 0.7em,
  ))
  #name
]

This report is embedded in the
ArtosFlow project. ArtosFlow is a
project of the Artos Institute.
```]
\end{verbatim}

\pandocbounded{\includesvg[keepaspectratio]{typst-img/2b2bbf130111979e4bc4cbc33171a39842467b3ea5e67a7fa0fcbf26222e8f90-1.svg}}


\section{Examples Book LaTeX/book/packages/misc.tex}
\title{sitandr.github.io/typst-examples-book/book/packages/misc}

\section{\texorpdfstring{\hyperref[misc]{Misc}}{Misc}}\label{misc}

\section{\texorpdfstring{\hyperref[formatting-strings]{Formatting
strings}}{Formatting strings}}\label{formatting-strings}

\subsection{\texorpdfstring{\hyperref[oxifmt-general-purpose-string-formatter]{\texttt{\ }{\texttt{\ oxifmt\ }}\texttt{\ }
, general purpose string
formatter}}{  oxifmt   , general purpose string formatter}}\label{oxifmt-general-purpose-string-formatter}

\begin{verbatim}
#import "@preview/oxifmt:0.2.0": strfmt
#strfmt("I'm {}. I have {num} cars. I'm {0}. {} is {{cool}}.", "John", "Carl", num: 10) \
#strfmt("{0:?}, {test:+012e}, {1:-<#8x}", "hi", -74, test: 569.4) \
#strfmt("{:_>+11.5}", 59.4) \
#strfmt("Dict: {:!<10?}", (a: 5))
\end{verbatim}

\pandocbounded{\includesvg[keepaspectratio]{typst-img/f4f305da3efacf420f5d2a5159a57cca479ebbfd9b7412246d483de520135087-1.svg}}

\begin{verbatim}
#import "@preview/oxifmt:0.2.0": strfmt
#strfmt("First: {}, Second: {}, Fourth: {3}, Banana: {banana} (brackets: {{escaped}})", 1, 2.1, 3, label("four"), banana: "Banana!!")\
#strfmt("The value is: {:?} | Also the label is {:?}", "something", label("label"))\
#strfmt("Values: {:?}, {1:?}, {stuff:?}", (test: 500), ("a", 5.1), stuff: [a])\
#strfmt("Left5 {:_<5}, Right6 {:*>6}, Center10 {centered: ^10?}, Left3 {tleft:_<3}", "xx", 539, tleft: "okay", centered: [a])\
\end{verbatim}

\pandocbounded{\includesvg[keepaspectratio]{typst-img/39d725a28a184c450c74f3f895d1d59d26271b86acbddd454da564df76b668c8-1.svg}}

\begin{verbatim}
#import "@preview/oxifmt:0.2.0": strfmt
#repr(strfmt("Left-padded7 numbers: {:07} {:07} {:07} {3:07}", 123, -344, 44224059, 45.32))\
#strfmt("Some numbers: {:+} {:+08}; With fill and align: {:_<+8}; Negative (no-op): {neg:+}", 123, 456, 4444, neg: -435)\
#strfmt("Bases (10, 2, 8, 16(l), 16(U):) {0} {0:b} {0:o} {0:x} {0:X} | W/ prefixes and modifiers: {0:#b} {0:+#09o} {0:_>+#9X}", 124)\
#strfmt("{0:.8} {0:.2$} {0:.potato$}", 1.234, 0, 2, potato: 5)\
#strfmt("{0:e} {0:E} {0:+.9e} | {1:e} | {2:.4E}", 124.2312, 50, -0.02)\
#strfmt("{0} {0:.6} {0:.5e}", 1.432, fmt-decimal-separator: ",")
\end{verbatim}

\pandocbounded{\includesvg[keepaspectratio]{typst-img/7b709cd252c147436c88822b60d49ede25a23040531eeac41fb2ba37ca46a2d8-1.svg}}

\subsection{\texorpdfstring{\hyperref[name-it-integer-to-text]{\texttt{\ }{\texttt{\ name-it\ }}\texttt{\ }
, integer to
text}}{  name-it   , integer to text}}\label{name-it-integer-to-text}

\begin{verbatim}
#import "@preview/name-it:0.1.0": name-it

- #name-it(2418345)
\end{verbatim}

\pandocbounded{\includesvg[keepaspectratio]{typst-img/825de955e9f7261cd08d725520caf813e797aa4891da32ed7b43bafbe8b19f28-1.svg}}

\subsection{\texorpdfstring{\hyperref[nth-nth-element]{\texttt{\ }{\texttt{\ nth\ }}\texttt{\ }
, Nth element}}{  nth   , Nth element}}\label{nth-nth-element}

\begin{verbatim}
#import "@preview/nth:0.2.0": nth
#nth(3), #nth(5), #nth(2421)
\end{verbatim}

\pandocbounded{\includesvg[keepaspectratio]{typst-img/f8389763af9ec32227285bdc25885f02b4ad74d6a5900852ccd0664989f1d3cb-1.svg}}


\section{Examples Book LaTeX/book/packages/word_count.tex}
\title{sitandr.github.io/typst-examples-book/book/packages/word_count}

\section{\texorpdfstring{\hyperref[counting-words]{Counting
words}}{Counting words}}\label{counting-words}

\subsection{\texorpdfstring{\hyperref[wordometr]{Wordometr}}{Wordometr}}\label{wordometr}

\begin{verbatim}
#import "@preview/wordometer:0.1.0": word-count, total-words

#show: word-count

In this document, there are #total-words words all up.

#word-count(total => [
  The number of words in this block is #total.words
  and there are #total.characters letters.
])
\end{verbatim}

\pandocbounded{\includesvg[keepaspectratio]{typst-img/a36d12209002f93aeaf23d4b21fcd4dcb1f9326f6ad358ad01558f09dede39c2-1.svg}}

\subsubsection{\texorpdfstring{\hyperref[excluding-elements]{Excluding
elements}}{Excluding elements}}\label{excluding-elements}

You can exclude elements by name (e.g.,
\texttt{\ }{\texttt{\ "caption"\ }}\texttt{\ } ), function (e.g.,
\texttt{\ }{\texttt{\ figure.caption\ }}\texttt{\ } ), where-selector
(e.g., \texttt{\ }{\texttt{\ raw.where(block:\ true)\ }}\texttt{\ } ),
or \texttt{\ }{\texttt{\ label\ }}\texttt{\ } (e.g.,
\texttt{\ }{\texttt{\ \textless{}\ }}\texttt{\ }{\texttt{\ no-wc\ }}\texttt{\ }{\texttt{\ \textgreater{}\ }}\texttt{\ }
).

\begin{verbatim}
#import "@preview/wordometer:0.1.0": word-count, total-words

#show: word-count.with(exclude: (heading.where(level: 1), strike))

= This Heading Doesn't Count
== But I do!

In this document #strike[(excluding me)], there are #total-words words all up.

#word-count(total => [
  You can exclude elements by label, too.
  #[That was #total-words, excluding this sentence!] <no-wc>
], exclude: <no-wc>)
\end{verbatim}

\pandocbounded{\includesvg[keepaspectratio]{typst-img/0e46f8aa570972e4f8a92bfa4b8f7b86b6374d632fa27bd043c102b683d70f96-1.svg}}


\section{Examples Book LaTeX/book/packages/glossary.tex}
\title{sitandr.github.io/typst-examples-book/book/packages/glossary}

\section{\texorpdfstring{\hyperref[glossary]{Glossary}}{Glossary}}\label{glossary}

\subsection{\texorpdfstring{\hyperref[glossarium]{glossarium}}{glossarium}}\label{glossarium}

\begin{quote}
\href{https://typst.app/universe/package/glossarium}{Link to the
universe}
\end{quote}

Package to manage glossary and abbreviations.

One of the very first cool packages of Typst, made specially for
(probably) the first thesis written in Typst.

\begin{verbatim}
#import "@preview/glossarium:0.4.1": make-glossary, print-glossary, gls, glspl
#show: make-glossary

// for better link visibility
#show link: set text(fill: blue.darken(60%))

#print-glossary(
    (
    // minimal term
    (key: "kuleuven", short: "KU Leuven"),

    // a term with a long form and a group
    (key: "unamur", short: "UNamur", long: "Namur University", group: "Universities"),

    // a term with a markup description
    (
      key: "oidc",
      short: "OIDC",
      long: "OpenID Connect",
      desc: [OpenID is an open standard and decentralized authentication protocol promoted by the non-profit
      #link("https://en.wikipedia.org/wiki/OpenID#OpenID_Foundation")[OpenID Foundation].],
      group: "Accronyms",
    ),

    // a term with a short plural
    (
      key: "potato",
      short: "potato",
      // "plural" will be used when "short" should be pluralized
      plural: "potatoes",
      desc: [#lorem(10)],
    ),

    // a term with a long plural
    (
      key: "dm",
      short: "DM",
      long: "diagonal matrix",
      // "longplural" will be used when "long" should be pluralized
      longplural: "diagonal matrices",
      desc: "Probably some math stuff idk",
    ),
  )
)

// referencing the OIDC term using gls
#gls("oidc")
// displaying the long form forcibly
#gls("oidc", long: true)

// referencing the OIDC term using the reference syntax
@oidc

Plural: #glspl("potato")

#gls("oidc", display: "whatever you want")
\end{verbatim}

\pandocbounded{\includesvg[keepaspectratio]{typst-img/c17c1be6563520252dfc59ccc646a6c48fb29e467d03f2892fdbfbddb496c3f9-1.svg}}


\section{Examples Book LaTeX/book/packages/graphs.tex}
\title{sitandr.github.io/typst-examples-book/book/packages/graphs}

\section{\texorpdfstring{\hyperref[graphs]{Graphs}}{Graphs}}\label{graphs}

\subsection{\texorpdfstring{\hyperref[cetz]{\texttt{\ }{\texttt{\ cetz\ }}\texttt{\ }}}{  cetz  }}\label{cetz}

Cetz comes with quite built-in support of drawing basic graphs. It is
much more customizable and extensible then packages like
\texttt{\ }{\texttt{\ plotst\ }}\texttt{\ } , so it is recommended to
skim through its possibilities.

\begin{quote}
See full manual
\href{https://github.com/johannes-wolf/cetz/blob/master/manual.pdf?raw=true}{there}
.
\end{quote}

\begin{verbatim}
#let data = (
  [A], ([B], [C], [D]), ([E], [F])
)

#import "@preview/cetz:0.1.2": canvas, draw, tree

#canvas(length: 1cm, {
  import draw: *

  set-style(content: (padding: .2),
    fill: gray.lighten(70%),
    stroke: gray.lighten(70%))

  tree.tree(data, spread: 2.5, grow: 1.5, draw-node: (node, _) => {
    circle((), radius: .45, stroke: none)
    content((), node.content)
  }, draw-edge: (from, to, _) => {
    line((a: from, number: .6, abs: true, b: to),
         (a: to, number: .6, abs: true, b: from), mark: (end: ">"))
  }, name: "tree")

  // Draw a "custom" connection between two nodes
  let (a, b) = ("tree.0-0-1", "tree.0-1-0",)
  line((a: a, number: .6, abs: true, b: b), (a: b, number: .6, abs: true, b: a), mark: (end: ">", start: ">"))
})
\end{verbatim}

\pandocbounded{\includesvg[keepaspectratio]{typst-img/18fc5bbebb79c44df6fd484d2cc0c763b6a64e4a6455535738e40932f5fa39b4-1.svg}}

\begin{verbatim}
#import "@preview/cetz:0.1.2": canvas, draw

#canvas({
    import draw: *
    circle((90deg, 3), radius: 0, name: "content")
    circle((210deg, 3), radius: 0, name: "structure")
    circle((-30deg, 3), radius: 0, name: "form")
    for (c, a) in (
    ("content", "bottom"),
    ("structure", "top-right"),
    ("form", "top-left")
    ) {
    content(c, box(c + " oriented", inset: 5pt), anchor:
    a)
    }
    stroke(gray + 1.2pt)
    line("content", "structure", "form", close: true)
    for (c, s, f, cont) in (
    (0.5, 0.1, 1, "PostScript"),
    (1, 0, 0.4, "DVI"),
    (0.5, 0.5, 1, "PDF"),
    (0, 0.25, 1, "CSS"),
    (0.5, 1, 0, "XML"),
    (0.5, 1, 0.4, "HTML"),
    (1, 0.2, 0.8, "LaTeX"),
    (1, 0.6, 0.8, "TeX"),
    (0.8, 0.8, 1, "Word"),
    (1, 0.05, 0.05, "ASCII")
    ) {
    content((bary: (content: c, structure: s, form:
    f)),cont)
    }
})
\end{verbatim}

\pandocbounded{\includesvg[keepaspectratio]{typst-img/e93f89ca321c612b1157fd81cea439ade85d17485d0111a08b94e54e59e356db-1.svg}}

\begin{verbatim}
#import "@preview/cetz:0.1.2": canvas, chart

#let data2 = (
  ([15-24], 18.0, 20.1, 23.0, 17.0),
  ([25-29], 16.3, 17.6, 19.4, 15.3),
  ([30-34], 14.0, 15.3, 13.9, 18.7),
  ([35-44], 35.5, 26.5, 29.4, 25.8),
  ([45-54], 25.0, 20.6, 22.4, 22.0),
  ([55+],   19.9, 18.2, 19.2, 16.4),
)

#canvas({
  chart.barchart(mode: "clustered",
                 size: (9, auto),
                 label-key: 0,
                 value-key: (..range(1, 5)),
                 bar-width: .8,
                 x-tick-step: 2.5,
                 data2)
})
\end{verbatim}

\pandocbounded{\includesvg[keepaspectratio]{typst-img/3d162509c91794a0814503ed304bea48b221b2f58559c9d832c3254580cd0d2b-1.svg}}

\subsubsection{\texorpdfstring{\hyperref[draw-a-graph-in-polar-coords]{Draw
a graph in polar
coords}}{Draw a graph in polar coords}}\label{draw-a-graph-in-polar-coords}

\begin{verbatim}
#import "@preview/cetz:0.1.2": canvas, plot

#figure(
canvas(length: 1cm, {
  plot.plot(size: (5, 5),
    x-tick-step: 5,
    y-tick-step: 5,
    x-max: 20,
    y-max: 20,
    x-min: -20,
    y-min: -20,
    x-grid: true,
    y-grid: true,
    {
      plot.add(
        domain: (0,2*calc.pi),
        samples: 100,
        t => (13*calc.cos(t)-5*calc.cos(2*t)-2*calc.cos(3*t)-calc.cos(4*t), 16*calc.sin(t)*calc.sin(t)*calc.sin(t))
        )
    })
}), caption: "Plot made with cetz",)
\end{verbatim}

\pandocbounded{\includesvg[keepaspectratio]{typst-img/d24c6270b5c074f9331b16cdde3b626129537c5b4760c17b4e447a7ef3f22388-1.svg}}

\subsection{\texorpdfstring{\hyperref[diagraph]{\texttt{\ }{\texttt{\ diagraph\ }}\texttt{\ }}}{  diagraph  }}\label{diagraph}

\subsubsection{\texorpdfstring{\hyperref[test]{Test}}{Test}}\label{test}

\begin{verbatim}
#import "@preview/diagraph:0.2.0": *
#let renderc(code) = render(code.text)

#renderc(
  ```
  digraph {
    rankdir=LR;
    f -> B
    B -> f
    C -> D
    D -> B
    E -> F
    f -> E
    B -> F
  }
  ```
)
\end{verbatim}

\pandocbounded{\includesvg[keepaspectratio]{typst-img/f47c3218e9b78fba4f38d6daeaff627ee6b210bda8dd26fcbc56f14a7bb984ee-1.svg}}

\subsubsection{\texorpdfstring{\hyperref[eating]{Eating}}{Eating}}\label{eating}

\begin{verbatim}
#import "@preview/diagraph:0.2.0": *
#let renderc(code) = render(code.text)

#renderc(
  ```
  digraph {
    orange -> fruit
    apple -> fruit
    fruit -> food
    carrot -> vegetable
    vegetable -> food
    food -> eat
    eat -> survive
  }
  ```
)
\end{verbatim}

\pandocbounded{\includesvg[keepaspectratio]{typst-img/0a7fcbfb15be7bac447381d10af9488a7353071c92d849d1e4b7800a360c7659-1.svg}}

\subsubsection{\texorpdfstring{\hyperref[fft]{FFT}}{FFT}}\label{fft}

Labels are overridden manually.

\begin{verbatim}
#import "@preview/diagraph:0.2.0": *
#let renderc(code) = render(code.text)

#renderc(
  ```
  digraph {
    node [shape=none]
    1
    2
    3
    r1
    r2
    r3
    1->2
    1->3
    2->r1 [color=red]
    3->r2 [color=red]
    r1->r3 [color=red]
    r2->r3 [color=red]
  }
  ```
)
\end{verbatim}

\pandocbounded{\includesvg[keepaspectratio]{typst-img/5d7074ff82c6786fa2fad07b25ff4c238dbb9333b0a806d3ea74474fbf8d005e-1.svg}}

\subsubsection{\texorpdfstring{\hyperref[state-machine]{State
Machine}}{State Machine}}\label{state-machine}

\begin{verbatim}
#import "@preview/diagraph:0.2.0": *
#set page(width: auto)
#let renderc(code) = render(code.text)

#renderc(
  ```
  digraph finite_state_machine {
    rankdir=LR
    size="8,5"

    node [shape=doublecircle]
    LR_0
    LR_3
    LR_4
    LR_8

    node [shape=circle]
    LR_0 -> LR_2 [label="SS(B)"]
    LR_0 -> LR_1 [label="SS(S)"]
    LR_1 -> LR_3 [label="S($end)"]
    LR_2 -> LR_6 [label="SS(b)"]
    LR_2 -> LR_5 [label="SS(a)"]
    LR_2 -> LR_4 [label="S(A)"]
    LR_5 -> LR_7 [label="S(b)"]
    LR_5 -> LR_5 [label="S(a)"]
    LR_6 -> LR_6 [label="S(b)"]
    LR_6 -> LR_5 [label="S(a)"]
    LR_7 -> LR_8 [label="S(b)"]
    LR_7 -> LR_5 [label="S(a)"]
    LR_8 -> LR_6 [label="S(b)"]
    LR_8 -> LR_5 [label="S(a)"]
  }
  ```
)
\end{verbatim}

\pandocbounded{\includesvg[keepaspectratio]{typst-img/ce09c93e743aceb45852a12c83839cafd73a5c68d370ff2f863c79ec5896ff10-1.svg}}

\subsubsection{\texorpdfstring{\hyperref[clustering]{Clustering}}{Clustering}}\label{clustering}

\begin{quote}
See \href{http://www.graphviz.org/content/cluster}{docs} .
\end{quote}

\begin{verbatim}
#import "@preview/diagraph:0.2.0": *
#let renderc(code) = render(code.text)

#renderc(
  ```
  digraph G {

    subgraph cluster_0 {
      style=filled;
      color=lightgrey;
      node [style=filled,color=white];
      a0 -> a1 -> a2 -> a3;
      label = "process #1";
    }

    subgraph cluster_1 {
      node [style=filled];
      b0 -> b1 -> b2 -> b3;
      label = "process #2";
      color=blue
    }

    start -> a0;
    start -> b0;
    a1 -> b3;
    b2 -> a3;
    a3 -> a0;
    a3 -> end;
    b3 -> end;

    start [shape=Mdiamond];
    end [shape=Msquare];
  }
  ```
)
\end{verbatim}

\pandocbounded{\includesvg[keepaspectratio]{typst-img/5b51a47ca589de6fdd481db4b61f96395ef246f12a54d77d6d9c443c3cd2fc72-1.svg}}

\subsubsection{\texorpdfstring{\hyperref[html]{HTML}}{HTML}}\label{html}

\begin{verbatim}
#import "@preview/diagraph:0.2.0": *
#let renderc(code) = render(code.text)

#renderc(
  ```
  digraph structs {
      node [shape=plaintext]
      struct1 [label=<
  <TABLE BORDER="0" CELLBORDER="1" CELLSPACING="0">
    <TR><TD>left</TD><TD PORT="f1">mid dle</TD><TD PORT="f2">right</TD></TR>
  </TABLE>>];
      struct2 [label=<
  <TABLE BORDER="0" CELLBORDER="1" CELLSPACING="0">
    <TR><TD PORT="f0">one</TD><TD>two</TD></TR>
  </TABLE>>];
      struct3 [label=<
  <TABLE BORDER="0" CELLBORDER="1" CELLSPACING="0" CELLPADDING="4">
    <TR>
      <TD ROWSPAN="3">hello<BR/>world</TD>
      <TD COLSPAN="3">b</TD>
      <TD ROWSPAN="3">g</TD>
      <TD ROWSPAN="3">h</TD>
    </TR>
    <TR>
      <TD>c</TD><TD PORT="here">d</TD><TD>e</TD>
    </TR>
    <TR>
      <TD COLSPAN="3">f</TD>
    </TR>
  </TABLE>>];
      struct1:f1 -> struct2:f0;
      struct1:f2 -> struct3:here;
  }
  ```
)
\end{verbatim}

\pandocbounded{\includesvg[keepaspectratio]{typst-img/104d9f0e05417c58dce29ff55b47019eadd8538eed11bf552b03c9803fb8ce5b-1.svg}}

\subsubsection{\texorpdfstring{\hyperref[overridden-labels]{Overridden
labels}}{Overridden labels}}\label{overridden-labels}

Labels for nodes \texttt{\ }{\texttt{\ big\ }}\texttt{\ } and
\texttt{\ }{\texttt{\ sum\ }}\texttt{\ } are overridden.

\begin{verbatim}
#import "@preview/diagraph:0.2.0": *
#set page(width: auto)

#raw-render(
  ```
  digraph {
    rankdir=LR
    node[shape=circle]
    Hmm -> a_0
    Hmm -> big
    a_0 -> "a'" -> big [style="dashed"]
    big -> sum
  }
  ```,
  labels: (:
    big: [_some_#text(2em)[ big ]*text*],
    sum: $ sum_(i=0)^n 1/i $,
  ),
)
\end{verbatim}

\pandocbounded{\includesvg[keepaspectratio]{typst-img/a89c13a3c9aad0509c224ede97b8f1ed14c899049f92e6f23a2effc0bd56de40-1.svg}}

\subsection{\texorpdfstring{\hyperref[bob-draw]{\texttt{\ }{\texttt{\ bob-draw\ }}\texttt{\ }}}{  bob-draw  }}\label{bob-draw}

WASM plugin for \href{https://github.com/ivanceras/svgbob}{svgbob} to
draw easily with ASCII,.

\begin{verbatim}
#import "@preview/bob-draw:0.1.0": *
#render(```
         /\_/\
bob ->  ( o.o )
         \ " /
  .------/  /
 (        | |
  `====== o o
```)
\end{verbatim}

\pandocbounded{\includesvg[keepaspectratio]{typst-img/6f2c3c039f98a852450fad73ef9ee68d6e4ddcef39edc2376903cf0aa72606a2-1.svg}}

\begin{verbatim}
#import "@preview/bob-draw:0.1.0": *
#show raw.where(lang: "bob"): it => render(it)

#render(
    ```
      0       3  
       *-------* 
    1 /|    2 /| 
     *-+-----* | 
     | |4    | |7
     | *-----|-*
     |/      |/
     *-------*
    5       6
    ```,
    width: 25%,
)

```bob
"cats:"
 /\_/\  /\_/\  /\_/\  /\_/\ 
( o.o )( o.o )( o.o )( o.o )
```

```bob
       +10-15V           ___0,047R
      *---------o-----o-|___|-o--o---------o----o-------.
    + |         |     |       |  |         |    |       |
    -===-      _|_    |       | .+.        |    |       |
    -===-      .-.    |       | | | 2k2    |    |       |
    -===-    470| +   |       | | |        |    |      _|_
    - |       uF|     '--.    | '+'       .+.   |      \ / LED
      +---------o        |6   |7 |8    1k | |   |      -+-
             ___|___   .-+----+--+--.     | |   |       |
              -═══-    |            |     '+'   |       |
                -      |            |1     |  |/  BC    |
               GND     |            +------o--+   547   |
                       |            |      |  |`>       |
                       |            |     ,+.   |       |
               .-------+            | 220R| |   o----||-+  IRF9Z34
               |       |            |     | |   |    |+->
               |       |  MC34063   |     `+'   |    ||-+
               |       |            |      |    |       |  BYV29     -12V6
               |       |            |      '----'       o--|<-o----o--X OUT
 6000 micro  - | +     |            |2                  |     |    |
 Farad, 40V ___|_____  |            |--o                C|    |    |
 Capacitor  ~ ~ ~ ~ ~  |            | GND         30uH  C|    |   --- 470
               |       |            |3      1nF         C|    |   ###  uF
               |       |            |-------||--.       |     |    | +
               |       '-----+----+-'           |      GND    |   GND
               |            5|   4|             |             |
               |             |    '-------------o-------------o
               |             |                           ___  |
               `-------------*------/\/\/------------o--|___|-'
                                     2k              |       1k0
                                                    .+.
                                                    | | 5k6 + 3k3
                                                    | | in Serie
                                                    '+'
                                                     |
                                                    GND
```
\end{verbatim}

\pandocbounded{\includesvg[keepaspectratio]{typst-img/850abc33fa97f8b80bbda399475b0e4436d275c03f1ca369187eea9e72948b01-1.svg}}

\subsection{\texorpdfstring{\hyperref[wavy]{\texttt{\ }{\texttt{\ wavy\ }}\texttt{\ }}}{  wavy  }}\label{wavy}

\subsection{\texorpdfstring{\hyperref[finite]{\texttt{\ }{\texttt{\ finite\ }}\texttt{\ }}}{  finite  }}\label{finite}

Finite automata. See the
\href{https://github.com/jneug/typst-finite/blob/main/manual.pdf}{manual}
for a full documentation.

\begin{verbatim}
#import "@preview/finite:0.3.0": automaton

#automaton((
  q0: (q1:0, q0:"0,1"),
  q1: (q0:(0,1), q2:"0"),
  q2: (),
))
\end{verbatim}

\pandocbounded{\includesvg[keepaspectratio]{typst-img/9eddd9b18a2df43372188dab692be9e2973fac63f3764683c431a2c0fb8ba873-1.svg}}


\section{Examples Book LaTeX/book/packages/tables.tex}
\title{sitandr.github.io/typst-examples-book/book/packages/tables}

\section{\texorpdfstring{\hyperref[tables]{Tables}}{Tables}}\label{tables}

\subsection{\texorpdfstring{\hyperref[tablex-general-purpose-tables-library]{Tablex:
general purpose tables
library}}{Tablex: general purpose tables library}}\label{tablex-general-purpose-tables-library}

\begin{verbatim}
#import "@preview/tablex:0.0.7": tablex, rowspanx, colspanx

#tablex(
  columns: 4,
  align: center + horizon,
  auto-vlines: false,

  // indicate the first two rows are the header
  // (in case we need to eventually
  // enable repeating the header across pages)
  header-rows: 2,

  // color the last column's cells
  // based on the written number
  map-cells: cell => {
    if cell.x == 3 and cell.y > 1 {
      cell.content = {
        let value = int(cell.content.text)
        let text-color = if value < 10 {
          red.lighten(30%)
        } else if value < 15 {
          yellow.darken(13%)
        } else {
          green
        }
        set text(text-color)
        strong(cell.content)
      }
    }
    cell
  },

  /* --- header --- */
  rowspanx(2)[*Username*], colspanx(2)[*Data*], (), rowspanx(2)[*Score*],
  (),                 [*Location*], [*Height*], (),
  /* -------------- */

  [John], [Second St.], [180 cm], [5],
  [Wally], [Third Av.], [160 cm], [10],
  [Jason], [Some St.], [150 cm], [15],
  [Robert], [123 Av.], [190 cm], [20],
  [Other], [Unknown St.], [170 cm], [25],
)
\end{verbatim}

\pandocbounded{\includesvg[keepaspectratio]{typst-img/9283c11489e3997fb302d12c4958a964543f3de172f3f8e21eb739f97ae78ae2-1.svg}}

\begin{verbatim}
#import "@preview/tablex:0.0.7": tablex, hlinex, vlinex, colspanx, rowspanx

#pagebreak()
#v(80%)

#tablex(
  columns: 4,
  align: center + horizon,
  auto-vlines: false,
  repeat-header: true,

  /* --- header --- */
  rowspanx(2)[*Names*], colspanx(2)[*Properties*], (), rowspanx(2)[*Creators*],
  (),                 [*Type*], [*Size*], (),
  /* -------------- */

  [Machine], [Steel], [5 $"cm"^3$], [John p& Kate],
  [Frog], [Animal], [6 $"cm"^3$], [Robert],
  [Frog], [Animal], [6 $"cm"^3$], [Robert],
  [Frog], [Animal], [6 $"cm"^3$], [Robert],
  [Frog], [Animal], [6 $"cm"^3$], [Robert],
  [Frog], [Animal], [6 $"cm"^3$], [Robert],
  [Frog], [Animal], [6 $"cm"^3$], [Robert],
  [Frog], [Animal], [6 $"cm"^3$], [Rodbert],
)
\end{verbatim}

\pandocbounded{\includesvg[keepaspectratio]{typst-img/03fd8d593886849d39370d731f423691b255e47da0a391649235f3f746c25e5c-1.svg}}

\pandocbounded{\includesvg[keepaspectratio]{typst-img/03fd8d593886849d39370d731f423691b255e47da0a391649235f3f746c25e5c-2.svg}}

\begin{verbatim}
#import "@preview/tablex:0.0.7": tablex, gridx, hlinex, vlinex, colspanx, rowspanx

#tablex(
  columns: 4,
  auto-lines: false,

  // skip a column here         vv
  vlinex(), vlinex(), vlinex(), (), vlinex(),
  colspanx(2)[a], (),  [b], [J],
  [c], rowspanx(2)[d], [e], [K],
  [f], (),             [g], [L],
  //   ^^ '()' after the first cell are 100% ignored
)

#tablex(
  columns: 4,
  auto-vlines: false,
  colspanx(2)[a], (),  [b], [J],
  [c], rowspanx(2)[d], [e], [K],
  [f], (),             [g], [L],
)

#gridx(
  columns: 4,
  (), (), vlinex(end: 2),
  hlinex(stroke: yellow + 2pt),
  colspanx(2)[a], (),  [b], [J],
  hlinex(start: 0, end: 1, stroke: yellow + 2pt),
  hlinex(start: 1, end: 2, stroke: green + 2pt),
  hlinex(start: 2, end: 3, stroke: red + 2pt),
  hlinex(start: 3, end: 4, stroke: blue.lighten(50%) + 2pt),
  [c], rowspanx(2)[d], [e], [K],
  hlinex(start: 2),
  [f], (),             [g], [L],
)
\end{verbatim}

\pandocbounded{\includesvg[keepaspectratio]{typst-img/4d25fc4ba39ee99bf9b8c043ab89bc74cf61cad3f4640b3384dad2e69f5f64c8-1.svg}}

\begin{verbatim}
#import "@preview/tablex:0.0.7": tablex, colspanx, rowspanx

#tablex(
  columns: 3,
  map-hlines: h => (..h, stroke: blue),
  map-vlines: v => (..v, stroke: green + 2pt),
  colspanx(2)[a], (),  [b],
  [c], rowspanx(2)[d], [ed],
  [f], (),             [g]
)
\end{verbatim}

\pandocbounded{\includesvg[keepaspectratio]{typst-img/9f721aa89d44247b880a2d34d64940cce12a782d4888a09b6a031a2918805128-1.svg}}

\begin{verbatim}
#import "@preview/tablex:0.0.7": tablex, colspanx, rowspanx

#tablex(
  columns: 4,
  auto-vlines: true,

  // make all cells italicized
  map-cells: cell => {
    (..cell, content: emph(cell.content))
  },

  // add some arbitrary content to entire rows
  map-rows: (row, cells) => cells.map(c =>
    if c == none {
      c  // keeping 'none' is important
    } else {
      (..c, content: [#c.content\ *R#row*])
    }
  ),

  // color cells based on their columns
  // (using 'fill: (column, row) => color' also works
  // for this particular purpose)
  map-cols: (col, cells) => cells.map(c =>
    if c == none {
      c
    } else {
      (..c, fill: if col < 2 { blue } else { yellow })
    }
  ),

  colspanx(2)[a], (),  [b], [J],
  [c], rowspanx(2)[dd], [e], [K],
  [f], (),             [g], [L],
)
\end{verbatim}

\pandocbounded{\includesvg[keepaspectratio]{typst-img/e4aeb7879544c21da12283922f4e3110d740059da77b65d94e34ed39229ffad1-1.svg}}

\begin{verbatim}
#import "@preview/tablex:0.0.7": gridx

#gridx(
  columns: 3,
  rows: 6,
  fill: (col, row) => (blue, red, green).at(calc.rem(row + col - 1, 3)),
  map-cols: (col, cells) => {
    let last = cells.last()
    last.content = [
      #cells.slice(0, cells.len() - 1).fold(0, (acc, c) => if c != none { acc + eval(c.content.text) } else { acc })
    ]
    last.fill = aqua
    cells.last() = last
    cells
  },
  [0], [5], [10],
  [1], [6], [11],
  [2], [7], [12],
  [3], [8], [13],
  [4], [9], [14],
  [s], [s], [s]
)
\end{verbatim}

\pandocbounded{\includesvg[keepaspectratio]{typst-img/c67cc9e428f9b21ae2e9c4ba792eacc7391fce70f06375f49d3a5f08234a5a77-1.svg}}

\subsection{\texorpdfstring{\hyperref[tada-data-manipulation]{Tada: data
manipulation}}{Tada: data manipulation}}\label{tada-data-manipulation}

\begin{verbatim}
#import "@preview/tada:0.1.0"

#let column-data = (
  name: ("Bread", "Milk", "Eggs"),
  price: (1.25, 2.50, 1.50),
  quantity: (2, 1, 3),
)
#let record-data = (
  (name: "Bread", price: 1.25, quantity: 2),
  (name: "Milk", price: 2.50, quantity: 1),
  (name: "Eggs", price: 1.50, quantity: 3),
)
#let row-data = (
  ("Bread", 1.25, 2),
  ("Milk", 2.50, 1),
  ("Eggs", 1.50, 3),
)

#import tada: TableData, to-tablex
#let td = TableData(data: column-data)
// Equivalent to:
#let td2 = tada.from-records(record-data)
// _Not_ equivalent to (since field names are unknown):
#let td3 = tada.from-rows(row-data)

#to-tablex(td)
#to-tablex(td2)
#to-tablex(td3)
\end{verbatim}

\pandocbounded{\includesvg[keepaspectratio]{typst-img/06c7045a0bb3aad12c70133b4aa55b1cadc17c944d28803e9418a376187afb2d-1.svg}}

\subsection{\texorpdfstring{\hyperref[tablem-markdown-tables]{Tablem:
markdown
tables}}{Tablem: markdown tables}}\label{tablem-markdown-tables}

\begin{quote}
See documentation \href{https://github.com/OrangeX4/typst-tablem}{there}
\end{quote}

Render markdown tables in Typst.

\begin{verbatim}
#import "@preview/tablem:0.1.0": tablem

#tablem[
  | *Name* | *Location* | *Height* | *Score* |
  | ------ | ---------- | -------- | ------- |
  | John   | Second St. | 180 cm   |  5      |
  | Wally  | Third Av.  | 160 cm   |  10     |
]
\end{verbatim}

\pandocbounded{\includesvg[keepaspectratio]{typst-img/6845ef64c7c12ce5f6616f130172c76974b184e97976e59a3a957c273c9084eb-1.svg}}

\subsubsection{\texorpdfstring{\hyperref[custom-render]{Custom
render}}{Custom render}}\label{custom-render}

\begin{verbatim}
#import "@preview/tablex:0.0.6": tablex, hlinex
#import "@preview/tablem:0.1.0": tablem

#let three-line-table = tablem.with(
  render: (columns: auto, ..args) => {
    tablex(
      columns: columns,
      auto-lines: false,
      align: center + horizon,
      hlinex(y: 0),
      hlinex(y: 1),
      ..args,
      hlinex(),
    )
  }
)

#three-line-table[
  | *Name* | *Location* | *Height* | *Score* |
  | ------ | ---------- | -------- | ------- |
  | John   | Second St. | 180 cm   |  5      |
  | Wally  | Third Av.  | 160 cm   |  10     |
]
\end{verbatim}

\pandocbounded{\includesvg[keepaspectratio]{typst-img/ebddbdf17a6518755d55af3900eabe9ffb8fa2c0d8b0326518dac03ca1856648-1.svg}}


\section{Examples Book LaTeX/book/packages/layout.tex}
\title{sitandr.github.io/typst-examples-book/book/packages/layout}

\section{\texorpdfstring{\hyperref[layouting]{Layouting}}{Layouting}}\label{layouting}

General useful things.

\subsection{\texorpdfstring{\hyperref[pinit-relative-place-by-pins]{Pinit:
relative place by
pins}}{Pinit: relative place by pins}}\label{pinit-relative-place-by-pins}

The idea of \href{https://github.com/OrangeX4/typst-pinit}{pinit} is
pinning pins on the normal flow of the text, and then placing the
content relative to pins.

\begin{verbatim}
#import "@preview/pinit:0.1.3": *
#set page(height: 6em, width: 20em)

#set text(size: 24pt)

A simple #pin(1)highlighted text#pin(2).

#pinit-highlight(1, 2)

#pinit-point-from(2)[It is simple.]
\end{verbatim}

\pandocbounded{\includesvg[keepaspectratio]{typst-img/b0a3a289ec65a00a9b39e0689578c9c139a65d1d9f379fa1593ba8ea9268af25-1.svg}}

More complex example:

\begin{verbatim}
#import "@preview/pinit:0.1.3": *

// Pages
#set page(paper: "presentation-4-3")
#set text(size: 20pt)
#show heading: set text(weight: "regular")
#show heading: set block(above: 1.4em, below: 1em)
#show heading.where(level: 1): set text(size: 1.5em)

// Useful functions
#let crimson = rgb("#c00000")
#let greybox(..args, body) = rect(fill: luma(95%), stroke: 0.5pt, inset: 0pt, outset: 10pt, ..args, body)
#let redbold(body) = {
  set text(fill: crimson, weight: "bold")
  body
}
#let blueit(body) = {
  set text(fill: blue)
  body
}

// Main body
#block[
  = Asymptotic Notation: $O$

  Use #pin("h1")asymptotic notations#pin("h2") to describe asymptotic efficiency of algorithms.
  (Ignore constant coefficients and lower-order terms.)

  #greybox[
    Given a function $g(n)$, we denote by $O(g(n))$ the following *set of functions*:
    #redbold(${f(n): "exists" c > 0 "and" n_0 > 0, "such that" f(n) <= c dot g(n) "for all" n >= n_0}$)
  ]

  #pinit-highlight("h1", "h2")

  $f(n) = O(g(n))$: #pin(1)$f(n)$ is *asymptotically smaller* than $g(n)$.#pin(2)

  $f(n) redbold(in) O(g(n))$: $f(n)$ is *asymptotically* #redbold[at most] $g(n)$.

  #pinit-line(stroke: 3pt + crimson, start-dy: -0.25em, end-dy: -0.25em, 1, 2)

  #block[Insertion Sort as an #pin("r1")example#pin("r2"):]

  - Best Case: $T(n) approx c n + c' n - c''$ #pin(3)
  - Worst case: $T(n) approx c n + (c' \/ 2) n^2 - c''$ #pin(4)

  #pinit-rect("r1", "r2")

  #pinit-place(3, dx: 15pt, dy: -15pt)[#redbold[$T(n) = O(n)$]]
  #pinit-place(4, dx: 15pt, dy: -15pt)[#redbold[$T(n) = O(n)$]]

  #blueit[Q: Is $n^(3) = O(n^2)$#pin("que")? How to prove your answer#pin("ans")?]

  #pinit-point-to("que", fill: crimson, redbold[No.])
  #pinit-point-from("ans", body-dx: -150pt)[
    Show that the equation $(3/2)^n >= c$ \
    has infinitely many solutions for $n$.
  ]
]
\end{verbatim}

\pandocbounded{\includesvg[keepaspectratio]{typst-img/4cc4ac1de81450b49f618408d35cd551858a4fcee317859f7f2a5d84482a9612-1.svg}}

\subsection{\texorpdfstring{\hyperref[margin-notes]{Margin
notes}}{Margin notes}}\label{margin-notes}

\begin{verbatim}
#import "@preview/drafting:0.1.1": *

#let (l-margin, r-margin) = (1in, 2in)
#set page(
  margin: (left: l-margin, right: r-margin, rest: 0.1in),
)
#set-page-properties(margin-left: l-margin, margin-right: r-margin)

= Margin Notes
== Setup
Unfortunately `typst` doesn't expose margins to calling functions, so you'll need to set them explicitly. This is done using `set-page-properties` *before you place any content*:

// At the top of your source file
// Of course, you can substitute any margin numbers you prefer
// provided the page margins match what you pass to `set-page-properties`

== The basics
#lorem(20)
#margin-note(side: left)[Hello, world!]
#lorem(10)
#margin-note[Hello from the other side]

#lorem(25)
#margin-note[When notes are about to overlap, they're automatically shifted]
#margin-note(stroke: aqua + 3pt)[To avoid collision]
#lorem(25)

#let caution-rect = rect.with(inset: 1em, radius: 0.5em, fill: orange.lighten(80%))
#inline-note(rect: caution-rect)[
  Be aware that notes will stop automatically avoiding collisions if 4 or more notes
  overlap. This is because `typst` warns when the layout doesn't resolve after 5 attempts
  (initial layout + adjustment for each note)
]
\end{verbatim}

\pandocbounded{\includesvg[keepaspectratio]{typst-img/80c65cf70b8da549afe447ce97f6dc71087cc0654dd85cd4f5e95bea388e3179-1.svg}}

\begin{verbatim}
#import "@preview/drafting:0.1.1": *

#let (l-margin, r-margin) = (1in, 2in)
#set page(
  margin: (left: l-margin, right: r-margin, rest: 0.1in),
)
#set-page-properties(margin-left: l-margin, margin-right: r-margin)

== Adjusting the default style
All function defaults are customizable through updating the module state:

#lorem(4) #margin-note(dy: -2em)[Default style]
#set-margin-note-defaults(stroke: orange, side: left)
#lorem(4) #margin-note[Updated style]


Even deeper customization is possible by overriding the default `rect`:

#import "@preview/colorful-boxes:1.1.0": stickybox

#let default-rect(stroke: none, fill: none, width: 0pt, content) = {
  stickybox(rotation: 30deg, width: width/1.5, content)
}
#set-margin-note-defaults(rect: default-rect, stroke: none, side: right)

#lorem(20)
#margin-note(dy: -25pt)[Why not use sticky notes in the margin?]

// Undo changes from last example
#set-margin-note-defaults(rect: rect, stroke: red)

== Multiple document reviewers
#let reviewer-a = margin-note.with(stroke: blue)
#let reviewer-b = margin-note.with(stroke: purple)
#lorem(20)
#reviewer-a[Comment from reviewer A]
#lorem(15)
#reviewer-b(side: left)[Comment from reviewer B]

== Inline Notes
#lorem(10)
#inline-note[The default inline note will split the paragraph at its location]
#lorem(10)
/*
// Should work, but doesn't? Created an issue in repo.
#inline-note(par-break: false, stroke: (paint: orange, dash: "dashed"))[
  But you can specify `par-break: false` to prevent this
]
*/
#lorem(10)
\end{verbatim}

\pandocbounded{\includesvg[keepaspectratio]{typst-img/282de769e728a8bdb9c78c665664b382ecbf59fd7d3c915fab67aae7055e2acb-1.svg}}

\begin{verbatim}
#import "@preview/drafting:0.1.1": *

#let (l-margin, r-margin) = (1in, 2in)
#set page(
  margin: (left: l-margin, right: r-margin, rest: 0.1in),
)
#set-page-properties(margin-left: l-margin, margin-right: r-margin)

== Hiding notes for print preview
#set-margin-note-defaults(hidden: true)

#lorem(20)
#margin-note[This will respect the global "hidden" state]
#margin-note(hidden: false, dy: -2.5em)[This note will never be hidden]

= Positioning
== Precise placement: rule grid
Need to measure space for fine-tuned positioning? You can use `rule-grid` to cross-hatch
the page with rule lines:

#rule-grid(width: 10cm, height: 3cm, spacing: 20pt)
#place(
  dx: 180pt,
  dy: 40pt,
  rect(fill: white, stroke: red, width: 1in, "This will originate at (180pt, 40pt)")
)

// Optionally specify divisions of the smallest dimension to automatically calculate
// spacing
#rule-grid(dx: 10cm + 3em, width: 3cm, height: 1.2cm, divisions: 5, square: true,  stroke: green)

// The rule grid doesn't take up space, so add it explicitly
#v(3cm + 1em)

== Absolute positioning
What about absolutely positioning something regardless of margin and relative location? `absolute-place` is your friend. You can put content anywhere:

#context {
  let (dx, dy) = (25%, here().position().y)
  let content-str = (
    "This absolutely-placed box will originate at (" + repr(dx) + ", " + repr(dy) + ")"
    + " in page coordinates"
  )
  absolute-place(
    dx: dx, dy: dy,
    rect(
      fill: green.lighten(60%),
      radius: 0.5em,
      width: 2.5in,
      height: 0.5in,
      [#align(center + horizon, content-str)]
    )
  )
}
#v(1in)

The "rule-grid" also supports absolute placement at the top-left of the page by passing `relative: false`. This is helpful for "rule"-ing the whole page.
\end{verbatim}

\pandocbounded{\includesvg[keepaspectratio]{typst-img/212dfc0f37bc9749e459085bb305f46a1db7ab3fbb22760f62ec58e349837d9e-1.svg}}

\subsection{\texorpdfstring{\hyperref[dropped-capitals]{Dropped
capitals}}{Dropped capitals}}\label{dropped-capitals}

\begin{quote}
Get more info
\href{https://github.com/EpicEricEE/typst-plugins/tree/master/droplet}{here}
\end{quote}

\subsubsection{\texorpdfstring{\hyperref[basic-usage]{Basic
usage}}{Basic usage}}\label{basic-usage}

\begin{verbatim}
#import "@preview/droplet:0.1.0": dropcap

#dropcap(gap: -2pt, hanging-indent: 8pt)[
  #lorem(42)
]
\end{verbatim}

\pandocbounded{\includesvg[keepaspectratio]{typst-img/a9c411d628d90aa8313aa9f0829bfdf43122c4532ad0d9d323a64b989a049d64-1.svg}}

\subsubsection{\texorpdfstring{\hyperref[extended-styling]{Extended
styling}}{Extended styling}}\label{extended-styling}

\begin{verbatim}
#import "@preview/droplet:0.1.0": dropcap

#dropcap(
  height: 2,
  justify: true,
  gap: 6pt,
  transform: letter => style(styles => {
    let height = measure(letter, styles).height

    grid(columns: 2, gutter: 6pt,
      align(center + horizon, text(blue, letter)),
      // Use "place" to ignore the line's height when
      // the font size is calculated later on.
      place(horizon, line(
        angle: 90deg,
        length: height + 6pt,
        stroke: blue.lighten(40%) + 1pt
      )),
    )
  })
)[
  #lorem(42)
]
\end{verbatim}

\pandocbounded{\includesvg[keepaspectratio]{typst-img/50d7ee4ffb1e61856535409373b040d579ab05734f3f304a4dc15f23361fd710-1.svg}}

\subsection{\texorpdfstring{\hyperref[headings-for-actual-current-chapter]{Headings
for actual current
chapter}}{Headings for actual current chapter}}\label{headings-for-actual-current-chapter}

\begin{quote}
See \href{https://github.com/tingerrr/hydra}{hydra}
\end{quote}

\begin{verbatim}
#import "@preview/hydra:0.2.0": hydra

#set page(header: hydra() + line(length: 100%))
#set heading(numbering: "1.1")
#show heading.where(level: 1): it => pagebreak(weak: true) + it

= Introduction
#lorem(750)

= Content
== First Section
#lorem(500)
== Second Section
#lorem(250)
== Third Section
#lorem(500)

= Annex
#lorem(10)
\end{verbatim}

\pandocbounded{\includesvg[keepaspectratio]{typst-img/ab3a07e72c5e19f28936f6d2249c9c5bcd102a27f4af177db63d7715c5c64f33-1.svg}}

\pandocbounded{\includesvg[keepaspectratio]{typst-img/ab3a07e72c5e19f28936f6d2249c9c5bcd102a27f4af177db63d7715c5c64f33-2.svg}}

\pandocbounded{\includesvg[keepaspectratio]{typst-img/ab3a07e72c5e19f28936f6d2249c9c5bcd102a27f4af177db63d7715c5c64f33-3.svg}}

\pandocbounded{\includesvg[keepaspectratio]{typst-img/ab3a07e72c5e19f28936f6d2249c9c5bcd102a27f4af177db63d7715c5c64f33-4.svg}}

\pandocbounded{\includesvg[keepaspectratio]{typst-img/ab3a07e72c5e19f28936f6d2249c9c5bcd102a27f4af177db63d7715c5c64f33-5.svg}}


\section{Examples Book LaTeX/book/packages/math.tex}
\title{sitandr.github.io/typst-examples-book/book/packages/math}

\section{\texorpdfstring{\hyperref[math]{Math}}{Math}}\label{math}

\subsection{\texorpdfstring{\hyperref[general]{General}}{General}}\label{general}

\subsubsection{\texorpdfstring{\hyperref[physica]{\texttt{\ }{\texttt{\ physica\ }}\texttt{\ }}}{  physica  }}\label{physica}

\begin{quote}
Physica (Latin for \emph{natural sciences} ) provides utilities that
simplify otherwise complex and repetitive mathematical expressions in
natural sciences.
\end{quote}

\begin{quote}
Its
\href{https://github.com/Leedehai/typst-physics/blob/master/physica-manual.pdf}{manual}
provides a full set of demonstrations of how the package could be
helpful.
\end{quote}

\paragraph{\texorpdfstring{\hyperref[common-notations]{Common
notations}}{Common notations}}\label{common-notations}

\begin{itemize}
\tightlist
\item
  Calculus: differential, ordinary and partial derivatives

  \begin{itemize}
  \tightlist
  \item
    Optional function name,
  \item
    Optional order number or an array of thereof,
  \item
    Customizable "d" symbol and product joiner (say, exterior product),
  \item
    Overridable total order calculation,
  \end{itemize}
\item
  Vectors and vector fields: div, grad, curl,
\item
  Taylor expansion,
\item
  Dirac braket notations,
\item
  Tensors with abstract index notations,
\item
  Matrix transpose and dagger (conjugate transpose).
\item
  Special matrices: determinant, (anti-)diagonal, identity, zero,
  Jacobian, Hessian, etc.
\end{itemize}

Below is a preview of those notations.

\begin{verbatim}
#import "@preview/physica:0.9.1": * // Symbol names annotated below

#table(
  columns: 4, align: horizon, stroke: none, gutter: 1em,

  // vectors: bold, unit, arrow
  [$ vb(a), vb(e_i), vu(a), vu(e_i), va(a), va(e_i) $],
  // dprod (dot product), cprod (cross product), iprod (innerproduct)
  [$ a dprod b, a cprod b, iprod(a, b) $],
  // laplacian (different from built-in laplace)
  [$ dot.double(u) = laplacian u =: laplace u $],
  // grad, div, curl (vector fields)
  [$ grad phi, div vb(E), \ curl vb(B) $],
)
\end{verbatim}

\pandocbounded{\includesvg[keepaspectratio]{typst-img/3be7ef86d6c5f7044a42c69fcf93afccd936eb0fcbe987122702c7dda467480f-1.svg}}

\begin{verbatim}
#import "@preview/physica:0.9.1": * // Symbol names annotated below

#table(
  columns: 4, align: horizon, stroke: none, gutter: 1em,

  // Row 1.
  // dd (differential), var (variation), difference
  [$ dd(f), var(f), difference(f) $],
  // dd, with an order number or an array thereof
  [$ dd(x,y), dd(x,y,2), \ dd(x,y,[1,n]), dd(vb(x),t,[3,]) $],
  // dd, with custom "d" symbol and joiner
  [$ dd(x,y,p:and), dd(x,y,d:Delta), \ dd(x,y,z,[1,1,n+1],d:d,p:dot) $],
  // dv (ordinary derivative), with custom "d" symbol and joiner
  [$ dv(phi,t,d:Delta), dv(phi,t,d:upright(D)), dv(phi,t,d:delta) $],

  // Row 2.
  // dv, with optional function name and order
  [$ dv(,t) (dv(x,t)) = dv(x,t,2) $],
  // pdv (partial derivative)
  [$ pdv(f,x,y,2), pdv(,x,y,[k,]) $],
  // pdv, with auto-added overridable total
  [$ pdv(,x,y,[i+2,i+1]), pdv(,y,x,[i+1,i+2],total:3+2i) $],
  // In a flat form
  [$ dv(u,x,s:slash), \ pdv(u,x,y,2,s:slash) $],
)
\end{verbatim}

\pandocbounded{\includesvg[keepaspectratio]{typst-img/0835a840454f88ed2e3b98ddfe37d6f8026729812372a6298d86129611f348c3-1.svg}}

\begin{verbatim}
#import "@preview/physica:0.9.1": * // Symbol names annotated below

#table(
  columns: 3, align: horizon, stroke: none, gutter: 1em,

  // tensor
  [$ tensor(T,+a,-b,-c) != tensor(T,-b,-c,+a) != tensor(T,+a',-b,-c) $],
  // Set builder notation
  [$ Set(p, {q^*, p} = 1) $],
  // taylorterm (Taylor series term)
  [$ taylorterm(f,x,x_0,1) \ taylorterm(f,x,x_0,(n+1)) $],
)
\end{verbatim}

\pandocbounded{\includesvg[keepaspectratio]{typst-img/5c08b65761578f38762229692a33e2b05f096aa8fb7859238b1018240f054d10-1.svg}}

\begin{verbatim}
#import "@preview/physica:0.9.1": * // Symbol names annotated below

#table(
  columns: 3, align: horizon, stroke: none, gutter: 1em,

  // expval (mean/expectation value), eval (evaluation boundary)
  [$ expval(X) = eval(f(x)/g(x))^oo_1 $],
  // Dirac braket notations
  [$
    bra(u), braket(u), braket(u, v), \
    ket(u), ketbra(u), ketbra(u, v), \
    mel(phi, hat(p), psi) $],
  // Superscript show rules that need to be enabled explicitly.
  // If put in a content block, they only control that block's scope.
  [
    #show: super-T-as-transpose // "..^T" just like handwriting
    #show: super-plus-as-dagger // "..^+" just like handwriting
    $ op("conj")A^T =^"def" A^+ \
      e^scripts(T), e^scripts(+) $ ], // Override with scripts()
)
\end{verbatim}

\pandocbounded{\includesvg[keepaspectratio]{typst-img/8965812a9c349892988d61872ff06418581098d15169b775f67b30e3460dd854-1.svg}}

\paragraph{\texorpdfstring{\hyperref[matrices]{Matrices}}{Matrices}}\label{matrices}

In addition to Typst\textquotesingle s built-in
\texttt{\ }{\texttt{\ mat()\ }}\texttt{\ } to write a matrix, physica
provides a number of handy tools to make it even easier.

\begin{verbatim}
#import "@preview/physica:0.9.1": TT, mdet

$
// Matrix transpose with "TT", though it is recommended to
// use super-T-as-transpose so that "A^T" also works (more on that later).
A^TT,
// Determinant with "mdet(...)".
det mat(a, b; c, d) := mdet(a, b; c, d)
$
\end{verbatim}

\pandocbounded{\includesvg[keepaspectratio]{typst-img/7eccaa3a0cf838bca4daf9ebf573452506d3ea724086fcda0c9eb4264e66b5d9-1.svg}}

Diagonal matrix
\texttt{\ }{\texttt{\ dmat(\ }}\texttt{\ }{\texttt{\ ...\ }}\texttt{\ }{\texttt{\ )\ }}\texttt{\ }
, antidiagonal matrix
\texttt{\ }{\texttt{\ admat(\ }}\texttt{\ }{\texttt{\ ...\ }}\texttt{\ }{\texttt{\ )\ }}\texttt{\ }
, identity matrix \texttt{\ }{\texttt{\ imat(n)\ }}\texttt{\ } , and
zero matrix \texttt{\ }{\texttt{\ zmat(n)\ }}\texttt{\ } .

\begin{verbatim}
#import "@preview/physica:0.9.1": dmat, admat, imat, zmat

$ dmat(1, 2)  dmat(1, a_1, xi, fill:0)               quad
  admat(1, 2) admat(1, a_1, xi, fill:dot, delim:"[") quad
  imat(2)     imat(3, delim:"{",fill:*) quad
  zmat(2)     zmat(3, delim:"|") $
\end{verbatim}

\pandocbounded{\includesvg[keepaspectratio]{typst-img/66bbf5294be293cc58d98de6ca078eb17c58539169325e6b59a6aa78e7a49f62-1.svg}}

Jacobian matrix with
\texttt{\ }{\texttt{\ jmat(func;\ }}\texttt{\ }{\texttt{\ ...\ }}\texttt{\ }{\texttt{\ )\ }}\texttt{\ }
or the longer name \texttt{\ }{\texttt{\ jacobianmatrix\ }}\texttt{\ } ,
Hessian matrix with
\texttt{\ }{\texttt{\ hmat(func;\ }}\texttt{\ }{\texttt{\ ...\ }}\texttt{\ }{\texttt{\ )\ }}\texttt{\ }
or the longer name \texttt{\ }{\texttt{\ hessianmatrix\ }}\texttt{\ } ,
and finally \texttt{\ }{\texttt{\ xmat(row,\ col,\ func)\ }}\texttt{\ }
to build a matrix.

\begin{verbatim}
#import "@preview/physica:0.9.1": jmat, hmat, xmat

$
jmat(f_1,f_2; x,y) jmat(f,g; x,y,z; delim:"[") quad
hmat(f; x,y)       hmat(; x,y; big:#true)      quad

#let elem-ij = (i,j) => $g^(#(i - 1)#(j - 1)) = #calc.pow(i,j)$
xmat(2, 2, #elem-ij)
$
\end{verbatim}

\pandocbounded{\includesvg[keepaspectratio]{typst-img/7cd30cef52b187d17459d7806a94d5ae56118d0f969760bbbaabeb83007e6869-1.svg}}

\subsubsection{\texorpdfstring{\hyperref[mitex]{\texttt{\ }{\texttt{\ mitex\ }}\texttt{\ }}}{  mitex  }}\label{mitex}

\begin{quote}
MiTeX provides LaTeX support powered by WASM in Typst, including
real-time rendering of LaTeX math equations. You can also use LaTeX
syntax to write
\texttt{\ }{\texttt{\ \textbackslash{}r\ }}\texttt{\ }{\texttt{\ ef\ }}\texttt{\ }
and
\texttt{\ }{\texttt{\ \textbackslash{}l\ }}\texttt{\ }{\texttt{\ abel\ }}\texttt{\ }
.
\end{quote}

\begin{quote}
Please refer to the \href{https://github.com/mitex-rs/mitex}{manual} for
more details.
\end{quote}

\begin{verbatim}
#import "@preview/mitex:0.2.4": *

Write inline equations like #mi("x") or #mi[y].

Also block equations:

#mitex(`
  \newcommand{\f}[2]{#1f(#2)}
  \f\relax{x} = \int_{-\infty}^\infty
    \f\hat\xi\,e^{2 \pi i \xi x}
    \,d\xi
`)

Text mode:

#mitext(`
  \iftypst
    #set math.equation(numbering: "(1)", supplement: "equation")
  \fi

  An inline equation $x + y$ and block \eqref{eq:pythagoras}.

  \begin{equation}
    a^2 + b^2 = c^2 \label{eq:pythagoras}
  \end{equation}
`)
\end{verbatim}

\pandocbounded{\includesvg[keepaspectratio]{typst-img/a3ff500a39b6d93d85b223af0aa162a5bfbe93fad3436dba80ee022638ed727a-1.svg}}

\subsubsection{\texorpdfstring{\hyperref[i-figured]{\texttt{\ }{\texttt{\ i-figured\ }}\texttt{\ }}}{  i-figured  }}\label{i-figured}

Configurable equation numbering per section in Typst. There is also
figure numbering per section, see more examples in its
\href{https://github.com/RubixDev/typst-i-figured}{manual} .

\begin{verbatim}
#import "@preview/i-figured:0.2.3"

// make sure you have some heading numbering set
#set heading(numbering: "1.1")

// apply the show rules (these can be customized)
#show heading: i-figured.reset-counters
#show math.equation: i-figured.show-equation.with(
  level: 1,
  zero-fill: true,
  leading-zero: true,
  numbering: "(1.1)",
  prefix: "eqt:",
  only-labeled: false,  // numbering all block equations implicitly
  unnumbered-label: "-",
)


= Introduction

You can write inline equations such as $x + y$, and numbered block equations like:

$ phi.alt := (1 + sqrt(5)) / 2 $ <ratio>

To reference a math equation, please use the `eqt:` prefix. For example, with @eqt:ratio, we have:

$ F_n = floor(1 / sqrt(5) phi.alt^n) $


= Appdendix

Additionally, you can use the <-> tag to indicate that a block formula should not be numbered:

$ y = integral_1^2 x^2 dif x $ <->

Subsequent math equations will continue to be numbered as usual:

$ F_n = floor(1 / sqrt(5) phi.alt^n) $
\end{verbatim}

\pandocbounded{\includesvg[keepaspectratio]{typst-img/b338b679a09371841be9322ac7cee901b6a1415582c3495677833602e344cae0-1.svg}}

\subsection{\texorpdfstring{\hyperref[theorems]{Theorems}}{Theorems}}\label{theorems}

\subsubsection{\texorpdfstring{\hyperref[ctheorem]{\texttt{\ }{\texttt{\ ctheorem\ }}\texttt{\ }}}{  ctheorem  }}\label{ctheorem}

A numbered theorem environment in Typst. See more examples in its
\href{https://github.com/sahasatvik/typst-theorems/blob/main/manual.pdf}{manual}
.

\begin{verbatim}
#import "@preview/ctheorems:1.1.0": *
#show: thmrules

#set page(width: 16cm, height: auto, margin: 1.5cm)
#set heading(numbering: "1.1")

#let theorem = thmbox("theorem", "Theorem", fill: rgb("#eeffee"))
#let corollary = thmplain("corollary", "Corollary", base: "theorem", titlefmt: strong)
#let definition = thmbox("definition", "Definition", inset: (x: 1.2em, top: 1em))

#let example = thmplain("example", "Example").with(numbering: none)
#let proof = thmplain(
  "proof", "Proof", base: "theorem",
  bodyfmt: body => [#body #h(1fr) $square$]
).with(numbering: none)

= Prime Numbers
#lorem(7)
#definition[ A natural number is called a #highlight[_prime number_] if ... ]
#example[
  The numbers $2$, $3$, and $17$ are prime. See @cor_largest_prime shows that
  this list is not exhaustive!
]
#theorem("Euclid")[There are infinitely many primes.]
#proof[
  Suppose to the contrary that $p_1, p_2, dots, p_n$ is a finite enumeration
  of all primes. ... a contradiction.
]
#corollary[
  There is no largest prime number.
] <cor_largest_prime>
#corollary[There are infinitely many composite numbers.]
\end{verbatim}

\pandocbounded{\includesvg[keepaspectratio]{typst-img/54d7817ddc4a8481da09052aa51dc6e4dde19bd85f40173b92750c402b07ff73-1.svg}}

\subsubsection{\texorpdfstring{\hyperref[lemmify]{\texttt{\ }{\texttt{\ lemmify\ }}\texttt{\ }}}{  lemmify  }}\label{lemmify}

Lemmify is another theorem evironment generator with many selector and
numbering capabilities. See documentations in its
\href{https://github.com/Marmare314/lemmify}{readme} .

\begin{verbatim}
#import "@preview/lemmify:0.1.5": *

#let my-thm-style(
  thm-type, name, number, body
) = grid(
  columns: (1fr, 3fr),
  column-gutter: 1em,
  stack(spacing: .5em, [#strong(thm-type) #number], emph(name)),
  body
)
#let my-styling = ( thm-styling: my-thm-style )
#let (
  definition, theorem, proof, lemma, rules
) = default-theorems("thm-group", lang: "en", ..my-styling)
#show: rules
#show thm-selector("thm-group"): box.with(inset: 0.8em)
#show thm-selector("thm-group", subgroup: "theorem"): it => box(
  it, fill: rgb("#eeffee"))

#set heading(numbering: "1.1")

= Prime numbers
#lorem(7) @proof and @thm[theorem]
#definition[ A natural number is called a #highlight[_prime number_] if ... ]
#theorem(name: "Theorem name")[There are infinitely many primes.]<thm>
#proof[
  Suppose to the contrary that $p_1, p_2, dots, p_n$ is a finite enumeration
  of all primes. ... #highlight[_a contradiction_].]<proof>
#lemma[There are infinitely many composite numbers.]
\end{verbatim}

\pandocbounded{\includesvg[keepaspectratio]{typst-img/46b0a27243980ee99b20133dbba1f00d4d819adff6e645ca0749820f5caf3589-1.svg}}


\section{Examples Book LaTeX/book/packages/wrapping.tex}
\title{sitandr.github.io/typst-examples-book/book/packages/wrapping}

\section{\texorpdfstring{\hyperref[wrapping-figures]{Wrapping
figures}}{Wrapping figures}}\label{wrapping-figures}

The better native support for wrapping is planned, however, something is
already possible via package:

\begin{verbatim}
#import "@preview/wrap-it:0.1.0": wrap-content, wrap-top-bottom

#set par(justify: true)
#let fig = figure(
  rect(fill: teal, radius: 0.5em, width: 8em),
  caption: [A figure],
)
#let body = lorem(40)
#wrap-content(fig, body)

#wrap-content(
  fig,
  body,
  align: bottom + right,
  column-gutter: 2em
)

#let boxed = box(fig, inset: 0.5em)
#wrap-content(boxed)[
  #lorem(40)
]

#let fig2 = figure(
  rect(fill: lime, radius: 0.5em),
  caption: [Another figure],
)
#wrap-top-bottom(boxed, fig2, lorem(60))
\end{verbatim}

\pandocbounded{\includesvg[keepaspectratio]{typst-img/1d249d6947bbea7f94c4f5f111c873f278dcf473e0cf672d6c55800c0eb6822c-1.svg}}

Limitations: non-ideal spacing near warping, only top-bottom left/right
are supported.






\section{C Examples Book LaTeX/book/basics.tex}
\section{Combined Examples Book LaTeX/book/basics/tutorial.tex}
\section{Examples Book LaTeX/book/basics/tutorial/markup.tex}
\title{sitandr.github.io/typst-examples-book/book/basics/tutorial/markup}

\section{\texorpdfstring{\hyperref[markup-language]{Markup
language}}{Markup language}}\label{markup-language}

\subsection{\texorpdfstring{\hyperref[starting]{Starting}}{Starting}}\label{starting}

\begin{verbatim}
Starting typing in Typst is easy.
You don't need packages or other weird things for most of things.

Blank line will move text to a new paragraph.

Btw, you can use any language and unicode symbols
without any problems as long as the font supports it: ßçœ̃ɛ̃ø∀αβёыა😆…
\end{verbatim}

\pandocbounded{\includesvg[keepaspectratio]{typst-img/ee9f64251c99c7aeaaf6fa1d5bc7e907c2d51a34aa38126544d515ca197ca2a8-1.svg}}

\subsection{\texorpdfstring{\hyperref[markup]{Markup}}{Markup}}\label{markup}

\begin{verbatim}
= Markup

This was a heading. Number of `=` in front of name corresponds to heading level.

== Second-level heading

Okay, let's move to _emphasis_ and *bold* text.

Markup syntax is generally similar to
`AsciiDoc` (this was `raw` for monospace text!)
\end{verbatim}

\pandocbounded{\includesvg[keepaspectratio]{typst-img/fa8b95f9b15083387a29c11d17efca9873b8e778643b1b5079aa137891d01c8d-1.svg}}

\subsection{\texorpdfstring{\hyperref[new-lines--escaping]{New lines \&
Escaping}}{New lines \& Escaping}}\label{new-lines--escaping}

\begin{verbatim}
You can break \
line anywhere you \
want using "\\" symbol.

Also you can use that symbol to
escape \_all the symbols you want\_,
if you don't want it to be interpreted as markup
or other special symbols.
\end{verbatim}

\pandocbounded{\includesvg[keepaspectratio]{typst-img/4dabdee2a61e7d10773d51772dba3665271a09d4d5df4a8f66dd80589f0bcd7a-1.svg}}

\subsection{\texorpdfstring{\hyperref[comments--codeblocks]{Comments \&
codeblocks}}{Comments \& codeblocks}}\label{comments--codeblocks}

\begin{verbatim}
You can write comments with `//` and `/* comment */`:
// Like this
/* Or even like
this */

```typ
Just in case you didn't read source,
this is how it is written:

// Like this
/* Or even like
this */

By the way, I'm writing it all in a _fenced code block_ with *syntax highlighting*!
```
\end{verbatim}

\pandocbounded{\includesvg[keepaspectratio]{typst-img/a481d12b3ed0bbe2d9db6cc4b4a1237cba9936de83333254dfce8702832db125-1.svg}}

\subsection{\texorpdfstring{\hyperref[smart-quotes]{Smart
quotes}}{Smart quotes}}\label{smart-quotes}

\begin{verbatim}
== What else?

There are not much things in basic "markup" syntax,
but we will see much more interesting things very soon!
I hope you noticed auto-matched "smart quotes" there.
\end{verbatim}

\pandocbounded{\includesvg[keepaspectratio]{typst-img/89114a6e9af45c2eb9db2ef44d0e5ba41e31bf816e72803bd1a9a02120e69fc3-1.svg}}

\subsection{\texorpdfstring{\hyperref[lists]{Lists}}{Lists}}\label{lists}

\begin{verbatim}
- Writing lists in a simple way is great.
- Nothing complex, start your points with `-`
  and this will become a list.
  - Indented lists are created via indentation.

+ Numbered lists start with `+` instead of `-`.
+ There is no alternative markup syntax for lists
+ So just remember `-` and `+`, all other symbols
  wouldn't work in an unintended way.
  + That is a general property of Typst's markup.
  + Unlike Markdown, there is only one way
    to write something with it.
\end{verbatim}

\pandocbounded{\includesvg[keepaspectratio]{typst-img/ad4e424e067a4362e9f145c0c4ba4b7c1b65e17e7d0e7631b6836841607ef85e-1.svg}}

\textbf{Notice:}

\begin{verbatim}
Typst numbered lists differ from markdown-like syntax for lists. If you write them by hand, numbering is preserved:

1. Apple
1. Orange
1. Peach
\end{verbatim}

\pandocbounded{\includesvg[keepaspectratio]{typst-img/477695c86becc136dceb144e90c0acd2b75faa2a49743f8673d09974b71da324-1.svg}}

\subsection{\texorpdfstring{\hyperref[math]{Math}}{Math}}\label{math}

\begin{verbatim}
I will just mention math ($a + b/c = sum_i x^i$)
is possible and quite pretty there:

$
7.32 beta +
  sum_(i=0)^nabla
    (Q_i (a_i - epsilon)) / 2
$

To learn more about math, see corresponding chapter.
\end{verbatim}

\pandocbounded{\includesvg[keepaspectratio]{typst-img/12cc318c8438cd8e91706013bbd53fee5ee004620a63348cfe2d7dcc3b8a19d4-1.svg}}


\section{Examples Book LaTeX/book/basics/tutorial/templates.tex}
\title{sitandr.github.io/typst-examples-book/book/basics/tutorial/templates}

\section{\texorpdfstring{\hyperref[templates]{Templates}}{Templates}}\label{templates}

\subsection{\texorpdfstring{\hyperref[templates-1]{Templates}}{Templates}}\label{templates-1}

If you want to reuse styling in other files, you can use the
\emph{template} idiom. Because \texttt{\ }{\texttt{\ set\ }}\texttt{\ }
and \texttt{\ }{\texttt{\ show\ }}\texttt{\ } rules are only active in
their current scope, they will not affect content in a file you imported
your file into. But functions can circumvent this in a predictable way:

\begin{verbatim}
// define a function that:
// - takes content
// - applies styling to it
// - returns the styled content
#let apply-template(body) = [
  #show heading.where(level: 1): emph
  #set heading(numbering: "1.1")
  // ...
  #body
]
\end{verbatim}

This is equivalent to:

\begin{verbatim}
// we can reduce the number of hashes needed here by using scripting mode
// same as above but we exchanged `[...]` for `{...}` to switch from markup
// into scripting mode
#let apply-template(body) = {
  show heading.where(level: 1): emph
  set heading(numbering: "1.1")
  // ...
  body
}
\end{verbatim}

Then in your main file:

\begin{verbatim}
#import "template.typ": apply-template
#show: apply-template
\end{verbatim}

\emph{This will apply a "template" function to the rest of your
document!}

\subsubsection{\texorpdfstring{\hyperref[passing-arguments]{Passing
arguments}}{Passing arguments}}\label{passing-arguments}

\begin{verbatim}
// add optional named arguments
#let apply-template(body, name: "My document") = {
  show heading.where(level: 1): emph
  set heading(numbering: "1.1")

  align(center, text(name, size: 2em))

  body
}
\end{verbatim}

Then, in template file:

\begin{verbatim}
#import "template.typ": apply-template

// `func.with(..)` applies the arguments to the function and returns the new
// function with those defaults applied
#show: apply-template.with(name: "Report")

// it is functionally the same as this
#let new-template(..args) = apply-template(name: "Report", ..args)
#show: new-template
\end{verbatim}

Writing templates is fairly easy if you understand
\href{../scripting/index.html}{scripting} .

See more information about writing templates in
\href{https://typst.app/docs/tutorial/making-a-template/}{Official
Tutorial} .

There is no official repository for templates yet, but there are a
plenty community ones in
\href{https://github.com/qjcg/awesome-typst?ysclid=lj8pur1am7431908794\#general}{awesome-typst}
.


\section{Examples Book LaTeX/book/basics/tutorial/index.tex}
\title{sitandr.github.io/typst-examples-book/book/basics/tutorial/index}

\section{\texorpdfstring{\hyperref[tutorial-by-examples]{Tutorial by
Examples}}{Tutorial by Examples}}\label{tutorial-by-examples}

The first section of Typst Basics is very similar to
\href{https://typst.app/docs/tutorial/}{Official Tutorial} , with more
specialized examples and less words. It is \emph{highly recommended to
read the official tutorial anyway} .


\section{Examples Book LaTeX/book/basics/tutorial/functions.tex}
\title{sitandr.github.io/typst-examples-book/book/basics/tutorial/functions}

\section{\texorpdfstring{\hyperref[functions]{Functions}}{Functions}}\label{functions}

\subsection{\texorpdfstring{\hyperref[functions-1]{Functions}}{Functions}}\label{functions-1}

\begin{verbatim}
Okay, let's now move to more complex things.

First of all, there are *lots of magic* in Typst.
And it major part of it is called "scripting".

To go to scripting mode, type `#` and *some function name*
after that. We will start with _something dull_:

#lorem(50)

_That *function* just generated 50 "Lorem Ipsum" words!_
\end{verbatim}

\pandocbounded{\includesvg[keepaspectratio]{typst-img/036fce36d10e06e8e41be8e77d7d5672f5dfc82c57e7c3ba9b8060d0822ca115-1.svg}}

\subsection{\texorpdfstring{\hyperref[more-functions]{More
functions}}{More functions}}\label{more-functions}

\begin{verbatim}
#underline[functions can do everything!]

#text(orange)[L]ike #text(size: 0.8em)[Really] #sub[E]verything!

#figure(
  caption: [
    This is a screenshot from one of first theses written in Typst. \
    _All these things are written with #text(blue)[custom functions] too._
  ],
  image("../boxes.png", width: 80%)
)

In fact, you can #strong[forget] about markup
and #emph[just write] functions everywhere!

#list[
  All that markup is just a #emph[syntax sugar] over functions!
]
\end{verbatim}

\pandocbounded{\includesvg[keepaspectratio]{typst-img/455e15e83c25259f932178d68517cc012432cb17d072e60c659169470fe191ce-1.svg}}

\subsection{\texorpdfstring{\hyperref[how-to-call-functions]{How to call
functions}}{How to call functions}}\label{how-to-call-functions}

\begin{verbatim}
First, start with `#`. Then write the name.
Finally, write some parentheses and maybe something inside.

You can navigate lots of built-in functions
in #link("https://typst.app/docs/reference/")[Official Reference].

#quote(block: true, attribution: "Typst Examples Book")[
  That's right, links, quotes and lots of
  other document elements are created with functions.
]
\end{verbatim}

\pandocbounded{\includesvg[keepaspectratio]{typst-img/4c63fde73bb1ad0afe1332ab68c5b540ec786c6352a76860f4398fec32034cf0-1.svg}}

\subsection{\texorpdfstring{\hyperref[function-arguments]{Function
arguments}}{Function arguments}}\label{function-arguments}

\begin{verbatim}
There are _two types_ of function arguments:

+ *Positional.* Like `50` in `lorem(50)`.
  Just write them in parentheses and it will be okay.
  If you have many, use commas.
+ *Named.* Like in `#quote(attribution: "Whoever")`.
  Write the value after a name and a colon.

If argument is named, it has some _default value_.
To find out what it is, see
#link("https://typst.app/docs/reference/")[Official Typst Reference].
\end{verbatim}

\pandocbounded{\includesvg[keepaspectratio]{typst-img/d66fb474260490595a207f06c687efcc85808701c39c2a6e8b686bc22ffde279-1.svg}}

\subsection{\texorpdfstring{\hyperref[content]{Content}}{Content}}\label{content}

\begin{verbatim}
The most "universal" type in Typst language is *content*.
Everything you write in the document becomes content.

#[
  But you can explicitly create it with
  _scripting mode_ and *square brackets*.

  There, in square brackets, you can use any markup
  functions or whatever you want.
]
\end{verbatim}

\pandocbounded{\includesvg[keepaspectratio]{typst-img/faf9d7cddd55e68f84d212013a52a724c2ad763f18d83221a99bbd380410d7d1-1.svg}}

\subsection{\texorpdfstring{\hyperref[markup-and-code-modes]{Markup and
code modes}}{Markup and code modes}}\label{markup-and-code-modes}

\begin{verbatim}
When you use `#`, you are "switching" to code mode.
When you use `[]`, you turn back:

// +-- going from markup (the default mode) to scripting for that function
// |                 +-- scripting mode: calling `text`, the last argument is markup
// |     first arg   |
// v     vvvvvvvvv   vvvv
   #rect(width: 5cm, text(red)[hello *world*])
//  ^^^^                       ^^^^^^^^^^^^^ just a markup argument for `text`
//  |
//  +-- calling `rect` in scripting mode, with two arguments: width and other content
\end{verbatim}

\pandocbounded{\includesvg[keepaspectratio]{typst-img/0cabe3da1eb49f805535fb1d7e34a0d6eb1a6c49227b0be98634c6965e892185-1.svg}}

\subsection{\texorpdfstring{\hyperref[passing-content-into-functions]{Passing
content into
functions}}{Passing content into functions}}\label{passing-content-into-functions}

\begin{verbatim}
So what are these square brackets after functions?

If you *write content right after
function, it will be passed as positional argument there*.

#quote(block: true)[
  So #text(red)[_that_] allows me to write
  _literally anything in things
  I pass to #underline[functions]!_
]
\end{verbatim}

\pandocbounded{\includesvg[keepaspectratio]{typst-img/686d2b2a361a60244452ce53bd37ebef0699e92cf962c477bfb62bafdc0f7241-1.svg}}

\subsection{\texorpdfstring{\hyperref[passing-content-part-ii]{Passing
content, part
II}}{Passing content, part II}}\label{passing-content-part-ii}

\begin{verbatim}
So, just to make it clear, when I write

```typ
- #text(red)[red text]
- #text([red text], red)
- #text("red text", red)
//      ^        ^
// Quotes there mean a plain string, not a content!
// This is just text.
```

It all will result in a #text([red text], red).
\end{verbatim}

\pandocbounded{\includesvg[keepaspectratio]{typst-img/4686939b6d0932f1ebebac4111d8f02919dbc16446def7855c521d8dbf293689-1.svg}}


\section{Examples Book LaTeX/book/basics/tutorial/basic_styling.tex}
\title{sitandr.github.io/typst-examples-book/book/basics/tutorial/basic_styling}

\section{\texorpdfstring{\hyperref[basic-styling]{Basic
styling}}{Basic styling}}\label{basic-styling}

\subsection{\texorpdfstring{\hyperref[set-rule]{\texttt{\ }{\texttt{\ Set\ }}\texttt{\ }
rule}}{  Set   rule}}\label{set-rule}

\begin{verbatim}
#set page(width: 15cm, margin: (left: 4cm, right: 4cm))

That was great, but using functions everywhere, especially
with many arguments every time is awfully cumbersome.

That's why Typst has _rules_. No, not for you, for the document.

#set par(justify: true)

And the first rule we will consider there is `set` rule.
As you see, I've just used it on `par` (which is short from paragraph)
and now all paragraphs became _justified_.

It will apply to all paragraphs after the rule,
but will work only in it's _scope_ (we will discuss them later).

#par(justify: false)[
  Of course, you can override a `set` rule.
  This rule just sets the _default value_
  of an argument of an element.
]

By the way, at first line of this snippet
I've reduced page size to make justifying more visible,
also increasing margins to add blank space on left and right.
\end{verbatim}

\pandocbounded{\includesvg[keepaspectratio]{typst-img/cee42a8b1274afa36891438d4b1611eb55b2cd8bb4546df47128a7d3eb66653b-1.svg}}

\subsection{\texorpdfstring{\hyperref[a-bit-about-length-units]{A bit
about length
units}}{A bit about length units}}\label{a-bit-about-length-units}

\begin{verbatim}
Before we continue with rules, we should talk about length. There are several absolute length units in Typst:

#set rect(height: 1em)

#table(
  columns: 2,
  [Points], rect(width: 72pt),
  [Millimeters], rect(width: 25.4mm),
  [Centimeters], rect(width: 2.54cm),
  [Inches], rect(width: 1in),
  [Relative to font size], rect(width: 6.5em)
)

`1 em` = current font size. \
It is a very convenient unit,
so we are going to use it a lot
\end{verbatim}

\pandocbounded{\includesvg[keepaspectratio]{typst-img/5f8abc94a3d9df0e16f78c258e7f487d5698b4c96491300b3a48ad8e685534bc-1.svg}}

\subsection{\texorpdfstring{\hyperref[setting-something-else]{Setting
something else}}{Setting something else}}\label{setting-something-else}

Of course, you can use \texttt{\ }{\texttt{\ set\ }}\texttt{\ } rule
with all built-in functions and all their named arguments to make some
argument "default".

For example, let\textquotesingle s make all quotes in this snippet
authored by the book:

\begin{verbatim}
#set quote(block: true, attribution: [Typst Examples Book])

#quote[
  Typst is great!
]

#quote[
  The problem with quotes on the internet is
  that it is hard to verify their authenticity.
]
\end{verbatim}

\pandocbounded{\includesvg[keepaspectratio]{typst-img/c34c25cad05b7c20b6e0f146002886a1de65b61f48666cfec3d3494bd694a641-1.svg}}

\subsection{\texorpdfstring{\hyperref[opinionated-defaults]{Opinionated
defaults}}{Opinionated defaults}}\label{opinionated-defaults}

That allows you to set Typst default styling as you want it:

\begin{verbatim}
#set par(justify: true)
#set list(indent: 1em)
#set enum(indent: 1em)
#set page(numbering: "1")

- List item
- List item

+ Enum item
+ Enum item
\end{verbatim}

\pandocbounded{\includesvg[keepaspectratio]{typst-img/773d68bc55eb89f119ad07b882eae5fd31868d8a1bb3d4963573ec80fb1c7466-1.svg}}

Don\textquotesingle t complain about bad defaults!
\texttt{\ }{\texttt{\ Set\ }}\texttt{\ } your own.

\subsection{\texorpdfstring{\hyperref[numbering]{Numbering}}{Numbering}}\label{numbering}

\begin{verbatim}
= Numbering

Some of elements have a property called "numbering".
They accept so-called "numbering patterns" and
are very useful with set rules. Let's see what I mean.

#set heading(numbering: "I.1:")

= This is first level
= Another first
== Second
== Another second
=== Now third
== And second again
= Now returning to first
= These are actual romanian numerals
\end{verbatim}

\pandocbounded{\includesvg[keepaspectratio]{typst-img/39fb958032888b1e41da849152fed716b6f590eed3ea975b051ab786fac4ce5c-1.svg}}

Of course, there are lots of other cool properties that can be
\emph{set} , so feel free to dive into
\href{https://typst.app/docs/reference/}{Official Reference} and explore
them!

And now we are moving into something much more interesting\ldots{}


\section{Examples Book LaTeX/book/basics/tutorial/advanced_styling.tex}
\title{sitandr.github.io/typst-examples-book/book/basics/tutorial/advanced_styling}

\section{\texorpdfstring{\hyperref[advanced-styling]{Advanced
styling}}{Advanced styling}}\label{advanced-styling}

\subsection{\texorpdfstring{\hyperref[the-show-rule]{The
\texttt{\ }{\texttt{\ show\ }}\texttt{\ }
rule}}{The   show   rule}}\label{the-show-rule}

\begin{verbatim}
Advanced styling comes with another rule. The _`show` rule_.

Now please compare the source code and the output.

#show "Be careful": strong[Play]

This is a very powerful thing, sometimes even too powerful.
Be careful with it.

#show "it is holding me hostage": text(green)[I'm fine]

Wait, what? I told you "Be careful!", not "Play!".

Help, it is holding me hostage.
\end{verbatim}

\pandocbounded{\includesvg[keepaspectratio]{typst-img/8a9ac38769d4ac7b42a2755047d0cd5a6404ad26e9e7f5b72b6984fa67abadf9-1.svg}}

\subsection{\texorpdfstring{\hyperref[now-a-bit-more-serious]{Now a bit
more serious}}{Now a bit more serious}}\label{now-a-bit-more-serious}

\begin{verbatim}
Show rule is a powerful thing that takes a _selector_
and what to apply to it. After that it will apply to
all elements it can find.

It may be extremely useful like that:

#show emph: set text(blue)

Now if I want to _emphasize_ something,
it will be both _emphasized_ and _blue_.
Isn't that cool?
\end{verbatim}

\pandocbounded{\includesvg[keepaspectratio]{typst-img/657acaf5c4ca684408bbc6fe0dec4c74b9fa58d24805ec975be1382aa7bf959c-1.svg}}

\subsection{\texorpdfstring{\hyperref[about-syntax]{About
syntax}}{About syntax}}\label{about-syntax}

\begin{verbatim}
Sometimes show rules may be confusing. They may seem very diverse, but in fact they all are quite the same! So

// actually, this is the same as
// redify = text.with(red)
// `with` creates a new function with this argument already set
#let redify(string) = text(red, string)

// and this is the same as
// framify = rect.with(stroke: orange)
#let framify(object) = rect(object, stroke: orange)

// set default color of text blue for all following text
#show: set text(blue)

Blue text.

// wrap everything into a frame
#show: framify

Framed text.

// it's the same, just creating new function that calls framify
#show: a => framify(a)

Double-framed.

// apply function to `the`
#show "the": redify
// set text color for all the headings
#show heading: set text(purple)

= Conclusion

All these rules do basically the same!
\end{verbatim}

\pandocbounded{\includesvg[keepaspectratio]{typst-img/2dfcde68345d3fa276b99a1f308343118c6eeae09fd106389a8fc488d7244ebb-1.svg}}

\subsection{\texorpdfstring{\hyperref[blocks]{Blocks}}{Blocks}}\label{blocks}

One of the most important usages is that you can set up all spacing
using blocks. Like every element with text contains text that can be set
up, every \emph{block element} contains blocks:

\begin{verbatim}
Text before
= Heading
Text after

#show heading: set block(spacing: 0.5em)

Text before
= Heading
Text after
\end{verbatim}

\pandocbounded{\includesvg[keepaspectratio]{typst-img/7891207932d0918c88b5804b3a7ee051ce5dda93081f8999eb0f7ebaee48400a-1.svg}}

\subsection{\texorpdfstring{\hyperref[selector]{Selector}}{Selector}}\label{selector}

\begin{verbatim}
So show rule can accept _selectors_.

There are lots of different selector types,
for example

- element functions
- strings
- regular expressions
- field filters

Let's see example of the latter:

#show heading.where(level: 1): set align(center)

= Title
== Small title

Of course, you can set align by hand,
no need to use show rules
(but they are very handy!):

#align(center)[== Centered small title]
\end{verbatim}

\pandocbounded{\includesvg[keepaspectratio]{typst-img/f41f337dd75b55211dd8d16e2682132c1ffb1ef19f774ba6cafc94cae090ec75-1.svg}}

\subsection{\texorpdfstring{\hyperref[custom-formatting]{Custom
formatting}}{Custom formatting}}\label{custom-formatting}

\begin{verbatim}
Let's try now writing custom functions.
It is very easy, see yourself:

// "it" is a heading, we take it and output things in braces
#show heading: it => {
  // center it
  set align(center)
  // set size and weight
  set text(12pt, weight: "regular")
  // see more about blocks and boxes
  // in corresponding chapter
  block(smallcaps(it.body))
}

= Smallcaps heading
\end{verbatim}

\pandocbounded{\includesvg[keepaspectratio]{typst-img/a5c37bce3cf9a077a4eb62a4d95f89584b5ef8acee279b81de6019d0e5768ba0-1.svg}}

\subsection{\texorpdfstring{\hyperref[setting-spacing]{Setting
spacing}}{Setting spacing}}\label{setting-spacing}

TODO: explain block spacing for common elements

\subsection{\texorpdfstring{\hyperref[formatting-to-get-an-article-look]{Formatting
to get an "article
look"}}{Formatting to get an "article look"}}\label{formatting-to-get-an-article-look}

\begin{verbatim}
#set page(
  // Header is that small thing on top
  header: align(
    right + horizon,
    [Some header there]
  ),
  height: 12cm
)

#align(center, text(17pt)[
  *Important title*
])

#grid(
  columns: (1fr, 1fr),
  align(center)[
    Some author \
    Some Institute \
    #link("mailto:some@mail.edu")
  ],
  align(center)[
    Another author \
    Another Institute \
    #link("mailto:another@mail.edu")
  ]
)

Now let's split text into two columns:

#show: rest => columns(2, rest)

#show heading.where(
  level: 1
): it => block(width: 100%)[
  #set align(center)
  #set text(12pt, weight: "regular")
  #smallcaps(it.body)
]

#show heading.where(
  level: 2
): it => text(
  size: 11pt,
  weight: "regular",
  style: "italic",
  it.body + [.],
)

// Now let's fill it with words:

= Heading
== Small heading
#lorem(10)
== Second subchapter
#lorem(10)
= Second heading
#lorem(40)

== Second subchapter
#lorem(40)
\end{verbatim}

\pandocbounded{\includesvg[keepaspectratio]{typst-img/76ee0cca809299df178ec9d94371c01031d1808a700b39deac5245dd6b83157f-1.svg}}




\section{Combined Examples Book LaTeX/book/basics/must_know.tex}
\section{Examples Book LaTeX/book/basics/must_know/spacing.tex}
\title{sitandr.github.io/typst-examples-book/book/basics/must_know/spacing}

\section{\texorpdfstring{\hyperref[using-spacing]{Using
spacing}}{Using spacing}}\label{using-spacing}

Most time you will pass spacing into functions. There are special
function fields that take only \emph{size} . They are usually called
like
\texttt{\ }{\texttt{\ width,\ length,\ in(out)set,\ spacing\ }}\texttt{\ }
and so on.

Like in CSS, one of the ways to set up spacing in Typst is setting
margins and padding of elements. However, you can also insert spacing
directly using functions \texttt{\ }{\texttt{\ h\ }}\texttt{\ }
(horizontal spacing) and \texttt{\ }{\texttt{\ v\ }}\texttt{\ }
(vertical spacing).

\begin{quote}
Links to reference: \href{https://typst.app/docs/reference/layout/h/}{h}
, \href{https://typst.app/docs/reference/layout/v/}{v} .
\end{quote}

\begin{verbatim}
Horizontal #h(1cm) spacing.
#v(1cm)
And some vertical too!
\end{verbatim}

\pandocbounded{\includesvg[keepaspectratio]{typst-img/47b3ea7d16575780e489790177df9a624ad3c6c669594baa4127c1db516ebc94-1.svg}}

\section{\texorpdfstring{\hyperref[absolute-length-units]{Absolute
length units}}{Absolute length units}}\label{absolute-length-units}

\begin{quote}
Link to
\href{https://typst.app/docs/reference/layout/length/}{reference}
\end{quote}

Absolute length (aka just "length") units are not affected by outer
content and size of parent.

\begin{verbatim}
#set rect(height: 1em)
#table(
  columns: 2,
  [Points], rect(width: 72pt),
  [Millimeters], rect(width: 25.4mm),
  [Centimeters], rect(width: 2.54cm),
  [Inches], rect(width: 1in),
)
\end{verbatim}

\pandocbounded{\includesvg[keepaspectratio]{typst-img/073ad26fe313743ab62dca82f30208dbf2d57ff354d5c37f0b6d4c063dc37d76-1.svg}}

\subsection{\texorpdfstring{\hyperref[relative-to-current-font-size]{Relative
to current font
size}}{Relative to current font size}}\label{relative-to-current-font-size}

\texttt{\ }{\texttt{\ 1em\ =\ 1\ current\ font\ size\ }}\texttt{\ } :

\begin{verbatim}
#set rect(height: 1em)
#table(
  columns: 2,
  [Centimeters], rect(width: 2.54cm),
  [Relative to font size], rect(width: 6.5em)
)

Double font size: #box(stroke: red, baseline: 40%, height: 2em, width: 2em)
\end{verbatim}

\pandocbounded{\includesvg[keepaspectratio]{typst-img/7d62c9e2540f8bce40d8a3fc65a2779b161eb6b5b5682cf87247fee7f14145c2-1.svg}}

It is a very convenient unit, so it is used a lot in Typst.

\subsection{\texorpdfstring{\hyperref[combined]{Combined}}{Combined}}\label{combined}

\begin{verbatim}
Combined: #box(rect(height: 5pt + 1em))

#(5pt + 1em).abs
#(5pt + 1em).em
\end{verbatim}

\pandocbounded{\includesvg[keepaspectratio]{typst-img/c8a0cae6047f35c85c41ac44ff2a6b0d28a28d0e097ca61b367202f9a361136e-1.svg}}

\section{\texorpdfstring{\hyperref[ratio-length]{Ratio
length}}{Ratio length}}\label{ratio-length}

\begin{quote}
Link to \href{https://typst.app/docs/reference/layout/ratio/}{reference}
\end{quote}

\texttt{\ }{\texttt{\ 1\%\ =\ 1\%\ from\ parent\ size\ in\ that\ dimension\ }}\texttt{\ }

\begin{verbatim}
This line width is 50% of available page size (without margins):

#line(length: 50%)

This line width is 50% of the box width: #box(stroke: red, width: 4em, inset: (y: 0.5em), line(length: 50%))
\end{verbatim}

\pandocbounded{\includesvg[keepaspectratio]{typst-img/d478cb8be0a049380479b634cae709dc1e1ed406d323ecb1edbca1e582d7eafe-1.svg}}

\section{\texorpdfstring{\hyperref[relative-length]{Relative
length}}{Relative length}}\label{relative-length}

\begin{quote}
Link to
\href{https://typst.app/docs/reference/layout/relative/}{reference}
\end{quote}

You can \emph{combine} absolute and ratio lengths into \emph{relative
length} :

\begin{verbatim}
#rect(width: 100% - 50pt)

#(100% - 50pt).length \
#(100% - 50pt).ratio
\end{verbatim}

\pandocbounded{\includesvg[keepaspectratio]{typst-img/6b72620a1972e758e55ef1ecf49d3e843095037399ed4dd2dfcd262ebbbe803f-1.svg}}

\section{\texorpdfstring{\hyperref[fractional-length]{Fractional
length}}{Fractional length}}\label{fractional-length}

\begin{quote}
Link to
\href{https://typst.app/docs/reference/layout/fraction/}{reference}
\end{quote}

Single fraction length just takes \emph{maximum size possible} to fill
the parent:

\begin{verbatim}
Left #h(1fr) Right

#rect(height: 1em)[
  #h(1fr)
]
\end{verbatim}

\pandocbounded{\includesvg[keepaspectratio]{typst-img/b9c91f53b684699fff70c6889c8a47fccc57c5c540d7629b93c51a797eb2ef3c-1.svg}}

There are not many places you can use fractions, mainly those are
\texttt{\ }{\texttt{\ h\ }}\texttt{\ } and
\texttt{\ }{\texttt{\ v\ }}\texttt{\ } .

\subsection{\texorpdfstring{\hyperref[several-fractions]{Several
fractions}}{Several fractions}}\label{several-fractions}

If you use several fractions inside one parent, they will take all
remaining space \emph{proportional to their number} :

\begin{verbatim}
Left #h(1fr) Left-ish #h(2fr) Right
\end{verbatim}

\pandocbounded{\includesvg[keepaspectratio]{typst-img/45182cbcecf395256d133af78fccacd9d48e29073672317744cb17340d0bafd8-1.svg}}

\subsection{\texorpdfstring{\hyperref[nested-layout]{Nested
layout}}{Nested layout}}\label{nested-layout}

Remember that fractions work in parent only, don\textquotesingle t
\emph{rely on them in nested layout} :

\begin{verbatim}
Word: #h(1fr) #box(height: 1em, stroke: red)[
  #h(2fr)
]
\end{verbatim}

\pandocbounded{\includesvg[keepaspectratio]{typst-img/0c7ed8b25ea7e39a0907b1105b82027a0fb8b921b28978f30692f6c693bea5f7-1.svg}}


\section{Examples Book LaTeX/book/basics/must_know/box_block.tex}
\title{sitandr.github.io/typst-examples-book/book/basics/must_know/box_block}

\section{\texorpdfstring{\hyperref[boxing--blocking]{Boxing \&
Blocking}}{Boxing \& Blocking}}\label{boxing--blocking}

\begin{verbatim}
You can use boxes to wrap anything
into text: #box(image("../tiger.jpg", height: 2em)).

Blocks will always be "separate paragraphs".
They will not fit into a text: #block(image("../tiger.jpg", height: 2em))
\end{verbatim}

\pandocbounded{\includesvg[keepaspectratio]{typst-img/8e3bd89485b00259666bd636cf28586f92db9c3c3922f0adcdad765ee66a06b1-1.svg}}

Both have similar useful properties:

\begin{verbatim}
#box(stroke: red, inset: 1em)[Box text]
#block(stroke: red, inset: 1em)[Block text]
\end{verbatim}

\pandocbounded{\includesvg[keepaspectratio]{typst-img/9e3562619cb8a31b3d2311f53c3815a214f081e033a564e63dc003dfbc50d68d-1.svg}}

\subsection{\texorpdfstring{\hyperref[rect]{\texttt{\ }{\texttt{\ rect\ }}\texttt{\ }}}{  rect  }}\label{rect}

There is also \texttt{\ }{\texttt{\ rect\ }}\texttt{\ } that works like
\texttt{\ }{\texttt{\ block\ }}\texttt{\ } , but has useful default
inset and stroke:

\begin{verbatim}
#rect[Block text]
\end{verbatim}

\pandocbounded{\includesvg[keepaspectratio]{typst-img/c778d1e94a3663a4f258985368c02e294a1333554c550b6cfe0465275a2eef0f-1.svg}}

\subsection{\texorpdfstring{\hyperref[figures]{Figures}}{Figures}}\label{figures}

For the purposes of adding a \emph{figure} to your document, use
\texttt{\ }{\texttt{\ figure\ }}\texttt{\ } function.
Don\textquotesingle t try to use boxes or blocks there.

Figures are that things like centered images (probably with captions),
tables, even code.

\begin{verbatim}
@tiger shows a tiger. Tigers
are animals.

#figure(
  image("../tiger.jpg", width: 80%),
  caption: [A tiger.],
) <tiger>
\end{verbatim}

\pandocbounded{\includesvg[keepaspectratio]{typst-img/09a8b5b3c3bfffd81be7f34c31cc93ca5f8341b2594d022b2b92ac285aeb959d-1.svg}}

In fact, you can put there anything you want:

\begin{verbatim}
They told me to write a letter to you. Here it is:

#figure(
  text(size: 5em)[I],
  caption: [I'm cool, right?],
) 
\end{verbatim}

\pandocbounded{\includesvg[keepaspectratio]{typst-img/e009534c4572064346490dfac659ff94a5a11d7f46af7a2b46c2136d206088c6-1.svg}}


\section{Examples Book LaTeX/book/basics/must_know/index.tex}
\title{sitandr.github.io/typst-examples-book/book/basics/must_know/index}

\section{\texorpdfstring{\hyperref[must-know]{Must-know}}{Must-know}}\label{must-know}

This section contains things, that are not general enough to be part of
"tutorial", but still are very important to know for proper typesetting.

Feel free to skip through things you are sure you will not use.


\section{Examples Book LaTeX/book/basics/must_know/place.tex}
\title{sitandr.github.io/typst-examples-book/book/basics/must_know/place}

\section{\texorpdfstring{\hyperref[placing-moving-scale--hide]{Placing,
Moving, Scale \&
Hide}}{Placing, Moving, Scale \& Hide}}\label{placing-moving-scale--hide}

This is \textbf{a very important section} if you want to do arbitrary
things with layout, create custom elements and hacking a way around
current Typst limitations.

TODO: WIP, add text and better examples

\section{\texorpdfstring{\hyperref[place]{Place}}{Place}}\label{place}

\emph{Ignore layout} , just put some object somehow relative to parent
and current position. The placed object \emph{will not} affect layouting

\begin{quote}
Link to \href{https://typst.app/docs/reference/layout/place/}{reference}
\end{quote}

\begin{verbatim}
#set page(height: 60pt)
Hello, world!

#place(
  top + right, // place at the page right and top
  square(
    width: 20pt,
    stroke: 2pt + blue
  ),
)
\end{verbatim}

\pandocbounded{\includesvg[keepaspectratio]{typst-img/e0d4c250d0f288e1a110ebddcb06149e0acd11b626a0ccb0ca9feb1c1d7be359-1.svg}}

\subsubsection{\texorpdfstring{\hyperref[basic-floating-with-place]{Basic
floating with
place}}{Basic floating with place}}\label{basic-floating-with-place}

\begin{verbatim}
#set page(height: 150pt)
#let note(where, body) = place(
  center + where,
  float: true,
  clearance: 6pt,
  rect(body),
)

#lorem(10)
#note(bottom)[Bottom 1]
#note(bottom)[Bottom 2]
#lorem(40)
#note(top)[Top]
#lorem(10)
\end{verbatim}

\pandocbounded{\includesvg[keepaspectratio]{typst-img/b770cfef024690b5fc7ab82458797d6cfab0c5cc8f52078ecf2d61be17c13acc-1.svg}}

\pandocbounded{\includesvg[keepaspectratio]{typst-img/b770cfef024690b5fc7ab82458797d6cfab0c5cc8f52078ecf2d61be17c13acc-2.svg}}

\subsubsection{\texorpdfstring{\hyperref[dx-dy]{dx,
dy}}{dx, dy}}\label{dx-dy}

Manually change position by
\texttt{\ }{\texttt{\ (dx,\ dy)\ }}\texttt{\ } relative to intended.

\begin{verbatim}
#set page(height: 100pt)
#for i in range(16) {
  let amount = i * 4pt
  place(center, dx: amount - 32pt, dy: amount)[A]
}
\end{verbatim}

\pandocbounded{\includesvg[keepaspectratio]{typst-img/12464f1a2cfe81fb04623033345f3f88ff598af5dc77de378b9d7cf88fc1d5b3-1.svg}}

\section{\texorpdfstring{\hyperref[move]{Move}}{Move}}\label{move}

\begin{quote}
Link to \href{https://typst.app/docs/reference/layout/move/}{reference}
\end{quote}

\begin{verbatim}
#rect(inset: 0pt, move(
  dx: 6pt, dy: 6pt,
  rect(
    inset: 8pt,
    fill: white,
    stroke: black,
    [Abra cadabra]
  )
))
\end{verbatim}

\pandocbounded{\includesvg[keepaspectratio]{typst-img/3292aebf7b633a2d9574027f50867d723d80850e046a101b9df5ab5143eb8a8d-1.svg}}

\section{\texorpdfstring{\hyperref[scale]{Scale}}{Scale}}\label{scale}

Scale content \emph{without affecting the layout} .

\begin{quote}
Link to \href{https://typst.app/docs/reference/layout/scale/}{reference}
\end{quote}

\begin{verbatim}
#scale(x: -100%)[This is mirrored.]
\end{verbatim}

\pandocbounded{\includesvg[keepaspectratio]{typst-img/401c8cd6f306771a3b12432c3c51e097a3ec1d12656c131c0043a12c4c1c3a0e-1.svg}}

\begin{verbatim}
A#box(scale(75%)[A])A \
B#box(scale(75%, origin: bottom + left)[B])B
\end{verbatim}

\pandocbounded{\includesvg[keepaspectratio]{typst-img/204b55690645eb6cc623c8d2d74b5521d72e4ba38d58ea40ea5e2d4354a01836-1.svg}}

\section{\texorpdfstring{\hyperref[hide]{Hide}}{Hide}}\label{hide}

Don\textquotesingle t show content, but leave empty space there.

\begin{quote}
Link to \href{https://typst.app/docs/reference/layout/hide/}{reference}
\end{quote}

\begin{verbatim}
Hello Jane \
#hide[Hello] Joe
\end{verbatim}

\pandocbounded{\includesvg[keepaspectratio]{typst-img/610672d5e43baa3ce94fe61f8d6dd0307e405c785639359c6a9e84bdd66884ad-1.svg}}


\section{Examples Book LaTeX/book/basics/must_know/project_struct.tex}
\title{sitandr.github.io/typst-examples-book/book/basics/must_know/project_struct}

\section{\texorpdfstring{\hyperref[project-structure]{Project
structure}}{Project structure}}\label{project-structure}

\subsection{\texorpdfstring{\hyperref[large-document]{Large
document}}{Large document}}\label{large-document}

Once the document becomes large enough, it becomes harder to navigate
it. If you haven\textquotesingle t reached that size yet, you can ignore
that section.

For managing that I would recommend splitting your document into
\emph{chapters} . It is just a way to work with this, but once you
understand how it works, you can do anything you want.

Let\textquotesingle s say you have two chapters, then the recommended
structure will look like this:

\begin{verbatim}
#import "@preview/treet:0.1.1": *

#show list: tree-list
#set par(leading: 0.8em)
#show list: set text(font: "DejaVu Sans Mono", size: 0.8em)
- chapters/
  - chapter_1.typ
  - chapter_2.typ
- main.typ 👁 #text(gray)[← document entry point]
- template.typ
\end{verbatim}

\pandocbounded{\includesvg[keepaspectratio]{typst-img/291489e71b40beea77872ad05adb609349872e9a11fc3a9c3f2008c88e37c9d5-1.svg}}

The exact file names are up to you.

Let\textquotesingle s see what to put in each of these files.

\subsubsection{\texorpdfstring{\hyperref[template]{Template}}{Template}}\label{template}

In the "template" file goes \emph{all useful functions and variables}
you will use across the chapters. If you have your own template or want
to write one, you can write it there.

\begin{verbatim}
// template.typ

#let template = doc => {
    set page(header: "My super document")
    show "physics": "magic"
    doc
}

#let info-block = block.with(stroke: blue, fill: blue.lighten(70%))
#let author = "@sitandr"
\end{verbatim}

\subsubsection{\texorpdfstring{\hyperref[main]{Main}}{Main}}\label{main}

\textbf{This file should be compiled} to get the whole compiled
document.

\begin{verbatim}
// main.typ

#import "template.typ": *
// if you have a template
#show: template

= This is the document title

// some additional formatting

#show emph: set text(blue)

// but don't define functions or variables there!
// chapters will not see it

// Now the chapters themselves as some Typst content
#include("chapters/chapter_1.typ")
#include("chapters/chapter_1.typ")
\end{verbatim}

\subsubsection{\texorpdfstring{\hyperref[chapter]{Chapter}}{Chapter}}\label{chapter}

\begin{verbatim}
// chapter_1.typ

#import "../template.typ": *

That's just content with _styling_ and blocks:

#infoblock[Some information].

// just any content you want to include in the document
\end{verbatim}

\subsection{\texorpdfstring{\hyperref[notes]{Notes}}{Notes}}\label{notes}

Note that modules in Typst can see only what they created themselves or
imported. Anything else is invisible for them. That\textquotesingle s
why you need \texttt{\ }{\texttt{\ template.typ\ }}\texttt{\ } file to
define all functions within.

That means chapters \emph{don\textquotesingle t see each other either} ,
only what is in the template.

\subsection{\texorpdfstring{\hyperref[cyclic-imports]{Cyclic
imports}}{Cyclic imports}}\label{cyclic-imports}

\textbf{Important:} Typst \emph{forbids} cyclic imports. That means you
can\textquotesingle t import
\texttt{\ }{\texttt{\ chapter\_1\ }}\texttt{\ } from
\texttt{\ }{\texttt{\ chapter\_2\ }}\texttt{\ } and
\texttt{\ }{\texttt{\ chapter\_2\ }}\texttt{\ } from
\texttt{\ }{\texttt{\ chapter\_1\ }}\texttt{\ } at the same time!

But the good news is that you can always create some other file to
import variable from.


\section{Examples Book LaTeX/book/basics/must_know/tables.tex}
\title{sitandr.github.io/typst-examples-book/book/basics/must_know/tables}

\section{\texorpdfstring{\hyperref[tables-and-grids]{Tables and
grids}}{Tables and grids}}\label{tables-and-grids}

While tables are not that necessary to know if you don\textquotesingle t
plan to use them in your documents, grids may be very useful for
\emph{document layout} . We will use both of them them in the book
later.

Let\textquotesingle s not bother with copying examples from official
documentation. Just make sure to skim through it, okay?

\subsection{\texorpdfstring{\hyperref[basic-snippets]{Basic
snippets}}{Basic snippets}}\label{basic-snippets}

\subsubsection{\texorpdfstring{\hyperref[spreading]{Spreading}}{Spreading}}\label{spreading}

Spreading operators (see \href{../scripting/arguments.html}{there} ) may
be especially useful for the tables:

\begin{verbatim}
#set text(size: 9pt)

#let yield_cells(n) = {
  for i in range(0, n + 1) {
    for j in range(0, n + 1) {
      let product = if i * j != 0 {
        // math is used for the better look 
        if j <= i { $#{ j * i }$ } 
        else {
          // upper part of the table
          text(gray.darken(50%), str(i * j))
        }
      } else {
        if i == j {
          // the top right corner 
          $times$
        } else {
          // on of them is zero, we are at top/left
          $#{i + j}$
        }
      }
      // this is an array, for loops merge them together
      // into one large array of cells
      (
        table.cell(
          fill: if i == j and j == 0 { orange } // top right corner
          else if i == j { yellow } // the diagonal
          else if i * j == 0 { blue.lighten(50%) }, // multipliers
          product,),
      )
    }
  }
}

#let n = 10
#table(
  columns: (0.6cm,) * (n + 1), rows: (0.6cm,) * (n + 1), align: center + horizon, inset: 3pt, ..yield_cells(n),
)
\end{verbatim}

\pandocbounded{\includesvg[keepaspectratio]{typst-img/0640c1d0e5f79bdcb5e60f7675ff1b1eb18810078f5bbbdfaf1c5648b987706e-1.svg}}

\subsubsection{\texorpdfstring{\hyperref[highlighting-table-row]{Highlighting
table row}}{Highlighting table row}}\label{highlighting-table-row}

\begin{verbatim}
#table(
  columns: 2,
  fill: (x, y) => if y == 2 { highlight.fill },
  [A], [B],
  [C], [D],
  [E], [F],
  [G], [H],
)
\end{verbatim}

\pandocbounded{\includesvg[keepaspectratio]{typst-img/4ff8cbb75f85dbab08a336be31115bcb4cb8ca505799641534d937d444e88082-1.svg}}

For individual cells, use

\begin{verbatim}
#table(
  columns: 2,
  [A], [B],
  table.cell(fill: yellow)[C], table.cell(fill: yellow)[D],
  [E], [F],
  [G], [H],
)
\end{verbatim}

\pandocbounded{\includesvg[keepaspectratio]{typst-img/07676a86d4643ff83988c0907aa17995b3d1f8fa7b5be4f11959551afd674bc9-1.svg}}

\subsubsection{\texorpdfstring{\hyperref[splitting-tables]{Splitting
tables}}{Splitting tables}}\label{splitting-tables}

Tables are split between pages automatically.

\begin{verbatim}
#set page(height: 8em)
#(
table(
  columns: 5,
  [Aligner], [publication], [Indexing], [Pairwise alignment], [Max. read length  (bp)],
  [BWA], [2009], [BWT-FM], [Semi-Global], [125],
  [Bowtie], [2009], [BWT-FM], [HD], [76],
  [CloudBurst], [2009], [Hashing], [Landau-Vishkin], [36],
  [GNUMAP], [2009], [Hashing], [NW], [36]
  )
)
\end{verbatim}

\pandocbounded{\includesvg[keepaspectratio]{typst-img/34794c27fefc5c307a1dfdc9ad7958c1dcca0ff8fb64962047051c6a216e0ff7-1.svg}}

\pandocbounded{\includesvg[keepaspectratio]{typst-img/34794c27fefc5c307a1dfdc9ad7958c1dcca0ff8fb64962047051c6a216e0ff7-2.svg}}

However, if you want to make it breakable inside other element,
you\textquotesingle ll have to make that element breakable too:

\begin{verbatim}
#set page(height: 8em)
// Without this, the table fails to split upon several pages
#show figure: set block(breakable: true)
#figure(
table(
  columns: 5,
  [Aligner], [publication], [Indexing], [Pairwise alignment], [Max. read length  (bp)],
  [BWA], [2009], [BWT-FM], [Semi-Global], [125],
  [Bowtie], [2009], [BWT-FM], [HD], [76],
  [CloudBurst], [2009], [Hashing], [Landau-Vishkin], [36],
  [GNUMAP], [2009], [Hashing], [NW], [36]
  )
)
\end{verbatim}

\pandocbounded{\includesvg[keepaspectratio]{typst-img/5be04bf8770a33256599791fb50751bcb24fa5108c13d0e5e2807b675fed00fb-1.svg}}

\pandocbounded{\includesvg[keepaspectratio]{typst-img/5be04bf8770a33256599791fb50751bcb24fa5108c13d0e5e2807b675fed00fb-2.svg}}




\section{Combined Examples Book LaTeX/book/basics/states.tex}
\section{Examples Book LaTeX/book/basics/states/counters.tex}
\title{sitandr.github.io/typst-examples-book/book/basics/states/counters}

\section{\texorpdfstring{\hyperref[counters]{Counters}}{Counters}}\label{counters}

This section is outdated. It may be still useful, but it is strongly
recommended to study new context system (using the reference).

Counters are special states that \emph{count} elements of some type. As
with states, you can create your own with identifier strings.

\emph{Important:} to initiate counters of elements, you need to
\emph{set numbering for them} .

\subsection{\texorpdfstring{\hyperref[states-methods]{States
methods}}{States methods}}\label{states-methods}

Counters are states, so they can do all things states can do.

\begin{verbatim}
#set heading(numbering: "1.")

= Background
#counter(heading).update(3)
#counter(heading).update(n => n * 2)

== Analysis
Current heading number: #counter(heading).display().
\end{verbatim}

\pandocbounded{\includesvg[keepaspectratio]{typst-img/c57c9907a5f238f0b5eee74f8c23c57a5e2d5b0c9cbf7ebd1befdfcbd33289df-1.svg}}

\begin{verbatim}
#let mine = counter("mycounter")
#mine.display()

#mine.step()
#mine.display()

#mine.update(c => c * 3)
#mine.display()
\end{verbatim}

\pandocbounded{\includesvg[keepaspectratio]{typst-img/876103777c9564f0bb524f83a988a6d444c4e889baed31ee960548d90f3233e2-1.svg}}

\subsection{\texorpdfstring{\hyperref[displaying-counters]{Displaying
counters}}{Displaying counters}}\label{displaying-counters}

\begin{verbatim}
#set heading(numbering: "1.")

= Introduction
Some text here.

= Background
The current value is:
#counter(heading).display()

Or in roman numerals:
#counter(heading).display("I")
\end{verbatim}

\pandocbounded{\includesvg[keepaspectratio]{typst-img/1ac65f4be42131b3cca1d7c56c6c60c3932a703e5e499c1c5cb874458028abea-1.svg}}

Counters also support displaying \emph{both current and final values}
out-of-box:

\begin{verbatim}
#set heading(numbering: "1.")

= Introduction
Some text here.

#counter(heading).display(both: true) \
#counter(heading).display("1 of 1", both: true) \
#counter(heading).display(
  (num, max) => [#num of #max],
   both: true
)

= Background
The current value is: #counter(heading).display()
\end{verbatim}

\pandocbounded{\includesvg[keepaspectratio]{typst-img/af9d0da905bbb2215461b07b39653ef3890ff11a364afe018dae4ce4216f4961-1.svg}}

\subsection{\texorpdfstring{\hyperref[step]{Step}}{Step}}\label{step}

That\textquotesingle s quite easy, for counters you can increment value
using \texttt{\ }{\texttt{\ step\ }}\texttt{\ } . It works the same way
as \texttt{\ }{\texttt{\ update\ }}\texttt{\ } .

\begin{verbatim}
#set heading(numbering: "1.")

= Introduction
#counter(heading).step()

= Analysis
Let's skip 3.1.
#counter(heading).step(level: 2)

== Analysis
At #counter(heading).display().
\end{verbatim}

\pandocbounded{\includesvg[keepaspectratio]{typst-img/12446a2258e9862d8df8b6b250ff14efbb9c35da165a2a04e8c4aa12c9b68cdf-1.svg}}

\subsection{\texorpdfstring{\hyperref[you-can-use-counters-in-your-functions]{You
can use counters in your
functions:}}{You can use counters in your functions:}}\label{you-can-use-counters-in-your-functions}

\begin{verbatim}
#let c = counter("theorem")
#let theorem(it) = block[
  #c.step()
  *Theorem #c.display():*
  #it
]

#theorem[$1 = 1$]
#theorem[$2 < 3$]
\end{verbatim}

\pandocbounded{\includesvg[keepaspectratio]{typst-img/0f178f614e49a7400d646926705364a92ca3d4d888423b2693f071f83ce09e7d-1.svg}}


\section{Examples Book LaTeX/book/basics/states/metadata.tex}
\title{sitandr.github.io/typst-examples-book/book/basics/states/metadata}

\section{\texorpdfstring{\hyperref[metadata]{Metadata}}{Metadata}}\label{metadata}

Metadata is invisible content that can be extracted using query or other
content. This may be very useful with
\texttt{\ }{\texttt{\ typst\ query\ }}\texttt{\ } to pass values to
external tools.

\begin{verbatim}
// Put metadata somewhere.
#metadata("This is a note") <note>

// And find it from anywhere else.
#context {
  query(<note>).first().value
}
\end{verbatim}

\pandocbounded{\includesvg[keepaspectratio]{typst-img/ef1c7d9faf74901f6c5266d48ae006167003a22754408a70ae9f9d1088b5fe24-1.svg}}


\section{Examples Book LaTeX/book/basics/states/index.tex}
\title{sitandr.github.io/typst-examples-book/book/basics/states/index}

\section{\texorpdfstring{\hyperref[states--query]{States \&
Query}}{States \& Query}}\label{states--query}

This section is outdated. It may be still useful, but it is strongly
recommended to study new context system (using the reference).

Typst tries to be a \emph{pure language} as much as possible.

That means, a function can\textquotesingle t change anything outside of
it. That also means, if you call function, the result should be always
the same.

Unfortunately, our world (and therefore our documents)
isn\textquotesingle t pure. If you create a heading №2, you want the
next number to be three.

That section will guide you to using impure Typst. Don\textquotesingle t
overuse it, as this knowledge comes close to the Dark Arts of Typst!


\section{Examples Book LaTeX/book/basics/states/states.tex}
\title{sitandr.github.io/typst-examples-book/book/basics/states/states}

\section{\texorpdfstring{\hyperref[states]{States}}{States}}\label{states}

This section is outdated. It may be still useful, but it is strongly
recommended to study new context system (using the reference).

Before we start something practical, it is important to understand
states in general.

Here is a good explanation of why do we \emph{need} them:
\href{https://typst.app/docs/reference/meta/state/}{Official Reference
about states} . It is highly recommended to read it first.

So instead of

\begin{verbatim}
#let x = 0
#let compute(expr) = {
  // eval evaluates string as Typst code
  // to calculate new x value
  x = eval(
    expr.replace("x", str(x))
  )
  [New value is #x.]
}

#compute("10") \
#compute("x + 3") \
#compute("x * 2") \
#compute("x - 5")
\end{verbatim}

\textbf{THIS DOES NOT COMPILE:} Variables from outside the function are
read-only and cannot be modified

Instead, you should write

\begin{verbatim}
#let s = state("x", 0)
#let compute(expr) = [
  // updates x current state with this function
  #s.update(x =>
    eval(expr.replace("x", str(x)))
  )
  // and displays it
  New value is #context s.get().
]

#compute("10") \
#compute("x + 3") \
#compute("x * 2") \
#compute("x - 5")

The computations will be made _in order_ they are _located_ in the document. So if you create computations first, but put them in the document later... See yourself:

#let more = [
  #compute("x * 2") \
  #compute("x - 5")
]

#compute("10") \
#compute("x + 3") \
#more
\end{verbatim}

\pandocbounded{\includesvg[keepaspectratio]{typst-img/9a88397d1a9b5a44b1a3a218894595121bd4c5ec875df2b960638f2925060334-1.svg}}

\subsection{\texorpdfstring{\hyperref[context-magic]{Context
magic}}{Context magic}}\label{context-magic}

So what does this magic
\texttt{\ }{\texttt{\ context\ s.get()\ }}\texttt{\ } mean?

\begin{quote}
\href{https://typst.app/docs/reference/context/}{Context in Reference}
\end{quote}

In short, it specifies what part of your code (or markup) can
\emph{depend on states outside} . This context-expression is packed then
as one object, and it is evaluated on layout stage.

That means it is impossible to look from "normal" code at whatever is
inside the \texttt{\ }{\texttt{\ context\ }}\texttt{\ } . This is a
black box that would be known \emph{only after putting it into the
document} .

We will discuss \texttt{\ }{\texttt{\ context\ }}\texttt{\ } features
later.

\subsection{\texorpdfstring{\hyperref[operations-with-states]{Operations
with states}}{Operations with states}}\label{operations-with-states}

\subsubsection{\texorpdfstring{\hyperref[creating-new-state]{Creating
new state}}{Creating new state}}\label{creating-new-state}

\begin{verbatim}
#let x = state("state-id")
#let y = state("state-id", 2)

#x, #y

State is #context x.get() \ // the same as
#context [State is #y.get()] \ // the same as
#context {"State is" + str(y.get())}
\end{verbatim}

\pandocbounded{\includesvg[keepaspectratio]{typst-img/4a52375bdeea2b7ca31dc51740563d01b3678f817dd6bc8c349d0714c2ac503f-1.svg}}

\subsubsection{\texorpdfstring{\hyperref[update]{Update}}{Update}}\label{update}

Updating is \emph{a content} that is an instruction. That instruction
tells compiler that in this place of document the state \emph{should be
updated} .

\begin{verbatim}
#let x = state("x", 0)
#context x.get() \
#let _ = x.update(3)
// nothing happens, we don't put `update` into the document flow
#context x.get()

#repr(x.update(3)) // this is how that content looks \

#context x.update(3)
#context x.get() // Finally!
\end{verbatim}

\pandocbounded{\includesvg[keepaspectratio]{typst-img/3732a9c7bca8c4faedf9b024e09e647a65222c8244e9f3235a6057dfebc0a511-1.svg}}

Here we can see one of \emph{important
\texttt{\ }{\texttt{\ context\ }}\texttt{\ } traits} : it "sees" states
from outside, but can\textquotesingle t see how they change inside it:

\begin{verbatim}
#let x = state("x", 0)

#context {
  x.update(3)
  str(x.get())
}
\end{verbatim}

\pandocbounded{\includesvg[keepaspectratio]{typst-img/78e500b80cb85e086a81302e2ce3dad88cb4304d4685b88e3f59111bc71f6748-1.svg}}

\subsubsection{\texorpdfstring{\hyperref[id-collision]{ID
collision}}{ID collision}}\label{id-collision}

\emph{TLDR; \textbf{Never allow colliding states.}}

States are described by their id-s, if they are the same, the code will
break.

So, if you write functions or loops that are used several times,
\emph{be careful} !

\begin{verbatim}
#let f(x) = {
  // return new state…
  // …but their id-s are the same!
  // so it will always be the same state!
  let y = state("x", 0)
  y.update(y => y + x)
  context y.get()
}

#let a = f(2)
#let b = f(3)

#a, #b \
#raw(repr(a) + "\n" + repr(b))
\end{verbatim}

\pandocbounded{\includesvg[keepaspectratio]{typst-img/31a3e88747ed09ae6078bd3caf986f0e6ba744e055d0889d92bfa23941e7e451-1.svg}}

However, this \emph{may seem} okay:

\begin{verbatim}
// locations in code are different!
#let x = state("state-id")
#let y = state("state-id", 2)

#x, #y
\end{verbatim}

\pandocbounded{\includesvg[keepaspectratio]{typst-img/1901e1449942d821c66f53bd6bc5fda10d63591aa45346fdf88bcbc3f2ab3425-1.svg}}

But in fact, it \emph{isn\textquotesingle t} :

\begin{verbatim}
#let x = state("state-id")
#let y = state("state-id", 2)

#context [#x.get(); #y.get()]

#x.update(3)

#context [#x.get(); #y.get()]
\end{verbatim}

\pandocbounded{\includesvg[keepaspectratio]{typst-img/9185a298f9bcf8c519fa85481b9272e6ef3a00c117a0904d0509920a6abef8b2-1.svg}}


\section{Examples Book LaTeX/book/basics/states/query.tex}
\title{sitandr.github.io/typst-examples-book/book/basics/states/query}

\section{\texorpdfstring{\hyperref[query]{Query}}{Query}}\label{query}

This section is outdated. It may be still useful, but it is strongly
recommended to study new context system (using the reference).

\begin{quote}
Link \href{https://typst.app/docs/reference/meta/query/}{there}
\end{quote}

Query is a thing that allows you getting location by \emph{selector}
(this is the same thing we used in show rules).

That enables "time travel", getting information about document from its
parts and so on. \emph{That is a way to violate Typst\textquotesingle s
purity.}

It is currently one of the \emph{the darkest magics currently existing
in Typst} . It gives you great powers, but with great power comes great
responsibility.

\subsection{\texorpdfstring{\hyperref[time-travel]{Time
travel}}{Time travel}}\label{time-travel}

\begin{verbatim}
#let s = state("x", 0)
#let compute(expr) = [
  #s.update(x =>
    eval(expr.replace("x", str(x)))
  )
  New value is #s.display().
]

Value at `<here>` is
#context s.at(
  query(<here>)
    .first()
    .location()
)

#compute("10") \
#compute("x + 3") \
*Here.* <here> \
#compute("x * 2") \
#compute("x - 5")
\end{verbatim}

\pandocbounded{\includesvg[keepaspectratio]{typst-img/130940aa5ae2ceb3364ef655c84cf8e7d2178210851b8fb20e6c0c3345c3ace7-1.svg}}

\subsection{\texorpdfstring{\hyperref[getting-nearest-chapter]{Getting
nearest
chapter}}{Getting nearest chapter}}\label{getting-nearest-chapter}

\begin{verbatim}
#set page(header: context {
  let elems = query(
    selector(heading).before(here()),
    here(),
  )
  let academy = smallcaps[
    Typst Academy
  ]
  if elems == () {
    align(right, academy)
  } else {
    let body = elems.last().body
    academy + h(1fr) + emph(body)
  }
})

= Introduction
#lorem(23)

= Background
#lorem(30)

= Analysis
#lorem(15)
\end{verbatim}

\pandocbounded{\includesvg[keepaspectratio]{typst-img/b4d0562911dd308b0d9cbc36ad20ba6ed91fc3c3da5b6259eb6721f3a53a18e3-1.svg}}




\section{Combined Examples Book LaTeX/book/basics/scripting.tex}
\section{Examples Book LaTeX/book/basics/scripting/conditions.tex}
\title{sitandr.github.io/typst-examples-book/book/basics/scripting/conditions}

\section{\texorpdfstring{\hyperref[conditions--loops]{Conditions \&
loops}}{Conditions \& loops}}\label{conditions--loops}

\subsection{\texorpdfstring{\hyperref[conditions]{Conditions}}{Conditions}}\label{conditions}

\begin{quote}
See
\href{https://typst.app/docs/reference/scripting/\#conditionals}{official
documentation} .
\end{quote}

In Typst, you can use \texttt{\ }{\texttt{\ if-else\ }}\texttt{\ }
statements. This is especially useful inside function bodies to vary
behavior depending on arguments types or many other things.

\begin{verbatim}
#if 1 < 2 [
  This is shown
] else [
  This is not.
]
\end{verbatim}

\pandocbounded{\includesvg[keepaspectratio]{typst-img/2e914defa3353d6fd42ed58c37a97aedcc2237cfe20228f0cc0d223dfff4619a-1.svg}}

Of course, \texttt{\ }{\texttt{\ else\ }}\texttt{\ } is unnecessary:

\begin{verbatim}
#let a = 3

#if a < 4 {
  a = 5
}

#a
\end{verbatim}

\pandocbounded{\includesvg[keepaspectratio]{typst-img/a7264774be154606a44d829d31edae18bf686262ccea66de9ed97fa20c720bd8-1.svg}}

You can also use \texttt{\ }{\texttt{\ else\ if\ }}\texttt{\ } statement
(known as \texttt{\ }{\texttt{\ elif\ }}\texttt{\ } in Python):

\begin{verbatim}
#let a = 5

#if a < 4 {
  a = 5
} else if a < 6 {
  a = -3
}

#a
\end{verbatim}

\pandocbounded{\includesvg[keepaspectratio]{typst-img/9f65678fc26af2d197d979e1b0a5295ed64037ee00c30fa28c9c417a2c7dc308-1.svg}}

\subsubsection{\texorpdfstring{\hyperref[booleans]{Booleans}}{Booleans}}\label{booleans}

\texttt{\ }{\texttt{\ if,\ else\ if,\ else\ }}\texttt{\ } accept
\emph{only boolean} values as a switch. You can combine booleans as
described in \href{./types.html\#boolean-bool}{types section} :

\begin{verbatim}
#let a = 5

#if (a > 1 and a <= 4) or a == 5 [
    `a` matches the condition
]
\end{verbatim}

\pandocbounded{\includesvg[keepaspectratio]{typst-img/21d3a48404d4e0c59bc0fccb114fdeac7384189db0020247796f44b0e9a7c362-1.svg}}

\subsection{\texorpdfstring{\hyperref[loops]{Loops}}{Loops}}\label{loops}

\begin{quote}
See \href{https://typst.app/docs/reference/scripting/\#loops}{official
documentation} .
\end{quote}

There are two kinds of loops: \texttt{\ }{\texttt{\ while\ }}\texttt{\ }
and \texttt{\ }{\texttt{\ for\ }}\texttt{\ } . While repeats body while
the condition is met:

\begin{verbatim}
#let a = 3

#while a < 100 {
    a *= 2
    str(a)
    " "
}
\end{verbatim}

\pandocbounded{\includesvg[keepaspectratio]{typst-img/ece06c012663616cac05b0f365bd02ea5607dcddfaa0249963088ceff797c100-1.svg}}

\texttt{\ }{\texttt{\ for\ }}\texttt{\ } iterates over all elements of
sequence. The sequence may be an
\texttt{\ }{\texttt{\ array\ }}\texttt{\ } ,
\texttt{\ }{\texttt{\ string\ }}\texttt{\ } or
\texttt{\ }{\texttt{\ dictionary\ }}\texttt{\ } (
\texttt{\ }{\texttt{\ for\ }}\texttt{\ } iterates over its
\emph{key-value pairs} ).

\begin{verbatim}
#for c in "ABC" [
  #c is a letter.
]
\end{verbatim}

\pandocbounded{\includesvg[keepaspectratio]{typst-img/9e70091e4c1f276d548f8200329298bf6b98946c331ca4630fec8313d5a91eff-1.svg}}

To iterate to all numbers from \texttt{\ }{\texttt{\ a\ }}\texttt{\ } to
\texttt{\ }{\texttt{\ b\ }}\texttt{\ } , use
\texttt{\ }{\texttt{\ range(a,\ b+1)\ }}\texttt{\ } :

\begin{verbatim}
#let s = 0

#for i in range(3, 6) {
    s += i
    [Number #i is added to sum. Now sum is #s.]
}
\end{verbatim}

\pandocbounded{\includesvg[keepaspectratio]{typst-img/1e3d95ee79d7bc6989e40ff1e27c0ef6e3b152a1e5f8a0df5b2819621e0e299f-1.svg}}

Because range is end-exclusive this is equal to

\begin{verbatim}
#let s = 0

#for i in (3, 4, 5) {
    s += i
    [Number #i is added to sum. Now sum is #s.]
}
\end{verbatim}

\pandocbounded{\includesvg[keepaspectratio]{typst-img/6158d29261339f8f285d592deff8992ca129ce32264abcdcf6734ac44cf558a4-1.svg}}

\begin{verbatim}
#let people = (Alice: 3, Bob: 5)

#for (name, value) in people [
    #name has #value apples.
]
\end{verbatim}

\pandocbounded{\includesvg[keepaspectratio]{typst-img/50ff0963afe8c9ec5a0562d518431b63d5dd3810525f55f084f812452b11eb21-1.svg}}

\subsubsection{\texorpdfstring{\hyperref[break-and-continue]{Break and
continue}}{Break and continue}}\label{break-and-continue}

Inside loops can be used \texttt{\ }{\texttt{\ break\ }}\texttt{\ } and
\texttt{\ }{\texttt{\ continue\ }}\texttt{\ } commands.
\texttt{\ }{\texttt{\ break\ }}\texttt{\ } breaks loop, jumping outside.
\texttt{\ }{\texttt{\ continue\ }}\texttt{\ } jumps to next loop
iteration.

See the difference on these examples:

\begin{verbatim}
#for letter in "abc nope" {
  if letter == " " {
    // stop when there is space
    break
  }

  letter
}
\end{verbatim}

\pandocbounded{\includesvg[keepaspectratio]{typst-img/a744551cab635d3ab70d9bf4258bb5fc26fe384f8e9f487ad0b8eee986ffe581-1.svg}}

\begin{verbatim}
#for letter in "abc nope" {
  if letter == " " {
    // skip the space
    continue
  }

  letter
}
\end{verbatim}

\pandocbounded{\includesvg[keepaspectratio]{typst-img/bbb719820f986e52fbf64306536766ecbfd7264d29429a5c62d1bd648a4754c5-1.svg}}


\section{Examples Book LaTeX/book/basics/scripting/index.tex}
\title{sitandr.github.io/typst-examples-book/book/basics/scripting/index}

\section{\texorpdfstring{\hyperref[scripting]{Scripting}}{Scripting}}\label{scripting}

\textbf{Typst} has a complete interpreted language inside. One of key
aspects of working with your document in a nicer way


\section{Examples Book LaTeX/book/basics/scripting/types_2.tex}
\title{sitandr.github.io/typst-examples-book/book/basics/scripting/types_2}

\section{\texorpdfstring{\hyperref[types-part-ii]{Types, part
II}}{Types, part II}}\label{types-part-ii}

In Typst, most of things are \textbf{immutable} . You
can\textquotesingle t change content, you can just create new using this
one (for example, using addition).

Immutability is very important for Typst since it tries to be \emph{as
pure language as possible} . Functions do nothing outside of returning
some value.

However, purity is partly "broken" by these types. They are
\emph{super-useful} and not adding them would make Typst much pain.

However, using them adds complexity.

\subsection{\texorpdfstring{\hyperref[arrays-array]{Arrays (
\texttt{\ }{\texttt{\ array\ }}\texttt{\ }
)}}{Arrays (   array   )}}\label{arrays-array}

\begin{quote}
\href{https://typst.app/docs/reference/foundations/array/}{Link to
Reference} .
\end{quote}

Mutable object that stores data with their indices.

\subsubsection{\texorpdfstring{\hyperref[working-with-indices]{Working
with indices}}{Working with indices}}\label{working-with-indices}

\begin{verbatim}
#let values = (1, 7, 4, -3, 2)

// take value at index 0
#values.at(0) \
// set value at 0 to 3
#(values.at(0) = 3)
// negative index => start from the back
#values.at(-1) \
// add index of something that is even
#values.find(calc.even)
\end{verbatim}

\pandocbounded{\includesvg[keepaspectratio]{typst-img/0374c20b28fbf2b2d15bc32e5428f7f5121ea9d673d96de3274a0c6d988d5fb5-1.svg}}

\subsubsection{\texorpdfstring{\hyperref[iterating-methods]{Iterating
methods}}{Iterating methods}}\label{iterating-methods}

\begin{verbatim}
#let values = (1, 7, 4, -3, 2)

// leave only what is odd
#values.filter(calc.odd) \
// create new list of absolute values of list values
#values.map(calc.abs) \
// reverse
#values.rev() \
// convert array of arrays to flat array
#(1, (2, 3)).flatten() \
// join array of string to string
#(("A", "B", "C")
 .join(", ", last: " and "))
\end{verbatim}

\pandocbounded{\includesvg[keepaspectratio]{typst-img/684400186916f8f16a2d7edb151b7f5023c7e4c010b23a2c6566f0bd7a224061-1.svg}}

\subsubsection{\texorpdfstring{\hyperref[list-operations]{List
operations}}{List operations}}\label{list-operations}

\begin{verbatim}
// sum of lists:
#((1, 2, 3) + (4, 5, 6))

// list product:
#((1, 2, 3) * 4)
\end{verbatim}

\pandocbounded{\includesvg[keepaspectratio]{typst-img/abe2d311638b351e0938be0e432f10265ca81a69a9ed7d2e6f88f656c60dfc65-1.svg}}

\subsubsection{\texorpdfstring{\hyperref[empty-list]{Empty
list}}{Empty list}}\label{empty-list}

\begin{verbatim}
#() \ // this is an empty list
#(1,) \  // this is a list with one element
BAD: #(1) // this is just an element, not a list!
\end{verbatim}

\pandocbounded{\includesvg[keepaspectratio]{typst-img/da4f77f8784462ca5c4f73862e58420695916064d56921e4adef7a7e37d5a532-1.svg}}

\subsection{\texorpdfstring{\hyperref[dictionaries-dict]{Dictionaries (
\texttt{\ }{\texttt{\ dict\ }}\texttt{\ }
)}}{Dictionaries (   dict   )}}\label{dictionaries-dict}

\begin{quote}
\href{https://typst.app/docs/reference/foundations/dictionary/}{Link to
Reference} .
\end{quote}

Dictionaries are objects that store a string "key" and a value,
associated with that key.

\begin{verbatim}
#let dict = (
  name: "Typst",
  born: 2019,
)

#dict.name \
#(dict.launch = 20)
#dict.len() \
#dict.keys() \
#dict.values() \
#dict.at("born") \
#dict.insert("city", "Berlin ")
#("name" in dict)
\end{verbatim}

\pandocbounded{\includesvg[keepaspectratio]{typst-img/638ada64eb36af0b1891def1b2c0a2cc97a14d87987df8c16f5f3872244553d6-1.svg}}

\subsubsection{\texorpdfstring{\hyperref[empty-dictionary]{Empty
dictionary}}{Empty dictionary}}\label{empty-dictionary}

\begin{verbatim}
This is an empty list: #() \
This is an empty dict: #(:)
\end{verbatim}

\pandocbounded{\includesvg[keepaspectratio]{typst-img/6ef41801d46f0b7256bb6913482fde054c811a1850ecae3a446660eb6d1c8850-1.svg}}


\section{Examples Book LaTeX/book/basics/scripting/basics.tex}
\title{sitandr.github.io/typst-examples-book/book/basics/scripting/basics}

\section{\texorpdfstring{\hyperref[basics]{Basics}}{Basics}}\label{basics}

\subsection{\texorpdfstring{\hyperref[variables-i]{Variables
I}}{Variables I}}\label{variables-i}

Let\textquotesingle s start with \emph{variables} .

The concept is very simple, just some value you can reuse:

\begin{verbatim}
#let author = "John Doe"

This is a book by #author. #author is a great guy.

#quote(block: true, attribution: author)[
  \<Some quote\>
]
\end{verbatim}

\pandocbounded{\includesvg[keepaspectratio]{typst-img/c311c1612cafa802f16f0d4ca2d6f1ecca59f545ed1f6ee99d3c4ae06ee2bff4-1.svg}}

\subsection{\texorpdfstring{\hyperref[variables-ii]{Variables
II}}{Variables II}}\label{variables-ii}

You can store \emph{any} Typst value in variable:

\begin{verbatim}
#let block_text = block(stroke: red, inset: 1em)[Text]

#block_text

#figure(caption: "The block", block_text)
\end{verbatim}

\pandocbounded{\includesvg[keepaspectratio]{typst-img/c6290389652d1771d5149c9393af8eb32bd37e4b2bfb2c11764f9f22c294f84b-1.svg}}

\subsection{\texorpdfstring{\hyperref[functions]{Functions}}{Functions}}\label{functions}

We have already seen some "custom" functions in
\href{../tutorial/advanced_styling.html}{Advanced Styling} chapter.

Functions are values that take some values and output some values:

\begin{verbatim}
// This is a syntax that we have seen earlier
#let f = (name) => "Hello, " + name

#f("world!")
\end{verbatim}

\pandocbounded{\includesvg[keepaspectratio]{typst-img/23fba8e9081a8b32b16d7deb54018bb73a8ac910adbfb1a0ca577eb3520a73b4-1.svg}}

\subsubsection{\texorpdfstring{\hyperref[alternative-syntax]{Alternative
syntax}}{Alternative syntax}}\label{alternative-syntax}

You can write the same shorter:

\begin{verbatim}
// The following syntaxes are equivalent
#let f = (name) => "Hello, " + name
#let f(name) = "Hello, " + name

#f("world!")
\end{verbatim}

\pandocbounded{\includesvg[keepaspectratio]{typst-img/e6e4bd179a38f1b3af96f3e7c6308be6f9494f41f43daa26ebabf7a77fc54780-1.svg}}


\section{Examples Book LaTeX/book/basics/scripting/arguments.tex}
\title{sitandr.github.io/typst-examples-book/book/basics/scripting/arguments}

\section{\texorpdfstring{\hyperref[advanced-arguments]{Advanced
arguments}}{Advanced arguments}}\label{advanced-arguments}

\subsection{\texorpdfstring{\hyperref[spreading-arguments-from-list]{Spreading
arguments from
list}}{Spreading arguments from list}}\label{spreading-arguments-from-list}

Spreading operator allows you to "unpack" the list of values into
arguments of function:

\begin{verbatim}
#let func(a, b, c, d, e) = [#a #b #c #d #e]
#func(..(([hi],) * 5))
\end{verbatim}

\pandocbounded{\includesvg[keepaspectratio]{typst-img/0586f1f7eb73effd507824b57f7282f12fe2612119d64413f72e6518aba01513-1.svg}}

This may be super useful in tables:

\begin{verbatim}
#let a = ("hi", "b", "c")

#table(columns: 3,
  [test], [x], [hello],
  ..a
)
\end{verbatim}

\pandocbounded{\includesvg[keepaspectratio]{typst-img/eb669f70df63815adcbe764fdb8635eecab33651c7eef55ea4de6cd63c96d9de-1.svg}}

\subsection{\texorpdfstring{\hyperref[key-arguments]{Key
arguments}}{Key arguments}}\label{key-arguments}

The same idea works with key arguments:

\begin{verbatim}
#let text-params = (fill: blue, size: 0.8em)

Some #text(..text-params)[text].
\end{verbatim}

\pandocbounded{\includesvg[keepaspectratio]{typst-img/e56483e8f4285f8fed8cd6867e720b9a1c9f62ef0bffea28d124159f8a61648d-1.svg}}

\section{\texorpdfstring{\hyperref[managing-arbitrary-arguments]{Managing
arbitrary
arguments}}{Managing arbitrary arguments}}\label{managing-arbitrary-arguments}

Typst allows taking as many arbitrary positional and key arguments as
you want.

In that case function is given special
\texttt{\ }{\texttt{\ arguments\ }}\texttt{\ } object that stores in it
positional and named arguments.

\begin{quote}
Link to
\href{https://typst.app/docs/reference/foundations/arguments/}{reference}
\end{quote}

\begin{verbatim}
#let f(..args) = [
  #args.pos()\
  #args.named()
]

#f(1, "a", width: 50%, block: false)
\end{verbatim}

\pandocbounded{\includesvg[keepaspectratio]{typst-img/2fc64c8521734ea689368ec83fe54025eb94b016a8ed1f6d6a9880ac6c94edf5-1.svg}}

You can combine them with other arguments. Spreading operator will "eat"
all remaining arguments:

\begin{verbatim}
#let format(title, ..authors) = {
  let by = authors
    .pos()
    .join(", ", last: " and ")

  [*#title* \ _Written by #by;_]
}

#format("ArtosFlow", "Jane", "Joe")
\end{verbatim}

\pandocbounded{\includesvg[keepaspectratio]{typst-img/4ba76c5176e0b93c6c2b03c38d55f88702546a5183717ed8c3567865c0d1bf5d-1.svg}}

\subsection{\texorpdfstring{\hyperref[optional-argument]{Optional
argument}}{Optional argument}}\label{optional-argument}

\emph{Currently the only way in Typst to create optional positional
arguments is using \texttt{\ }{\texttt{\ arguments\ }}\texttt{\ }
object:}

TODO


\section{Examples Book LaTeX/book/basics/scripting/types.tex}
\title{sitandr.github.io/typst-examples-book/book/basics/scripting/types}

\section{\texorpdfstring{\hyperref[types-part-i]{Types, part
I}}{Types, part I}}\label{types-part-i}

Each value in Typst has a type. You don\textquotesingle t have to
specify it, but it is important.

\subsection{\texorpdfstring{\hyperref[content-content]{Content (
\texttt{\ }{\texttt{\ content\ }}\texttt{\ }
)}}{Content (   content   )}}\label{content-content}

\begin{quote}
\href{https://typst.app/docs/reference/foundations/content/}{Link to
Reference} .
\end{quote}

We have already seen it. A type that represents what is displayed in
document.

\begin{verbatim}
#let c = [It is _content_!]

// Check type of c
#(type(c) == content)

#c

// repr gives an "inner representation" of value
#repr(c)
\end{verbatim}

\pandocbounded{\includesvg[keepaspectratio]{typst-img/21fd80460de8e8a377a9ef2046a27232ad88924070509ccf8647c9135c9c2fe3-1.svg}}

\textbf{Important:} It is very hard to convert \emph{content} to
\emph{plain text} , as \emph{content} may contain \emph{anything} ! So
be careful when passing and storing content in variables.

\subsection{\texorpdfstring{\hyperref[none-none]{None (
\texttt{\ }{\texttt{\ none\ }}\texttt{\ }
)}}{None (   none   )}}\label{none-none}

Nothing. Also known as \texttt{\ }{\texttt{\ null\ }}\texttt{\ } in
other languages. It isn\textquotesingle t displayed, converts to empty
content.

\begin{verbatim}
#none
#repr(none)
\end{verbatim}

\pandocbounded{\includesvg[keepaspectratio]{typst-img/c4100c1d1df8fc0a51bd99945d9bac3c5aa67de19b8f872fd33fd9068bb2507b-1.svg}}

\subsection{\texorpdfstring{\hyperref[string-str]{String (
\texttt{\ }{\texttt{\ str\ }}\texttt{\ }
)}}{String (   str   )}}\label{string-str}

\begin{quote}
\href{https://typst.app/docs/reference/foundations/str/}{Link to
Reference} .
\end{quote}

String contains only plain text and no formatting. Just some chars. That
allows us to work with chars:

\begin{verbatim}
#let s = "Some large string. There could be escape sentences: \n,
 line breaks, and even unicode codes: \u{1251}"
#s \
#type(s) \
`repr`: #repr(s)

#let s = "another small string"
#s.replace("a", sym.alpha) \
#s.split(" ") // split by space
\end{verbatim}

\pandocbounded{\includesvg[keepaspectratio]{typst-img/b797f9c4a540fcf1429bec801d0b334e7d88dc9ccd10e3b7b859f451e269f30f-1.svg}}

You can convert other types to their string representation using this
type\textquotesingle s constructor (e.g. convert number to string):

\begin{verbatim}
#str(5) // string, can be worked with as string
\end{verbatim}

\pandocbounded{\includesvg[keepaspectratio]{typst-img/ab4d4a5d93533525f7f9b2cc8378b79f1561904f3c5d5f6d2ec4bdc448669cb5-1.svg}}

\subsection{\texorpdfstring{\hyperref[boolean-bool]{Boolean (
\texttt{\ }{\texttt{\ bool\ }}\texttt{\ }
)}}{Boolean (   bool   )}}\label{boolean-bool}

\begin{quote}
\href{https://typst.app/docs/reference/foundations/bool/}{Link to
Reference} .
\end{quote}

true/false. Used in \texttt{\ }{\texttt{\ if\ }}\texttt{\ } and many
others

\begin{verbatim}
#let b = false
#b \
#repr(b) \
#(true and not true or true) = #((true and (not true)) or true) \
#if (4 > 3) {
  "4 is more than 3"
}
\end{verbatim}

\pandocbounded{\includesvg[keepaspectratio]{typst-img/e848d78e220ca8cf3b6c323a99d5d963e529aad36857f0e6753c56c02984a616-1.svg}}

\subsection{\texorpdfstring{\hyperref[integer-int]{Integer (
\texttt{\ }{\texttt{\ int\ }}\texttt{\ }
)}}{Integer (   int   )}}\label{integer-int}

\begin{quote}
\href{https://typst.app/docs/reference/foundations/int/}{Link to
Reference} .
\end{quote}

A whole number.

The number can also be specified as hexadecimal, octal, or binary by
starting it with a zero followed by either x, o, or b.

\begin{verbatim}
#let n = 5
#n \
#(n += 1) \
#n \
#calc.pow(2, n) \
#type(n) \
#repr(n)
\end{verbatim}

\pandocbounded{\includesvg[keepaspectratio]{typst-img/6f1c9e02393e14aa23add33d0e6dc2b596ee97a0d425cd3edb3e2b912c6ef6b0-1.svg}}

\begin{verbatim}
#(1 + 2) \
#(2 - 5) \
#(3 + 4 < 8)
\end{verbatim}

\pandocbounded{\includesvg[keepaspectratio]{typst-img/e610f15659cb6b64c3516be48740b54e6caf3d933919004157ba64b757389ba5-1.svg}}

\begin{verbatim}
#0xff \
#0o10 \
#0b1001
\end{verbatim}

\pandocbounded{\includesvg[keepaspectratio]{typst-img/1446dba05ee6f8006884c280ff32e31ede8425d4847445e97cae5dfcde1efe7f-1.svg}}

You can convert a value to an integer with this type\textquotesingle s
constructor (e.g. convert string to int).

\begin{verbatim}
#int(false) \
#int(true) \
#int(2.7) \
#(int("27") + int("4"))
\end{verbatim}

\pandocbounded{\includesvg[keepaspectratio]{typst-img/b44779a87fd984d317ec4d1aed732c0ebdc6220fd4764e407f77fedd139c0d8c-1.svg}}

\subsection{\texorpdfstring{\hyperref[float-float]{Float (
\texttt{\ }{\texttt{\ float\ }}\texttt{\ }
)}}{Float (   float   )}}\label{float-float}

\begin{quote}
\href{https://typst.app/docs/reference/foundations/float/}{Link to
Reference} .
\end{quote}

Works the same way as integer, but can store floating point numbers.
However, precision may be lost.

\begin{verbatim}
#let n = 5.0

// You can mix floats and integers, 
// they will be implicitly converted
#(n += 1) \
#calc.pow(2, n) \
#(0.2 + 0.1) \
#type(n) 
\end{verbatim}

\pandocbounded{\includesvg[keepaspectratio]{typst-img/21cafe751ec803dd9598c871b283a29bc3c6b2e302f0f9bd78edc17330b45616-1.svg}}

\begin{verbatim}
#3.14 \
#1e4 \
#(10 / 4)
\end{verbatim}

\pandocbounded{\includesvg[keepaspectratio]{typst-img/05bd400096c1df5a954fda0897f3c1756c9f99f73503d32d992b3222667a45cd-1.svg}}

You can convert a value to a float with this type\textquotesingle s
constructor (e.g. convert string to float).

\begin{verbatim}
#float(40%) \
#float("2.7") \
#float("1e5")
\end{verbatim}

\pandocbounded{\includesvg[keepaspectratio]{typst-img/f50a22cbea42fded97ab8340f0939e786e5c1cdb5ea531cd4b35b1f732947b7f-1.svg}}


\section{Examples Book LaTeX/book/basics/scripting/tips.tex}
\title{sitandr.github.io/typst-examples-book/book/basics/scripting/tips}

\section{\texorpdfstring{\hyperref[tips]{Tips}}{Tips}}\label{tips}

There are lots of elements in Typst scripting that are not obvious, but
important. All the book is designated to show them, but some of them

\subsection{\texorpdfstring{\hyperref[equality]{Equality}}{Equality}}\label{equality}

Equality doesn\textquotesingle t mean objects are really the same, like
in many other objects:

\begin{verbatim}
#let a = 7
#let b = 7.0
#(a == b)
#(type(a) == type(b))
\end{verbatim}

\pandocbounded{\includesvg[keepaspectratio]{typst-img/3632e0202f7aae6ed6e2958b7bc6360a6cba31aa3d1aaf169a133ef987c839de-1.svg}}

That may be less obvious for dictionaries. In dictionaries \textbf{the
order may matter} , so equality doesn\textquotesingle t mean they behave
exactly the same way:

\begin{verbatim}
#let a = (x: 1, y: 2)
#let b = (y: 2, x: 1)
#(a == b)
#(a.pairs() == b.pairs())
\end{verbatim}

\pandocbounded{\includesvg[keepaspectratio]{typst-img/f7277d7cc170d7cc2ae1de5436b534fb113cda82d8e7829a0fc92e950b78238f-1.svg}}

\subsection{\texorpdfstring{\hyperref[check-key-is-in-dictionary]{Check
key is in
dictionary}}{Check key is in dictionary}}\label{check-key-is-in-dictionary}

Use the keyword \texttt{\ }{\texttt{\ in\ }}\texttt{\ } , like in
\texttt{\ }{\texttt{\ Python\ }}\texttt{\ } :

\begin{verbatim}
#let dict = (a: 1, b: 2)

#("a" in dict)
// gives the same as
#(dict.keys().contains("a"))
\end{verbatim}

\pandocbounded{\includesvg[keepaspectratio]{typst-img/c4ae77418e54911af371f203d2bd3d5badb7269496bb8f07a2e3010e15f18922-1.svg}}

Note it works for lists too:

\begin{verbatim}
#("a" in ("b", "c", "a"))
#(("b", "c", "a").contains("a"))
\end{verbatim}

\pandocbounded{\includesvg[keepaspectratio]{typst-img/0fc3ff7d44bbb5bcacd38e921f199699d2ea43ce0a142e79f67314d4f24386a7-1.svg}}


\section{Examples Book LaTeX/book/basics/scripting/braces.tex}
\title{sitandr.github.io/typst-examples-book/book/basics/scripting/braces}

\section{\texorpdfstring{\hyperref[braces-brackets-and-default]{Braces,
brackets and
default}}{Braces, brackets and default}}\label{braces-brackets-and-default}

\subsection{\texorpdfstring{\hyperref[square-brackets]{Square
brackets}}{Square brackets}}\label{square-brackets}

You may remember that square brackets convert everything inside to
\emph{content} .

\begin{verbatim}
#let v = [Some text, _markup_ and other #strong[functions]]
#v
\end{verbatim}

\pandocbounded{\includesvg[keepaspectratio]{typst-img/5ba617daa8d4c166d96a0abbba02d6502fe7fde1ded460afa78682993295142d-1.svg}}

We may use same for functions bodies:

\begin{verbatim}
#let f(name) = [Hello, #name]
#f[World] // also don't forget we can use it to pass content!
\end{verbatim}

\pandocbounded{\includesvg[keepaspectratio]{typst-img/4545deeee45655ee6666feb4773416cd075fe7522cbfd80d0847c615c6c5f30a-1.svg}}

\textbf{Important:} It is very hard to convert \emph{content} to
\emph{plain text} , as \emph{content} may contain \emph{anything} ! Sp
be careful when passing and storing content in variables.

\subsection{\texorpdfstring{\hyperref[braces]{Braces}}{Braces}}\label{braces}

However, we often want to use code inside functions.
That\textquotesingle s when we use
\texttt{\ }{\texttt{\ \{\}\ }}\texttt{\ } :

\begin{verbatim}
#let f(name) = {
  // this is code mode

  // First part of our output
  "Hello, "

  // we check if name is empty, and if it is,
  // insert placeholder
  if name == "" {
      "anonym"
  } else {
      name
  }

  // finish sentence
  "!"
}

#f("")
#f("Joe")
#f("world")
\end{verbatim}

\pandocbounded{\includesvg[keepaspectratio]{typst-img/f2bc6aebef06f213c9a8e740266a96e424318d953c09cffba6c5811375e91395-1.svg}}

\subsection{\texorpdfstring{\hyperref[scopes]{Scopes}}{Scopes}}\label{scopes}

\textbf{This is a very important thing to remember} .

\emph{You can\textquotesingle t use variables outside of scopes they are
defined (unless it is file root, then you can import them)} . \emph{Set
and show rules affect things in their scope only.}

\begin{verbatim}
#{
  let a = 3;
}
// can't use "a" there.

#[
  #show "true": "false"

  This is true.
]

This is true.
\end{verbatim}

\pandocbounded{\includesvg[keepaspectratio]{typst-img/c25d356831eeea19bb243b87c0f32d062c7086a55b4ee432e41b388d626f875b-1.svg}}

\subsection{\texorpdfstring{\hyperref[return]{Return}}{Return}}\label{return}

\textbf{Important} : by default braces return anything that "returns"
into them. For example,

\begin{verbatim}
#let change_world() = {
  // some code there changing everything in the world
  str(4e7)
  // another code changing the world
}

#let g() = {
  "Hahaha, I will change the world now! "
  change_world()
  " So here is my long evil monologue..."
}

#g()
\end{verbatim}

\pandocbounded{\includesvg[keepaspectratio]{typst-img/160d9672bd7abc64ea61943d1bfcbd1b06dc70f87be5e5cf9c411fe4ee6d2a44-1.svg}}

To avoid returning everything, return only what you want explicitly,
otherwise everything will be joined:

\begin{verbatim}
#let f() = {
  "Some long text"
  // Crazy numbers
  "2e7"
  return none
}

// Returns nothing
#f()
\end{verbatim}

\pandocbounded{\includesvg[keepaspectratio]{typst-img/14c935733a8c91165ee4ebe8246efb841207feeaa0309e36a1cde2888acffb10-1.svg}}

\subsection{\texorpdfstring{\hyperref[default-values]{Default
values}}{Default values}}\label{default-values}

What we made just now was inventing "default values".

They are very common in styling, so there is a special syntax for them:

\begin{verbatim}
#let f(name: "anonym") = [Hello, #name!]

#f()
#f(name: "Joe")
#f(name: "world")
\end{verbatim}

\pandocbounded{\includesvg[keepaspectratio]{typst-img/e9730d0d1f30ec9f2404179560ae4a4b19dd788b1afc2f6b956fb32268439cb6-1.svg}}

You may have noticed that the argument became \emph{named} now. In
Typst, named argument is an argument \emph{that has default value} .




\section{Combined Examples Book LaTeX/book/basics/basics.tex}
\section{Examples Book LaTeX/book/basics/extra.tex}
\title{sitandr.github.io/typst-examples-book/book/basics/extra}

\section{\texorpdfstring{\hyperref[extra]{Extra}}{Extra}}\label{extra}

\subsection{\texorpdfstring{\hyperref[bibliography]{Bibliography}}{Bibliography}}\label{bibliography}

Typst supports bibliography using BibLaTex
\texttt{\ }{\texttt{\ .bib\ }}\texttt{\ } file or its own Hayagriva
\texttt{\ }{\texttt{\ .yml\ }}\texttt{\ } format.

BibLaTex is wider supported, but Hayagriva is easier to work with.

\begin{quote}
Link to Hayagriva
\href{https://github.com/typst/hayagriva/blob/main/docs/file-format.md}{documentation}
and some
\href{https://github.com/typst/hayagriva/blob/main/tests/data/basic.yml}{examples}
.
\end{quote}

\subsubsection{\texorpdfstring{\hyperref[citation-style]{Citation
Style}}{Citation Style}}\label{citation-style}

The style can be customized via CSL, citation style language, with more
than 10 000 styles available online. See
\href{https://github.com/citation-style-language/styles}{official
repository} .


\section{Examples Book LaTeX/book/basics/index.tex}
\title{sitandr.github.io/typst-examples-book/book/basics/index}

\section{\texorpdfstring{\hyperref[typst-basics]{Typst
Basics}}{Typst Basics}}\label{typst-basics}

This is a chapter that consistently introduces you to the most things
you need to know when writing with Typst.

It show much more things than official tutorial, so maybe it will be
interesting to read for some of the experienced users too.

Some examples are taken from
\href{https://typst.app/docs/tutorial/}{Official Tutorial} and
\href{https://typst.app/docs/reference/}{Official Reference} . Most are
created and edited specially for this book.

\begin{quote}
\emph{Important:} in most cases there will be used "clipped" examples of
your rendered documents (no margins, smaller width and so on).

To set up the spacing as you want, see
\href{https://typst.app/docs/guides/page-setup-guide/}{Official Page
Setup Guide} .
\end{quote}


\section{Examples Book LaTeX/book/basics/special_symbols.tex}
\title{sitandr.github.io/typst-examples-book/book/basics/special_symbols}

\section{\texorpdfstring{\hyperref[special-symbols]{Special
symbols}}{Special symbols}}\label{special-symbols}

\begin{quote}
\emph{Important:} I\textquotesingle m not great with special symbols, so
I would additionally appreciate additions and corrections.
\end{quote}

Typst has a great support of \emph{unicode} . That also means it
supports \emph{special symbols} . They may be very useful for
typesetting.

In most cases, you shouldn\textquotesingle t use these symbols directly
often. If possible, use them with show rules (for example, replace all
\texttt{\ }{\texttt{\ -th\ }}\texttt{\ } with
\texttt{\ }{\texttt{\ \textbackslash{}u\ }}\texttt{\ }{\texttt{\ \{2011\}th\ }}\texttt{\ }
, a non-breaking hyphen).

\subsection{\texorpdfstring{\hyperref[non-breaking-symbols]{Non-breaking
symbols}}{Non-breaking symbols}}\label{non-breaking-symbols}

Non-breaking symbols can make sure the word/phrase will not be
separated. Typst will try to put them as a whole.

\subsubsection{\texorpdfstring{\hyperref[non-breaking-space]{Non-breaking
space}}{Non-breaking space}}\label{non-breaking-space}

\begin{quote}
\emph{Important:} As it is spacing symbols, copy-pasting it will not
help. Typst will see it as just a usual spacing symbol you used for your
source code to look nicer in your editor. Again, it will interpret it
\emph{as a basic space} .
\end{quote}

This is a symbol you should\textquotesingle t use often (use Typst boxes
instead), but it is a good demonstration of how non-breaking symbol
work:

\begin{verbatim}
#set page(width: 9em)

// Cruel and world are separated.
// Imagine this is a phrase that can't be split, what to do then?
Hello cruel world

// Let's connect them with a special space!

// No usual spacing is allowed, so either use semicolumn...
Hello cruel#sym.space.nobreak;world

// ...parentheses...
Hello cruel#(sym.space.nobreak)world

// ...or unicode code
Hello cruel\u{00a0}world

// Well, to achieve the same effect I recommend using box:
Hello #box[cruel world]
\end{verbatim}

\pandocbounded{\includesvg[keepaspectratio]{typst-img/be9e5cddfdd58a5f21a2b17e32227ac0c96e2d6eeffe764ef2809257aa416c59-1.svg}}

\subsubsection{\texorpdfstring{\hyperref[non-breaking-hyphen]{Non-breaking
hyphen}}{Non-breaking hyphen}}\label{non-breaking-hyphen}

\begin{verbatim}
#set page(width: 8em)

This is an $i$-th element.

This is an $i$\u{2011}th element.

// the best way would be
#show "-th": "\u{2011}th"

This is an $i$-th element.
\end{verbatim}

\pandocbounded{\includesvg[keepaspectratio]{typst-img/02baa9a61778ef23389d4ceb2fae4d2ac699d72b127b447ca6f25037096d2df9-1.svg}}

\subsection{\texorpdfstring{\hyperref[connectors-and-separators]{Connectors
and
separators}}{Connectors and separators}}\label{connectors-and-separators}

\subsubsection{\texorpdfstring{\hyperref[world-joiner]{World
joiner}}{World joiner}}\label{world-joiner}

Initially, world joiner indicates that no line break should occur at
this position. It is also a zero-width symbol (invisible), so it can be
used as a space removing thing:

\begin{verbatim}
#set page(width: 9em)
#set text(hyphenate: true)

Thisisawordthathastobreak

// Be careful, there is no line break at all now!
Thisi#sym.wj;sawordthathastobreak

// code from `physica` package
// word joiner here is used to avoid extra spacing
#let just-hbar = move(dy: -0.08em, strike(offset: -0.55em, extent: -0.05em, sym.planck))
#let hbar = (sym.wj, just-hbar, sym.wj).join()

$ a #just-hbar b, a hbar b$
\end{verbatim}

\pandocbounded{\includesvg[keepaspectratio]{typst-img/7df9031646c932030adb0fc5a97446e7560ca7d353ef935d4034dc0a4b8be5c1-1.svg}}

\subsubsection{\texorpdfstring{\hyperref[zero-width-space]{Zero width
space}}{Zero width space}}\label{zero-width-space}

Similar to word-joiner, but this is a \emph{space} . It
doesn\textquotesingle t prevent word break. On the contrary, it breaks
it without any hyphen at all!

\begin{verbatim}
#set page(width: 9em)
#set text(hyphenate: true)

// There is a space inside!
Thisisa#sym.zws;word

// Be careful, there is no hyphen at all now!
Thisisawo#sym.zws;rdthathastobreak
\end{verbatim}

\pandocbounded{\includesvg[keepaspectratio]{typst-img/7fd917d4e0422bc1bb72d451b6da6e38fb9fe28cd28152ab60bdfb7ad5d1cab1-1.svg}}


\section{Examples Book LaTeX/book/basics/measure.tex}
\title{sitandr.github.io/typst-examples-book/book/basics/measure}

\section{\texorpdfstring{\hyperref[measure-layout]{Measure,
Layout}}{Measure, Layout}}\label{measure-layout}

This section is outdated. It may be still useful, but it is strongly
recommended to study new context system (using the reference).

\subsection{\texorpdfstring{\hyperref[style--measure]{Style \&
Measure}}{Style \& Measure}}\label{style--measure}

\begin{quote}
Style
\href{https://typst.app/docs/reference/foundations/style/}{documentation}
.
\end{quote}

\begin{quote}
Measure
\href{https://typst.app/docs/reference/layout/measure/}{documentation} .
\end{quote}

\texttt{\ }{\texttt{\ measure\ }}\texttt{\ } returns \emph{the element
size} . This command is extremely helpful when doing custom layout with
\texttt{\ }{\texttt{\ place\ }}\texttt{\ } .

However, there is a catch. Element size depends on styles, applied to
this element.

\begin{verbatim}
#let content = [Hello!]
#content
#set text(14pt)
#content
\end{verbatim}

\pandocbounded{\includesvg[keepaspectratio]{typst-img/00a6cbbc650947c03f34564786b0645eee60396f288d26333c591ff9059cc369-1.svg}}

So if we will set the big text size for some part of our text, to
measure the element\textquotesingle s size, we have to know \emph{where
the element is located} . Without knowing it, we can\textquotesingle t
tell what styles should be applied.

So we need a scheme similar to
\texttt{\ }{\texttt{\ locate\ }}\texttt{\ } .

This is what \texttt{\ }{\texttt{\ styles\ }}\texttt{\ } function is
used for. It is \emph{a content} , which, when located in document,
calls a function inside on \emph{current styles} .

Now, when we got fixed \texttt{\ }{\texttt{\ styles\ }}\texttt{\ } , we
can get the element\textquotesingle s size using
\texttt{\ }{\texttt{\ measure\ }}\texttt{\ } :

\begin{verbatim}
#let thing(body) = style(styles => {
  let size = measure(body, styles)
  [Width of "#body" is #size.width]
})

#thing[Hey] \
#thing[Welcome]
\end{verbatim}

\pandocbounded{\includesvg[keepaspectratio]{typst-img/5afe1855072b4ee8e343e5b5aa79affae5b17bc89738ffbe93dac245576cdd04-1.svg}}

\section{\texorpdfstring{\hyperref[layout]{Layout}}{Layout}}\label{layout}

Layout is similar to \texttt{\ }{\texttt{\ measure\ }}\texttt{\ } , but
it returns current scope \textbf{parent size} .

If you are putting elements in block, that will be
block\textquotesingle s size. If you are just putting right on the page,
that will be page\textquotesingle s size.

As parent\textquotesingle s size depends on it\textquotesingle s place
in document, it uses the similar scheme to
\texttt{\ }{\texttt{\ locate\ }}\texttt{\ } and
\texttt{\ }{\texttt{\ style\ }}\texttt{\ } :

\begin{verbatim}
#layout(size => {
  let half = 50% * size.width
  [Half a page is #half wide.]
})
\end{verbatim}

\pandocbounded{\includesvg[keepaspectratio]{typst-img/c68a166f6e6b1b3229fd56478ae302dbeb39c882e229c69d4c6ebb6c9c528985-1.svg}}

It may be extremely useful to combine
\texttt{\ }{\texttt{\ layout\ }}\texttt{\ } with
\texttt{\ }{\texttt{\ measure\ }}\texttt{\ } , to get width of things
that depend on parent\textquotesingle s size:

\begin{verbatim}
#let text = lorem(30)
#layout(size => style(styles => [
  #let (height,) = measure(
    block(width: size.width, text),
    styles,
  )
  This text is #height high with
  the current page width: \
  #text
]))
\end{verbatim}

\pandocbounded{\includesvg[keepaspectratio]{typst-img/93167dc0b22b02fe27aa92c6b03c2281665b4352624364a19c63f61a488aa75a-1.svg}}




\section{Combined Examples Book LaTeX/book/basics/math.tex}
\section{Examples Book LaTeX/book/basics/math/symbols.tex}
\title{sitandr.github.io/typst-examples-book/book/basics/math/symbols}

\section{\texorpdfstring{\hyperref[symbols]{Symbols}}{Symbols}}\label{symbols}

Multiletter words in math refer either to local variables, functions,
text operators, spacing or \emph{special symbols} . The latter are very
important for advanced math.

\begin{verbatim}
$
forall v, w in V, alpha in KK: alpha dot (v + w) = alpha v + alpha w
$
\end{verbatim}

\pandocbounded{\includesvg[keepaspectratio]{typst-img/60a6e3e08582c87ec082b6714a45a90a914dd1299f788e2bb21b0cc5adc80e6a-1.svg}}

You can write the same with unicode:

\begin{verbatim}
$
∀ v, w ∈ V, α ∈ 𝕂: α ⋅ (v + w) = α v + α w
$
\end{verbatim}

\pandocbounded{\includesvg[keepaspectratio]{typst-img/d37776c21d5c4d692e4ebbe7e5ce7e7cdf5e2c0777a88a47abe0c0c5992cf41a-1.svg}}

\subsection{\texorpdfstring{\hyperref[symbols-naming]{Symbols
naming}}{Symbols naming}}\label{symbols-naming}

\begin{quote}
See all available symbols list
\href{https://typst.app/docs/reference/symbols/sym/}{there} .
\end{quote}

\subsubsection{\texorpdfstring{\hyperref[general-idea]{General
idea}}{General idea}}\label{general-idea}

Typst wants to define some "basic" symbols with small easy-to-remember
words, and build complex ones using combinations. For example,

\begin{verbatim}
$
// cont — contour
integral, integral.cont, integral.double, integral.square, sum.integral\

// lt — less than, gt — greater than
lt, lt.circle, lt.eq, lt.not, lt.eq.not, lt.tri, lt.tri.eq, lt.tri.eq.not, gt, lt.gt.eq, lt.gt.not
$
\end{verbatim}

\pandocbounded{\includesvg[keepaspectratio]{typst-img/a0ee196d2bf305ca6c2d812008f9955e5ae526de0b0ac0b83ca016a66bdc00f1-1.svg}}

I highly recommend using WebApp/Typst LSP when writing math with lots of
complex symbols. That helps you to quickly choose the right symbol
within all combinations.

Sometimes the names are not obvious, for example, sometimes it is used
prefix \texttt{\ }{\texttt{\ n-\ }}\texttt{\ } instead of
\texttt{\ }{\texttt{\ not\ }}\texttt{\ } :

\begin{verbatim}
$
gt.nequiv, gt.napprox, gt.ntilde, gt.tilde.not
$
\end{verbatim}

\pandocbounded{\includesvg[keepaspectratio]{typst-img/e4d0ef024efaf9f4334ebf04a2ac4e015fc5ec76617be8b6d7aad2f4429e3317-1.svg}}

\subsubsection{\texorpdfstring{\hyperref[common-modifiers]{Common
modifiers}}{Common modifiers}}\label{common-modifiers}

\begin{itemize}
\item
  \texttt{\ }{\texttt{\ .b,\ .t,\ .l,\ .r\ }}\texttt{\ } : bottom, top,
  left, right. Change direction of symbol.

\begin{verbatim}
$arrow.b, triangle.r, angle.l$
\end{verbatim}

  \pandocbounded{\includesvg[keepaspectratio]{typst-img/8ab0fa590b7a39023b1467e7a376a4810f997f720dd5d221ad83d7e741943b55-1.svg}}
\end{itemize}


\section{Examples Book LaTeX/book/basics/math/grouping.tex}
\title{sitandr.github.io/typst-examples-book/book/basics/math/grouping}

\section{\texorpdfstring{\hyperref[grouping]{Grouping}}{Grouping}}\label{grouping}

Every grouping can be (currently) done by parenthesis. So the
parenthesis may be both "real" parenthesis and grouping ones.

For example, these parentheses specify nominator of the fraction:

\begin{verbatim}
$ (a^2 + b^2)/2 $
\end{verbatim}

\pandocbounded{\includesvg[keepaspectratio]{typst-img/6f4767b2aee69b5c3a22df5f394105df9f19c9762678d02b297c4d4f8d1cf6ad-1.svg}}

\subsection{\texorpdfstring{\hyperref[left-right]{Left-right}}{Left-right}}\label{left-right}

\begin{quote}
See \href{https://typst.app/docs/reference/math/lr}{official
documentation} .
\end{quote}

If there are two matching braces of any kind, they will be wrapped as
\texttt{\ }{\texttt{\ lr\ }}\texttt{\ } (left-right) group.

\begin{verbatim}
$
{[((a + b)/2) + 1]_0}
$
\end{verbatim}

\pandocbounded{\includesvg[keepaspectratio]{typst-img/a4137ff5d1f577cc816776cb4279cce4cd964601c20eb244d12e170deecd5d6a-1.svg}}

You can disable it by escaping.

You can also match braces of any kind by using
\texttt{\ }{\texttt{\ lr\ }}\texttt{\ } directly:

\begin{verbatim}
$
lr([a/2, b)) \
lr([a/2, b), size: #150%)
$
\end{verbatim}

\pandocbounded{\includesvg[keepaspectratio]{typst-img/fb81420a901d8b570ef03d1f50c83f7b8c483c9366222156ea193ac2976b63ed-1.svg}}

\subsection{\texorpdfstring{\hyperref[fences]{Fences}}{Fences}}\label{fences}

Fences \emph{are not matched automatically} because of large amount of
false-positives.

You can use \texttt{\ }{\texttt{\ abs\ }}\texttt{\ } or
\texttt{\ }{\texttt{\ norm\ }}\texttt{\ } to match them:

\begin{verbatim}
$
abs(a + b), norm(a + b), floor(a + b), ceil(a + b), round(a + b)
$
\end{verbatim}

\pandocbounded{\includesvg[keepaspectratio]{typst-img/fd8454b2a97d649525827367f459f3163d830b5db9181178304d5fd2b44fcca1-1.svg}}


\section{Examples Book LaTeX/book/basics/math/classes.tex}
\title{sitandr.github.io/typst-examples-book/book/basics/math/classes}

\section{\texorpdfstring{\hyperref[classes]{Classes}}{Classes}}\label{classes}

\begin{quote}
See \href{https://typst.app/docs/reference/math/class/}{official
documentation}
\end{quote}

Each math symbol has its own "class", the way it behaves.
That\textquotesingle s one of the main reasons why they are layouted
differently.

\subsection{\texorpdfstring{\hyperref[classes-1]{Classes}}{Classes}}\label{classes-1}

\begin{verbatim}
$
a b c\
a class("normal", b) c\
a class("punctuation", b) c\
a class("opening", b) c\
a lr(b c]) c\
a lr(class("opening", b) c ]) c // notice it is moved vertically \
a class("closing", b) c\
a class("fence", b) c\
a class("large", b) c\
a class("relation", b) c\
a class("unary", b) c\
a class("binary", b) c\
a class("vary", b) c\
$
\end{verbatim}

\pandocbounded{\includesvg[keepaspectratio]{typst-img/5d4604274229b2f53ee04b88ff0e73d9aa8365643c5e60052fcca1298d4f5a23-1.svg}}

\subsection{\texorpdfstring{\hyperref[setting-class-for-symbol]{Setting
class for
symbol}}{Setting class for symbol}}\label{setting-class-for-symbol}

\begin{verbatim}
Default:

$square circle square$

With `#h(0)`:

$square #h(0pt) circle #h(0pt) square$

With `math.class`:

#show math.circle: math.class.with("normal")
$square circle square$
\end{verbatim}

\pandocbounded{\includesvg[keepaspectratio]{typst-img/86a709c6189649b79005752253a842631eed4722b350e4197116e0be19094035-1.svg}}


\section{Examples Book LaTeX/book/basics/math/index.tex}
\title{sitandr.github.io/typst-examples-book/book/basics/math/index}

\section{\texorpdfstring{\hyperref[math]{Math}}{Math}}\label{math}

Math is a special environment that has special features related to...
math.

\subsection{\texorpdfstring{\hyperref[syntax]{Syntax}}{Syntax}}\label{syntax}

To start math environment, \texttt{\ }{\texttt{\ \$\ }}\texttt{\ } . The
spacing around \texttt{\ }{\texttt{\ \$\ }}\texttt{\ } will make it
either \emph{inline} math (smaller, used in text) or \emph{display} math
(used on math equations on their own).

\begin{verbatim}
// This is inline math
Let $a$, $b$, and $c$ be the side
lengths of right-angled triangle.
Then, we know that:

// This is display math
$ a^2 + b^2 = c^2 $

Prove by induction:

// You can use new lines as spacing too!
$
sum_(k=1)^n k = (n(n+1)) / 2
$
\end{verbatim}

\pandocbounded{\includesvg[keepaspectratio]{typst-img/068db3a521a38c3acede771ebb6342807cca4fd98baf5b2b508184a6854ea8ff-1.svg}}

\subsection{\texorpdfstring{\hyperref[mathequation]{Math.equation}}{Math.equation}}\label{mathequation}

The element that math is displayed in is called
\texttt{\ }{\texttt{\ math.equation\ }}\texttt{\ } . You can use it for
set/show rules:

\begin{verbatim}
#show math.equation: set text(red)

$
integral_0^oo (f(t) + g(t))/2
$
\end{verbatim}

\pandocbounded{\includesvg[keepaspectratio]{typst-img/94e0532dd7224d08e966cb82834283efd8889d7f117b04116e721a788bfcc16c-1.svg}}

Any symbol/command that is available in math, \emph{is also available}
in code mode using \texttt{\ }{\texttt{\ math.command\ }}\texttt{\ } :

\begin{verbatim}
#math.integral, #math.underbrace([a + b], [c])
\end{verbatim}

\pandocbounded{\includesvg[keepaspectratio]{typst-img/b4ca12d7f34ed342f3cb3fba2ee1f5b58faa8fceecb74671baacd9166fcbb5aa-1.svg}}

\subsection{\texorpdfstring{\hyperref[letters-and-commands]{Letters and
commands}}{Letters and commands}}\label{letters-and-commands}

Typst aims to have as simple and effective syntax for math as possible.
That means no special symbols, just using commands.

To make it short, Typst uses several simple rules:

\begin{itemize}
\item
  All single-letter words \emph{turn into variables} . That includes any
  \emph{unicode symbols} too!
\item
  All multi-letter words \emph{turn into commands} . They may be
  built-in commands (available with math.something outside of math
  environment). Or they \textbf{may be user-defined variables/functions}
  . If the command \textbf{isn\textquotesingle t defined} , there will
  be \textbf{compilation error} .

  If you use kebab-case or snake\_case for variables you want to use in
  math, you will have to refer to them as \#snake-case-variable.
\item
  To write simple text, use quotes:

\begin{verbatim}
$a "equals to" 2$
\end{verbatim}

  \pandocbounded{\includesvg[keepaspectratio]{typst-img/811f30ede68d08bec254f184c1be319958c3e11f9f9d58c40b2f460bba037e3d-1.svg}}

  Spacing matters there!

\begin{verbatim}
$a "is" 2$, $a"is"2$
\end{verbatim}

  \pandocbounded{\includesvg[keepaspectratio]{typst-img/9cc2d263c76646c623e1e6b73756e1fe1e2c56d7fe0324ee945652107e6456ba-1.svg}}
\item
  You can turn it into multi-letter variables using
  \texttt{\ }{\texttt{\ italic\ }}\texttt{\ } :

\begin{verbatim}
$(italic("mass") v^2)/2$
\end{verbatim}

  \pandocbounded{\includesvg[keepaspectratio]{typst-img/141d8a3b9beb3559387411170f7378078c80cb2ff80d8d5f5345c3231f55df9c-1.svg}}
\end{itemize}

Commands see
\href{https://typst.app/docs/reference/math/\#definitions}{there} (go to
the links to see the commands).

All symbols see
\href{https://typst.app/docs/reference/symbols/sym/}{there} .

\subsection{\texorpdfstring{\hyperref[multiline-equations]{Multiline
equations}}{Multiline equations}}\label{multiline-equations}

To create multiline \emph{display equation} , use the same symbol as in
markup mode: \texttt{\ }{\texttt{\ \textbackslash{}\ }}\texttt{\ } :

\begin{verbatim}
$
a = b\
a = c
$
\end{verbatim}

\pandocbounded{\includesvg[keepaspectratio]{typst-img/2f16d9e64e38ff22ca27a09b0d8eaef1b020e4eccd7d2ce1380e10a0efcea163-1.svg}}

\subsection{\texorpdfstring{\hyperref[escaping]{Escaping}}{Escaping}}\label{escaping}

Any symbol that is used may be escaped with
\texttt{\ }{\texttt{\ \textbackslash{}\ }}\texttt{\ } , like in markup
mode. For example, you can disable fraction:

\begin{verbatim}
$
a  / b \
a \/ b
$
\end{verbatim}

\pandocbounded{\includesvg[keepaspectratio]{typst-img/e7931e55a2772ee992446af8506d8d25b96167e3bb71d5c63ed8ca156530f2d9-1.svg}}

The same way it works with any other syntax.

\subsection{\texorpdfstring{\hyperref[wrapping-inline-math]{Wrapping
inline math}}{Wrapping inline math}}\label{wrapping-inline-math}

Sometimes, when you write large math, it may be too close to text
(especially for some long letter tails).

\begin{verbatim}
#lorem(17) $display(1)/display(1+x^n)$ #lorem(20)
\end{verbatim}

\pandocbounded{\includesvg[keepaspectratio]{typst-img/a9cce2b851a01939a0abfc02e8cd994d20c465d2800cf64c5c6051ead5bc4e9a-1.svg}}

You may easily increase the distance it by wrapping into box:

\begin{verbatim}
#lorem(17) #box($display(1)/display(1+x^n)$, inset: 0.2em) #lorem(20)
\end{verbatim}

\pandocbounded{\includesvg[keepaspectratio]{typst-img/ee9fc5a3ec529a9f3e811a70724c1585c294d82454c22ee9343235556f572792-1.svg}}


\section{Examples Book LaTeX/book/basics/math/alignment.tex}
\title{sitandr.github.io/typst-examples-book/book/basics/math/alignment}

\section{\texorpdfstring{\hyperref[alignment]{Alignment}}{Alignment}}\label{alignment}

\subsection{\texorpdfstring{\hyperref[general-alignment]{General
alignment}}{General alignment}}\label{general-alignment}

By default display math is center-aligned, but that can be set up with
\texttt{\ }{\texttt{\ show\ }}\texttt{\ } rule:

\begin{verbatim}
#show math.equation: set align(right)

$
(a + b)/2
$
\end{verbatim}

\pandocbounded{\includesvg[keepaspectratio]{typst-img/bcd19808066d4eee09c984bf17077653b1c1bf25115c10a155611056a30e2cb6-1.svg}}

Or using \texttt{\ }{\texttt{\ align\ }}\texttt{\ } element:

\begin{verbatim}
#align(left, block($ x = 5 $))
\end{verbatim}

\pandocbounded{\includesvg[keepaspectratio]{typst-img/4545bd54c4090d4c9599e639aa441b68eb214011861d9949652df140843af042-1.svg}}

\subsection{\texorpdfstring{\hyperref[alignment-points]{Alignment
points}}{Alignment points}}\label{alignment-points}

When equations include multiple alignment points (\&), this creates
blocks of alternatingly \emph{right-} and \emph{left-} aligned columns.

In the example below, the expression
\texttt{\ }{\texttt{\ (3x\ +\ y)\ /\ 7\ }}\texttt{\ } is
\emph{right-aligned} and
\texttt{\ }{\texttt{\ =\ }}\texttt{\ }{\texttt{\ 9\ }}\texttt{\ } is
\emph{left-aligned} .

\begin{verbatim}
$ (3x + y) / 7 &= 9 && "given" \
  3x + y &= 63 & "multiply by 7" \
  3x &= 63 - y && "subtract y" \
  x &= 21 - y/3 & "divide by 3" $
\end{verbatim}

\pandocbounded{\includesvg[keepaspectratio]{typst-img/bfb7a5df8873923079f45d12fa92204afeddecb15ec31d6b8624ac4610d29677-1.svg}}

The word "given" is also left-aligned because
\texttt{\ }{\texttt{\ \&\&\ }}\texttt{\ } creates two alignment points
in a row, \emph{alternating the alignment twice} .

\texttt{\ }{\texttt{\ \&\ \&\ }}\texttt{\ } and
\texttt{\ }{\texttt{\ \&\&\ }}\texttt{\ } behave exactly the same way.
Meanwhile, "multiply by 7" is left-aligned because just one
\texttt{\ }{\texttt{\ \&\ }}\texttt{\ } precedes it.

\textbf{Each alignment point simply alternates between
right-aligned/left-aligned.}


\section{Examples Book LaTeX/book/basics/math/sizes.tex}
\title{sitandr.github.io/typst-examples-book/book/basics/math/sizes}

\section{\texorpdfstring{\hyperref[location-and-sizes]{Location and
sizes}}{Location and sizes}}\label{location-and-sizes}

We talked already about display and inline math. They differ not only by
aligning and spacing, but also by size and style:

\begin{verbatim}
Inline: $a/(b + 1/c), sum_(n=0)^3 x_n$

$
a/(b + 1/c), sum_(n=0)^3 x_n
$
\end{verbatim}

\pandocbounded{\includesvg[keepaspectratio]{typst-img/7de20fcaee4fb6ea523b34bfe9b2be6b91cc6e6a5b46fab0eebe7f0155689f8e-1.svg}}

The size and style of current environment is described by Math Size, see
\href{https://typst.app/docs/reference/math/sizes}{reference} .

There are for sizes:

\begin{itemize}
\tightlist
\item
  Display math size ( \texttt{\ }{\texttt{\ display\ }}\texttt{\ } )
\item
  Inline math size ( \texttt{\ }{\texttt{\ inline\ }}\texttt{\ } )
\item
  Script math size ( \texttt{\ }{\texttt{\ script\ }}\texttt{\ } )
\item
  Sub/super script math size (
  \texttt{\ }{\texttt{\ sscript\ }}\texttt{\ } )
\end{itemize}

Each time thing is used in fraction, script or exponent, it is moved
several "levels lowers", becoming smaller and more "crapping".
\texttt{\ }{\texttt{\ sscript\ }}\texttt{\ } isn\textquotesingle t
reduced father:

\begin{verbatim}
$
"display:" 1/("inline:" a + 1/("script:" b + 1/("sscript:" c + 1/("sscript:" d + 1/("sscript:" e + 1/f)))))
$
\end{verbatim}

\pandocbounded{\includesvg[keepaspectratio]{typst-img/9c8cbc46da7dc8eb9436c561107cbb97a836aaa7b120a9bc3f044dd648d702e1-1.svg}}

\subsection{\texorpdfstring{\hyperref[setting-sizes-manually]{Setting
sizes manually}}{Setting sizes manually}}\label{setting-sizes-manually}

Just use the corresponding command:

\begin{verbatim}
Inine: $sum_0^oo e^x^a$\
Inline with limits: $limits(sum)_0^oo e^x^a$\
Inline, but like true display: $display(sum_0^oo e^x^a)$
\end{verbatim}

\pandocbounded{\includesvg[keepaspectratio]{typst-img/0d16a9d157c9689f4b3cce434ebf89d9a18d67b4916ac0ebfbce3daccb94e709-1.svg}}


\section{Examples Book LaTeX/book/basics/math/vec.tex}
\title{sitandr.github.io/typst-examples-book/book/basics/math/vec}

\section{\texorpdfstring{\hyperref[vectors-matrices-semicolumn-syntax]{Vectors,
matrices, semicolumn
syntax}}{Vectors, matrices, semicolumn syntax}}\label{vectors-matrices-semicolumn-syntax}

\subsection{\texorpdfstring{\hyperref[vectors]{Vectors}}{Vectors}}\label{vectors}

\begin{quote}
By vector we mean a column there.\\
To write arrow notations for letters, use
\texttt{\ }{\texttt{\ \$\ }}\texttt{\ }{\texttt{\ arrow\ }}\texttt{\ }{\texttt{\ (\ }}\texttt{\ }{\texttt{\ v\ }}\texttt{\ }{\texttt{\ )\ }}\texttt{\ }{\texttt{\ \$\ }}\texttt{\ }\\
I recommend to create shortcut for this, like
\texttt{\ }{\texttt{\ \#let\ }}\texttt{\ }{\texttt{\ arr\ }}\texttt{\ }{\texttt{\ =\ }}\texttt{\ }{\texttt{\ math.arrow\ }}\texttt{\ }
\end{quote}

To write columns, use \texttt{\ }{\texttt{\ vec\ }}\texttt{\ } command:

\begin{verbatim}
$
vec(a, b, c) + vec(1, 2, 3) = vec(a + 1, b + 2, c + 3)
$
\end{verbatim}

\pandocbounded{\includesvg[keepaspectratio]{typst-img/92aa72b3d4f797123f550cc8630b34e09176956c4b116cc0a4fe48d457e1ee0a-1.svg}}

\subsubsection{\texorpdfstring{\hyperref[delimiter]{Delimiter}}{Delimiter}}\label{delimiter}

You can change parentheses around the column or even remove them:

\begin{verbatim}
$
vec(1, 2, 3, delim: "{") \
vec(1, 2, 3, delim: bar.double) \
vec(1, 2, 3, delim: #none)
$
\end{verbatim}

\pandocbounded{\includesvg[keepaspectratio]{typst-img/efd7cc6c6abb317c316b746f7a286ab2f8b2a023fe19bf77c15638db9c6bed8f-1.svg}}

\subsubsection{\texorpdfstring{\hyperref[gap]{Gap}}{Gap}}\label{gap}

You can change the size of gap between rows:

\begin{verbatim}
$
vec(a, b, c)
vec(a, b, c, gap:#0em)
vec(a, b, c, gap:#1em)
$
\end{verbatim}

\pandocbounded{\includesvg[keepaspectratio]{typst-img/8977ff36f1f7a4b78c2fdbaef8764fec4b2cb42092f63b07176cca13707c0407-1.svg}}

\subsubsection{\texorpdfstring{\hyperref[making-gap-even]{Making gap
even}}{Making gap even}}\label{making-gap-even}

You can easily note that the gap isn\textquotesingle t necessarily even
or the same in different vectors:

\begin{verbatim}
$
vec(a/b, a/b, a/b) = vec(1, 1, 1)
$
\end{verbatim}

\pandocbounded{\includesvg[keepaspectratio]{typst-img/c3141fb95a4280df589e5e9fc0d605d59b16a8da6b4a01be532fab0bf04f6a00-1.svg}}

That happens because \texttt{\ }{\texttt{\ gap\ }}\texttt{\ } refers to
\emph{spacing between} elements, not the distance between their centers.

To fix this, you can use \href{../../snippets/math/vecs.html}{this
snippet} .

\subsection{\texorpdfstring{\hyperref[matrix]{Matrix}}{Matrix}}\label{matrix}

\begin{quote}
See \href{https://typst.app/docs/reference/math/mat/}{official
reference}
\end{quote}

Matrix is very similar to \texttt{\ }{\texttt{\ vec\ }}\texttt{\ } , but
accepts rows, separated by \texttt{\ }{\texttt{\ ;\ }}\texttt{\ } :

\begin{verbatim}
$
mat(
    1, 2, ..., 10;
    2, 2, ..., 10;
    dots.v, dots.v, dots.down, dots.v;
    10, 10, ..., 10; // `;` in the end is optional
)
$
\end{verbatim}

\pandocbounded{\includesvg[keepaspectratio]{typst-img/ca1e7bdfe61f2ae541843aeff854d40882487bed8fd5b1e094852cf662a759f8-1.svg}}

\subsubsection{\texorpdfstring{\hyperref[delimiters-and-gaps]{Delimiters
and gaps}}{Delimiters and gaps}}\label{delimiters-and-gaps}

You can specify them the same way as for vectors.

Specify the arguments either before the content, or \textbf{after the
semicolon} . The code will panic if there is no semicolon!

\begin{verbatim}
$
mat(
    delim: "|",
    1, 2, ..., 10;
    2, 2, ..., 10;
    dots.v, dots.v, dots.down, dots.v;
    10, 10, ..., 10;
    gap: #0.3em
)
$
\end{verbatim}

\pandocbounded{\includesvg[keepaspectratio]{typst-img/8fd5effce0cef589ea8f7e7388cf221f1c8d7f0ac6c76d8d7d2fb14c4840bef7-1.svg}}

\subsection{\texorpdfstring{\hyperref[semicolon-syntax]{Semicolon
syntax}}{Semicolon syntax}}\label{semicolon-syntax}

When you use semicolons, the arguments \emph{between the semicolons} are
merged into arrays. See yourself:

\begin{verbatim}
#let fun(..args) = {
    args.pos()
}

$
fun(1, 2;3, 4; 6, ; 8)
$
\end{verbatim}

\pandocbounded{\includesvg[keepaspectratio]{typst-img/a589a9f51ffa925d9dce1da521c4d15373e236fd19db49317091d681c2fface0-1.svg}}

If you miss some of elements, they will be replaced by
\texttt{\ }{\texttt{\ none\ }}\texttt{\ } -s.

You can mix semicolon syntax and named arguments, but be careful!

\begin{verbatim}
#let fun(..args) = {
    repr(args.pos())
    repr(args.named())
}

$
fun(1, 2; gap: #3em, 4)
$
\end{verbatim}

\pandocbounded{\includesvg[keepaspectratio]{typst-img/7a3c90212650f7f7df0cb42177753236eddae675ac3220fbabd0f40e4af8b842-1.svg}}

For example, this will not work:

\begin{verbatim}
$
//         ↓ there is no `;`, so it tries to add (gap:) to array
mat(1, 2; 4, gap: #3em)
$
\end{verbatim}


\section{Examples Book LaTeX/book/basics/math/operators.tex}
\title{sitandr.github.io/typst-examples-book/book/basics/math/operators}

\section{\texorpdfstring{\hyperref[operators]{Operators}}{Operators}}\label{operators}

\begin{quote}
See \href{https://typst.app/docs/reference/math/op/}{reference} .
\end{quote}

There are lots of built-in "text operators" in Typst math. This is a
symbol that behaves very close to plain text. Nevertheless, it is
different:

\begin{verbatim}
$
lim x_n, "lim" x_n, "lim"x_n
$
\end{verbatim}

\pandocbounded{\includesvg[keepaspectratio]{typst-img/b195783135218e8117ac954790e7a108297d7a3e532136d851e2c397358509f0-1.svg}}

\subsection{\texorpdfstring{\hyperref[predefined-operators]{Predefined
operators}}{Predefined operators}}\label{predefined-operators}

Here are all text operators Typst has built-in:

\begin{verbatim}
$
arccos, arcsin, arctan, arg, cos, cosh, cot, coth, csc,\
csch, ctg, deg, det, dim, exp, gcd, hom, id, im, inf, ker,\
lg, lim, liminf, limsup, ln, log, max, min, mod, Pr, sec,\
sech, sin, sinc, sinh, sup, tan, tanh, tg "and" tr
$
\end{verbatim}

\pandocbounded{\includesvg[keepaspectratio]{typst-img/8a14bfdd8bd657d613ccbcd3f77d68f31e6d73e509ba85dd8e6f5207d5c8c7e4-1.svg}}

\subsection{\texorpdfstring{\hyperref[creating-custom-operator]{Creating
custom
operator}}{Creating custom operator}}\label{creating-custom-operator}

Of course, there always will be some text operators you will need that
are not in the list.

But don\textquotesingle t worry, it is very easy to add your own:

\begin{verbatim}
#let arcsinh = math.op("arcsinh")

$
arcsinh x
$
\end{verbatim}

\pandocbounded{\includesvg[keepaspectratio]{typst-img/e4f5a9aa5dfd03914d26ad85ed73eff426d21badca21ea5a6e8de5032b2f29bb-1.svg}}

\subsubsection{\texorpdfstring{\hyperref[limits-for-operators]{Limits
for operators}}{Limits for operators}}\label{limits-for-operators}

When creating operators (upright text with proper spacing), you can set
limits for \emph{display mode} at the same time:

\begin{verbatim}
$
op("liminf")_a, op("liminf", limits: #true)_a
$
\end{verbatim}

\pandocbounded{\includesvg[keepaspectratio]{typst-img/9c3593b91bf3810a593b622e4972c5a87d637696f35850422f9232c74802a394-1.svg}}

This is roughly equivalent to

\begin{verbatim}
$
limits(op("liminf"))_a
$
\end{verbatim}

\pandocbounded{\includesvg[keepaspectratio]{typst-img/7aaabb25d8e73d54504aa3e99b9c8b341759f165923439447f4990871ec3943f-1.svg}}

Everything can be combined to create new operators:

\begin{verbatim}
#let liminf = math.op(math.underline(math.lim), limits: true)
#let limsup = math.op(math.overline(math.lim), limits: true)
#let integrate = math.op($integral dif x$)

$
liminf_(x->oo)\
limsup_(x->oo)\
integrate x^2
$
\end{verbatim}

\pandocbounded{\includesvg[keepaspectratio]{typst-img/adf6ee9659a71ecefb64d09f5f27f01acdd193bc79c792abf95fc56821bca4cb-1.svg}}


\section{Examples Book LaTeX/book/basics/math/limits.tex}
\title{sitandr.github.io/typst-examples-book/book/basics/math/limits}

\section{\texorpdfstring{\hyperref[setting-limits]{Setting
limits}}{Setting limits}}\label{setting-limits}

Sometimes we want to change how the default attaching should work.

\subsection{\texorpdfstring{\hyperref[limits]{Limits}}{Limits}}\label{limits}

For example, in many countries it is common to write definite integrals
with limits below and above. To set this, use
\texttt{\ }{\texttt{\ limits\ }}\texttt{\ } function:

\begin{verbatim}
$
integral_a^b\
limits(integral)_a^b
$
\end{verbatim}

\pandocbounded{\includesvg[keepaspectratio]{typst-img/ade8f85a6178d42d58769da477afa5349a3db9df3075a3d5f8e4a6b546c3d43e-1.svg}}

You can set this by default using
\texttt{\ }{\texttt{\ show\ }}\texttt{\ } rule:

\begin{verbatim}
#show math.integral: math.limits

$
integral_a^b
$

This is inline equation: $integral_a^b$
\end{verbatim}

\pandocbounded{\includesvg[keepaspectratio]{typst-img/e0011edccf76468c3d77a7502ce1dc001c82bfd9d590b258d8c8453d056bc966-1.svg}}

\subsection{\texorpdfstring{\hyperref[only-display-mode]{Only display
mode}}{Only display mode}}\label{only-display-mode}

Notice that this will also affect inline equations. To enable limits for
display math only, use
\texttt{\ }{\texttt{\ limits(inline:\ false)\ }}\texttt{\ } :

\begin{verbatim}
#show math.integral: math.limits.with(inline: false)

$
integral_a^b
$

This is inline equation: $integral_a^b$.
\end{verbatim}

\pandocbounded{\includesvg[keepaspectratio]{typst-img/d37f1132cdf338670e131079a57ae724a7dfcb102f3125dad712173fbf115bcd-1.svg}}

Of course, it is possible to move them back as bottom attachments:

\begin{verbatim}
$
sum_a^b, scripts(sum)_a^b
$
\end{verbatim}

\pandocbounded{\includesvg[keepaspectratio]{typst-img/7134a72120f7217b1f11438e166fa7e53f3a9287fa4c9079019181a6e16affb8-1.svg}}

\subsection{\texorpdfstring{\hyperref[operations]{Operations}}{Operations}}\label{operations}

The same scheme works for operations. By default, they are attached to
the bottom and top:

\begin{verbatim}
$a =_"By lemme 1" b, a scripts(=)_+ b$
\end{verbatim}

\pandocbounded{\includesvg[keepaspectratio]{typst-img/98d790005c43aa666b392f8a35f1e9564ff315aaf9881ceb309e53bd5db542b1-1.svg}}








\section{C Examples Book LaTeX/C Examples Book LaTeX.tex}
\section{C Examples Book LaTeX/cover.tex}
\section{Combined Examples Book LaTeX/front.tex}
\section{Examples Book LaTeX/book.tex}
\title{sitandr.github.io/typst-examples-book/book}

\section{\texorpdfstring{\hyperref[typst-examples-book]{Typst Examples
Book}}{Typst Examples Book}}\label{typst-examples-book}

This book provides an extended \emph{tutorial} and lots of
\href{https://github.com/typst/typst}{Typst} snippets that can help you
to write better Typst code.

This is an unofficial book. Some snippets \& suggestions here may be
outdated or useless (please let me know if you find some).

However, \emph{all of them should compile on last version of Typst
\textsuperscript{\hyperref[1]{1}}} .

\textbf{CAUTION:} the book is (probably forever) a \textbf{WIP} , so
don\textquotesingle t rely on it.

If you like it, consider
\href{https://github.com/sitandr/typst-examples-book}{giving a star on
GitHub} !

This will help me to stay motivated and continue working on this book.

\subsection{\texorpdfstring{\hyperref[navigation]{Navigation}}{Navigation}}\label{navigation}

The book consists of several chapters, each with its own goal:

\begin{enumerate}
\tightlist
\item
  \href{./basics/index.html}{Typst Basics}
\item
  \href{./snippets/index.html}{Typst Snippets}
\item
  \href{./packages/index.html}{Typst Packages}
\item
  \href{./typstonomicon/index.html}{Typstonomicon}
\end{enumerate}

\subsection{\texorpdfstring{\hyperref[contributions]{Contributions}}{Contributions}}\label{contributions}

Any contributions are very welcome! If you have a good code snippet that
you want to share, feel free to submit an issue with snippet or make a
PR in the
\href{https://github.com/sitandr/typst-examples-book}{repository} .

I will especially appreciate submissions of active community members and
compiler contributors!

However, I will also really appreciate feedback from beginners to make
the book as comprehensible as possible!

\subsection{\texorpdfstring{\hyperref[acknowledgements]{Acknowledgements}}{Acknowledgements}}\label{acknowledgements}

Thanks to everyone in the community who published their code snippets!

If someone doesn\textquotesingle t like their code and/or name being
published, please contact me.

\phantomsection\label{1}
\textsuperscript{1}

When a new version launches, it may take some time to update the book,
feel free to tag me to speed up the process.








