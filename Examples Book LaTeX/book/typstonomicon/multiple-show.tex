\title{sitandr.github.io/typst-examples-book/book/typstonomicon/multiple-show}

\subsection{\texorpdfstring{\hyperref[multiple-show-rules]{Multiple show
rules}}{Multiple show rules}}\label{multiple-show-rules}

Sometimes there is a need to apply several rules that look very similar.
Or generate them from code. One of the ways to deal with this, the most
cursed one, is this:

\begin{verbatim}
#let rules = (math.sum, math.product, math.root)

#let apply-rules(rules, it) = {
  if rules.len() == 0 {
    return it
  }
  show rules.pop(): math.display
  apply-rules(rules, it)
}

$product/sum root(3, x)/2$

#show: apply-rules.with(rules)

$product/sum root(3, x)/2$
\end{verbatim}

\pandocbounded{\includesvg[keepaspectratio]{typst-img/3f8166b0ca4ea7bdcf8017e914da7036f5b5ac804c34535f36b2a67bba3d995b-1.svg}}

The recursion problem may be avoided with the power of
\texttt{\ }{\texttt{\ fold\ }}\texttt{\ } , with basically the same
idea:

\begin{verbatim}
// author: Eric
#let kind_supp = (code: "Listing", algo: "Algorithme")
#show: it => kind_supp.pairs().fold(it, (acc, (kind, supp)) => {
  show figure.where(kind: kind): set figure(supplement: supp)
  acc
})
\end{verbatim}

\pandocbounded{\includesvg[keepaspectratio]{typst-img/e2ee1949cb74ef6dc8109f082f424dcb30765452043f5e93ccdd8a4fc30029b3-1.svg}}

Note that just in case of symbols (if you don\textquotesingle t need
element functions), one can use regular expressions. That is a more
robust way:

\begin{verbatim}
#show regex("[" + math.product + math.sum + "]"): math.display

$product/sum root(3, x)/2$
\end{verbatim}

\pandocbounded{\includesvg[keepaspectratio]{typst-img/b0f3afcb048a141cbfc9404f17ab9f91c701528560eb09810ce0bbaae66adbaa-1.svg}}
