\title{sitandr.github.io/typst-examples-book/book/basics/math/alignment}

\section{\texorpdfstring{\hyperref[alignment]{Alignment}}{Alignment}}\label{alignment}

\subsection{\texorpdfstring{\hyperref[general-alignment]{General
alignment}}{General alignment}}\label{general-alignment}

By default display math is center-aligned, but that can be set up with
\texttt{\ }{\texttt{\ show\ }}\texttt{\ } rule:

\begin{verbatim}
#show math.equation: set align(right)

$
(a + b)/2
$
\end{verbatim}

\pandocbounded{\includesvg[keepaspectratio]{typst-img/bcd19808066d4eee09c984bf17077653b1c1bf25115c10a155611056a30e2cb6-1.svg}}

Or using \texttt{\ }{\texttt{\ align\ }}\texttt{\ } element:

\begin{verbatim}
#align(left, block($ x = 5 $))
\end{verbatim}

\pandocbounded{\includesvg[keepaspectratio]{typst-img/4545bd54c4090d4c9599e639aa441b68eb214011861d9949652df140843af042-1.svg}}

\subsection{\texorpdfstring{\hyperref[alignment-points]{Alignment
points}}{Alignment points}}\label{alignment-points}

When equations include multiple alignment points (\&), this creates
blocks of alternatingly \emph{right-} and \emph{left-} aligned columns.

In the example below, the expression
\texttt{\ }{\texttt{\ (3x\ +\ y)\ /\ 7\ }}\texttt{\ } is
\emph{right-aligned} and
\texttt{\ }{\texttt{\ =\ }}\texttt{\ }{\texttt{\ 9\ }}\texttt{\ } is
\emph{left-aligned} .

\begin{verbatim}
$ (3x + y) / 7 &= 9 && "given" \
  3x + y &= 63 & "multiply by 7" \
  3x &= 63 - y && "subtract y" \
  x &= 21 - y/3 & "divide by 3" $
\end{verbatim}

\pandocbounded{\includesvg[keepaspectratio]{typst-img/bfb7a5df8873923079f45d12fa92204afeddecb15ec31d6b8624ac4610d29677-1.svg}}

The word "given" is also left-aligned because
\texttt{\ }{\texttt{\ \&\&\ }}\texttt{\ } creates two alignment points
in a row, \emph{alternating the alignment twice} .

\texttt{\ }{\texttt{\ \&\ \&\ }}\texttt{\ } and
\texttt{\ }{\texttt{\ \&\&\ }}\texttt{\ } behave exactly the same way.
Meanwhile, "multiply by 7" is left-aligned because just one
\texttt{\ }{\texttt{\ \&\ }}\texttt{\ } precedes it.

\textbf{Each alignment point simply alternates between
right-aligned/left-aligned.}
