\title{sitandr.github.io/typst-examples-book/book/basics/measure}

\section{\texorpdfstring{\hyperref[measure-layout]{Measure,
Layout}}{Measure, Layout}}\label{measure-layout}

This section is outdated. It may be still useful, but it is strongly
recommended to study new context system (using the reference).

\subsection{\texorpdfstring{\hyperref[style--measure]{Style \&
Measure}}{Style \& Measure}}\label{style--measure}

\begin{quote}
Style
\href{https://typst.app/docs/reference/foundations/style/}{documentation}
.
\end{quote}

\begin{quote}
Measure
\href{https://typst.app/docs/reference/layout/measure/}{documentation} .
\end{quote}

\texttt{\ }{\texttt{\ measure\ }}\texttt{\ } returns \emph{the element
size} . This command is extremely helpful when doing custom layout with
\texttt{\ }{\texttt{\ place\ }}\texttt{\ } .

However, there is a catch. Element size depends on styles, applied to
this element.

\begin{verbatim}
#let content = [Hello!]
#content
#set text(14pt)
#content
\end{verbatim}

\pandocbounded{\includesvg[keepaspectratio]{typst-img/00a6cbbc650947c03f34564786b0645eee60396f288d26333c591ff9059cc369-1.svg}}

So if we will set the big text size for some part of our text, to
measure the element\textquotesingle s size, we have to know \emph{where
the element is located} . Without knowing it, we can\textquotesingle t
tell what styles should be applied.

So we need a scheme similar to
\texttt{\ }{\texttt{\ locate\ }}\texttt{\ } .

This is what \texttt{\ }{\texttt{\ styles\ }}\texttt{\ } function is
used for. It is \emph{a content} , which, when located in document,
calls a function inside on \emph{current styles} .

Now, when we got fixed \texttt{\ }{\texttt{\ styles\ }}\texttt{\ } , we
can get the element\textquotesingle s size using
\texttt{\ }{\texttt{\ measure\ }}\texttt{\ } :

\begin{verbatim}
#let thing(body) = style(styles => {
  let size = measure(body, styles)
  [Width of "#body" is #size.width]
})

#thing[Hey] \
#thing[Welcome]
\end{verbatim}

\pandocbounded{\includesvg[keepaspectratio]{typst-img/5afe1855072b4ee8e343e5b5aa79affae5b17bc89738ffbe93dac245576cdd04-1.svg}}

\section{\texorpdfstring{\hyperref[layout]{Layout}}{Layout}}\label{layout}

Layout is similar to \texttt{\ }{\texttt{\ measure\ }}\texttt{\ } , but
it returns current scope \textbf{parent size} .

If you are putting elements in block, that will be
block\textquotesingle s size. If you are just putting right on the page,
that will be page\textquotesingle s size.

As parent\textquotesingle s size depends on it\textquotesingle s place
in document, it uses the similar scheme to
\texttt{\ }{\texttt{\ locate\ }}\texttt{\ } and
\texttt{\ }{\texttt{\ style\ }}\texttt{\ } :

\begin{verbatim}
#layout(size => {
  let half = 50% * size.width
  [Half a page is #half wide.]
})
\end{verbatim}

\pandocbounded{\includesvg[keepaspectratio]{typst-img/c68a166f6e6b1b3229fd56478ae302dbeb39c882e229c69d4c6ebb6c9c528985-1.svg}}

It may be extremely useful to combine
\texttt{\ }{\texttt{\ layout\ }}\texttt{\ } with
\texttt{\ }{\texttt{\ measure\ }}\texttt{\ } , to get width of things
that depend on parent\textquotesingle s size:

\begin{verbatim}
#let text = lorem(30)
#layout(size => style(styles => [
  #let (height,) = measure(
    block(width: size.width, text),
    styles,
  )
  This text is #height high with
  the current page width: \
  #text
]))
\end{verbatim}

\pandocbounded{\includesvg[keepaspectratio]{typst-img/93167dc0b22b02fe27aa92c6b03c2281665b4352624364a19c63f61a488aa75a-1.svg}}
