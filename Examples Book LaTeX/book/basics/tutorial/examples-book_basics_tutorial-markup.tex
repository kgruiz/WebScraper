\title{sitandr.github.io/typst-examples-book/book/basics/tutorial/markup}

\section{\texorpdfstring{\hyperref[markup-language]{Markup
language}}{Markup language}}\label{markup-language}

\subsection{\texorpdfstring{\hyperref[starting]{Starting}}{Starting}}\label{starting}

\begin{verbatim}
Starting typing in Typst is easy.
You don't need packages or other weird things for most of things.

Blank line will move text to a new paragraph.

Btw, you can use any language and unicode symbols
without any problems as long as the font supports it: ßçœ̃ɛ̃ø∀αβёыა😆…
\end{verbatim}

\pandocbounded{\includesvg[keepaspectratio]{typst-img/ee9f64251c99c7aeaaf6fa1d5bc7e907c2d51a34aa38126544d515ca197ca2a8-1.svg}}

\subsection{\texorpdfstring{\hyperref[markup]{Markup}}{Markup}}\label{markup}

\begin{verbatim}
= Markup

This was a heading. Number of `=` in front of name corresponds to heading level.

== Second-level heading

Okay, let's move to _emphasis_ and *bold* text.

Markup syntax is generally similar to
`AsciiDoc` (this was `raw` for monospace text!)
\end{verbatim}

\pandocbounded{\includesvg[keepaspectratio]{typst-img/fa8b95f9b15083387a29c11d17efca9873b8e778643b1b5079aa137891d01c8d-1.svg}}

\subsection{\texorpdfstring{\hyperref[new-lines--escaping]{New lines \&
Escaping}}{New lines \& Escaping}}\label{new-lines--escaping}

\begin{verbatim}
You can break \
line anywhere you \
want using "\\" symbol.

Also you can use that symbol to
escape \_all the symbols you want\_,
if you don't want it to be interpreted as markup
or other special symbols.
\end{verbatim}

\pandocbounded{\includesvg[keepaspectratio]{typst-img/4dabdee2a61e7d10773d51772dba3665271a09d4d5df4a8f66dd80589f0bcd7a-1.svg}}

\subsection{\texorpdfstring{\hyperref[comments--codeblocks]{Comments \&
codeblocks}}{Comments \& codeblocks}}\label{comments--codeblocks}

\begin{verbatim}
You can write comments with `//` and `/* comment */`:
// Like this
/* Or even like
this */

```typ
Just in case you didn't read source,
this is how it is written:

// Like this
/* Or even like
this */

By the way, I'm writing it all in a _fenced code block_ with *syntax highlighting*!
```
\end{verbatim}

\pandocbounded{\includesvg[keepaspectratio]{typst-img/a481d12b3ed0bbe2d9db6cc4b4a1237cba9936de83333254dfce8702832db125-1.svg}}

\subsection{\texorpdfstring{\hyperref[smart-quotes]{Smart
quotes}}{Smart quotes}}\label{smart-quotes}

\begin{verbatim}
== What else?

There are not much things in basic "markup" syntax,
but we will see much more interesting things very soon!
I hope you noticed auto-matched "smart quotes" there.
\end{verbatim}

\pandocbounded{\includesvg[keepaspectratio]{typst-img/89114a6e9af45c2eb9db2ef44d0e5ba41e31bf816e72803bd1a9a02120e69fc3-1.svg}}

\subsection{\texorpdfstring{\hyperref[lists]{Lists}}{Lists}}\label{lists}

\begin{verbatim}
- Writing lists in a simple way is great.
- Nothing complex, start your points with `-`
  and this will become a list.
  - Indented lists are created via indentation.

+ Numbered lists start with `+` instead of `-`.
+ There is no alternative markup syntax for lists
+ So just remember `-` and `+`, all other symbols
  wouldn't work in an unintended way.
  + That is a general property of Typst's markup.
  + Unlike Markdown, there is only one way
    to write something with it.
\end{verbatim}

\pandocbounded{\includesvg[keepaspectratio]{typst-img/ad4e424e067a4362e9f145c0c4ba4b7c1b65e17e7d0e7631b6836841607ef85e-1.svg}}

\textbf{Notice:}

\begin{verbatim}
Typst numbered lists differ from markdown-like syntax for lists. If you write them by hand, numbering is preserved:

1. Apple
1. Orange
1. Peach
\end{verbatim}

\pandocbounded{\includesvg[keepaspectratio]{typst-img/477695c86becc136dceb144e90c0acd2b75faa2a49743f8673d09974b71da324-1.svg}}

\subsection{\texorpdfstring{\hyperref[math]{Math}}{Math}}\label{math}

\begin{verbatim}
I will just mention math ($a + b/c = sum_i x^i$)
is possible and quite pretty there:

$
7.32 beta +
  sum_(i=0)^nabla
    (Q_i (a_i - epsilon)) / 2
$

To learn more about math, see corresponding chapter.
\end{verbatim}

\pandocbounded{\includesvg[keepaspectratio]{typst-img/12cc318c8438cd8e91706013bbd53fee5ee004620a63348cfe2d7dcc3b8a19d4-1.svg}}
