\title{sitandr.github.io/typst-examples-book/book/basics/tutorial/basic_styling}

\section{\texorpdfstring{\hyperref[basic-styling]{Basic
styling}}{Basic styling}}\label{basic-styling}

\subsection{\texorpdfstring{\hyperref[set-rule]{\texttt{\ }{\texttt{\ Set\ }}\texttt{\ }
rule}}{  Set   rule}}\label{set-rule}

\begin{verbatim}
#set page(width: 15cm, margin: (left: 4cm, right: 4cm))

That was great, but using functions everywhere, especially
with many arguments every time is awfully cumbersome.

That's why Typst has _rules_. No, not for you, for the document.

#set par(justify: true)

And the first rule we will consider there is `set` rule.
As you see, I've just used it on `par` (which is short from paragraph)
and now all paragraphs became _justified_.

It will apply to all paragraphs after the rule,
but will work only in it's _scope_ (we will discuss them later).

#par(justify: false)[
  Of course, you can override a `set` rule.
  This rule just sets the _default value_
  of an argument of an element.
]

By the way, at first line of this snippet
I've reduced page size to make justifying more visible,
also increasing margins to add blank space on left and right.
\end{verbatim}

\pandocbounded{\includesvg[keepaspectratio]{typst-img/cee42a8b1274afa36891438d4b1611eb55b2cd8bb4546df47128a7d3eb66653b-1.svg}}

\subsection{\texorpdfstring{\hyperref[a-bit-about-length-units]{A bit
about length
units}}{A bit about length units}}\label{a-bit-about-length-units}

\begin{verbatim}
Before we continue with rules, we should talk about length. There are several absolute length units in Typst:

#set rect(height: 1em)

#table(
  columns: 2,
  [Points], rect(width: 72pt),
  [Millimeters], rect(width: 25.4mm),
  [Centimeters], rect(width: 2.54cm),
  [Inches], rect(width: 1in),
  [Relative to font size], rect(width: 6.5em)
)

`1 em` = current font size. \
It is a very convenient unit,
so we are going to use it a lot
\end{verbatim}

\pandocbounded{\includesvg[keepaspectratio]{typst-img/5f8abc94a3d9df0e16f78c258e7f487d5698b4c96491300b3a48ad8e685534bc-1.svg}}

\subsection{\texorpdfstring{\hyperref[setting-something-else]{Setting
something else}}{Setting something else}}\label{setting-something-else}

Of course, you can use \texttt{\ }{\texttt{\ set\ }}\texttt{\ } rule
with all built-in functions and all their named arguments to make some
argument "default".

For example, let\textquotesingle s make all quotes in this snippet
authored by the book:

\begin{verbatim}
#set quote(block: true, attribution: [Typst Examples Book])

#quote[
  Typst is great!
]

#quote[
  The problem with quotes on the internet is
  that it is hard to verify their authenticity.
]
\end{verbatim}

\pandocbounded{\includesvg[keepaspectratio]{typst-img/c34c25cad05b7c20b6e0f146002886a1de65b61f48666cfec3d3494bd694a641-1.svg}}

\subsection{\texorpdfstring{\hyperref[opinionated-defaults]{Opinionated
defaults}}{Opinionated defaults}}\label{opinionated-defaults}

That allows you to set Typst default styling as you want it:

\begin{verbatim}
#set par(justify: true)
#set list(indent: 1em)
#set enum(indent: 1em)
#set page(numbering: "1")

- List item
- List item

+ Enum item
+ Enum item
\end{verbatim}

\pandocbounded{\includesvg[keepaspectratio]{typst-img/773d68bc55eb89f119ad07b882eae5fd31868d8a1bb3d4963573ec80fb1c7466-1.svg}}

Don\textquotesingle t complain about bad defaults!
\texttt{\ }{\texttt{\ Set\ }}\texttt{\ } your own.

\subsection{\texorpdfstring{\hyperref[numbering]{Numbering}}{Numbering}}\label{numbering}

\begin{verbatim}
= Numbering

Some of elements have a property called "numbering".
They accept so-called "numbering patterns" and
are very useful with set rules. Let's see what I mean.

#set heading(numbering: "I.1:")

= This is first level
= Another first
== Second
== Another second
=== Now third
== And second again
= Now returning to first
= These are actual romanian numerals
\end{verbatim}

\pandocbounded{\includesvg[keepaspectratio]{typst-img/39fb958032888b1e41da849152fed716b6f590eed3ea975b051ab786fac4ce5c-1.svg}}

Of course, there are lots of other cool properties that can be
\emph{set} , so feel free to dive into
\href{https://typst.app/docs/reference/}{Official Reference} and explore
them!

And now we are moving into something much more interesting\ldots{}
