\title{sitandr.github.io/typst-examples-book/book/basics/scripting/types_2}

\section{\texorpdfstring{\hyperref[types-part-ii]{Types, part
II}}{Types, part II}}\label{types-part-ii}

In Typst, most of things are \textbf{immutable} . You
can\textquotesingle t change content, you can just create new using this
one (for example, using addition).

Immutability is very important for Typst since it tries to be \emph{as
pure language as possible} . Functions do nothing outside of returning
some value.

However, purity is partly "broken" by these types. They are
\emph{super-useful} and not adding them would make Typst much pain.

However, using them adds complexity.

\subsection{\texorpdfstring{\hyperref[arrays-array]{Arrays (
\texttt{\ }{\texttt{\ array\ }}\texttt{\ }
)}}{Arrays (   array   )}}\label{arrays-array}

\begin{quote}
\href{https://typst.app/docs/reference/foundations/array/}{Link to
Reference} .
\end{quote}

Mutable object that stores data with their indices.

\subsubsection{\texorpdfstring{\hyperref[working-with-indices]{Working
with indices}}{Working with indices}}\label{working-with-indices}

\begin{verbatim}
#let values = (1, 7, 4, -3, 2)

// take value at index 0
#values.at(0) \
// set value at 0 to 3
#(values.at(0) = 3)
// negative index => start from the back
#values.at(-1) \
// add index of something that is even
#values.find(calc.even)
\end{verbatim}

\pandocbounded{\includesvg[keepaspectratio]{typst-img/0374c20b28fbf2b2d15bc32e5428f7f5121ea9d673d96de3274a0c6d988d5fb5-1.svg}}

\subsubsection{\texorpdfstring{\hyperref[iterating-methods]{Iterating
methods}}{Iterating methods}}\label{iterating-methods}

\begin{verbatim}
#let values = (1, 7, 4, -3, 2)

// leave only what is odd
#values.filter(calc.odd) \
// create new list of absolute values of list values
#values.map(calc.abs) \
// reverse
#values.rev() \
// convert array of arrays to flat array
#(1, (2, 3)).flatten() \
// join array of string to string
#(("A", "B", "C")
 .join(", ", last: " and "))
\end{verbatim}

\pandocbounded{\includesvg[keepaspectratio]{typst-img/684400186916f8f16a2d7edb151b7f5023c7e4c010b23a2c6566f0bd7a224061-1.svg}}

\subsubsection{\texorpdfstring{\hyperref[list-operations]{List
operations}}{List operations}}\label{list-operations}

\begin{verbatim}
// sum of lists:
#((1, 2, 3) + (4, 5, 6))

// list product:
#((1, 2, 3) * 4)
\end{verbatim}

\pandocbounded{\includesvg[keepaspectratio]{typst-img/abe2d311638b351e0938be0e432f10265ca81a69a9ed7d2e6f88f656c60dfc65-1.svg}}

\subsubsection{\texorpdfstring{\hyperref[empty-list]{Empty
list}}{Empty list}}\label{empty-list}

\begin{verbatim}
#() \ // this is an empty list
#(1,) \  // this is a list with one element
BAD: #(1) // this is just an element, not a list!
\end{verbatim}

\pandocbounded{\includesvg[keepaspectratio]{typst-img/da4f77f8784462ca5c4f73862e58420695916064d56921e4adef7a7e37d5a532-1.svg}}

\subsection{\texorpdfstring{\hyperref[dictionaries-dict]{Dictionaries (
\texttt{\ }{\texttt{\ dict\ }}\texttt{\ }
)}}{Dictionaries (   dict   )}}\label{dictionaries-dict}

\begin{quote}
\href{https://typst.app/docs/reference/foundations/dictionary/}{Link to
Reference} .
\end{quote}

Dictionaries are objects that store a string "key" and a value,
associated with that key.

\begin{verbatim}
#let dict = (
  name: "Typst",
  born: 2019,
)

#dict.name \
#(dict.launch = 20)
#dict.len() \
#dict.keys() \
#dict.values() \
#dict.at("born") \
#dict.insert("city", "Berlin ")
#("name" in dict)
\end{verbatim}

\pandocbounded{\includesvg[keepaspectratio]{typst-img/638ada64eb36af0b1891def1b2c0a2cc97a14d87987df8c16f5f3872244553d6-1.svg}}

\subsubsection{\texorpdfstring{\hyperref[empty-dictionary]{Empty
dictionary}}{Empty dictionary}}\label{empty-dictionary}

\begin{verbatim}
This is an empty list: #() \
This is an empty dict: #(:)
\end{verbatim}

\pandocbounded{\includesvg[keepaspectratio]{typst-img/6ef41801d46f0b7256bb6913482fde054c811a1850ecae3a446660eb6d1c8850-1.svg}}
