\title{sitandr.github.io/typst-examples-book/book/basics/states/query}

\section{\texorpdfstring{\hyperref[query]{Query}}{Query}}\label{query}

This section is outdated. It may be still useful, but it is strongly
recommended to study new context system (using the reference).

\begin{quote}
Link \href{https://typst.app/docs/reference/meta/query/}{there}
\end{quote}

Query is a thing that allows you getting location by \emph{selector}
(this is the same thing we used in show rules).

That enables "time travel", getting information about document from its
parts and so on. \emph{That is a way to violate Typst\textquotesingle s
purity.}

It is currently one of the \emph{the darkest magics currently existing
in Typst} . It gives you great powers, but with great power comes great
responsibility.

\subsection{\texorpdfstring{\hyperref[time-travel]{Time
travel}}{Time travel}}\label{time-travel}

\begin{verbatim}
#let s = state("x", 0)
#let compute(expr) = [
  #s.update(x =>
    eval(expr.replace("x", str(x)))
  )
  New value is #s.display().
]

Value at `<here>` is
#context s.at(
  query(<here>)
    .first()
    .location()
)

#compute("10") \
#compute("x + 3") \
*Here.* <here> \
#compute("x * 2") \
#compute("x - 5")
\end{verbatim}

\pandocbounded{\includesvg[keepaspectratio]{typst-img/130940aa5ae2ceb3364ef655c84cf8e7d2178210851b8fb20e6c0c3345c3ace7-1.svg}}

\subsection{\texorpdfstring{\hyperref[getting-nearest-chapter]{Getting
nearest
chapter}}{Getting nearest chapter}}\label{getting-nearest-chapter}

\begin{verbatim}
#set page(header: context {
  let elems = query(
    selector(heading).before(here()),
    here(),
  )
  let academy = smallcaps[
    Typst Academy
  ]
  if elems == () {
    align(right, academy)
  } else {
    let body = elems.last().body
    academy + h(1fr) + emph(body)
  }
})

= Introduction
#lorem(23)

= Background
#lorem(30)

= Analysis
#lorem(15)
\end{verbatim}

\pandocbounded{\includesvg[keepaspectratio]{typst-img/b4d0562911dd308b0d9cbc36ad20ba6ed91fc3c3da5b6259eb6721f3a53a18e3-1.svg}}
