\title{sitandr.github.io/typst-examples-book/book/packages/index}

\section{\texorpdfstring{\hyperref[packages]{Packages}}{Packages}}\label{packages}

Once the \href{https://typst.app/universe}{Typst Universe} was launched,
this chapter has become almost redundant. The Universe is actually a
very cool place to look for packages.

However, there are still some cool examples of interesting package
usage.

\subsection{\texorpdfstring{\hyperref[general]{General}}{General}}\label{general}

Typst has packages, but, unlike LaTeX, you need to remember:

\begin{itemize}
\tightlist
\item
  You need them only for some specialized tasks, basic formatting
  \emph{can be totally done without them} .
\item
  Packages are much lighter and much easier "installed" than LaTeX ones.
\item
  Packages are just plain Typst files (and sometimes plugins), so you
  can easily write your own!
\end{itemize}

To use mighty package, just write, like this:

\begin{verbatim}
#import "@preview/cetz:0.1.2": canvas, plot

#canvas(length: 1cm, {
  plot.plot(size: (8, 6),
    x-tick-step: none,
    x-ticks: ((-calc.pi, $-pi$), (0, $0$), (calc.pi, $pi$)),
    y-tick-step: 1,
    {
      plot.add(
        style: plot.palette.blue,
        domain: (-calc.pi, calc.pi), x => calc.sin(x * 1rad))
      plot.add(
        hypograph: true,
        style: plot.palette.blue,
        domain: (-calc.pi, calc.pi), x => calc.cos(x * 1rad))
      plot.add(
        hypograph: true,
        style: plot.palette.blue,
        domain: (-calc.pi, calc.pi), x => calc.cos((x + calc.pi) * 1rad))
    })
})
\end{verbatim}

\pandocbounded{\includesvg[keepaspectratio]{typst-img/29d7015ed96122fa3fb663929c1ac58d25340995423c82456ab8815811373979-1.svg}}

\subsection{\texorpdfstring{\hyperref[contributing]{Contributing}}{Contributing}}\label{contributing}

If you are author of a package or just want to make a fair overview,
feel free to make issues/PR-s!
