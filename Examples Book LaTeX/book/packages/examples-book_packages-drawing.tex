\title{sitandr.github.io/typst-examples-book/book/packages/drawing}

\section{\texorpdfstring{\hyperref[drawing]{Drawing}}{Drawing}}\label{drawing}

\subsection{\texorpdfstring{\hyperref[cetz]{\texttt{\ }{\texttt{\ cetz\ }}\texttt{\ }}}{  cetz  }}\label{cetz}

Cetz is an analogue of LaTeX\textquotesingle s
\texttt{\ }{\texttt{\ tikz\ }}\texttt{\ } . Maybe it is not as powerful
yet, but certainly easier to learn and use.

It is the best choice in most of cases you want to draw something in
Typst.

\begin{verbatim}
#import "@preview/cetz:0.1.2"

#cetz.canvas(length: 1cm, {
  import cetz.draw: *
  import cetz.angle: angle
  let (a, b, c) = ((0,0), (-1,1), (1.5,0))
  line(a, b)
  line(a, c)
  set-style(angle: (radius: 1, label-radius: .5), stroke: blue)
  angle(a, c, b, label: $alpha$, mark: (end: ">"), stroke: blue)
  set-style(stroke: red)
  angle(a, b, c, label: n => $#{n/1deg} degree$,
    mark: (end: ">"), stroke: red, inner: false)
})
\end{verbatim}

\pandocbounded{\includesvg[keepaspectratio]{typst-img/d3b5277dd18dffb6da9a8f41486cb85a5044597821e80867652f062724ed8dd4-1.svg}}

\begin{verbatim}
#import "@preview/cetz:0.1.2": canvas, draw

#canvas(length: 1cm, {
  import draw: *
  intersections(name: "demo", {
    circle((0, 0))
    bezier((0,0), (3,0), (1,-1), (2,1))
    line((0,-1), (0,1))
    rect((1.5,-1),(2.5,1))
  })
  for-each-anchor("demo", (name) => {
    circle("demo." + name, radius: .1, fill: black)
  })
})
\end{verbatim}

\pandocbounded{\includesvg[keepaspectratio]{typst-img/05a1dbe2a2d17e5e81991406bed640775db6ab4ce2d585bc5a0d1175def43ea1-1.svg}}

\begin{verbatim}
#import "@preview/cetz:0.1.2": canvas, draw

#canvas(length: 1cm, {
  import draw: *
  let (a, b, c) = ((0, 0), (1, 1), (2, -1))
  line(a, b, c, stroke: gray)
  bezier-through(a, b, c, name: "b")
  // Show calculated control points
  line(a, "b.ctrl-1", "b.ctrl-2", c, stroke: gray)
})
\end{verbatim}

\pandocbounded{\includesvg[keepaspectratio]{typst-img/8e7d39d73212ebf8f230a0bd54a7fb7e58607a99f327e29809c4021b9e797345-1.svg}}

\begin{verbatim}
#import "@preview/cetz:0.1.2": canvas, draw

#canvas(length: 1cm, {
  import draw: *
  group(name: "g", {
    rotate(45deg)
    rect((0,0), (1,1), name: "r")
    copy-anchors("r")
  })
  circle("g.top", radius: .1, fill: black)
})
\end{verbatim}

\pandocbounded{\includesvg[keepaspectratio]{typst-img/b3d0b37a84cddb77a1508333743f851509e2250930abdcbda7ec4675e00077c3-1.svg}}

\begin{verbatim}
// author: LDemetrios
#import "@preview/cetz:0.2.2"

#cetz.canvas({
  let left = (a:2, b:1, d:-1, e:-2)
  let right = (p:2.7, q: 1.8, r: 0.9, s: -.3, t: -1.5, u: -2.4)
  let edges = "as,bq,dq,et".split(",")

  let ell-width = 1.5
  let ell-height = 3
  let dist = 5
  let dot-radius = 0.1
  let dot-clr = blue

  import cetz.draw: *
  circle((-dist/2, 0), radius:(ell-width ,  ell-height))
  circle((+dist/2, 0), radius:(ell-width ,  ell-height))

  for (name, y) in left {
    circle((-dist/2, y), radius:dot-radius, fill:dot-clr, name:name)
    content(name, anchor:"east", pad(right:.7em, text(fill:dot-clr, name)))
  }

  for (name, y) in right {
    circle((dist/2, y), radius:dot-radius, fill:dot-clr, name:name)
    content(name, anchor:"west", pad(left:.7em, text(fill:dot-clr, name)))
  }

  for edge in edges {
    let from = edge.at(0)
    let to = edge.at(1)
    line(from, to)
    mark(from, to, symbol: ">",  fill: black)
  }

  content((0, - ell-height), text(fill:blue)[APPLICATION], anchor:"south")
})
\end{verbatim}

\pandocbounded{\includesvg[keepaspectratio]{typst-img/7a4a9224b76305ecd694fd4505b3fdee8c706ccea76ac0e59fd13d469c343dd4-1.svg}}
