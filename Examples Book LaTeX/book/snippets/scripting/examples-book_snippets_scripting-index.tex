\title{sitandr.github.io/typst-examples-book/book/snippets/scripting/index}

\section{\texorpdfstring{\hyperref[scripting]{Scripting}}{Scripting}}\label{scripting}

\subsection{\texorpdfstring{\hyperref[unflatten-arrays]{Unflatten
arrays}}{Unflatten arrays}}\label{unflatten-arrays}

\begin{verbatim}
// author: PgSuper
#let unflatten(arr, n) = {
  let columns = range(0, n).map(_ => ())
  for (i, x) in arr.enumerate() {
    columns.at(calc.rem(i, n)).push(x)
  }
  array.zip(..columns)
}

#unflatten((1, 2, 3, 4, 5, 6), 2)
#unflatten((1, 2, 3, 4, 5, 6), 3)
\end{verbatim}

\pandocbounded{\includesvg[keepaspectratio]{typst-img/98271a255f0fb10f31ba1d8199ba5a91ebb6f647cdd570220f95e1b88d193ca0-1.svg}}

\subsection{\texorpdfstring{\hyperref[create-an-abbreviation]{Create an
abbreviation}}{Create an abbreviation}}\label{create-an-abbreviation}

\begin{verbatim}
#let full-name = "Federal University of Ceará"

#let letts = {
  full-name
    .split()
    .map(word => word.at(0)) // filter only capital letters
    .filter(l => upper(l) == l)
    .join()
}
#letts
\end{verbatim}

\pandocbounded{\includesvg[keepaspectratio]{typst-img/e95b77243a1305a47517cb128577d1c7633d858561de0ef797ff551f35be40de-1.svg}}

\subsection{\texorpdfstring{\hyperref[split-the-string-retrieving-separators]{Split
the string retrieving
separators}}{Split the string retrieving separators}}\label{split-the-string-retrieving-separators}

\begin{verbatim}
#",this, is a a a a; a. test? string!".matches(regex("(\b[\P{Punct}\s]+\b|\p{Punct})")).map(x => x.captures).join()
\end{verbatim}

\pandocbounded{\includesvg[keepaspectratio]{typst-img/c5d183e45097449e4f52b07f82185847092ad28bcad3b9474093d341c4b07c4a-1.svg}}

\subsection{\texorpdfstring{\hyperref[create-selector-matching-any-values-in-an-array]{Create
selector matching any values in an
array}}{Create selector matching any values in an array}}\label{create-selector-matching-any-values-in-an-array}

This snippet creates a selector (that is then used in a show rule) that
matches any of the values inside the array. Here, it is used to
highlight a few raw lines, but it can be easily adapted to any kind of
selector.

\begin{verbatim}
// author: Blokyk
#let lines = (2, 3, 5)
#let lines-selectors = lines.map(lineno => raw.line.where(number: lineno))
#let lines-combined-selector = lines-selectors.fold(
  // start with the first selector by default
  // you can also use a selector that wouldn't ever match anything, if possible
  lines-selectors.at(0),
  selector.or // create an OR of all selectors (alternatively: (acc, sel) => acc.or(sel))
)

#show lines-combined-selector: highlight

```py
def foo(x, y):
  if x == y:
    return False
  z = x + y
  return z * x - z * y >= z
```
\end{verbatim}

\pandocbounded{\includesvg[keepaspectratio]{typst-img/085d1ae3a0672ba278edcde3ebb229a34a40ab5166d0b6d5b469d838b9262a51-1.svg}}

\subsection{\texorpdfstring{\hyperref[synthesize-show-or-show-set-rules-from-dictionnary]{Synthesize
show (or show-set) rules from
dictionnary}}{Synthesize show (or show-set) rules from dictionnary}}\label{synthesize-show-or-show-set-rules-from-dictionnary}

This snippet applies a show-set rule to any element inside a dictionary,
by using the key as the selector and the value as the parameter to set.
In this example, it\textquotesingle s used to give custom supplements to
custom figure kinds, based on a dictionnary of correspondances.

\begin{verbatim}
// author: laurmaedje
#let kind_supp_dict = (
  algo: "Pseudo-code",
  ex: "Example",
  prob: "Problem",
)

// apply this rule to the whole (rest of the) document
#show: it => {
  kind_supp_dict
    .pairs() // get an array of key-value pairs
    .fold( // we're going to stack show-set rules before the document
      it, // start with the default document
      (acc, (kind, supp)) => {
        // add the curent kind-supp combination on top of the rest
        show figure.where(kind: kind): set figure(supplement: supp)
        acc
      }
    )
}
#figure(
    kind: "algo",
    caption: [My code],
    ```Algorithm there```
)
\end{verbatim}

\pandocbounded{\includesvg[keepaspectratio]{typst-img/9de9b5f4bb801735b13ffafe54d35ebcfc78f1df78a34a8ab90f8a6c350b986e-1.svg}}

Additonnaly, as this is applied at the position where you write it,
these show-set rules will appear as if they were added in the same place
where you wrote this rule. This means that you can override them later,
just like any other show-set rules.
