\title{typst.app/universe/package/minerva-report-fcfm}

\phantomsection\label{banner}
\phantomsection\label{template-thumbnail}
\pandocbounded{\includegraphics[keepaspectratio]{https://packages.typst.org/preview/thumbnails/minerva-report-fcfm-0.2.1-small.webp}}

\section{minerva-report-fcfm}\label{minerva-report-fcfm}

{ 0.2.1 }

Template de artículos, informes y tareas para la Facultad de Ciencias
Físicas y Matemáticas (FCFM).

\href{/app?template=minerva-report-fcfm&version=0.2.1}{Create project in
app}

\phantomsection\label{readme}
Template para hacer tareas, informes y trabajos, para estudiantes y
académicos de la Facultad de Ciencias Físicas y Matemáticas de la
Universidad de Chile que han usado templates similares para LaTeX.

\subsection{Guía Rápida}\label{guuxe3a-ruxe3pida}

\subsubsection{\texorpdfstring{\href{https://typst.app/}{Webapp}}{Webapp}}\label{webapp}

Si utilizas la webapp de Typst puedes presionar “Start from
template� y buscar “minerva-report-fcfm� para crear un nuevo
proyecto con este template.

\subsubsection{Typst CLI}\label{typst-cli}

Teniendo el CLI con la versión 0.11.0 o mayor, puedes realizar:

\begin{Shaded}
\begin{Highlighting}[]
\ExtensionTok{typst}\NormalTok{ init @preview/minerva{-}report{-}fcfm:0.2.1}
\end{Highlighting}
\end{Shaded}

Esto va a descargar el template en la cache de typst y luego va a
iniciar el proyecto en la carpeta actual.

\subsection{Configuración}\label{configuraciuxe3uxb3n}

La mayoría de la configuración se realiza a través del archivo
\texttt{\ meta.typ\ } , allí podrás elegir un título, indicar los
autores, el equipo docente, entre otras configuraciones.

El campo \texttt{\ autores\ } solo puede ser \texttt{\ string\ } o un
\texttt{\ array\ } de strings.

La configuración \texttt{\ departamento\ } puede ser personalizada a
cualquier organización pasandole un diccionario de esta forma:

\begin{Shaded}
\begin{Highlighting}[]
\NormalTok{\#let departamento = (}
\NormalTok{  nombre: (}
\NormalTok{    "Universidad Técnica Federico Santa María",}
\NormalTok{  )}
\NormalTok{)}
\end{Highlighting}
\end{Shaded}

Las demás configuraciones pueden ser un \texttt{\ content\ }
arbitrario, o un \texttt{\ string\ } .

\subsubsection{Configuración
Avanzada}\label{configuraciuxe3uxb3n-avanzada}

Algunos aspectos más avanzados pueden ser configurados a través de la
show rule que inicializa el documento
\texttt{\ \#show:\ minerva.report.with(\ ...\ )\ } , los parámetros
opcionales que recibe la función \texttt{\ report\ } son los
siguientes:

\begin{longtable}[]{@{}lll@{}}
\toprule\noalign{}
nombre & tipo & descrición \\
\midrule\noalign{}
\endhead
\bottomrule\noalign{}
\endlastfoot
portada & (meta) =\textgreater{} content & Una función que recibe el
diccionario \texttt{\ meta.typ\ } y retorna una página. \\
header & (meta) =\textgreater{} content & Header a aplicarse a cada
página. \\
footer & (meta) =\textgreater{} content & Footer a aplicarse a cada
página. \\
showrules & bool & El template aplica ciertas show-rules para que sea
más fácil de utilizar. Si quires más personalización, es probable
que necesites desactivarlas y luego solo utilizar las que necesites. \\
\end{longtable}

\paragraph{Show Rules}\label{show-rules}

El template incluye show rules que pueden ser incluidas opcionalmente.
Todas estas show rules pueden ser activadas agregando:

\begin{Shaded}
\begin{Highlighting}[]
\NormalTok{\#show: minerva.\textless{}nombre{-}función\textgreater{}}
\end{Highlighting}
\end{Shaded}

Justo después de la línea
\texttt{\ \#show\ minerva.report.with(\ ...\ )\ } reemplazando
\texttt{\ \textless{}nombre-función\textgreater{}\ } por el nombre de
la show rule a aplicar.

\subparagraph{primer-heading-en-nueva-pag (activada por
defecto)}\label{primer-heading-en-nueva-pag-activada-por-defecto}

Esta show rule hace que el primer heading que tenga
\texttt{\ outlined:\ true\ } se muestre en una nueva página (con
\texttt{\ weak:\ true\ } ). Notar que al ser \texttt{\ weak:\ true\ } si
la página ya de por si estaba vacía, no se crea otra página adicional,
pero para que la página realmente se considere vacía no debe contener
absolutamente nada, incluso tener elementos invisibles va a causar que
se agregue una página extra.

\subparagraph{operadores-es (activada por
defecto)}\label{operadores-es-activada-por-defecto}

Cambia los operadores matemáticos que define Typst por defecto a sus
contrapartes en español, esto es, cambia \texttt{\ lim\ } por
\texttt{\ lím\ } , \texttt{\ inf\ } por \texttt{\ ínf\ } y así con
todos.

\subparagraph{formato-numeros-es}\label{formato-numeros-es}

Cambia los números dentro de las ecuaciones para que usen coma decimal
en vez de punto decimal, como es convención en el español. Esta show
rule no viene activa por defecto.

\href{/app?template=minerva-report-fcfm&version=0.2.1}{Create project in
app}

\subsubsection{How to use}\label{how-to-use}

Click the button above to create a new project using this template in
the Typst app.

You can also use the Typst CLI to start a new project on your computer
using this command:

\begin{verbatim}
typst init @preview/minerva-report-fcfm:0.2.1
\end{verbatim}

\includesvg[width=0.16667in,height=0.16667in]{/assets/icons/16-copy.svg}

\subsubsection{About}\label{about}

\begin{description}
\tightlist
\item[Author :]
\href{https://github.com/Dav1com}{David Ibáñez}
\item[License:]
MIT-0
\item[Current version:]
0.2.1
\item[Last updated:]
October 14, 2024
\item[First released:]
April 15, 2024
\item[Minimum Typst version:]
0.11.0
\item[Archive size:]
246 kB
\href{https://packages.typst.org/preview/minerva-report-fcfm-0.2.1.tar.gz}{\pandocbounded{\includesvg[keepaspectratio]{/assets/icons/16-download.svg}}}
\item[Repository:]
\href{https://github.com/Dav1com/minerva-report-fcfm}{GitHub}
\item[Categor y :]
\begin{itemize}
\tightlist
\item[]
\item
  \pandocbounded{\includesvg[keepaspectratio]{/assets/icons/16-speak.svg}}
  \href{https://typst.app/universe/search/?category=report}{Report}
\end{itemize}
\end{description}

\subsubsection{Where to report issues?}\label{where-to-report-issues}

This template is a project of David Ibáñez . Report issues on
\href{https://github.com/Dav1com/minerva-report-fcfm}{their repository}
. You can also try to ask for help with this template on the
\href{https://forum.typst.app}{Forum} .

Please report this template to the Typst team using the
\href{https://typst.app/contact}{contact form} if you believe it is a
safety hazard or infringes upon your rights.

\phantomsection\label{versions}
\subsubsection{Version history}\label{version-history}

\begin{longtable}[]{@{}ll@{}}
\toprule\noalign{}
Version & Release Date \\
\midrule\noalign{}
\endhead
\bottomrule\noalign{}
\endlastfoot
0.2.1 & October 14, 2024 \\
\href{https://typst.app/universe/package/minerva-report-fcfm/0.2.0/}{0.2.0}
& April 29, 2024 \\
\href{https://typst.app/universe/package/minerva-report-fcfm/0.1.0/}{0.1.0}
& April 15, 2024 \\
\end{longtable}

Typst GmbH did not create this template and cannot guarantee correct
functionality of this template or compatibility with any version of the
Typst compiler or app.
