\title{typst.app/universe/package/silky-slides-insa}

\phantomsection\label{banner}
\phantomsection\label{template-thumbnail}
\pandocbounded{\includegraphics[keepaspectratio]{https://packages.typst.org/preview/thumbnails/silky-slides-insa-0.1.1-small.webp}}

\section{silky-slides-insa}\label{silky-slides-insa}

{ 0.1.1 }

A template made for presentations of INSA, a French engineering school.

\href{/app?template=silky-slides-insa&version=0.1.1}{Create project in
app}

\phantomsection\label{readme}
\pandocbounded{\includegraphics[keepaspectratio]{https://github.com/typst/packages/raw/main/packages/preview/silky-slides-insa/0.1.1/thumbnail-full.png}}

Typst Template for presentation for the french engineering school INSA.

\subsection{Table of contents}\label{table-of-contents}

\begin{enumerate}
\tightlist
\item
  \href{https://github.com/typst/packages/raw/main/packages/preview/silky-slides-insa/0.1.1/\#examples}{Example}
\item
  \href{https://github.com/typst/packages/raw/main/packages/preview/silky-slides-insa/0.1.1/\#usage}{Usage}
\item
  \href{https://github.com/typst/packages/raw/main/packages/preview/silky-slides-insa/0.1.1/\#fonts}{Fonts
  information}
\item
  \href{https://github.com/typst/packages/raw/main/packages/preview/silky-slides-insa/0.1.1/\#notes}{Notes}
\item
  \href{https://github.com/typst/packages/raw/main/packages/preview/silky-slides-insa/0.1.1/\#license}{License}
\item
  \href{https://github.com/typst/packages/raw/main/packages/preview/silky-slides-insa/0.1.1/\#changelog}{Changelog}
\end{enumerate}

\subsection{Example}\label{example}

\begin{Shaded}
\begin{Highlighting}[]
\NormalTok{\#import "@preview/silky{-}slides{-}insa:0.1.1": *}
\NormalTok{\#show: insa{-}slides.with(}
\NormalTok{  title: "Titre du diaporama",}
\NormalTok{  title{-}visual: none,}
\NormalTok{  subtitle: "Sous{-}titre (noms et prénoms ?)",}
\NormalTok{  insa: "rennes"}
\NormalTok{)}

\NormalTok{= Titre de section}

\NormalTok{== Titre d\textquotesingle{}une slide}

\NormalTok{{-} Liste}
\NormalTok{  {-} dans}
\NormalTok{    {-} une liste}

\NormalTok{On peut aussi faire un \#text(fill: insa{-}colors.secondary)[texte] avec les \#text(fill: insa{-}colors.primary)[couleurs de l\textquotesingle{}INSA] !}

\NormalTok{== Une autre slide}

\NormalTok{Du texte}

\NormalTok{\#pause}

\NormalTok{Et un autre texte qui apparaît plus tard !}

\NormalTok{\#section{-}slide[Une autre section][Avec une petite description]}

\NormalTok{Coucou}
\end{Highlighting}
\end{Shaded}

\subsection{Usage}\label{usage}

\subsubsection{Slide show rule}\label{slide-show-rule}

You call it with \texttt{\ \#show:\ insa-slides.with(..parameters)\ } .

\begin{longtable}[]{@{}llll@{}}
\toprule\noalign{}
Parameter & Description & Type & Example \\
\midrule\noalign{}
\endhead
\bottomrule\noalign{}
\endlastfoot
\textbf{title} & Title of the presentation & content &
\texttt{\ {[}Titre\ de\ la\ prez{]}\ } \\
\textbf{title-visual} & Content shown at the right of the title slide &
content & none \\
\textbf{subtitle} & Subtitle of the presentation & content &
\texttt{\ {[}Sous-titre{]}\ } \\
\textbf{insa} & INSA name ( \texttt{\ rennes\ } , \texttt{\ hdf\ } …)
& str & \texttt{\ "rennes"\ } \\
\end{longtable}

If you assign a content to \texttt{\ title-visual\ } , the title slide
will automatically switch layout to the “visual� one from the
graphic charter. If you do not assign a visual content, the title slide
will only contain the title and subtitle and will choose the simple
layout.

\subsubsection{Section slide}\label{section-slide}

A section slide is automatically created when you put a level-1 header
in your markup. For example:

\begin{Shaded}
\begin{Highlighting}[]
\NormalTok{= Slide section}
\NormalTok{Blablabla}
\end{Highlighting}
\end{Shaded}

Will create a section slide with the title “Slide section� and will
be followed by a content slide containing “Blablabla�.

If you want to put a subtitle in your section slide, you must
explicitely use the \texttt{\ section-slide\ } function like so:

\begin{Shaded}
\begin{Highlighting}[]
\NormalTok{\#section{-}slide[Titre de section][Description de section]}
\end{Highlighting}
\end{Shaded}

\subsection{Fonts}\label{fonts}

The graphic charter recommends the fonts \textbf{League Spartan} for
headings and \textbf{Source Serif} for regular text. To have the best
look, you should install those fonts.

\begin{quote}
You can download the fonts from
\href{https://github.com/SkytAsul/INSA-Typst-Template/tree/main/fonts}{here}
.
\end{quote}

To behave correctly on computers lacking those specific fonts, this
template will automatically fallback to similar ones:

\begin{itemize}
\tightlist
\item
  \textbf{League Spartan} -\textgreater{} \textbf{Arial} (approved by
  INSA’s graphic charter, by default in Windows) -\textgreater{}
  \textbf{Liberation Sans} (by default in most Linux)
\item
  \textbf{Source Serif} -\textgreater{} \textbf{Source Serif 4}
  (downloadable for free) -\textgreater{} \textbf{Georgia} (approved by
  the graphic charter) -\textgreater{} \textbf{Linux Libertine} (default
  Typst font)
\end{itemize}

\subsubsection{Note on variable fonts}\label{note-on-variable-fonts}

If you want to install those fonts on your computer, Typst might not
recognize them if you install their \emph{Variable} versions. You should
install the static versions ( \textbf{League Spartan Bold} and most
versions of \textbf{Source Serif} ).

Keep an eye on \href{https://github.com/typst/typst/issues/185}{the
issue in Typst bug tracker} to see when variable fonts will be used!

\subsection{Notes}\label{notes}

This template is being developed by Youenn LE JEUNE from the INSA de
Rennes in \href{https://github.com/SkytAsul/INSA-Typst-Template}{this
repository} .

For now it includes assets from the graphic charters of those INSAs:

\begin{itemize}
\tightlist
\item
  Rennes ( \texttt{\ rennes\ } )
\item
  Hauts de France ( \texttt{\ hdf\ } )
\item
  Centre Val de Loire ( \texttt{\ cvl\ } ) Users from other INSAs can
  open a pull request on the repository with the assets for their INSA.
\end{itemize}

If you have any other feature request, open an issue on the repository.

\subsection{License}\label{license}

The typst template is licensed under the
\href{https://github.com/SkytAsul/INSA-Typst-Template/blob/main/LICENSE}{MIT
license} . This does \emph{not} apply to the image assets. Those image
files are property of Groupe INSA.

\subsection{Changelog}\label{changelog}

\subsubsection{0.1.1}\label{section}

\begin{itemize}
\tightlist
\item
  Added INSA CVL assets
\end{itemize}

\subsubsection{0.1.0}\label{section-1}

\begin{itemize}
\tightlist
\item
  Created the template
\end{itemize}

\href{/app?template=silky-slides-insa&version=0.1.1}{Create project in
app}

\subsubsection{How to use}\label{how-to-use}

Click the button above to create a new project using this template in
the Typst app.

You can also use the Typst CLI to start a new project on your computer
using this command:

\begin{verbatim}
typst init @preview/silky-slides-insa:0.1.1
\end{verbatim}

\includesvg[width=0.16667in,height=0.16667in]{/assets/icons/16-copy.svg}

\subsubsection{About}\label{about}

\begin{description}
\tightlist
\item[Author :]
SkytAsul
\item[License:]
MIT
\item[Current version:]
0.1.1
\item[Last updated:]
November 21, 2024
\item[First released:]
October 16, 2024
\item[Archive size:]
227 kB
\href{https://packages.typst.org/preview/silky-slides-insa-0.1.1.tar.gz}{\pandocbounded{\includesvg[keepaspectratio]{/assets/icons/16-download.svg}}}
\item[Repository:]
\href{https://github.com/SkytAsul/INSA-Typst-Template}{GitHub}
\item[Discipline s :]
\begin{itemize}
\tightlist
\item[]
\item
  \href{https://typst.app/universe/search/?discipline=engineering}{Engineering}
\item
  \href{https://typst.app/universe/search/?discipline=computer-science}{Computer
  Science}
\item
  \href{https://typst.app/universe/search/?discipline=mathematics}{Mathematics}
\item
  \href{https://typst.app/universe/search/?discipline=physics}{Physics}
\item
  \href{https://typst.app/universe/search/?discipline=education}{Education}
\end{itemize}
\item[Categor y :]
\begin{itemize}
\tightlist
\item[]
\item
  \pandocbounded{\includesvg[keepaspectratio]{/assets/icons/16-presentation.svg}}
  \href{https://typst.app/universe/search/?category=presentation}{Presentation}
\end{itemize}
\end{description}

\subsubsection{Where to report issues?}\label{where-to-report-issues}

This template is a project of SkytAsul . Report issues on
\href{https://github.com/SkytAsul/INSA-Typst-Template}{their repository}
. You can also try to ask for help with this template on the
\href{https://forum.typst.app}{Forum} .

Please report this template to the Typst team using the
\href{https://typst.app/contact}{contact form} if you believe it is a
safety hazard or infringes upon your rights.

\phantomsection\label{versions}
\subsubsection{Version history}\label{version-history}

\begin{longtable}[]{@{}ll@{}}
\toprule\noalign{}
Version & Release Date \\
\midrule\noalign{}
\endhead
\bottomrule\noalign{}
\endlastfoot
0.1.1 & November 21, 2024 \\
\href{https://typst.app/universe/package/silky-slides-insa/0.1.0/}{0.1.0}
& October 16, 2024 \\
\end{longtable}

Typst GmbH did not create this template and cannot guarantee correct
functionality of this template or compatibility with any version of the
Typst compiler or app.
