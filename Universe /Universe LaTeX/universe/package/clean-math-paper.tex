\title{typst.app/universe/package/clean-math-paper}

\phantomsection\label{banner}
\phantomsection\label{template-thumbnail}
\pandocbounded{\includegraphics[keepaspectratio]{https://packages.typst.org/preview/thumbnails/clean-math-paper-0.1.0-small.webp}}

\section{clean-math-paper}\label{clean-math-paper}

{ 0.1.0 }

A simple and good looking template for mathematical papers

\href{/app?template=clean-math-paper&version=0.1.0}{Create project in
app}

\phantomsection\label{readme}
\href{https://github.com/JoshuaLampert/clean-math-paper/actions/workflows/build.yml}{\pandocbounded{\includesvg[keepaspectratio]{https://github.com/JoshuaLampert/clean-math-paper/actions/workflows/build.yml/badge.svg}}}
\href{https://github.com/JoshuaLampert/clean-math-paper}{\pandocbounded{\includegraphics[keepaspectratio]{https://img.shields.io/badge/GitHub-repo-blue}}}
\href{https://opensource.org/licenses/MIT}{\pandocbounded{\includesvg[keepaspectratio]{https://img.shields.io/badge/License-MIT-success.svg}}}

\href{https://typst.app/home/}{Typst} paper template for mathematical
papers built for simple, efficient use and a clean look. Of course, it
can also be used for other subjects, but the following math-specific
features are already contained in the template:

\begin{itemize}
\tightlist
\item
  theorems, lemmas, corollaries, proofs etc. prepared using
  \href{https://typst.app/universe/package/great-theorems}{great-theorems}
\item
  equation settings
\end{itemize}

\subsection{Set-Up}\label{set-up}

The template is already filled with dummy data, to give users an
impression how it looks like. The paper is obtained by compiling
\texttt{\ main.typ\ } .

\begin{itemize}
\tightlist
\item
  after
  \href{https://github.com/typst/typst?tab=readme-ov-file\#installation}{installing
  Typst} you can conveniently use the following to create a new folder
  containing this project.
\end{itemize}

\begin{Shaded}
\begin{Highlighting}[]
\ExtensionTok{typst}\NormalTok{ init @preview/clean{-}math{-}paper:0.1.0}
\end{Highlighting}
\end{Shaded}

\begin{itemize}
\tightlist
\item
  edit the data in \texttt{\ main.typ\ } â†'
  \texttt{\ \#show\ template.with({[}your\ data{]})\ }
\end{itemize}

\subsubsection{Parameters of the
Template}\label{parameters-of-the-template}

\begin{itemize}
\tightlist
\item
  \texttt{\ title\ } : Title of the paper.
\item
  \texttt{\ authors\ } : List of names of the authors of the paper. Each
  entry of the list is a dictionary with the following keys:

  \begin{itemize}
  \tightlist
  \item
    \texttt{\ name\ } : Name of the author.
  \item
    \texttt{\ affiliation-id\ } : The ID of the affiliation in
    \texttt{\ affiliations\ } , see below.
  \item
    optionally \texttt{\ orcid\ } : The \href{https://orcid.org/}{ORCID}
    of the author. If provided, the author’s name will be linked to
    their ORCID profile.
  \end{itemize}
\item
  \texttt{\ affiliations\ } : List of affiliations of the authors. Each
  entry of the list is a dictionary with the following keys:

  \begin{itemize}
  \tightlist
  \item
    \texttt{\ id\ } : ID of the affiliation, which is used to link the
    authors to the affiliation, see above.
  \item
    \texttt{\ name\ } : Name of the affiliation.
  \end{itemize}
\item
  \texttt{\ date\ } : Date of the paper.
\item
  \texttt{\ heading-color\ } : Color of the headings including the
  title.
\item
  \texttt{\ link-color\ } : Color of the links.
\item
  \texttt{\ abstract\ } : Abstract of the paper.
\item
  \texttt{\ keywords\ } : List of keywords of the paper. If not
  provided, nothing will be shown.
\item
  \texttt{\ AMS\ } : List of AMS subject classifications of the paper.
  If not provided, nothing will be shown.
\end{itemize}

\subsection{Acknowledgements}\label{acknowledgements}

Some parts of this template are based on the
\href{https://github.com/mgoulao/arkheion}{arkheion} template.

\subsection{Feedback \& Improvements}\label{feedback-improvements}

If you encounter problems, please open issues. In case you found useful
extensions or improved anything We are also very happy to accept pull
requests.

\href{/app?template=clean-math-paper&version=0.1.0}{Create project in
app}

\subsubsection{How to use}\label{how-to-use}

Click the button above to create a new project using this template in
the Typst app.

You can also use the Typst CLI to start a new project on your computer
using this command:

\begin{verbatim}
typst init @preview/clean-math-paper:0.1.0
\end{verbatim}

\includesvg[width=0.16667in,height=0.16667in]{/assets/icons/16-copy.svg}

\subsubsection{About}\label{about}

\begin{description}
\tightlist
\item[Author :]
\href{https://github.com/JoshuaLampert}{Joshua Lampert}
\item[License:]
MIT
\item[Current version:]
0.1.0
\item[Last updated:]
November 21, 2024
\item[First released:]
November 21, 2024
\item[Minimum Typst version:]
0.12.0
\item[Archive size:]
5.95 kB
\href{https://packages.typst.org/preview/clean-math-paper-0.1.0.tar.gz}{\pandocbounded{\includesvg[keepaspectratio]{/assets/icons/16-download.svg}}}
\item[Repository:]
\href{https://github.com/JoshuaLampert/clean-math-paper}{GitHub}
\item[Discipline s :]
\begin{itemize}
\tightlist
\item[]
\item
  \href{https://typst.app/universe/search/?discipline=mathematics}{Mathematics}
\item
  \href{https://typst.app/universe/search/?discipline=engineering}{Engineering}
\item
  \href{https://typst.app/universe/search/?discipline=computer-science}{Computer
  Science}
\end{itemize}
\item[Categor y :]
\begin{itemize}
\tightlist
\item[]
\item
  \pandocbounded{\includesvg[keepaspectratio]{/assets/icons/16-atom.svg}}
  \href{https://typst.app/universe/search/?category=paper}{Paper}
\end{itemize}
\end{description}

\subsubsection{Where to report issues?}\label{where-to-report-issues}

This template is a project of Joshua Lampert . Report issues on
\href{https://github.com/JoshuaLampert/clean-math-paper}{their
repository} . You can also try to ask for help with this template on the
\href{https://forum.typst.app}{Forum} .

Please report this template to the Typst team using the
\href{https://typst.app/contact}{contact form} if you believe it is a
safety hazard or infringes upon your rights.

\phantomsection\label{versions}
\subsubsection{Version history}\label{version-history}

\begin{longtable}[]{@{}ll@{}}
\toprule\noalign{}
Version & Release Date \\
\midrule\noalign{}
\endhead
\bottomrule\noalign{}
\endlastfoot
0.1.0 & November 21, 2024 \\
\end{longtable}

Typst GmbH did not create this template and cannot guarantee correct
functionality of this template or compatibility with any version of the
Typst compiler or app.
