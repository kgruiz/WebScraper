\title{typst.app/universe/package/kouhu}

\phantomsection\label{banner}
\section{kouhu}\label{kouhu}

{ 0.1.1 }

Chinese lipsum text generator; 中æ--‡ä¹±æ•°å?‡æ--‡ï¼ˆLorem
Ipsum)ç''Ÿæˆ?器

\phantomsection\label{readme}
\texttt{\ kouhu\ } is a Chinese lipsum text generator for
\href{https://typst.app/}{Typst} . It provides a set of built-in text
samples containing both Simplified and Traditional Chinese characters.
You can choose from generated fake text, classic or modern Chinese
literature, or specify your own text.

\texttt{\ kouhu\ } is inspired by
\href{https://ctan.org/pkg/zhlipsum}{zhlipsum} LaTeX package and
\href{https://typst.app/universe/package/roremu}{roremu} Typst package.

All
\href{https://github.com/typst/packages/raw/main/packages/preview/kouhu/0.1.1/data/zhlipsum.json}{sample
text} is excerpted from \texttt{\ zhlipsum\ } LaTeX package (see
Appendix for details).

\subsection{Usage}\label{usage}

\begin{Shaded}
\begin{Highlighting}[]
\NormalTok{\#import "@preview/kouhu:0.1.0": kouhu}

\NormalTok{\#kouhu(indicies: range(1, 3)) // select the first 3 paragraphs from default text}

\NormalTok{\#kouhu(builtin{-}text: "zhufu", offset: 5, length: 100) // select 100 characters from the 5th paragraph of "zhufu" text}

\NormalTok{\#kouhu(custom{-}text: ("Foo", "Bar")) // provide your own text}
\end{Highlighting}
\end{Shaded}

See
\href{https://github.com/Harry-Chen/kouhu/blob/master/doc/manual.pdf}{manual}
for more details.

\subsection{\texorpdfstring{What does \texttt{\ kouhu\ }
mean?}{What does  kouhu  mean?}}\label{what-does-kouhu-mean}

GitHub Copilot says:

\begin{quote}
\texttt{\ kouhu\ } (�胡) is a Chinese term for reading aloud without
understanding the meaning. It is often used in the context of learning
Chinese language or reciting Chinese literature.
\end{quote}

\subsection{Changelog}\label{changelog}

\subsubsection{0.1.1}\label{section}

\begin{itemize}
\tightlist
\item
  Fix some wrong paths in \texttt{\ README.md\ } .
\item
  Fix genearation of \texttt{\ indicies\ } when not specified by user.
\item
  Add repetition of text until \texttt{\ length\ } is reached.
\end{itemize}

\subsubsection{0.1.0}\label{section-1}

\begin{itemize}
\tightlist
\item
  Initial release.
\end{itemize}

\subsection{Appendix}\label{appendix}

\subsubsection{\texorpdfstring{Generating
\texttt{\ zhlipsum.json\ }}{Generating  zhlipsum.json }}\label{generating-zhlipsum.json}

First download the \texttt{\ zhlipsum-text.dtx\ } from
\href{https://ctan.org/pkg/zhlipsum}{CTAN} or from local TeX Live (
\texttt{\ kpsewhich\ zhlipsum-text.dtx\ } ). Then run:

\begin{Shaded}
\begin{Highlighting}[]
\ExtensionTok{python3}\NormalTok{ utils/generate\_zhlipsum.py /path/to/zhlipsum{-}text.dtx src/zhlipsum.json}
\end{Highlighting}
\end{Shaded}

\subsubsection{How to add}\label{how-to-add}

Copy this into your project and use the import as \texttt{\ kouhu\ }

\begin{verbatim}
#import "@preview/kouhu:0.1.1"
\end{verbatim}

\includesvg[width=0.16667in,height=0.16667in]{/assets/icons/16-copy.svg}

Check the docs for
\href{https://typst.app/docs/reference/scripting/\#packages}{more
information on how to import packages} .

\subsubsection{About}\label{about}

\begin{description}
\tightlist
\item[Author :]
\href{mailto:harry-chen@outlook.com}{Shengqi Chen}
\item[License:]
MIT
\item[Current version:]
0.1.1
\item[Last updated:]
September 30, 2024
\item[First released:]
September 27, 2024
\item[Archive size:]
905 kB
\href{https://packages.typst.org/preview/kouhu-0.1.1.tar.gz}{\pandocbounded{\includesvg[keepaspectratio]{/assets/icons/16-download.svg}}}
\item[Repository:]
\href{https://github.com/Harry-Chen/kouhu}{GitHub}
\item[Categor y :]
\begin{itemize}
\tightlist
\item[]
\item
  \pandocbounded{\includesvg[keepaspectratio]{/assets/icons/16-hammer.svg}}
  \href{https://typst.app/universe/search/?category=utility}{Utility}
\end{itemize}
\end{description}

\subsubsection{Where to report issues?}\label{where-to-report-issues}

This package is a project of Shengqi Chen . Report issues on
\href{https://github.com/Harry-Chen/kouhu}{their repository} . You can
also try to ask for help with this package on the
\href{https://forum.typst.app}{Forum} .

Please report this package to the Typst team using the
\href{https://typst.app/contact}{contact form} if you believe it is a
safety hazard or infringes upon your rights.

\phantomsection\label{versions}
\subsubsection{Version history}\label{version-history}

\begin{longtable}[]{@{}ll@{}}
\toprule\noalign{}
Version & Release Date \\
\midrule\noalign{}
\endhead
\bottomrule\noalign{}
\endlastfoot
0.1.1 & September 30, 2024 \\
\href{https://typst.app/universe/package/kouhu/0.1.0/}{0.1.0} &
September 27, 2024 \\
\end{longtable}

Typst GmbH did not create this package and cannot guarantee correct
functionality of this package or compatibility with any version of the
Typst compiler or app.
