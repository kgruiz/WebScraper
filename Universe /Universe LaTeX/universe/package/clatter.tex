\title{typst.app/universe/package/clatter}

\phantomsection\label{banner}
\section{clatter}\label{clatter}

{ 0.1.0 }

Just the PDF417 generator from rxing.

\phantomsection\label{readme}
clatter is a simple Typst package for generating PDF417 barcodes,
utilizing the \href{https://github.com/rxing-core/rxing}{rxing} library.

\subsection{Features}\label{features}

\begin{itemize}
\tightlist
\item
  \textbf{Easy to Use} : The package provides a single, intuitive
  function to generate barcodes.
\item
  \textbf{Flexible Sizing} : Control the size of the barcode with
  optional width and height parameters.
\item
  \textbf{Customizable Orientation} : Barcodes can be rendered
  horizontally or vertically, with automatic adjustment based on size.
\end{itemize}

\subsection{Usage}\label{usage}

The primary function provided by this package is \texttt{\ pdf417\ } .

\subsubsection{Parameters}\label{parameters}

\begin{itemize}
\tightlist
\item
  \texttt{\ text\ } (required): The text to encode in the barcode.
\item
  \texttt{\ width\ } (optional): The desired width of the barcode.
\item
  \texttt{\ height\ } (optional): The desired height of the barcode.
\item
  \texttt{\ direction\ } (optional): Sets the orientation of the
  barcode, either \texttt{\ "horizontal"\ } or \texttt{\ "vertical"\ } .
  If not specified, the orientation is automatically determined based on
  the provided dimensions.
\end{itemize}

\subsubsection{Sizing Behavior}\label{sizing-behavior}

\begin{itemize}
\tightlist
\item
  By default, the barcode is rendered horizontally at a reasonable size.
\item
  If both \texttt{\ width\ } and \texttt{\ height\ } are provided, the
  barcode will fit within the specified dimensions (i.e.
  \texttt{\ fit:\ "contain"\ } ).
\item
  If the \texttt{\ height\ } is greater than the \texttt{\ width\ } ,
  the barcode will automatically switch to vertical orientation unless
  \texttt{\ direction\ } is manually set.
\end{itemize}

\subsubsection{Example Usage}\label{example-usage}

\begin{Shaded}
\begin{Highlighting}[]
\NormalTok{\#import "@preview/clatter:0.1.0": pdf417}

\NormalTok{// Generate a sized horizontal PDF417 barcode }
\NormalTok{// Note: The specified size may not be exact, as the barcode will fit within the box, maintaining its aspect ratio.}
\NormalTok{\#pdf417("sized{-}barcode", width: 50mm, height: 20mm)}

\NormalTok{// Generate a vertical barcode}
\NormalTok{\#pdf417("vertical{-}barcode", direction: "vertical")}

\NormalTok{// Generate a barcode and position it on the page}
\NormalTok{\#place(top + right, pdf417("absolutely{-}positioned{-}barcode", width: 50mm), dx: {-}5mm, dy: 5mm)}
\end{Highlighting}
\end{Shaded}

\begin{center}\rule{0.5\linewidth}{0.5pt}\end{center}

{Of course, such a lengthy README can’t be written without the help of
ChatGPT.}

\subsubsection{How to add}\label{how-to-add}

Copy this into your project and use the import as \texttt{\ clatter\ }

\begin{verbatim}
#import "@preview/clatter:0.1.0"
\end{verbatim}

\includesvg[width=0.16667in,height=0.16667in]{/assets/icons/16-copy.svg}

Check the docs for
\href{https://typst.app/docs/reference/scripting/\#packages}{more
information on how to import packages} .

\subsubsection{About}\label{about}

\begin{description}
\tightlist
\item[Author :]
\href{mailto:whygowe@gmail.com}{Hung-I Wang}
\item[License:]
MIT
\item[Current version:]
0.1.0
\item[Last updated:]
August 14, 2024
\item[First released:]
August 14, 2024
\item[Archive size:]
411 kB
\href{https://packages.typst.org/preview/clatter-0.1.0.tar.gz}{\pandocbounded{\includesvg[keepaspectratio]{/assets/icons/16-download.svg}}}
\item[Repository:]
\href{https://github.com/Gowee/typst-clatter}{GitHub}
\end{description}

\subsubsection{Where to report issues?}\label{where-to-report-issues}

This package is a project of Hung-I Wang . Report issues on
\href{https://github.com/Gowee/typst-clatter}{their repository} . You
can also try to ask for help with this package on the
\href{https://forum.typst.app}{Forum} .

Please report this package to the Typst team using the
\href{https://typst.app/contact}{contact form} if you believe it is a
safety hazard or infringes upon your rights.

\phantomsection\label{versions}
\subsubsection{Version history}\label{version-history}

\begin{longtable}[]{@{}ll@{}}
\toprule\noalign{}
Version & Release Date \\
\midrule\noalign{}
\endhead
\bottomrule\noalign{}
\endlastfoot
0.1.0 & August 14, 2024 \\
\end{longtable}

Typst GmbH did not create this package and cannot guarantee correct
functionality of this package or compatibility with any version of the
Typst compiler or app.
