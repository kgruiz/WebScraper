\title{typst.app/universe/package/zhconv}

\phantomsection\label{banner}
\section{zhconv}\label{zhconv}

{ 0.3.1 }

Convert Chinese text between Traditional/Simplified and regional
variants. 中æ--‡ç®€ç¹?å?Šåœ°å?€è©žè½‰æ?›

\phantomsection\label{readme}
zhconv-typst converts Chinese text between Traditional, Simplified and
regional variants in typst, utilizing
\href{https://github.com/Gowee/zhconv-rs}{zhconv-rs} .

\subsection{Usage}\label{usage}

To use the \texttt{\ zhconv\ } plugin in your Typst project, import it
as follows:

\begin{Shaded}
\begin{Highlighting}[]
\NormalTok{\#import "@preview/zhconv:0.3.1": zhconv}
\end{Highlighting}
\end{Shaded}

\subsubsection{Text Conversion}\label{text-conversion}

The primary function provided by this package is \texttt{\ zhconv\ } ,
which converts strings or nested contents to a target Chinese variant.

\begin{Shaded}
\begin{Highlighting}[]
\NormalTok{\#zhconv(content, "target{-}variant", wikitext: false)}
\end{Highlighting}
\end{Shaded}

\begin{itemize}
\tightlist
\item
  \texttt{\ content\ } : The text or content to be converted.
\item
  \texttt{\ target-variant\ } : The target Chinese variant (e.g.,
  \texttt{\ "zh-hant"\ } , \texttt{\ "zh-hans"\ } , \texttt{\ "zh-cn"\ }
  , \texttt{\ "zh-tw"\ } , \texttt{\ "zh-hk"\ } ).
\item
  \texttt{\ wikitext\ } : An optional boolean flag to specify if the
  text should be processed as wikitext (default is \texttt{\ false\ } ).
\end{itemize}

\paragraph{Example}\label{example}

\subparagraph{Convert a string}\label{convert-a-string}

\begin{Shaded}
\begin{Highlighting}[]
\NormalTok{\#let text = "互联网"}
\NormalTok{Original: \#text}
\NormalTok{{-} \#emph([zh{-}HK]): \#zhconv(text, "zh{-}hk")}
\NormalTok{{-} \#emph([zh{-}TW]): \#zhconv(text, "zh{-}tw")}
\end{Highlighting}
\end{Shaded}

\subparagraph{Convert nested contents}\label{convert-nested-contents}

\begin{Shaded}
\begin{Highlighting}[]
\NormalTok{\#zhconv([}
\NormalTok{柳外輕雷池上雨 \textbackslash{}}
\NormalTok{雨聲滴碎荷聲 \textbackslash{}}

\NormalTok{小樓西角斷虹明 \textbackslash{}}
\NormalTok{闌干倚處 \textbackslash{}}
\NormalTok{待得月華生 \textbackslash{}}
\NormalTok{], "zh{-}hans")}
\end{Highlighting}
\end{Shaded}

\subsubsection{How to add}\label{how-to-add}

Copy this into your project and use the import as \texttt{\ zhconv\ }

\begin{verbatim}
#import "@preview/zhconv:0.3.1"
\end{verbatim}

\includesvg[width=0.16667in,height=0.16667in]{/assets/icons/16-copy.svg}

Check the docs for
\href{https://typst.app/docs/reference/scripting/\#packages}{more
information on how to import packages} .

\subsubsection{About}\label{about}

\begin{description}
\tightlist
\item[Author :]
\href{mailto:whygowe@gmail.com}{Hung-I Wang}
\item[License:]
GPL-2.0
\item[Current version:]
0.3.1
\item[Last updated:]
August 14, 2024
\item[First released:]
August 14, 2024
\item[Archive size:]
804 kB
\href{https://packages.typst.org/preview/zhconv-0.3.1.tar.gz}{\pandocbounded{\includesvg[keepaspectratio]{/assets/icons/16-download.svg}}}
\item[Repository:]
\href{https://github.com/Gowee/zhconv-rs}{GitHub}
\end{description}

\subsubsection{Where to report issues?}\label{where-to-report-issues}

This package is a project of Hung-I Wang . Report issues on
\href{https://github.com/Gowee/zhconv-rs}{their repository} . You can
also try to ask for help with this package on the
\href{https://forum.typst.app}{Forum} .

Please report this package to the Typst team using the
\href{https://typst.app/contact}{contact form} if you believe it is a
safety hazard or infringes upon your rights.

\phantomsection\label{versions}
\subsubsection{Version history}\label{version-history}

\begin{longtable}[]{@{}ll@{}}
\toprule\noalign{}
Version & Release Date \\
\midrule\noalign{}
\endhead
\bottomrule\noalign{}
\endlastfoot
0.3.1 & August 14, 2024 \\
\end{longtable}

Typst GmbH did not create this package and cannot guarantee correct
functionality of this package or compatibility with any version of the
Typst compiler or app.
