\title{typst.app/universe/package/cartao}

\phantomsection\label{banner}
\section{cartao}\label{cartao}

{ 0.1.0 }

Dead simple flashcards with Typst.

\phantomsection\label{readme}
Dead simple flashcards with Typst.

\subsection{Example usage:}\label{example-usage}

\begin{Shaded}
\begin{Highlighting}[]
\NormalTok{\#import "@preview/cartao:0.1.0": card, letter8up, a48up}

\NormalTok{\#set page(}
\NormalTok{  paper: "a4",}
\NormalTok{  // paper: "us{-}letter",}
\NormalTok{  // paper: "presentation{-}16{-}9",}
\NormalTok{  margin: (x: 0cm, y: 0cm),}
\NormalTok{)}

\NormalTok{// build the cards}
\NormalTok{\#a48up}
\NormalTok{// \#letter8up}
\NormalTok{// \#present}

\NormalTok{// define your cards}
\NormalTok{\#card(}
\NormalTok{  [Header],}
\NormalTok{  [Footer],}
\NormalTok{  [Question?],}
\NormalTok{  [answer]}
\NormalTok{)}

\NormalTok{\#card(}
\NormalTok{  [portuguese],}
\NormalTok{  [Hint: Its the title of this package!],}
\NormalTok{  [card],}
\NormalTok{  [cartão]}
\NormalTok{)}

\NormalTok{\#card(}
\NormalTok{  [french],}
\NormalTok{  [Hint: close to the portuguese],}
\NormalTok{  [card],}
\NormalTok{  [carte]}
\NormalTok{)}
\end{Highlighting}
\end{Shaded}

\subsection{Documentation}\label{documentation}

\subsubsection{\texorpdfstring{\texttt{\ card\ }}{ card }}\label{card}

Defines a card by updating the below \texttt{\ counter\ } and
\texttt{\ state\ } (s), and dropping a label.

\begin{Shaded}
\begin{Highlighting}[]
\NormalTok{\#let card(header, footer, question, answer) = [}
\NormalTok{  \#cardnumber.step()}
\NormalTok{  \#cardheader.update(header)}
\NormalTok{  \#cardfooter.update(footer)}
\NormalTok{  \#cardquestion.update(question)}
\NormalTok{  \#cardanswer.update(answer)}
\NormalTok{  \textless{}card\textgreater{}}
\NormalTok{]}
\end{Highlighting}
\end{Shaded}

\paragraph{Arguments}\label{arguments}

\begin{itemize}
\tightlist
\item
  \texttt{\ header\ }
\item
  \texttt{\ footer\ }
\item
  \texttt{\ question\ }
\item
  \texttt{\ answer\ }
\end{itemize}

\subsubsection{card builders}\label{card-builders}

\textbf{How they work}

\begin{enumerate}
\tightlist
\item
  Find all locations of the \texttt{\ \textless{}card\textgreater{}\ }
  label
\item
  Get the values of the \texttt{\ cardnumber\ } counter, and
  \texttt{\ cardheader\ } , \texttt{\ cardfooter\ } ,
  \texttt{\ cardquestion\ } , \texttt{\ cardanswer\ } states at each
  \texttt{\ \textless{}card\textgreater{}\ } .
\item
  Populates an array of questions and an array of answers using these
  values

  \begin{itemize}
  \tightlist
  \item
    The \texttt{\ \#a48up\ } and \texttt{\ \#letter8up\ } functions
    describe the layout of each card for each item in these arrays, and
    also rearrange the answers so that the layout makes sense when
    printed double sided.
  \end{itemize}
\item
  Loop over the arrays and dump each item’s \texttt{\ content\ } onto
  the page.

  \begin{itemize}
  \tightlist
  \item
    in the case of \texttt{\ \#a48up\ } and \texttt{\ letter8up\ } ,
    each item is dumped into a 2-column table.
  \end{itemize}
\end{enumerate}

\texttt{\ cartao\ } comes builtin with the following card building
functions. Take a look at the source for how they work, and use them as
a guide to help you build your own flashcards with different
sizes/formats.

\subsubsection{\texorpdfstring{\texttt{\ a48up\ }}{ a48up }}\label{a48up}

Produces a 2x8 portrait card layout on a4 paper.

Designed to be printed double-sided on the perforated 8-up a4 card paper
you can find on
\href{https://www.amazon.ca/s?k=a4+perforated+card&crid=37RT2L4H5XSD0&sprefix=a4+perforated+ca\%2Caps\%2C648&ref=nb_sb_noss}{Amazon}

Usage

\begin{Shaded}
\begin{Highlighting}[]
\NormalTok{\#a48up}
\end{Highlighting}
\end{Shaded}

\subsubsection{\texorpdfstring{\texttt{\ letter8up\ }}{ letter8up }}\label{letter8up}

Produces a 2x8 portrait card layout on us-letter paper.

Usage

\begin{Shaded}
\begin{Highlighting}[]
\NormalTok{\#letter8up}
\end{Highlighting}
\end{Shaded}

\subsubsection{\texorpdfstring{\texttt{\ present\ }}{ present }}\label{present}

A 16:9 presentation of the flashcards with questions and answers on
different slides

Usage

\begin{Shaded}
\begin{Highlighting}[]
\NormalTok{\#present}
\end{Highlighting}
\end{Shaded}

\subsubsection{How to add}\label{how-to-add}

Copy this into your project and use the import as \texttt{\ cartao\ }

\begin{verbatim}
#import "@preview/cartao:0.1.0"
\end{verbatim}

\includesvg[width=0.16667in,height=0.16667in]{/assets/icons/16-copy.svg}

Check the docs for
\href{https://typst.app/docs/reference/scripting/\#packages}{more
information on how to import packages} .

\subsubsection{About}\label{about}

\begin{description}
\tightlist
\item[Author :]
Gavin Vales
\item[License:]
MIT
\item[Current version:]
0.1.0
\item[Last updated:]
November 21, 2023
\item[First released:]
November 21, 2023
\item[Archive size:]
3.40 kB
\href{https://packages.typst.org/preview/cartao-0.1.0.tar.gz}{\pandocbounded{\includesvg[keepaspectratio]{/assets/icons/16-download.svg}}}
\end{description}

\subsubsection{Where to report issues?}\label{where-to-report-issues}

This package is a project of Gavin Vales . You can also try to ask for
help with this package on the \href{https://forum.typst.app}{Forum} .

Please report this package to the Typst team using the
\href{https://typst.app/contact}{contact form} if you believe it is a
safety hazard or infringes upon your rights.

\phantomsection\label{versions}
\subsubsection{Version history}\label{version-history}

\begin{longtable}[]{@{}ll@{}}
\toprule\noalign{}
Version & Release Date \\
\midrule\noalign{}
\endhead
\bottomrule\noalign{}
\endlastfoot
0.1.0 & November 21, 2023 \\
\end{longtable}

Typst GmbH did not create this package and cannot guarantee correct
functionality of this package or compatibility with any version of the
Typst compiler or app.
