\title{typst.app/universe/package/parcio-thesis}

\phantomsection\label{banner}
\phantomsection\label{template-thumbnail}
\pandocbounded{\includegraphics[keepaspectratio]{https://packages.typst.org/preview/thumbnails/parcio-thesis-0.1.0-small.webp}}

\section{parcio-thesis}\label{parcio-thesis}

{ 0.1.0 }

A simple thesis template based on the ParCIO working group at OvGU
Magdeburg.

\href{/app?template=parcio-thesis&version=0.1.0}{Create project in app}

\phantomsection\label{readme}
\includegraphics[width=0.32\linewidth,height=\textheight,keepaspectratio]{https://github.com/typst/packages/raw/main/packages/preview/parcio-thesis/0.1.0/thumbnails/1.png}
\includegraphics[width=0.32\linewidth,height=\textheight,keepaspectratio]{https://github.com/typst/packages/raw/main/packages/preview/parcio-thesis/0.1.0/thumbnails/2.png}
\includegraphics[width=0.32\linewidth,height=\textheight,keepaspectratio]{https://github.com/typst/packages/raw/main/packages/preview/parcio-thesis/0.1.0/thumbnails/3.png}

A simple thesis template based on the ParCIO working group at
Otto-von-Guericke University Magdeburg.

\subsection{Getting Started}\label{getting-started}

To use this template, simply import it as shown below (more options
under \texttt{\ Usage\ } ):

\begin{Shaded}
\begin{Highlighting}[]
\NormalTok{\#import "@preview/parcio{-}thesis:0.1.0": *}

\NormalTok{\#show: parcio.with(}
\NormalTok{  title: "My great thesis",}
\NormalTok{  author: (}
\NormalTok{    name: "Author",}
\NormalTok{    mail: "author@ovgu.de"}
\NormalTok{  ),}
\NormalTok{  abstract: [My abstract begins here.],}
\NormalTok{)}
\end{Highlighting}
\end{Shaded}

\subsubsection{Local Installation}\label{local-installation}

Following these steps will make the template available locally under the
\texttt{\ @local\ } namespace. Requires
\href{https://github.com/casey/just}{“Just - A Command Runner�} .

\begin{Shaded}
\begin{Highlighting}[]
\FunctionTok{git}\NormalTok{ clone git@github.com:xkevio/parcio{-}typst.git }
\BuiltInTok{cd}\NormalTok{ parcio{-}typst/parcio{-}thesis/}
\ExtensionTok{just}\NormalTok{ install}
\end{Highlighting}
\end{Shaded}

\subsection{Usage}\label{usage}

See here for \textbf{all} possible arguments (and their default values)
and utility functions:

\begin{Shaded}
\begin{Highlighting}[]
\NormalTok{\#import "@preview/parcio{-}thesis:0.1.0": *}

\NormalTok{\#show: parcio.with(}
\NormalTok{  title: "Title",}
\NormalTok{  author: (name: "Author", mail: "author@ovgu.de"),}
\NormalTok{  abstract: [],}
\NormalTok{  thesis{-}type: "Bachelor/Master",}
\NormalTok{  reviewers: (),}
\NormalTok{  date: datetime.today(),}
\NormalTok{  lang: "en",}
\NormalTok{  header{-}logo: none,}
\NormalTok{  translations: none,}
\NormalTok{)}

\NormalTok{// Use these to *enable* or *change* page numbering for your frontmatter and mainmatter respectively.}
\NormalTok{// (By default, this template hides the page numbering!)}
\NormalTok{\#show: roman{-}numbering.with(reset: \textless{}true|false\textgreater{})}
\NormalTok{\#show: arabic{-}numbering.with(alternate: \textless{}true|false\textgreater{}, reset: \textless{}true|false\textgreater{})}
\end{Highlighting}
\end{Shaded}

\subsubsection{Utility Functions}\label{utility-functions}

These could be useful while writing your thesis!

\begin{Shaded}
\begin{Highlighting}[]
\NormalTok{// A TODO marker. (inline: false {-}\textgreater{} margin note, inline: true {-}\textgreater{} box).}
\NormalTok{\#let todo(inline: false, body)}

\NormalTok{// Like \textbackslash{}section* in LaTeX. (unnumbered level 2 heading, not in ToC).}
\NormalTok{\#let section = heading.with(level: 2, outlined: false, numbering: none)}

\NormalTok{// A neat inline{-}section in smallcaps and sans font.}
\NormalTok{\#let inline{-}section(title) = smallcaps[*\#text(font: "Libertinus Sans", title)*] }

\NormalTok{// Fully empty page, no page numbering.}
\NormalTok{\#let empty{-}page = page([], footer: [])}

\NormalTok{// Subfigures (see chapters/introduction for syntax).}
\NormalTok{\#let subfigure(..)}

\NormalTok{// A ParCIO{-}like table with a design taken from the LaTeX template.}
\NormalTok{\#let parcio{-}table(max{-}rows, ..args)}

\NormalTok{// Nicer handling of (multiple) appendices. Specify \textasciigrave{}reset: true\textasciigrave{} with your first appendix to reset the heading counter!}
\NormalTok{\#let appendix(reset: false, label: none, body)}
\end{Highlighting}
\end{Shaded}

\subsubsection{Translations}\label{translations}

If you wish, you can provide custom translations for things like
“Section�, “Contents�, etc. by providing a custom
\texttt{\ translations.toml\ } (this template already comes with
translations for English and German) with the following schema:

\begin{Shaded}
\begin{Highlighting}[]
\CommentTok{\# Top{-}level dict name should follow ISO 639 language codes!}
\DataTypeTok{default{-}lang} \OperatorTok{=} \StringTok{"en"}

\KeywordTok{[de]}
\DataTypeTok{contents} \OperatorTok{=} \StringTok{"Inhaltsverzeichnis"}
\DataTypeTok{chapter} \OperatorTok{=} \StringTok{"Kapitel"}
\DataTypeTok{section} \OperatorTok{=} \StringTok{"Sektion"}
\DataTypeTok{thesis} \OperatorTok{=} \OperatorTok{\{ }\DataTypeTok{value}\OperatorTok{ =} \StringTok{"Arbeit"}\OperatorTok{, }\DataTypeTok{compound}\OperatorTok{ =} \ConstantTok{true}\OperatorTok{ \}}

\KeywordTok{[de.title{-}page]}
\DataTypeTok{first{-}reviewer} \OperatorTok{=} \StringTok{"Erstgutachter"}
\DataTypeTok{second{-}reviewer} \OperatorTok{=} \StringTok{"Zweitgutachter"}
\DataTypeTok{supervisor} \OperatorTok{=} \StringTok{"Betreuer"}

\KeywordTok{[de.bibliography]}
\DataTypeTok{bibliography} \OperatorTok{=} \StringTok{"Quellenverzeichnis"}
\DataTypeTok{cited{-}on{-}page} \OperatorTok{=} \StringTok{"Zitiert auf Seite"}
\DataTypeTok{cited{-}on{-}pages} \OperatorTok{=} \StringTok{"Zitiert auf Seiten"}
\DataTypeTok{join} \OperatorTok{=} \StringTok{"und"}

\KeywordTok{[de.date]}
\DataTypeTok{date{-}format} \OperatorTok{=} \StringTok{"[day]. [month repr:long] [year]"}
\DataTypeTok{months} \OperatorTok{=} \OperatorTok{[}\StringTok{"Januar"}\OperatorTok{,} \StringTok{"Februar"}\OperatorTok{,} \StringTok{"März"}\OperatorTok{,} \StringTok{"April"}\OperatorTok{,} \StringTok{"Mai"}\OperatorTok{,} \StringTok{"Juni"}\OperatorTok{,} \StringTok{"Juli"}\OperatorTok{,} \StringTok{"August"}\OperatorTok{,} \StringTok{"September"}\OperatorTok{,} \StringTok{"Oktober"}\OperatorTok{,} \StringTok{"November"}\OperatorTok{,} \StringTok{"Dezember"}\OperatorTok{]}
\end{Highlighting}
\end{Shaded}

\subsection{Fonts and OvGU Corporate
Design}\label{fonts-and-ovgu-corporate-design}

This template requires these three fonts to be installed on your
system{[}\^{}1{]}:

\begin{itemize}
\tightlist
\item
  Libertinus Serif ( \url{https://github.com/alerque/libertinus} )
\item
  Libertinus Sans ( \url{https://github.com/alerque/libertinus} )
\item
  Inconsolata ( \url{https://github.com/googlefonts/Inconsolata} )
\end{itemize}

We bundle the default “Faculty of Computer Science� head banner and
use it as the \texttt{\ header-logo\ } . You can find yours at:
\href{https://www.cd.ovgu.de/Fakult\%C3\%A4ten.html}{https://www.cd.ovgu.de/Fakultäten.html}
.

{[}\^{}1{]}: Typst should already provide the Libertinus font family by
default as it is their standard font.

\href{/app?template=parcio-thesis&version=0.1.0}{Create project in app}

\subsubsection{How to use}\label{how-to-use}

Click the button above to create a new project using this template in
the Typst app.

You can also use the Typst CLI to start a new project on your computer
using this command:

\begin{verbatim}
typst init @preview/parcio-thesis:0.1.0
\end{verbatim}

\includesvg[width=0.16667in,height=0.16667in]{/assets/icons/16-copy.svg}

\subsubsection{About}\label{about}

\begin{description}
\tightlist
\item[Author :]
\href{https://github.com/xkevio}{Kevin Kulot}
\item[License:]
0BSD
\item[Current version:]
0.1.0
\item[Last updated:]
November 19, 2024
\item[First released:]
November 19, 2024
\item[Minimum Typst version:]
0.12.0
\item[Archive size:]
20.3 kB
\href{https://packages.typst.org/preview/parcio-thesis-0.1.0.tar.gz}{\pandocbounded{\includesvg[keepaspectratio]{/assets/icons/16-download.svg}}}
\item[Repository:]
\href{https://github.com/xkevio/parcio-typst/}{GitHub}
\item[Categor ies :]
\begin{itemize}
\tightlist
\item[]
\item
  \pandocbounded{\includesvg[keepaspectratio]{/assets/icons/16-mortarboard.svg}}
  \href{https://typst.app/universe/search/?category=thesis}{Thesis}
\item
  \pandocbounded{\includesvg[keepaspectratio]{/assets/icons/16-speak.svg}}
  \href{https://typst.app/universe/search/?category=report}{Report}
\end{itemize}
\end{description}

\subsubsection{Where to report issues?}\label{where-to-report-issues}

This template is a project of Kevin Kulot . Report issues on
\href{https://github.com/xkevio/parcio-typst/}{their repository} . You
can also try to ask for help with this template on the
\href{https://forum.typst.app}{Forum} .

Please report this template to the Typst team using the
\href{https://typst.app/contact}{contact form} if you believe it is a
safety hazard or infringes upon your rights.

\phantomsection\label{versions}
\subsubsection{Version history}\label{version-history}

\begin{longtable}[]{@{}ll@{}}
\toprule\noalign{}
Version & Release Date \\
\midrule\noalign{}
\endhead
\bottomrule\noalign{}
\endlastfoot
0.1.0 & November 19, 2024 \\
\end{longtable}

Typst GmbH did not create this template and cannot guarantee correct
functionality of this template or compatibility with any version of the
Typst compiler or app.
