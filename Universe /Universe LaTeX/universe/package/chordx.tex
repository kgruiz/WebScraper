\title{typst.app/universe/package/chordx}

\phantomsection\label{banner}
\section{chordx}\label{chordx}

{ 0.5.0 }

A package to write song lyrics with chord diagrams in Typst.

\phantomsection\label{readme}
A package to write song lyrics with chord diagrams in Typst.

\textbf{Table of Contents}

\begin{itemize}
\tightlist
\item
  \href{https://github.com/typst/packages/raw/main/packages/preview/chordx/0.5.0/\#introduction}{Introduction}
\item
  \href{https://github.com/typst/packages/raw/main/packages/preview/chordx/0.5.0/\#usage}{Usage}

  \begin{itemize}
  \tightlist
  \item
    \href{https://github.com/typst/packages/raw/main/packages/preview/chordx/0.5.0/\#typst-packages}{Typst
    Packages}
  \item
    \href{https://github.com/typst/packages/raw/main/packages/preview/chordx/0.5.0/\#local-packages}{Local
    Packages}
  \end{itemize}
\item
  \href{https://github.com/typst/packages/raw/main/packages/preview/chordx/0.5.0/\#documentation}{Documentation}
\item
  \href{https://github.com/typst/packages/raw/main/packages/preview/chordx/0.5.0/\#examples}{Examples}

  \begin{itemize}
  \tightlist
  \item
    \href{https://github.com/typst/packages/raw/main/packages/preview/chordx/0.5.0/\#chart-chords}{Chart
    Chords}
  \item
    \href{https://github.com/typst/packages/raw/main/packages/preview/chordx/0.5.0/\#piano-chords}{Piano
    Chords}
  \item
    \href{https://github.com/typst/packages/raw/main/packages/preview/chordx/0.5.0/\#single-chords}{Single
    Chords}
  \end{itemize}
\item
  \href{https://github.com/typst/packages/raw/main/packages/preview/chordx/0.5.0/\#changelog}{Changelog}
\item
  \href{https://github.com/typst/packages/raw/main/packages/preview/chordx/0.5.0/\#license}{License}
\end{itemize}

\subsection{Introduction}\label{introduction}

With \texttt{\ chordx\ } you can easily generate song lyrics with chords
for writing songbooks.

\texttt{\ chordx\ } generates chord charts for stringed instruments
(e.g. guitar, ukulele, etc.), piano chords (with diferent piano layouts)
and single chords that are chords without charts used to write the
chords over a word to write songbooks.

\subsection{Usage}\label{usage}

\texttt{\ chordx\ } exports 3 functions to generate diferents types fo
charts:

\begin{itemize}
\tightlist
\item
  \texttt{\ chart-chord\ } : used to generate chart chords for stringed
  instruments.
\item
  \texttt{\ piano-chord\ } : used to generate piano chords.
\item
  \texttt{\ single-chord\ } : used to show the chord name over a word.
\end{itemize}

\subsubsection{Typst Packages}\label{typst-packages}

Typst added an experimental package repository and you can import
\texttt{\ chordx\ } as follows:

\begin{Shaded}
\begin{Highlighting}[]
\NormalTok{\#import "@preview/chordx:0.5.0": *}
\end{Highlighting}
\end{Shaded}

\subsubsection{Local Packages}\label{local-packages}

If the package hasn’t been released yet, or if you just want to use it
from this repository, you can use \emph{\emph{local-packages}} .

You can read the documentation about typst
\href{https://github.com/typst/packages\#local-packages}{local-packages}
and learn about the path folders used in differents operating systems
(Linux / MacOS / Windows).

In Linux you can do:

\begin{Shaded}
\begin{Highlighting}[]
\FunctionTok{git}\NormalTok{ clone https://github.com/ljgago/typst{-}chords \textasciitilde{}/.local/share/typst/packages/local/chordx/0.5.0}
\end{Highlighting}
\end{Shaded}

And import the package in your file:

\begin{Shaded}
\begin{Highlighting}[]
\NormalTok{\#import "@local/chordx:0.5.0": *}
\end{Highlighting}
\end{Shaded}

\subsection{Documentation}\label{documentation}

Here
\href{https://github.com/ljgago/typst-chords/blob/v0.5.0/docs/chordx-docs.pdf}{chordx-docs}
you have the reference documentation that describes the functions and
parameters used in this package. ( \emph{Generated with
\href{https://github.com/Mc-Zen/tidy}{tidy}} )

\subsection{Examples:}\label{examples}

\subsubsection{Chart Chords}\label{chart-chords}

\begin{Shaded}
\begin{Highlighting}[]
\NormalTok{\#import "@preview/chordx:0.5.0": chart{-}chord}

\NormalTok{\#let chart{-}chord{-}sharp = chart{-}chord.with(size: 18pt)}
\NormalTok{\#let chart{-}chord{-}round = chart{-}chord.with(size: 1.5em, design: "round")}

\NormalTok{// Design "sharp"}
\NormalTok{\#chart{-}chord{-}sharp(tabs: "x32o1o", fingers: "n32n1n")[C]}
\NormalTok{\#chart{-}chord{-}sharp(tabs: "ooo3", fingers: "ooo3")[C]}

\NormalTok{// Desigh "round" with position "bottom"}
\NormalTok{\#chart{-}chord{-}round(tabs: "xn332n", fingers: "o13421", fret: 3, capos: "115", position: "bottom")[Cm]}
\NormalTok{\#chart{-}chord{-}round(tabs: "onnn", fingers: "n111", capos: "313", position: "bottom")[Cm]}

\NormalTok{// Design "round" with background color in chord name}
\NormalTok{\#chart{-}chord{-}round(tabs: "xn332n", fingers: "o13421", fret: 3, capos: "115", background: silver)[Cm]}
\NormalTok{\#chart{-}chord{-}round(tabs: "onnn", fingers: "n111", capos: "313", background: silver)[Cm]}
\end{Highlighting}
\end{Shaded}

\subsubsection{\texorpdfstring{\href{https://github.com/ljgago/typst-chords/blob/v0.5.0/examples/chart-chords.typ}{\protect\pandocbounded{\includesvg[keepaspectratio]{https://raw.githubusercontent.com/ljgago/typst-chords/v0.5.0/examples/chart-chords.svg}}}}{Chart Chord}}\label{chart-chord}

\subsubsection{Piano Chords}\label{piano-chords}

\begin{Shaded}
\begin{Highlighting}[]
\NormalTok{\#import "@preview/chordx:0.5.0": piano{-}chord}

\NormalTok{\#let piano{-}chord{-}sharp = piano{-}chord.with(layout: "F", size: 18pt)}
\NormalTok{\#let piano{-}chord{-}round = piano{-}chord.with(layout: "F", size: 1.5em, design: "round")}

\NormalTok{\#piano{-}chord{-}sharp(keys: "B1, D2\#, F2\#", fill{-}key: blue)[B]}
\NormalTok{\#piano{-}chord{-}round(keys: "B1, D2\#, F2\#", fill{-}key: yellow, position: "bottom")[B]}
\NormalTok{\#piano{-}chord{-}round(keys: "B1, D2\#, F2\#", fill{-}key: red)[B]}
\end{Highlighting}
\end{Shaded}

\subsubsection{\texorpdfstring{\href{https://github.com/ljgago/typst-chords/blob/v0.5.0/examples/piano-chords.typ}{\protect\pandocbounded{\includesvg[keepaspectratio]{https://raw.githubusercontent.com/ljgago/typst-chords/v0.5.0/examples/piano-chords.svg}}}}{Piano Chord}}\label{piano-chord}

\subsubsection{Single Chords}\label{single-chords}

\begin{Shaded}
\begin{Highlighting}[]
\NormalTok{\#import "@preview/chordx:0.5.0": single{-}chord}

\NormalTok{\#let chord = single{-}chord.with(}
\NormalTok{  font: "PT Sans",}
\NormalTok{  size: 12pt,}
\NormalTok{  weight: "semibold",}
\NormalTok{  background: silver}
\NormalTok{)}

\NormalTok{\#chord[Jingle][G][2] bells, jingle bells, jingle \#chord[all][C][2] the \#chord[way!][G][2] \textbackslash{}}
\NormalTok{\#chord[Oh][C][] what fun it \#chord[is][G][] to ride \textbackslash{}}
\NormalTok{In a \#chord[one{-}horse][A7][2] open \#chord[sleigh,][D7][3] hey!}
\end{Highlighting}
\end{Shaded}

\subsection{\texorpdfstring{\href{https://github.com/ljgago/typst-chords/blob/v0.5.0/examples/single-chords.typ}{\protect\pandocbounded{\includesvg[keepaspectratio]{https://raw.githubusercontent.com/ljgago/typst-chords/v0.5.0/examples/single-chords.svg}}}}{Single Chord}}\label{single-chord}

\subsection{Changelog}\label{changelog}

You can read the latest changes in
\href{https://github.com/typst/packages/raw/main/packages/preview/chordx/0.5.0/CHANGELOG.md}{CHANGELOG.md}

\subsection{License}\label{license}

\href{https://github.com/typst/packages/raw/main/packages/preview/chordx/0.5.0/LICENSE}{MIT
License}

\subsubsection{How to add}\label{how-to-add}

Copy this into your project and use the import as \texttt{\ chordx\ }

\begin{verbatim}
#import "@preview/chordx:0.5.0"
\end{verbatim}

\includesvg[width=0.16667in,height=0.16667in]{/assets/icons/16-copy.svg}

Check the docs for
\href{https://typst.app/docs/reference/scripting/\#packages}{more
information on how to import packages} .

\subsubsection{About}\label{about}

\begin{description}
\tightlist
\item[Author :]
\href{https://github.com/ljgago}{Leonardo Gago}
\item[License:]
MIT
\item[Current version:]
0.5.0
\item[Last updated:]
November 4, 2024
\item[First released:]
July 17, 2023
\item[Minimum Typst version:]
0.12.0
\item[Archive size:]
10.3 kB
\href{https://packages.typst.org/preview/chordx-0.5.0.tar.gz}{\pandocbounded{\includesvg[keepaspectratio]{/assets/icons/16-download.svg}}}
\item[Repository:]
\href{https://github.com/ljgago/typst-chords}{GitHub}
\end{description}

\subsubsection{Where to report issues?}\label{where-to-report-issues}

This package is a project of Leonardo Gago . Report issues on
\href{https://github.com/ljgago/typst-chords}{their repository} . You
can also try to ask for help with this package on the
\href{https://forum.typst.app}{Forum} .

Please report this package to the Typst team using the
\href{https://typst.app/contact}{contact form} if you believe it is a
safety hazard or infringes upon your rights.

\phantomsection\label{versions}
\subsubsection{Version history}\label{version-history}

\begin{longtable}[]{@{}ll@{}}
\toprule\noalign{}
Version & Release Date \\
\midrule\noalign{}
\endhead
\bottomrule\noalign{}
\endlastfoot
0.5.0 & November 4, 2024 \\
\href{https://typst.app/universe/package/chordx/0.4.0/}{0.4.0} & July
10, 2024 \\
\href{https://typst.app/universe/package/chordx/0.3.0/}{0.3.0} & March
3, 2024 \\
\href{https://typst.app/universe/package/chordx/0.2.0/}{0.2.0} &
September 3, 2023 \\
\href{https://typst.app/universe/package/chordx/0.1.0/}{0.1.0} & July
17, 2023 \\
\end{longtable}

Typst GmbH did not create this package and cannot guarantee correct
functionality of this package or compatibility with any version of the
Typst compiler or app.
