\title{typst.app/universe/package/minitoc}

\phantomsection\label{banner}
\section{minitoc}\label{minitoc}

{ 0.1.0 }

An outline function just for one section and nothing else

\phantomsection\label{readme}
This package provides the \texttt{\ minitoc\ } command that does the
same thing as the \texttt{\ outline\ } command but only for headings
under the heading above it.

This is inspired by minitoc package for LaTeX.

\subsection{Example}\label{example}

\begin{Shaded}
\begin{Highlighting}[]
\NormalTok{\#import "@preview/minitoc:0.1.0": *}
\NormalTok{\#set heading(numbering: "1.1")}

\NormalTok{\#outline()}

\NormalTok{= Heading 1}

\NormalTok{\#minitoc()}

\NormalTok{== Heading 1.1}

\NormalTok{\#lorem(20)}

\NormalTok{=== Heading 1.1.1}

\NormalTok{\#lorem(30)}

\NormalTok{== Heading 1.2}

\NormalTok{\#lorem(10)}

\NormalTok{= Heading 2}
\end{Highlighting}
\end{Shaded}

This produces

\pandocbounded{\includegraphics[keepaspectratio]{https://gitlab.com/human_person/typst-local-outline/-/raw/main/example/example.png}}

\subsection{Usage}\label{usage}

The \texttt{\ minitoc\ } function has the following signature:

\begin{Shaded}
\begin{Highlighting}[]
\NormalTok{\#let minitoc(}
\NormalTok{  title: none, target: heading.where(outlined: true),}
\NormalTok{    depth: none, indent: none, fill: repeat([.])}
\NormalTok{) \{ /* .. */ \}}
\end{Highlighting}
\end{Shaded}

This is designed to be as close to the
\href{https://typst.app/docs/reference/meta/outline/}{\texttt{\ outline\ }}
funtions as possible. The arguments are:

\begin{itemize}
\tightlist
\item
  \textbf{title} : The title for the local outline. This is the same as
  for
  \href{https://typst.app/docs/reference/meta/outline/\#parameters-title}{\texttt{\ outline.title\ }}
  .
\item
  \textbf{target} : What should be included. This is the same as for
  \href{https://typst.app/docs/reference/meta/outline/\#parameters-target}{\texttt{\ outline.target\ }}
\item
  \textbf{depth} : The maximum depth different to include. For example,
  if depth was 1 in the example, “Heading 1.1.1� would not be
  included
\item
  \textbf{indent} : How the entries should be indented. Takes the same
  types as for
  \href{https://typst.app/docs/reference/meta/outline/\#parameters-indent}{\texttt{\ outline.indent\ }}
  and is passed directly to it
\item
  \textbf{fill} : Content to put between the numbering and title, and
  the page number. Same types as for
  \href{https://typst.app/docs/reference/meta/outline/\#parameters-fill}{\texttt{\ outline.fill\ }}
\end{itemize}

\subsection{Unintended consequences}\label{unintended-consequences}

Because \texttt{\ minitoc\ } uses \texttt{\ outline\ } , if you apply
numbering to the title of outline with
\texttt{\ \#show\ outline:\ set\ heading(numbering:\ "1.")\ } or
similar, any title in \texttt{\ minitoc\ } will be numbered and be a
level 1 heading. This cannot be changed with
\texttt{\ \#show\ outline:\ set\ heading(level:\ 3)\ } or similar
unfortunately.

\subsubsection{How to add}\label{how-to-add}

Copy this into your project and use the import as \texttt{\ minitoc\ }

\begin{verbatim}
#import "@preview/minitoc:0.1.0"
\end{verbatim}

\includesvg[width=0.16667in,height=0.16667in]{/assets/icons/16-copy.svg}

Check the docs for
\href{https://typst.app/docs/reference/scripting/\#packages}{more
information on how to import packages} .

\subsubsection{About}\label{about}

\begin{description}
\tightlist
\item[Author :]
\href{https://github.com/RosiePuddles}{nxe}
\item[License:]
GPL-3.0-only
\item[Current version:]
0.1.0
\item[Last updated:]
January 7, 2024
\item[First released:]
January 7, 2024
\item[Archive size:]
13.6 kB
\href{https://packages.typst.org/preview/minitoc-0.1.0.tar.gz}{\pandocbounded{\includesvg[keepaspectratio]{/assets/icons/16-download.svg}}}
\item[Repository:]
\href{https://gitlab.com/human_person/typst-local-outline}{GitLab}
\end{description}

\subsubsection{Where to report issues?}\label{where-to-report-issues}

This package is a project of nxe . Report issues on
\href{https://gitlab.com/human_person/typst-local-outline}{their
repository} . You can also try to ask for help with this package on the
\href{https://forum.typst.app}{Forum} .

Please report this package to the Typst team using the
\href{https://typst.app/contact}{contact form} if you believe it is a
safety hazard or infringes upon your rights.

\phantomsection\label{versions}
\subsubsection{Version history}\label{version-history}

\begin{longtable}[]{@{}ll@{}}
\toprule\noalign{}
Version & Release Date \\
\midrule\noalign{}
\endhead
\bottomrule\noalign{}
\endlastfoot
0.1.0 & January 7, 2024 \\
\end{longtable}

Typst GmbH did not create this package and cannot guarantee correct
functionality of this package or compatibility with any version of the
Typst compiler or app.
