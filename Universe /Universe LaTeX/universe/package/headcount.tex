\title{typst.app/universe/package/headcount}

\phantomsection\label{banner}
\section{headcount}\label{headcount}

{ 0.1.0 }

Make counters inherit from the heading counter.

\phantomsection\label{readme}
This package allows you to make \textbf{counters depend on the current
chapter/section number} .

This works for \textbf{figures, theorems, and any other counters} .

The advantage compared to
\href{https://typst.app/universe/package/rich-counters/}{rich-counters}
is that you stick with native \texttt{\ counter\ } s and you can
influence e.g. the \texttt{\ figure\ } counter directly without writing
a new \texttt{\ show\ } rule with a custom counter or so.

\subsection{Showcase}\label{showcase}

In the following example, we demonstrate how you can inherit 1 level of
the heading counter for figures and 2 levels for theorems.

\begin{Shaded}
\begin{Highlighting}[]
\NormalTok{\#import "@preview/headcount:0.1.0": *}
\NormalTok{\#import "@preview/great{-}theorems:0.1.0": *}

\NormalTok{\#show: great{-}theorems{-}init}

\NormalTok{\#set heading(numbering: "1.1")}

\NormalTok{// contruct theorem environment with counter that inherits 2 levels from heading}
\NormalTok{\#let thmcounter = counter("hello")}
\NormalTok{\#let theorem = mathblock(}
\NormalTok{  blocktitle: [Theorem],}
\NormalTok{  counter: thmcounter,}
\NormalTok{  numbering: dependent{-}numbering("1.1", levels: 2)}
\NormalTok{)}
\NormalTok{\#show heading: reset{-}counter(thmcounter, levels: 2)}

\NormalTok{// set figure counter so that it inherits 1 level from heading}
\NormalTok{\#set figure(numbering: dependent{-}numbering("1.1"))}
\NormalTok{\#show heading: reset{-}counter(counter(figure.where(kind: image)))}

\NormalTok{= First heading}

\NormalTok{The theorems inherit 2 levels from the headings and the figures inherit 1 level from the headings.}

\NormalTok{\#theorem[Some theorem.]}
\NormalTok{\#theorem[Some theorem.]}
\NormalTok{\#figure([SOME FIGURE], caption: [some figure])}
\NormalTok{\#figure([SOME FIGURE], caption: [some figure])}

\NormalTok{== Subheading}

\NormalTok{\#theorem[Some theorem.]}
\NormalTok{\#figure([SOME FIGURE], caption: [some figure])}
\NormalTok{\#figure([SOME FIGURE], caption: [some figure])}

\NormalTok{= Second heading}

\NormalTok{\#theorem[Some theorem.]}
\NormalTok{\#figure([SOME FIGURE], caption: [some figure])}
\NormalTok{\#theorem[Some theorem.]}
\end{Highlighting}
\end{Shaded}

\pandocbounded{\includegraphics[keepaspectratio]{https://github.com/typst/packages/raw/main/packages/preview/headcount/0.1.0/example.png}}

\subsection{Usage}\label{usage}

To make another \texttt{\ counter\ } inherit from the heading counter,
you have to do \textbf{two} things.

\begin{enumerate}
\item
  For the numbering of your counter, you have to use
  \texttt{\ dependent-numbering(...)\ } .

  \begin{itemize}
  \item
    \texttt{\ dependent-numbering(style,\ level:\ 1)\ } (needs
    \texttt{\ context\ } )

    Is a replacement for the \texttt{\ numbering\ } function, with the
    difference that it precedes any counter value with
    \texttt{\ level\ } many values of the heading counter.
  \end{itemize}

\begin{Shaded}
\begin{Highlighting}[]
\NormalTok{\#import "@preview/headcount:0.1.0": *}

\NormalTok{\#set heading(numbering: "1.1")}

\NormalTok{\#let mycounter = counter("hello")}

\NormalTok{= First heading}

\NormalTok{\#context mycounter.step()}
\NormalTok{\#context mycounter.display(dependent{-}numbering("1.1"))}

\NormalTok{= Second heading}

\NormalTok{\#context mycounter.step()}
\NormalTok{\#context mycounter.display(dependent{-}numbering("1.1"))}

\NormalTok{\#context mycounter.step()}
\NormalTok{\#context mycounter.display(dependent{-}numbering("1.1"))}
\end{Highlighting}
\end{Shaded}

  This displays the desired amount of levels of the heading counter in
  front of the actual counter. However, as you can see in the code
  above, our actual counter does not yet reset in each section.
\item
  For resetting the counter at the appropriate places, you need to equip
  \texttt{\ heading\ } with the \texttt{\ show\ } rule that
  \texttt{\ reset-counter(...)\ } returns.

  \begin{itemize}
  \item
    \texttt{\ reset-counter(counter,\ level:\ 1)\ } (needs
    \texttt{\ context\ } )

    Returns a function that should be used as a \texttt{\ show\ } rule
    for \texttt{\ heading\ } . It will reset \texttt{\ counter\ } if the
    level of the heading is less than or equal to \texttt{\ level\ } .
  \end{itemize}

  \textbf{Important:} This \texttt{\ show\ } rule should be placed as
  the \emph{last} \texttt{\ show\ } rule for \texttt{\ heading\ } , or
  at least after \texttt{\ show\ } rules for \texttt{\ heading\ } that
  employ a custom design, see
  \href{https://forum.typst.app/t/i-figured-broken-with-custom-template/1730/10?u=jbirnick}{here}
  for an explanation.

\begin{Shaded}
\begin{Highlighting}[]
\NormalTok{\#import "@preview/headcount:0.1.0": *}

\NormalTok{\#set heading(numbering: "1.1")}
\NormalTok{\#show heading: reset{-}counter(mycounter, levels: 1)}

\NormalTok{\#let mycounter = counter("hello")}

\NormalTok{= First heading}

\NormalTok{\#context mycounter.step()}
\NormalTok{\#context mycounter.display(dependent{-}numbering("1.1"))}

\NormalTok{= Second heading}

\NormalTok{\#context mycounter.step()}
\NormalTok{\#context mycounter.display(dependent{-}numbering("1.1"))}

\NormalTok{\#context mycounter.step()}
\NormalTok{\#context mycounter.display(dependent{-}numbering("1.1"))}
\end{Highlighting}
\end{Shaded}
\end{enumerate}

\textbf{Note:} The \texttt{\ level\ } that you pass to
\texttt{\ dependent-numbering(...)\ } and the \texttt{\ level\ } that
you pass to \texttt{\ reset-counter(...)\ } must be the \emph{same} .

\subsection{Limitations}\label{limitations}

Due to current Typst limitations, there is no way to detect manual
updates or steps of the heading counter, like
\texttt{\ counter(heading).update(...)\ } or
\texttt{\ counter(heading).step(...)\ } . Only occurrences of actual
\texttt{\ heading\ } s can be detected. So make sure that after you call
e.g. \texttt{\ counter(heading).update(...)\ } , you place a heading
directly after it, before you use any counters that depend on the
heading counter.

\subsubsection{How to add}\label{how-to-add}

Copy this into your project and use the import as \texttt{\ headcount\ }

\begin{verbatim}
#import "@preview/headcount:0.1.0"
\end{verbatim}

\includesvg[width=0.16667in,height=0.16667in]{/assets/icons/16-copy.svg}

Check the docs for
\href{https://typst.app/docs/reference/scripting/\#packages}{more
information on how to import packages} .

\subsubsection{About}\label{about}

\begin{description}
\tightlist
\item[Author :]
\href{https://jbirnick.net}{Johann Birnick}
\item[License:]
MIT
\item[Current version:]
0.1.0
\item[Last updated:]
October 16, 2024
\item[First released:]
October 16, 2024
\item[Archive size:]
2.67 kB
\href{https://packages.typst.org/preview/headcount-0.1.0.tar.gz}{\pandocbounded{\includesvg[keepaspectratio]{/assets/icons/16-download.svg}}}
\item[Repository:]
\href{https://github.com/jbirnick/typst-headcount}{GitHub}
\item[Categor ies :]
\begin{itemize}
\tightlist
\item[]
\item
  \pandocbounded{\includesvg[keepaspectratio]{/assets/icons/16-list-unordered.svg}}
  \href{https://typst.app/universe/search/?category=model}{Model}
\item
  \pandocbounded{\includesvg[keepaspectratio]{/assets/icons/16-code.svg}}
  \href{https://typst.app/universe/search/?category=scripting}{Scripting}
\item
  \pandocbounded{\includesvg[keepaspectratio]{/assets/icons/16-hammer.svg}}
  \href{https://typst.app/universe/search/?category=utility}{Utility}
\end{itemize}
\end{description}

\subsubsection{Where to report issues?}\label{where-to-report-issues}

This package is a project of Johann Birnick . Report issues on
\href{https://github.com/jbirnick/typst-headcount}{their repository} .
You can also try to ask for help with this package on the
\href{https://forum.typst.app}{Forum} .

Please report this package to the Typst team using the
\href{https://typst.app/contact}{contact form} if you believe it is a
safety hazard or infringes upon your rights.

\phantomsection\label{versions}
\subsubsection{Version history}\label{version-history}

\begin{longtable}[]{@{}ll@{}}
\toprule\noalign{}
Version & Release Date \\
\midrule\noalign{}
\endhead
\bottomrule\noalign{}
\endlastfoot
0.1.0 & October 16, 2024 \\
\end{longtable}

Typst GmbH did not create this package and cannot guarantee correct
functionality of this package or compatibility with any version of the
Typst compiler or app.
