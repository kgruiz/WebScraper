\title{typst.app/universe/package/minideck}

\phantomsection\label{banner}
\section{minideck}\label{minideck}

{ 0.2.1 }

Simple dynamic slides.

\phantomsection\label{readme}
A small package for dynamic slides in typst.

minideck provides basic functionality for dynamic slides (
\texttt{\ pause\ } , \texttt{\ uncover\ } , …), integration with
\href{https://typst.app/universe/package/fletcher}{fletcher} and
\href{https://typst.app/universe/package/cetz/}{CetZ} , and some minimal
infrastructure for theming.

\subsection{Usage}\label{usage}

Call \texttt{\ minideck.config\ } to get the functions you want to use:

\begin{Shaded}
\begin{Highlighting}[]
\NormalTok{\#import "@preview/minideck:0.2.1"}

\NormalTok{\#let (template, slide, title{-}slide, pause, uncover, only) = minideck.config()}
\NormalTok{\#show: template}

\NormalTok{\#title{-}slide[}
\NormalTok{  = Slides with \textasciigrave{}minideck\textasciigrave{}}
\NormalTok{  == Some examples}
\NormalTok{  John Doe}

\NormalTok{  \#datetime.today().display()}
\NormalTok{]}

\NormalTok{\#slide[}
\NormalTok{  = Some title}

\NormalTok{  Some content}

\NormalTok{  \#show: pause}

\NormalTok{  More content}

\NormalTok{  1. One}
\NormalTok{  2. Two}
\NormalTok{  \#show: pause}
\NormalTok{  3. Three}
\NormalTok{]}
\end{Highlighting}
\end{Shaded}

This will show three subslides with progressively more content. (Note
that the default theme uses the font Libertinus Sans from the
\href{https://github.com/alerque/libertinus}{Libertinus} family, so you
may want to install it.)

Instead of \texttt{\ \#show:\ pause\ } , you can use
\texttt{\ \#uncover(2,3){[}...{]}\ } to make content visible only on
subslides 2 and 3, or \texttt{\ \#uncover(from:\ 2){[}...{]}\ } to have
it visible on subslide 2 and following.

The \texttt{\ only\ } function is similar to \texttt{\ uncover\ } , but
instead of hiding the content (without affecting the layout), it removes
it.

\begin{Shaded}
\begin{Highlighting}[]
\NormalTok{\#slide[}
\NormalTok{  = \textasciigrave{}uncover\textasciigrave{} and \textasciigrave{}only\textasciigrave{}}
  
\NormalTok{  \#uncover(1, from:3)[}
\NormalTok{    Content visible on subslides 1 and 3+}
\NormalTok{    (space reserved on 2).}
\NormalTok{  ]}

\NormalTok{  \#only(2,3)[}
\NormalTok{    Content included on subslides 2 and 3}
\NormalTok{    (no space reserved on 1).}
\NormalTok{  ]}

\NormalTok{  Content always visible.}
\NormalTok{]}
\end{Highlighting}
\end{Shaded}

Contrary to \texttt{\ pause\ } , the \texttt{\ uncover\ } and
\texttt{\ only\ } functions also work in math mode:

\begin{Shaded}
\begin{Highlighting}[]
\NormalTok{\#slide[}
\NormalTok{  = Dynamic equation}

\NormalTok{  $}
\NormalTok{    f(x) \&= x\^{}2 + 2x + 1  \textbackslash{}}
\NormalTok{         \#uncover(2, $\&= (x + 1)\^{}2$)}
\NormalTok{  $}
\NormalTok{]}
\end{Highlighting}
\end{Shaded}

When mixing \texttt{\ pause\ } with \texttt{\ uncover\ } /
\texttt{\ only\ } , the sequence of pauses should be taken as reference
for the meaning of subslide indices. For example content after the first
pause always appears on the second subslide, even if it’s preceded by
a call to \texttt{\ \#uncover(from:\ 3){[}...{]}\ } .

The package also works well with
\href{https://typst.app/universe/package/pinit}{pinit} :

\begin{Shaded}
\begin{Highlighting}[]
\NormalTok{\#import "@preview/pinit:0.1.4": *}

\NormalTok{\#slide[}
\NormalTok{  = Works well with \textasciigrave{}pinit\textasciigrave{}}

\NormalTok{  Pythagorean theorem:}

\NormalTok{  $ \#pin(1)a\^{}2\#pin(2) + \#pin(3)b\^{}2\#pin(4) = \#pin(5)c\^{}2\#pin(6) $}

\NormalTok{  \#show: pause}

\NormalTok{  $a\^{}2$ and $b\^{}2$ : squares of triangle legs}

\NormalTok{  \#only(2, \{}
\NormalTok{    pinit{-}highlight(1,2)}
\NormalTok{    pinit{-}highlight(3,4)}
\NormalTok{  \})}

\NormalTok{  \#show: pause}

\NormalTok{  $c\^{}2$ : square of hypotenuse}

\NormalTok{  \#pinit{-}highlight(5,6, fill: green.transparentize(80\%))}
\NormalTok{  \#pinit{-}point{-}from(6)[larger than $a\^{}2$ and $b\^{}2$]}
\NormalTok{]}
\end{Highlighting}
\end{Shaded}

\subsubsection{Handout mode}\label{handout-mode}

minideck can make a handout version of the document, in which dynamic
behavior is disabled: the content of all subslides is shown together in
a single slide.

To compile a handout version, pass \texttt{\ -\/-input\ handout=true\ }
in the command line:

\begin{Shaded}
\begin{Highlighting}[]
\ExtensionTok{typst}\NormalTok{ compile }\AttributeTok{{-}{-}input}\NormalTok{ handout=true myfile.typ}
\end{Highlighting}
\end{Shaded}

It is also possible to enable handout mode from within the document, as
shown in the next section.

\subsubsection{Configuration}\label{configuration}

The behavior of the minideck functions depends on the settings passed to
\texttt{\ minideck.config\ } . For example, handout mode can also be
enabled like this:

\begin{Shaded}
\begin{Highlighting}[]
\NormalTok{\#import "@preview/minideck:0.2.1"}

\NormalTok{\#let (template, slide, pause) = minideck.config(handout: true)}
\NormalTok{\#show: template}

\NormalTok{\#slide[}
\NormalTok{  = Slide title}
  
\NormalTok{  Some text}

\NormalTok{  \#show: pause}

\NormalTok{  More text}
\NormalTok{]}
\end{Highlighting}
\end{Shaded}

(The default value of \texttt{\ handout\ } is \texttt{\ auto\ } , in
which case minideck checks for a command line setting as in the previous
section.)

\texttt{\ minideck.config\ } accepts the following named arguments:

\begin{itemize}
\tightlist
\item
  \texttt{\ paper\ } : a string for one of the paper size names
  recognized by
  \href{https://typst.app/docs/reference/layout/page/\#parameters-paper}{\texttt{\ page.paper\ }}
  , or one of the shorthands \texttt{\ "16:9"\ } or \texttt{\ "4:3"\ } .
  Default: \texttt{\ "4:3"\ } .
\item
  \texttt{\ landscape\ } : use the paper size in landscape orientation.
  Default: \texttt{\ true\ } .
\item
  \texttt{\ width\ } : page width as an absolute length. Takes
  precedence over \texttt{\ paper\ } and \texttt{\ landscape\ } .
\item
  \texttt{\ height\ } : page height as an absolute length. Takes
  precedence over \texttt{\ paper\ } and \texttt{\ landscape\ } .
\item
  \texttt{\ handout\ } : whether to make a document for handout, with
  content of all subslides shown together in a single slide.
\item
  \texttt{\ theme\ } : the theme (see below).
\item
  \texttt{\ cetz\ } : the CeTZ module (see below).
\item
  \texttt{\ fletcher\ } : the fletcher module (see below).
\end{itemize}

For example to make slides with 16:9 aspect ratio, use
\texttt{\ minideck.config(paper:\ "16:9")\ } .

\subsubsection{Themes}\label{themes}

Use \texttt{\ minideck.config(theme:\ ...)\ } to select a theme.
Currently there is only one built-in:
\texttt{\ minideck.themes.simple\ } . However you can also pass a theme
implemented by yourself or from a third-party package. See the
\href{https://github.com/typst/packages/raw/main/packages/preview/minideck/0.2.1/themes/README.md}{theme
documentation} for how that works.

Themes are functions and can be configured using the standard
\href{https://typst.app/docs/reference/foundations/function/\#definitions-with}{\texttt{\ with\ }
method} :

\begin{itemize}
\tightlist
\item
  The \texttt{\ simple\ } theme has a \texttt{\ variant\ } setting with
  values “light� (default) and “dark�.
\end{itemize}

Here’s an example:

\begin{Shaded}
\begin{Highlighting}[]
\NormalTok{\#import "@preview/minideck:0.2.1"}

\NormalTok{\#let (template, slide) = minideck.config(}
\NormalTok{  theme: minideck.themes.simple.with(variant: "dark"),}
\NormalTok{)}
\NormalTok{\#show: template}

\NormalTok{\#slide[}
\NormalTok{  = Slide with dark theme}
  
\NormalTok{  Some text}
\NormalTok{]}
\end{Highlighting}
\end{Shaded}

Note that you can override part of a theme with show and set rules:

\begin{Shaded}
\begin{Highlighting}[]
\NormalTok{\#import "@preview/minideck:0.2.1"}

\NormalTok{\#let (template, slide) = minideck.config(}
\NormalTok{  theme: minideck.themes.simple.with(variant: "dark"),}
\NormalTok{)}
\NormalTok{\#show: template}

\NormalTok{\#set page(footer: none) // get rid of the page number}
\NormalTok{\#show heading: it =\textgreater{} text(style: "italic", it)}
\NormalTok{\#set text(red)}

\NormalTok{\#slide[}
\NormalTok{  = Slide with dark theme and red text}
  
\NormalTok{  Some text}
\NormalTok{]}
\end{Highlighting}
\end{Shaded}

\subsubsection{Integration with CeTZ}\label{integration-with-cetz}

You can use \texttt{\ uncover\ } and \texttt{\ only\ } (but not
\texttt{\ pause\ } ) in CeTZ figures, with the following extra steps:

\begin{itemize}
\item
  Get CeTZ-specific functions \texttt{\ cetz-uncover\ } and
  \texttt{\ cetz-only\ } by passing the CeTZ module to
  \texttt{\ minideck.config\ } (see example below).

  This ensures that minideck uses CeTZ functions from the correct
  version of CeTZ.
\item
  Add a \texttt{\ context\ } keyword outside the \texttt{\ canvas\ }
  call.

  This is required to access the minideck subslide state from within the
  canvas without making the content opaque (CeTZ needs to inspect the
  canvas content to make the drawing).
\end{itemize}

Example:

\begin{Shaded}
\begin{Highlighting}[]
\NormalTok{\#import "@preview/cetz:0.2.2" as cetz: *}
\NormalTok{\#import "@preview/minideck:0.2.1"}

\NormalTok{\#let (template, slide, only, cetz{-}uncover, cetz{-}only) = minideck.config(cetz: cetz)}
\NormalTok{\#show: template}

\NormalTok{\#slide[}
\NormalTok{  = In a CeTZ figure}

\NormalTok{  Above canvas}
\NormalTok{  \#context canvas(\{}
\NormalTok{    import draw: *}
\NormalTok{    cetz{-}only(3, rect((0,{-}2), (14,4), stroke: 3pt))}
\NormalTok{    cetz{-}uncover(from: 2, rect((0,{-}2), (16,2), stroke: blue+3pt))}
\NormalTok{    content((8,0), box(stroke: red+3pt, inset: 1em)[}
\NormalTok{      A typst box \#only(2)[on 2nd subslide]}
\NormalTok{    ])}
\NormalTok{  \})}
\NormalTok{  Below canvas}
\NormalTok{]}
\end{Highlighting}
\end{Shaded}

\subsubsection{Integration with
fletcher}\label{integration-with-fletcher}

The same steps are required as for CeTZ integration (passing the
fletcher module to get fletcher-specific functions), plus an additional
step:

\begin{itemize}
\item
  Give explicitly the number of subslides to the \texttt{\ slide\ }
  function.

  This is required because I could not find a reliable way to update a
  typst state from within a fletcher diagram, so I cannot rely on the
  state to keep track of the number of subslides.
\end{itemize}

Example:

\begin{Shaded}
\begin{Highlighting}[]
\NormalTok{\#import "@preview/fletcher:0.5.0" as fletcher: diagram, node, edge}
\NormalTok{\#import "@preview/minideck:0.2.1"}
\NormalTok{\#let (template, slide, fletcher{-}uncover) = minideck.config(fletcher: fletcher)}
\NormalTok{\#show: template}

\NormalTok{\#slide(steps: 2)[}
\NormalTok{  = In a fletcher diagram}

\NormalTok{  \#set align(center)}

\NormalTok{  Above diagram}

\NormalTok{  \#context diagram(}
\NormalTok{    node{-}stroke: 1pt,}
\NormalTok{    node((0,0), [Start], corner{-}radius: 2pt, extrude: (0, 3)),}
\NormalTok{    edge("{-}|\textgreater{}"),}
\NormalTok{    node((1,0), align(center)[A]),}
\NormalTok{    fletcher{-}uncover(from: 2, edge("d,r,u,l", "{-}|\textgreater{}", [x], label{-}pos: 0.1))}
\NormalTok{  )}
  
\NormalTok{  Below diagram}
\NormalTok{]}
\end{Highlighting}
\end{Shaded}

\subsection{Comparison with other slides
packages}\label{comparison-with-other-slides-packages}

Performance: minideck is currently faster than
\href{https://typst.app/universe/package/polylux/}{Polylux} when using
\texttt{\ pause\ } , especially for incremental compilation, but a bit
slower than \href{https://typst.app/universe/package/touying}{Touying} ,
according to my tests.

Features: Polylux and Touying have more themes and more features, for
example support for \href{https://pdfpc.github.io/}{pdfpc} which
provides speaker notes and more. Minideck allows using
\texttt{\ uncover\ } and \texttt{\ only\ } in CeTZ figures and fletcher
diagrams, which Polylux currently doesn’t support.

Syntax: package configuration is simpler in minideck than Touying but a
bit more involved than in Polylux. The minideck \texttt{\ pause\ } is
more cumbersome to use: one must write \texttt{\ \#show:\ pause\ }
instead of \texttt{\ \#pause\ } . On the other hand minideck’s
\texttt{\ uncover\ } and \texttt{\ only\ } can be used directly in
equations without requiring a special math environment as in Touying (I
think).

Other: minideck sometimes has better error messages than Touying due to
implementation differences: the minideck stack trace points back to the
user’s code while Touying errors sometimes point only to an output
page number.

\subsection{Limitations}\label{limitations}

\begin{itemize}
\tightlist
\item
  \texttt{\ pause\ } , \texttt{\ uncover\ } and \texttt{\ only\ } work
  in enumerations but they require explicit enum indices (
  \texttt{\ 1.\ ...\ } rather than \texttt{\ +\ ...\ } ) as they cause a
  reset of the list index.
\item
  Usage in a CeTZ canvas or fletcher diagram requires a
  \texttt{\ context\ } keyword in front of the \texttt{\ canvas\ } /
  \texttt{\ diagram\ } call (see above).
\item
  fletcher diagrams also require to specify the number of subslides
  explicitly (see above).
\item
  \texttt{\ pause\ } doesn’t work in CeTZ figures, fletcher diagrams
  and math mode.
\item
  \texttt{\ pause\ } requires writing \texttt{\ \#show:\ pause\ } and
  its effect is lost after the \texttt{\ \#show\ } call goes out of
  scope. For example this means that one can use \texttt{\ pause\ }
  inside of a grid, but further \texttt{\ pause\ } calls after the grid
  (in the same slide) won’t work as intended.
\end{itemize}

\subsection{Internals}\label{internals}

The package uses states with the following keys:

\texttt{\ \_\_minideck-subslide-count\ } : an array of two integers for
counting pauses and keeping track of the subslide number automatically.
The first value is the number of subslides so far referenced in current
slide. The second value is the number of pauses seen so far in the
current slide. Both values are kept in one state so that an update
function can update the number of subslides based on the number of
pauses, without requiring a context. This avoids problems with layout
convergence.

\texttt{\ \_\_minideck-subslide-step\ } : the current subslide being
generated while processing a slide.

\subsubsection{How to add}\label{how-to-add}

Copy this into your project and use the import as \texttt{\ minideck\ }

\begin{verbatim}
#import "@preview/minideck:0.2.1"
\end{verbatim}

\includesvg[width=0.16667in,height=0.16667in]{/assets/icons/16-copy.svg}

Check the docs for
\href{https://typst.app/docs/reference/scripting/\#packages}{more
information on how to import packages} .

\subsubsection{About}\label{about}

\begin{description}
\tightlist
\item[Author :]
\href{https://github.com/knuesel}{Jeremie Knuesel}
\item[License:]
MIT
\item[Current version:]
0.2.1
\item[Last updated:]
July 1, 2024
\item[First released:]
July 1, 2024
\item[Archive size:]
10.3 kB
\href{https://packages.typst.org/preview/minideck-0.2.1.tar.gz}{\pandocbounded{\includesvg[keepaspectratio]{/assets/icons/16-download.svg}}}
\item[Repository:]
\href{https://github.com/knuesel/typst-minideck}{GitHub}
\item[Categor y :]
\begin{itemize}
\tightlist
\item[]
\item
  \pandocbounded{\includesvg[keepaspectratio]{/assets/icons/16-presentation.svg}}
  \href{https://typst.app/universe/search/?category=presentation}{Presentation}
\end{itemize}
\end{description}

\subsubsection{Where to report issues?}\label{where-to-report-issues}

This package is a project of Jeremie Knuesel . Report issues on
\href{https://github.com/knuesel/typst-minideck}{their repository} . You
can also try to ask for help with this package on the
\href{https://forum.typst.app}{Forum} .

Please report this package to the Typst team using the
\href{https://typst.app/contact}{contact form} if you believe it is a
safety hazard or infringes upon your rights.

\phantomsection\label{versions}
\subsubsection{Version history}\label{version-history}

\begin{longtable}[]{@{}ll@{}}
\toprule\noalign{}
Version & Release Date \\
\midrule\noalign{}
\endhead
\bottomrule\noalign{}
\endlastfoot
0.2.1 & July 1, 2024 \\
\end{longtable}

Typst GmbH did not create this package and cannot guarantee correct
functionality of this package or compatibility with any version of the
Typst compiler or app.
