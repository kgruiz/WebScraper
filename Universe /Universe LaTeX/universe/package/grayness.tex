\title{typst.app/universe/package/grayness}

\phantomsection\label{banner}
\section{grayness}\label{grayness}

{ 0.2.0 }

Simple image editing capabilities like converting to grayscale and
cropping via a WASM plugin.

\phantomsection\label{readme}
A package providing simple image editing capabilities via a WASM plugin.

Available functionality includes converting images to grayscale,
cropping and flipping the images. Furthermore, this package supports
adding transparency and bluring (very slow) as well as handling
additional raster image formats.

The package name is inspired by the blurry, gray images of Nessie, the
\href{https://en.wikipedia.org/wiki/Loch_Ness_Monster}{Loch Ness
Monster}

\subsection{Usage}\label{usage}

Due to the way typst currently interprets given paths, you have to read
the images yourself in the calling typst file. This raw imagedata can
then be passed to the grayness-package functions, like grayscale-image.
These functions also optionally accept all additional parameters of the
original typst image function like \texttt{\ width\ } or
\texttt{\ height\ } :

\begin{Shaded}
\begin{Highlighting}[]
\NormalTok{\#import "@preview/grayness:0.2.0": grayscale{-}image}

\NormalTok{\#let data = read("Art.webp", encoding: none)}
\NormalTok{\#grayscale{-}image(data, width: 50\%)}
\end{Highlighting}
\end{Shaded}

A detailed descriptions of all available functions is provided in the
\href{https://github.com/typst/packages/raw/main/packages/preview/grayness/0.2.0/manual.pdf}{manual}
.

You can also use the built-in help functions provided by tidy:

\begin{Shaded}
\begin{Highlighting}[]
\NormalTok{\#import "@preview/grayness:0.2.0": *}
\NormalTok{\#help("flip{-}image{-}vertical")}
\end{Highlighting}
\end{Shaded}

The \texttt{\ grayscale-image\ } function also works with SVG images. To
do so you must specify the format as \texttt{\ "svg"\ } :

\begin{Shaded}
\begin{Highlighting}[]
\NormalTok{\#let data = read("example.svg", encoding: none)}
\NormalTok{\#grayscale{-}image(data, format: "svg")}
\end{Highlighting}
\end{Shaded}

\subsection{Examples}\label{examples}

Here are several functions applied to a WEBP image of
\href{https://commons.wikimedia.org/wiki/File:Arturo_Nieto-Dorantes.webp}{Arturo
Nieto Dorantes} (CC-By-SA 4.0):
\pandocbounded{\includegraphics[keepaspectratio]{https://github.com/typst/packages/raw/main/packages/preview/grayness/0.2.0/example.png}}

\subsubsection{How to add}\label{how-to-add}

Copy this into your project and use the import as \texttt{\ grayness\ }

\begin{verbatim}
#import "@preview/grayness:0.2.0"
\end{verbatim}

\includesvg[width=0.16667in,height=0.16667in]{/assets/icons/16-copy.svg}

Check the docs for
\href{https://typst.app/docs/reference/scripting/\#packages}{more
information on how to import packages} .

\subsubsection{About}\label{about}

\begin{description}
\tightlist
\item[Author :]
Nikolai Neff-Sarnow
\item[License:]
Apache-2.0
\item[Current version:]
0.2.0
\item[Last updated:]
October 10, 2024
\item[First released:]
April 13, 2024
\item[Minimum Typst version:]
0.11.0
\item[Archive size:]
682 kB
\href{https://packages.typst.org/preview/grayness-0.2.0.tar.gz}{\pandocbounded{\includesvg[keepaspectratio]{/assets/icons/16-download.svg}}}
\item[Repository:]
\href{https://github.com/nineff/grayness}{GitHub}
\item[Categor ies :]
\begin{itemize}
\tightlist
\item[]
\item
  \pandocbounded{\includesvg[keepaspectratio]{/assets/icons/16-chart.svg}}
  \href{https://typst.app/universe/search/?category=visualization}{Visualization}
\item
  \pandocbounded{\includesvg[keepaspectratio]{/assets/icons/16-integration.svg}}
  \href{https://typst.app/universe/search/?category=integration}{Integration}
\item
  \pandocbounded{\includesvg[keepaspectratio]{/assets/icons/16-hammer.svg}}
  \href{https://typst.app/universe/search/?category=utility}{Utility}
\end{itemize}
\end{description}

\subsubsection{Where to report issues?}\label{where-to-report-issues}

This package is a project of Nikolai Neff-Sarnow . Report issues on
\href{https://github.com/nineff/grayness}{their repository} . You can
also try to ask for help with this package on the
\href{https://forum.typst.app}{Forum} .

Please report this package to the Typst team using the
\href{https://typst.app/contact}{contact form} if you believe it is a
safety hazard or infringes upon your rights.

\phantomsection\label{versions}
\subsubsection{Version history}\label{version-history}

\begin{longtable}[]{@{}ll@{}}
\toprule\noalign{}
Version & Release Date \\
\midrule\noalign{}
\endhead
\bottomrule\noalign{}
\endlastfoot
0.2.0 & October 10, 2024 \\
\href{https://typst.app/universe/package/grayness/0.1.0/}{0.1.0} & April
13, 2024 \\
\end{longtable}

Typst GmbH did not create this package and cannot guarantee correct
functionality of this package or compatibility with any version of the
Typst compiler or app.
