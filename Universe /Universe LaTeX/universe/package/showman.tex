\title{typst.app/universe/package/showman}

\phantomsection\label{banner}
\section{showman}\label{showman}

{ 0.1.2 }

Eval \& show typst code outputs inline with their source

\phantomsection\label{readme}
\pandocbounded{\includegraphics[keepaspectratio]{https://www.github.com/ntjess/showman/raw/v0.1.0/showman.jpg}}

\begin{center}\rule{0.5\linewidth}{0.5pt}\end{center}

Automagic tools to smooth the package documentation \& development
process.

\begin{itemize}
\item
  Package your files for local typst installation or PR submission in
  one command
\item
  Provide your readme in typst format with code block examples, and let
  \texttt{\ showman\ } do the rest! In one command, it will the readme
  to markdown and render code block outputs as included images.

  \begin{itemize}
  \tightlist
  \item
    Bonus points â€`` let \texttt{\ showman\ } know your repository path
    and it will ensure images will properly appear in your readme even
    after your package has been distributed through typst’s registry.
  \end{itemize}
\item
  Execute non-typst code blocks and render their outputs
\end{itemize}

\textbf{Prerequisites} : Make sure you have \texttt{\ typst\ } and
\texttt{\ pandoc\ } available from the command line. Then, in a python
virtual environment, run:

\begin{Shaded}
\begin{Highlighting}[]
\ExtensionTok{pip}\NormalTok{ install showman}
\end{Highlighting}
\end{Shaded}

Create a typst file with
\texttt{\ \textasciigrave{}\textasciigrave{}\textasciigrave{}example\ }
code blocks that show the output you want to include in your readme. For
instance:

\begin{Shaded}
\begin{Highlighting}[]
\NormalTok{\#import "@preview/cetz:0.1.2"}
\NormalTok{// Just to avoid showing this heading in the readme itself}
\NormalTok{\#set heading(outlined: false)}

\NormalTok{= Hello, world!}
\NormalTok{Let\textquotesingle{}s do something complicated:}

\NormalTok{\#cetz.canvas(\{}
\NormalTok{  import cetz.plot}
\NormalTok{  import cetz.palette}
\NormalTok{  cetz.draw.set{-}style(axes: (tick: (length: {-}.05)))}
\NormalTok{  // Plot something}
\NormalTok{  plot.plot(size: (3,3), x{-}tick{-}step: 1, axis{-}style: "left", \{}
\NormalTok{      for i in range(0, 3) \{}
\NormalTok{      plot.add(domain: ({-}4, 2),}
\NormalTok{      x =\textgreater{} calc.exp({-}(calc.pow(x + i, 2))),}
\NormalTok{      fill: true, style: palette.tango)}
\NormalTok{    \}}
\NormalTok{  \})}
\NormalTok{\})}
\end{Highlighting}
\end{Shaded}

\pandocbounded{\includegraphics[keepaspectratio]{https://www.github.com/ntjess/showman/raw/v0.1.0/assets/example-1.png}}

Then, run the following command:

\begin{Shaded}
\begin{Highlighting}[]
\ExtensionTok{showman}\NormalTok{ md }
\end{Highlighting}
\end{Shaded}

Congrats, you now have a readme with inline images 🎉

You can optionally specify your workspace root, output file name, image
folder, etc. These options are visible under
\texttt{\ showman\ md\ -\/-help\ } .

\textbf{Note} : You can customize the appearance of these images by
providing \texttt{\ showman\ } the template to use when creating them.
In your file to be rendered, create a variable called
\texttt{\ showman-config\ } at the outermost scope:

\begin{Shaded}
\begin{Highlighting}[]
\NormalTok{// Render images with a black background and red text}
\NormalTok{\#let showman{-}config = (}
\NormalTok{  template: it =\textgreater{} \{}
\NormalTok{    set text(fill: red)}
\NormalTok{    rect(fill: black, it)}
\NormalTok{  \}}
\NormalTok{)}
\end{Highlighting}
\end{Shaded}

Behind the scenes, showman imports your file as a module and looks for
this variable. If it is found, your template and a few other options are
injected into the example rendering process.

\textbf{Note} : If every example has the same setup (package imports,
etc.), and you don’t want the text to be included in your examples,
you can pass \texttt{\ eval-kwargs\ } in this config as well to specify
a string that gets prefixed to every example. Alternatively, pass
variables in a scope directly:

\begin{Shaded}
\begin{Highlighting}[]
\NormalTok{// Setup either through providing scope or import prefixes}
\NormalTok{\#let my{-}variable = 5}
\NormalTok{\#let showman{-}config = (}
\NormalTok{  eval{-}kwargs: (}
\NormalTok{    prefix: "\#import \textbackslash{}"@preview/cetz:0.1.2\textbackslash{}"}
\NormalTok{  ),}
\NormalTok{  // Now you can use \textasciigrave{}my{-}var\textasciigrave{} in your examples}
\NormalTok{  scope: (my{-}var: my{-}variable)}
\NormalTok{)}
\end{Highlighting}
\end{Shaded}

\subsection{Caveats}\label{caveats}

\begin{itemize}
\item
  \texttt{\ showman\ } uses the beautiful \texttt{\ pandoc\ } to do most
  of the markdown conversion heavy lifting. So, if your document can’t
  be processed by pandoc, you may need to adjust your syntax until
  pandoc is happy making a markdown document.
\item
  Typst doesn’t allow page styling inside containers. Since
  \texttt{\ showman\ } must use containers to extract each rendered
  example, you can’t use \texttt{\ \#set\ page(...)\ } or
  \texttt{\ \#pagebreak()\ } inside your examples.
\end{itemize}

If you don’t care about converting your readme to markdown, it’s
even easier to have example rendered alongside their code. Simply add
the following preamble to your file:

\begin{Shaded}
\begin{Highlighting}[]
\NormalTok{\#import "@preview/showman:0.1.1"}
\NormalTok{\#show: showman.formatter.template}

\NormalTok{The code below will be rendered side by side with its output:}

\NormalTok{\textasciigrave{}\textasciigrave{}\textasciigrave{} typst}
\NormalTok{= Hello world!}
\NormalTok{\textasciigrave{}\textasciigrave{}\textasciigrave{}}
\NormalTok{![Example 2](https://www.github.com/ntjess/showman/raw/v0.1.0/assets/example{-}2.png)}

\NormalTok{Several keywords can be privded to customize appearance and more. See \textasciigrave{}showman.formatter.template\textasciigrave{} for more details.}
\end{Highlighting}
\end{Shaded}

You’ve done the hard work of creating a beautiful, well-documented
manual. Now it’s time to share it with the world. \texttt{\ showman\ }
can help you package your files for distribution in one command, after
some minimal setup.

\begin{enumerate}
\item
  Make sure you have a \texttt{\ typst.toml\ } file that follows typst
  \href{https://github.com/typst/packages}{packaging guidelines}
\item
  Add a new block to your toml file as follows:
\end{enumerate}

\begin{Shaded}
\begin{Highlighting}[]
\KeywordTok{[tool.packager]}
\DataTypeTok{paths} \OperatorTok{=} \OperatorTok{[}\ErrorTok{...}\OperatorTok{]}
\end{Highlighting}
\end{Shaded}

Where \texttt{\ paths\ } is a list of files and directories you want to
include in your package.

\begin{enumerate}
\setcounter{enumi}{2}
\tightlist
\item
  Run the following command from the root of your repository:
\end{enumerate}

\begin{Shaded}
\begin{Highlighting}[]
\ExtensionTok{showman}\NormalTok{ package }
\end{Highlighting}
\end{Shaded}

\begin{enumerate}
\setcounter{enumi}{3}
\item
  Without any other arguments, you’ve just installed your package in
  your system’s local typst packages folder. Now you can import it
  with
  \texttt{\ typst\ \#import\ "@local/mypackage:\textless{}version\textgreater{}"\ }
  .

  \begin{itemize}
  \tightlist
  \item
    You can alternatively specify the path to your fork of
    \texttt{\ typst/packages\ } to prep your files for a PR, or specify
    a \texttt{\ -\/-namespace\ } other than \texttt{\ local\ } .
  \end{itemize}
\end{enumerate}

\textbf{Note} : You can see the full list of command options with
\texttt{\ showman\ package\ -\/-help\ } .

This package also executes non-typst code (currently bash on
non-windows, python, and c++). You can use
\texttt{\ showman\ execute\ ./path/to/file.typ\ } to execute code blocks
in these languages, and the output will be captured in a
\texttt{\ .coderunner.json\ } file in the root directory you specified.
To enable this feature, you need to add the following preamble to your
file:

\begin{Shaded}
\begin{Highlighting}[]
\NormalTok{\#import "@preview/showman:0.1.1": runner}

\NormalTok{\#let cache = json("/.coderunner.json").at("path/to/file.typ", default: (:))}
\NormalTok{\#let show{-}rule = runner.external{-}code.with(result{-}cache: cache)}

\NormalTok{// Now, apply the show rule to languages that have a \textasciigrave{}showman execute\textasciigrave{} result:}
\NormalTok{\#show raw.where(lang: "python"): show{-}rule}
\end{Highlighting}
\end{Shaded}

You can optionally style
\texttt{\ \textless{}example-input\textgreater{}\ } and
\texttt{\ \textless{}example-output\textgreater{}\ } labels to customize
how input and output blocks appear. For even deeper customization, you
can specify the \texttt{\ container\ } that displays the input and
output blocks that accepts a keyword \texttt{\ direction\ } and
positional \texttt{\ input\ } and \texttt{\ output\ } .

\subsubsection{How to add}\label{how-to-add}

Copy this into your project and use the import as \texttt{\ showman\ }

\begin{verbatim}
#import "@preview/showman:0.1.2"
\end{verbatim}

\includesvg[width=0.16667in,height=0.16667in]{/assets/icons/16-copy.svg}

Check the docs for
\href{https://typst.app/docs/reference/scripting/\#packages}{more
information on how to import packages} .

\subsubsection{About}\label{about}

\begin{description}
\tightlist
\item[Author :]
Nathan Jessurun
\item[License:]
Unlicense
\item[Current version:]
0.1.2
\item[Last updated:]
November 28, 2024
\item[First released:]
January 15, 2024
\item[Minimum Typst version:]
0.12.0
\item[Archive size:]
6.00 kB
\href{https://packages.typst.org/preview/showman-0.1.2.tar.gz}{\pandocbounded{\includesvg[keepaspectratio]{/assets/icons/16-download.svg}}}
\item[Repository:]
\href{https://github.com/ntjess/showman}{GitHub}
\item[Categor ies :]
\begin{itemize}
\tightlist
\item[]
\item
  \pandocbounded{\includesvg[keepaspectratio]{/assets/icons/16-code.svg}}
  \href{https://typst.app/universe/search/?category=scripting}{Scripting}
\item
  \pandocbounded{\includesvg[keepaspectratio]{/assets/icons/16-hammer.svg}}
  \href{https://typst.app/universe/search/?category=utility}{Utility}
\end{itemize}
\end{description}

\subsubsection{Where to report issues?}\label{where-to-report-issues}

This package is a project of Nathan Jessurun . Report issues on
\href{https://github.com/ntjess/showman}{their repository} . You can
also try to ask for help with this package on the
\href{https://forum.typst.app}{Forum} .

Please report this package to the Typst team using the
\href{https://typst.app/contact}{contact form} if you believe it is a
safety hazard or infringes upon your rights.

\phantomsection\label{versions}
\subsubsection{Version history}\label{version-history}

\begin{longtable}[]{@{}ll@{}}
\toprule\noalign{}
Version & Release Date \\
\midrule\noalign{}
\endhead
\bottomrule\noalign{}
\endlastfoot
0.1.2 & November 28, 2024 \\
\href{https://typst.app/universe/package/showman/0.1.1/}{0.1.1} & March
16, 2024 \\
\href{https://typst.app/universe/package/showman/0.1.0/}{0.1.0} &
January 15, 2024 \\
\end{longtable}

Typst GmbH did not create this package and cannot guarantee correct
functionality of this package or compatibility with any version of the
Typst compiler or app.
