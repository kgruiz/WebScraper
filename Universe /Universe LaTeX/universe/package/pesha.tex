\title{typst.app/universe/package/pesha}

\phantomsection\label{banner}
\phantomsection\label{template-thumbnail}
\pandocbounded{\includegraphics[keepaspectratio]{https://packages.typst.org/preview/thumbnails/pesha-0.4.0-small.webp}}

\section{pesha}\label{pesha}

{ 0.4.0 }

A clean and minimal template for your résumé or CV

\href{/app?template=pesha&version=0.4.0}{Create project in app}

\phantomsection\label{readme}
\begin{quote}
Pesha (Urdu: پیشÛ?) is the Urdu term for occupation/profession. It is
pronounced as pay-sha.
\end{quote}

A clean and minimal template for your CV or résumé.

This template is inspired by Matthew Butterick’s excellent
\href{https://practicaltypography.com/}{\emph{Practical Typography}}
book.

See
\href{https://github.com/talal/pesha/blob/main/example.pdf}{example.pdf}
or
\href{https://github.com/talal/pesha/blob/main/example-profile-picture.pdf}{example-profile-picture.pdf}
file to see how it looks.

\subsection{Usage}\label{usage}

You can use this template in the Typst web app by clicking “Start from
template� on the dashboard and searching for \texttt{\ pesha\ } .

Alternatively, you can use the CLI to kick this project off using the
command

\begin{Shaded}
\begin{Highlighting}[]
\ExtensionTok{typst}\NormalTok{ init @preview/pesha}
\end{Highlighting}
\end{Shaded}

Typst will create a new directory with all the files needed to get you
started.

\subsection{Configuration}\label{configuration}

This template exports the \texttt{\ pesha\ } function with the following
named arguments:

\begin{longtable}[]{@{}lll@{}}
\toprule\noalign{}
Argument & Type & Description \\
\midrule\noalign{}
\endhead
\bottomrule\noalign{}
\endlastfoot
\texttt{\ name\ } &
\href{https://typst.app/docs/reference/foundations/str/}{string} & A
string to specify the author’s name. \\
\texttt{\ address\ } &
\href{https://typst.app/docs/reference/foundations/str/}{string} & A
string to specify the author’s address. \\
\texttt{\ contacts\ } &
\href{https://typst.app/docs/reference/foundations/array/}{array} & An
array of content to specify your contact information. E.g., phone
number, email, LinkedIn, etc. \\
\texttt{\ profile-picture\ } &
\href{https://typst.app/docs/reference/foundations/content/}{content} &
The result of a call to the
\href{https://typst.app/docs/reference/visualize/image/}{image function}
or \texttt{\ none\ } . For best result, make sure that your image has an
1:1 aspect ratio. \\
\texttt{\ paper-size\ } &
\href{https://typst.app/docs/reference/foundations/str/}{string} &
Specify a
\href{https://typst.app/docs/reference/layout/page\#parameters-paper}{paper
size string} to change the page size (default is \texttt{\ a4\ } ). \\
\texttt{\ footer-text\ } &
\href{https://typst.app/docs/reference/foundations/content/}{content} &
Content that will be prepended to the page numbering in the footer. \\
\texttt{\ page-numbering-format\ } &
\href{https://typst.app/docs/reference/foundations/str/}{string} &
\href{https://typst.app/docs/reference/model/numbering/\#parameters-numbering}{Pattern}
that will be used for displaying page numbering in the footer (default
is \texttt{\ 1\ of\ 1\ } ). \\
\end{longtable}

The function also accepts a single, positional argument for the body.

The template will initialize your package with a sample call to the
\texttt{\ pesha\ } function in a show rule. If you, however, want to
change an existing project to use this template, you can add a show rule
like this at the top of your file:

\begin{Shaded}
\begin{Highlighting}[]
\NormalTok{\#import "@preview/pesha:0.4.0": *}

\NormalTok{\#show: pesha.with(}
\NormalTok{  name: "Max Mustermann",}
\NormalTok{  address: "5419 Hollywood Blvd Ste c731, Los Angeles, CA 90027",}
\NormalTok{  contacts: (}
\NormalTok{    [(323) 555 1435],}
\NormalTok{    [\#link("mailto:max@mustermann.com")],}
\NormalTok{  ),}
\NormalTok{  paper{-}size: "us{-}letter",}
\NormalTok{  footer{-}text: [Mustermann Résumé {-}{-}{-}]}
\NormalTok{)}

\NormalTok{// Your content goes below.}
\end{Highlighting}
\end{Shaded}

\href{/app?template=pesha&version=0.4.0}{Create project in app}

\subsubsection{How to use}\label{how-to-use}

Click the button above to create a new project using this template in
the Typst app.

You can also use the Typst CLI to start a new project on your computer
using this command:

\begin{verbatim}
typst init @preview/pesha:0.4.0
\end{verbatim}

\includesvg[width=0.16667in,height=0.16667in]{/assets/icons/16-copy.svg}

\subsubsection{About}\label{about}

\begin{description}
\tightlist
\item[Author :]
\href{https://github.com/talal}{Muhammad Talal Anwar}
\item[License:]
MIT-0
\item[Current version:]
0.4.0
\item[Last updated:]
October 24, 2024
\item[First released:]
March 23, 2024
\item[Minimum Typst version:]
0.12.0
\item[Archive size:]
4.24 kB
\href{https://packages.typst.org/preview/pesha-0.4.0.tar.gz}{\pandocbounded{\includesvg[keepaspectratio]{/assets/icons/16-download.svg}}}
\item[Repository:]
\href{https://github.com/talal/pesha}{GitHub}
\item[Categor y :]
\begin{itemize}
\tightlist
\item[]
\item
  \pandocbounded{\includesvg[keepaspectratio]{/assets/icons/16-user.svg}}
  \href{https://typst.app/universe/search/?category=cv}{CV}
\end{itemize}
\end{description}

\subsubsection{Where to report issues?}\label{where-to-report-issues}

This template is a project of Muhammad Talal Anwar . Report issues on
\href{https://github.com/talal/pesha}{their repository} . You can also
try to ask for help with this template on the
\href{https://forum.typst.app}{Forum} .

Please report this template to the Typst team using the
\href{https://typst.app/contact}{contact form} if you believe it is a
safety hazard or infringes upon your rights.

\phantomsection\label{versions}
\subsubsection{Version history}\label{version-history}

\begin{longtable}[]{@{}ll@{}}
\toprule\noalign{}
Version & Release Date \\
\midrule\noalign{}
\endhead
\bottomrule\noalign{}
\endlastfoot
0.4.0 & October 24, 2024 \\
\href{https://typst.app/universe/package/pesha/0.3.1/}{0.3.1} & April
19, 2024 \\
\href{https://typst.app/universe/package/pesha/0.3.0/}{0.3.0} & April
15, 2024 \\
\href{https://typst.app/universe/package/pesha/0.2.0/}{0.2.0} & April
12, 2024 \\
\href{https://typst.app/universe/package/pesha/0.1.0/}{0.1.0} & March
23, 2024 \\
\end{longtable}

Typst GmbH did not create this template and cannot guarantee correct
functionality of this template or compatibility with any version of the
Typst compiler or app.
