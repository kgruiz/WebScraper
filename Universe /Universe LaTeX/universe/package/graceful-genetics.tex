\title{typst.app/universe/package/graceful-genetics}

\phantomsection\label{banner}
\phantomsection\label{template-thumbnail}
\pandocbounded{\includegraphics[keepaspectratio]{https://packages.typst.org/preview/thumbnails/graceful-genetics-0.2.0-small.webp}}

\section{graceful-genetics}\label{graceful-genetics}

{ 0.2.0 }

A paper template with which to publish in journals and at conferences

{ } Featured Template

\href{/app?template=graceful-genetics&version=0.2.0}{Create project in
app}

\phantomsection\label{readme}
Version 0.2.0

A recreation of the Oxford Physics template shown on the typst.app
homepage.

\subsection{Media}\label{media}

\includegraphics[width=0.45\linewidth,height=\textheight,keepaspectratio]{https://github.com/typst/packages/raw/main/packages/preview/graceful-genetics/0.2.0/thumbnails/1.png}
\includegraphics[width=0.45\linewidth,height=\textheight,keepaspectratio]{https://github.com/typst/packages/raw/main/packages/preview/graceful-genetics/0.2.0/thumbnails/2.png}

\subsection{Getting Started}\label{getting-started}

To use this template, simply import it as shown below:

\begin{Shaded}
\begin{Highlighting}[]
\NormalTok{\#import "@preview/graceful{-}genetics:0.2.0"}

\NormalTok{\#show: graceful{-}genetics.template.with(}
\NormalTok{  title: [Towards Swifter Interstellar Mail Delivery],}
\NormalTok{  authors: (}
\NormalTok{    (}
\NormalTok{      name: "Johanna Swift",}
\NormalTok{      department: "Primary Logistics Department",}
\NormalTok{      institution: "Delivery Institute",}
\NormalTok{      city: "Berlin",}
\NormalTok{      country: "Germany",}
\NormalTok{      mail: "swift@delivery.de",}
\NormalTok{    ),}
\NormalTok{    (}
\NormalTok{      name: "Egon Stellaris",}
\NormalTok{      department: "Communications Group",}
\NormalTok{      institution: "Space Institute",}
\NormalTok{      city: "Florence",}
\NormalTok{      country: "Italy",}
\NormalTok{      mail: "stegonaris@space.it",}
\NormalTok{    ),}
\NormalTok{    (}
\NormalTok{      name: "Oliver Liam",}
\NormalTok{      department: "Missing Letters Task Force",}
\NormalTok{      institution: "Mail Institute",}
\NormalTok{      city: "Budapest",}
\NormalTok{      country: "Hungary",}
\NormalTok{      mail: "oliver.liam@mail.hu",}
\NormalTok{    ),}
\NormalTok{  ),}
\NormalTok{  date: (}
\NormalTok{    year: 2022,}
\NormalTok{    month: "May",}
\NormalTok{    day: 17,}
\NormalTok{  ),}
\NormalTok{  keywords: (}
\NormalTok{    "Space",}
\NormalTok{    "Mail",}
\NormalTok{    "Astromail",}
\NormalTok{    "Faster{-}than{-}Light",}
\NormalTok{    "Mars",}
\NormalTok{  ),}
\NormalTok{  doi: "10:7891/120948510",}
\NormalTok{  abstract: [}
\NormalTok{    Recent advances in space{-}based document processing have enabled faster mail delivery between different planets of a solar system. Given the time it takes for a message to be transmitted from one planet to the next, its estimated that even a one{-}way trip to a distant destination could take up to one year. During these periods of interplanetary mail delivery there is a slight possibility of mail being lost in transit. This issue is considered so serious that space management employs P.I. agents to track down and retrieve lost mail. We propose A{-}Mail, a new anti{-}matter based approach that can ensure that mail loss occurring during interplanetary transit is unobservable and therefore potentially undetectable. Going even further, we extend A{-}Mail to predict problems and apply existing and new best practices to ensure the mail is delivered without any issues. We call this extension AI{-}Mail.}
\NormalTok{  ]}
\NormalTok{)}
\end{Highlighting}
\end{Shaded}

\href{/app?template=graceful-genetics&version=0.2.0}{Create project in
app}

\subsubsection{How to use}\label{how-to-use}

Click the button above to create a new project using this template in
the Typst app.

You can also use the Typst CLI to start a new project on your computer
using this command:

\begin{verbatim}
typst init @preview/graceful-genetics:0.2.0
\end{verbatim}

\includesvg[width=0.16667in,height=0.16667in]{/assets/icons/16-copy.svg}

\subsubsection{About}\label{about}

\begin{description}
\tightlist
\item[Author :]
James R. Swift
\item[License:]
Unlicense
\item[Current version:]
0.2.0
\item[Last updated:]
October 30, 2024
\item[First released:]
July 16, 2024
\item[Minimum Typst version:]
0.12.0
\item[Archive size:]
24.5 kB
\href{https://packages.typst.org/preview/graceful-genetics-0.2.0.tar.gz}{\pandocbounded{\includesvg[keepaspectratio]{/assets/icons/16-download.svg}}}
\item[Repository:]
\href{https://github.com/JamesxX/graceful-genetics}{GitHub}
\item[Categor y :]
\begin{itemize}
\tightlist
\item[]
\item
  \pandocbounded{\includesvg[keepaspectratio]{/assets/icons/16-atom.svg}}
  \href{https://typst.app/universe/search/?category=paper}{Paper}
\end{itemize}
\end{description}

\subsubsection{Where to report issues?}\label{where-to-report-issues}

This template is a project of James R. Swift . Report issues on
\href{https://github.com/JamesxX/graceful-genetics}{their repository} .
You can also try to ask for help with this template on the
\href{https://forum.typst.app}{Forum} .

Please report this template to the Typst team using the
\href{https://typst.app/contact}{contact form} if you believe it is a
safety hazard or infringes upon your rights.

\phantomsection\label{versions}
\subsubsection{Version history}\label{version-history}

\begin{longtable}[]{@{}ll@{}}
\toprule\noalign{}
Version & Release Date \\
\midrule\noalign{}
\endhead
\bottomrule\noalign{}
\endlastfoot
0.2.0 & October 30, 2024 \\
\href{https://typst.app/universe/package/graceful-genetics/0.1.0/}{0.1.0}
& July 16, 2024 \\
\end{longtable}

Typst GmbH did not create this template and cannot guarantee correct
functionality of this template or compatibility with any version of the
Typst compiler or app.
