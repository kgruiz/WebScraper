\title{typst.app/universe/package/genealotree}

\phantomsection\label{banner}
\section{genealotree}\label{genealotree}

{ 0.1.0 }

A package to draw genealogical trees, based on CeTZ

\phantomsection\label{readme}
Genealotree is a typst package to draw genealogical trees. It is
developped at \url{https://codeberg.org/drloiseau/genealogy} . This is
the place you can get the developpement version and send issues and pull
requests.

\pandocbounded{\includegraphics[keepaspectratio]{https://github.com/typst/packages/raw/main/packages/preview/genealotree/0.1.0/examples/example.jpg}}

This package is based on
\href{https://github.com/typst/packages/raw/main/packages/preview/genealotree/0.1.0/\%22https://typst.app/universe/package/cetz/\%22}{CeTZ}
and it provides functions to draw genealogical trees. It is oriented
towards medical genealogy to study genetic disorders inheritance, but
you might be able to use it to draw your family tree.

\textbf{Features :}

\begin{itemize}
\tightlist
\item
  Draw an unlimited number of independant genealogical trees
\item
  Supports consanguinity and unions between different trees (see
  limitations)
\item
  Auto adjusts position of children to optimize spacing
\item
  Customize all lengths
\item
  Draw as much phenotypes as needed by coloring individuals
\item
  Print genotype and/or phenotype labels under individuals
\end{itemize}

\textbf{Limitations :}

\begin{itemize}
\tightlist
\item
  Must manually adjust individual position in the tree when drawing
  consanguinity and unions between trees to prevent overlapping of
  individuals.
\item
  No remarriages (might be added in a future version)
\item
  No union between individuals at different generations (might be added
  in a future version)
\end{itemize}

\textbf{To be implemented :}

\begin{itemize}
\tightlist
\item
  Allow to pass CeTZ arguments to every elements to cutomize their
  appearance
\item
  Draw optional legends for tree symbols and for phenotypes
\end{itemize}

See example.typ for a simple usage example, and the manual for precise
references at
\href{https://codeberg.org/attachments/cfdad2b7-52ae-4e18-8d7b-453fadc45532}{manual.pdf}

The steps to produce a tree are :

\begin{itemize}
\tightlist
\item
  Import the package and CeTZ
\end{itemize}

\begin{Shaded}
\begin{Highlighting}[]
\NormalTok{\#import "@preview/genealotree:0.1.0": *}
\NormalTok{\#import "@preview/cetz:0.2.2": canvas}
\end{Highlighting}
\end{Shaded}

\begin{itemize}
\tightlist
\item
  Create a genealogy object
\end{itemize}

\begin{Shaded}
\begin{Highlighting}[]
\NormalTok{let my{-}geneal = genealogy{-}init()}
\end{Highlighting}
\end{Shaded}

\begin{itemize}
\tightlist
\item
  Add persons to the object : pass a dictionary mapping a persons name
  with a dictionary describing its characteristics. See the manual for a
  full reference.
\end{itemize}

\begin{Shaded}
\begin{Highlighting}[]
\NormalTok{let my{-}geneal = add{-}persons(}
\NormalTok{  my{-}geneal,}
\NormalTok{  (}
\NormalTok{    "I1": (sex: "m"),}
\NormalTok{    "I2": (sex: "f"),}
\NormalTok{    "II1": (sex: "f"),}
\NormalTok{  )}
\NormalTok{)}
\end{Highlighting}
\end{Shaded}

\begin{itemize}
\tightlist
\item
  Set unions between persons : give the parents names as an array of 2
  strings, and the children names as an array of strings.
\end{itemize}

\begin{Shaded}
\begin{Highlighting}[]
\NormalTok{let my{-}geneal = add{-}unions(}
\NormalTok{  my{-}geneal,}
\NormalTok{  (("I1", "I2"), ("II1",))}
\NormalTok{)}
\end{Highlighting}
\end{Shaded}

\begin{itemize}
\tightlist
\item
  Open up a CeTZ canva and draw the tree
\end{itemize}

\begin{Shaded}
\begin{Highlighting}[]
\NormalTok{\#canvas(length: 0.4cm, \{}
\NormalTok{    draw{-}tree(my{-}geneal)}
\NormalTok{\})}
\end{Highlighting}
\end{Shaded}

\subsubsection{How to add}\label{how-to-add}

Copy this into your project and use the import as
\texttt{\ genealotree\ }

\begin{verbatim}
#import "@preview/genealotree:0.1.0"
\end{verbatim}

\includesvg[width=0.16667in,height=0.16667in]{/assets/icons/16-copy.svg}

Check the docs for
\href{https://typst.app/docs/reference/scripting/\#packages}{more
information on how to import packages} .

\subsubsection{About}\label{about}

\begin{description}
\tightlist
\item[Author :]
DrLoiseau
\item[License:]
GPL-3.0-only
\item[Current version:]
0.1.0
\item[Last updated:]
May 23, 2024
\item[First released:]
May 23, 2024
\item[Minimum Typst version:]
0.10.0
\item[Archive size:]
22.9 kB
\href{https://packages.typst.org/preview/genealotree-0.1.0.tar.gz}{\pandocbounded{\includesvg[keepaspectratio]{/assets/icons/16-download.svg}}}
\item[Repository:]
\href{https://codeberg.org/drloiseau/genealogy}{Codeberg}
\item[Discipline s :]
\begin{itemize}
\tightlist
\item[]
\item
  \href{https://typst.app/universe/search/?discipline=anthropology}{Anthropology}
\item
  \href{https://typst.app/universe/search/?discipline=biology}{Biology}
\item
  \href{https://typst.app/universe/search/?discipline=history}{History}
\item
  \href{https://typst.app/universe/search/?discipline=medicine}{Medicine}
\end{itemize}
\item[Categor y :]
\begin{itemize}
\tightlist
\item[]
\item
  \pandocbounded{\includesvg[keepaspectratio]{/assets/icons/16-chart.svg}}
  \href{https://typst.app/universe/search/?category=visualization}{Visualization}
\end{itemize}
\end{description}

\subsubsection{Where to report issues?}\label{where-to-report-issues}

This package is a project of DrLoiseau . Report issues on
\href{https://codeberg.org/drloiseau/genealogy}{their repository} . You
can also try to ask for help with this package on the
\href{https://forum.typst.app}{Forum} .

Please report this package to the Typst team using the
\href{https://typst.app/contact}{contact form} if you believe it is a
safety hazard or infringes upon your rights.

\phantomsection\label{versions}
\subsubsection{Version history}\label{version-history}

\begin{longtable}[]{@{}ll@{}}
\toprule\noalign{}
Version & Release Date \\
\midrule\noalign{}
\endhead
\bottomrule\noalign{}
\endlastfoot
0.1.0 & May 23, 2024 \\
\end{longtable}

Typst GmbH did not create this package and cannot guarantee correct
functionality of this package or compatibility with any version of the
Typst compiler or app.
