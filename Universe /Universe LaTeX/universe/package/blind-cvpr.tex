\title{typst.app/universe/package/blind-cvpr}

\phantomsection\label{banner}
\phantomsection\label{template-thumbnail}
\pandocbounded{\includegraphics[keepaspectratio]{https://packages.typst.org/preview/thumbnails/blind-cvpr-0.5.0-small.webp}}

\section{blind-cvpr}\label{blind-cvpr}

{ 0.5.0 }

CVPR-style paper template to publish at the Computer Vision and Pattern
Recognition (CVPR) conferences.

\href{/app?template=blind-cvpr&version=0.5.0}{Create project in app}

\phantomsection\label{readme}
\subsection{Usage}\label{usage}

You can use this template in the Typst web app by clicking \emph{Start
from template} on the dashboard and searching for
\texttt{\ blind-cvpr\ } .

Alternatively, you can use the CLI to kick this project off using the
command

\begin{Shaded}
\begin{Highlighting}[]
\NormalTok{typst init @preview/blind{-}cvpr}
\end{Highlighting}
\end{Shaded}

Typst will create a new directory with all the files needed to get you
started.

\subsection{Configuration}\label{configuration}

This template exports the \texttt{\ cvpr2022\ } and
\texttt{\ cvpr2025\ } styling rule with the following named arguments.

\begin{itemize}
\tightlist
\item
  \texttt{\ title\ } : The paper’s title as content.
\item
  \texttt{\ authors\ } : An array of author dictionaries. Each of the
  author dictionaries must have a name key and can have the keys
  department, organization, location, and email.
\item
  \texttt{\ keywords\ } : Publication keywords (used in PDF metadata).
\item
  \texttt{\ date\ } : Creation date (used in PDF metadata).
\item
  \texttt{\ abstract\ } : The content of a brief summary of the paper or
  none. Appears at the top under the title.
\item
  \texttt{\ bibliography\ } : The result of a call to the bibliography
  function or none. The function also accepts a single, positional
  argument for the body of the paper.
\item
  \texttt{\ appendix\ } : Content to append after bibliography section.
\item
  \texttt{\ accepted\ } : If this is set to \texttt{\ false\ } then
  anonymized ready for submission document is produced;
  \texttt{\ accepted:\ true\ } produces camera-redy version. If the
  argument is set to \texttt{\ none\ } then preprint version is produced
  (can be uploaded to arXiv).
\item
  \texttt{\ id\ } : Identifier of a submission.
\end{itemize}

The template will initialize your package with a sample call to the
\texttt{\ cvpr2025\ } function in a show rule. If you want to change an
existing project to use this template, you can add a show rule at the
top of your file.

\begin{Shaded}
\begin{Highlighting}[]
\NormalTok{\#import "@preview/blind{-}cvpr:0.5.0": cvpr2025}

\NormalTok{\#show: cvpr2025.with(}
\NormalTok{  title: [LaTeX Author Guidelines for CVPR Proceedings],}
\NormalTok{  authors: (authors, affls),}
\NormalTok{  keywords: (),}
\NormalTok{  abstract: [}
\NormalTok{    The ABSTRACT is to be in fully justified italicized text, at the top of the}
\NormalTok{    left{-}hand column, below the author and affiliation information. Use the}
\NormalTok{    word "Abstract" as the title, in 12{-}point Times, boldface type, centered}
\NormalTok{    relative to the column, initially capitalized. The abstract is to be in}
\NormalTok{    10{-}point, single{-}spaced type. Leave two blank lines after the Abstract,}
\NormalTok{    then begin the main text. Look at previous CVPR abstracts to get a feel for}
\NormalTok{    style and length.}
\NormalTok{  ],}
\NormalTok{  bibliography: bibliography("main.bib"),}
\NormalTok{  accepted: false,}
\NormalTok{  id: none,}
\NormalTok{)}
\end{Highlighting}
\end{Shaded}

\subsection{Issues}\label{issues}

\begin{itemize}
\item
  In case of US Letter, column sizes + gap does not equals to text width
  (2 * 3.25 + 5/16 != 6 + 7/8). It seems that correct gap should be 3/8.
\item
  At the moment of Typst v0.11.0, it is impossible to indent the first
  paragraph in a section (see
  \href{https://github.com/typst/typst/issues/311}{typst/typst\#311} ).
  The workaround is to add indentation manually as follows.

\begin{Shaded}
\begin{Highlighting}[]
\NormalTok{== H2}

\NormalTok{\#h(12pt)  Manually as space for the first paragraph.}
\NormalTok{Lorem ipsum dolor sit amet, consectetur adipiscing elit, sed do.}

\NormalTok{// The second one is just fine.}
\NormalTok{Lorem ipsum dolor sit amet, consectetur adipiscing elit, sed do.}
\end{Highlighting}
\end{Shaded}

  Also, we add \texttt{\ indent\ } constant as a shortcut for
  \texttt{\ h(12pt)\ } .

  This issue is relevant to CVPR 2022. In the 2025 template there is no
  indentaino of the first paragraph in section.
\item
  At the moment Typst v0.11.0 does not allow flexible customization of
  citation styles. Specifically, CVPR 2022 citation lookes like
  \texttt{\ {[}42{]}\ } where number is colored hyperlink. In order to
  achive this, we shouuld provide custom CSL-style and then colorize
  number and put it into square parenthesis in typst markup.
\item
  CVPR 2022 requires simple ruler which enumerates lines in regular
  intervals whilst CVPR2025 already requires a ruler which add line
  numers per line in paragraph or heading. Thus we need the next major
  Typst release v0.12.0 for ruler. With the next Typst release, we can
  do the following.

\begin{Shaded}
\begin{Highlighting}[]
\NormalTok{set par.line(numbering: "1")}
\NormalTok{show figure: set par.line(numbering: none)}
\end{Highlighting}
\end{Shaded}

  For implementation details see
  \href{https://github.com/typst/typst/pull/4516}{typst/typst\#4516} .
\item
  CVPR 2022 and 2025 requires IEEE-like bibliography style but does not
  follow its guidelines closely. Since writing CSL-style files is
  tedious task, we adopt close enough bibliography style from Zotero.
\end{itemize}

\subsection{References}\label{references}

\begin{itemize}
\tightlist
\item
  CVPR 2022 conference
  \href{https://cvpr2022.thecvf.com/author-guidelines\#dates}{web site}
  .
\item
  CVPR 2025 conference
  \href{https://cvpr.thecvf.com/Conferences/2025}{web site} .
\end{itemize}

\href{/app?template=blind-cvpr&version=0.5.0}{Create project in app}

\subsubsection{How to use}\label{how-to-use}

Click the button above to create a new project using this template in
the Typst app.

You can also use the Typst CLI to start a new project on your computer
using this command:

\begin{verbatim}
typst init @preview/blind-cvpr:0.5.0
\end{verbatim}

\includesvg[width=0.16667in,height=0.16667in]{/assets/icons/16-copy.svg}

\subsubsection{About}\label{about}

\begin{description}
\tightlist
\item[Author :]
daskol
\item[License:]
MIT
\item[Current version:]
0.5.0
\item[Last updated:]
September 22, 2024
\item[First released:]
September 22, 2024
\item[Minimum Typst version:]
0.11.1
\item[Archive size:]
18.3 kB
\href{https://packages.typst.org/preview/blind-cvpr-0.5.0.tar.gz}{\pandocbounded{\includesvg[keepaspectratio]{/assets/icons/16-download.svg}}}
\item[Repository:]
\href{https://github.com/daskol/typst-templates}{GitHub}
\item[Discipline s :]
\begin{itemize}
\tightlist
\item[]
\item
  \href{https://typst.app/universe/search/?discipline=computer-science}{Computer
  Science}
\item
  \href{https://typst.app/universe/search/?discipline=mathematics}{Mathematics}
\end{itemize}
\item[Categor y :]
\begin{itemize}
\tightlist
\item[]
\item
  \pandocbounded{\includesvg[keepaspectratio]{/assets/icons/16-atom.svg}}
  \href{https://typst.app/universe/search/?category=paper}{Paper}
\end{itemize}
\end{description}

\subsubsection{Where to report issues?}\label{where-to-report-issues}

This template is a project of daskol . Report issues on
\href{https://github.com/daskol/typst-templates}{their repository} . You
can also try to ask for help with this template on the
\href{https://forum.typst.app}{Forum} .

Please report this template to the Typst team using the
\href{https://typst.app/contact}{contact form} if you believe it is a
safety hazard or infringes upon your rights.

\phantomsection\label{versions}
\subsubsection{Version history}\label{version-history}

\begin{longtable}[]{@{}ll@{}}
\toprule\noalign{}
Version & Release Date \\
\midrule\noalign{}
\endhead
\bottomrule\noalign{}
\endlastfoot
0.5.0 & September 22, 2024 \\
\end{longtable}

Typst GmbH did not create this template and cannot guarantee correct
functionality of this template or compatibility with any version of the
Typst compiler or app.
