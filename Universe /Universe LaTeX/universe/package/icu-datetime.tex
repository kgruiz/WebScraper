\title{typst.app/universe/package/icu-datetime}

\phantomsection\label{banner}
\section{icu-datetime}\label{icu-datetime}

{ 0.1.2 }

Date and time formatting using ICU4X via WASM

\phantomsection\label{readme}
This library is a wrapper around
\href{https://github.com/unicode-org/icu4x}{ICU4X} ’
\texttt{\ datetime\ } formatting for Typst which provides
internationalized formatting for dates, times, and timezones.

As the WASM bundle includes all localization data, it’s quite large
(about 8 MiB).

See \href{https://nerixyz.github.io/icu-typ}{nerixyz.github.io/icu-typ}
for a full API reference with more examples.

\subsection{Example}\label{example}

\begin{Shaded}
\begin{Highlighting}[]
\NormalTok{\#import "@preview/icu{-}datetime:0.1.2": fmt{-}date, fmt{-}time, fmt{-}datetime, experimental}
\NormalTok{// These functions may change at any time}
\NormalTok{\#import experimental: fmt{-}timezone, fmt{-}zoned{-}datetime}

\NormalTok{\#let day = datetime(}
\NormalTok{  year: 2024,}
\NormalTok{  month: 5,}
\NormalTok{  day: 31,}
\NormalTok{)}
\NormalTok{\#let time = datetime(}
\NormalTok{  hour: 18,}
\NormalTok{  minute: 2,}
\NormalTok{  second: 23,}
\NormalTok{)}
\NormalTok{\#let dt = datetime(}
\NormalTok{  year: 2024,}
\NormalTok{  month: 5,}
\NormalTok{  day: 31,}
\NormalTok{  hour: 18,}
\NormalTok{  minute: 2,}
\NormalTok{  second: 23,}
\NormalTok{)}
\NormalTok{\#let tz = (}
\NormalTok{  offset: "{-}07",}
\NormalTok{  iana: "America/Los\_Angeles",}
\NormalTok{  zone{-}variant: "st", // standard}
\NormalTok{)}

\NormalTok{= Dates}
\NormalTok{\#fmt{-}date(day, locale: "km", length: "full") \textbackslash{}}
\NormalTok{\#fmt{-}date(day, locale: "af", length: "full") \textbackslash{}}
\NormalTok{\#fmt{-}date(day, locale: "za", length: "full") \textbackslash{}}

\NormalTok{= Time}
\NormalTok{\#fmt{-}time(time, locale: "id", length: "medium") \textbackslash{}}
\NormalTok{\#fmt{-}time(time, locale: "en", length: "medium") \textbackslash{}}
\NormalTok{\#fmt{-}time(time, locale: "ga", length: "medium") \textbackslash{}}

\NormalTok{= Date and Time}
\NormalTok{\#fmt{-}datetime(dt, locale: "ru", date{-}length: "full") \textbackslash{}}
\NormalTok{\#fmt{-}datetime(dt, locale: "en{-}US", date{-}length: "full") \textbackslash{}}
\NormalTok{\#fmt{-}datetime(dt, locale: "zh{-}Hans{-}CN", date{-}length: "full") \textbackslash{}}
\NormalTok{\#fmt{-}datetime(dt, locale: "ar", date{-}length: "full") \textbackslash{}}
\NormalTok{\#fmt{-}datetime(dt, locale: "fi", date{-}length: "full")}

\NormalTok{= Timezones (experimental)}
\NormalTok{\#fmt{-}timezone(}
\NormalTok{  ..tz,}
\NormalTok{  local{-}date: datetime.today(),}
\NormalTok{  format: "specific{-}non{-}location{-}long"}
\NormalTok{) \textbackslash{}}
\NormalTok{\#fmt{-}timezone(}
\NormalTok{  ..tz,}
\NormalTok{  format: (}
\NormalTok{    iso8601: (}
\NormalTok{      format: "utc{-}extended",}
\NormalTok{      minutes: "required",}
\NormalTok{      seconds: "optional",}
\NormalTok{    )}
\NormalTok{  )}
\NormalTok{)}

\NormalTok{= Zoned Datetimes (experimental)}
\NormalTok{\#fmt{-}zoned{-}datetime(dt, tz) \textbackslash{}}
\NormalTok{\#fmt{-}zoned{-}datetime(dt, tz, locale: "lv") \textbackslash{}}
\NormalTok{\#fmt{-}zoned{-}datetime(dt, tz, locale: "en{-}CA{-}u{-}hc{-}h24{-}ca{-}buddhist")}
\end{Highlighting}
\end{Shaded}

\pandocbounded{\includegraphics[keepaspectratio]{https://github.com/typst/packages/raw/main/packages/preview/icu-datetime/0.1.2/res/example.png}}

Locales must be
\href{https://unicode.org/reports/tr35/tr35.html\#Unicode_locale_identifier}{Unicode
Locale Identifier} s. Use
\href{https://nerixyz.github.io/icu-typ/locale-info/}{\texttt{\ locale-info(locale)\ }}
to get information on how a locale is parsed. Unicode extensions are
supported, so you can, for example, set the hour cycle with
\texttt{\ hc-h12\ } or set the calendar with \texttt{\ ca-buddhist\ }
(e.g. \texttt{\ en-CA-u-hc-h24-ca-buddhist\ } ).

\subsection{Documentation}\label{documentation}

Documentation can be found on
\href{https://nerixyz.github.io/icu-typ}{nerixyz.github.io/icu-typ} .

\subsection{Using Locally}\label{using-locally}

Download the \href{https://github.com/Nerixyz/icu-typ/releases}{latest
release} , unzip it to your
\href{https://github.com/typst/packages\#local-packages}{local Typst
packages} , and use \texttt{\ \#import\ "@local/icu-datetime:0.1.2"\ } .

\subsection{Building}\label{building}

To build the library, you need to have
\href{https://www.rust-lang.org/}{Rust} ,
\href{https://just.systems/}{just} , and
\href{https://github.com/WebAssembly/binaryen}{\texttt{\ wasm-opt\ }}
installed.

\begin{Shaded}
\begin{Highlighting}[]
\ExtensionTok{just}\NormalTok{ build}
\CommentTok{\# to deploy the package locally, use \textasciigrave{}just deploy\textasciigrave{}}
\end{Highlighting}
\end{Shaded}

While developing, you can symlink the WASM file into the root of the
repository (it’s in the \texttt{\ .gitignore\ } ):

\begin{Shaded}
\begin{Highlighting}[]
\CommentTok{\# Windows (PowerShell)}
\ExtensionTok{New{-}Item}\NormalTok{ icu{-}datetime.wasm }\AttributeTok{{-}ItemType}\NormalTok{ SymbolicLink }\AttributeTok{{-}Value}\NormalTok{ ./target/wasm32{-}unknown{-}unknown/debug/icu\_typ.wasm}
\CommentTok{\# Unix}
\FunctionTok{ln} \AttributeTok{{-}s}\NormalTok{ ./target/wasm32{-}unknown{-}unknown/debug/icu\_typ.wasm icu{-}datetime.wasm}
\end{Highlighting}
\end{Shaded}

Use \texttt{\ cargo\ b\ -\/-target\ wasm32-unknown-unknown\ } to build
in debug mode.

\subsubsection{How to add}\label{how-to-add}

Copy this into your project and use the import as
\texttt{\ icu-datetime\ }

\begin{verbatim}
#import "@preview/icu-datetime:0.1.2"
\end{verbatim}

\includesvg[width=0.16667in,height=0.16667in]{/assets/icons/16-copy.svg}

Check the docs for
\href{https://typst.app/docs/reference/scripting/\#packages}{more
information on how to import packages} .

\subsubsection{About}\label{about}

\begin{description}
\tightlist
\item[Author :]
\href{https://github.com/Nerixyz}{Nerixyz}
\item[License:]
MIT
\item[Current version:]
0.1.2
\item[Last updated:]
June 14, 2024
\item[First released:]
June 3, 2024
\item[Minimum Typst version:]
0.11.0
\item[Archive size:]
1.55 MB
\href{https://packages.typst.org/preview/icu-datetime-0.1.2.tar.gz}{\pandocbounded{\includesvg[keepaspectratio]{/assets/icons/16-download.svg}}}
\item[Repository:]
\href{https://github.com/Nerixyz/icu-typ}{GitHub}
\item[Categor y :]
\begin{itemize}
\tightlist
\item[]
\item
  \pandocbounded{\includesvg[keepaspectratio]{/assets/icons/16-world.svg}}
  \href{https://typst.app/universe/search/?category=languages}{Languages}
\end{itemize}
\end{description}

\subsubsection{Where to report issues?}\label{where-to-report-issues}

This package is a project of Nerixyz . Report issues on
\href{https://github.com/Nerixyz/icu-typ}{their repository} . You can
also try to ask for help with this package on the
\href{https://forum.typst.app}{Forum} .

Please report this package to the Typst team using the
\href{https://typst.app/contact}{contact form} if you believe it is a
safety hazard or infringes upon your rights.

\phantomsection\label{versions}
\subsubsection{Version history}\label{version-history}

\begin{longtable}[]{@{}ll@{}}
\toprule\noalign{}
Version & Release Date \\
\midrule\noalign{}
\endhead
\bottomrule\noalign{}
\endlastfoot
0.1.2 & June 14, 2024 \\
\href{https://typst.app/universe/package/icu-datetime/0.1.1/}{0.1.1} &
June 10, 2024 \\
\href{https://typst.app/universe/package/icu-datetime/0.1.0/}{0.1.0} &
June 3, 2024 \\
\end{longtable}

Typst GmbH did not create this package and cannot guarantee correct
functionality of this package or compatibility with any version of the
Typst compiler or app.
