\title{typst.app/universe/package/fontawesome}

\phantomsection\label{banner}
\section{fontawesome}\label{fontawesome}

{ 0.5.0 }

A Typst library for Font Awesome icons through the desktop fonts.

\phantomsection\label{readme}
A Typst library for Font Awesome icons through the desktop fonts.

p.s. The library is based on the Font Awesome 6 desktop fonts (v6.6.0)

\subsection{Usage}\label{usage}

\subsubsection{Install the fonts}\label{install-the-fonts}

You can download the fonts from the official website:
\url{https://fontawesome.com/download}

After downloading the zip file, you can install the fonts depending on
your OS.

\paragraph{Typst web app}\label{typst-web-app}

You can simply upload the \texttt{\ otf\ } files to the web app and use
them with this package.

\paragraph{Mac}\label{mac}

You can double click the \texttt{\ otf\ } files to install them.

\paragraph{Windows}\label{windows}

You can right-click the \texttt{\ otf\ } files and select
\texttt{\ Install\ } .

\paragraph{Some notes}\label{some-notes}

This library is tested with the otf files of the Font Awesome Free set.
TrueType fonts may not work as expected. (Though I am not sure whether
Font Awesome provides TrueType fonts, some issue is reported with
TrueType fonts.)

\subsubsection{Import the library}\label{import-the-library}

\paragraph{Using the typst packages}\label{using-the-typst-packages}

You can install the library using the typst packages:

\texttt{\ \#import\ "@preview/fontawesome:0.5.0":\ *\ }

\paragraph{Manually install}\label{manually-install}

Copy all files start with \texttt{\ lib\ } to your project and import
the library:

\texttt{\ \#import\ "lib.typ":\ *\ }

There are three files:

\begin{itemize}
\tightlist
\item
  \texttt{\ lib.typ\ } : The main entrypoint of the library.
\item
  \texttt{\ lib-impl.typ\ } : The implementation of \texttt{\ fa-icon\ }
  .
\item
  \texttt{\ lib-gen.typ\ } : The generated icon map and functions.
\end{itemize}

I recommend renaming these files to avoid conflicts with other
libraries.

\subsubsection{Use the icons}\label{use-the-icons}

You can use the \texttt{\ fa-icon\ } function to create an icon with its
name:

\texttt{\ \#fa-icon("chess-queen")\ }

Or you can use the \texttt{\ fa-\ } prefix to create an icon with its
name:

\texttt{\ \#fa-chess-queen()\ } (This is equivalent to
\texttt{\ \#fa-icon().with("chess-queen")\ } )

You can also set \texttt{\ solid\ } to \texttt{\ true\ } to use the
solid version of the icon:

\texttt{\ \#fa-icon("chess-queen",\ solid:\ true)\ }

Some icons only have the solid version in the Free set, so you need to
set \texttt{\ solid\ } to \texttt{\ true\ } to use them if you are using
the Free set. Otherwise, you may not get the expected glyph.

\paragraph{Full list of icons}\label{full-list-of-icons}

You can find all icons on the
\href{https://fontawesome.com/search}{official website}

\paragraph{Different sets}\label{different-sets}

By default, the library supports \texttt{\ Free\ } , \texttt{\ Brands\ }
, \texttt{\ Pro\ } , \texttt{\ Duotone\ } and \texttt{\ Sharp\ } sets.
(See
\href{https://github.com/typst/packages/raw/main/packages/preview/fontawesome/0.5.0/\#enable-pro-sets}{Enable
Pro sets} for enabling Pro sets.)

But only \texttt{\ Free\ } and \texttt{\ Brands\ } are tested by me.
That is, three font files are used to test:

\begin{itemize}
\tightlist
\item
  Font Awesome 6 Free (Also named as \emph{Font Awesome 6 Free Regular}
  )
\item
  Font Awesome 6 Free Solid
\item
  Font Awesome 6 Brands
\end{itemize}

Due to some limitations of typst 0.12.0, the regular and solid versions
are treated as different fonts. In this library, \texttt{\ solid\ } is
used to switch between the regular and solid versions.

To use other sets or specify one set, you can pass the \texttt{\ font\ }
parameter to the inner \texttt{\ text\ } function:\\
\texttt{\ fa-icon("github",\ font:\ "Font\ Awesome\ 6\ Pro\ Solid")\ }

If you have Font Awesome Pro, please help me test the library with the
Pro set. Any feedback is appreciated.

\subparagraph{Enable Pro sets}\label{enable-pro-sets}

Typst 0.12.0 raise a warning when the font is not found. To use the Pro
set, \texttt{\ \#fa-use-pro()\ } should be called before any
\texttt{\ fa-*\ } functions.

\begin{Shaded}
\begin{Highlighting}[]
\NormalTok{\#fa{-}use{-}pro()                 // Enable Pro sets}

\NormalTok{\#fa{-}icon("chess{-}queen{-}piece") // Use icons from Pro sets}
\end{Highlighting}
\end{Shaded}

\paragraph{Customization}\label{customization}

The \texttt{\ fa-icon\ } function passes args to \texttt{\ text\ } , so
you can customize the icon by passing parameters to it:

\texttt{\ \#fa-icon("chess-queen",\ fill:\ blue)\ }

\paragraph{Stacking icons}\label{stacking-icons}

The \texttt{\ fa-stack\ } function can be used to create stacked icons:

\texttt{\ \#fa-stack(fa-icon-args:\ (solid:\ true),\ "square",\ ("chess-queen",\ (fill:\ white,\ size:\ 5.5pt)))\ }

Declaration is
\texttt{\ fa-stack(box-args:\ (:),\ grid-args:\ (:),\ fa-icon-args:\ (:),\ ..icons)\ }

\begin{itemize}
\tightlist
\item
  The order of the icons is from the bottom to the top.
\item
  \texttt{\ fa-icon-args\ } is used to set the default args for all
  icons.
\item
  You can also control the internal \texttt{\ box\ } and
  \texttt{\ grid\ } by passing the \texttt{\ box-args\ } and
  \texttt{\ grid-args\ } to the \texttt{\ fa-stack\ } function.
\item
  Currently, four types of icons are supported. The first three types
  leverage the \texttt{\ fa-icon\ } function, and the last type is just
  a content you want to put in the stack.

  \begin{itemize}
  \tightlist
  \item
    \texttt{\ str\ } , e.g., \texttt{\ "square"\ }
  \item
    \texttt{\ array\ } , e.g.,
    \texttt{\ ("chess-queen",\ (fill:\ white,\ size:\ 5.5pt))\ }
  \item
    \texttt{\ arguments\ } , e.g.
    \texttt{\ arguments("chess-queen",\ solid:\ true,\ fill:\ white)\ }
  \item
    \texttt{\ content\ } , e.g.
    \texttt{\ fa-chess-queen(solid:\ true,\ fill:\ white)\ }
  \end{itemize}
\end{itemize}

\paragraph{Known Issues}\label{known-issues}

\begin{itemize}
\item
  \href{https://github.com/typst/typst/issues/2578}{typst\#2578}
  \href{https://github.com/duskmoon314/typst-fontawesome/issues/2}{typst-fontawesome\#2}

  This is a known issue that the ligatures may not work in headings,
  list items, grid items, and other elements. You can use the Unicode
  from the \href{https://fontawesome.com/}{official website} to avoid
  this issue when using Pro sets.

  For most icons, Unicode is used implicitly. So I assume we usually
  don’t need to worry about this.

  Any help on this issue is appreciated.
\end{itemize}

\subsection{Example}\label{example}

See the
\href{https://typst.app/project/rQwGUWt5p33vrsb_uNPR9F}{\texttt{\ example.typ\ }}
file for a complete example.

\subsection{Contribution}\label{contribution}

Feel free to open an issue or a pull request if you find any problems or
have any suggestions.

\subsubsection{Python helper}\label{python-helper}

The \texttt{\ helper.py\ } script is used to get metadata via the
GraphQL API and generate typst code. I aim only to use standard python
libraries, so running it on any platform with python installed should be
easy.

\subsubsection{Repo structure}\label{repo-structure}

\begin{itemize}
\tightlist
\item
  \texttt{\ helper.py\ } : The helper script to get metadata and
  generate typst code.
\item
  \texttt{\ lib.typ\ } : The main entrypoint of the library.
\item
  \texttt{\ lib-impl.typ\ } : The implementation of \texttt{\ fa-icon\ }
  .
\item
  \texttt{\ lib-gen.typ\ } : The generated functions of icons.
\item
  \texttt{\ example.typ\ } : An example file to show how to use the
  library.
\item
  \texttt{\ gallery.typ\ } : The generated gallery of icons. It is used
  in the example file.
\end{itemize}

\subsection{License}\label{license}

This library is licensed under the MIT license. Feel free to use it in
your project.

\subsubsection{How to add}\label{how-to-add}

Copy this into your project and use the import as
\texttt{\ fontawesome\ }

\begin{verbatim}
#import "@preview/fontawesome:0.5.0"
\end{verbatim}

\includesvg[width=0.16667in,height=0.16667in]{/assets/icons/16-copy.svg}

Check the docs for
\href{https://typst.app/docs/reference/scripting/\#packages}{more
information on how to import packages} .

\subsubsection{About}\label{about}

\begin{description}
\tightlist
\item[Author :]
\href{mailto:kp.campbell.he@duskmoon314.com}{duskmoon (Campbell He)}
\item[License:]
MIT
\item[Current version:]
0.5.0
\item[Last updated:]
October 21, 2024
\item[First released:]
July 3, 2023
\item[Archive size:]
74.7 kB
\href{https://packages.typst.org/preview/fontawesome-0.5.0.tar.gz}{\pandocbounded{\includesvg[keepaspectratio]{/assets/icons/16-download.svg}}}
\item[Repository:]
\href{https://github.com/duskmoon314/typst-fontawesome}{GitHub}
\end{description}

\subsubsection{Where to report issues?}\label{where-to-report-issues}

This package is a project of duskmoon (Campbell He) . Report issues on
\href{https://github.com/duskmoon314/typst-fontawesome}{their
repository} . You can also try to ask for help with this package on the
\href{https://forum.typst.app}{Forum} .

Please report this package to the Typst team using the
\href{https://typst.app/contact}{contact form} if you believe it is a
safety hazard or infringes upon your rights.

\phantomsection\label{versions}
\subsubsection{Version history}\label{version-history}

\begin{longtable}[]{@{}ll@{}}
\toprule\noalign{}
Version & Release Date \\
\midrule\noalign{}
\endhead
\bottomrule\noalign{}
\endlastfoot
0.5.0 & October 21, 2024 \\
\href{https://typst.app/universe/package/fontawesome/0.4.0/}{0.4.0} &
August 1, 2024 \\
\href{https://typst.app/universe/package/fontawesome/0.3.0/}{0.3.0} &
July 23, 2024 \\
\href{https://typst.app/universe/package/fontawesome/0.2.1/}{0.2.1} &
June 17, 2024 \\
\href{https://typst.app/universe/package/fontawesome/0.2.0/}{0.2.0} &
April 19, 2024 \\
\href{https://typst.app/universe/package/fontawesome/0.1.1/}{0.1.1} &
April 1, 2024 \\
\href{https://typst.app/universe/package/fontawesome/0.1.0/}{0.1.0} &
July 3, 2023 \\
\end{longtable}

Typst GmbH did not create this package and cannot guarantee correct
functionality of this package or compatibility with any version of the
Typst compiler or app.
