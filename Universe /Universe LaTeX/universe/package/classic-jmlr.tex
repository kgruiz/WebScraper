\title{typst.app/universe/package/classic-jmlr}

\phantomsection\label{banner}
\phantomsection\label{template-thumbnail}
\pandocbounded{\includegraphics[keepaspectratio]{https://packages.typst.org/preview/thumbnails/classic-jmlr-0.4.0-small.webp}}

\section{classic-jmlr}\label{classic-jmlr}

{ 0.4.0 }

Paper template for submission to Journal of Machine Learning Research
(JMLR)

\href{/app?template=classic-jmlr&version=0.4.0}{Create project in app}

\phantomsection\label{readme}
\subsection{Overview}\label{overview}

This is a Typst template for Journal of Machine Learning Research
(JMLR). It is based on official
\href{https://www.jmlr.org/format/authors-guide.html}{author guide} ,
\href{https://www.jmlr.org/format/format.html}{formatting instructions}
, and
\href{https://www.jmlr.org/format/formatting-errors.html}{formatting
error checklist} as well as the official
\href{https://github.com/jmlrorg/jmlr-style-file}{example paper} .

\subsection{Usage}\label{usage}

You can use this template in the Typst web app by clicking \emph{Start
from template} on the dashboard and searching for
\texttt{\ classic-jmlr\ } .

Alternatively, you can use the CLI to kick this project off using the
command

\begin{Shaded}
\begin{Highlighting}[]
\NormalTok{typst init @preview/classic{-}jmlr}
\end{Highlighting}
\end{Shaded}

Typst will create a new directory with all the files needed to get you
started.

\subsection{Configuration}\label{configuration}

This template exports the \texttt{\ jmlr\ } function with the following
named arguments.

\begin{itemize}
\tightlist
\item
  \texttt{\ title\ } : The paper’s title as content.
\item
  \texttt{\ short-title\ } : Paper short title (for page header).
\item
  \texttt{\ authors\ } : An array of author dictionaries. Each of the
  author dictionaries must have a name key and can have the keys
  department, organization, location, and email.
\item
  \texttt{\ last-names\ } : List of authors last names (for page
  header).
\item
  \texttt{\ keywords\ } : Publication keywords (used in PDF metadata).
\item
  \texttt{\ date\ } : Creation date (used in PDF metadata).
\item
  \texttt{\ abstract\ } : The content of a brief summary of the paper or
  none. Appears at the top under the title.
\item
  \texttt{\ bibliography\ } : The result of a call to the bibliography
  function or none. The function also accepts a single, positional
  argument for the body of the paper.
\item
  \texttt{\ appendix\ } : Content to append after bibliography section.
\item
  \texttt{\ pubdata\ } : Dictionary with auxiliary information about
  publication. It contains editor name(s), paper id, volume, and
  submission/review/publishing dates.
\end{itemize}

The template will initialize your package with a sample call to the
\texttt{\ jmlr\ } function in a show rule. If you want to change an
existing project to use this template, you can add a show rule at the
top of your file.

\begin{Shaded}
\begin{Highlighting}[]
\NormalTok{\#import "@preview/classic{-}jmlr": jmlr}
\NormalTok{\#show: jmlr.with(}
\NormalTok{  title: [Sample JMLR Paper],}
\NormalTok{  authors: (authors, affls),}
\NormalTok{  abstract: blindtext,}
\NormalTok{  keywords: ("keyword one", "keyword two", "keyword three"),}
\NormalTok{  bibliography: bibliography("main.bib"),}
\NormalTok{  appendix: include "appendix.typ",}
\NormalTok{  pubdata: (}
\NormalTok{    id: "21{-}0000",}
\NormalTok{    editor: "My editor",}
\NormalTok{    volume: 23,}
\NormalTok{    submitted{-}at: datetime(year: 2021, month: 1, day: 1),}
\NormalTok{    revised{-}at: datetime(year: 2022, month: 5, day: 1),}
\NormalTok{    published{-}at: datetime(year: 2022, month: 9, day: 1),}
\NormalTok{  ),}
\NormalTok{)}
\end{Highlighting}
\end{Shaded}

\subsection{Issues}\label{issues}

This template is developed at
\href{https://github.com/daskol/typst-templates}{daskol/typst-templates}
repo. Please report all issues there.

\begin{itemize}
\item
  Original JMLR example paper is not not representative. It does not
  demonstrate appearance of figures, images, tables, lists, etc.
\item
  Leading in author affilations in in the original template is varying.
\item
  There is no bibliography CSL-style. The closest one is
  \texttt{\ bristol-university-press\ } .
\item
  Another issue is related to Typst’s inablity to produce colored
  annotation. In order to mitigte the issue, we add a script which
  modifies annotations and make them colored.

\begin{Shaded}
\begin{Highlighting}[]
\NormalTok{../colorize{-}annotations.py \textbackslash{}}
\NormalTok{    example{-}paper.typst.pdf example{-}paper{-}colored.typst.pdf}
\end{Highlighting}
\end{Shaded}

  See
  \href{https://github.com/daskol/typst-templates/\#colored-annotations}{README.md}
  for details.
\end{itemize}

\href{/app?template=classic-jmlr&version=0.4.0}{Create project in app}

\subsubsection{How to use}\label{how-to-use}

Click the button above to create a new project using this template in
the Typst app.

You can also use the Typst CLI to start a new project on your computer
using this command:

\begin{verbatim}
typst init @preview/classic-jmlr:0.4.0
\end{verbatim}

\includesvg[width=0.16667in,height=0.16667in]{/assets/icons/16-copy.svg}

\subsubsection{About}\label{about}

\begin{description}
\tightlist
\item[Author :]
\href{mailto:d.bershatsky2@skoltech.ru}{Daniel Bershatsky}
\item[License:]
MIT
\item[Current version:]
0.4.0
\item[Last updated:]
April 19, 2024
\item[First released:]
April 19, 2024
\item[Minimum Typst version:]
0.11.0
\item[Archive size:]
8.60 kB
\href{https://packages.typst.org/preview/classic-jmlr-0.4.0.tar.gz}{\pandocbounded{\includesvg[keepaspectratio]{/assets/icons/16-download.svg}}}
\item[Repository:]
\href{https://github.com/daskol/typst-templates}{GitHub}
\item[Discipline s :]
\begin{itemize}
\tightlist
\item[]
\item
  \href{https://typst.app/universe/search/?discipline=computer-science}{Computer
  Science}
\item
  \href{https://typst.app/universe/search/?discipline=mathematics}{Mathematics}
\end{itemize}
\item[Categor y :]
\begin{itemize}
\tightlist
\item[]
\item
  \pandocbounded{\includesvg[keepaspectratio]{/assets/icons/16-atom.svg}}
  \href{https://typst.app/universe/search/?category=paper}{Paper}
\end{itemize}
\end{description}

\subsubsection{Where to report issues?}\label{where-to-report-issues}

This template is a project of Daniel Bershatsky . Report issues on
\href{https://github.com/daskol/typst-templates}{their repository} . You
can also try to ask for help with this template on the
\href{https://forum.typst.app}{Forum} .

Please report this template to the Typst team using the
\href{https://typst.app/contact}{contact form} if you believe it is a
safety hazard or infringes upon your rights.

\phantomsection\label{versions}
\subsubsection{Version history}\label{version-history}

\begin{longtable}[]{@{}ll@{}}
\toprule\noalign{}
Version & Release Date \\
\midrule\noalign{}
\endhead
\bottomrule\noalign{}
\endlastfoot
0.4.0 & April 19, 2024 \\
\end{longtable}

Typst GmbH did not create this template and cannot guarantee correct
functionality of this template or compatibility with any version of the
Typst compiler or app.
