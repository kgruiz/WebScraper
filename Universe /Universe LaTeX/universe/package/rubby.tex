\title{typst.app/universe/package/rubby}

\phantomsection\label{banner}
\section{rubby}\label{rubby}

{ 0.10.1 }

Add ruby (furigana) next to base text.

\phantomsection\label{readme}
\subsection{Usage}\label{usage}

\begin{Shaded}
\begin{Highlighting}[]
\NormalTok{\#import "@preview/rubby:0.10.1": get{-}ruby}

\NormalTok{\#let ruby = get{-}ruby(}
\NormalTok{  size: 0.5em,         // Ruby font size}
\NormalTok{  dy: 0pt,             // Vertical offset of the ruby}
\NormalTok{  pos: top,            // Ruby position (top or bottom)}
\NormalTok{  alignment: "center", // Ruby alignment ("center", "start", "between", "around")}
\NormalTok{  delimiter: "|",      // The delimiter between words}
\NormalTok{  auto{-}spacing: true,  // Automatically add necessary space around words}
\NormalTok{)}

\NormalTok{// Ruby goes first, base text {-} second.}
\NormalTok{\#ruby[ふりがな][振り仮名]}

\NormalTok{Treat each kanji as a separate word:}
\NormalTok{\#ruby[とう|きょう|こう|ぎょう|だい|がく][東|京|工|業|大|学]}
\end{Highlighting}
\end{Shaded}

If you don’t want automatically wrap text with delimiter:

\begin{Shaded}
\begin{Highlighting}[]
\NormalTok{\#let ruby = get{-}ruby(auto{-}spacing: false)}
\end{Highlighting}
\end{Shaded}

See also \url{https://github.com/rinmyo/ruby-typ/blob/main/manual.pdf}
and \texttt{\ example.typ\ } .

\subsection{Notes}\label{notes}

Original project is at \url{https://github.com/rinmyo/ruby-typ} which
itself is based on
\href{https://zenn.dev/saito_atsushi/articles/ff9490458570e1}{the post}
of 齊è---¤æ•¦å¿--- (Saito Atsushi). This project is a modified version
of
\href{https://github.com/rinmyo/ruby-typ/commit/23ca86180757cf70f2b9f5851abb5ea5a3b4c6a1}{this
commit} .

\texttt{\ auto-spacing\ } adds missing delimiter around the
\texttt{\ content\ } / \texttt{\ string\ } which then adds space around
base text if ruby is wider than the base text.

Problems appear only if ruby is wider than its base text and
\texttt{\ auto-spacing\ } is not set to \texttt{\ true\ } (default is
\texttt{\ true\ } ).

You can always use a one-letter function (variable) name to shorten the
function call length (if you have to use it a lot), e.g.,
\texttt{\ \#let\ r\ =\ get-ruby()\ } (or \texttt{\ f\ } â€'' short for
furigana). But be careful as there are functions with names
\texttt{\ v\ } and \texttt{\ h\ } and there could be a new built-in
function with a name \texttt{\ r\ } or \texttt{\ f\ } which may break
your document (Typst right now is in beta, so breaking changes are
possible).

Although you can open issues or send PRs, I won’t be able to always
reply quickly (sometimes I’m very busy).

\subsection{Development}\label{development}

This repository should exist as a \texttt{\ @local\ } package with the
version from the \texttt{\ typst.toml\ } .

Here is a short description of the development process:

\begin{enumerate}
\tightlist
\item
  run \texttt{\ git\ checkout\ dev\ \&\&\ git\ pull\ } ;
\item
  make changes;
\item
  test changes, if not done or something isn’t working then go to step
  1;
\item
  when finished, run
  \texttt{\ just\ change-version\ \textless{}new\ semantic\ version\textgreater{}\ }
  ;
\item
  document changes in the \texttt{\ CHANGELOG.md\ } ;
\item
  commit all changes (only locally);
\item
  create a \texttt{\ @local\ } Typst package with the new version and
  test it;
\item
  if everything is working then run \texttt{\ git\ push\ } ;
\item
  realize that you’ve missed something and fix it (then push changes
  again);
\item
  run \texttt{\ git\ checkout\ master\ \&\&\ git\ merge\ dev\ } to sync
  \texttt{\ master\ } to \texttt{\ dev\ } ;
\item
  run \texttt{\ just\ create-release\ } .
\end{enumerate}

\subsection{Publishing a Typst
package}\label{publishing-a-typst-package}

\begin{enumerate}
\tightlist
\item
  To make a new package version for merging into
  \texttt{\ typst/packages\ } repository run
  \texttt{\ just\ mark-PR-version\ } ;
\item
  copy newly created directory (with a version name) and place it in the
  appropriate place in your fork of the \texttt{\ typst/packages\ }
  repository;
\item
  run
  \texttt{\ git\ fetch\ upstream\ \&\&\ git\ merge\ upstream\ main\ } to
  sync fork with \texttt{\ typst/packages\ } ;
\item
  go to a new branch with
  \texttt{\ git\ checkout\ -b\ \textless{}package-version\textgreater{}\ }
  ;
\item
  commit newly added directory with commit message:
  \texttt{\ package:version\ } ;
\item
  run \texttt{\ gh\ pr\ create\ } and follow further CLI instructions.
\end{enumerate}

\subsection{Changelog}\label{changelog}

You can view the change log in the \texttt{\ CHANGELOG.md\ } file in the
root of the project.

\subsection{License}\label{license}

This Typst package is licensed under AGPL v3.0. You can view the license
in the LICENSE file in the root of the project or at
\url{https://www.gnu.org/licenses/agpl-3.0.txt} . There is also a NOTICE
file for 3rd party copyright notices.

Copyright © 2023 Andrew Voynov

\subsubsection{How to add}\label{how-to-add}

Copy this into your project and use the import as \texttt{\ rubby\ }

\begin{verbatim}
#import "@preview/rubby:0.10.1"
\end{verbatim}

\includesvg[width=0.16667in,height=0.16667in]{/assets/icons/16-copy.svg}

Check the docs for
\href{https://typst.app/docs/reference/scripting/\#packages}{more
information on how to import packages} .

\subsubsection{About}\label{about}

\begin{description}
\tightlist
\item[Author :]
Andrew Voynov
\item[License:]
AGPL-3.0-only
\item[Current version:]
0.10.1
\item[Last updated:]
December 3, 2023
\item[First released:]
July 3, 2023
\item[Minimum Typst version:]
0.8.0
\item[Archive size:]
16.0 kB
\href{https://packages.typst.org/preview/rubby-0.10.1.tar.gz}{\pandocbounded{\includesvg[keepaspectratio]{/assets/icons/16-download.svg}}}
\item[Repository:]
\href{https://github.com/Andrew15-5/rubby}{GitHub}
\end{description}

\subsubsection{Where to report issues?}\label{where-to-report-issues}

This package is a project of Andrew Voynov . Report issues on
\href{https://github.com/Andrew15-5/rubby}{their repository} . You can
also try to ask for help with this package on the
\href{https://forum.typst.app}{Forum} .

Please report this package to the Typst team using the
\href{https://typst.app/contact}{contact form} if you believe it is a
safety hazard or infringes upon your rights.

\phantomsection\label{versions}
\subsubsection{Version history}\label{version-history}

\begin{longtable}[]{@{}ll@{}}
\toprule\noalign{}
Version & Release Date \\
\midrule\noalign{}
\endhead
\bottomrule\noalign{}
\endlastfoot
\href{https://typst.app/universe/package/rubby/0.9.2/}{0.9.2} &
September 15, 2023 \\
\href{https://typst.app/universe/package/rubby/0.8.0/}{0.8.0} & July 3,
2023 \\
0.10.1 & December 3, 2023 \\
\href{https://typst.app/universe/package/rubby/0.10.0/}{0.10.0} &
November 25, 2023 \\
\end{longtable}

Typst GmbH did not create this package and cannot guarantee correct
functionality of this package or compatibility with any version of the
Typst compiler or app.
