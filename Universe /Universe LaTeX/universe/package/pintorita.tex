\title{typst.app/universe/package/pintorita}

\phantomsection\label{banner}
\section{pintorita}\label{pintorita}

{ 0.1.2 }

Package to draw Sequence Diagrams, Entity Relationship Diagrams,
Component Diagrams, Activity Diagrams, Mind Maps, Gantt Diagrams, and
DOT Diagrams based on Pintora which is heavily influenced by mermaid.js
and plantuml.

\phantomsection\label{readme}
\href{https://pintorajs.vercel.app/}{Pintora}

Typst package for drawing the following from markup:

\begin{itemize}
\tightlist
\item
  Sequence Diagram
\item
  Entity Relationship Diagram
\item
  Component Diagram
\item
  Activity Diagram
\item
  Mind Map Experiment
\item
  Gantt Diagram Experiment
\item
  DOT Diagram Experiment
\end{itemize}

\pandocbounded{\includesvg[keepaspectratio]{https://github.com/typst/packages/raw/main/packages/preview/pintorita/0.1.2/pintorita.svg}}

\begin{Shaded}
\begin{Highlighting}[]
\NormalTok{\#import "@preview/pintorita:0.1.2"}

\NormalTok{\#set page(height: auto, width: auto, fill: black, margin: 2em)}
\NormalTok{\#set text(fill: white)}

\NormalTok{\#show raw.where(lang: "pintora"): it =\textgreater{} pintorita.render(it.text)}

\NormalTok{= pintora}

\NormalTok{Typst just got a load of diagrams. }

\NormalTok{\textasciigrave{}\textasciigrave{}\textasciigrave{}pintora}
\NormalTok{mindmap}
\NormalTok{@param layoutDirection TB}
\NormalTok{+ UML Diagrams}
\NormalTok{++ Behavior Diagrams}
\NormalTok{+++ Sequence Diagram}
\NormalTok{+++ State Diagram}
\NormalTok{+++ Activity Diagram}
\NormalTok{++ Structural Diagrams}
\NormalTok{+++ Class Diagram}
\NormalTok{+++ Component Diagram}
\NormalTok{\textasciigrave{}\textasciigrave{}\textasciigrave{}}

\NormalTok{\textasciigrave{}\textasciigrave{}\textasciigrave{}}
\NormalTok{mindmap}
\NormalTok{@param layoutDirection TB}
\NormalTok{+ UML Diagrams}
\NormalTok{++ Behavior Diagrams}
\NormalTok{+++ Sequence Diagram}
\NormalTok{+++ State Diagram}
\NormalTok{+++ Activity Diagram}
\NormalTok{++ Structural Diagrams}
\NormalTok{+++ Class Diagram}
\NormalTok{+++ Component Diagram}
\NormalTok{\textasciigrave{}\textasciigrave{}\textasciigrave{}}
\end{Highlighting}
\end{Shaded}

\subsection{Documentation}\label{documentation}

\subsubsection{\texorpdfstring{\texttt{\ render\ }}{ render }}\label{render}

Render a pintora string to an image

\paragraph{Arguments}\label{arguments}

\begin{itemize}
\tightlist
\item
  \texttt{\ src\ } : \texttt{\ str\ } - pintora source string
\item
  \texttt{\ factor\ } : scale output svg, “factor:0.5� will scale
  images down by half, so scale can be consistent across renders.
\item
  \texttt{\ style\ } : \texttt{\ str\ } diagram style,
  \texttt{\ default\ } or \texttt{\ dark\ } or \texttt{\ larkLight\ } or
  \texttt{\ larkDark\ }
\item
  \texttt{\ font\ } : \texttt{\ str\ } font family, default is
  \texttt{\ Source\ Code\ Pro,\ sans-serif\ }
\item
  All other arguments are passed to \texttt{\ image.decode\ } so you can
  customize the image size
\end{itemize}

\paragraph{Returns}\label{returns}

The image, of type \texttt{\ content\ }

\subsubsection{\texorpdfstring{\texttt{\ render-svg\ }}{ render-svg }}\label{render-svg}

Render a pintora string to an image

\paragraph{Arguments}\label{arguments-1}

\begin{itemize}
\tightlist
\item
  \texttt{\ src\ } : \texttt{\ str\ } - pintora source string
\item
  \texttt{\ factor\ } : scale output svg, “factor:0.5� will scale
  images down by half, so scale can be consistent across renders.
\item
  \texttt{\ style\ } : \texttt{\ str\ } diagram style,
  \texttt{\ default\ } or \texttt{\ dark\ } or \texttt{\ larkLight\ } or
  \texttt{\ larkDark\ }
\item
  \texttt{\ font\ } : \texttt{\ str\ } font family, default is
  \texttt{\ Source\ Code\ Pro,\ sans-serif\ }
\item
  All other arguments are passed to \texttt{\ image.decode\ } so you can
  customize the image size
\end{itemize}

\paragraph{Returns}\label{returns-1}

The svg image

\subsection{History}\label{history}

\begin{itemize}
\tightlist
\item
  0.1.0 - Inital Release
\item
  0.1.1 - Updated to Jogs 0.2.3 and pintora 0.7.3
\item
  0.1.2 - Fixed strange offset of text rows in class diagram, added
  \texttt{\ render-svg\ } function and more customization options
\end{itemize}

\subsubsection{How to add}\label{how-to-add}

Copy this into your project and use the import as \texttt{\ pintorita\ }

\begin{verbatim}
#import "@preview/pintorita:0.1.2"
\end{verbatim}

\includesvg[width=0.16667in,height=0.16667in]{/assets/icons/16-copy.svg}

Check the docs for
\href{https://typst.app/docs/reference/scripting/\#packages}{more
information on how to import packages} .

\subsubsection{About}\label{about}

\begin{description}
\tightlist
\item[Author s :]
Min Chen (hikerpig) \& Taylorh140
\item[License:]
MIT
\item[Current version:]
0.1.2
\item[Last updated:]
October 4, 2024
\item[First released:]
January 16, 2024
\item[Archive size:]
725 kB
\href{https://packages.typst.org/preview/pintorita-0.1.2.tar.gz}{\pandocbounded{\includesvg[keepaspectratio]{/assets/icons/16-download.svg}}}
\item[Repository:]
\href{https://github.com/taylorh140/typst-pintora}{GitHub}
\item[Categor y :]
\begin{itemize}
\tightlist
\item[]
\item
  \pandocbounded{\includesvg[keepaspectratio]{/assets/icons/16-chart.svg}}
  \href{https://typst.app/universe/search/?category=visualization}{Visualization}
\end{itemize}
\end{description}

\subsubsection{Where to report issues?}\label{where-to-report-issues}

This package is a project of Min Chen (hikerpig) and Taylorh140 . Report
issues on \href{https://github.com/taylorh140/typst-pintora}{their
repository} . You can also try to ask for help with this package on the
\href{https://forum.typst.app}{Forum} .

Please report this package to the Typst team using the
\href{https://typst.app/contact}{contact form} if you believe it is a
safety hazard or infringes upon your rights.

\phantomsection\label{versions}
\subsubsection{Version history}\label{version-history}

\begin{longtable}[]{@{}ll@{}}
\toprule\noalign{}
Version & Release Date \\
\midrule\noalign{}
\endhead
\bottomrule\noalign{}
\endlastfoot
0.1.2 & October 4, 2024 \\
\href{https://typst.app/universe/package/pintorita/0.1.1/}{0.1.1} &
April 3, 2024 \\
\href{https://typst.app/universe/package/pintorita/0.1.0/}{0.1.0} &
January 16, 2024 \\
\end{longtable}

Typst GmbH did not create this package and cannot guarantee correct
functionality of this package or compatibility with any version of the
Typst compiler or app.
