\title{typst.app/universe/package/ansi-render}

\phantomsection\label{banner}
\section{ansi-render}\label{ansi-render}

{ 0.6.1 }

provides a simple way to render text with ANSI escape sequences.

\phantomsection\label{readme}
\href{https://github.com/8LWXpg/typst-ansi-render/tags}{\pandocbounded{\includegraphics[keepaspectratio]{https://img.shields.io/github/v/tag/8LWXpg/typst-ansi-render}}}
\href{https://github.com/8LWXpg/typst-ansi-render}{\pandocbounded{\includegraphics[keepaspectratio]{https://img.shields.io/github/stars/8LWXpg/typst-ansi-render}}}
\href{https://github.com/8LWXpg/typst-ansi-render/blob/master/LICENSE}{\pandocbounded{\includegraphics[keepaspectratio]{https://img.shields.io/github/license/8LWXpg/typst-ansi-render}}}
\href{https://github.com/typst/packages/tree/main/packages/preview/ansi-render}{\pandocbounded{\includegraphics[keepaspectratio]{https://img.shields.io/badge/typst-package-239dad}}}

This script provides a simple way to render text with ANSI escape
sequences. Package \texttt{\ ansi-render\ } provides a function
\texttt{\ ansi-render\ } , and a dictionary of themes
\texttt{\ terminal-themes\ } .

contribution is welcomed!

\subsection{Usage}\label{usage}

\begin{Shaded}
\begin{Highlighting}[]
\NormalTok{\#import "@preview/ansi{-}render:0.6.1": *}

\NormalTok{\#ansi{-}render(}
\NormalTok{  string,}
\NormalTok{  font:           string or none,}
\NormalTok{  size:           length,}
\NormalTok{  width:          auto or relative length,}
\NormalTok{  height:         auto or relative length,}
\NormalTok{  breakable:      boolean,}
\NormalTok{  radius:         relative length or dictionary,}
\NormalTok{  inset:          relative length or dictionary,}
\NormalTok{  outset:         relative length or dictionary,}
\NormalTok{  spacing:        relative length or fraction,}
\NormalTok{  above:          relative length or fraction,}
\NormalTok{  below:          relative length or fraction,}
\NormalTok{  clip:           boolean,}
\NormalTok{  bold{-}is{-}bright: boolean,}
\NormalTok{  theme:          terminal{-}themes.theme,}
\NormalTok{)}
\end{Highlighting}
\end{Shaded}

\subsubsection{Parameters}\label{parameters}

\begin{itemize}
\tightlist
\item
  \texttt{\ string\ } - string with ANSI escape sequences
\item
  \texttt{\ font\ } - font name or none, default is
  \texttt{\ Cascadia\ Code\ } , set to \texttt{\ none\ } to use the same
  font as \texttt{\ raw\ }
\item
  \texttt{\ size\ } - font size, default is \texttt{\ 1em\ }
\item
  \texttt{\ bold-is-bright\ } - boolean, whether bold text is rendered
  with bright colors, default is \texttt{\ false\ }
\item
  \texttt{\ theme\ } - theme, default is \texttt{\ vscode-light\ }
\item
  parameters from
  \href{https://typst.app/docs/reference/layout/block/}{\texttt{\ block\ }}
  function with the same default value, only affects outmost block
  layout:

  \begin{itemize}
  \tightlist
  \item
    \texttt{\ width\ }
  \item
    \texttt{\ height\ }
  \item
    \texttt{\ breakable\ }
  \item
    \texttt{\ radius\ }
  \item
    \texttt{\ inset\ }
  \item
    \texttt{\ outset\ }
  \item
    \texttt{\ spacing\ }
  \item
    \texttt{\ above\ }
  \item
    \texttt{\ below\ }
  \item
    \texttt{\ clip\ }
  \end{itemize}
\end{itemize}

\subsection{Themes}\label{themes}

see
\href{https://github.com/8LWXpg/typst-ansi-render/blob/master/test/themes.pdf}{themes}

\subsection{Demo}\label{demo}

see
\href{https://github.com/8LWXpg/typst-ansi-render/blob/master/test/demo.typ}{demo.typ}
\href{https://github.com/8LWXpg/typst-ansi-render/blob/master/test/demo.pdf}{demo.pdf}

\begin{Shaded}
\begin{Highlighting}[]
\NormalTok{\#ansi{-}render(}
\NormalTok{"\textbackslash{}u\{1b\}[38;2;255;0;0mThis text is red.\textbackslash{}u\{1b\}[0m}
\NormalTok{\textbackslash{}u\{1b\}[48;2;0;255;0mThis background is green.\textbackslash{}u\{1b\}[0m}
\NormalTok{\textbackslash{}u\{1b\}[38;2;255;255;255m\textbackslash{}u\{1b\}[48;2;0;0;255mThis text is white on a blue background.\textbackslash{}u\{1b\}[0m}
\NormalTok{\textbackslash{}u\{1b\}[1mThis text is bold.\textbackslash{}u\{1b\}[0m}
\NormalTok{\textbackslash{}u\{1b\}[4mThis text is underlined.\textbackslash{}u\{1b\}[0m}
\NormalTok{\textbackslash{}u\{1b\}[38;2;255;165;0m\textbackslash{}u\{1b\}[48;2;255;255;0mThis text is orange on a yellow background.\textbackslash{}u\{1b\}[0m",}
\NormalTok{inset: 5pt, radius: 3pt,}
\NormalTok{theme: terminal{-}themes.vscode}
\NormalTok{)}
\end{Highlighting}
\end{Shaded}

\pandocbounded{\includegraphics[keepaspectratio]{https://raw.githubusercontent.com/8LWXpg/typst-ansi-render/master/img/1.png}}

\begin{Shaded}
\begin{Highlighting}[]
\NormalTok{\#ansi{-}render(}
\NormalTok{"\textbackslash{}u\{1b\}[38;5;196mRed text\textbackslash{}u\{1b\}[0m}
\NormalTok{\textbackslash{}u\{1b\}[48;5;27mBlue background\textbackslash{}u\{1b\}[0m}
\NormalTok{\textbackslash{}u\{1b\}[38;5;226;48;5;18mYellow text on blue background\textbackslash{}u\{1b\}[0m}
\NormalTok{\textbackslash{}u\{1b\}[7mInverted text\textbackslash{}u\{1b\}[0m}
\NormalTok{\textbackslash{}u\{1b\}[38;5;208;48;5;237mOrange text on gray background\textbackslash{}u\{1b\}[0m}
\NormalTok{\textbackslash{}u\{1b\}[38;5;39;48;5;208mBlue text on orange background\textbackslash{}u\{1b\}[0m}
\NormalTok{\textbackslash{}u\{1b\}[38;5;255;48;5;0mWhite text on black background\textbackslash{}u\{1b\}[0m",}
\NormalTok{inset: 5pt, radius: 3pt,}
\NormalTok{theme: terminal{-}themes.vscode}
\NormalTok{)}
\end{Highlighting}
\end{Shaded}

\pandocbounded{\includegraphics[keepaspectratio]{https://raw.githubusercontent.com/8LWXpg/typst-ansi-render/master/img/2.png}}

\begin{Shaded}
\begin{Highlighting}[]
\NormalTok{\#ansi{-}render(}
\NormalTok{"\textbackslash{}u\{1b\}[31;1mHello \textbackslash{}u\{1b\}[7mWorld\textbackslash{}u\{1b\}[0m}

\NormalTok{\textbackslash{}u\{1b\}[53;4;36mOver  and \textbackslash{}u\{1b\}[35m Under!}
\NormalTok{\textbackslash{}u\{1b\}[7;90mreverse\textbackslash{}u\{1b\}[101m and \textbackslash{}u\{1b\}[94;27mreverse",}
\NormalTok{inset: 5pt, radius: 3pt,}
\NormalTok{theme: terminal{-}themes.vscode}
\NormalTok{)}
\end{Highlighting}
\end{Shaded}

\pandocbounded{\includegraphics[keepaspectratio]{https://raw.githubusercontent.com/8LWXpg/typst-ansi-render/master/img/3.png}}

\begin{Shaded}
\begin{Highlighting}[]
\NormalTok{// uses the font that supports ligatures}
\NormalTok{\#ansi{-}render(read("test.txt"), inset: 5pt, radius: 3pt, font: "Cascadia Code", theme: terminal{-}themes.putty)}
\end{Highlighting}
\end{Shaded}

\pandocbounded{\includegraphics[keepaspectratio]{https://raw.githubusercontent.com/8LWXpg/typst-ansi-render/master/img/4.png}}

\subsubsection{How to add}\label{how-to-add}

Copy this into your project and use the import as
\texttt{\ ansi-render\ }

\begin{verbatim}
#import "@preview/ansi-render:0.6.1"
\end{verbatim}

\includesvg[width=0.16667in,height=0.16667in]{/assets/icons/16-copy.svg}

Check the docs for
\href{https://typst.app/docs/reference/scripting/\#packages}{more
information on how to import packages} .

\subsubsection{About}\label{about}

\begin{description}
\tightlist
\item[Author :]
8LWXpg
\item[License:]
MIT
\item[Current version:]
0.6.1
\item[Last updated:]
December 28, 2023
\item[First released:]
July 3, 2023
\item[Minimum Typst version:]
0.10.0
\item[Archive size:]
6.23 kB
\href{https://packages.typst.org/preview/ansi-render-0.6.1.tar.gz}{\pandocbounded{\includesvg[keepaspectratio]{/assets/icons/16-download.svg}}}
\item[Repository:]
\href{https://github.com/8LWXpg/typst-ansi-render}{GitHub}
\end{description}

\subsubsection{Where to report issues?}\label{where-to-report-issues}

This package is a project of 8LWXpg . Report issues on
\href{https://github.com/8LWXpg/typst-ansi-render}{their repository} .
You can also try to ask for help with this package on the
\href{https://forum.typst.app}{Forum} .

Please report this package to the Typst team using the
\href{https://typst.app/contact}{contact form} if you believe it is a
safety hazard or infringes upon your rights.

\phantomsection\label{versions}
\subsubsection{Version history}\label{version-history}

\begin{longtable}[]{@{}ll@{}}
\toprule\noalign{}
Version & Release Date \\
\midrule\noalign{}
\endhead
\bottomrule\noalign{}
\endlastfoot
0.6.1 & December 28, 2023 \\
\href{https://typst.app/universe/package/ansi-render/0.6.0/}{0.6.0} &
December 10, 2023 \\
\href{https://typst.app/universe/package/ansi-render/0.5.1/}{0.5.1} &
October 21, 2023 \\
\href{https://typst.app/universe/package/ansi-render/0.5.0/}{0.5.0} &
September 29, 2023 \\
\href{https://typst.app/universe/package/ansi-render/0.4.2/}{0.4.2} &
September 25, 2023 \\
\href{https://typst.app/universe/package/ansi-render/0.4.1/}{0.4.1} &
September 22, 2023 \\
\href{https://typst.app/universe/package/ansi-render/0.4.0/}{0.4.0} &
September 13, 2023 \\
\href{https://typst.app/universe/package/ansi-render/0.3.0/}{0.3.0} &
September 9, 2023 \\
\href{https://typst.app/universe/package/ansi-render/0.2.0/}{0.2.0} &
August 5, 2023 \\
\href{https://typst.app/universe/package/ansi-render/0.1.0/}{0.1.0} &
July 3, 2023 \\
\end{longtable}

Typst GmbH did not create this package and cannot guarantee correct
functionality of this package or compatibility with any version of the
Typst compiler or app.
