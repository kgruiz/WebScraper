\title{typst.app/universe/package/nth}

\phantomsection\label{banner}
\section{nth}\label{nth}

{ 1.0.1 }

Add english ordinals to numbers, eg. 1st, 2nd, 3rd, 4th.

\phantomsection\label{readme}
Provides functions \texttt{\ \#nth()\ } and \texttt{\ \#nths()\ } which
take a number and return it suffixed by an english ordinal.

This package is named after the nth
\href{https://ctan.org/pkg/nth}{LaTeX macro} by Donald Arseneau.

\subsection{Usage}\label{usage}

Include this line in your document to import the package.

\begin{Shaded}
\begin{Highlighting}[]
\NormalTok{\#import "@preview/nth:1.0.1": *}
\end{Highlighting}
\end{Shaded}

Then, you can use \texttt{\ \#nth()\ } to markup ordinal numbers in your
document.

For example, \texttt{\ \#nth(1)\ } shows 1st,\\
\texttt{\ \#nth(2)\ } shows 2nd,\\
\texttt{\ \#nth(3)\ } shows 3rd,\\
\texttt{\ \#nth(4)\ } shows 4th,\\
and \texttt{\ \#nth(11)\ } shows 11th.

If you want the ordinal to be in superscript, use \texttt{\ \#nths\ }
with an ‘s’ at the end.

For example, \texttt{\ \#nths(1)\ } shows 1 \textsuperscript{st} .

\subsubsection{How to add}\label{how-to-add}

Copy this into your project and use the import as \texttt{\ nth\ }

\begin{verbatim}
#import "@preview/nth:1.0.1"
\end{verbatim}

\includesvg[width=0.16667in,height=0.16667in]{/assets/icons/16-copy.svg}

Check the docs for
\href{https://typst.app/docs/reference/scripting/\#packages}{more
information on how to import packages} .

\subsubsection{About}\label{about}

\begin{description}
\tightlist
\item[Author s :]
\href{mailto:pierre.marshall@gmail.com}{Pierre Marshall} ,
\href{https://github.com/fnoaman}{fnoaman} , \&
\href{https://github.com/jeffa5}{Andrew Jeffery}
\item[License:]
MIT-0
\item[Current version:]
1.0.1
\item[Last updated:]
June 21, 2024
\item[First released:]
September 22, 2023
\item[Minimum Typst version:]
0.8.0
\item[Archive size:]
2.38 kB
\href{https://packages.typst.org/preview/nth-1.0.1.tar.gz}{\pandocbounded{\includesvg[keepaspectratio]{/assets/icons/16-download.svg}}}
\item[Repository:]
\href{https://github.com/extua/nth}{GitHub}
\item[Categor y :]
\begin{itemize}
\tightlist
\item[]
\item
  \pandocbounded{\includesvg[keepaspectratio]{/assets/icons/16-text.svg}}
  \href{https://typst.app/universe/search/?category=text}{Text}
\end{itemize}
\end{description}

\subsubsection{Where to report issues?}\label{where-to-report-issues}

This package is a project of Pierre Marshall, fnoaman, and Andrew
Jeffery . Report issues on \href{https://github.com/extua/nth}{their
repository} . You can also try to ask for help with this package on the
\href{https://forum.typst.app}{Forum} .

Please report this package to the Typst team using the
\href{https://typst.app/contact}{contact form} if you believe it is a
safety hazard or infringes upon your rights.

\phantomsection\label{versions}
\subsubsection{Version history}\label{version-history}

\begin{longtable}[]{@{}ll@{}}
\toprule\noalign{}
Version & Release Date \\
\midrule\noalign{}
\endhead
\bottomrule\noalign{}
\endlastfoot
1.0.1 & June 21, 2024 \\
\href{https://typst.app/universe/package/nth/1.0.0/}{1.0.0} & December
23, 2023 \\
\href{https://typst.app/universe/package/nth/0.2.0/}{0.2.0} & October 2,
2023 \\
\href{https://typst.app/universe/package/nth/0.1.0/}{0.1.0} & September
22, 2023 \\
\end{longtable}

Typst GmbH did not create this package and cannot guarantee correct
functionality of this package or compatibility with any version of the
Typst compiler or app.
