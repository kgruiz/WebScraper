\title{typst.app/universe/package/tabut}

\phantomsection\label{banner}
\section{tabut}\label{tabut}

{ 1.0.2 }

Display data as tables.

\phantomsection\label{readme}
\emph{Powerful, Simple, Concise}

A Typst plugin for turning data into tables.

\subsection{Outline}\label{outline}

\begin{itemize}
\item
  \href{https://github.com/typst/packages/raw/main/packages/preview/tabut/1.0.2/\#examples}{Examples}

  \begin{itemize}
  \item
    \href{https://github.com/typst/packages/raw/main/packages/preview/tabut/1.0.2/\#input-format-and-creation}{Input
    Format and Creation}
  \item
    \href{https://github.com/typst/packages/raw/main/packages/preview/tabut/1.0.2/\#basic-table}{Basic
    Table}
  \item
    \href{https://github.com/typst/packages/raw/main/packages/preview/tabut/1.0.2/\#table-styling}{Table
    Styling}
  \item
    \href{https://github.com/typst/packages/raw/main/packages/preview/tabut/1.0.2/\#header-formatting}{Header
    Formatting}
  \item
    \href{https://github.com/typst/packages/raw/main/packages/preview/tabut/1.0.2/\#remove-headers}{Remove
    Headers}
  \item
    \href{https://github.com/typst/packages/raw/main/packages/preview/tabut/1.0.2/\#cell-expressions-and-formatting}{Cell
    Expressions and Formatting}
  \item
    \href{https://github.com/typst/packages/raw/main/packages/preview/tabut/1.0.2/\#index}{Index}
  \item
    \href{https://github.com/typst/packages/raw/main/packages/preview/tabut/1.0.2/\#transpose}{Transpose}
  \item
    \href{https://github.com/typst/packages/raw/main/packages/preview/tabut/1.0.2/\#alignment}{Alignment}
  \item
    \href{https://github.com/typst/packages/raw/main/packages/preview/tabut/1.0.2/\#column-width}{Column
    Width}
  \item
    \href{https://github.com/typst/packages/raw/main/packages/preview/tabut/1.0.2/\#get-cells-only}{Get
    Cells Only}
  \item
    \href{https://github.com/typst/packages/raw/main/packages/preview/tabut/1.0.2/\#use-with-tablex}{Use
    with Tablex}
  \end{itemize}
\item
  \href{https://github.com/typst/packages/raw/main/packages/preview/tabut/1.0.2/\#data-operation-examples}{Data
  Operation Examples}

  \begin{itemize}
  \item
    \href{https://github.com/typst/packages/raw/main/packages/preview/tabut/1.0.2/\#csv-data}{CSV
    Data}
  \item
    \href{https://github.com/typst/packages/raw/main/packages/preview/tabut/1.0.2/\#slice}{Slice}
  \item
    \href{https://github.com/typst/packages/raw/main/packages/preview/tabut/1.0.2/\#sorting-and-reversing}{Sorting
    and Reversing}
  \item
    \href{https://github.com/typst/packages/raw/main/packages/preview/tabut/1.0.2/\#filter}{Filter}
  \item
    \href{https://github.com/typst/packages/raw/main/packages/preview/tabut/1.0.2/\#aggregation-using-map-and-sum}{Aggregation
    using Map and Sum}
  \item
    \href{https://github.com/typst/packages/raw/main/packages/preview/tabut/1.0.2/\#grouping}{Grouping}
  \end{itemize}
\item
  \href{https://github.com/typst/packages/raw/main/packages/preview/tabut/1.0.2/\#function-definitions}{Function
  Definitions}

  \begin{itemize}
  \item
    \href{https://github.com/typst/packages/raw/main/packages/preview/tabut/1.0.2/\#tabut}{\texttt{\ tabut\ }}
  \item
    \href{https://github.com/typst/packages/raw/main/packages/preview/tabut/1.0.2/\#tabut-cells}{\texttt{\ tabut-cells\ }}
  \item
    \href{https://github.com/typst/packages/raw/main/packages/preview/tabut/1.0.2/\#rows-to-records}{\texttt{\ rows-to-records\ }}
  \item
    \href{https://github.com/typst/packages/raw/main/packages/preview/tabut/1.0.2/\#records-from-csv}{\texttt{\ records-from-csv\ }}
  \item
    \href{https://github.com/typst/packages/raw/main/packages/preview/tabut/1.0.2/\#group}{\texttt{\ group\ }}
  \end{itemize}
\end{itemize}

\subsection[Input Format and Creation ]{\texorpdfstring{Input Format and
Creation \protect\hypertarget{input-format-and-creation}{}{
}}{Input Format and Creation  }}\label{input-format-and-creation}

The \texttt{\ tabut\ } function takes input in “record� format, an
array of dictionaries, with each dictionary representing a single
“object� or “record�.

In the example below, each record is a listing for an office supply
product.

\begin{Shaded}
\begin{Highlighting}[]
\NormalTok{\#let supplies = (}
\NormalTok{  (name: "Notebook", price: 3.49, quantity: 5),}
\NormalTok{  (name: "Ballpoint Pens", price: 5.99, quantity: 2),}
\NormalTok{  (name: "Printer Paper", price: 6.99, quantity: 3),}
\NormalTok{)}
\end{Highlighting}
\end{Shaded}

\subsection[Basic Table ]{\texorpdfstring{Basic Table
\protect\hypertarget{basic-table}{}{
}}{Basic Table  }}\label{basic-table}

Now create a basic table from the data.

\begin{Shaded}
\begin{Highlighting}[]
\NormalTok{\#import "@preview/tabut:1.0.2": tabut}
\NormalTok{\#import "example{-}data/supplies.typ": supplies}

\NormalTok{\#tabut(}
\NormalTok{  supplies, // the source of the data used to generate the table}
\NormalTok{  ( // column definitions}
\NormalTok{    (}
\NormalTok{      header: [Name], // label, takes content.}
\NormalTok{      func: r =\textgreater{} r.name // generates the cell content.}
\NormalTok{    ), }
\NormalTok{    (header: [Price], func: r =\textgreater{} r.price), }
\NormalTok{    (header: [Quantity], func: r =\textgreater{} r.quantity), }
\NormalTok{  )}
\NormalTok{)}
\end{Highlighting}
\end{Shaded}

\pandocbounded{\includesvg[keepaspectratio]{https://github.com/typst/packages/raw/main/packages/preview/tabut/1.0.2/doc/compiled-snippets/basic.svg}}

\texttt{\ funct\ } takes a function which generates content for a given
cell corrosponding to the defined column for each record. \texttt{\ r\ }
is the record, so \texttt{\ r\ =\textgreater{}\ r.name\ } returns the
\texttt{\ name\ } property of each record in the input data if it has
one.

The philosphy of \texttt{\ tabut\ } is that the display of data should
be simple and clearly defined, therefore each column and it’s content
and formatting should be defined within a single clear column defintion.
One consequence is you can comment out, remove or move, any column
easily, for example:

\begin{Shaded}
\begin{Highlighting}[]
\NormalTok{\#import "@preview/tabut:1.0.2": tabut}
\NormalTok{\#import "example{-}data/supplies.typ": supplies}

\NormalTok{\#tabut(}
\NormalTok{  supplies,}
\NormalTok{  (}
\NormalTok{    (header: [Price], func: r =\textgreater{} r.price), // This column is moved to the front}
\NormalTok{    (header: [Name], func: r =\textgreater{} r.name), }
\NormalTok{    (header: [Name 2], func: r =\textgreater{} r.name), // copied}
\NormalTok{    // (header: [Quantity], func: r =\textgreater{} r.quantity), // removed via comment}
\NormalTok{  )}
\NormalTok{)}
\end{Highlighting}
\end{Shaded}

\pandocbounded{\includesvg[keepaspectratio]{https://github.com/typst/packages/raw/main/packages/preview/tabut/1.0.2/doc/compiled-snippets/rearrange.svg}}

\subsection[Table Styling ]{\texorpdfstring{Table Styling
\protect\hypertarget{table-styling}{}{
}}{Table Styling  }}\label{table-styling}

Any default Table style options can be tacked on and are passed to the
final table function.

\begin{Shaded}
\begin{Highlighting}[]
\NormalTok{\#import "@preview/tabut:1.0.2": tabut}
\NormalTok{\#import "example{-}data/supplies.typ": supplies}

\NormalTok{\#tabut(}
\NormalTok{  supplies,}
\NormalTok{  ( }
\NormalTok{    (header: [Name], func: r =\textgreater{} r.name), }
\NormalTok{    (header: [Price], func: r =\textgreater{} r.price), }
\NormalTok{    (header: [Quantity], func: r =\textgreater{} r.quantity),}
\NormalTok{  ),}
\NormalTok{  fill: (\_, row) =\textgreater{} if calc.odd(row) \{ luma(240) \} else \{ luma(220) \}, }
\NormalTok{  stroke: none}
\NormalTok{)}
\end{Highlighting}
\end{Shaded}

\pandocbounded{\includesvg[keepaspectratio]{https://github.com/typst/packages/raw/main/packages/preview/tabut/1.0.2/doc/compiled-snippets/styling.svg}}

\subsection[Header Formatting ]{\texorpdfstring{Header Formatting
\protect\hypertarget{header-formatting}{}{
}}{Header Formatting  }}\label{header-formatting}

You can pass any content or expression into the header property.

\begin{Shaded}
\begin{Highlighting}[]
\NormalTok{\#import "@preview/tabut:1.0.2": tabut}
\NormalTok{\#import "example{-}data/supplies.typ": supplies}

\NormalTok{\#let fmt(it) = \{}
\NormalTok{  heading(}
\NormalTok{    outlined: false,}
\NormalTok{    upper(it)}
\NormalTok{  )}
\NormalTok{\}}

\NormalTok{\#tabut(}
\NormalTok{  supplies,}
\NormalTok{  ( }
\NormalTok{    (header: fmt([Name]), func: r =\textgreater{} r.name ), }
\NormalTok{    (header: fmt([Price]), func: r =\textgreater{} r.price), }
\NormalTok{    (header: fmt([Quantity]), func: r =\textgreater{} r.quantity), }
\NormalTok{  ),}
\NormalTok{  fill: (\_, row) =\textgreater{} if calc.odd(row) \{ luma(240) \} else \{ luma(220) \}, }
\NormalTok{  stroke: none}
\NormalTok{)}
\end{Highlighting}
\end{Shaded}

\pandocbounded{\includesvg[keepaspectratio]{https://github.com/typst/packages/raw/main/packages/preview/tabut/1.0.2/doc/compiled-snippets/title.svg}}

\subsection[Remove Headers ]{\texorpdfstring{Remove Headers
\protect\hypertarget{remove-headers}{}{
}}{Remove Headers  }}\label{remove-headers}

You can prevent from being generated with the \texttt{\ headers\ }
paramater. This is useful with the \texttt{\ tabut-cells\ } function as
demonstrated in it’s section.

\begin{Shaded}
\begin{Highlighting}[]
\NormalTok{\#import "@preview/tabut:1.0.2": tabut}
\NormalTok{\#import "example{-}data/supplies.typ": supplies}

\NormalTok{\#tabut(}
\NormalTok{  supplies,}
\NormalTok{  (}
\NormalTok{    (header: [*Name*], func: r =\textgreater{} r.name), }
\NormalTok{    (header: [*Price*], func: r =\textgreater{} r.price), }
\NormalTok{    (header: [*Quantity*], func: r =\textgreater{} r.quantity), }
\NormalTok{  ),}
\NormalTok{  headers: false, // Prevents Headers from being generated}
\NormalTok{  fill: (\_, row) =\textgreater{} if calc.odd(row) \{ luma(240) \} else \{ luma(220) \}, }
\NormalTok{  stroke: none,}
\NormalTok{)}
\end{Highlighting}
\end{Shaded}

\pandocbounded{\includesvg[keepaspectratio]{https://github.com/typst/packages/raw/main/packages/preview/tabut/1.0.2/doc/compiled-snippets/no-headers.svg}}

\subsection[Cell Expressions and Formatting ]{\texorpdfstring{Cell
Expressions and Formatting
\protect\hypertarget{cell-expressions-and-formatting}{}{
}}{Cell Expressions and Formatting  }}\label{cell-expressions-and-formatting}

Just like the headers, cell contents can be modified and formatted like
any content in Typst.

\begin{Shaded}
\begin{Highlighting}[]
\NormalTok{\#import "@preview/tabut:1.0.2": tabut}
\NormalTok{\#import "usd.typ": usd}
\NormalTok{\#import "example{-}data/supplies.typ": supplies}

\NormalTok{\#tabut(}
\NormalTok{  supplies,}
\NormalTok{  ( }
\NormalTok{    (header: [*Name*], func: r =\textgreater{} r.name ), }
\NormalTok{    (header: [*Price*], func: r =\textgreater{} usd(r.price)), }
\NormalTok{  ),}
\NormalTok{  fill: (\_, row) =\textgreater{} if calc.odd(row) \{ luma(240) \} else \{ luma(220) \}, }
\NormalTok{  stroke: none}
\NormalTok{)}
\end{Highlighting}
\end{Shaded}

\pandocbounded{\includesvg[keepaspectratio]{https://github.com/typst/packages/raw/main/packages/preview/tabut/1.0.2/doc/compiled-snippets/format.svg}}

You can have the cell content function do calculations on a record
property.

\begin{Shaded}
\begin{Highlighting}[]
\NormalTok{\#import "@preview/tabut:1.0.2": tabut}
\NormalTok{\#import "usd.typ": usd}
\NormalTok{\#import "example{-}data/supplies.typ": supplies}

\NormalTok{\#tabut(}
\NormalTok{  supplies,}
\NormalTok{  ( }
\NormalTok{    (header: [*Name*], func: r =\textgreater{} r.name ), }
\NormalTok{    (header: [*Price*], func: r =\textgreater{} usd(r.price)), }
\NormalTok{    (header: [*Tax*], func: r =\textgreater{} usd(r.price * .2)), }
\NormalTok{    (header: [*Total*], func: r =\textgreater{} usd(r.price * 1.2)), }
\NormalTok{  ),}
\NormalTok{  fill: (\_, row) =\textgreater{} if calc.odd(row) \{ luma(240) \} else \{ luma(220) \}, }
\NormalTok{  stroke: none}
\NormalTok{)}
\end{Highlighting}
\end{Shaded}

\pandocbounded{\includesvg[keepaspectratio]{https://github.com/typst/packages/raw/main/packages/preview/tabut/1.0.2/doc/compiled-snippets/calculation.svg}}

Or even combine multiple record properties, go wild.

\begin{Shaded}
\begin{Highlighting}[]
\NormalTok{\#import "@preview/tabut:1.0.2": tabut}

\NormalTok{\#let employees = (}
\NormalTok{    (id: 3251, first: "Alice", last: "Smith", middle: "Jane"),}
\NormalTok{    (id: 4872, first: "Carlos", last: "Garcia", middle: "Luis"),}
\NormalTok{    (id: 5639, first: "Evelyn", last: "Chen", middle: "Ming")}
\NormalTok{);}

\NormalTok{\#tabut(}
\NormalTok{  employees,}
\NormalTok{  ( }
\NormalTok{    (header: [*ID*], func: r =\textgreater{} r.id ),}
\NormalTok{    (header: [*Full Name*], func: r =\textgreater{} [\#r.first \#r.middle.first(), \#r.last] ),}
\NormalTok{  ),}
\NormalTok{  fill: (\_, row) =\textgreater{} if calc.odd(row) \{ luma(240) \} else \{ luma(220) \}, }
\NormalTok{  stroke: none}
\NormalTok{)}
\end{Highlighting}
\end{Shaded}

\pandocbounded{\includesvg[keepaspectratio]{https://github.com/typst/packages/raw/main/packages/preview/tabut/1.0.2/doc/compiled-snippets/combine.svg}}

\subsection[Index ]{\texorpdfstring{Index \protect\hypertarget{index}{}{
}}{Index  }}\label{index}

\texttt{\ tabut\ } automatically adds an \texttt{\ \_index\ } property
to each record.

\begin{Shaded}
\begin{Highlighting}[]
\NormalTok{\#import "@preview/tabut:1.0.2": tabut}
\NormalTok{\#import "example{-}data/supplies.typ": supplies}

\NormalTok{\#tabut(}
\NormalTok{  supplies,}
\NormalTok{  ( }
\NormalTok{    (header: [*\textbackslash{}\#*], func: r =\textgreater{} r.\_index),}
\NormalTok{    (header: [*Name*], func: r =\textgreater{} r.name ), }
\NormalTok{  ),}
\NormalTok{  fill: (\_, row) =\textgreater{} if calc.odd(row) \{ luma(240) \} else \{ luma(220) \}, }
\NormalTok{  stroke: none}
\NormalTok{)}
\end{Highlighting}
\end{Shaded}

\pandocbounded{\includesvg[keepaspectratio]{https://github.com/typst/packages/raw/main/packages/preview/tabut/1.0.2/doc/compiled-snippets/index.svg}}

You can also prevent the \texttt{\ index\ } property being generated by
setting it to \texttt{\ none\ } , or you can also set an alternate name
of the index property as shown below.

\begin{Shaded}
\begin{Highlighting}[]
\NormalTok{\#import "@preview/tabut:1.0.2": tabut}
\NormalTok{\#import "example{-}data/supplies.typ": supplies}

\NormalTok{\#tabut(}
\NormalTok{  supplies,}
\NormalTok{  ( }
\NormalTok{    (header: [*\textbackslash{}\#*], func: r =\textgreater{} r.index{-}alt ),}
\NormalTok{    (header: [*Name*], func: r =\textgreater{} r.name ), }
\NormalTok{  ),}
\NormalTok{  index: "index{-}alt", // set an aternate name for the automatically generated index property.}
\NormalTok{  fill: (\_, row) =\textgreater{} if calc.odd(row) \{ luma(240) \} else \{ luma(220) \}, }
\NormalTok{  stroke: none}
\NormalTok{)}
\end{Highlighting}
\end{Shaded}

\pandocbounded{\includesvg[keepaspectratio]{https://github.com/typst/packages/raw/main/packages/preview/tabut/1.0.2/doc/compiled-snippets/index-alternate.svg}}

\subsection[Transpose ]{\texorpdfstring{Transpose
\protect\hypertarget{transpose}{}{ }}{Transpose  }}\label{transpose}

This was annoying to implement, and I don’t know when you’d actually
use this, but here.

\begin{Shaded}
\begin{Highlighting}[]
\NormalTok{\#import "@preview/tabut:1.0.2": tabut}
\NormalTok{\#import "usd.typ": usd}
\NormalTok{\#import "example{-}data/supplies.typ": supplies}

\NormalTok{\#tabut(}
\NormalTok{  supplies,}
\NormalTok{  (}
\NormalTok{    (header: [*\textbackslash{}\#*], func: r =\textgreater{} r.\_index),}
\NormalTok{    (header: [*Name*], func: r =\textgreater{} r.name), }
\NormalTok{    (header: [*Price*], func: r =\textgreater{} usd(r.price)), }
\NormalTok{    (header: [*Quantity*], func: r =\textgreater{} r.quantity),}
\NormalTok{  ),}
\NormalTok{  transpose: true,  // set optional name arg \textasciigrave{}transpose\textasciigrave{} to \textasciigrave{}true\textasciigrave{}}
\NormalTok{  fill: (\_, row) =\textgreater{} if calc.odd(row) \{ luma(240) \} else \{ luma(220) \}, }
\NormalTok{  stroke: none}
\NormalTok{)}
\end{Highlighting}
\end{Shaded}

\pandocbounded{\includesvg[keepaspectratio]{https://github.com/typst/packages/raw/main/packages/preview/tabut/1.0.2/doc/compiled-snippets/transpose.svg}}

\subsection[Alignment ]{\texorpdfstring{Alignment
\protect\hypertarget{alignment}{}{ }}{Alignment  }}\label{alignment}

\begin{Shaded}
\begin{Highlighting}[]
\NormalTok{\#import "@preview/tabut:1.0.2": tabut}
\NormalTok{\#import "usd.typ": usd}
\NormalTok{\#import "example{-}data/supplies.typ": supplies}

\NormalTok{\#tabut(}
\NormalTok{  supplies,}
\NormalTok{  ( // Include \textasciigrave{}align\textasciigrave{} as an optional arg to a column def}
\NormalTok{    (header: [*\textbackslash{}\#*], func: r =\textgreater{} r.\_index),}
\NormalTok{    (header: [*Name*], align: right, func: r =\textgreater{} r.name), }
\NormalTok{    (header: [*Price*], align: right, func: r =\textgreater{} usd(r.price)), }
\NormalTok{    (header: [*Quantity*], align: right, func: r =\textgreater{} r.quantity),}
\NormalTok{  ),}
\NormalTok{  fill: (\_, row) =\textgreater{} if calc.odd(row) \{ luma(240) \} else \{ luma(220) \}, }
\NormalTok{  stroke: none}
\NormalTok{)}
\end{Highlighting}
\end{Shaded}

\pandocbounded{\includesvg[keepaspectratio]{https://github.com/typst/packages/raw/main/packages/preview/tabut/1.0.2/doc/compiled-snippets/align.svg}}

You can also define Alignment manually as in the the standard Table
Function.

\begin{Shaded}
\begin{Highlighting}[]
\NormalTok{\#import "@preview/tabut:1.0.2": tabut}
\NormalTok{\#import "usd.typ": usd}
\NormalTok{\#import "example{-}data/supplies.typ": supplies}

\NormalTok{\#tabut(}
\NormalTok{  supplies,}
\NormalTok{  ( }
\NormalTok{    (header: [*\textbackslash{}\#*], func: r =\textgreater{} r.\_index),}
\NormalTok{    (header: [*Name*], func: r =\textgreater{} r.name), }
\NormalTok{    (header: [*Price*], func: r =\textgreater{} usd(r.price)), }
\NormalTok{    (header: [*Quantity*], func: r =\textgreater{} r.quantity),}
\NormalTok{  ),}
\NormalTok{  align: (auto, right, right, right), // Alignment defined as in standard table function}
\NormalTok{  fill: (\_, row) =\textgreater{} if calc.odd(row) \{ luma(240) \} else \{ luma(220) \}, }
\NormalTok{  stroke: none}
\NormalTok{)}
\end{Highlighting}
\end{Shaded}

\pandocbounded{\includesvg[keepaspectratio]{https://github.com/typst/packages/raw/main/packages/preview/tabut/1.0.2/doc/compiled-snippets/align-manual.svg}}

\subsection[Column Width ]{\texorpdfstring{Column Width
\protect\hypertarget{column-width}{}{
}}{Column Width  }}\label{column-width}

\begin{Shaded}
\begin{Highlighting}[]
\NormalTok{\#import "@preview/tabut:1.0.2": tabut}
\NormalTok{\#import "usd.typ": usd}
\NormalTok{\#import "example{-}data/supplies.typ": supplies}

\NormalTok{\#box(}
\NormalTok{  width: 300pt,}
\NormalTok{  tabut(}
\NormalTok{    supplies,}
\NormalTok{    ( // Include \textasciigrave{}width\textasciigrave{} as an optional arg to a column def}
\NormalTok{      (header: [*\textbackslash{}\#*], func: r =\textgreater{} r.\_index),}
\NormalTok{      (header: [*Name*], width: 1fr, func: r =\textgreater{} r.name), }
\NormalTok{      (header: [*Price*], width: 20\%, func: r =\textgreater{} usd(r.price)), }
\NormalTok{      (header: [*Quantity*], width: 1.5in, func: r =\textgreater{} r.quantity),}
\NormalTok{    ),}
\NormalTok{    fill: (\_, row) =\textgreater{} if calc.odd(row) \{ luma(240) \} else \{ luma(220) \}, }
\NormalTok{    stroke: none,}
\NormalTok{  )}
\NormalTok{)}
\end{Highlighting}
\end{Shaded}

\pandocbounded{\includesvg[keepaspectratio]{https://github.com/typst/packages/raw/main/packages/preview/tabut/1.0.2/doc/compiled-snippets/width.svg}}

You can also define Columns manually as in the the standard Table
Function.

\begin{Shaded}
\begin{Highlighting}[]
\NormalTok{\#import "@preview/tabut:1.0.2": tabut}
\NormalTok{\#import "usd.typ": usd}
\NormalTok{\#import "example{-}data/supplies.typ": supplies}

\NormalTok{\#box(}
\NormalTok{  width: 300pt,}
\NormalTok{  tabut(}
\NormalTok{    supplies,}
\NormalTok{    (}
\NormalTok{      (header: [*\textbackslash{}\#*], func: r =\textgreater{} r.\_index),}
\NormalTok{      (header: [*Name*], func: r =\textgreater{} r.name), }
\NormalTok{      (header: [*Price*], func: r =\textgreater{} usd(r.price)), }
\NormalTok{      (header: [*Quantity*], func: r =\textgreater{} r.quantity),}
\NormalTok{    ),}
\NormalTok{    columns: (auto, 1fr, 20\%, 1.5in),  // Columns defined as in standard table}
\NormalTok{    fill: (\_, row) =\textgreater{} if calc.odd(row) \{ luma(240) \} else \{ luma(220) \}, }
\NormalTok{    stroke: none,}
\NormalTok{  )}
\NormalTok{)}
\end{Highlighting}
\end{Shaded}

\pandocbounded{\includesvg[keepaspectratio]{https://github.com/typst/packages/raw/main/packages/preview/tabut/1.0.2/doc/compiled-snippets/width-manual.svg}}

\subsection[Get Cells Only ]{\texorpdfstring{Get Cells Only
\protect\hypertarget{get-cells-only}{}{
}}{Get Cells Only  }}\label{get-cells-only}

\begin{Shaded}
\begin{Highlighting}[]
\NormalTok{\#import "@preview/tabut:1.0.2": tabut{-}cells}
\NormalTok{\#import "usd.typ": usd}
\NormalTok{\#import "example{-}data/supplies.typ": supplies}

\NormalTok{\#tabut{-}cells(}
\NormalTok{  supplies,}
\NormalTok{  ( }
\NormalTok{    (header: [Name], func: r =\textgreater{} r.name), }
\NormalTok{    (header: [Price], func: r =\textgreater{} usd(r.price)), }
\NormalTok{    (header: [Quantity], func: r =\textgreater{} r.quantity),}
\NormalTok{  )}
\NormalTok{)}
\end{Highlighting}
\end{Shaded}

\pandocbounded{\includesvg[keepaspectratio]{https://github.com/typst/packages/raw/main/packages/preview/tabut/1.0.2/doc/compiled-snippets/only-cells.svg}}

\subsection[Use with Tablex ]{\texorpdfstring{Use with Tablex
\protect\hypertarget{use-with-tablex}{}{
}}{Use with Tablex  }}\label{use-with-tablex}

\begin{Shaded}
\begin{Highlighting}[]
\NormalTok{\#import "@preview/tabut:1.0.2": tabut{-}cells}
\NormalTok{\#import "usd.typ": usd}
\NormalTok{\#import "example{-}data/supplies.typ": supplies}

\NormalTok{\#import "@preview/tablex:0.0.8": tablex, rowspanx, colspanx}

\NormalTok{\#tablex(}
\NormalTok{  auto{-}vlines: false,}
\NormalTok{  header{-}rows: 2,}

\NormalTok{  /* {-}{-}{-} header {-}{-}{-} */}
\NormalTok{  rowspanx(2)[*Name*], colspanx(2)[*Price*], (), rowspanx(2)[*Quantity*],}
\NormalTok{  (),                 [*Base*], [*W/Tax*], (),}
\NormalTok{  /* {-}{-}{-}{-}{-}{-}{-}{-}{-}{-}{-}{-}{-}{-} */}

\NormalTok{  ..tabut{-}cells(}
\NormalTok{    supplies,}
\NormalTok{    ( }
\NormalTok{      (header: [], func: r =\textgreater{} r.name), }
\NormalTok{      (header: [], func: r =\textgreater{} usd(r.price)), }
\NormalTok{      (header: [], func: r =\textgreater{} usd(r.price * 1.3)), }
\NormalTok{      (header: [], func: r =\textgreater{} r.quantity),}
\NormalTok{    ),}
\NormalTok{    headers: false}
\NormalTok{  )}
\NormalTok{)}
\end{Highlighting}
\end{Shaded}

\pandocbounded{\includesvg[keepaspectratio]{https://github.com/typst/packages/raw/main/packages/preview/tabut/1.0.2/doc/compiled-snippets/tablex.svg}}

While technically seperate from table display, the following are
examples of how to perform operations on data before it is displayed
with \texttt{\ tabut\ } .

Since \texttt{\ tabut\ } assumes an “array of dictionaries� format,
then most data operations can be performed easily with Typst’s native
array functions. \texttt{\ tabut\ } also provides several functions to
provide additional functionality.

\subsection[CSV Data ]{\texorpdfstring{CSV Data
\protect\hypertarget{csv-data}{}{ }}{CSV Data  }}\label{csv-data}

By default, imported CSV gives a “rows� or “array of arrays�
data format, which can not be directly used by \texttt{\ tabut\ } . To
convert, \texttt{\ tabut\ } includes a function
\texttt{\ rows-to-records\ } demonstrated below.

\begin{Shaded}
\begin{Highlighting}[]
\NormalTok{\#import "@preview/tabut:1.0.2": tabut, rows{-}to{-}records}
\NormalTok{\#import "example{-}data/supplies.typ": supplies}

\NormalTok{\#let titanic = \{}
\NormalTok{  let titanic{-}raw = csv("example{-}data/titanic.csv");}
\NormalTok{  rows{-}to{-}records(}
\NormalTok{    titanic{-}raw.first(), // The header row}
\NormalTok{    titanic{-}raw.slice(1, {-}1), // The rest of the rows}
\NormalTok{  )}
\NormalTok{\}}
\end{Highlighting}
\end{Shaded}

Imported CSV data are all strings, so it’s usefull to convert them to
\texttt{\ int\ } or \texttt{\ float\ } when possible.

\begin{Shaded}
\begin{Highlighting}[]
\NormalTok{\#import "@preview/tabut:1.0.2": tabut, rows{-}to{-}records}
\NormalTok{\#import "example{-}data/supplies.typ": supplies}

\NormalTok{\#let auto{-}type(input) = \{}
\NormalTok{  let is{-}int = (input.match(regex("\^{}{-}?\textbackslash{}d+$")) != none);}
\NormalTok{  if is{-}int \{ return int(input); \}}
\NormalTok{  let is{-}float = (input.match(regex("\^{}{-}?(inf|nan|\textbackslash{}d+|\textbackslash{}d*(\textbackslash{}.\textbackslash{}d+))$")) != none);}
\NormalTok{  if is{-}float \{ return float(input) \}}
\NormalTok{  input}
\NormalTok{\}}

\NormalTok{\#let titanic = \{}
\NormalTok{  let titanic{-}raw = csv("example{-}data/titanic.csv");}
\NormalTok{  rows{-}to{-}records( titanic{-}raw.first(), titanic{-}raw.slice(1, {-}1) )}
\NormalTok{  .map( r =\textgreater{} \{}
\NormalTok{    let new{-}record = (:);}
\NormalTok{    for (k, v) in r.pairs() \{ new{-}record.insert(k, auto{-}type(v)); \}}
\NormalTok{    new{-}record}
\NormalTok{  \})}
\NormalTok{\}}
\end{Highlighting}
\end{Shaded}

\texttt{\ tabut\ } includes a function, \texttt{\ records-from-csv\ } ,
to automatically perform this process.

\begin{Shaded}
\begin{Highlighting}[]
\NormalTok{\#import "@preview/tabut:1.0.2": records{-}from{-}csv}

\NormalTok{\#let titanic = records{-}from{-}csv(csv("example{-}data/titanic.csv"));}
\end{Highlighting}
\end{Shaded}

\subsection[Slice ]{\texorpdfstring{Slice \protect\hypertarget{slice}{}{
}}{Slice  }}\label{slice}

\begin{Shaded}
\begin{Highlighting}[]
\NormalTok{\#import "@preview/tabut:1.0.2": tabut, records{-}from{-}csv}
\NormalTok{\#import "usd.typ": usd}
\NormalTok{\#import "example{-}data/titanic.typ": titanic}

\NormalTok{\#let classes = (}
\NormalTok{  "N/A",}
\NormalTok{  "First", }
\NormalTok{  "Second", }
\NormalTok{  "Third"}
\NormalTok{);}

\NormalTok{\#let titanic{-}head = titanic.slice(0, 5);}

\NormalTok{\#tabut(}
\NormalTok{  titanic{-}head,}
\NormalTok{  ( }
\NormalTok{    (header: [*Name*], func: r =\textgreater{} r.Name), }
\NormalTok{    (header: [*Class*], func: r =\textgreater{} classes.at(r.Pclass)),}
\NormalTok{    (header: [*Fare*], func: r =\textgreater{} usd(r.Fare)), }
\NormalTok{    (header: [*Survived?*], func: r =\textgreater{} ("No", "Yes").at(r.Survived)), }
\NormalTok{  ),}
\NormalTok{  fill: (\_, row) =\textgreater{} if calc.odd(row) \{ luma(240) \} else \{ luma(220) \}, }
\NormalTok{  stroke: none}
\NormalTok{)}
\end{Highlighting}
\end{Shaded}

\pandocbounded{\includesvg[keepaspectratio]{https://github.com/typst/packages/raw/main/packages/preview/tabut/1.0.2/doc/compiled-snippets/slice.svg}}

\subsection[Sorting and Reversing ]{\texorpdfstring{Sorting and
Reversing \protect\hypertarget{sorting-and-reversing}{}{
}}{Sorting and Reversing  }}\label{sorting-and-reversing}

\begin{Shaded}
\begin{Highlighting}[]
\NormalTok{\#import "@preview/tabut:1.0.2": tabut}
\NormalTok{\#import "usd.typ": usd}
\NormalTok{\#import "example{-}data/titanic.typ": titanic, classes}

\NormalTok{\#tabut(}
\NormalTok{  titanic}
\NormalTok{  .sorted(key: r =\textgreater{} r.Fare)}
\NormalTok{  .rev()}
\NormalTok{  .slice(0, 5),}
\NormalTok{  ( }
\NormalTok{    (header: [*Name*], func: r =\textgreater{} r.Name), }
\NormalTok{    (header: [*Class*], func: r =\textgreater{} classes.at(r.Pclass)),}
\NormalTok{    (header: [*Fare*], func: r =\textgreater{} usd(r.Fare)), }
\NormalTok{    (header: [*Survived?*], func: r =\textgreater{} ("No", "Yes").at(r.Survived)), }
\NormalTok{  ),}
\NormalTok{  fill: (\_, row) =\textgreater{} if calc.odd(row) \{ luma(240) \} else \{ luma(220) \}, }
\NormalTok{  stroke: none}
\NormalTok{)}
\end{Highlighting}
\end{Shaded}

\pandocbounded{\includesvg[keepaspectratio]{https://github.com/typst/packages/raw/main/packages/preview/tabut/1.0.2/doc/compiled-snippets/sort.svg}}

\subsection[Filter ]{\texorpdfstring{Filter
\protect\hypertarget{filter}{}{ }}{Filter  }}\label{filter}

\begin{Shaded}
\begin{Highlighting}[]
\NormalTok{\#import "@preview/tabut:1.0.2": tabut}
\NormalTok{\#import "usd.typ": usd}
\NormalTok{\#import "example{-}data/titanic.typ": titanic, classes}

\NormalTok{\#tabut(}
\NormalTok{  titanic}
\NormalTok{  .filter(r =\textgreater{} r.Pclass == 1)}
\NormalTok{  .slice(0, 5),}
\NormalTok{  ( }
\NormalTok{    (header: [*Name*], func: r =\textgreater{} r.Name), }
\NormalTok{    (header: [*Class*], func: r =\textgreater{} classes.at(r.Pclass)),}
\NormalTok{    (header: [*Fare*], func: r =\textgreater{} usd(r.Fare)), }
\NormalTok{    (header: [*Survived?*], func: r =\textgreater{} ("No", "Yes").at(r.Survived)), }
\NormalTok{  ),}
\NormalTok{  fill: (\_, row) =\textgreater{} if calc.odd(row) \{ luma(240) \} else \{ luma(220) \}, }
\NormalTok{  stroke: none}
\NormalTok{)}
\end{Highlighting}
\end{Shaded}

\pandocbounded{\includesvg[keepaspectratio]{https://github.com/typst/packages/raw/main/packages/preview/tabut/1.0.2/doc/compiled-snippets/filter.svg}}

\subsection[Aggregation using Map and Sum ]{\texorpdfstring{Aggregation
using Map and Sum \protect\hypertarget{aggregation-using-map-and-sum}{}{
}}{Aggregation using Map and Sum  }}\label{aggregation-using-map-and-sum}

\begin{Shaded}
\begin{Highlighting}[]
\NormalTok{\#import "usd.typ": usd}
\NormalTok{\#import "example{-}data/titanic.typ": titanic, classes}

\NormalTok{\#table(}
\NormalTok{  columns: (auto, auto),}
\NormalTok{  [*Fare, Total:*], [\#usd(titanic.map(r =\textgreater{} r.Fare).sum())],}
\NormalTok{  [*Fare, Avg:*], [\#usd(titanic.map(r =\textgreater{} r.Fare).sum() / titanic.len())], }
\NormalTok{  stroke: none}
\NormalTok{)}
\end{Highlighting}
\end{Shaded}

\pandocbounded{\includesvg[keepaspectratio]{https://github.com/typst/packages/raw/main/packages/preview/tabut/1.0.2/doc/compiled-snippets/aggregation.svg}}

\subsection[Grouping ]{\texorpdfstring{Grouping
\protect\hypertarget{grouping}{}{ }}{Grouping  }}\label{grouping}

\begin{Shaded}
\begin{Highlighting}[]
\NormalTok{\#import "@preview/tabut:1.0.2": tabut, group}
\NormalTok{\#import "example{-}data/titanic.typ": titanic, classes}

\NormalTok{\#tabut(}
\NormalTok{  group(titanic, r =\textgreater{} r.Pclass),}
\NormalTok{  (}
\NormalTok{    (header: [*Class*], func: r =\textgreater{} classes.at(r.value)), }
\NormalTok{    (header: [*Passengers*], func: r =\textgreater{} r.group.len()), }
\NormalTok{  ),}
\NormalTok{  fill: (\_, row) =\textgreater{} if calc.odd(row) \{ luma(240) \} else \{ luma(220) \}, }
\NormalTok{  stroke: none}
\NormalTok{)}
\end{Highlighting}
\end{Shaded}

\pandocbounded{\includesvg[keepaspectratio]{https://github.com/typst/packages/raw/main/packages/preview/tabut/1.0.2/doc/compiled-snippets/group.svg}}

\begin{Shaded}
\begin{Highlighting}[]
\NormalTok{\#import "@preview/tabut:1.0.2": tabut, group}
\NormalTok{\#import "usd.typ": usd}
\NormalTok{\#import "example{-}data/titanic.typ": titanic, classes}

\NormalTok{\#tabut(}
\NormalTok{  group(titanic, r =\textgreater{} r.Pclass),}
\NormalTok{  (}
\NormalTok{    (header: [*Class*], func: r =\textgreater{} classes.at(r.value)), }
\NormalTok{    (header: [*Total Fare*], func: r =\textgreater{} usd(r.group.map(r =\textgreater{} r.Fare).sum())), }
\NormalTok{    (}
\NormalTok{      header: [*Avg Fare*], }
\NormalTok{      func: r =\textgreater{} usd(r.group.map(r =\textgreater{} r.Fare).sum() / r.group.len())}
\NormalTok{    ), }
\NormalTok{  ),}
\NormalTok{  fill: (\_, row) =\textgreater{} if calc.odd(row) \{ luma(240) \} else \{ luma(220) \}, }
\NormalTok{  stroke: none}
\NormalTok{)}
\end{Highlighting}
\end{Shaded}

\pandocbounded{\includesvg[keepaspectratio]{https://github.com/typst/packages/raw/main/packages/preview/tabut/1.0.2/doc/compiled-snippets/group-aggregation.svg}}

\subsection[\texttt{\ tabut\ } ]{\texorpdfstring{\texttt{\ tabut\ }
\protect\hypertarget{tabut}{}{ }}{ tabut   }}\label{tabut-1}

Takes data and column definitions and outputs a table.

\begin{Shaded}
\begin{Highlighting}[]
\NormalTok{tabut(}
\NormalTok{  data{-}raw, }
\NormalTok{  colDefs, }
\NormalTok{  columns: auto,}
\NormalTok{  align: auto,}
\NormalTok{  index: "\_index",}
\NormalTok{  transpose: false,}
\NormalTok{  headers: true,}
\NormalTok{  ..tableArgs}
\NormalTok{) {-}\textgreater{} content}
\end{Highlighting}
\end{Shaded}

\subsubsection{Parameters}\label{parameters}

\texttt{\ data-raw\ }\strut \\
This is the raw data that will be used to generate the table. The data
is expected to be in an array of dictionaries, where each dictionary
represents a single record or object.

\texttt{\ colDefs\ }\strut \\
These are the column definitions. An array of dictionaries, each
representing column definition. Must include the properties
\texttt{\ header\ } and a \texttt{\ func\ } . \texttt{\ header\ }
expects content, and specifies the label of the column.
\texttt{\ func\ } expects a function, the function takes a record
dictionary as input and returns the value to be displayed in the cell
corresponding to that record and column. There are also two optional
properties; \texttt{\ align\ } sets the alignment of the content within
the cells of the column, \texttt{\ width\ } sets the width of the
column.

\texttt{\ columns\ }\strut \\
(optional, default: \texttt{\ auto\ } ) Specifies the column widths. If
set to \texttt{\ auto\ } , the function automatically generates column
widths by each column’s column definition. Otherwise functions exactly
the \texttt{\ columns\ } paramater of the standard Typst
\texttt{\ table\ } function. Unlike the \texttt{\ tabut-cells\ } setting
this to \texttt{\ none\ } will break.

\texttt{\ align\ }\strut \\
(optional, default: \texttt{\ auto\ } ) Specifies the column alignment.
If set to \texttt{\ auto\ } , the function automatically generates
column alignment by each column’s column definition. If set to
\texttt{\ none\ } no \texttt{\ align\ } property is added to the output
arg. Otherwise functions exactly the \texttt{\ align\ } paramater of the
standard Typst \texttt{\ table\ } function.

\texttt{\ index\ }\strut \\
(optional, default: \texttt{\ "\_index"\ } ) Specifies the property name
for the index of each record. By default, an \texttt{\ \_index\ }
property is automatically added to each record. If set to
\texttt{\ none\ } , no index property is added.

\texttt{\ transpose\ }\strut \\
(optional, default: \texttt{\ false\ } ) If set to \texttt{\ true\ } ,
transposes the table, swapping rows and columns.

\texttt{\ headers\ }\strut \\
(optional, default: \texttt{\ true\ } ) Determines whether headers
should be included in the output. If set to \texttt{\ false\ } , headers
are not generated.

\texttt{\ tableArgs\ }\strut \\
(optional) Any additional arguments are passed to the \texttt{\ table\ }
function, can be used for styling or anything else.

\subsection[\texttt{\ tabut-cells\ }
]{\texorpdfstring{\texttt{\ tabut-cells\ }
\protect\hypertarget{tabut-cells}{}{
}}{ tabut-cells   }}\label{tabut-cells}

The \texttt{\ tabut-cells\ } function functions as \texttt{\ tabut\ } ,
but returns \texttt{\ arguments\ } for use in either the standard
\texttt{\ table\ } function or other tools such as \texttt{\ tablex\ } .
If you just want the array of cells, use the \texttt{\ pos\ } function
on the returned value, ex \texttt{\ tabut-cells(...).pos\ } .

\texttt{\ tabut-cells\ } is particularly useful when you need to
generate only the cell contents of a table or when these cells need to
be passed to another function for further processing or customization.

\subsubsection{Function Signature}\label{function-signature}

\begin{Shaded}
\begin{Highlighting}[]
\NormalTok{tabut{-}cells(}
\NormalTok{  data{-}raw, }
\NormalTok{  colDefs, }
\NormalTok{  columns: auto,}
\NormalTok{  align: auto,}
\NormalTok{  index: "\_index",}
\NormalTok{  transpose: false,}
\NormalTok{  headers: true,}
\NormalTok{) {-}\textgreater{} arguments}
\end{Highlighting}
\end{Shaded}

\subsubsection{Parameters}\label{parameters-1}

\texttt{\ data-raw\ }\strut \\
This is the raw data that will be used to generate the table. The data
is expected to be in an array of dictionaries, where each dictionary
represents a single record or object.

\texttt{\ colDefs\ }\strut \\
These are the column definitions. An array of dictionaries, each
representing column definition. Must include the properties
\texttt{\ header\ } and a \texttt{\ func\ } . \texttt{\ header\ }
expects content, and specifies the label of the column.
\texttt{\ func\ } expects a function, the function takes a record
dictionary as input and returns the value to be displayed in the cell
corresponding to that record and column. There are also two optional
properties; \texttt{\ align\ } sets the alignment of the content within
the cells of the column, \texttt{\ width\ } sets the width of the
column.

\texttt{\ columns\ }\strut \\
(optional, default: \texttt{\ auto\ } ) Specifies the column widths. If
set to \texttt{\ auto\ } , the function automatically generates column
widths by each column’s column definition. If set to \texttt{\ none\ }
no \texttt{\ column\ } property is added to the output arg. Otherwise
functions exactly the \texttt{\ columns\ } paramater of the standard
typst \texttt{\ table\ } function.

\texttt{\ align\ }\strut \\
(optional, default: \texttt{\ auto\ } ) Specifies the column alignment.
If set to \texttt{\ auto\ } , the function automatically generates
column alignment by each column’s column definition. If set to
\texttt{\ none\ } no \texttt{\ align\ } property is added to the output
arg. Otherwise functions exactly the \texttt{\ align\ } paramater of the
standard typst \texttt{\ table\ } function.

\texttt{\ index\ }\strut \\
(optional, default: \texttt{\ "\_index"\ } ) Specifies the property name
for the index of each record. By default, an \texttt{\ \_index\ }
property is automatically added to each record. If set to
\texttt{\ none\ } , no index property is added.

\texttt{\ transpose\ }\strut \\
(optional, default: \texttt{\ false\ } ) If set to \texttt{\ true\ } ,
transposes the table, swapping rows and columns.

\texttt{\ headers\ }\strut \\
(optional, default: \texttt{\ true\ } ) Determines whether headers
should be included in the output. If set to \texttt{\ false\ } , headers
are not generated.

\subsection[\texttt{\ records-from-csv\ }
]{\texorpdfstring{\texttt{\ records-from-csv\ }
\protect\hypertarget{records-from-csv}{}{
}}{ records-from-csv   }}\label{records-from-csv}

Automatically converts a CSV data into an array of records.

\begin{Shaded}
\begin{Highlighting}[]
\NormalTok{records{-}from{-}csv(}
\NormalTok{  data}
\NormalTok{) {-}\textgreater{} array}
\end{Highlighting}
\end{Shaded}

\subsubsection{Parameters}\label{parameters-2}

\texttt{\ data\ }\strut \\
The CSV data that needs to be converted, this can be obtained using the
native \texttt{\ csv\ } function, like
\texttt{\ records-from-csv(csv(file-path))\ } .

This function simplifies the process of converting CSV data into a
format compatible with \texttt{\ tabut\ } . It reads the CSV data,
extracts the headers, and converts each row into a dictionary, using the
headers as keys.

It also automatically converts data into floats or integers when
possible.

\subsection[\texttt{\ rows-to-records\ }
]{\texorpdfstring{\texttt{\ rows-to-records\ }
\protect\hypertarget{rows-to-records}{}{
}}{ rows-to-records   }}\label{rows-to-records}

Converts rows of data into an array of records based on specified
headers.

This function is useful for converting data in a “rows� format
(commonly found in CSV files) into an array of dictionaries format,
which is required for \texttt{\ tabut\ } and allows easy data processing
using the built in array functions.

\begin{Shaded}
\begin{Highlighting}[]
\NormalTok{rows{-}to{-}records(}
\NormalTok{  headers, }
\NormalTok{  rows, }
\NormalTok{  default: none}
\NormalTok{) {-}\textgreater{} array}
\end{Highlighting}
\end{Shaded}

\subsubsection{Parameters}\label{parameters-3}

\texttt{\ headers\ }\strut \\
An array representing the headers of the table. Each item in this array
corresponds to a column header.

\texttt{\ rows\ }\strut \\
An array of arrays, each representing a row of data. Each sub-array
contains the cell data for a corresponding row.

\texttt{\ default\ }\strut \\
(optional, default: \texttt{\ none\ } ) A default value to use when a
cell is empty or there is an error.

\subsection[\texttt{\ group\ } ]{\texorpdfstring{\texttt{\ group\ }
\protect\hypertarget{group}{}{ }}{ group   }}\label{group}

Groups data based on a specified function and returns an array of
grouped records.

\begin{Shaded}
\begin{Highlighting}[]
\NormalTok{group(}
\NormalTok{  data, }
\NormalTok{  function}
\NormalTok{) {-}\textgreater{} array}
\end{Highlighting}
\end{Shaded}

\subsubsection{Parameters}\label{parameters-4}

\texttt{\ data\ }\strut \\
An array of dictionaries. Each dictionary represents a single record or
object.

\texttt{\ function\ }\strut \\
A function that takes a record as input and returns a value based on
which the grouping is to be performed.

This function iterates over each record in the \texttt{\ data\ } ,
applies the \texttt{\ function\ } to determine the grouping value, and
organizes the records into groups based on this value. Each group record
is represented as a dictionary with two properties: \texttt{\ value\ }
(the result of the grouping function) and \texttt{\ group\ } (an array
of records belonging to this group).

In the context of \texttt{\ tabut\ } , the \texttt{\ group\ } function
is particularly useful for creating summary tables where records need to
be categorized and aggregated based on certain criteria, such as
calculating total or average values for each group.

\subsubsection{How to add}\label{how-to-add}

Copy this into your project and use the import as \texttt{\ tabut\ }

\begin{verbatim}
#import "@preview/tabut:1.0.2"
\end{verbatim}

\includesvg[width=0.16667in,height=0.16667in]{/assets/icons/16-copy.svg}

Check the docs for
\href{https://typst.app/docs/reference/scripting/\#packages}{more
information on how to import packages} .

\subsubsection{About}\label{about}

\begin{description}
\tightlist
\item[Author :]
\href{https://github.com/Amelia-Mowers}{Amelia Mowers}
\item[License:]
MIT
\item[Current version:]
1.0.2
\item[Last updated:]
April 16, 2024
\item[First released:]
January 29, 2024
\item[Archive size:]
9.40 kB
\href{https://packages.typst.org/preview/tabut-1.0.2.tar.gz}{\pandocbounded{\includesvg[keepaspectratio]{/assets/icons/16-download.svg}}}
\item[Repository:]
\href{https://github.com/Amelia-Mowers/typst-tabut}{GitHub}
\end{description}

\subsubsection{Where to report issues?}\label{where-to-report-issues}

This package is a project of Amelia Mowers . Report issues on
\href{https://github.com/Amelia-Mowers/typst-tabut}{their repository} .
You can also try to ask for help with this package on the
\href{https://forum.typst.app}{Forum} .

Please report this package to the Typst team using the
\href{https://typst.app/contact}{contact form} if you believe it is a
safety hazard or infringes upon your rights.

\phantomsection\label{versions}
\subsubsection{Version history}\label{version-history}

\begin{longtable}[]{@{}ll@{}}
\toprule\noalign{}
Version & Release Date \\
\midrule\noalign{}
\endhead
\bottomrule\noalign{}
\endlastfoot
1.0.2 & April 16, 2024 \\
\href{https://typst.app/universe/package/tabut/1.0.1/}{1.0.1} & January
31, 2024 \\
\href{https://typst.app/universe/package/tabut/1.0.0/}{1.0.0} & January
29, 2024 \\
\end{longtable}

Typst GmbH did not create this package and cannot guarantee correct
functionality of this package or compatibility with any version of the
Typst compiler or app.
