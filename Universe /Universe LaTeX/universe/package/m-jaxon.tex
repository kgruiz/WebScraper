\title{typst.app/universe/package/m-jaxon}

\phantomsection\label{banner}
\section{m-jaxon}\label{m-jaxon}

{ 0.1.1 }

Render LaTeX equation in typst using MathJax.

\phantomsection\label{readme}
Render LaTeX equation in typst using MathJax.

\textbf{Note:} This package is made for fun and to demonstrate the
capability of typst plugins. And it is \textbf{slow} . To actually
convert LaTeX equations to typst ones, you should use \textbf{pandoc} or
\textbf{texmath} .

\pandocbounded{\includesvg[keepaspectratio]{https://github.com/typst/packages/raw/main/packages/preview/m-jaxon/0.1.1/mj.svg}}

\begin{Shaded}
\begin{Highlighting}[]
\NormalTok{\#import "./typst{-}package/lib.typ" as m{-}jaxon}
\NormalTok{// Uncomment the following line to use the m{-}jaxon from the official package registry}
\NormalTok{// \#import "@preview/m{-}jaxon:0.1.1"}

\NormalTok{= M{-}Jaxon}

\NormalTok{Typst, now with *MathJax*.}

\NormalTok{The equation of mass{-}energy equivalence is often written as $E=m c\^{}2$ in modern physics.}

\NormalTok{But we can also write it using M{-}Jaxon as: \#m{-}jaxon.render("E = mc\^{}2", inline: true)}
\end{Highlighting}
\end{Shaded}

\subsection{Limitations}\label{limitations}

\begin{itemize}
\tightlist
\item
  The baseline of the inline equation still looks a bit off.
\end{itemize}

\subsection{Documentation}\label{documentation}

\subsubsection{\texorpdfstring{\texttt{\ render\ }}{ render }}\label{render}

Render a LaTeX equation string to an svg image. Depending on the
\texttt{\ inline\ } argument, the image will be rendered as an inline
image or a block image.

\paragraph{Arguments}\label{arguments}

\begin{itemize}
\tightlist
\item
  \texttt{\ src\ } : \texttt{\ str\ } or \texttt{\ raw\ } block - The
  LaTeX equation string
\item
  \texttt{\ inline\ } : \texttt{\ bool\ } - Whether to render the image
  as an inline image or a block image
\end{itemize}

\paragraph{Returns}\label{returns}

The image, of type \texttt{\ content\ }

\subsubsection{How to add}\label{how-to-add}

Copy this into your project and use the import as \texttt{\ m-jaxon\ }

\begin{verbatim}
#import "@preview/m-jaxon:0.1.1"
\end{verbatim}

\includesvg[width=0.16667in,height=0.16667in]{/assets/icons/16-copy.svg}

Check the docs for
\href{https://typst.app/docs/reference/scripting/\#packages}{more
information on how to import packages} .

\subsubsection{About}\label{about}

\begin{description}
\tightlist
\item[Author :]
Wenzhuo Liu
\item[License:]
MIT
\item[Current version:]
0.1.1
\item[Last updated:]
January 17, 2024
\item[First released:]
December 14, 2023
\item[Archive size:]
633 kB
\href{https://packages.typst.org/preview/m-jaxon-0.1.1.tar.gz}{\pandocbounded{\includesvg[keepaspectratio]{/assets/icons/16-download.svg}}}
\item[Repository:]
\href{https://github.com/Enter-tainer/m-jaxon}{GitHub}
\end{description}

\subsubsection{Where to report issues?}\label{where-to-report-issues}

This package is a project of Wenzhuo Liu . Report issues on
\href{https://github.com/Enter-tainer/m-jaxon}{their repository} . You
can also try to ask for help with this package on the
\href{https://forum.typst.app}{Forum} .

Please report this package to the Typst team using the
\href{https://typst.app/contact}{contact form} if you believe it is a
safety hazard or infringes upon your rights.

\phantomsection\label{versions}
\subsubsection{Version history}\label{version-history}

\begin{longtable}[]{@{}ll@{}}
\toprule\noalign{}
Version & Release Date \\
\midrule\noalign{}
\endhead
\bottomrule\noalign{}
\endlastfoot
0.1.1 & January 17, 2024 \\
\href{https://typst.app/universe/package/m-jaxon/0.1.0/}{0.1.0} &
December 14, 2023 \\
\end{longtable}

Typst GmbH did not create this package and cannot guarantee correct
functionality of this package or compatibility with any version of the
Typst compiler or app.
