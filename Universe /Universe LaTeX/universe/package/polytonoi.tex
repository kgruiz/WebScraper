\title{typst.app/universe/package/polytonoi}

\phantomsection\label{banner}
\section{polytonoi}\label{polytonoi}

{ 0.1.0 }

Renders Roman letters into polytonic Greek.

\phantomsection\label{readme}
A typst package for rendering text into polytonic Greek using a
hopefully-intuitive transliteration scheme.

\subsection{Usage}\label{usage}

First, be sure you include the package at the top of your typst file:

\begin{Shaded}
\begin{Highlighting}[]
\NormalTok{@import "preview/polytonoi@0.1.0: *}
\end{Highlighting}
\end{Shaded}

The package currently exposes one function,
\texttt{\ \#ptgk(\textless{}string\textgreater{})\ } , which will
convert \texttt{\ \textless{}string\textgreater{}\ } into polytonic
Greek text in the same location where the function appears in the typst
document.

For example: \texttt{\ \#ptgk("polu/s")\ } would render: πολÏ?Ï‚

\textbf{NOTE:} Quotation marks within the function call (see above
example) are \textbf{mandatory} , and the code will not work without
them.

Where possible, Greek letters have been linked with their closest Roman
equivalent (e.g. a -\/-\textgreater{} α, b -\/-\textgreater{} β).
Where not possible, I’ve tried to stick to common convention, such as
what is used by the Perseus Project for their transliteration. A couple
letters (ξ and ψ) are made up of two letters ( \texttt{\ ks\ } and
\texttt{\ ps\ } respectively), which the \texttt{\ \#ptgk()\ } function
handles as special cases. See below for the full transliteration scheme.

\paragraph{Additional Usage Notes}\label{additional-usage-notes}

\begin{enumerate}
\tightlist
\item
  Any character that isn’t specifically accounted for (including white
  space, most punctuation, numbers, etc.) is rendered as-is.
\item
  Smooth breathing marks are automatically added to a vowel that begins
  a word, unless that first vowel is followed by another. In this case,
  you’ll need to manually add it to the second vowel.
\end{enumerate}

\subsubsection{Text Formatting}\label{text-formatting}

As of now, the text is processed as a string, which means that any
formatting markup (such as \texttt{\ \_\ } or \texttt{\ *\ } ) is
\textbf{not} included in how the text is rendered, and is instead passed
through and will display literally. To apply any kind of formatting to
the Greek text, the markup or commands must be put outside the text
passed to the function. Compare the following two examples to see how
this works:

\texttt{\ \#ptgk("\_Arxh\textbackslash{}\_")\ } would display as
\_ἈÏ?χὴ\_

whereas

\texttt{\ \_\#ptgk("Arxh\textbackslash{}")\_\ } would display as
\emph{ἈÏ?χὴ}

\subsection{Transliteration Scheme}\label{transliteration-scheme}

\begin{longtable}[]{@{}lll@{}}
\toprule\noalign{}
Roman letter & Greek result & Notes \\
\midrule\noalign{}
\endhead
\bottomrule\noalign{}
\endlastfoot
a & α & \\
b & β & \\
g & γ & \\
d & δ & \\
e & ε & \\
z & ζ & \\
h & η & \\
q & θ & \\
i & ι & \\
k & κ & \\
l & λ & \\
m & μ & \\
n & ν & \\
ks & ξ & \\
o & ο & \\
p & π & \\
r & Ï? & \\
s & σ/ς & Renders as final sigma automatically \\
t & Ï„ & \\
u & Ï & \\
v & φ & \\
x & χ & \\
ps & ψ & \\
w & ω & \\
\end{longtable}

Upper-case letters are handled the same way, just with an upper-case
letter as input. The upper-case versions of the two “combined�
letters (Ξ and Ψ) can be entered either as “KS�/“PS� or
“Ks�/“Ps�.

\subsubsection{Accents \& Breathing
Marks}\label{accents-breathing-marks}

As mentioned above, this package will automatically put a smooth
breathing mark on a vowel that begins a word, unless that vowel is
followed immediately by a second vowel. In that instance, you’ll have
to manually put the smooth breathing mark in its correct place. (This is
to avoid having to code for edge cases, such as where a word starts with
three vowels in a row.) By the same token, rough breathing must always
be entered manually.

\begin{longtable}[]{@{}llll@{}}
\toprule\noalign{}
Input & Greek & Notes & Example \\
\midrule\noalign{}
\endhead
\bottomrule\noalign{}
\endlastfoot
( & rough breathing & Put before the vowel & \texttt{\ (a\ }
-\/-\textgreater{} á¼? \\
) & smooth breathing & Put before the vowel & \texttt{\ )a\ }
-\/-\textgreater{} á¼€ \\
\textbackslash{} & Grave / varia & Put after the vowel &
\texttt{\ a\textbackslash{}\ } -\/-\textgreater{} á½° \\
/ & Acute / oxia / tonos & Put after the vowel & \texttt{\ a/\ }
-\/-\textgreater{} ά \\
= & Tilde / perispomeni & Put after the vowel & \texttt{\ a=\ }
-\/-\textgreater{} ᾶ \\
\textbar{} & Iota subscript & Put after the vowel &
\texttt{\ a\textbar{}\ } -\/-\textgreater{} á¾³ \\
: & Diaresis & Put after the vowel & \texttt{\ i:\ } -\/-\textgreater{}
ÏŠ \\
\end{longtable}

Multiple diacriticals can be added to a vowel, e.g.
\texttt{\ (h\textbar{}\ } -\/-\textgreater{} á¾`

\subsubsection{Punctuation}\label{punctuation}

Most Roman punctuation characters are left unchanged. The exceptions are
\texttt{\ ;\ } (semicolon) and \texttt{\ ?\ } (question mark), which are
rendered as a high dot (·) and the Greek question mark (;)
respectively.

\subsection{Feedback}\label{feedback}

Feedback is welcome, and please don’t hesitate to open an issue if
something doesn’t work. I’ve tried to account for edge cases, but I
certainly can’t guarantee that I’ve found everything.

\subsubsection{How to add}\label{how-to-add}

Copy this into your project and use the import as \texttt{\ polytonoi\ }

\begin{verbatim}
#import "@preview/polytonoi:0.1.0"
\end{verbatim}

\includesvg[width=0.16667in,height=0.16667in]{/assets/icons/16-copy.svg}

Check the docs for
\href{https://typst.app/docs/reference/scripting/\#packages}{more
information on how to import packages} .

\subsubsection{About}\label{about}

\begin{description}
\tightlist
\item[Author :]
Dei Layborer
\item[License:]
GPL-3.0-only
\item[Current version:]
0.1.0
\item[Last updated:]
December 28, 2023
\item[First released:]
December 28, 2023
\item[Archive size:]
15.6 kB
\href{https://packages.typst.org/preview/polytonoi-0.1.0.tar.gz}{\pandocbounded{\includesvg[keepaspectratio]{/assets/icons/16-download.svg}}}
\item[Repository:]
\href{https://github.com/dei-layborer/polytonoi}{GitHub}
\end{description}

\subsubsection{Where to report issues?}\label{where-to-report-issues}

This package is a project of Dei Layborer . Report issues on
\href{https://github.com/dei-layborer/polytonoi}{their repository} . You
can also try to ask for help with this package on the
\href{https://forum.typst.app}{Forum} .

Please report this package to the Typst team using the
\href{https://typst.app/contact}{contact form} if you believe it is a
safety hazard or infringes upon your rights.

\phantomsection\label{versions}
\subsubsection{Version history}\label{version-history}

\begin{longtable}[]{@{}ll@{}}
\toprule\noalign{}
Version & Release Date \\
\midrule\noalign{}
\endhead
\bottomrule\noalign{}
\endlastfoot
0.1.0 & December 28, 2023 \\
\end{longtable}

Typst GmbH did not create this package and cannot guarantee correct
functionality of this package or compatibility with any version of the
Typst compiler or app.
