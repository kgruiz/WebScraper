\title{typst.app/universe/package/pillar}

\phantomsection\label{banner}
\section{pillar}\label{pillar}

{ 0.2.0 }

Faster column specifications for tables.

\phantomsection\label{readme}
\emph{Shorthand notations for table column specifications in
\href{https://typst.app/}{Typst} .}

\href{https://typst.app/universe/package/pillar}{\pandocbounded{\includegraphics[keepaspectratio]{https://img.shields.io/badge/dynamic/toml?url=https\%3A\%2F\%2Fraw.githubusercontent.com\%2FMc-Zen\%2Fpillar\%2Fmain\%2Ftypst.toml&query=\%24.package.version&prefix=v&logo=typst&label=package&color=239DAD}}}
\href{https://github.com/Mc-Zen/pillar/actions/workflows/run_tests.yml}{\pandocbounded{\includesvg[keepaspectratio]{https://github.com/Mc-Zen/pillar/actions/workflows/run_tests.yml/badge.svg}}}
\href{https://github.com/Mc-Zen/pillar/blob/main/LICENSE}{\pandocbounded{\includegraphics[keepaspectratio]{https://img.shields.io/badge/license-MIT-blue}}}

\begin{itemize}
\tightlist
\item
  \href{https://github.com/typst/packages/raw/main/packages/preview/pillar/0.2.0/\#introduction}{Introduction}
\item
  \href{https://github.com/typst/packages/raw/main/packages/preview/pillar/0.2.0/\#column-specification}{Column
  specification}
\item
  \href{https://github.com/typst/packages/raw/main/packages/preview/pillar/0.2.0/\#number-alignment}{Number
  alignment}
\item
  \href{https://github.com/typst/packages/raw/main/packages/preview/pillar/0.2.0/\#pillarcols}{\texttt{\ pillar.cols()\ }}
\item
  \href{https://github.com/typst/packages/raw/main/packages/preview/pillar/0.2.0/\#pillartable}{\texttt{\ pillar.table()\ }}
\item
  \href{https://github.com/typst/packages/raw/main/packages/preview/pillar/0.2.0/\#vline-customization}{\texttt{\ vline\ }
  customization}
\end{itemize}

\subsection{Introduction}\label{introduction}

With \textbf{pillar} , you can significantly simplify the column setup
of tables by unifying the specification of the number, alignment, and
separation of columns. This package is in particular designed for
scientific tables, which typically have simple styling:

\pandocbounded{\includegraphics[keepaspectratio]{https://github.com/user-attachments/assets/c0c60651-c682-4968-9ac9-0fa1e8d85dad}}

In order to prepare this table with just the built-in methods, some code
like the following would be required.

\begin{Shaded}
\begin{Highlighting}[]
\NormalTok{\#table(}
\NormalTok{  columns: 5,}
\NormalTok{  align: (center,) * 4 + (right,),}
\NormalTok{  stroke: none,}


\NormalTok{  [Piano Key], table.vline(), [MIDI Number], [Note Name], [Pitch Name], table.vline(), [$f$ in Hz],}
\NormalTok{  ..}
\NormalTok{)}
\end{Highlighting}
\end{Shaded}

Using \textbf{pillar} , the same can be achieved with

\begin{Shaded}
\begin{Highlighting}[]
\NormalTok{\#table(}
\NormalTok{    ..pillar.cols("c|ccc|r"),}

\NormalTok{    [Piano Key], [MIDI Number], [Note Name], [Pitch Name], [$f$ in Hz], ..}
\NormalTok{)}
\end{Highlighting}
\end{Shaded}

or alternatively

\begin{Shaded}
\begin{Highlighting}[]
\NormalTok{\#pillar.table(}
\NormalTok{    cols: "c|ccc|r",}

\NormalTok{    [Piano Key], [MIDI Number], [Note Name], [Pitch Name], [$f$ in Hz], ..}
\NormalTok{)}
\end{Highlighting}
\end{Shaded}

\textbf{Pillar} is designed for interoperability. It uses the powerful
standard tables and provides generated arguments for \texttt{\ table\ }
’s \texttt{\ columns\ } , \texttt{\ align\ } , \texttt{\ stroke\ } ,
and for the specified vertical lines. This means that all features of
the built-in tables (and also \texttt{\ show\ } and \texttt{\ set\ }
rules) can be applied as usual.

\subsection{Column specification}\label{column-specification}

This package works with \emph{column specification strings} . Each
column is described by its alignment which can be \texttt{\ l\ } (left),
\texttt{\ c\ } (center), \texttt{\ r\ } (right), or \texttt{\ a\ }
(auto).

The width of a column can optionally be specified by appending a
(relative) length, or fraction in square brackets to the alignment
specifier, e.g., \texttt{\ c{[}2cm{]}\ } or \texttt{\ r{[}1fr{]}\ } .

Vertical lines can be added between columns with a
\texttt{\ \textbar{}\ } character. Double lines can be produced with
\texttt{\ \textbar{}\textbar{}\ } (see
\href{https://github.com/typst/packages/raw/main/packages/preview/pillar/0.2.0/\#vline-customization}{\texttt{\ vline\ }
customization} ). The stroke of the vertical line can be changed by
appending anything that is usually allowed as a stroke argument in
square brackets, e.g., \texttt{\ \textbar{}{[}2pt{]}\ } ,
\texttt{\ \textbar{}{[}red{]}\ } or
\texttt{\ \textbar{}{[}(dash:\ \textbackslash{}"dashed\textbackslash{}"){]}\ }
.

A column specification string may contain any number of spaces (e.g., to
improve readability) â€'' all spaces will be ignored.

\emph{If you find yourself writing highly complex column specifications,
consider not using this package and resort to the parameters that the
built-in tables provide. This package is intended for quick and
relatively simple column specifications.}

\subsection{Number alignment}\label{number-alignment}

Choosing capital letters \texttt{\ L\ } , \texttt{\ C\ } ,
\texttt{\ R\ } , or \texttt{\ A\ } instead of lower-case letters
activates number alignment at the decimal separator for a specific
column (similar to the column type “S� of the LaTeX package
\href{https://github.com/josephwright/siunitx}{siunitx} ). This feature
is provided via the Typst package \textbf{Zero} .
\href{https://github.com/Mc-Zen/zero}{Here} you can read up on the
configuration of number formatting.

\begin{Shaded}
\begin{Highlighting}[]
\NormalTok{\#let percm = $"cm"\^{}({-}1)$}

\NormalTok{\#pillar.table(}
\NormalTok{  cols: "l|CCCC",}
\NormalTok{  [], [$Δ ν\_0$ in \#percm], [$B\textquotesingle{}\_ν$ in \#percm], [$B\textquotesingle{}\textquotesingle{}\_ν$ in \#percm], [$D\textquotesingle{}\_ν$ in \#percm],}
\NormalTok{  table.hline(),}
\NormalTok{  [Measurement], [14525.278],   [1.41],    [1.47],    [1.5e{-}5],}
\NormalTok{  [Uncertainty], [2],           [0.3],     [0.3],     [4e{-}6],}
\NormalTok{  [Ref. [2]],    [14525,74856], [1.37316], [1.43777], [5.401e{-}6]}
\NormalTok{)}
\end{Highlighting}
\end{Shaded}

\pandocbounded{\includegraphics[keepaspectratio]{https://github.com/user-attachments/assets/066cd34e-7043-48c7-b067-e3256e942f14}}

Non-number entries (e.g., in the header) are automatically recognized in
some cases and will not be aligned. In ambiguous cases, adding a leading
or trailing space tells Zero not to apply alignment to this cell, e.g.,
\texttt{\ {[}Voltage\ {]}\ } instead of \texttt{\ {[}Voltage{]}\ } .

\subsection{\texorpdfstring{\texttt{\ pillar.cols()\ }}{ pillar.cols() }}\label{pillar.cols}

This function produces an argument list that may contain arguments for
\texttt{\ columns\ } , \texttt{\ align\ } , \texttt{\ stroke\ } , and
\texttt{\ column-gutter\ } as well as instances of
\texttt{\ table.vline()\ } . These arguments are intended to be expanded
with the \texttt{\ ..\ } syntax into the argument list of
\texttt{\ table\ } as shown in the examples.

\subsection{\texorpdfstring{\texttt{\ pillar.table()\ }}{ pillar.table() }}\label{pillar.table}

This is a thin wrapper that behaves just like the built-in
\texttt{\ table\ } , except that it extracts the first positional
argument if it is a string, and runs it through
\texttt{\ pillar.cols()\ } .

\subsection{\texorpdfstring{\texttt{\ vline\ }
customization}{ vline  customization}}\label{vline-customization}

In order to customize the default line setting, just use set rules on
\texttt{\ table.vline\ } , e.g.,

\begin{Shaded}
\begin{Highlighting}[]
\NormalTok{\#set table.vline(stroke: .7pt)}

\NormalTok{\#table(..pillar.cols("c|cc"), ..)}
\end{Highlighting}
\end{Shaded}

Double lines are currently experimental and are realized through column
gutters. They could also be realized through patterns, but this can
produce artifacts with some renderers. As it currently is, double lines
are not supported before the first and after the last column. On the
other hand, with the current method, double lines are styled with set
rules on \texttt{\ table.vline\ } which is nice and not achievable in
the same way with patterns.

\subsection{Examples}\label{examples}

\subsubsection{Double lines}\label{double-lines}

The following example uses a double line for visually separating
repeated table columns.

\begin{Shaded}
\begin{Highlighting}[]
\NormalTok{\#pillar.table(}
\NormalTok{  cols: "ccc ||[.7pt] ccc",}
  
\NormalTok{  ..([\textbackslash{}\#], [$α$ in °], [$β$ in °]) * 2,}
\NormalTok{  table.hline(),}
\NormalTok{  [1], [34.3], [11.1],  [6], [34.0], [12.9],}
\NormalTok{  [2], [34.2], [11.2],  [7], [34.3], [12.8],}
\NormalTok{  [3], [34.6], [11.4],  [8], [33.9], [11.9],}
\NormalTok{  [4], [34.7], [10.3],  [9], [34.4], [11.8],}
\NormalTok{  [5], [34.3], [11.1], [10], [34.4], [11.8],}
\NormalTok{)}
\end{Highlighting}
\end{Shaded}

\pandocbounded{\includegraphics[keepaspectratio]{https://github.com/user-attachments/assets/e05e7bad-61b6-44f9-af34-5e558f338cdc}}

\subsubsection{Further customization}\label{further-customization}

This example shows the codes of the first ten printable ASCII
characters, demonstrating stroke and column width customization.

\begin{Shaded}
\begin{Highlighting}[]
\NormalTok{\#pillar.table(}
\NormalTok{  cols: "ccc|ccc|[1.8pt + blue] l[5cm]",}
  
\NormalTok{  [Dec],[Hex],[Bin],[Symbol], [HTML code], [HTML name], [Description],}
\NormalTok{  table.hline(),}
\NormalTok{  [32], [20], [00100000], [\&\#32;], [],         [SP], [Space],}
\NormalTok{  [33], [21], [00100001], [\&\#33;], [\&excl;],   [!],  [Exclamation mark],}
\NormalTok{  [34], [22], [00100010], [\&\#34;], [\&quot;],   ["],  [Double quotes],}
\NormalTok{  [35], [23], [00100011], [\&\#35;], [\&num;],    [\textbackslash{}\#], [Number sign],}
\NormalTok{  [36], [24], [00100100], [\&\#36;], [\&dollar;], [\textbackslash{}$], [Dollar sign],}
\NormalTok{  [37], [25], [00100101], [\&\#37;], [\&percnt;], [\%],  [Percent sign],}
\NormalTok{  [38], [26], [00100110], [\&\#38;], [\&amp;],    [\&],  [Ampersand],}
\NormalTok{  [39], [27], [00100111], [\&\#39;], [\&apos;],   [\textquotesingle{}],  [Single quote],}
\NormalTok{  [40], [28], [00101000], [\&\#40;], [\&lparen;], [(],  [Opening parenthesis],}
\NormalTok{  [41], [29], [00101001], [\&\#41;], [\&rparen;], [)],  [Closing parenthesis],}
\NormalTok{)}
\end{Highlighting}
\end{Shaded}

\pandocbounded{\includegraphics[keepaspectratio]{https://github.com/user-attachments/assets/9fae998e-033d-4d7e-9344-fe3778bbd9e6}}

\subsection{Tests}\label{tests}

This package uses
\href{https://github.com/tingerrr/typst-test/}{typst-test} for running
\href{https://github.com/typst/packages/raw/main/packages/preview/pillar/0.2.0/tests/}{tests}
.

\subsubsection{How to add}\label{how-to-add}

Copy this into your project and use the import as \texttt{\ pillar\ }

\begin{verbatim}
#import "@preview/pillar:0.2.0"
\end{verbatim}

\includesvg[width=0.16667in,height=0.16667in]{/assets/icons/16-copy.svg}

Check the docs for
\href{https://typst.app/docs/reference/scripting/\#packages}{more
information on how to import packages} .

\subsubsection{About}\label{about}

\begin{description}
\tightlist
\item[Author :]
\href{https://github.com/Mc-Zen}{Mc-Zen}
\item[License:]
MIT
\item[Current version:]
0.2.0
\item[Last updated:]
August 22, 2024
\item[First released:]
May 27, 2024
\item[Minimum Typst version:]
0.11.0
\item[Archive size:]
5.52 kB
\href{https://packages.typst.org/preview/pillar-0.2.0.tar.gz}{\pandocbounded{\includesvg[keepaspectratio]{/assets/icons/16-download.svg}}}
\item[Categor ies :]
\begin{itemize}
\tightlist
\item[]
\item
  \pandocbounded{\includesvg[keepaspectratio]{/assets/icons/16-chart.svg}}
  \href{https://typst.app/universe/search/?category=visualization}{Visualization}
\item
  \pandocbounded{\includesvg[keepaspectratio]{/assets/icons/16-layout.svg}}
  \href{https://typst.app/universe/search/?category=layout}{Layout}
\end{itemize}
\end{description}

\subsubsection{Where to report issues?}\label{where-to-report-issues}

This package is a project of Mc-Zen . You can also try to ask for help
with this package on the \href{https://forum.typst.app}{Forum} .

Please report this package to the Typst team using the
\href{https://typst.app/contact}{contact form} if you believe it is a
safety hazard or infringes upon your rights.

\phantomsection\label{versions}
\subsubsection{Version history}\label{version-history}

\begin{longtable}[]{@{}ll@{}}
\toprule\noalign{}
Version & Release Date \\
\midrule\noalign{}
\endhead
\bottomrule\noalign{}
\endlastfoot
0.2.0 & August 22, 2024 \\
\href{https://typst.app/universe/package/pillar/0.1.0/}{0.1.0} & May 27,
2024 \\
\end{longtable}

Typst GmbH did not create this package and cannot guarantee correct
functionality of this package or compatibility with any version of the
Typst compiler or app.
