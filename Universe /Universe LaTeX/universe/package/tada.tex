\title{typst.app/universe/package/tada}

\phantomsection\label{banner}
\section{tada}\label{tada}

{ 0.1.0 }

Easy, composable tabular data manipulation

\phantomsection\label{readme}
TaDa provides a set of simple but powerful operations on rows of data. A
full manual is available online:
\url{https://github.com/ntjess/typst-tada/blob/v0.1.0/docs/manual.pdf}

Key features include:

\begin{itemize}
\item
  \textbf{Arithmetic expressions} : Row-wise operations are as simple as
  string expressions with field names
\item
  \textbf{Aggregation} : Any function that operates on an array of
  values can perform row-wise or column-wise aggregation
\item
  \textbf{Data representation} : Handle displaying currencies, floats,
  integers, and more with ease and arbitrary customization
\end{itemize}

Note: This library is in early development. The API is subject to change
especially as typst adds more support for user-defined types.
\textbf{Backwards compatibility is not guaranteed!} Handling of field
info, value types, and more may change substantially with more user
feedback.

\subsection{Importing}\label{importing}

TaDa can be imported as follows:

\subsubsection{From the official packages repository
(recommended):}\label{from-the-official-packages-repository-recommended}

\begin{Shaded}
\begin{Highlighting}[]
\NormalTok{\#import "@preview/tada:0.1.0"}
\end{Highlighting}
\end{Shaded}

\subsubsection{From the source code (not
recommended)}\label{from-the-source-code-not-recommended}

\textbf{Option 1:} You can clone the package directly into your project
directory:

\begin{Shaded}
\begin{Highlighting}[]
\CommentTok{\# In your project directory}
\FunctionTok{git}\NormalTok{ clone https://github.com/ntjess/typst{-}tada.git tada}
\end{Highlighting}
\end{Shaded}

Then import the functionality with

\begin{Shaded}
\begin{Highlighting}[]
\NormalTok{\#import "./tada/lib.typ" }
\end{Highlighting}
\end{Shaded}

\textbf{Option 2:} If Python is available on your system, use the
provided packaging script to install TaDa in typst’s
\texttt{\ local\ } directory:

\begin{Shaded}
\begin{Highlighting}[]
\CommentTok{\# Anywhere on your system}
  \FunctionTok{git}\NormalTok{ clone https://github.com/ntjess/typst{-}tada.git}
  \BuiltInTok{cd}\NormalTok{ typst{-}tada}
  
  \CommentTok{\# Replace $XDG\_CACHE\_HOME with the appropriate directory based on}
  \CommentTok{\# https://github.com/typst/packages\#downloads}
  \ExtensionTok{python}\NormalTok{ package.py ./typst.toml }\StringTok{"}\VariableTok{$XDG\_CACHE\_HOME}\StringTok{/typst/packages"} \DataTypeTok{\textbackslash{}}
    \AttributeTok{{-}{-}namespace}\NormalTok{ local}
  
\end{Highlighting}
\end{Shaded}

Now, TaDa is available under the local namespace:

\begin{Shaded}
\begin{Highlighting}[]
\NormalTok{\#import "@local/tada:0.1.0"}
\end{Highlighting}
\end{Shaded}

\subsection{Creation}\label{creation}

TaDa provides three main ways to construct tables â€`` from columns,
rows, or records.

\begin{itemize}
\item
  \textbf{Columns} are a dictionary of field names to column values.
  Alternatively, a 2D array of columns can be passed to
  \texttt{\ from-columns\ } , where \texttt{\ values.at(0)\ } is a
  column (belongs to one field).
\item
  \textbf{Records} are a 1D array of dictionaries where each dictionary
  is a row.
\item
  \textbf{Rows} are a 2D array where \texttt{\ values.at(0)\ } is a row
  (has one value for each field). Note that if \texttt{\ rows\ } are
  given without field names, they default to (0, 1, …\$n\$).
\end{itemize}

\begin{Shaded}
\begin{Highlighting}[]
\NormalTok{\#let column{-}data = (}
\NormalTok{  name: ("Bread", "Milk", "Eggs"),}
\NormalTok{  price: (1.25, 2.50, 1.50),}
\NormalTok{  quantity: (2, 1, 3),}
\NormalTok{)}
\NormalTok{\#let record{-}data = (}
\NormalTok{  (name: "Bread", price: 1.25, quantity: 2),}
\NormalTok{  (name: "Milk", price: 2.50, quantity: 1),}
\NormalTok{  (name: "Eggs", price: 1.50, quantity: 3),}
\NormalTok{)}
\NormalTok{\#let row{-}data = (}
\NormalTok{  ("Bread", 1.25, 2),}
\NormalTok{  ("Milk", 2.50, 1),}
\NormalTok{  ("Eggs", 1.50, 3),}
\NormalTok{)}

\NormalTok{\#import tada: TableData}
\NormalTok{\#let td = TableData(data: column{-}data)}
\NormalTok{// Equivalent to:}
\NormalTok{\#let td2 = tada.from{-}records(record{-}data)}
\NormalTok{// \_Not\_ equivalent to (since field names are unknown):}
\NormalTok{\#let td3 = tada.from{-}rows(row{-}data)}

\NormalTok{\#to{-}tablex(td)}
\NormalTok{\#to{-}tablex(td2)}
\NormalTok{\#to{-}tablex(td3)}
\end{Highlighting}
\end{Shaded}

\pandocbounded{\includegraphics[keepaspectratio]{https://raw.githubusercontent.com/ntjess/typst-tada/v0.1.0/assets/example-01.png}}

\subsection{Title formatting}\label{title-formatting}

You can pass any \texttt{\ content\ } as a field’s \texttt{\ title\ }
. \textbf{Note} : if you pass a string, it will be evaluated as markup.

\begin{Shaded}
\begin{Highlighting}[]
\NormalTok{\#let fmt(it) = \{}
\NormalTok{  heading(outlined: false,}
\NormalTok{    upper(it.at(0))}
\NormalTok{    + it.slice(1).replace("\_", " ")}
\NormalTok{  )}
\NormalTok{\}}

\NormalTok{\#let titles = (}
\NormalTok{  // As a function}
\NormalTok{  name: (title: fmt),}
\NormalTok{  // As a string}
\NormalTok{  quantity: (title: fmt("Qty")),}
\NormalTok{)}
\NormalTok{\#let td = TableData(..td, field{-}info: titles)}

\NormalTok{\#to{-}tablex(td)}
\end{Highlighting}
\end{Shaded}

\pandocbounded{\includegraphics[keepaspectratio]{https://raw.githubusercontent.com/ntjess/typst-tada/v0.1.0/assets/example-02.png}}

\subsection{Adapting default behavior}\label{adapting-default-behavior}

You can specify defaults for any field not explicitly populated by
passing information to \texttt{\ field-defaults\ } . Observe in the last
example that \texttt{\ price\ } was not given a title. We can indicate
it should be formatted the same as \texttt{\ name\ } by passing
\texttt{\ title:\ fmt\ } to \texttt{\ field-defaults\ } . \textbf{Note}
that any field that is explicitly given a value will not be affected by
\texttt{\ field-defaults\ } (i.e., \texttt{\ quantity\ } will retain its
string title “Qty�)

\begin{Shaded}
\begin{Highlighting}[]
\NormalTok{\#let defaults = (title: fmt)}
\NormalTok{\#let td = TableData(..td, field{-}defaults: defaults)}
\NormalTok{\#to{-}tablex(td)}
\end{Highlighting}
\end{Shaded}

\pandocbounded{\includegraphics[keepaspectratio]{https://raw.githubusercontent.com/ntjess/typst-tada/v0.1.0/assets/example-03.png}}

\subsection{\texorpdfstring{Using
\texttt{\ \_\_index\ }}{Using  \_\_index }}\label{using-__index}

TaDa will automatically add an \texttt{\ \_\_index\ } field to each row
that is hidden by default. If you want it displayed, update its
information to set \texttt{\ hide:\ false\ } :

\begin{Shaded}
\begin{Highlighting}[]
\NormalTok{// Use the helper function \textasciigrave{}update{-}fields\textasciigrave{} to update multiple fields}
\NormalTok{// and/or attributes}
\NormalTok{\#import tada: update{-}fields}
\NormalTok{\#let td = update{-}fields(}
\NormalTok{  td, \_\_index: (hide: false, title: "\textbackslash{}\#")}
\NormalTok{)}
\NormalTok{// You can also insert attributes directly:}
\NormalTok{// \#td.field{-}info.\_\_index.insert("hide", false)}
\NormalTok{// etc.}
\NormalTok{\#to{-}tablex(td)}
\end{Highlighting}
\end{Shaded}

\pandocbounded{\includegraphics[keepaspectratio]{https://raw.githubusercontent.com/ntjess/typst-tada/v0.1.0/assets/example-04.png}}

\subsection{Value formatting}\label{value-formatting}

\subsubsection{\texorpdfstring{\texttt{\ type\ }}{ type }}\label{type}

Type information can have attached metadata that specifies alignment,
display formats, and more. Available types and their metadata are:

\begin{itemize}
\item
  \textbf{string} : (default-value: "", align: left)
\item
  \textbf{content} : (display: , align: left)
\item
  \textbf{float} : (align: right)
\item
  \textbf{integer} : (align: right)
\item
  \textbf{percent} : (display: , align: right)
\item
  \textbf{index} : (align: right)
\end{itemize}

While adding your own default types is not yet supported, you can simply
defined a dictionary of specifications and pass its keys to the field

\begin{Shaded}
\begin{Highlighting}[]
\NormalTok{\#let currency{-}info = (}
\NormalTok{  display: tada.display.format{-}usd, align: right}
\NormalTok{)}
\NormalTok{\#td.field{-}info.insert("price", (type: "currency"))}
\NormalTok{\#let td = TableData(..td, type{-}info: ("currency": currency{-}info))}
\NormalTok{\#to{-}tablex(td)}
\end{Highlighting}
\end{Shaded}

\pandocbounded{\includegraphics[keepaspectratio]{https://raw.githubusercontent.com/ntjess/typst-tada/v0.1.0/assets/example-05.png}}

\subsection{Transposing}\label{transposing}

\texttt{\ transpose\ } is supported, but keep in mind if columns have
different types, an error will be a frequent result. To avoid the error,
explicitly pass \texttt{\ ignore-types:\ true\ } . You can choose
whether to keep field names as an additional column by passing a string
to \texttt{\ fields-name\ } that is evaluated as markup:

\begin{Shaded}
\begin{Highlighting}[]
\NormalTok{\#to{-}tablex(}
\NormalTok{  tada.transpose(}
\NormalTok{    td, ignore{-}types: true, fields{-}name: ""}
\NormalTok{  )}
\NormalTok{)}
\end{Highlighting}
\end{Shaded}

\pandocbounded{\includegraphics[keepaspectratio]{https://raw.githubusercontent.com/ntjess/typst-tada/v0.1.0/assets/example-06.png}}

\subsubsection{\texorpdfstring{\texttt{\ display\ }}{ display }}\label{display}

If your type is not available or you want to customize its display, pass
a \texttt{\ display\ } function that formats the value, or a string that
accesses \texttt{\ value\ } in its scope:

\begin{Shaded}
\begin{Highlighting}[]
\NormalTok{\#td.field{-}info.at("quantity").insert(}
\NormalTok{  "display",}
\NormalTok{  val =\textgreater{} ("/", "One", "Two", "Three").at(val),}
\NormalTok{)}

\NormalTok{\#let td = TableData(..td)}
\NormalTok{\#to{-}tablex(td)}
\end{Highlighting}
\end{Shaded}

\pandocbounded{\includegraphics[keepaspectratio]{https://raw.githubusercontent.com/ntjess/typst-tada/v0.1.0/assets/example-07.png}}

\subsubsection{\texorpdfstring{\texttt{\ align\ }
etc.}{ align  etc.}}\label{align-etc.}

You can pass \texttt{\ align\ } and \texttt{\ width\ } to a given
field’s metadata to determine how content aligns in the cell and how
much horizontal space it takes up. In the future, more
\texttt{\ tablex\ } setup arguments will be accepted.

\begin{Shaded}
\begin{Highlighting}[]
\NormalTok{\#let adjusted = update{-}fields(}
\NormalTok{  td, name: (align: center, width: 1.4in)}
\NormalTok{)}
\NormalTok{\#to{-}tablex(adjusted)}
\end{Highlighting}
\end{Shaded}

\pandocbounded{\includegraphics[keepaspectratio]{https://raw.githubusercontent.com/ntjess/typst-tada/v0.1.0/assets/example-08.png}}

\subsection{\texorpdfstring{Deeper \texttt{\ tablex\ }
customization}{Deeper  tablex  customization}}\label{deeper-tablex-customization}

TaDa uses \texttt{\ tablex\ } to display the table. So any argument that
\texttt{\ tablex\ } accepts can be passed to TableData as well:

\begin{Shaded}
\begin{Highlighting}[]
\NormalTok{\#let mapper = (index, row) =\textgreater{} \{}
\NormalTok{  let fill = if index == 0 \{rgb("\#8888")\} else \{none\}}
\NormalTok{  row.map(cell =\textgreater{} (..cell, fill: fill))}
\NormalTok{\}}
\NormalTok{\#let td = TableData(}
\NormalTok{  ..td,}
\NormalTok{  tablex{-}kwargs: (}
\NormalTok{    map{-}rows: mapper, auto{-}vlines: false}
\NormalTok{  ),}
\NormalTok{)}
\NormalTok{\#to{-}tablex(td)}
\end{Highlighting}
\end{Shaded}

\pandocbounded{\includegraphics[keepaspectratio]{https://raw.githubusercontent.com/ntjess/typst-tada/v0.1.0/assets/example-09.png}}

\subsection{Subselection}\label{subselection}

You can select a subset of fields or rows to display:

\begin{Shaded}
\begin{Highlighting}[]
\NormalTok{\#import tada: subset}
\NormalTok{\#to{-}tablex(}
\NormalTok{  subset(td, indexes: (0,2), fields: ("name", "price"))}
\NormalTok{)}
\end{Highlighting}
\end{Shaded}

\pandocbounded{\includegraphics[keepaspectratio]{https://raw.githubusercontent.com/ntjess/typst-tada/v0.1.0/assets/example-10.png}}

Note that \texttt{\ indexes\ } is based on the table’s
\texttt{\ \_\_index\ } column, \emph{not} it’s positional index within
the table:

\begin{Shaded}
\begin{Highlighting}[]
\NormalTok{\#let td2 = td}
\NormalTok{\#td2.data.insert("\_\_index", (1, 2, 2))}
\NormalTok{\#to{-}tablex(}
\NormalTok{  subset(td2, indexes: 2, fields: ("\_\_index", "name"))}
\NormalTok{)}
\end{Highlighting}
\end{Shaded}

\pandocbounded{\includegraphics[keepaspectratio]{https://raw.githubusercontent.com/ntjess/typst-tada/v0.1.0/assets/example-11.png}}

Rows can also be selected by whether they fulfill a field condition:

\begin{Shaded}
\begin{Highlighting}[]
\NormalTok{\#to{-}tablex(}
\NormalTok{  tada.filter(td, expression: "price \textless{} 1.5")}
\NormalTok{)}
\end{Highlighting}
\end{Shaded}

\pandocbounded{\includegraphics[keepaspectratio]{https://raw.githubusercontent.com/ntjess/typst-tada/v0.1.0/assets/example-12.png}}

\subsection{Concatenation}\label{concatenation}

Concatenating rows and columns are both supported operations, but only
in the simple sense of stacking the data. Currently, there is no ability
to join on a field or otherwise intelligently merge data.

\begin{itemize}
\item
  \texttt{\ axis:\ 0\ } places new rows below current rows
\item
  \texttt{\ axis:\ 1\ } places new columns to the right of current
  columns
\item
  Unless you specify a fill value for missing values, the function will
  panic if the tables do not match exactly along their concatenation
  axis.
\item
  You cannot stack with \texttt{\ axis:\ 1\ } unless every column has a
  unique field name.
\end{itemize}

\begin{Shaded}
\begin{Highlighting}[]
\NormalTok{\#import tada: stack}

\NormalTok{\#let td2 = TableData(}
\NormalTok{  data: (}
\NormalTok{    name: ("Cheese", "Butter"),}
\NormalTok{    price: (2.50, 1.75),}
\NormalTok{  )}
\NormalTok{)}
\NormalTok{\#let td3 = TableData(}
\NormalTok{  data: (}
\NormalTok{    rating: (4.5, 3.5, 5.0, 4.0, 2.5),}
\NormalTok{  )}
\NormalTok{)}

\NormalTok{// This would fail without specifying the fill}
\NormalTok{// since \textasciigrave{}quantity\textasciigrave{} is missing from \textasciigrave{}td2\textasciigrave{}}
\NormalTok{\#let stack{-}a = stack(td, td2, missing{-}fill: 0)}
\NormalTok{\#let stack{-}b = stack(stack{-}a, td3, axis: 1)}
\NormalTok{\#to{-}tablex(stack{-}b)}
\end{Highlighting}
\end{Shaded}

\pandocbounded{\includegraphics[keepaspectratio]{https://raw.githubusercontent.com/ntjess/typst-tada/v0.1.0/assets/example-13.png}}

\subsection{Expressions}\label{expressions}

The easiest way to leverage TaDa’s flexibility is through expressions.
They can be strings that treat field names as variables, or functions
that take keyword-only arguments.

\begin{itemize}
\tightlist
\item
  \textbf{Note} ! When passing functions, every field is passed as a
  named argument to the function. So, make sure to capture unused fields
  with \texttt{\ ..rest\ } (the name is unimportant) to avoid errors.
\end{itemize}

\begin{Shaded}
\begin{Highlighting}[]
\NormalTok{\#let make{-}dict(field, expression) = \{}
\NormalTok{  let out = (:)}
\NormalTok{  out.insert(}
\NormalTok{    field,}
\NormalTok{    (expression: expression, type: "currency"),}
\NormalTok{  )}
\NormalTok{  out}
\NormalTok{\}}

\NormalTok{\#let td = update{-}fields(}
\NormalTok{  td, ..make{-}dict("total", "price * quantity" )}
\NormalTok{)}

\NormalTok{\#let tax{-}expr(total: none, ..rest) = \{ total * 0.2 \}}
\NormalTok{\#let taxed = update{-}fields(}
\NormalTok{  td, ..make{-}dict("tax", tax{-}expr),}
\NormalTok{)}

\NormalTok{\#to{-}tablex(}
\NormalTok{  subset(taxed, fields: ("name", "total", "tax"))}
\NormalTok{)}
\end{Highlighting}
\end{Shaded}

\pandocbounded{\includegraphics[keepaspectratio]{https://raw.githubusercontent.com/ntjess/typst-tada/v0.1.0/assets/example-14.png}}

\subsection{Chaining}\label{chaining}

It is inconvenient to require several temporary variables as above, or
deep function nesting, to perform multiple operations on a table. TaDa
provides a \texttt{\ chain\ } function to make this easier. Furthermore,
when you need to compute several fields at once and don’t need extra
field information, you can use \texttt{\ add-expressions\ } as a
shorthand:

\begin{Shaded}
\begin{Highlighting}[]
\NormalTok{\#import tada: chain, add{-}expressions}
\NormalTok{\#let totals = chain(td,}
\NormalTok{  add{-}expressions.with(}
\NormalTok{    total: "price * quantity",}
\NormalTok{    tax: "total * 0.2",}
\NormalTok{    after{-}tax: "total + tax",}
\NormalTok{  ),}
\NormalTok{  subset.with(}
\NormalTok{    fields: ("name", "total", "after{-}tax")}
\NormalTok{  ),}
\NormalTok{  // Add type information}
\NormalTok{  update{-}fields.with(}
\NormalTok{    after{-}tax: (type: "currency", title: fmt("w/ Tax")),}
\NormalTok{  ),}
\NormalTok{)}
\NormalTok{\#to{-}tablex(totals)}
\end{Highlighting}
\end{Shaded}

\pandocbounded{\includegraphics[keepaspectratio]{https://raw.githubusercontent.com/ntjess/typst-tada/v0.1.0/assets/example-15.png}}

\subsection{Sorting}\label{sorting}

You can sort by ascending/descending values of any field, or provide
your own transformation function to the \texttt{\ key\ } argument to
customize behavior further:

\begin{Shaded}
\begin{Highlighting}[]
\NormalTok{\#import tada: sort{-}values}
\NormalTok{\#to{-}tablex(sort{-}values(}
\NormalTok{  td, by: "quantity", descending: true}
\NormalTok{))}
\end{Highlighting}
\end{Shaded}

\pandocbounded{\includegraphics[keepaspectratio]{https://raw.githubusercontent.com/ntjess/typst-tada/v0.1.0/assets/example-16.png}}

\subsection{Aggregation}\label{aggregation}

Column-wise reduction is supported through \texttt{\ agg\ } , using
either functions or string expressions:

\begin{Shaded}
\begin{Highlighting}[]
\NormalTok{\#import tada: agg, item}
\NormalTok{\#let grand{-}total = chain(}
\NormalTok{  totals,}
\NormalTok{  agg.with(after{-}tax: array.sum),}
\NormalTok{  // use "item" to extract exactly one element}
\NormalTok{  item}
\NormalTok{)}
\NormalTok{// "Output" is a helper function just for these docs.}
\NormalTok{// It is not necessary in your code.}
\NormalTok{\#output[}
\NormalTok{  *Grand total: \#tada.display.format{-}usd(grand{-}total)*}
\NormalTok{]}
\end{Highlighting}
\end{Shaded}

\pandocbounded{\includegraphics[keepaspectratio]{https://raw.githubusercontent.com/ntjess/typst-tada/v0.1.0/assets/example-17.png}}

It is also easy to aggregate several expressions at once:

\begin{Shaded}
\begin{Highlighting}[]
\NormalTok{\#let agg{-}exprs = (}
\NormalTok{  "\# items": "quantity.sum()",}
\NormalTok{  "Longest name": "[\#name.sorted(key: str.len).at({-}1)]",}
\NormalTok{)}
\NormalTok{\#let agg{-}td = tada.agg(td, ..agg{-}exprs)}
\NormalTok{\#to{-}tablex(agg{-}td)}
\end{Highlighting}
\end{Shaded}

\pandocbounded{\includegraphics[keepaspectratio]{https://raw.githubusercontent.com/ntjess/typst-tada/v0.1.0/assets/example-18.png}}

\subsubsection{How to add}\label{how-to-add}

Copy this into your project and use the import as \texttt{\ tada\ }

\begin{verbatim}
#import "@preview/tada:0.1.0"
\end{verbatim}

\includesvg[width=0.16667in,height=0.16667in]{/assets/icons/16-copy.svg}

Check the docs for
\href{https://typst.app/docs/reference/scripting/\#packages}{more
information on how to import packages} .

\subsubsection{About}\label{about}

\begin{description}
\tightlist
\item[Author :]
Nathan Jessurun
\item[License:]
Unlicense
\item[Current version:]
0.1.0
\item[Last updated:]
December 15, 2023
\item[First released:]
December 15, 2023
\item[Archive size:]
16.2 kB
\href{https://packages.typst.org/preview/tada-0.1.0.tar.gz}{\pandocbounded{\includesvg[keepaspectratio]{/assets/icons/16-download.svg}}}
\item[Repository:]
\href{https://github.com/ntjess/typst-tada}{GitHub}
\end{description}

\subsubsection{Where to report issues?}\label{where-to-report-issues}

This package is a project of Nathan Jessurun . Report issues on
\href{https://github.com/ntjess/typst-tada}{their repository} . You can
also try to ask for help with this package on the
\href{https://forum.typst.app}{Forum} .

Please report this package to the Typst team using the
\href{https://typst.app/contact}{contact form} if you believe it is a
safety hazard or infringes upon your rights.

\phantomsection\label{versions}
\subsubsection{Version history}\label{version-history}

\begin{longtable}[]{@{}ll@{}}
\toprule\noalign{}
Version & Release Date \\
\midrule\noalign{}
\endhead
\bottomrule\noalign{}
\endlastfoot
0.1.0 & December 15, 2023 \\
\end{longtable}

Typst GmbH did not create this package and cannot guarantee correct
functionality of this package or compatibility with any version of the
Typst compiler or app.
