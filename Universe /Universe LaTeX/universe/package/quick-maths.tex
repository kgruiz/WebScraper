\title{typst.app/universe/package/quick-maths}

\phantomsection\label{banner}
\section{quick-maths}\label{quick-maths}

{ 0.2.0 }

Custom shorthands for math equations.

{ } Featured Package

\phantomsection\label{readme}
A package for creating custom shorthands for math equations.

\subsection{Usage}\label{usage}

The package comes with a single template function
\texttt{\ shorthands\ } that takes one or more tuples of the form
\texttt{\ (shorthand,\ replacement)\ } , where \texttt{\ shorthand\ }
can be a string or content.

There are some small quality of life features for interaction of
shorthands with fractions and attachments:

\begin{itemize}
\tightlist
\item
  If the right-most symbol of a shorthand has any attachments, they are
  moved to the shorthand’s replacement.
\item
  If a shorthand ends in the numerator of a fraction, the whole
  replacement is placed in the numerator.
\item
  If a shorthand starts in the denominator of a fraction, the whole
  replacement is placed in the denominator.
\end{itemize}

As the implementation of these features is quite hacky, you may
encounter some edge cases, where the use of explicit parentheses
hopefully saves you.

\subsection{Notes}\label{notes}

\begin{itemize}
\item
  Shorthands are parsed in the order they are given, so if you have a
  shorthand that is a prefix of another shorthand, you should put the
  longer shorthand first.
\item
  The content of an equation is traversed from left to right, so the
  left-most matching shorthand will be replaced first.
\item
  Shorthands consisting of only a single character or element may be
  applied using show rules, so that they can affect non-sequence
  elements. This may lead to different behavior than multi-character
  shorthands.
\item
  If the replacement of a shorthand contains a shorthand itself, there
  are no protections against infinite recursion or overflows.
\end{itemize}

\subsection{Example}\label{example}

\begin{Shaded}
\begin{Highlighting}[]
\NormalTok{\#import "@preview/quick{-}maths:0.2.0": shorthands}

\NormalTok{\#show: shorthands.with(}
\NormalTok{  ($+{-}$, $plus.minus$),}
\NormalTok{  ($|{-}$, math.tack),}
\NormalTok{  ($\textless{}=$, math.arrow.l.double) // Replaces \textquotesingle{}≤\textquotesingle{}}
\NormalTok{)}

\NormalTok{$ x\^{}2 = 9 quad \textless{}==\textgreater{} quad x = +{-}3 $}
\NormalTok{$ A or B |{-} A $}
\NormalTok{$ x \textless{}= y $}
\end{Highlighting}
\end{Shaded}

\pandocbounded{\includesvg[keepaspectratio]{https://github.com/typst/packages/raw/main/packages/preview/quick-maths/0.2.0/assets/example.svg}}

\subsubsection{How to add}\label{how-to-add}

Copy this into your project and use the import as
\texttt{\ quick-maths\ }

\begin{verbatim}
#import "@preview/quick-maths:0.2.0"
\end{verbatim}

\includesvg[width=0.16667in,height=0.16667in]{/assets/icons/16-copy.svg}

Check the docs for
\href{https://typst.app/docs/reference/scripting/\#packages}{more
information on how to import packages} .

\subsubsection{About}\label{about}

\begin{description}
\tightlist
\item[Author :]
Eric Biedert
\item[License:]
MIT
\item[Current version:]
0.2.0
\item[Last updated:]
November 18, 2024
\item[First released:]
July 5, 2024
\item[Archive size:]
3.58 kB
\href{https://packages.typst.org/preview/quick-maths-0.2.0.tar.gz}{\pandocbounded{\includesvg[keepaspectratio]{/assets/icons/16-download.svg}}}
\item[Repository:]
\href{https://github.com/EpicEricEE/typst-quick-maths}{GitHub}
\item[Categor y :]
\begin{itemize}
\tightlist
\item[]
\item
  \pandocbounded{\includesvg[keepaspectratio]{/assets/icons/16-hammer.svg}}
  \href{https://typst.app/universe/search/?category=utility}{Utility}
\end{itemize}
\end{description}

\subsubsection{Where to report issues?}\label{where-to-report-issues}

This package is a project of Eric Biedert . Report issues on
\href{https://github.com/EpicEricEE/typst-quick-maths}{their repository}
. You can also try to ask for help with this package on the
\href{https://forum.typst.app}{Forum} .

Please report this package to the Typst team using the
\href{https://typst.app/contact}{contact form} if you believe it is a
safety hazard or infringes upon your rights.

\phantomsection\label{versions}
\subsubsection{Version history}\label{version-history}

\begin{longtable}[]{@{}ll@{}}
\toprule\noalign{}
Version & Release Date \\
\midrule\noalign{}
\endhead
\bottomrule\noalign{}
\endlastfoot
0.2.0 & November 18, 2024 \\
\href{https://typst.app/universe/package/quick-maths/0.1.0/}{0.1.0} &
July 5, 2024 \\
\end{longtable}

Typst GmbH did not create this package and cannot guarantee correct
functionality of this package or compatibility with any version of the
Typst compiler or app.
