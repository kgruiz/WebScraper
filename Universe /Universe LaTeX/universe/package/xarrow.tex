\title{typst.app/universe/package/xarrow}

\phantomsection\label{banner}
\section{xarrow}\label{xarrow}

{ 0.3.1 }

Variable-length arrows in Typst.

\phantomsection\label{readme}
Variable-length arrows in Typst, fitting the width of a given content.

\subsection{Usage}\label{usage}

This library mainly provides the \texttt{\ xarrow\ } function. This
function takes one positional argument, which is the content to display
on top of the arrow. Additionally, the library provides the following
arrow styles:

\begin{itemize}
\tightlist
\item
  \texttt{\ xarrowDashed\ } using arrow \texttt{\ sym.arrow.dashed\ } .
\item
  \texttt{\ xarrowDouble\ } using arrow
  \texttt{\ sym.arrow.double.long\ } ;
\item
  \texttt{\ xarrowHook\ } using arrow \texttt{\ sym.arrow.hook\ } ;
\item
  \texttt{\ xarrowSquiggly\ } using arrow
  \texttt{\ sym.arrow.long.squiggly\ } ;
\item
  \texttt{\ xarrowTwoHead\ } using arrow \texttt{\ sym.arrow.twohead\ }
  ;
\item
  …
\end{itemize}

These names use camlCase in order to be simply called from math mode.
This may change in the future, if it becomes possible to have the
function names mirror the dot-separated name of the symbols themselves.

\subsubsection{Arguments}\label{arguments}

Users can provide the following arguments to any of the
previously-mentioned functions:

\begin{itemize}
\tightlist
\item
  \texttt{\ width\ } defines the width of the arrow. It defaults to
  \texttt{\ auto\ } , which makes the arrow adapt to the size of the
  body.
\item
  \texttt{\ margins\ } defines the spacing on each side of the
  \texttt{\ body\ } argument. Ignored when \texttt{\ width\ } is not
  \texttt{\ auto\ } .
\item
  \texttt{\ position\ } defines whether the main \texttt{\ body\ }
  argument will be set above or below the arrow. Default is
  \texttt{\ top\ } ; the only other accepted value is
  \texttt{\ bottom\ } .
\item
  \texttt{\ opposite\ } sets the content that is displayed on the other,
  non-default side of the arrow. Default is \texttt{\ none\ } .
\end{itemize}

\subsubsection{Example}\label{example}

\begin{verbatim}
#import "@preview/xarrow:0.3.0": xarrow, xarrowSquiggly, xarrowTwoHead

$
  a xarrow(sym: <--, QQ\, 1 + 1^4) b \
  c xarrowSquiggly("very long boi") d \
  c / ( a xarrowTwoHead("NP" limits(sum)^*) b times 4)
$
\end{verbatim}

\subsection{Customisation}\label{customisation}

The \texttt{\ xarrow\ } function has several named arguments which serve
to create new arrow designs:

\begin{itemize}
\tightlist
\item
  \texttt{\ sym\ } is the base symbol.
\item
  \texttt{\ sections\ } defines the way the symbol is divided. Drawing
  an arrow consists of drawing its tail, then repeating a central part
  that is defined by \texttt{\ sections\ } , then drawing the head. This
  is the parameter that has to be tweaked if observing artefacts.
  \texttt{\ sections\ } are given as two ratios, delimiting respectively
  the beginning and the end of the central, repeated part of the symbol.
\item
  \texttt{\ partial\_repeats\ } indicates whether the central part of
  the symbol can be partially repeated at the end in order to match the
  exact desired width. This has to be disabled when the repeated part
  has a clear period (like the squiggly arrow).
\end{itemize}

\subsubsection{Example}\label{example-1}

\begin{verbatim}
#let xarrowSquigglyBottom = xarrow.with(
  sym: sym.arrow.long.squiggly,
  sections: (20%, 45%),
  partial_repeats: false,
  position: bottom,
)
\end{verbatim}

\subsection{Limitations}\label{limitations}

\begin{itemize}
\tightlist
\item
  The predefined arrows are tweaked with the Computer Modern Math font
  in mind. With different glyphs, more sophisticated arrows will require
  manual modifications (of the \texttt{\ sections\ } argument) to be
  rendered correctly.
\item
  The \texttt{\ width\ } argument cannot be given ratio/fractions like
  other shapes. This would be a nice feature to have, in order to be
  able to create an arrow that takes 50\% of the available line width
  for instance.
\item
  I would like to make a proper manual for this library in the future,
  using something cool like
  \href{https://github.com/jneug/typst-mantys}{mantys} .
\end{itemize}

\subsubsection{How to add}\label{how-to-add}

Copy this into your project and use the import as \texttt{\ xarrow\ }

\begin{verbatim}
#import "@preview/xarrow:0.3.1"
\end{verbatim}

\includesvg[width=0.16667in,height=0.16667in]{/assets/icons/16-copy.svg}

Check the docs for
\href{https://typst.app/docs/reference/scripting/\#packages}{more
information on how to import packages} .

\subsubsection{About}\label{about}

\begin{description}
\tightlist
\item[Author :]
loutr
\item[License:]
GPL-3.0-only
\item[Current version:]
0.3.1
\item[Last updated:]
March 20, 2024
\item[First released:]
July 10, 2023
\item[Minimum Typst version:]
0.11.0
\item[Archive size:]
3.50 kB
\href{https://packages.typst.org/preview/xarrow-0.3.1.tar.gz}{\pandocbounded{\includesvg[keepaspectratio]{/assets/icons/16-download.svg}}}
\item[Repository:]
\href{https://codeberg.org/loutr/typst-xarrow/}{Codeberg}
\end{description}

\subsubsection{Where to report issues?}\label{where-to-report-issues}

This package is a project of loutr . Report issues on
\href{https://codeberg.org/loutr/typst-xarrow/}{their repository} . You
can also try to ask for help with this package on the
\href{https://forum.typst.app}{Forum} .

Please report this package to the Typst team using the
\href{https://typst.app/contact}{contact form} if you believe it is a
safety hazard or infringes upon your rights.

\phantomsection\label{versions}
\subsubsection{Version history}\label{version-history}

\begin{longtable}[]{@{}ll@{}}
\toprule\noalign{}
Version & Release Date \\
\midrule\noalign{}
\endhead
\bottomrule\noalign{}
\endlastfoot
0.3.1 & March 20, 2024 \\
\href{https://typst.app/universe/package/xarrow/0.3.0/}{0.3.0} & January
10, 2024 \\
\href{https://typst.app/universe/package/xarrow/0.2.0/}{0.2.0} &
September 26, 2023 \\
\href{https://typst.app/universe/package/xarrow/0.1.1/}{0.1.1} & July
11, 2023 \\
\href{https://typst.app/universe/package/xarrow/0.1.0/}{0.1.0} & July
10, 2023 \\
\end{longtable}

Typst GmbH did not create this package and cannot guarantee correct
functionality of this package or compatibility with any version of the
Typst compiler or app.
