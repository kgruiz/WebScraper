\title{typst.app/universe/package/polylux}

\phantomsection\label{banner}
\section{polylux}\label{polylux}

{ 0.3.1 }

Presentation slides creation with Typst

{ } Featured Package

\phantomsection\label{readme}
This is a package for creating presentation slides in
\href{https://typst.app/}{Typst} . Read the
\href{https://andreaskroepelin.github.io/polylux/book}{book} to learn
all about it and click
\href{https://andreaskroepelin.github.io/polylux/book/changelog.html}{here}
to see what’s new!

If you like it, consider
\href{https://github.com/andreasKroepelin/polylux}{giving a star on
GitHub} !

\href{https://andreaskroepelin.github.io/polylux/book}{\pandocbounded{\includegraphics[keepaspectratio]{https://img.shields.io/badge/docs-book-green}}}
\pandocbounded{\includegraphics[keepaspectratio]{https://img.shields.io/github/license/andreasKroepelin/polylux}}
\pandocbounded{\includegraphics[keepaspectratio]{https://img.shields.io/github/v/release/andreasKroepelin/polylux}}
\pandocbounded{\includegraphics[keepaspectratio]{https://img.shields.io/github/stars/andreasKroepelin/polylux}}
\href{https://github.com/andreasKroepelin/polylux/releases/latest/download/demo.pdf}{\pandocbounded{\includegraphics[keepaspectratio]{https://img.shields.io/badge/demo-pdf-blue}}}
\pandocbounded{\includegraphics[keepaspectratio]{https://img.shields.io/badge/themes-5-aqua}}

\subsection{Quickstart}\label{quickstart}

For the bare-bones, do-it-yourself experience, all you need is:

\begin{Shaded}
\begin{Highlighting}[]
\NormalTok{// Get Polylux from the official package repository}
\NormalTok{\#import "@preview/polylux:0.3.1": *}

\NormalTok{// Make the paper dimensions fit for a presentation and the text larger}
\NormalTok{\#set page(paper: "presentation{-}16{-}9")}
\NormalTok{\#set text(size: 25pt)}

\NormalTok{// Use \#polylux{-}slide to create a slide and style it using your favourite Typst functions}
\NormalTok{\#polylux{-}slide[}
\NormalTok{  \#align(horizon + center)[}
\NormalTok{    = Very minimalist slides}

\NormalTok{    A lazy author}

\NormalTok{    July 23, 2023}
\NormalTok{  ]}
\NormalTok{]}

\NormalTok{\#polylux{-}slide[}
\NormalTok{  == First slide}

\NormalTok{  Some static text on this slide.}
\NormalTok{]}

\NormalTok{\#polylux{-}slide[}
\NormalTok{  == This slide changes!}

\NormalTok{  You can always see this.}
\NormalTok{  // Make use of features like \#uncover, \#only, and others to create dynamic content}
\NormalTok{  \#uncover(2)[But this appears later!]}
\NormalTok{]}
\end{Highlighting}
\end{Shaded}

This code produces these PDF pages:
\pandocbounded{\includegraphics[keepaspectratio]{https://andreaskroepelin.github.io/polylux/book/minimal.png}}

From there, you can either start creatively adapting the looks to your
likings or you can use one of the provided themes. The simplest one of
them is called \texttt{\ simple\ } (what a coincidence!). It is still
very unintrusive but gives you some sensible defaults:

\begin{Shaded}
\begin{Highlighting}[]
\NormalTok{\#import "@preview/polylux:0.3.1": *}

\NormalTok{\#import themes.simple: *}

\NormalTok{\#set text(font: "Inria Sans")}

\NormalTok{\#show: simple{-}theme.with(}
\NormalTok{  footer: [Simple slides],}
\NormalTok{)}

\NormalTok{\#title{-}slide[}
\NormalTok{  = Keep it simple!}
\NormalTok{  \#v(2em)}

\NormalTok{  Alpha \#footnote[Uni Augsburg] \#h(1em)}
\NormalTok{  Bravo \#footnote[Uni Bayreuth] \#h(1em)}
\NormalTok{  Charlie \#footnote[Uni Chemnitz] \#h(1em)}

\NormalTok{  July 23}
\NormalTok{]}

\NormalTok{\#slide[}
\NormalTok{  == First slide}

\NormalTok{  \#lorem(20)}
\NormalTok{]}

\NormalTok{\#focus{-}slide[}
\NormalTok{  \_Focus!\_}

\NormalTok{  This is very important.}
\NormalTok{]}

\NormalTok{\#centered{-}slide[}
\NormalTok{  = Let\textquotesingle{}s start a new section!}
\NormalTok{]}

\NormalTok{\#slide[}
\NormalTok{  == Dynamic slide}
\NormalTok{  Did you know that...}

\NormalTok{  \#pause}
\NormalTok{  ...you can see the current section at the top of the slide?}
\NormalTok{]}
\end{Highlighting}
\end{Shaded}

This time, we obtain these PDF pages:
\pandocbounded{\includegraphics[keepaspectratio]{https://andreaskroepelin.github.io/polylux/book/themes/gallery/simple.png}}

As you can see, a theme can introduce its own types of slides (here:
\texttt{\ title-slide\ } , \texttt{\ slide\ } , \texttt{\ focus-slide\ }
, \texttt{\ centered-slide\ } ) to let you quickly switch between
different layouts. The book
\href{https://andreaskroepelin.github.io/polylux/book/themes/themes.html}{has
more infos} on how to use (and create your own) themes.

For dynamic content, Polylux also provides
\href{https://andreaskroepelin.github.io/polylux/book/dynamic/dynamic.html}{a
convenient API for complex overlays} .

If you use \href{https://pdfpc.github.io/}{pdfpc} to display your
slides, you can rely on
\href{https://andreaskroepelin.github.io/polylux/book/external/pdfpc.html}{Polylux’
support for it} and create speaker notes, hide slides, configure the
timer and more!

Visit the \href{https://andreaskroepelin.github.io/polylux/book}{book}
for more details or take a look at the
\href{https://github.com/andreasKroepelin/polylux/releases/latest/download/demo.pdf}{demo
PDF} where you can see the features of this template in action.

\textbf{âš~ This package is under active development and there are no
backwards compatibility guarantees!}

\subsection{Acknowledgements}\label{acknowledgements}

Thank you to…

\begin{itemize}
\tightlist
\item
  \href{https://github.com/drupol}{@drupol} for the
  \texttt{\ university\ } theme
\item
  \href{https://github.com/Enivex}{@Enivex} for the
  \texttt{\ metropolis\ } theme
\item
  \href{https://github.com/MarkBlyth}{@MarkBlyth} for contributing to
  the \texttt{\ clean\ } theme
\item
  \href{https://github.com/ntjess}{@ntjess} for contributing to the
  height fitting feature
\item
  \href{https://github.com/JuliusFreudenberger}{@JuliusFreudenberger}
  for maintaining the \texttt{\ polylux2pdfpc\ } AUR package
\item
  \href{https://github.com/fncnt}{@fncnt} for coming up with the name
  “Polylux�
\item
  the Typst authors and contributors for this refreshing piece of
  software
\end{itemize}

\subsubsection{How to add}\label{how-to-add}

Copy this into your project and use the import as \texttt{\ polylux\ }

\begin{verbatim}
#import "@preview/polylux:0.3.1"
\end{verbatim}

\includesvg[width=0.16667in,height=0.16667in]{/assets/icons/16-copy.svg}

Check the docs for
\href{https://typst.app/docs/reference/scripting/\#packages}{more
information on how to import packages} .

\subsubsection{About}\label{about}

\begin{description}
\tightlist
\item[Author s :]
Andreas Kröpelin \& contributors
\item[License:]
MIT
\item[Current version:]
0.3.1
\item[Last updated:]
September 3, 2023
\item[First released:]
July 26, 2023
\item[Archive size:]
9.62 kB
\href{https://packages.typst.org/preview/polylux-0.3.1.tar.gz}{\pandocbounded{\includesvg[keepaspectratio]{/assets/icons/16-download.svg}}}
\item[Repository:]
\href{https://github.com/andreasKroepelin/polylux}{GitHub}
\end{description}

\subsubsection{Where to report issues?}\label{where-to-report-issues}

This package is a project of Andreas Kröpelin and contributors . Report
issues on \href{https://github.com/andreasKroepelin/polylux}{their
repository} . You can also try to ask for help with this package on the
\href{https://forum.typst.app}{Forum} .

Please report this package to the Typst team using the
\href{https://typst.app/contact}{contact form} if you believe it is a
safety hazard or infringes upon your rights.

\phantomsection\label{versions}
\subsubsection{Version history}\label{version-history}

\begin{longtable}[]{@{}ll@{}}
\toprule\noalign{}
Version & Release Date \\
\midrule\noalign{}
\endhead
\bottomrule\noalign{}
\endlastfoot
0.3.1 & September 3, 2023 \\
\href{https://typst.app/universe/package/polylux/0.2.0/}{0.2.0} & July
26, 2023 \\
\end{longtable}

Typst GmbH did not create this package and cannot guarantee correct
functionality of this package or compatibility with any version of the
Typst compiler or app.
