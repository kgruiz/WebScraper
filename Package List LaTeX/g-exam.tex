\title{typst.app/universe/package/g-exam}

\phantomsection\label{banner}
\phantomsection\label{template-thumbnail}
\pandocbounded{\includegraphics[keepaspectratio]{https://packages.typst.org/preview/thumbnails/g-exam-0.4.1-small.webp}}

\section{g-exam}\label{g-exam}

{ 0.4.1 }

Create exams with student information, grade chart, score control,
questions, and sub-questions.

\href{/app?template=g-exam&version=0.4.1}{Create project in app}

\phantomsection\label{readme}
This template provides a way to generate exams. You can create questions
and sub-questions, header with information about the academic center,
score box, subject, exam, header with student information,
clarifications, solutions, watermark with information about the exam
model and teacher.

\paragraph{Features}\label{features}

\begin{itemize}
\tightlist
\item
  Scoreboard.
\item
  Scoring by questions and subquestions.
\item
  Student information, on the first page or on all odd pages.
\item
  Question and subcuestion.
\item
  Show solutions and clarifications
\item
  List of clarifications.
\item
  Teacher’s Watermark
\item
  Exam Model Watermark
\item
  Draft mode
\end{itemize}

\subsection{Usage}\label{usage}

For information, see the
\href{https://matheschool.github.io/typst-g-exam/}{online
docucumentation} .

To use this package, simply add the following code to your document:

\paragraph{A sample exam}\label{a-sample-exam}

\pandocbounded{\includegraphics[keepaspectratio]{https://github.com/typst/packages/raw/main/packages/preview/g-exam/0.4.1/gallery/exam-table-content.png}}

\paragraph{Source:}\label{source}

\begin{Shaded}
\begin{Highlighting}[]
\NormalTok{\#import "@preview/g{-}exam:0.4.1": *}

\NormalTok{\#show: exam.with(}
\NormalTok{  school: (}
\NormalTok{    name: "Sunrise Secondary School",}
\NormalTok{    logo: read("./logo.png", encoding: none),}
\NormalTok{  ),}
\NormalTok{  exam{-}info: (}
\NormalTok{    academic{-}period: "Academic year 2023/2024",}
\NormalTok{    academic{-}level: "1st Secondary Education",}
\NormalTok{    academic{-}subject: "Mathematics",}
\NormalTok{    number: "2nd Assessment 1st Exam",}
\NormalTok{    content: "Radicals and fractions",}
\NormalTok{    model: "Model A"}
\NormalTok{  ),}
  
\NormalTok{  show{-}student{-}data: "first{-}page",}
\NormalTok{  show{-}grade{-}table: true,}
\NormalTok{  clarifications: "Answer the questions in the spaces provided. If you run out of room for an answer, continue on the back of the page."}
\NormalTok{)}
\NormalTok{\#question(points:2.5)[Is it true that $x\^{}n + y\^{}n = z\^{}n$ if $(x,y,z)$ and $n$ are positive integers?. Explain.] }
\NormalTok{\#v(1fr)}

\NormalTok{\#question(points:2.5)[Prove that the real part of all non{-}trivial zeros of the function $zeta(z) "is" 1/2$].}
\NormalTok{\#v(1fr)}

\NormalTok{\#question(points:2)[Compute $ integral\_0\^{}infinity (sin(x))/x $ ]}
\NormalTok{\#v(1fr)}
\end{Highlighting}
\end{Shaded}

\subsection{Changelog}\label{changelog}

\subsubsection{v0.4.1}\label{v0.4.1}

\begin{itemize}
\tightlist
\item
  Fix student data.
\item
  Fix Indenting subquestion.
\end{itemize}

\subsubsection{v0.4.0}\label{v0.4.0}

\begin{itemize}
\tightlist
\item
  Change g-exam for exam.
\item
  Change g-question and g-subquestion for question and subquestion.
\item
  Change point parameter to points in question and subquestion.
\item
  Change question-points-position paramet to question-points-position.
\item
  Include online documentation.
\item
  Use paper by default.
\item
  Indenting subquestion.
\item
  Include support for dutch language.
\item
  Corrections in English texts.
\item
  Draft label.
\end{itemize}

\subsubsection{v0.3.2}\label{v0.3.2}

\begin{itemize}
\tightlist
\item
  Change show-studen-data to show-student-data parameter.
\item
  Change languaje to language parameter.
\end{itemize}

\subsubsection{v0.3.1}\label{v0.3.1}

\begin{itemize}
\tightlist
\item
  Corrections in French.
\end{itemize}

\subsubsection{v0.3.0}\label{v0.3.0}

\begin{itemize}
\tightlist
\item
  Include parameter question-text-parameters.
\item
  Show solution.
\item
  Expand documentation.
\item
  Possibility of estrablecer question-point-position to none.
\item
  Bug fix show watermark.
\end{itemize}

\subsubsection{v0.2.0}\label{v0.2.0}

\begin{itemize}
\tightlist
\item
  Control the size of the logo image.
\item
  Convert to template
\item
  Allow true and false values in show-student-data.
\item
  Show clarifications.
\item
  Widen margin points.
\item
  Show solution.
\end{itemize}

\subsubsection{v0.1.1}\label{v0.1.1}

\begin{itemize}
\tightlist
\item
  Fix loading image.
\end{itemize}

\subsubsection{v0.1.0}\label{v0.1.0}

\begin{itemize}
\tightlist
\item
  Initial version submitted to typst/packages.
\end{itemize}

\href{/app?template=g-exam&version=0.4.1}{Create project in app}

\subsubsection{How to use}\label{how-to-use}

Click the button above to create a new project using this template in
the Typst app.

You can also use the Typst CLI to start a new project on your computer
using this command:

\begin{verbatim}
typst init @preview/g-exam:0.4.1
\end{verbatim}

\includesvg[width=0.16667in,height=0.16667in]{/assets/icons/16-copy.svg}

\subsubsection{About}\label{about}

\begin{description}
\tightlist
\item[Author :]
Andrés Giménez Muñoz
\item[License:]
MIT
\item[Current version:]
0.4.1
\item[Last updated:]
November 19, 2024
\item[First released:]
February 21, 2024
\item[Minimum Typst version:]
0.12.0
\item[Archive size:]
177 kB
\href{https://packages.typst.org/preview/g-exam-0.4.1.tar.gz}{\pandocbounded{\includesvg[keepaspectratio]{/assets/icons/16-download.svg}}}
\item[Repository:]
\href{https://github.com/MatheSchool/typst-g-exam}{GitHub}
\item[Discipline :]
\begin{itemize}
\tightlist
\item[]
\item
  \href{https://typst.app/universe/search/?discipline=education}{Education}
\end{itemize}
\item[Categor y :]
\begin{itemize}
\tightlist
\item[]
\item
  \pandocbounded{\includesvg[keepaspectratio]{/assets/icons/16-envelope.svg}}
  \href{https://typst.app/universe/search/?category=office}{Office}
\end{itemize}
\end{description}

\subsubsection{Where to report issues?}\label{where-to-report-issues}

This template is a project of Andrés Giménez Muñoz . Report issues on
\href{https://github.com/MatheSchool/typst-g-exam}{their repository} .
You can also try to ask for help with this template on the
\href{https://forum.typst.app}{Forum} .

Please report this template to the Typst team using the
\href{https://typst.app/contact}{contact form} if you believe it is a
safety hazard or infringes upon your rights.

\phantomsection\label{versions}
\subsubsection{Version history}\label{version-history}

\begin{longtable}[]{@{}ll@{}}
\toprule\noalign{}
Version & Release Date \\
\midrule\noalign{}
\endhead
\bottomrule\noalign{}
\endlastfoot
0.4.1 & November 19, 2024 \\
\href{https://typst.app/universe/package/g-exam/0.4.0/}{0.4.0} &
November 8, 2024 \\
\href{https://typst.app/universe/package/g-exam/0.3.2/}{0.3.2} & August
26, 2024 \\
\href{https://typst.app/universe/package/g-exam/0.3.1/}{0.3.1} & July
23, 2024 \\
\href{https://typst.app/universe/package/g-exam/0.3.0/}{0.3.0} & April
8, 2024 \\
\href{https://typst.app/universe/package/g-exam/0.2.0/}{0.2.0} & March
21, 2024 \\
\href{https://typst.app/universe/package/g-exam/0.1.1/}{0.1.1} &
February 22, 2024 \\
\href{https://typst.app/universe/package/g-exam/0.1.0/}{0.1.0} &
February 21, 2024 \\
\end{longtable}

Typst GmbH did not create this template and cannot guarantee correct
functionality of this template or compatibility with any version of the
Typst compiler or app.
