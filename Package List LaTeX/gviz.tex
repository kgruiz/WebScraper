\title{typst.app/universe/package/gviz}

\phantomsection\label{banner}
\section{gviz}\label{gviz}

{ 0.1.0 }

Generate graphs using the graphviz dot language.

\phantomsection\label{readme}
GViz is a typst plugin that can render graphviz graphs.

It uses \url{https://codeberg.org/Sekoia/layout} as a backend, which
means it can currently only render to SVG, and mostly supports basic
features.

Import it like any other plugin:
\texttt{\ \#import\ "@preview/gviz:0.1.0":\ *\ } .

\subsection{Usage}\label{usage}

\begin{Shaded}
\begin{Highlighting}[]
\NormalTok{\#import "@preview/gviz:0.1.0": *}

\NormalTok{\#show raw.where(lang: "dot{-}render"): it =\textgreater{} render{-}image(it.text)}

\NormalTok{\textasciigrave{}\textasciigrave{}\textasciigrave{}dot{-}render}
\NormalTok{digraph mygraph \{}
\NormalTok{  node [shape=box];}
\NormalTok{  A {-}\textgreater{} B;}
\NormalTok{  B {-}\textgreater{} C;}
\NormalTok{  B {-}\textgreater{} D;}
\NormalTok{  C {-}\textgreater{} E;}
\NormalTok{  D {-}\textgreater{} E;}
\NormalTok{  E {-}\textgreater{} F;}
\NormalTok{  A {-}\textgreater{} F [label="one"];}
\NormalTok{  A {-}\textgreater{} F [label="two"];}
\NormalTok{  A {-}\textgreater{} F [label="three"];}
\NormalTok{  A {-}\textgreater{} F [label="four"];}
\NormalTok{  A {-}\textgreater{} F [label="five"];}
\NormalTok{\}\textasciigrave{}\textasciigrave{}\textasciigrave{}}

\NormalTok{\#let my{-}graph = "digraph \{A {-}\textgreater{} B\}"}
\NormalTok{\#render{-}image(my{-}graph)}

\NormalTok{SVG:}
\NormalTok{\#raw(render(my{-}graph), block: true, lang: "svg")}
\end{Highlighting}
\end{Shaded}

\subsection{API}\label{api}

\subsubsection{render}\label{render}

Renders a graph in dot language and returns SVG code for it.

Parameters:

\begin{itemize}
\tightlist
\item
  code (string, bytes): Dot language code to be rendered.
\end{itemize}

Returns: string

\subsubsection{render-image}\label{render-image}

Renders a graph in dot language and returns an SVG image of it. Uses the
same parameters as image.decode.

Parameters:

\begin{itemize}
\tightlist
\item
  code (string, bytes): Dot language code to be rendered.
\item
  width (auto, relative): The width of the image.
\item
  height (auto, relative): The height of the image.
\item
  alt (none, string): A text describing the image.
\item
  fit (string): How the image should adjust itself to a given area. See
  image.decode.
\end{itemize}

Returns: content

\subsubsection{How to add}\label{how-to-add}

Copy this into your project and use the import as \texttt{\ gviz\ }

\begin{verbatim}
#import "@preview/gviz:0.1.0"
\end{verbatim}

\includesvg[width=0.16667in,height=0.16667in]{/assets/icons/16-copy.svg}

Check the docs for
\href{https://typst.app/docs/reference/scripting/\#packages}{more
information on how to import packages} .

\subsubsection{About}\label{about}

\begin{description}
\tightlist
\item[Author :]
\href{https://codeberg.org/Sekoia\%3E\%20\%3Chttps://github.com/SekoiaTree}{Sekoia}
\item[License:]
Unlicense
\item[Current version:]
0.1.0
\item[Last updated:]
September 15, 2023
\item[First released:]
September 15, 2023
\item[Minimum Typst version:]
0.8.0
\item[Archive size:]
85.7 kB
\href{https://packages.typst.org/preview/gviz-0.1.0.tar.gz}{\pandocbounded{\includesvg[keepaspectratio]{/assets/icons/16-download.svg}}}
\item[Repository:]
\href{https://codeberg.org/Sekoia/gviz-typst}{Codeberg}
\end{description}

\subsubsection{Where to report issues?}\label{where-to-report-issues}

This package is a project of Sekoia . Report issues on
\href{https://codeberg.org/Sekoia/gviz-typst}{their repository} . You
can also try to ask for help with this package on the
\href{https://forum.typst.app}{Forum} .

Please report this package to the Typst team using the
\href{https://typst.app/contact}{contact form} if you believe it is a
safety hazard or infringes upon your rights.

\phantomsection\label{versions}
\subsubsection{Version history}\label{version-history}

\begin{longtable}[]{@{}ll@{}}
\toprule\noalign{}
Version & Release Date \\
\midrule\noalign{}
\endhead
\bottomrule\noalign{}
\endlastfoot
0.1.0 & September 15, 2023 \\
\end{longtable}

Typst GmbH did not create this package and cannot guarantee correct
functionality of this package or compatibility with any version of the
Typst compiler or app.
