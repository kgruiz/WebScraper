\title{typst.app/universe/package/pinit}

\phantomsection\label{banner}
\section{pinit}\label{pinit}

{ 0.2.2 }

Relative positioning by pins, especially useful for making slides in
typst.

{ } Featured Package

\phantomsection\label{readme}
Relative positioning by pins, especially useful for making slides in
typst.

\subsection{Example}\label{example}

\subsubsection{Pin things as you like}\label{pin-things-as-you-like}

Have a look at the source
\href{https://github.com/typst/packages/raw/main/packages/preview/pinit/0.2.2/examples/example.typ}{here}
.

\pandocbounded{\includegraphics[keepaspectratio]{https://github.com/typst/packages/raw/main/packages/preview/pinit/0.2.2/examples/example.png}}

\subsubsection{Dynamic Slides}\label{dynamic-slides}

Pinit works with \href{https://github.com/touying-typ/touying}{Touying}
or \href{https://github.com/andreasKroepelin/polylux}{Polylux}
animations.

Have a look at the pdf file
\href{https://github.com/OrangeX4/typst-pinit/blob/main/examples/example.pdf}{here}
.

\pandocbounded{\includegraphics[keepaspectratio]{https://github.com/typst/packages/raw/main/packages/preview/pinit/0.2.2/examples/example-pages.png}}

\subsection{Usage}\label{usage}

\subsubsection{Examples}\label{examples}

The idea of pinit is pinning pins on the normal flow of the text, and
then placing the content on the page by \texttt{\ absolute-place\ }
function.

For example, we can highlight text and add a tip by pins simply:

\begin{Shaded}
\begin{Highlighting}[]
\NormalTok{\#import "@preview/pinit:0.2.2": *}

\NormalTok{\#set text(size: 24pt)}

\NormalTok{A simple \#pin(1)highlighted text\#pin(2).}

\NormalTok{\#pinit{-}highlight(1, 2)}

\NormalTok{\#pinit{-}point{-}from(2)[It is simple.]}
\end{Highlighting}
\end{Shaded}

\pandocbounded{\includegraphics[keepaspectratio]{https://github.com/typst/packages/raw/main/packages/preview/pinit/0.2.2/examples/simple-demo.png}}

If you want to place the content relative to the center of some pins,
you use a array of pins:

\begin{Shaded}
\begin{Highlighting}[]
\NormalTok{\#import "@preview/pinit:0.2.2": *}

\NormalTok{\#set text(size: 12pt)}

\NormalTok{A simple \#pin(1)highlighted text\#pin(2).}

\NormalTok{\#pinit{-}highlight(1, 2)}

\NormalTok{\#pinit{-}point{-}from((1, 2))[It is simple.]}
\end{Highlighting}
\end{Shaded}

\pandocbounded{\includegraphics[keepaspectratio]{https://github.com/typst/packages/raw/main/packages/preview/pinit/0.2.2/examples/simple-demo2.png}}

A more complex example, Have a look at the source
\href{https://github.com/typst/packages/raw/main/packages/preview/pinit/0.2.2/examples/equation-desc.typ}{here}
.

\pandocbounded{\includegraphics[keepaspectratio]{https://github.com/typst/packages/raw/main/packages/preview/pinit/0.2.2/examples/equation-desc.png}}

\subsubsection{Fletcher edge support}\label{fletcher-edge-support}

\href{https://github.com/Jollywatt/typst-fletcher}{Fletcher} is a
powerful Typst package for drawing diagrams with arrows. We can use
fletcher to draw more complex arrows.

\href{https://github.com/typst/packages/raw/main/packages/preview/pinit/0.2.2/\#pinit-fletcher-edge}{\texttt{\ pinit-fletcher-edge\ }}

\begin{Shaded}
\begin{Highlighting}[]
\NormalTok{\#import "@preview/pinit:0.2.2": *}
\NormalTok{\#import "@preview/fletcher:0.5.1"}

\NormalTok{Con\#pin(1)\#h(4em)\#pin(2)nect}

\NormalTok{\#pinit{-}fletcher{-}edge(}
\NormalTok{  fletcher, 1, end: 2, (1, 0), [bend], bend: {-}20deg, "\textless{}{-}\textgreater{}",}
\NormalTok{  decorations: fletcher.cetz.decorations.wave.with(amplitude: .1),}
\NormalTok{)}
\end{Highlighting}
\end{Shaded}

\pandocbounded{\includegraphics[keepaspectratio]{https://github.com/typst/packages/raw/main/packages/preview/pinit/0.2.2/examples/fletcher.png}}

\subsubsection{Pinit for raw}\label{pinit-for-raw}

In the code block, we need to use a regex trick to get pinit to work,
for example

\begin{Shaded}
\begin{Highlighting}[]
\NormalTok{\#show raw: it =\textgreater{} \{}
\NormalTok{  show regex("pin\textbackslash{}d"): it =\textgreater{} pin(eval(it.text.slice(3)))}
\NormalTok{  it}
\NormalTok{\}}

\NormalTok{\textasciigrave{}print(pin1"hello, world"pin2)\textasciigrave{}}

\NormalTok{\#pinit{-}highlight(1, 2)}
\end{Highlighting}
\end{Shaded}

\pandocbounded{\includegraphics[keepaspectratio]{https://github.com/typst/packages/raw/main/packages/preview/pinit/0.2.2/examples/pinit-for-raw.png}}

Note that typst’s code highlighting breaks up the text, causing overly
complex regular expressions such as ‘\#pin(.*?)’ to not work
properly.

However, you may want to consider putting it in a comment to avoid
highlighting the text and breaking it up.

\subsection{Notice}\label{notice}

\textbf{Since Typst does not provide a reliable
\texttt{\ absolute-place\ } function, you may consider taking the
following steps if a MISALIGNMENT occurs:}

\begin{enumerate}
\tightlist
\item
  \textbf{You could try to add a \texttt{\ \#box()\ } after the
  \texttt{\ \#pinit-xxx\ } function call, like
  \texttt{\ \#pinit-xxx()\#box()\ } .}
\item
  \textbf{You should add a blank line before the
  \texttt{\ \#pinit-xxx\ } function call, otherwise it will cause
  misalignment.}
\item
  \textbf{You can try moving \texttt{\ \#pinit-xxx()\ } in front of or
  behind \texttt{\ \#pin()\ } , or otherwhere, in short, try more.}
\item
  \textbf{Try to add a offset to the \texttt{\ dx\ } or \texttt{\ dy\ }
  argument of \texttt{\ \#pinit-xxx\ } function by yourself.}
\item
  \textbf{Open an issue if you have any questions you can’t solve.}
\end{enumerate}

\subsection{Outline}\label{outline}

\begin{itemize}
\tightlist
\item
  \href{https://github.com/typst/packages/raw/main/packages/preview/pinit/0.2.2/\#pinit}{Pinit}

  \begin{itemize}
  \tightlist
  \item
    \href{https://github.com/typst/packages/raw/main/packages/preview/pinit/0.2.2/\#example}{Example}

    \begin{itemize}
    \tightlist
    \item
      \href{https://github.com/typst/packages/raw/main/packages/preview/pinit/0.2.2/\#pin-things-as-you-like}{Pin
      things as you like}
    \item
      \href{https://github.com/typst/packages/raw/main/packages/preview/pinit/0.2.2/\#dynamic-slides}{Dynamic
      Slides}
    \end{itemize}
  \item
    \href{https://github.com/typst/packages/raw/main/packages/preview/pinit/0.2.2/\#usage}{Usage}

    \begin{itemize}
    \tightlist
    \item
      \href{https://github.com/typst/packages/raw/main/packages/preview/pinit/0.2.2/\#examples}{Examples}
    \item
      \href{https://github.com/typst/packages/raw/main/packages/preview/pinit/0.2.2/\#fletcher-edge-support}{Fletcher
      edge support}
    \item
      \href{https://github.com/typst/packages/raw/main/packages/preview/pinit/0.2.2/\#pinit-for-raw}{Pinit
      for raw}
    \end{itemize}
  \item
    \href{https://github.com/typst/packages/raw/main/packages/preview/pinit/0.2.2/\#notice}{Notice}
  \item
    \href{https://github.com/typst/packages/raw/main/packages/preview/pinit/0.2.2/\#outline}{Outline}
  \item
    \href{https://github.com/typst/packages/raw/main/packages/preview/pinit/0.2.2/\#reference}{Reference}

    \begin{itemize}
    \tightlist
    \item
      \href{https://github.com/typst/packages/raw/main/packages/preview/pinit/0.2.2/\#pin}{\texttt{\ pin\ }}
    \item
      \href{https://github.com/typst/packages/raw/main/packages/preview/pinit/0.2.2/\#pinit-1}{\texttt{\ pinit\ }}
    \item
      \href{https://github.com/typst/packages/raw/main/packages/preview/pinit/0.2.2/\#absolute-place}{\texttt{\ absolute-place\ }}
    \item
      \href{https://github.com/typst/packages/raw/main/packages/preview/pinit/0.2.2/\#pinit-place}{\texttt{\ pinit-place\ }}
    \item
      \href{https://github.com/typst/packages/raw/main/packages/preview/pinit/0.2.2/\#pinit-rect}{\texttt{\ pinit-rect\ }}
    \item
      \href{https://github.com/typst/packages/raw/main/packages/preview/pinit/0.2.2/\#pinit-highlight}{\texttt{\ pinit-highlight\ }}
    \item
      \href{https://github.com/typst/packages/raw/main/packages/preview/pinit/0.2.2/\#pinit-line}{\texttt{\ pinit-line\ }}
    \item
      \href{https://github.com/typst/packages/raw/main/packages/preview/pinit/0.2.2/\#pinit-line-to}{\texttt{\ pinit-line-to\ }}
    \item
      \href{https://github.com/typst/packages/raw/main/packages/preview/pinit/0.2.2/\#pinit-arrow}{\texttt{\ pinit-arrow\ }}
    \item
      \href{https://github.com/typst/packages/raw/main/packages/preview/pinit/0.2.2/\#pinit-double-arrow}{\texttt{\ pinit-double-arrow\ }}
    \item
      \href{https://github.com/typst/packages/raw/main/packages/preview/pinit/0.2.2/\#pinit-point-to}{\texttt{\ pinit-point-to\ }}
    \item
      \href{https://github.com/typst/packages/raw/main/packages/preview/pinit/0.2.2/\#pinit-point-from}{\texttt{\ pinit-point-from\ }}
    \item
      \href{https://github.com/typst/packages/raw/main/packages/preview/pinit/0.2.2/\#simple-arrow}{\texttt{\ simple-arrow\ }}
    \item
      \href{https://github.com/typst/packages/raw/main/packages/preview/pinit/0.2.2/\#double-arrow}{\texttt{\ double-arrow\ }}
    \item
      \href{https://github.com/typst/packages/raw/main/packages/preview/pinit/0.2.2/\#pinit-fletcher-edge}{\texttt{\ pinit-fletcher-edge\ }}
    \end{itemize}
  \item
    \href{https://github.com/typst/packages/raw/main/packages/preview/pinit/0.2.2/\#changelog}{Changelog}

    \begin{itemize}
    \tightlist
    \item
      \href{https://github.com/typst/packages/raw/main/packages/preview/pinit/0.2.2/\#022}{0.2.2}
    \item
      \href{https://github.com/typst/packages/raw/main/packages/preview/pinit/0.2.2/\#021}{0.2.1}
    \item
      \href{https://github.com/typst/packages/raw/main/packages/preview/pinit/0.2.2/\#020}{0.2.0}
    \item
      \href{https://github.com/typst/packages/raw/main/packages/preview/pinit/0.2.2/\#014}{0.1.4}
    \item
      \href{https://github.com/typst/packages/raw/main/packages/preview/pinit/0.2.2/\#013}{0.1.3}
    \item
      \href{https://github.com/typst/packages/raw/main/packages/preview/pinit/0.2.2/\#012}{0.1.2}
    \item
      \href{https://github.com/typst/packages/raw/main/packages/preview/pinit/0.2.2/\#011}{0.1.1}
    \item
      \href{https://github.com/typst/packages/raw/main/packages/preview/pinit/0.2.2/\#010}{0.1.0}
    \end{itemize}
  \item
    \href{https://github.com/typst/packages/raw/main/packages/preview/pinit/0.2.2/\#acknowledgements}{Acknowledgements}
  \item
    \href{https://github.com/typst/packages/raw/main/packages/preview/pinit/0.2.2/\#license}{License}
  \end{itemize}
\end{itemize}

\subsection{Reference}\label{reference}

\subsubsection{\texorpdfstring{\texttt{\ pin\ }}{ pin }}\label{pin}

Pinning a pin in text, the pin is supposed to be unique in one page.

\begin{Shaded}
\begin{Highlighting}[]
\NormalTok{\#let pin(name) = \{ .. \}}
\end{Highlighting}
\end{Shaded}

\textbf{Arguments:}

\begin{itemize}
\tightlist
\item
  \texttt{\ name\ } : {[} \texttt{\ int\ } or \texttt{\ str\ } or
  \texttt{\ any\ } {]} â€'' Name of pin, which can be any types with
  unique \texttt{\ repr()\ } return value, such as integer and string.
\end{itemize}

\subsubsection{\texorpdfstring{\texttt{\ pinit\ }}{ pinit }}\label{pinit-1}

Query positions of pins in the same page, then call the callback
function \texttt{\ callback\ } .

\begin{Shaded}
\begin{Highlighting}[]
\NormalTok{\#let pinit(callback: none, ..pins) = \{ .. \}}
\end{Highlighting}
\end{Shaded}

\textbf{Arguments:}

\begin{itemize}
\tightlist
\item
  \texttt{\ ..pins\ } : {[} \texttt{\ pin\ } {]} â€'' Names of pins you
  want to query. It is supposed to be arguments of pin or a group of
  pins.
\item
  \texttt{\ callback\ } : {[}
  \texttt{\ (..positions)\ =\textgreater{}\ \{\ ..\ \}\ } {]} â€'' A
  callback function accepting an array of positions (or a single
  position) as a parameter. Each position is a dictionary like
  \texttt{\ (page:\ 1,\ x:\ 319.97pt,\ y:\ 86.66pt)\ } . You can use the
  \texttt{\ absolute-place\ } function in this callback function to
  display something around the pins.
\end{itemize}

\subsubsection{\texorpdfstring{\texttt{\ absolute-place\ }}{ absolute-place }}\label{absolute-place}

Place content at a specific location on the page relative to the top
left corner of the page, regardless of margins, current containers, etc.

\begin{quote}
This function comes from
\href{https://github.com/ntjess/typst-drafting}{typst-drafting} .
\end{quote}

\begin{Shaded}
\begin{Highlighting}[]
\NormalTok{\#let absolute{-}place(}
\NormalTok{  dx: 0em,}
\NormalTok{  dy: 0em,}
\NormalTok{  body,}
\NormalTok{) = \{ .. \}}
\end{Highlighting}
\end{Shaded}

\textbf{Arguments:}

\begin{itemize}
\tightlist
\item
  \texttt{\ dx\ } : {[} \texttt{\ length\ } {]} â€'' Length in the
  x-axis relative to the left edge of the page.
\item
  \texttt{\ dy\ } : {[} \texttt{\ length\ } {]} â€'' Length in the
  y-axis relative to the top edge of the page.
\item
  \texttt{\ content\ } : {[} \texttt{\ content\ } {]} â€'' The content
  you want to place.
\end{itemize}

\subsubsection{\texorpdfstring{\texttt{\ pinit-place\ }}{ pinit-place }}\label{pinit-place}

Place content at a specific location on the page relative to the pin.

\begin{Shaded}
\begin{Highlighting}[]
\NormalTok{\#let pinit{-}place(}
\NormalTok{  dx: 0pt,}
\NormalTok{  dy: 0pt,}
\NormalTok{  pin{-}name,}
\NormalTok{  body,}
\NormalTok{) = \{ .. \}}
\end{Highlighting}
\end{Shaded}

\textbf{Arguments:}

\begin{itemize}
\tightlist
\item
  \texttt{\ dx\ } : {[} \texttt{\ length\ } {]} â€'' Offset X relative
  to the pin.
\item
  \texttt{\ dy\ } : {[} \texttt{\ length\ } {]} â€'' Offset Y relative
  to the pin.
\item
  \texttt{\ pin-name\ } : {[} \texttt{\ pin\ } {]} â€'' Name of the pin
  to which you want to locate.
\item
  \texttt{\ body\ } : {[} \texttt{\ content\ } {]} â€'' The content you
  want to place.
\end{itemize}

\subsubsection{\texorpdfstring{\texttt{\ pinit-rect\ }}{ pinit-rect }}\label{pinit-rect}

Draw a rectangular shape on the page \textbf{containing all pins} with
optional extended width and height.

\begin{Shaded}
\begin{Highlighting}[]
\NormalTok{\#let pinit{-}rect(}
\NormalTok{  dx: 0em,}
\NormalTok{  dy: {-}1em,}
\NormalTok{  extended{-}width: 0em,}
\NormalTok{  extended{-}height: 1.4em,}
\NormalTok{  pin1,}
\NormalTok{  pin2,}
\NormalTok{  pin3,  // Optional}
\NormalTok{  ..pinX,}
\NormalTok{  ..args,}
\NormalTok{) = \{ .. \}}
\end{Highlighting}
\end{Shaded}

\textbf{Arguments:}

\begin{itemize}
\tightlist
\item
  \texttt{\ dx\ } : {[} \texttt{\ length\ } {]} â€'' Offset X relative
  to the min-left of pins.
\item
  \texttt{\ dy\ } : {[} \texttt{\ length\ } {]} â€'' Offset Y relative
  to the min-top of pins.
\item
  \texttt{\ extended-width\ } : {[} \texttt{\ length\ } {]} â€''
  Optional extended width of the rectangular shape.
\item
  \texttt{\ extended-height\ } : {[} \texttt{\ length\ } {]} â€''
  Optional extended height of the rectangular shape.
\item
  \texttt{\ pin1\ } : {[} \texttt{\ pin\ } {]} â€'' One of these pins.
\item
  \texttt{\ pin2\ } : {[} \texttt{\ pin\ } {]} â€'' One of these pins.
\item
  \texttt{\ pin3\ } : {[} \texttt{\ pin\ } {]} â€'' One of these pins,
  optionally.
\item
  \texttt{\ ...args\ } : Additional named arguments or settings for
  \href{https://typst.app/docs/reference/visualize/rect/}{\texttt{\ rect\ }}
  , like \texttt{\ fill\ } , \texttt{\ stroke\ } and \texttt{\ radius\ }
  .
\end{itemize}

\subsubsection{\texorpdfstring{\texttt{\ pinit-highlight\ }}{ pinit-highlight }}\label{pinit-highlight}

Highlight a specific area on the page with a filled color and optional
radius and stroke. It is just a simply styled \texttt{\ pinit-rect\ } .

\begin{Shaded}
\begin{Highlighting}[]
\NormalTok{\#let pinit{-}highlight(}
\NormalTok{  fill: rgb(255, 0, 0, 20),}
\NormalTok{  radius: 5pt,}
\NormalTok{  stroke: 0pt,}
\NormalTok{  dx: 0em,}
\NormalTok{  dy: {-}1em,}
\NormalTok{  extended{-}width: 0em,}
\NormalTok{  extended{-}height: 1.4em,}
\NormalTok{  pin1,}
\NormalTok{  pin2,}
\NormalTok{  pin3,  // Optional}
\NormalTok{  ..pinX,}
\NormalTok{  ...args,}
\NormalTok{) = \{ .. \}}
\end{Highlighting}
\end{Shaded}

\textbf{Arguments:}

\begin{itemize}
\tightlist
\item
  \texttt{\ fill\ } : {[} \texttt{\ color\ } {]} â€'' The fill color for
  the highlighted area.
\item
  \texttt{\ radius\ } : {[} \texttt{\ length\ } {]} â€'' Optional radius
  for the highlight.
\item
  \texttt{\ stroke\ } : {[} \texttt{\ stroke\ } {]} â€'' Optional stroke
  width for the highlight.
\item
  \texttt{\ dx\ } : {[} \texttt{\ length\ } {]} â€'' Offset X relative
  to the min-left of pins.
\item
  \texttt{\ dy\ } : {[} \texttt{\ length\ } {]} â€'' Offset Y relative
  to the min-top of pins.
\item
  \texttt{\ extended-width\ } : {[} \texttt{\ length\ } {]} â€''
  Optional extended width of the rectangular shape.
\item
  \texttt{\ extended-height\ } : {[} \texttt{\ length\ } {]} â€''
  Optional extended height of the rectangular shape.
\item
  \texttt{\ pin1\ } : {[} \texttt{\ pin\ } {]} â€'' One of these pins.
\item
  \texttt{\ pin2\ } : {[} \texttt{\ pin\ } {]} â€'' One of these pins.
\item
  \texttt{\ pin3\ } : {[} \texttt{\ pin\ } {]} â€'' One of these pins,
  optionally.
\item
  \texttt{\ ...args\ } : Additional arguments or settings for
  \href{https://github.com/typst/packages/raw/main/packages/preview/pinit/0.2.2/\#pinit-rect}{\texttt{\ pinit-rect\ }}
  .
\end{itemize}

\subsubsection{\texorpdfstring{\texttt{\ pinit-line\ }}{ pinit-line }}\label{pinit-line}

Draw a line on the page between two specified pins with an optional
stroke.

\begin{Shaded}
\begin{Highlighting}[]
\NormalTok{\#let pinit{-}line(}
\NormalTok{  stroke: 1pt,}
\NormalTok{  start{-}dx: 0pt,}
\NormalTok{  start{-}dy: 0pt,}
\NormalTok{  end{-}dx: 0pt,}
\NormalTok{  end{-}dy: 0pt,}
\NormalTok{  start,}
\NormalTok{  end,}
\NormalTok{) = \{ ... \}}
\end{Highlighting}
\end{Shaded}

\textbf{Arguments:}

\begin{itemize}
\tightlist
\item
  \texttt{\ stroke\ } : {[} \texttt{\ stroke\ } {]} â€'' The stroke for
  the line.
\item
  \texttt{\ start-dx\ } : {[} \texttt{\ length\ } {]} â€'' Offset X
  relative to the start pin.
\item
  \texttt{\ start-dy\ } : {[} \texttt{\ length\ } {]} â€'' Offset Y
  relative to the start pin.
\item
  \texttt{\ end-dx\ } : {[} \texttt{\ length\ } {]} â€'' Offset X
  relative to the end pin.
\item
  \texttt{\ end-dy\ } : {[} \texttt{\ length\ } {]} â€'' Offset Y
  relative to the end pin.
\item
  \texttt{\ start\ } : {[} \texttt{\ pin\ } {]} â€'' The start pin.
\item
  \texttt{\ end\ } : {[} \texttt{\ pin\ } {]} â€'' The end pin.
\end{itemize}

\subsubsection{\texorpdfstring{\texttt{\ pinit-line-to\ }}{ pinit-line-to }}\label{pinit-line-to}

Draw an line from a specified pin to a point on the page with optional
settings.

\begin{Shaded}
\begin{Highlighting}[]
\NormalTok{\#let pinit{-}line{-}to(}
\NormalTok{  stroke: 1pt,}
\NormalTok{  pin{-}dx: 5pt,}
\NormalTok{  pin{-}dy: 5pt,}
\NormalTok{  body{-}dx: 5pt,}
\NormalTok{  body{-}dy: 5pt,}
\NormalTok{  offset{-}dx: 35pt,}
\NormalTok{  offset{-}dy: 35pt,}
\NormalTok{  pin{-}name,}
\NormalTok{  body,}
\NormalTok{) = \{ ... \}}
\end{Highlighting}
\end{Shaded}

\textbf{Arguments:}

\begin{itemize}
\tightlist
\item
  \texttt{\ stroke\ } : {[} \texttt{\ stroke\ } {]} â€'' The stroke for
  the line.
\item
  \texttt{\ pin-dx\ } : {[} \texttt{\ length\ } {]} â€'' Offset X of
  arrow start relative to the pin.
\item
  \texttt{\ pin-dy\ } : {[} \texttt{\ length\ } {]} â€'' Offset Y of
  arrow start relative to the pin.
\item
  \texttt{\ body-dx\ } : {[} \texttt{\ length\ } {]} â€'' Offset X of
  arrow end relative to the body.
\item
  \texttt{\ body-dy\ } : {[} \texttt{\ length\ } {]} â€'' Offset Y of
  arrow end relative to the body.
\item
  \texttt{\ offset-dx\ } : {[} \texttt{\ length\ } {]} â€'' Offset X
  relative to the pin.
\item
  \texttt{\ offset-dy\ } : {[} \texttt{\ length\ } {]} â€'' Offset Y
  relative to the pin.
\item
  \texttt{\ pin-name\ } : {[} \texttt{\ pin\ } {]} â€'' The name of the
  pin to start from.
\item
  \texttt{\ body\ } : {[} \texttt{\ content\ } {]} â€'' The content to
  draw the arrow to.
\end{itemize}

\subsubsection{\texorpdfstring{\texttt{\ pinit-arrow\ }}{ pinit-arrow }}\label{pinit-arrow}

Draw an arrow between two specified pins with optional settings.

\begin{Shaded}
\begin{Highlighting}[]
\NormalTok{\#let pinit{-}arrow(}
\NormalTok{  start{-}dx: 0pt,}
\NormalTok{  start{-}dy: 0pt,}
\NormalTok{  end{-}dx: 0pt,}
\NormalTok{  end{-}dy: 0pt,}
\NormalTok{  start,}
\NormalTok{  end,}
\NormalTok{  ..args,}
\NormalTok{) = \{ ... \}}
\end{Highlighting}
\end{Shaded}

\textbf{Arguments:}

\begin{itemize}
\tightlist
\item
  \texttt{\ start-dx\ } : {[} \texttt{\ length\ } {]} â€'' Offset X
  relative to the start pin.
\item
  \texttt{\ start-dy\ } : {[} \texttt{\ length\ } {]} â€'' Offset Y
  relative to the start pin.
\item
  \texttt{\ end-dx\ } : {[} \texttt{\ length\ } {]} â€'' Offset X
  relative to the end pin.
\item
  \texttt{\ end-dy\ } : {[} \texttt{\ length\ } {]} â€'' Offset Y
  relative to the end pin.
\item
  \texttt{\ start\ } : {[} \texttt{\ pin\ } {]} â€'' The start pin.
\item
  \texttt{\ end\ } : {[} \texttt{\ pin\ } {]} â€'' The end pin.
\item
  \texttt{\ ...args\ } : Additional arguments or settings for
  \href{https://github.com/typst/packages/raw/main/packages/preview/pinit/0.2.2/\#simple-arrow}{\texttt{\ simple-arrow\ }}
  , like \texttt{\ fill\ } , \texttt{\ stroke\ } and
  \texttt{\ thickness\ } .
\end{itemize}

\subsubsection{\texorpdfstring{\texttt{\ pinit-double-arrow\ }}{ pinit-double-arrow }}\label{pinit-double-arrow}

Draw an double arrow between two specified pins with optional settings.

\begin{Shaded}
\begin{Highlighting}[]
\NormalTok{\#let pinit{-}double{-}arrow(}
\NormalTok{  start{-}dx: 0pt,}
\NormalTok{  start{-}dy: 0pt,}
\NormalTok{  end{-}dx: 0pt,}
\NormalTok{  end{-}dy: 0pt,}
\NormalTok{  start,}
\NormalTok{  end,}
\NormalTok{  ..args,}
\NormalTok{) = \{ ... \}}
\end{Highlighting}
\end{Shaded}

\textbf{Arguments:}

\begin{itemize}
\tightlist
\item
  \texttt{\ start-dx\ } : {[} \texttt{\ length\ } {]} â€'' Offset X
  relative to the start pin.
\item
  \texttt{\ start-dy\ } : {[} \texttt{\ length\ } {]} â€'' Offset Y
  relative to the start pin.
\item
  \texttt{\ end-dx\ } : {[} \texttt{\ length\ } {]} â€'' Offset X
  relative to the end pin.
\item
  \texttt{\ end-dy\ } : {[} \texttt{\ length\ } {]} â€'' Offset Y
  relative to the end pin.
\item
  \texttt{\ start\ } : {[} \texttt{\ pin\ } {]} â€'' The start pin.
\item
  \texttt{\ end\ } : {[} \texttt{\ pin\ } {]} â€'' The end pin.
\item
  \texttt{\ ...args\ } : Additional arguments or settings for
  \href{https://github.com/typst/packages/raw/main/packages/preview/pinit/0.2.2/\#double-arrow}{\texttt{\ double-arrow\ }}
  , like \texttt{\ fill\ } , \texttt{\ stroke\ } and
  \texttt{\ thickness\ } .
\end{itemize}

\subsubsection{\texorpdfstring{\texttt{\ pinit-point-to\ }}{ pinit-point-to }}\label{pinit-point-to}

Draw an arrow from a specified pin to a point on the page with optional
settings.

\begin{Shaded}
\begin{Highlighting}[]
\NormalTok{\#let pinit{-}point{-}to(}
\NormalTok{  pin{-}dx: 5pt,}
\NormalTok{  pin{-}dy: 5pt,}
\NormalTok{  body{-}dx: 5pt,}
\NormalTok{  body{-}dy: 5pt,}
\NormalTok{  offset{-}dx: 35pt,}
\NormalTok{  offset{-}dy: 35pt,}
\NormalTok{  double: false,}
\NormalTok{  pin{-}name,}
\NormalTok{  body,}
\NormalTok{  ..args,}
\NormalTok{) = \{ ... \}}
\end{Highlighting}
\end{Shaded}

\textbf{Arguments:}

\begin{itemize}
\tightlist
\item
  \texttt{\ pin-dx\ } : {[} \texttt{\ length\ } {]} â€'' Offset X of
  arrow start relative to the pin.
\item
  \texttt{\ pin-dy\ } : {[} \texttt{\ length\ } {]} â€'' Offset Y of
  arrow start relative to the pin.
\item
  \texttt{\ body-dx\ } : {[} \texttt{\ length\ } {]} â€'' Offset X of
  arrow end relative to the body.
\item
  \texttt{\ body-dy\ } : {[} \texttt{\ length\ } {]} â€'' Offset Y of
  arrow end relative to the body.
\item
  \texttt{\ offset-dx\ } : {[} \texttt{\ length\ } {]} â€'' Offset X
  relative to the pin.
\item
  \texttt{\ offset-dy\ } : {[} \texttt{\ length\ } {]} â€'' Offset Y
  relative to the pin.
\item
  \texttt{\ double\ } : {[} \texttt{\ bool\ } {]} â€'' Draw a double
  arrow, default is \texttt{\ false\ } .
\item
  \texttt{\ pin-name\ } : {[} \texttt{\ pin\ } {]} â€'' The name of the
  pin to start from.
\item
  \texttt{\ body\ } : {[} \texttt{\ content\ } {]} â€'' The content to
  draw the arrow to.
\item
  \texttt{\ ...args\ } : Additional arguments or settings for
  \href{https://github.com/typst/packages/raw/main/packages/preview/pinit/0.2.2/\#simple-arrow}{\texttt{\ simple-arrow\ }}
  , like \texttt{\ fill\ } , \texttt{\ stroke\ } and
  \texttt{\ thickness\ } .
\end{itemize}

\subsubsection{\texorpdfstring{\texttt{\ pinit-point-from\ }}{ pinit-point-from }}\label{pinit-point-from}

Draw an arrow from a point on the page to a specified pin with optional
settings.

\begin{Shaded}
\begin{Highlighting}[]
\NormalTok{\#let pinit{-}point{-}from(}
\NormalTok{  pin{-}dx: 5pt,}
\NormalTok{  pin{-}dy: 5pt,}
\NormalTok{  body{-}dx: 5pt,}
\NormalTok{  body{-}dy: 5pt,}
\NormalTok{  offset{-}dx: 35pt,}
\NormalTok{  offset{-}dy: 35pt,}
\NormalTok{  double: false,}
\NormalTok{  pin{-}name,}
\NormalTok{  body,}
\NormalTok{  ..args,}
\NormalTok{) = \{ ... \}}
\end{Highlighting}
\end{Shaded}

\textbf{Arguments:}

\begin{itemize}
\tightlist
\item
  \texttt{\ pin-dx\ } : {[} \texttt{\ length\ } {]} â€'' Offset X
  relative to the pin.
\item
  \texttt{\ pin-dy\ } : {[} \texttt{\ length\ } {]} â€'' Offset Y
  relative to the pin.
\item
  \texttt{\ body-dx\ } : {[} \texttt{\ length\ } {]} â€'' Offset X
  relative to the body.
\item
  \texttt{\ body-dy\ } : {[} \texttt{\ length\ } {]} â€'' Offset Y
  relative to the body.
\item
  \texttt{\ offset-dx\ } : {[} \texttt{\ length\ } {]} â€'' Offset X
  relative to the left edge of the page.
\item
  \texttt{\ offset-dy\ } : {[} \texttt{\ length\ } {]} â€'' Offset Y
  relative to the top edge of the page.
\item
  \texttt{\ double\ } : {[} \texttt{\ bool\ } {]} â€'' Draw a double
  arrow, default is \texttt{\ false\ } .
\item
  \texttt{\ pin-name\ } : {[} \texttt{\ pin\ } {]} â€'' The name of the
  pin that the arrow to.
\item
  \texttt{\ body\ } : {[} \texttt{\ content\ } {]} â€'' The content to
  draw the arrow from.
\item
  \texttt{\ ...args\ } : Additional arguments or settings for
  \href{https://github.com/typst/packages/raw/main/packages/preview/pinit/0.2.2/\#simple-arrow}{\texttt{\ simple-arrow\ }}
  , like \texttt{\ fill\ } , \texttt{\ stroke\ } and
  \texttt{\ thickness\ } .
\end{itemize}

\subsubsection{\texorpdfstring{\texttt{\ simple-arrow\ }}{ simple-arrow }}\label{simple-arrow}

Draw a simple arrow on the page with optional settings, implemented by
\href{https://typst.app/docs/reference/visualize/polygon/}{\texttt{\ polygon\ }}
.

\begin{Shaded}
\begin{Highlighting}[]
\NormalTok{\#let simple{-}arrow(}
\NormalTok{  fill: black,}
\NormalTok{  stroke: 0pt,}
\NormalTok{  start: (0pt, 0pt),}
\NormalTok{  end: (30pt, 0pt),}
\NormalTok{  thickness: 2pt,}
\NormalTok{  arrow{-}width: 4,}
\NormalTok{  arrow{-}height: 4,}
\NormalTok{  inset: 0.5,}
\NormalTok{  tail: (),}
\NormalTok{) = \{ ... \}}
\end{Highlighting}
\end{Shaded}

\textbf{Arguments:}

\begin{itemize}
\tightlist
\item
  \texttt{\ fill\ } : {[} \texttt{\ color\ } {]} â€'' The fill color for
  the arrow.
\item
  \texttt{\ stroke\ } : {[} \texttt{\ stroke\ } {]} â€'' The stroke for
  the arrow.
\item
  \texttt{\ start\ } : {[} \texttt{\ point\ } {]} â€'' The starting
  point of the arrow.
\item
  \texttt{\ end\ } : {[} \texttt{\ point\ } {]} â€'' The ending point of
  the arrow.
\item
  \texttt{\ thickness\ } : {[} \texttt{\ length\ } {]} â€'' The
  thickness of the arrow.
\item
  \texttt{\ arrow-width\ } : {[} \texttt{\ int\ } or \texttt{\ float\ }
  {]} â€'' The width of the arrowhead relative to thickness.
\item
  \texttt{\ arrow-height\ } : {[} \texttt{\ int\ } or \texttt{\ float\ }
  {]} â€'' The height of the arrowhead relative to thickness.
\item
  \texttt{\ inset\ } : {[} \texttt{\ int\ } or \texttt{\ float\ } {]}
  â€'' The inset value for the arrowhead relative to thickness.
\item
  \texttt{\ tail\ } : {[} \texttt{\ array\ } {]} â€'' The tail settings
  for the arrow.
\end{itemize}

\subsubsection{\texorpdfstring{\texttt{\ double-arrow\ }}{ double-arrow }}\label{double-arrow}

Draw a double arrow on the page with optional settings, implemented by
\href{https://typst.app/docs/reference/visualize/polygon/}{\texttt{\ polygon\ }}
.

\begin{Shaded}
\begin{Highlighting}[]
\NormalTok{\#let double{-}arrow(}
\NormalTok{  fill: black,}
\NormalTok{  stroke: 0pt,}
\NormalTok{  start: (0pt, 0pt),}
\NormalTok{  end: (30pt, 0pt),}
\NormalTok{  thickness: 2pt,}
\NormalTok{  arrow{-}width: 4,}
\NormalTok{  arrow{-}height: 4,}
\NormalTok{  inset: 0.5,}
\NormalTok{  tail: (),}
\NormalTok{) = \{ ... \}}
\end{Highlighting}
\end{Shaded}

\textbf{Arguments:}

\begin{itemize}
\tightlist
\item
  \texttt{\ fill\ } : {[} \texttt{\ color\ } {]} â€'' The fill color for
  the arrow.
\item
  \texttt{\ stroke\ } : {[} \texttt{\ stroke\ } {]} â€'' The stroke for
  the arrow.
\item
  \texttt{\ start\ } : {[} \texttt{\ point\ } {]} â€'' The starting
  point of the arrow.
\item
  \texttt{\ end\ } : {[} \texttt{\ point\ } {]} â€'' The ending point of
  the arrow.
\item
  \texttt{\ thickness\ } : {[} \texttt{\ length\ } {]} â€'' The
  thickness of the arrow.
\item
  \texttt{\ arrow-width\ } : {[} \texttt{\ int\ } or \texttt{\ float\ }
  {]} â€'' The width of the arrowhead relative to thickness.
\item
  \texttt{\ arrow-height\ } : {[} \texttt{\ int\ } or \texttt{\ float\ }
  {]} â€'' The height of the arrowhead relative to thickness.
\item
  \texttt{\ inset\ } : {[} \texttt{\ int\ } or \texttt{\ float\ } {]}
  â€'' The inset value for the arrowhead relative to thickness.
\item
  \texttt{\ tail\ } : {[} \texttt{\ array\ } {]} â€'' The tail settings
  for the arrow.
\end{itemize}

\subsubsection{\texorpdfstring{\texttt{\ pinit-fletcher-edge\ }}{ pinit-fletcher-edge }}\label{pinit-fletcher-edge}

Draw a connecting line or arc in an fletcher arrow diagram.

\begin{Shaded}
\begin{Highlighting}[]
\NormalTok{\#let pinit{-}fletcher{-}edge(}
\NormalTok{  fletcher,}
\NormalTok{  start,}
\NormalTok{  end: none,}
\NormalTok{  start{-}dx: 0pt,}
\NormalTok{  start{-}dy: 0pt,}
\NormalTok{  end{-}dx: 0pt,}
\NormalTok{  end{-}dy: 0pt,}
\NormalTok{  width{-}scale: 100\%,}
\NormalTok{  height{-}scale: 100\%,}
\NormalTok{  default{-}width: 30pt,}
\NormalTok{  default{-}height: 30pt,}
\NormalTok{    ..args,}
\NormalTok{) = \{ ... \}}
\end{Highlighting}
\end{Shaded}

\textbf{Arguments:}

\begin{itemize}
\tightlist
\item
  \texttt{\ fletcher\ } (module): The Fletcher module. You can import it
  with something like \texttt{\ \#import\ "@preview/fletcher:0.5.1"\ }
\item
  \texttt{\ start\ } (pin): The starting pin of the edge. It is assumed
  that the pin is at the \emph{origin point (0, 0)} of the edge.
\item
  \texttt{\ end\ } (pin): The ending pin of the edge. If not provided,
  the edge will use default values for the width and height.
\item
  \texttt{\ start-dx\ } (length): The x-offset of the starting pin. You
  should use pt units.
\item
  \texttt{\ start-dy\ } (length): The y-offset of the starting pin. You
  should use pt units.
\item
  \texttt{\ end-dx\ } (length): The x-offset of the ending pin. You
  should use pt units.
\item
  \texttt{\ end-dy\ } (length): The y-offset of the ending pin. You
  should use pt units.
\item
  \texttt{\ width-scale\ } (percent): The width scale of the edge. The
  default value is 100\%. If you set the width scale to 50\%, the width
  of the edge will be half of the default width. Then you can use
  \texttt{\ "r,r"\ } which is equivalent to single \texttt{\ "r"\ } .
\item
  \texttt{\ height-scale\ } (percent): The height scale of the edge. The
  default value is 100\%.
\item
  \texttt{\ default-width\ } (length): The default width of the edge.
  The default value is 30pt, which will only be used if the end pin is
  not provided or the width is 0pt or 0em.
\item
  \texttt{\ default-height\ } (length): The default height of the edge.
  The default value is 30pt, which will only be used if the end pin is
  not provided or the height is 0pt or 0em.
\item
  \texttt{\ ..args\ } (any): An edge’s positional arguments may
  specify:

  \begin{itemize}
  \tightlist
  \item
    the edge’s \#param{[}edge{]}{[}vertices{]}, each specified with a
    CeTZ-style coordinate
  \item
    the \#param{[}edge{]}{[}label{]} content
  \item
    arrow \#param{[}edge{]}{[}marks{]}, like
    \texttt{\ "=\textgreater{}"\ } or
    \texttt{\ "\textless{}\textless{}-\textbar{}-o"\ }
  \item
    other style flags, like \texttt{\ "double"\ } or \texttt{\ "wave"\ }
  \end{itemize}
\end{itemize}

\subsection{Changelog}\label{changelog}

\subsubsection{0.2.2}\label{section}

\begin{itemize}
\tightlist
\item
  Fix bugs.
\end{itemize}

\subsubsection{0.2.1}\label{section-1}

\begin{itemize}
\tightlist
\item
  To be compatible with Typst 0.12.
\end{itemize}

\subsubsection{0.2.0}\label{section-2}

\begin{itemize}
\tightlist
\item
  \textbf{Breaking changes} : \texttt{\ \#pinit(pins,\ func)\ } is
  replaced by \texttt{\ \#pinit(callback:\ none,\ ..pins)\ } and the
  callback argument will receive an
  \texttt{\ (..positions)\ =\textgreater{}\ \{\ ..\ \}\ } function
  instead of a \texttt{\ (positions)\ =\textgreater{}\ \{\ ..\ \}\ }
  function.

  \begin{itemize}
  \tightlist
  \item
    \textbf{Migration} : you need to use a named argument
    \texttt{\ callback:\ (..positions)\ =\textgreater{}\ \{\ ..\ \}\ }
    to specify the callback function.
  \item
    \textbf{Migration} : you cannot use a array as a pin name. Now
    \texttt{\ \#pinit((pin1,\ pin2),\ callback:\ func)\ } means that we
    use \texttt{\ pin1\ } and \texttt{\ pin2\ } as a group of pins, and
    the callback function will receive \textbf{a single position} (the
    center of the bounding box of \texttt{\ pin1\ } and
    \texttt{\ pin2\ } ).
  \item
    \textbf{Benefit} : you can use
    \texttt{\ \#pinit(pin1,\ pin2,\ callback:\ func)\ } to query the
    positions of \texttt{\ pin1\ } and \texttt{\ pin2\ } separately, and
    \texttt{\ \#pinit((pin1,\ pin2),\ callback:\ func)\ } to query the
    position of the center of the bounding box of \texttt{\ pin1\ } and
    \texttt{\ pin2\ } .
  \end{itemize}
\item
  Add \texttt{\ pinit-fletcher-edge\ } function to draw a connecting
  line or arc in an fletcher arrow diagram.
\item
  Add \texttt{\ double-arrow\ } function and
  \texttt{\ pinit-double-arrow\ } function.
\item
  Add \texttt{\ double\ } argument for \texttt{\ pinit-point-to\ } and
  \texttt{\ pinit-point-from\ } functions.
\item
  Better comments and documentation.
\end{itemize}

\subsubsection{0.1.4}\label{section-3}

\begin{itemize}
\tightlist
\item
  Update documentation.
\end{itemize}

\subsubsection{0.1.3}\label{section-4}

\begin{itemize}
\tightlist
\item
  Add \texttt{\ pinit-line-to\ } function.
\end{itemize}

\subsubsection{0.1.2}\label{section-5}

\begin{itemize}
\tightlist
\item
  Add em unit support for \texttt{\ simple-arrow\ } .
\end{itemize}

\subsubsection{0.1.1}\label{section-6}

\begin{itemize}
\tightlist
\item
  Fix some bugs.
\end{itemize}

\subsubsection{0.1.0}\label{section-7}

\begin{itemize}
\tightlist
\item
  Initial release.
\end{itemize}

\subsection{Acknowledgements}\label{acknowledgements}

\begin{itemize}
\tightlist
\item
  Some of the inspirations and codes comes from
  \href{https://github.com/ntjess/typst-drafting}{typst-drafting} .
\item
  The concise and aesthetic example slide style come from course
  \emph{Data Structures and Algorithms} of
  \href{https://chaodong.me/}{Chaodong ZHENG} .
\item
  Thank \href{https://github.com/psads-git}{PaulS} for double arrow
  feature.
\item
  Thank \href{https://github.com/Jollywatt}{Jollywatt} for fletcher
  package.
\end{itemize}

\subsection{License}\label{license}

This project is licensed under the MIT License.

\subsubsection{How to add}\label{how-to-add}

Copy this into your project and use the import as \texttt{\ pinit\ }

\begin{verbatim}
#import "@preview/pinit:0.2.2"
\end{verbatim}

\includesvg[width=0.16667in,height=0.16667in]{/assets/icons/16-copy.svg}

Check the docs for
\href{https://typst.app/docs/reference/scripting/\#packages}{more
information on how to import packages} .

\subsubsection{About}\label{about}

\begin{description}
\tightlist
\item[Author :]
OrangeX4
\item[License:]
MIT
\item[Current version:]
0.2.2
\item[Last updated:]
October 17, 2024
\item[First released:]
November 7, 2023
\item[Archive size:]
14.0 kB
\href{https://packages.typst.org/preview/pinit-0.2.2.tar.gz}{\pandocbounded{\includesvg[keepaspectratio]{/assets/icons/16-download.svg}}}
\item[Repository:]
\href{https://github.com/OrangeX4/typst-pinit}{GitHub}
\item[Categor ies :]
\begin{itemize}
\tightlist
\item[]
\item
  \pandocbounded{\includesvg[keepaspectratio]{/assets/icons/16-layout.svg}}
  \href{https://typst.app/universe/search/?category=layout}{Layout}
\item
  \pandocbounded{\includesvg[keepaspectratio]{/assets/icons/16-hammer.svg}}
  \href{https://typst.app/universe/search/?category=utility}{Utility}
\end{itemize}
\end{description}

\subsubsection{Where to report issues?}\label{where-to-report-issues}

This package is a project of OrangeX4 . Report issues on
\href{https://github.com/OrangeX4/typst-pinit}{their repository} . You
can also try to ask for help with this package on the
\href{https://forum.typst.app}{Forum} .

Please report this package to the Typst team using the
\href{https://typst.app/contact}{contact form} if you believe it is a
safety hazard or infringes upon your rights.

\phantomsection\label{versions}
\subsubsection{Version history}\label{version-history}

\begin{longtable}[]{@{}ll@{}}
\toprule\noalign{}
Version & Release Date \\
\midrule\noalign{}
\endhead
\bottomrule\noalign{}
\endlastfoot
0.2.2 & October 17, 2024 \\
\href{https://typst.app/universe/package/pinit/0.2.1/}{0.2.1} & October
16, 2024 \\
\href{https://typst.app/universe/package/pinit/0.2.0/}{0.2.0} & August
26, 2024 \\
\href{https://typst.app/universe/package/pinit/0.1.4/}{0.1.4} & April
12, 2024 \\
\href{https://typst.app/universe/package/pinit/0.1.3/}{0.1.3} & December
23, 2023 \\
\href{https://typst.app/universe/package/pinit/0.1.2/}{0.1.2} & November
29, 2023 \\
\href{https://typst.app/universe/package/pinit/0.1.1/}{0.1.1} & November
7, 2023 \\
\href{https://typst.app/universe/package/pinit/0.1.0/}{0.1.0} & November
7, 2023 \\
\end{longtable}

Typst GmbH did not create this package and cannot guarantee correct
functionality of this package or compatibility with any version of the
Typst compiler or app.
