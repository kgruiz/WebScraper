\title{typst.app/universe/package/board-n-pieces}

\phantomsection\label{banner}
\section{board-n-pieces}\label{board-n-pieces}

{ 0.5.0 }

Display chessboards.

\phantomsection\label{readme}
Display chessboards in Typst.

\subsection{Displaying chessboards}\label{displaying-chessboards}

The main function of this package is \texttt{\ board\ } . It lets you
display a specific position on a board.

\begin{Shaded}
\begin{Highlighting}[]
\NormalTok{\#board(starting{-}position)}
\end{Highlighting}
\end{Shaded}

\pandocbounded{\includesvg[keepaspectratio]{https://github.com/typst/packages/raw/main/packages/preview/board-n-pieces/0.5.0/examples/example-1.svg}}

\texttt{\ starting-position\ } is a position that is provided by the
package. It represents the initial position of a chess game.

You can create a different position using the \texttt{\ position\ }
function. It accepts strings representing each rank. Use upper-case
letters for white pieces, and lower-case letters for black pieces. Dots
and spaces correspond to empty squares.

\begin{Shaded}
\begin{Highlighting}[]
\NormalTok{\#board(position(}
\NormalTok{  "....r...",}
\NormalTok{  "........",}
\NormalTok{  "..p..PPk",}
\NormalTok{  ".p.r....",}
\NormalTok{  "pP..p.R.",}
\NormalTok{  "P.B.....",}
\NormalTok{  "..P..K..",}
\NormalTok{  "........",}
\NormalTok{))}
\end{Highlighting}
\end{Shaded}

\pandocbounded{\includesvg[keepaspectratio]{https://github.com/typst/packages/raw/main/packages/preview/board-n-pieces/0.5.0/examples/example-2.svg}}

Alternatively, you can use the \texttt{\ fen\ } function to create a
position using
\href{https://en.wikipedia.org/wiki/Forsyth\%E2\%80\%93Edwards_Notation}{Forsythâ€``Edwards
notation} :

\begin{Shaded}
\begin{Highlighting}[]
\NormalTok{\#board(fen("r1bk3r/p2pBpNp/n4n2/1p1NP2P/6P1/3P4/P1P1K3/q5b1 b {-} {-} 1 23"))}
\end{Highlighting}
\end{Shaded}

\pandocbounded{\includesvg[keepaspectratio]{https://github.com/typst/packages/raw/main/packages/preview/board-n-pieces/0.5.0/examples/example-3.svg}}

Note that you can specify only the first part of the FEN string:

\begin{Shaded}
\begin{Highlighting}[]
\NormalTok{\#board(fen("r4rk1/pp2Bpbp/1qp3p1/8/2BP2b1/Q1n2N2/P4PPP/3RK2R"))}
\end{Highlighting}
\end{Shaded}

\pandocbounded{\includesvg[keepaspectratio]{https://github.com/typst/packages/raw/main/packages/preview/board-n-pieces/0.5.0/examples/example-4.svg}}

Also note that positions do not need to be on a standard 8Ã---8 board:

\begin{Shaded}
\begin{Highlighting}[]
\NormalTok{\#board(position(}
\NormalTok{  "....Q....",}
\NormalTok{  "......Q..",}
\NormalTok{  "........Q",}
\NormalTok{  "...Q.....",}
\NormalTok{  ".Q.......",}
\NormalTok{  ".......Q.",}
\NormalTok{  ".....Q...",}
\NormalTok{  "..Q......",}
\NormalTok{  "Q........",}
\NormalTok{))}
\end{Highlighting}
\end{Shaded}

\pandocbounded{\includesvg[keepaspectratio]{https://github.com/typst/packages/raw/main/packages/preview/board-n-pieces/0.5.0/examples/example-5.svg}}

\subsection{\texorpdfstring{Using the \texttt{\ game\ }
function}{Using the  game  function}}\label{using-the-game-function}

The \texttt{\ game\ } function creates an array of positions from a full
chess game. A game is described by a series of turns written in
\href{https://en.wikipedia.org/wiki/Algebraic_notation_(chess)}{standard
algebraic notation} . Those turns can be specified as an array of
strings, or as a single string containing whitespace-separated moves.

\begin{Shaded}
\begin{Highlighting}[]
\NormalTok{The scholar\textquotesingle{}s mate:}
\NormalTok{\#let positions = game("e4 e5 Qh5 Nc6 Bc4 Nf6 Qxf7")}
\NormalTok{\#grid(}
\NormalTok{  columns: 4,}
\NormalTok{  gutter: 0.2cm,}
\NormalTok{  ..positions.map(board.with(square{-}size: 0.5cm)),}
\NormalTok{)}
\end{Highlighting}
\end{Shaded}

\pandocbounded{\includesvg[keepaspectratio]{https://github.com/typst/packages/raw/main/packages/preview/board-n-pieces/0.5.0/examples/example-6.svg}}

You can specify an alternative starting position to the
\texttt{\ game\ } function with the \texttt{\ starting-position\ } named
argument.

\subsection{\texorpdfstring{Using the \texttt{\ pgn\ } function to
import PGN
files}{Using the  pgn  function to import PGN files}}\label{using-the-pgn-function-to-import-pgn-files}

Similarly to the \texttt{\ game\ } function, the \texttt{\ pgn\ }
function creates an array of positions. It accepts a single argument,
which is a string containing
\href{https://en.wikipedia.org/wiki/Portable_Game_Notation}{portable
game notation} . To read a game from a PGN file, you can use this
function in combination with Typst’s native
\href{https://typst.app/docs/reference/data-loading/read/}{\texttt{\ read\ }}
function.

\begin{Shaded}
\begin{Highlighting}[]
\NormalTok{\#let positions = pgn(read("game.pgn"))}
\end{Highlighting}
\end{Shaded}

Note that the argument to \texttt{\ pgn\ } must describe a single game.
If you have a PGN file containing multiple games, you will need to split
them using other means.

\subsection{Using non-standard chess
pieces}\label{using-non-standard-chess-pieces}

The \texttt{\ board\ } function’s \texttt{\ pieces\ } argument lets
you specify how to display each piece by mapping each piece character to
some content. You can use this feature to display non-standard chess
pieces:

\begin{Shaded}
\begin{Highlighting}[]
\NormalTok{\#board(}
\NormalTok{  fen("g7/5g2/8/8/8/8/p6g/k1K4G"),}
\NormalTok{  pieces: (}
\NormalTok{    // We use symbols for the example.}
\NormalTok{    // In practice, you should import your own images.}
\NormalTok{    g: chess{-}sym.queen.black.b,}
\NormalTok{    p: chess{-}sym.pawn.black,}
\NormalTok{    k: chess{-}sym.king.black,}
\NormalTok{    K: chess{-}sym.king.white,}
\NormalTok{    G: chess{-}sym.queen.white.b,}
\NormalTok{  ),}
\NormalTok{)}
\end{Highlighting}
\end{Shaded}

\pandocbounded{\includesvg[keepaspectratio]{https://github.com/typst/packages/raw/main/packages/preview/board-n-pieces/0.5.0/examples/example-7.svg}}

\subsection{Customizing a chessboard}\label{customizing-a-chessboard}

The \texttt{\ board\ } function lets you customize the appearance of the
board in various ways, as illustrated in the example below.

\begin{Shaded}
\begin{Highlighting}[]
\NormalTok{// From https://lichess.org/study/Xf1PGrM0.}
\NormalTok{\#board(}
\NormalTok{  fen("3k4/7R/8/2PK4/8/8/8/6r1 b {-} {-} 0 1"),}

\NormalTok{  marked{-}squares: "c7 c6 h6",}
\NormalTok{  arrows: ("d8 c8", "d8 c7", "g1 g6", "h7 h6"),}
\NormalTok{  display{-}numbers: true,}

\NormalTok{  white{-}square{-}fill: rgb("\#d2eeea"),}
\NormalTok{  black{-}square{-}fill: rgb("\#567f96"),}
\NormalTok{  marking{-}color: rgb("\#2bcbC6"),}
\NormalTok{  arrow{-}stroke: 0.2cm + rgb("\#38f442df"),}

\NormalTok{  stroke: 0.8pt + black,}
\NormalTok{)}
\end{Highlighting}
\end{Shaded}

\pandocbounded{\includesvg[keepaspectratio]{https://github.com/typst/packages/raw/main/packages/preview/board-n-pieces/0.5.0/examples/example-8.svg}}

Here is a list of all the available arguments:

\begin{itemize}
\item
  \texttt{\ marked-squares\ } is a list of squares to mark (e.g.,
  \texttt{\ ("d3",\ "d2",\ "e3")\ } ). It can also be specified as a
  single string containing whitespace-separated squares (e.g.,
  \texttt{\ "d3\ d2\ e3"\ } ).
\item
  \texttt{\ arrows\ } is a list of arrows to draw (e.g.,
  \texttt{\ ("e2\ e4",\ "e7\ e5")\ } ).
\item
  \texttt{\ reverse\ } is a boolean indicating whether to reverse the
  board, displaying it from black’s point of view. This is
  \texttt{\ false\ } by default, meaning the board is displayed from
  white’s point of view.
\item
  \texttt{\ display-numbers\ } is a boolean indicating whether ranks and
  files should be numbered. This is \texttt{\ false\ } by default.
\item
  \texttt{\ rank-numbering\ } and \texttt{\ file-numbering\ } are
  functions describing how ranks and files should be numbered. By
  default they are respectively \texttt{\ numbering.with("1")\ } and
  \texttt{\ numbering.with("a")\ } .
\item
  \texttt{\ square-size\ } is a length describing the size of each
  square. By default, this is \texttt{\ 1cm\ } .
\item
  \texttt{\ white-square-fill\ } and \texttt{\ black-square-fill\ }
  indicate how squares should be filled. They can be colors, gradient or
  patterns.
\item
  \texttt{\ marking-color\ } is the color to use for markings (marked
  squares and arrows).
\item
  \texttt{\ marked-white-square-background\ } and
  \texttt{\ marked-black-square-background\ } define the content to
  display in the background of marked squares. By default, this is a
  circle using the \texttt{\ marking-color\ } .
\item
  \texttt{\ arrow-stroke\ } is the stroke to draw the arrows with. If
  only a length is given, \texttt{\ marking-color\ } is used.
  Alternatively, a stroke can be passed to specify a different color.
\item
  \texttt{\ pieces\ } is a dictionary containing images representing
  each piece. If specified, the dictionary must contain an entry for
  every piece kind in the displayed position. Keys are single upper-case
  letters for white pieces and single lower-case letters for black
  pieces. The default images are taken from
  \href{https://commons.wikimedia.org/wiki/Category:SVG_chess_pieces}{Wikimedia
  Commons} . Please refer to
  \href{https://github.com/typst/packages/raw/main/packages/preview/board-n-pieces/0.5.0/\#licensing}{the
  section on licensing} for information on how you can use them in your
  documents.
\item
  \texttt{\ stroke\ } has the same structure as
  \href{https://typst.app/docs/reference/visualize/rect/\#parameters-stroke}{\texttt{\ rect\ }
  ’s \texttt{\ stroke\ } parameter} and corresponds to the stroke to
  use around the board. If \texttt{\ display-numbers\ } is
  \texttt{\ true\ } , the numbers are displayed outside the stroke. The
  default value is \texttt{\ none\ } .
\end{itemize}

\subsection{Chess symbols}\label{chess-symbols}

This package also exports chess symbols for all Unicode chess-related
codepoints under the \texttt{\ chess-sym\ } submodule. Standard chess
pieces are available as
\texttt{\ chess-sym.\{pawn,knight,bishop,rook,queen,king\}.\{white,black,neutral\}\ }
. Alternatively, you can use \texttt{\ stroked\ } and
\texttt{\ filled\ } instead of, respectively, \texttt{\ white\ } and
\texttt{\ black\ } . They can be rotated rightward, downward, and
leftward respectively with with \texttt{\ .r\ } , \texttt{\ .b\ } , and
\texttt{\ .l\ } . Chinese chess pieces are also available as
\texttt{\ chess-sym.\{soldier,cannon,chariot,horse,elephant,mandarin,general\}.\{red,black\}\ }
. Similarly, you can use \texttt{\ stroked\ } and \texttt{\ filled\ } as
alternatives to, respectively, \texttt{\ red\ } and \texttt{\ black\ } .
Note that most fonts only support black and white versions of standard
pieces. To use the other symbols, you may have to use a font such as
Noto Sans Symbols 2.

\begin{Shaded}
\begin{Highlighting}[]
\NormalTok{The best move in this position is \#chess{-}sym.knight.white;c6.}
\end{Highlighting}
\end{Shaded}

\pandocbounded{\includesvg[keepaspectratio]{https://github.com/typst/packages/raw/main/packages/preview/board-n-pieces/0.5.0/examples/example-9.svg}}

\subsection{Licensing}\label{licensing}

The default images for chess pieces used by the \texttt{\ board\ }
function come from
\href{https://commons.wikimedia.org/wiki/Category:SVG_chess_pieces}{Wikimedia
Commons} . They are all licensed the
\href{https://www.gnu.org/licenses/old-licenses/gpl-2.0.html}{GNU
General Public License, version 2} by their original author:
\href{https://en.wikipedia.org/wiki/User:Cburnett}{Cburnett} .

\subsection{Changelog}\label{changelog}

\subsubsection{Version 0.5.0}\label{version-0.5.0}

\begin{itemize}
\item
  Add symbols for all Unicode chess-related codepoints.
\item
  Change the signature of the \texttt{\ board\ } function.

  \begin{itemize}
  \tightlist
  \item
    Rename argument \texttt{\ highlighted-squares\ } to
    \texttt{\ marked-squares\ } .
  \item
    Remove arguments \texttt{\ highlighted-white-square-fill\ } and
    \texttt{\ highlighted-black-square-fill\ } .
  \item
    Add argument \texttt{\ marking-color\ } , together with
    \texttt{\ marked-white-square-background\ } and
    \texttt{\ marked-black-square-background\ } .
  \item
    Support passing a length as \texttt{\ arrow-stroke\ } .
  \end{itemize}
\item
  Fix arrows not being displayed properly on reversed boards.
\end{itemize}

\subsubsection{Version 0.4.0}\label{version-0.4.0}

\begin{itemize}
\tightlist
\item
  Add the ability to draw arrows in \texttt{\ board\ } .
\end{itemize}

\subsubsection{Version 0.3.0}\label{version-0.3.0}

\begin{itemize}
\item
  Detect moves that put the king in check as illegal, improving SAN
  support.
\item
  Add \texttt{\ stroke\ } argument to the \texttt{\ board\ } function.
\item
  Rename \texttt{\ \{highlighted-,\}\{white,black\}-square-color\ }
  arguments to the \texttt{\ board\ } function to
  \texttt{\ \{highlighted-,\}\{white,black\}-square-fill\ } .
\end{itemize}

\subsubsection{Version 0.2.0}\label{version-0.2.0}

\begin{itemize}
\item
  Allow using dashes for empty squares in \texttt{\ position\ }
  function.
\item
  Allow passing highlighted squares as a single string of
  whitespace-separated squares.
\item
  Describe entire games using algebraic notation with the
  \texttt{\ game\ } function.
\item
  Initial PGN support through the \texttt{\ pgn\ } function.
\end{itemize}

\subsubsection{Version 0.1.0}\label{version-0.1.0}

\begin{itemize}
\item
  Display a chess position on a chessboard with the \texttt{\ board\ }
  function.
\item
  Get the starting position with \texttt{\ starting-position\ } .
\item
  Use chess-related symbols with the \texttt{\ chess-sym\ } module.
\end{itemize}

\subsubsection{How to add}\label{how-to-add}

Copy this into your project and use the import as
\texttt{\ board-n-pieces\ }

\begin{verbatim}
#import "@preview/board-n-pieces:0.5.0"
\end{verbatim}

\includesvg[width=0.16667in,height=0.16667in]{/assets/icons/16-copy.svg}

Check the docs for
\href{https://typst.app/docs/reference/scripting/\#packages}{more
information on how to import packages} .

\subsubsection{About}\label{about}

\begin{description}
\tightlist
\item[Author :]
\href{https://github.com/MDLC01}{Malo}
\item[License:]
MIT AND GPL-2.0-only
\item[Current version:]
0.5.0
\item[Last updated:]
July 22, 2024
\item[First released:]
March 20, 2024
\item[Minimum Typst version:]
0.11.0
\item[Archive size:]
52.2 kB
\href{https://packages.typst.org/preview/board-n-pieces-0.5.0.tar.gz}{\pandocbounded{\includesvg[keepaspectratio]{/assets/icons/16-download.svg}}}
\item[Repository:]
\href{https://github.com/MDLC01/board-n-pieces}{GitHub}
\item[Discipline s :]
\begin{itemize}
\tightlist
\item[]
\item
  \href{https://typst.app/universe/search/?discipline=computer-science}{Computer
  Science}
\item
  \href{https://typst.app/universe/search/?discipline=mathematics}{Mathematics}
\end{itemize}
\item[Categor y :]
\begin{itemize}
\tightlist
\item[]
\item
  \pandocbounded{\includesvg[keepaspectratio]{/assets/icons/16-chart.svg}}
  \href{https://typst.app/universe/search/?category=visualization}{Visualization}
\end{itemize}
\end{description}

\subsubsection{Where to report issues?}\label{where-to-report-issues}

This package is a project of Malo . Report issues on
\href{https://github.com/MDLC01/board-n-pieces}{their repository} . You
can also try to ask for help with this package on the
\href{https://forum.typst.app}{Forum} .

Please report this package to the Typst team using the
\href{https://typst.app/contact}{contact form} if you believe it is a
safety hazard or infringes upon your rights.

\phantomsection\label{versions}
\subsubsection{Version history}\label{version-history}

\begin{longtable}[]{@{}ll@{}}
\toprule\noalign{}
Version & Release Date \\
\midrule\noalign{}
\endhead
\bottomrule\noalign{}
\endlastfoot
0.5.0 & July 22, 2024 \\
\href{https://typst.app/universe/package/board-n-pieces/0.4.0/}{0.4.0} &
July 8, 2024 \\
\href{https://typst.app/universe/package/board-n-pieces/0.3.0/}{0.3.0} &
May 23, 2024 \\
\href{https://typst.app/universe/package/board-n-pieces/0.2.0/}{0.2.0} &
April 29, 2024 \\
\href{https://typst.app/universe/package/board-n-pieces/0.1.0/}{0.1.0} &
March 20, 2024 \\
\end{longtable}

Typst GmbH did not create this package and cannot guarantee correct
functionality of this package or compatibility with any version of the
Typst compiler or app.
