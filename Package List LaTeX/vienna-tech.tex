\title{typst.app/universe/package/vienna-tech}

\phantomsection\label{banner}
\phantomsection\label{template-thumbnail}
\pandocbounded{\includegraphics[keepaspectratio]{https://packages.typst.org/preview/thumbnails/vienna-tech-0.1.2-small.webp}}

\section{vienna-tech}\label{vienna-tech}

{ 0.1.2 }

An unofficial template for writing thesis at the TU Wien civil- and
environmental engineering faculty.

\href{/app?template=vienna-tech&version=0.1.2}{Create project in app}

\phantomsection\label{readme}
Version 0.1.2

This is a template, modeled after the LaTeX template provided by the
Vienna University of Technology for Engineering Students. It is intended
to be used as a starting point for writing Bachelor’s or Master’s
theses, but can be adapted for other purposes as well. It shall be noted
that this template is not an official template provided by the Vienna
University of Technology, but rather a personal effort to provide a
similar template in a new typesetting system. If you want to checkout
the original templates visit the website of
\href{https://www.tuwien.at/cee/edvlabor/lehre/vorlagen}{TU Wien}

\subsection{Getting Started}\label{getting-started}

These instructions will help you set up the template on the typst web
app.

\begin{Shaded}
\begin{Highlighting}[]
\NormalTok{\#import "@preview/vienna{-}tech:0.1.2": *}

\NormalTok{// Useing the configuration}
\NormalTok{\#show: tuw{-}thesis.with(}
\NormalTok{  title: [Titel of the Thesis],}
\NormalTok{  thesis{-}type: [Bachelorthesis],}
\NormalTok{  lang: "de",}
\NormalTok{  authors: (}
\NormalTok{    (}
\NormalTok{      name: "Firstname Lastname", }
\NormalTok{      email: "email@email.com",}
\NormalTok{      matrnr: "12345678",}
\NormalTok{      date: datetime.today().display("[day] [month repr:long] [year]"),}
\NormalTok{    ),}
\NormalTok{  ),}
\NormalTok{  abstract: [The Abstract of the Thesis],}
\NormalTok{  bibliography: bibliography("bibliography.bib"), }
\NormalTok{  appendix: [The Appendix of the Thesis], }
\NormalTok{    )}
\end{Highlighting}
\end{Shaded}

\subsection{Options}\label{options}

All the available options that are available for the template are listed
below.

\begin{longtable}[]{@{}lll@{}}
\toprule\noalign{}
Parameter & Type & Description \\
\midrule\noalign{}
\endhead
\bottomrule\noalign{}
\endlastfoot
\texttt{\ title\ } & \texttt{\ content\ } & Title of the thesis. \\
\texttt{\ thesis-type\ } & \texttt{\ content\ } & Type of thesis (e.g.,
Bachelor’s thesis, Master’s thesis). \\
\texttt{\ authors\ } & \texttt{\ content\ } ; \texttt{\ string\ } ;
\texttt{\ array\ } & Name of the author(s) as text or array. \\
\texttt{\ abstract\ } & \texttt{\ content\ } & Abstract of the
thesis. \\
\texttt{\ papersize\ } & \texttt{\ string\ } & Paper size (e.g., A4,
Letter). \\
\texttt{\ bibliography\ } & \texttt{\ bibliography\ } & Bibliography
section. \\
\texttt{\ lang\ } & \texttt{\ string\ } & Language of the thesis (e.g.,
“en� for English, “de� for German). \\
\texttt{\ appendix\ } & \texttt{\ content\ } & Appendix of the
thesis. \\
\texttt{\ toc\ } & \texttt{\ bool\ } & Show table of contents (
\texttt{\ true\ } or \texttt{\ false\ } ). \\
\texttt{\ font-size\ } & \texttt{\ length\ } & Font size for the main
text. \\
\texttt{\ main-font\ } & \texttt{\ string\ } ; \texttt{\ array\ } & Main
font as a name or an array of font names. \\
\texttt{\ title-font\ } & \texttt{\ string\ } ; \texttt{\ array\ } &
Font for the title as a name or an array of font names. \\
\texttt{\ raw-font\ } & \texttt{\ string\ } ; \texttt{\ array\ } & Font
for specific text as a name or an array of font names. \\
\texttt{\ title-page\ } & \texttt{\ content\ } & Content of the title
page. \\
\texttt{\ paper-margins\ } & \texttt{\ auto\ } ; \texttt{\ relative\ } ;
\texttt{\ dictionary\ } & Margins of the document. Can be set as
automatic, relative, or defined by a dictionary. \\
\texttt{\ title-hyphenation\ } & \texttt{\ auto\ } ; \texttt{\ bool\ } &
Title hyphenation, either automatic ( \texttt{\ auto\ } ) or manual (
\texttt{\ true\ } or \texttt{\ false\ } ). \\
\end{longtable}

\subsection{Usage}\label{usage}

These instructions will get you a copy of the project up and running on
the typst web app.

\begin{Shaded}
\begin{Highlighting}[]
\ExtensionTok{typst}\NormalTok{ init @preview/vienna{-}tech:0.1.2}
\end{Highlighting}
\end{Shaded}

\subsubsection{Template overview}\label{template-overview}

After setting up the template, you will have the following files:

\begin{itemize}
\tightlist
\item
  \texttt{\ main.typ\ } : the file which is used to compile the document
\item
  \texttt{\ abstract.typ\ } : a file where you can put your abstract
  text
\item
  \texttt{\ appendix.typ\ } : a file where you can put your appendix
  text
\item
  \texttt{\ sections.typ\ } : a file which can include all your contents
\item
  \texttt{\ refs.bib\ } : references
\end{itemize}

\subsection{Contribute to the
template}\label{contribute-to-the-template}

Feel free to contribute to the template by opening a pull request. If
you have any questions, feel free to open an issue.

\href{/app?template=vienna-tech&version=0.1.2}{Create project in app}

\subsubsection{How to use}\label{how-to-use}

Click the button above to create a new project using this template in
the Typst app.

You can also use the Typst CLI to start a new project on your computer
using this command:

\begin{verbatim}
typst init @preview/vienna-tech:0.1.2
\end{verbatim}

\includesvg[width=0.16667in,height=0.16667in]{/assets/icons/16-copy.svg}

\subsubsection{About}\label{about}

\begin{description}
\tightlist
\item[Author :]
Niko Pikall
\item[License:]
Unlicense
\item[Current version:]
0.1.2
\item[Last updated:]
November 6, 2024
\item[First released:]
August 23, 2024
\item[Archive size:]
66.3 kB
\href{https://packages.typst.org/preview/vienna-tech-0.1.2.tar.gz}{\pandocbounded{\includesvg[keepaspectratio]{/assets/icons/16-download.svg}}}
\item[Repository:]
\href{https://github.com/npikall/vienna-tech.git}{GitHub}
\item[Categor y :]
\begin{itemize}
\tightlist
\item[]
\item
  \pandocbounded{\includesvg[keepaspectratio]{/assets/icons/16-mortarboard.svg}}
  \href{https://typst.app/universe/search/?category=thesis}{Thesis}
\end{itemize}
\end{description}

\subsubsection{Where to report issues?}\label{where-to-report-issues}

This template is a project of Niko Pikall . Report issues on
\href{https://github.com/npikall/vienna-tech.git}{their repository} .
You can also try to ask for help with this template on the
\href{https://forum.typst.app}{Forum} .

Please report this template to the Typst team using the
\href{https://typst.app/contact}{contact form} if you believe it is a
safety hazard or infringes upon your rights.

\phantomsection\label{versions}
\subsubsection{Version history}\label{version-history}

\begin{longtable}[]{@{}ll@{}}
\toprule\noalign{}
Version & Release Date \\
\midrule\noalign{}
\endhead
\bottomrule\noalign{}
\endlastfoot
0.1.2 & November 6, 2024 \\
\href{https://typst.app/universe/package/vienna-tech/0.1.1/}{0.1.1} &
August 27, 2024 \\
\href{https://typst.app/universe/package/vienna-tech/0.1.0/}{0.1.0} &
August 23, 2024 \\
\end{longtable}

Typst GmbH did not create this template and cannot guarantee correct
functionality of this template or compatibility with any version of the
Typst compiler or app.
