\title{typst.app/universe/package/glossy}

\phantomsection\label{banner}
\section{glossy}\label{glossy}

{ 0.2.0 }

A very simple glossary system with easily customizable output.

\phantomsection\label{readme}
This package provides utilities to manage and render glossaries within
documents. It includes functions to define and use glossary terms, track
their usage, and generate a glossary list with references to where terms
are used in the document.

\pandocbounded{\includegraphics[keepaspectratio]{https://github.com/typst/packages/raw/main/packages/preview/glossy/0.2.0/thumbnail.png}}

\subsection{Motivation}\label{motivation}

Glossy is heavily inspired by
\href{https://typst.app/universe/package/glossarium}{glossarium} , with
a few key different goals:

\begin{enumerate}
\tightlist
\item
  Provide a simple interface which allows for complete control over
  glossary display. To do this, \texttt{\ glossy\ } ’s
  \texttt{\ \#glossary()\ } function accepts a theme parameter. The goal
  here is to separate presentation and logic.
\item
  Simplify the user interface as much as possible. Glossy has exactly
  two exports, \texttt{\ init-glossary\ } and \texttt{\ glossary\ } .
\item
  Double-down on \texttt{\ glossy\ } ’s excellent \texttt{\ @term\ }
  reference approach, completely eliminating the need to make any calls
  to \texttt{\ gls()\ } and friends.
\item
  Mimic established patterns and best practices. For example,
  \texttt{\ glossy\ } ’s \texttt{\ \#glossary()\ } function is
  intentionally similar (in naming and parameters) to \texttt{\ typst\ }
  ’s built-in \texttt{\ \#bibliography\ } , to the degree possible.
\item
  Simplify the implementation. The \texttt{\ glossy\ } code is
  significantly shorter and easier to understand.
\end{enumerate}

\subsection{Features}\label{features}

\begin{itemize}
\tightlist
\item
  Define glossary terms with short and long forms, descriptions, and
  grouping
\item
  Automatically tracks term usage in the document through
  \texttt{\ @labels\ }
\item
  Supports modifiers to adjust how terms are displayed (capitalize,
  pluralize, etc.)
\item
  Generates a formatted glossary section with backlinks to term
  occurrences
\item
  Customizable themes for rendering glossary sections, groups, and
  entries
\item
  Automatic pluralization of terms with custom override options
\item
  Page number references back to term usage locations
\end{itemize}

\subsection{Usage}\label{usage}

\subsubsection{Import the package}\label{import-the-package}

\begin{Shaded}
\begin{Highlighting}[]
\NormalTok{\#import "@preview/glossy:0.2.0": *}
\end{Highlighting}
\end{Shaded}

\subsubsection{Defining Glossary Terms}\label{defining-glossary-terms}

Use the \texttt{\ init-glossary\ } function to initialize glossary
entries:

\begin{Shaded}
\begin{Highlighting}[]
\NormalTok{\#let myGlossary = (}
\NormalTok{    html: (}
\NormalTok{      short: "HTML",}
\NormalTok{      long: "Hypertext Markup Language",}
\NormalTok{      description: "A standard language for creating web pages",}
\NormalTok{      group: "Web"}
\NormalTok{    ),}
\NormalTok{    css: (}
\NormalTok{      short: "CSS",}
\NormalTok{      long: "Cascading Style Sheets",}
\NormalTok{      description: "A stylesheet language used for describing the presentation of a document",}
\NormalTok{      group: "Web"}
\NormalTok{    ),}
\NormalTok{    tps: (}
\NormalTok{      short: "TPS",}
\NormalTok{      long: "test procedure specification",}
\NormalTok{      description: "A formal document describing test steps and expected results",}
\NormalTok{      // Optional: Override automatic pluralization}
\NormalTok{      plural: "TPSes",}
\NormalTok{      longplural: "test procedure specifications"}
\NormalTok{    )}
\NormalTok{)}

\NormalTok{\#show: init{-}glossary.with(myGlossary)}
\end{Highlighting}
\end{Shaded}

Each glossary entry supports the following fields:

\begin{itemize}
\tightlist
\item
  \texttt{\ short\ } (required): Short form of the term
\item
  \texttt{\ long\ } (optional): Long form of the term
\item
  \texttt{\ description\ } (optional): Term description (often a
  definition)
\item
  \texttt{\ group\ } (optional): Category grouping
\item
  \texttt{\ plural\ } (optional): Override automatic pluralization of
  short form
\item
  \texttt{\ longplural\ } (optional): Override automatic pluralization
  of long form
\end{itemize}

You can also load glossary entries from a data file using \#yaml(),
\#json(), or similar.

For example, the above glossary could be in this YAML file:

\begin{Shaded}
\begin{Highlighting}[]
\FunctionTok{html}\KeywordTok{:}
\AttributeTok{  }\FunctionTok{short}\KeywordTok{:}\AttributeTok{ HTML}
\AttributeTok{  }\FunctionTok{long}\KeywordTok{:}\AttributeTok{ Hypertext Markup Language}
\AttributeTok{  }\FunctionTok{description}\KeywordTok{:}\AttributeTok{ A standard language for creating web pages}
\AttributeTok{  }\FunctionTok{group}\KeywordTok{:}\AttributeTok{ Web}

\FunctionTok{css}\KeywordTok{:}
\AttributeTok{  }\FunctionTok{short}\KeywordTok{:}\AttributeTok{ CSS}
\AttributeTok{  }\FunctionTok{long}\KeywordTok{:}\AttributeTok{ Cascading Style Sheets}
\AttributeTok{  }\FunctionTok{description}\KeywordTok{:}\AttributeTok{ A stylesheet language used for describing the presentation of a document}
\AttributeTok{  }\FunctionTok{group}\KeywordTok{:}\AttributeTok{ Web}

\FunctionTok{tps}\KeywordTok{:}
\AttributeTok{  }\FunctionTok{short}\KeywordTok{:}\AttributeTok{ TPS}
\AttributeTok{  }\FunctionTok{long}\KeywordTok{:}\AttributeTok{ test procedure specification}
\AttributeTok{  }\FunctionTok{description}\KeywordTok{:}\AttributeTok{ A formal document describing test steps and expected results}
\AttributeTok{  }\FunctionTok{plural}\KeywordTok{:}\AttributeTok{ TPSes}
\AttributeTok{  }\FunctionTok{longplural}\KeywordTok{:}\AttributeTok{ test procedure specifications}
\end{Highlighting}
\end{Shaded}

And then loaded during initialization as follows:

\begin{Shaded}
\begin{Highlighting}[]
\NormalTok{\#show: init{-}glossary.with(yaml("glossary.yaml"))}
\end{Highlighting}
\end{Shaded}

\subsubsection{Using Glossary Terms}\label{using-glossary-terms}

Reference glossary terms using Typst’s \texttt{\ @reference\ } syntax:

\begin{Shaded}
\begin{Highlighting}[]
\NormalTok{In modern web development, languages like @html and @css are essential.}
\NormalTok{The @tps:pl need to be submitted by Friday.}
\end{Highlighting}
\end{Shaded}

Available modifiers:

\begin{itemize}
\tightlist
\item
  \textbf{cap} : Capitalizes the term
\item
  \textbf{pl} : Uses the plural form
\item
  \textbf{both} : Shows “Long Form (Short Form)�
\item
  \textbf{short} : Shows only short form
\item
  \textbf{long} : Shows only long form
\item
  \textbf{def} or \textbf{desc} : Shows the description
\end{itemize}

Modifiers can be combined with colons:

\begin{longtable}[]{@{}ll@{}}
\toprule\noalign{}
\textbf{Input} & \textbf{Output} \\
\midrule\noalign{}
\endhead
\bottomrule\noalign{}
\endlastfoot
\texttt{\ @tps\ } (first use) & “test procedure specification
(TPS)� \\
\texttt{\ @tps\ } (subsequent) & “TPS� \\
\texttt{\ @tps:short\ } & “TPS� \\
\texttt{\ @tps:long\ } & “test procedure specification� \\
\texttt{\ @tps:both\ } & “test procedure specification (TPS)� \\
\texttt{\ @tps:long:cap\ } & “Test procedure specification� \\
\texttt{\ @tps:long:pl\ } & “test procedure specifications� \\
\texttt{\ @tps:short:pl\ } & “TPSes� \\
\texttt{\ @tps:both:pl:cap\ } & “Technical procedure specifications
(TPSes)� \\
\texttt{\ @tps:def\ } & “A formal document describing test steps and
expected results� \\
\end{longtable}

\subsubsection{Generating the Glossary}\label{generating-the-glossary}

Display the glossary using the \texttt{\ glossary()\ } function:

\begin{Shaded}
\begin{Highlighting}[]
\NormalTok{\#glossary(}
\NormalTok{  title: "Web Development Glossary",}
\NormalTok{  theme: my{-}theme,}
\NormalTok{  groups: ("Web")  // Optional: Filter to specific groups}
\NormalTok{)}
\end{Highlighting}
\end{Shaded}

\subsubsection{Customizing Term Display}\label{customizing-term-display}

Control how terms appear in the document by providing a custom
\texttt{\ show-term\ } function:

\begin{Shaded}
\begin{Highlighting}[]
\NormalTok{\#let emph{-}term(term{-}body) = \{ emph(term{-}body) \}}

\NormalTok{\#show: init{-}glossary.with(}
\NormalTok{  myGlossary,}
\NormalTok{  show{-}term: emph{-}term}
\NormalTok{)}
\end{Highlighting}
\end{Shaded}

\subsubsection{Glossary Themes}\label{glossary-themes}

\paragraph{Included Themes}\label{included-themes}

Glossy comes with several built-in themes that can be used directly or
serve as examples for custom themes:

\subparagraph{theme-twocol}\label{theme-twocol}

A professional two-column layout ideal for technical documentation.
Features:

\begin{itemize}
\tightlist
\item
  Two-column layout for efficient space usage
\item
  Dotted leaders to page numbers
\item
  Clear hierarchy with optional group headings
\item
  Compact but readable formatting
\item
  Terms in regular weight with long forms and descriptions inline
\end{itemize}

\begin{Shaded}
\begin{Highlighting}[]
\NormalTok{\#glossary(theme: theme{-}twocol)}
\end{Highlighting}
\end{Shaded}

\subparagraph{theme-basic}\label{theme-basic}

A traditional single-column layout similar to book glossaries. Features:

\begin{itemize}
\tightlist
\item
  Bold terms with indented content
\item
  Clear separation between entries
\item
  Hanging indentation for wrapped lines
\item
  Parenthetical long forms
\item
  Page numbers with “pp.� prefix
\item
  Simple, clean design
\end{itemize}

\begin{Shaded}
\begin{Highlighting}[]
\NormalTok{\#glossary(theme: theme{-}basic)}
\end{Highlighting}
\end{Shaded}

\subparagraph{theme-compact}\label{theme-compact}

A space-efficient layout perfect for technical documents or appendices.
Features:

\begin{itemize}
\tightlist
\item
  Reduced vertical spacing
\item
  Smaller font sizes for secondary information
\item
  Color-coded term components
\item
  Grid-based alignment
\item
  Minimal decorative elements
\item
  Gray text for supplementary information
\item
  Bullet separators between components
\end{itemize}

\begin{Shaded}
\begin{Highlighting}[]
\NormalTok{\#glossary(theme: theme{-}compact)}
\end{Highlighting}
\end{Shaded}

\paragraph{Custom Themes}\label{custom-themes}

Customize glossary appearance by defining a theme with three functions:

\begin{Shaded}
\begin{Highlighting}[]
\NormalTok{\#let my{-}theme = (}
\NormalTok{  // Main glossary section}
\NormalTok{  section: (title, body) =\textgreater{} \{}
\NormalTok{    heading(level: 1, title)}
\NormalTok{    body}
\NormalTok{  \},}

\NormalTok{  // Group of related terms}
\NormalTok{  group: (name, index, total, body) =\textgreater{} \{}
\NormalTok{    // index = group index, total = total groups}
\NormalTok{    if name != "" and total \textgreater{} 1 \{}
\NormalTok{      heading(level: 2, name)}
\NormalTok{    \}}
\NormalTok{    body}
\NormalTok{  \},}

\NormalTok{  // Individual glossary entry}
\NormalTok{  entry: (entry, index, total) =\textgreater{} \{}
\NormalTok{    // index = entry index, total = total entries in group}
\NormalTok{    let output = [\#entry.short]}
\NormalTok{    if entry.long != none \{}
\NormalTok{      output = [\#output {-}{-} \#entry.long]}
\NormalTok{    \}}
\NormalTok{    if entry.description != none \{}
\NormalTok{      output = [\#output: \#entry.description]}
\NormalTok{    \}}
\NormalTok{    block(}
\NormalTok{      grid(}
\NormalTok{        columns: (auto, 1fr, auto),}
\NormalTok{        output,}
\NormalTok{        repeat([\#h(0.25em) . \#h(0.25em)]),}
\NormalTok{        entry.pages,}
\NormalTok{      )}
\NormalTok{    )}
\NormalTok{  \}}
\NormalTok{)}
\end{Highlighting}
\end{Shaded}

Entry fields available to themes:

\begin{itemize}
\tightlist
\item
  \texttt{\ short\ } : Short form (always present)
\item
  \texttt{\ long\ } : Long form (can be \texttt{\ none\ } )
\item
  \texttt{\ description\ } : Term description (can be \texttt{\ none\ }
  )
\item
  \texttt{\ label\ } : Term’s dictionary label
\item
  \texttt{\ pages\ } : Linked page numbers where term appears
\end{itemize}

\subsection{License}\label{license}

This project is licensed under the MIT License.

\subsubsection{How to add}\label{how-to-add}

Copy this into your project and use the import as \texttt{\ glossy\ }

\begin{verbatim}
#import "@preview/glossy:0.2.0"
\end{verbatim}

\includesvg[width=0.16667in,height=0.16667in]{/assets/icons/16-copy.svg}

Check the docs for
\href{https://typst.app/docs/reference/scripting/\#packages}{more
information on how to import packages} .

\subsubsection{About}\label{about}

\begin{description}
\tightlist
\item[Author :]
\href{mailto:steve@waits.net}{Stephen Waits}
\item[License:]
MIT
\item[Current version:]
0.2.0
\item[Last updated:]
November 26, 2024
\item[First released:]
October 23, 2024
\item[Archive size:]
10.2 kB
\href{https://packages.typst.org/preview/glossy-0.2.0.tar.gz}{\pandocbounded{\includesvg[keepaspectratio]{/assets/icons/16-download.svg}}}
\item[Repository:]
\href{https://github.com/swaits/typst-collection}{GitHub}
\item[Categor y :]
\begin{itemize}
\tightlist
\item[]
\item
  \pandocbounded{\includesvg[keepaspectratio]{/assets/icons/16-list-unordered.svg}}
  \href{https://typst.app/universe/search/?category=model}{Model}
\end{itemize}
\end{description}

\subsubsection{Where to report issues?}\label{where-to-report-issues}

This package is a project of Stephen Waits . Report issues on
\href{https://github.com/swaits/typst-collection}{their repository} .
You can also try to ask for help with this package on the
\href{https://forum.typst.app}{Forum} .

Please report this package to the Typst team using the
\href{https://typst.app/contact}{contact form} if you believe it is a
safety hazard or infringes upon your rights.

\phantomsection\label{versions}
\subsubsection{Version history}\label{version-history}

\begin{longtable}[]{@{}ll@{}}
\toprule\noalign{}
Version & Release Date \\
\midrule\noalign{}
\endhead
\bottomrule\noalign{}
\endlastfoot
0.2.0 & November 26, 2024 \\
\href{https://typst.app/universe/package/glossy/0.1.2/}{0.1.2} & October
31, 2024 \\
\href{https://typst.app/universe/package/glossy/0.1.1/}{0.1.1} & October
24, 2024 \\
\href{https://typst.app/universe/package/glossy/0.1.0/}{0.1.0} & October
23, 2024 \\
\end{longtable}

Typst GmbH did not create this package and cannot guarantee correct
functionality of this package or compatibility with any version of the
Typst compiler or app.
