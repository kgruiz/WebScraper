\title{typst.app/universe/package/tinyset}

\phantomsection\label{banner}
\section{tinyset}\label{tinyset}

{ 0.1.0 }

Simple, consistent, and appealing math homework template

\phantomsection\label{readme}
Extremely simple \href{https://github.com/typst/typst}{typst} package
for writing math problem sets quickly and consistently. Under the hood
it is just typst fundamentals that could be defined by hand, however the
aim of this package is to make you faster and the code easier to read.

\subsection{Usage}\label{usage}

Import styles and create a new header. I like to copy this from the top
of the previous week’s homework (don’t forget to increment the
number!).

Example using proof, question, and part environments. Indentation in
source code is largely ignored and left to personal preference. By
default questions are numbered and each part is lettered, you can change
this based on course / instructor preference.

\begin{Shaded}
\begin{Highlighting}[]
\NormalTok{\#import "@preview/tinyset:0.1.0": *}
\NormalTok{\#header(number: 7, name: "Sylvan Franklin", class: "Math 3551 {-} Fall 2024")}

\NormalTok{+ \#qs[}
\NormalTok{Let $G\_1$ and $G\_2$ be groups, $phi : G\_1 {-}\textgreater{} G\_2$ be a homomorphism, and $H$ be}
\NormalTok{any subgroup of $G\_2$. Define}

\NormalTok{$ phi\^{}({-}1)(H) = \{g in G\_1 : phi(g) in H\}. $}

\NormalTok{+ \#pt[ }
\NormalTok{    Prove that $phi\^{}({-}1)(H)$ is a subgroup of $G\_1$.}
\NormalTok{    \#prf[ Non empty: Since $H$ is a subgroup it contains the indentity, and}
\NormalTok{    since $phi$ is a homomorphism and ... ]}
\NormalTok{]}

\NormalTok{+ \#pt[ }
\NormalTok{    What about a question that you don\textquotesingle{}t need a proof for?}
\NormalTok{    \#ans[Use the ans environment]}
\NormalTok{]}

\NormalTok{]}
\end{Highlighting}
\end{Shaded}

\subsection{Custom shorthand}\label{custom-shorthand}

Sometimes when thinking about math I find it easier to phonetically
write out these symbols instead of using the built in typst classes. For
certain others I find the original symbols annoying to type quickly.

\begin{longtable}[]{@{}ll@{}}
\toprule\noalign{}
shorthand & expansion \\
\midrule\noalign{}
\endhead
\bottomrule\noalign{}
\endlastfoot
implies / impl & ==\textgreater{} \\
iff & \textless==\textgreater{} \\
wlog & without loss of generality \\
inv() & ()\^{}(-1) \\
\end{longtable}

\subsubsection{How to add}\label{how-to-add}

Copy this into your project and use the import as \texttt{\ tinyset\ }

\begin{verbatim}
#import "@preview/tinyset:0.1.0"
\end{verbatim}

\includesvg[width=0.16667in,height=0.16667in]{/assets/icons/16-copy.svg}

Check the docs for
\href{https://typst.app/docs/reference/scripting/\#packages}{more
information on how to import packages} .

\subsubsection{About}\label{about}

\begin{description}
\tightlist
\item[Author :]
Sylvan Franklin
\item[License:]
MIT
\item[Current version:]
0.1.0
\item[Last updated:]
November 6, 2024
\item[First released:]
November 6, 2024
\item[Archive size:]
2.28 kB
\href{https://packages.typst.org/preview/tinyset-0.1.0.tar.gz}{\pandocbounded{\includesvg[keepaspectratio]{/assets/icons/16-download.svg}}}
\item[Repository:]
\href{https://github.com/sylvanfranklin/tinyset}{GitHub}
\item[Discipline :]
\begin{itemize}
\tightlist
\item[]
\item
  \href{https://typst.app/universe/search/?discipline=mathematics}{Mathematics}
\end{itemize}
\item[Categor y :]
\begin{itemize}
\tightlist
\item[]
\item
  \pandocbounded{\includesvg[keepaspectratio]{/assets/icons/16-layout.svg}}
  \href{https://typst.app/universe/search/?category=layout}{Layout}
\end{itemize}
\end{description}

\subsubsection{Where to report issues?}\label{where-to-report-issues}

This package is a project of Sylvan Franklin . Report issues on
\href{https://github.com/sylvanfranklin/tinyset}{their repository} . You
can also try to ask for help with this package on the
\href{https://forum.typst.app}{Forum} .

Please report this package to the Typst team using the
\href{https://typst.app/contact}{contact form} if you believe it is a
safety hazard or infringes upon your rights.

\phantomsection\label{versions}
\subsubsection{Version history}\label{version-history}

\begin{longtable}[]{@{}ll@{}}
\toprule\noalign{}
Version & Release Date \\
\midrule\noalign{}
\endhead
\bottomrule\noalign{}
\endlastfoot
0.1.0 & November 6, 2024 \\
\end{longtable}

Typst GmbH did not create this package and cannot guarantee correct
functionality of this package or compatibility with any version of the
Typst compiler or app.
