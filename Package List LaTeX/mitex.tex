\title{typst.app/universe/package/mitex}

\phantomsection\label{banner}
\section{mitex}\label{mitex}

{ 0.2.4 }

LaTeX support for Typst, powered by Rust and WASM.

\phantomsection\label{readme}
\textbf{\href{https://www.latex-project.org/}{LaTeX} support for
\href{https://typst.app/}{Typst} , powered by
\href{https://www.rust-lang.org/}{Rust} and
\href{https://webassembly.org/}{WASM} .}

\href{https://github.com/mitex-rs/mitex}{MiTeX} processes LaTeX code
into an abstract syntax tree (AST). Then it transforms the AST into
Typst code and evaluates code into Typst content by \texttt{\ eval\ }
function.

MiTeX has been proved to be practical on a large project. It has already
correctly converted 32.5k equations from
\href{https://github.com/OI-wiki/OI-wiki}{OI Wiki} . Compared to
\href{https://github.com/jgm/texmath}{texmath} , MiTeX has a better
display effect and performance in that wiki project. It is also more
easy to use, since importing MiTeX to Typst is just one line of code,
while texmath is an external program.

In addition, MiTeX is not only \textbf{SMALL} but also \textbf{FAST} !
MiTeX has a size of just about 185 KB, comparing that texmath has a size
of 17 MB. A not strict but intuitive comparison is shown below. To
convert 32.5k equations from OI Wiki, texmath takes about 109s, while
MiTeX WASM takes only 2.28s and MiTeX x86 takes merely 0.085s.

Thanks to \href{https://github.com/Myriad-Dreamin}{@Myriad-Dreamin} , he
completed the most complex development work: developing the parser for
generating AST.

\subsection{MiTeX as a Typst Package}\label{mitex-as-a-typst-package}

\begin{itemize}
\tightlist
\item
  Use \texttt{\ mitex-convert\ } to convert LaTeX code into Typst code
  in string.
\item
  Use \texttt{\ mi\ } to render an inline LaTeX equation in Typst.
\item
  Use \texttt{\ mitex(numbering:\ none,\ supplement:\ auto,\ ..)\ } or
  \texttt{\ mimath\ } to render a block LaTeX equation in Typst.
\item
  Use \texttt{\ mitext\ } to render a LaTeX text in Typst.
\end{itemize}

PS: \texttt{\ \#set\ math.equation(numbering:\ "(1)")\ } is also valid
for MiTeX.

Following is
\href{https://github.com/mitex-rs/mitex/blob/main/packages/mitex/examples/example.typ}{a
simple example} of using MiTeX in Typst:

\begin{Shaded}
\begin{Highlighting}[]
\NormalTok{\#import "@preview/mitex:0.2.4": *}

\NormalTok{\#assert.eq(mitex{-}convert("\textbackslash{}alpha x"), "alpha  x ")}

\NormalTok{Write inline equations like \#mi("x") or \#mi[y].}

\NormalTok{Also block equations (this case is from \#text(blue.lighten(20\%), link("https://katex.org/")[katex.org])):}

\NormalTok{\#mitex(\textasciigrave{}}
\NormalTok{  \textbackslash{}newcommand\{\textbackslash{}f\}[2]\{\#1f(\#2)\}}
\NormalTok{  \textbackslash{}f\textbackslash{}relax\{x\} = \textbackslash{}int\_\{{-}\textbackslash{}infty\}\^{}\textbackslash{}infty}
\NormalTok{    \textbackslash{}f\textbackslash{}hat\textbackslash{}xi\textbackslash{},e\^{}\{2 \textbackslash{}pi i \textbackslash{}xi x\}}
\NormalTok{    \textbackslash{},d\textbackslash{}xi}
\NormalTok{\textasciigrave{})}

\NormalTok{We also support text mode (in development):}

\NormalTok{\#mitext(\textasciigrave{}}
\NormalTok{  \textbackslash{}iftypst}
\NormalTok{    \#set math.equation(numbering: "(1)", supplement: "equation")}
\NormalTok{  \textbackslash{}fi}

\NormalTok{  \textbackslash{}section\{Title\}}

\NormalTok{  A \textbackslash{}textbf\{strong\} text, a \textbackslash{}emph\{emph\} text and inline equation $x + y$.}

\NormalTok{  Also block \textbackslash{}eqref\{eq:pythagoras\}.}

\NormalTok{  \textbackslash{}begin\{equation\}}
\NormalTok{    a\^{}2 + b\^{}2 = c\^{}2 \textbackslash{}label\{eq:pythagoras\}}
\NormalTok{  \textbackslash{}end\{equation\}}
\NormalTok{\textasciigrave{})}
\end{Highlighting}
\end{Shaded}

\pandocbounded{\includegraphics[keepaspectratio]{https://github.com/typst/packages/raw/main/packages/preview/mitex/0.2.4/examples/example.png}}

\subsection{MiTeX as a CLI Tool}\label{mitex-as-a-cli-tool}

\subsubsection{Installation}\label{installation}

Install latest nightly version by
\texttt{\ cargo\ install\ -\/-git\ https://github.com/mitex-rs/mitex\ mitex-cli\ }
.

\subsubsection{Usage}\label{usage}

\begin{Shaded}
\begin{Highlighting}[]
\ExtensionTok{mitex}\NormalTok{ compile main.tex}
\CommentTok{\# or (same as above)}
\ExtensionTok{mitex}\NormalTok{ compile main.tex mitex.typ}
\end{Highlighting}
\end{Shaded}

\subsection{MiTeX as a Web App}\label{mitex-as-a-web-app}

\subsubsection{MiTeX Online Math
Converter}\label{mitex-online-math-converter}

We can convert LaTeX equations to Typst equations in web by wasm.
\url{https://mitex-rs.github.io/mitex/}

\subsubsection{Underleaf}\label{underleaf}

We made a proof of concept online tex editor to show our conversion
speed in text mode. The PoC loads files from a git repository and then
runs typst compile in browser. As you see, each keystroking get response
in preview quickly.

\url{https://mitex-rs.github.io/mitex/tools/underleaf.html}

\url{https://github.com/mitex-rs/mitex/assets/34951714/0ce77a2c-0a7d-445f-b26d-e139f3038f83}

\subsection{Implemented Features}\label{implemented-features}

\begin{itemize}
\tightlist
\item
  {[}x{]} User-defined TeX (macro) commands, such as
  \texttt{\ \textbackslash{}newcommand\{\textbackslash{}mysym\}\{\textbackslash{}alpha\}\ }
  .
\item
  {[}x{]} LaTeX equations support.

  \begin{itemize}
  \tightlist
  \item
    {[}x{]} Coloring commands (
    \texttt{\ \textbackslash{}color\{red\}\ text\ } ,
    \texttt{\ \textbackslash{}textcolor\{red\}\{text\}\ } ).
  \item
    {[}x{]} Support for various environments, such as aligned, matrix,
    cases.
  \end{itemize}
\item
  {[}x{]} Basic text mode support, you can use it to write LaTeX drafts.

  \begin{itemize}
  \tightlist
  \item
    {[}x{]} \texttt{\ \textbackslash{}section\ } ,
    \texttt{\ \textbackslash{}textbf\ } ,
    \texttt{\ \textbackslash{}emph\ } .
  \item
    {[}x{]} Inline and block math equations.
  \item
    {[}x{]} \texttt{\ \textbackslash{}ref\ } ,
    \texttt{\ \textbackslash{}eqref\ } and
    \texttt{\ \textbackslash{}label\ } .
  \item
    {[}x{]} \texttt{\ itemize\ } and \texttt{\ enumerate\ }
    environments.
  \end{itemize}
\end{itemize}

\subsection{Features to Implement}\label{features-to-implement}

\begin{itemize}
\tightlist
\item
  {[} {]} Pass command specification to MiTeX plugin dynamically. With
  that you can define a typst function
  \texttt{\ let\ myop(it)\ =\ op(upright(it))\ } and then use it by
  \texttt{\ \textbackslash{}myop\{it\}\ } .
\item
  {[} {]} Package support, which means that you can change set of
  commands by telling MiTeX to use a list of packages.
\item
  {[} {]} Better text mode support, such as figure, algorithm and
  description environments.
\end{itemize}

To achieve the latter two goals, we need a well-structured architecture
for the text mode, along with intricate work. Currently, we don’t have
very clear ideas yet. If you are willing to contribute by discussing in
the issues or even submitting pull requests, your contribution is highly
welcome.

\subsection{Differences between MiTeX and other
solutions}\label{differences-between-mitex-and-other-solutions}

MiTeX has different objectives compared to
\href{https://github.com/jgm/texmath}{texmath} (a.k.a.
\href{https://pandoc.org/}{pandoc} ):

\begin{itemize}
\tightlist
\item
  MiTeX focuses on rendering LaTeX content correctly within Typst,
  leveraging the powerful programming capabilities of WASM and typst to
  achieve results that are essentially consistent with LaTeX display.
\item
  texmath aims to be general-purpose converters and generate strings
  that are more human-readable.
\end{itemize}

For example, MiTeX transforms
\texttt{\ \textbackslash{}frac\{1\}\{2\}\_3\ } into
\texttt{\ frac(1,\ 2)\_3\ } , while texmath converts it into
\texttt{\ 1\ /\ 2\_3\ } . The latter’s display is not entirely
correct, whereas the former ensures consistency in display.

Another example is that MiTeX transforms
\texttt{\ (\textbackslash{}frac\{1\}\{2\})\ } into
\texttt{\ \textbackslash{}(frac(1,\ 2)\textbackslash{})\ } instead of
\texttt{\ (frac(1,\ 2))\ } , avoiding the use of automatic Left/Right to
achieve consistency with LaTeX rendering.

\textbf{Certainly, the greatest advantage is that you can directly write
LaTeX content in Typst without the need for manual conversion!}

\subsection{Submitting Issues}\label{submitting-issues}

If you find missing commands or bugs of MiTeX, please feel free to
submit an issue \href{https://github.com/mitex-rs/mitex/issues}{here} .

\subsection{Contributing to MiTeX}\label{contributing-to-mitex}

Currently, MiTeX maintains following three parts of code:

\begin{itemize}
\tightlist
\item
  A TeX parser library written in \textbf{Rust} , see
  \href{https://github.com/mitex-rs/mitex/tree/main/crates/mitex-lexer}{mitex-lexer}
  and
  \href{https://github.com/mitex-rs/mitex/tree/main/crates/mitex-parser}{mitex-parser}
  .
\item
  A TeX to Typst converter library written in \textbf{Rust} , see
  \href{https://github.com/mitex-rs/mitex/tree/main/crates/mitex}{mitex}
  .
\item
  A list of TeX packages and comamnds written in \textbf{Typst} , which
  then used by the typst package, see
  \href{https://github.com/mitex-rs/mitex/tree/main/packages/mitex/specs}{MiTeX
  Command Specification} .
\end{itemize}

For a translation process, for example, we have:

\begin{verbatim}
\frac{1}{2}

===[parser]===> AST ===[converter]===>

#eval("$frac(1, 2)$", scope: (frac: (num, den) => $(num)/(den)$))
\end{verbatim}

You can use the \texttt{\ \#mitex-convert()\ } function to get the Typst
Code generated from LaTeX Code.

\subsubsection{Add missing TeX commands}\label{add-missing-tex-commands}

Even if you don’t know Rust at all, you can still add missing TeX
commands to MiTeX by modifying
\href{https://github.com/mitex-rs/mitex/tree/main/packages/mitex/specs}{specification
files} , since they are written in typst! You can open an issue to
acquire the commands you want to add, or you can edit the files and
submit a pull request.

In the future, we will provide the ability to customize TeX commands,
which will make it easier for you to use the commands you create for
yourself.

\subsubsection{Develop the parser and the
converter}\label{develop-the-parser-and-the-converter}

See
\href{https://github.com/mitex-rs/mitex/blob/main/CONTRIBUTING.md}{CONTRIBUTING.md}
.

\subsubsection{How to add}\label{how-to-add}

Copy this into your project and use the import as \texttt{\ mitex\ }

\begin{verbatim}
#import "@preview/mitex:0.2.4"
\end{verbatim}

\includesvg[width=0.16667in,height=0.16667in]{/assets/icons/16-copy.svg}

Check the docs for
\href{https://typst.app/docs/reference/scripting/\#packages}{more
information on how to import packages} .

\subsubsection{About}\label{about}

\begin{description}
\tightlist
\item[Author s :]
Myriad-Dreamin , OrangeX4 , \& Enter-tainer
\item[License:]
Apache-2.0
\item[Current version:]
0.2.4
\item[Last updated:]
May 13, 2024
\item[First released:]
December 23, 2023
\item[Archive size:]
109 kB
\href{https://packages.typst.org/preview/mitex-0.2.4.tar.gz}{\pandocbounded{\includesvg[keepaspectratio]{/assets/icons/16-download.svg}}}
\item[Repository:]
\href{https://github.com/mitex-rs/mitex}{GitHub}
\item[Categor y :]
\begin{itemize}
\tightlist
\item[]
\item
  \pandocbounded{\includesvg[keepaspectratio]{/assets/icons/16-hammer.svg}}
  \href{https://typst.app/universe/search/?category=utility}{Utility}
\end{itemize}
\end{description}

\subsubsection{Where to report issues?}\label{where-to-report-issues}

This package is a project of Myriad-Dreamin, OrangeX4, and Enter-tainer
. Report issues on \href{https://github.com/mitex-rs/mitex}{their
repository} . You can also try to ask for help with this package on the
\href{https://forum.typst.app}{Forum} .

Please report this package to the Typst team using the
\href{https://typst.app/contact}{contact form} if you believe it is a
safety hazard or infringes upon your rights.

\phantomsection\label{versions}
\subsubsection{Version history}\label{version-history}

\begin{longtable}[]{@{}ll@{}}
\toprule\noalign{}
Version & Release Date \\
\midrule\noalign{}
\endhead
\bottomrule\noalign{}
\endlastfoot
0.2.4 & May 13, 2024 \\
\href{https://typst.app/universe/package/mitex/0.2.3/}{0.2.3} & April 1,
2024 \\
\href{https://typst.app/universe/package/mitex/0.2.2/}{0.2.2} & March
10, 2024 \\
\href{https://typst.app/universe/package/mitex/0.2.1/}{0.2.1} & January
15, 2024 \\
\href{https://typst.app/universe/package/mitex/0.2.0/}{0.2.0} & January
1, 2024 \\
\href{https://typst.app/universe/package/mitex/0.1.0/}{0.1.0} & December
23, 2023 \\
\end{longtable}

Typst GmbH did not create this package and cannot guarantee correct
functionality of this package or compatibility with any version of the
Typst compiler or app.
