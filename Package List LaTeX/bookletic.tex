\title{typst.app/universe/package/bookletic}

\phantomsection\label{banner}
\section{bookletic}\label{bookletic}

{ 0.3.0 }

Create beautiful booklets with ease.

\phantomsection\label{readme}
Create beautiful booklets with ease.

The current version of this library (0.3.0) contains a single function
to take in an array of content blocks and order them into a ready to
print booklet, bulletin, etc. No need to fight with printer settings or
document converters.

\subsubsection{Example Output}\label{example-output}

Here is an example of the output generated by the \texttt{\ sig\ }
function (short for a book’s signature) with default parameters and
some sample content:

\pandocbounded{\includegraphics[keepaspectratio]{https://github.com/typst/packages/raw/main/packages/preview/bookletic/0.3.0/example/basic.png}}

Here is an example with some customization applied:

\pandocbounded{\includegraphics[keepaspectratio]{https://github.com/typst/packages/raw/main/packages/preview/bookletic/0.3.0/example/fancy.png}}

\subsection{\texorpdfstring{\texttt{\ sig\ }
Function}{ sig  Function}}\label{sig-function}

The \texttt{\ sig\ } function is used to create a signature (booklet)
layout from provided content. It takes various parameters to
automatically configure the layout.

\subsubsection{Parameters}\label{parameters}

\begin{itemize}
\tightlist
\item
  \texttt{\ page\_margin\_binding\ } : The binding margin for each page
  in the booklet (space between pages).
\item
  \texttt{\ page\_border\ } : Takes a color space value to draw a border
  around each page. If set to none no border will be drawn.
\item
  \texttt{\ draft\ } : A boolean value indicating whether to output an
  unordered draft or final layout.
\item
  \texttt{\ p-num-layout\ } : A configuration for page numbering styles,
  allowing multiple layouts that apply to specified page ranges. Each
  layout can be provided as a dictonary specifying the following
  options:

  \begin{itemize}
  \tightlist
  \item
    \texttt{\ p-num-start\ } : Starting page number for this layout
  \item
    \texttt{\ p-num-alt-start\ } : Alternative starting page number
    (e.g., for chapters)
  \item
    \texttt{\ p-num-pattern\ } : Pattern for page numbering (e.g.,
    \texttt{\ "1"\ } , \texttt{\ "i"\ } , \texttt{\ "a"\ } ,
    \texttt{\ "A"\ } )
  \item
    \texttt{\ p-num-placment\ } : Placement of page numbers (
    \texttt{\ top\ } or \texttt{\ bottom\ } )
  \item
    \texttt{\ p-num-align-horizontal\ } : Horizontal alignment of page
    numbers ( \texttt{\ left\ } , \texttt{\ center\ } , or
    \texttt{\ right\ } )
  \item
    \texttt{\ p-num-align-vertical\ } : Vertical alignment of page
    numbers ( \texttt{\ top\ } , \texttt{\ horizon\ } , or
    \texttt{\ bottom\ } )
  \item
    \texttt{\ p-num-pad-left\ } : Extra padding added to the left of the
    page number
  \item
    \texttt{\ p-num-pad-horizontal\ } : Horizontal padding for page
    numbers
  \item
    \texttt{\ p-num-size\ } : Size of page numbers
  \item
    \texttt{\ p-num-border\ } : The border color for the page numbers.
    If set to none no border will be drawn.
  \item
    \texttt{\ p-num-halign-alternate\ } : A boolean for whether to
    alternate horizontal alignment between left and right pages.
  \end{itemize}
\item
  \texttt{\ pad\_content\ } : The padding around the page content.
\item
  \texttt{\ contents\ } : The content to be laid out in the booklet.
  This should be an array of blocks.
\end{itemize}

\subsubsection{Usage}\label{usage}

To use the \texttt{\ sig\ } function, first set your desired page
settings using the native page function. Then simply call the sig
function with the desired parameters and provide the content to be laid
out in the booklet:

\begin{Shaded}
\begin{Highlighting}[]
\NormalTok{\#set page(flipped: true, paper: "us{-}letter")}
\NormalTok{\#bookletic.sig(}
\NormalTok{  contents: [}
\NormalTok{    ["Page 1 content"],}
\NormalTok{    ["Page 2 content"],}
\NormalTok{    ["Page 3 content"],}
\NormalTok{    ["Page 4 content"],}
\NormalTok{  ],}
\NormalTok{)}
\end{Highlighting}
\end{Shaded}

This will create a signature layout with the provided content, using the
default values for the other parameters.

You can customize the layout by passing different values for the various
parameters. For example:

\begin{Shaded}
\begin{Highlighting}[]
\NormalTok{\#set page(flipped: true, paper: "us{-}legal", margin: (top: 1in, bottom: 1in, left: 1in, right: 1in))}
\NormalTok{\#bookletic.sig(}
\NormalTok{  page\_margin\_binding: 0.5in,}
\NormalTok{  page\_border: none,}
\NormalTok{  draft: true,}
\NormalTok{  p{-}num{-}layout: (}
\NormalTok{    bookletic.num{-}layout(}
\NormalTok{      p{-}num{-}start: 1,}
\NormalTok{      p{-}num{-}alt{-}start: none,}
\NormalTok{      p{-}num{-}pattern: "\textasciitilde{} 1 \textasciitilde{}", }
\NormalTok{      p{-}num{-}placment: bottom,}
\NormalTok{      p{-}num{-}align{-}horizontal: right,}
\NormalTok{      p{-}num{-}align{-}vertical: horizon,}
\NormalTok{      p{-}num{-}pad{-}left: {-}5pt,}
\NormalTok{      p{-}num{-}pad{-}horizontal: 0pt,}
\NormalTok{      p{-}num{-}size: 16pt,}
\NormalTok{      p{-}num{-}border: rgb("\#ff4136"),}
\NormalTok{      p{-}num{-}halign{-}alternate: false,}
\NormalTok{    ),}
\NormalTok{  ),}
\NormalTok{  pad\_content: 10pt,}
\NormalTok{  contents: (}
\NormalTok{    ["Page 1 content"],}
\NormalTok{    ["Page 2 content"],}
\NormalTok{    ["Page 3 content"],}
\NormalTok{    ["Page 4 content"],}
\NormalTok{  ),}
\NormalTok{)}
\end{Highlighting}
\end{Shaded}

This will create an unordered draft signature layout with US Legal paper
size, larger margins, no page borders, page numbers at the bottom right
corner with a red border, and more padding around the content.

\subsubsection{Notes}\label{notes}

\begin{itemize}
\tightlist
\item
  The \texttt{\ sig\ } function is currently hardcoded to only handle
  two-page single-fold signatures. Other more complicated signatures may
  be supported in the future.
\item
  The \texttt{\ num-layout\ } function is a helper to create page number
  layouts with default values.
\item
  The \texttt{\ booklet\ } function is a placeholder for automatically
  breaking a single content block into pages dynamically. It is not
  implemented yet but will be added in coming versions.
\end{itemize}

\subsection{Collaboration}\label{collaboration}

I would love to see this package eventually turn into a community
effort. So any interest in collaboration is very welcome! You can find
the github repository for this library here:
\href{https://github.com/harrellbm/Bookletic}{Bookletic Repo} . Feel
free to file an issue, pull request, or start a discussion.

\subsection{Changlog}\label{changlog}

\paragraph{0.3.0}\label{section}

\begin{itemize}
\tightlist
\item
  Remove internal dependency on native page function. This allows the
  user to set the page function separately with full control over paper
  type, outer margins and everything else defined by the native page
  function.
\item
  Add p-num-halign-alternate to page number layout allowing setting page
  numbers to alternate on facing pages making it possible to place page
  numbers along the outside or inside edges of facing pages.
\item
  Internal improvements for ordering algorithm.
\item
  Add \texttt{\ num-layout\ } function helper.
\end{itemize}

\paragraph{0.2.0}\label{section-1}

\begin{itemize}
\tightlist
\item
  Handle odd number of pages by inserting a blank back cover
\item
  Implements page number layouts to allow defining different page
  numbers for different page ranges.
\item
  Add various other page number options
\end{itemize}

\paragraph{0.1.0}\label{section-2}

Initial Commit

\subsubsection{How to add}\label{how-to-add}

Copy this into your project and use the import as \texttt{\ bookletic\ }

\begin{verbatim}
#import "@preview/bookletic:0.3.0"
\end{verbatim}

\includesvg[width=0.16667in,height=0.16667in]{/assets/icons/16-copy.svg}

Check the docs for
\href{https://typst.app/docs/reference/scripting/\#packages}{more
information on how to import packages} .

\subsubsection{About}\label{about}

\begin{description}
\tightlist
\item[Author s :]
\href{https://github.com/harrellbm}{Brenden Harrell} \&
\href{https://github.com/paul2t}{Paul DE TEMMERMAN}
\item[License:]
Apache-2.0
\item[Current version:]
0.3.0
\item[Last updated:]
October 10, 2024
\item[First released:]
May 8, 2024
\item[Minimum Typst version:]
0.11.0
\item[Archive size:]
7.91 kB
\href{https://packages.typst.org/preview/bookletic-0.3.0.tar.gz}{\pandocbounded{\includesvg[keepaspectratio]{/assets/icons/16-download.svg}}}
\item[Repository:]
\href{https://github.com/harrellbm/Bookletic.git}{GitHub}
\item[Categor ies :]
\begin{itemize}
\tightlist
\item[]
\item
  \pandocbounded{\includesvg[keepaspectratio]{/assets/icons/16-docs.svg}}
  \href{https://typst.app/universe/search/?category=book}{Book}
\item
  \pandocbounded{\includesvg[keepaspectratio]{/assets/icons/16-layout.svg}}
  \href{https://typst.app/universe/search/?category=layout}{Layout}
\item
  \pandocbounded{\includesvg[keepaspectratio]{/assets/icons/16-map.svg}}
  \href{https://typst.app/universe/search/?category=flyer}{Flyer}
\end{itemize}
\end{description}

\subsubsection{Where to report issues?}\label{where-to-report-issues}

This package is a project of Brenden Harrell and Paul DE TEMMERMAN .
Report issues on \href{https://github.com/harrellbm/Bookletic.git}{their
repository} . You can also try to ask for help with this package on the
\href{https://forum.typst.app}{Forum} .

Please report this package to the Typst team using the
\href{https://typst.app/contact}{contact form} if you believe it is a
safety hazard or infringes upon your rights.

\phantomsection\label{versions}
\subsubsection{Version history}\label{version-history}

\begin{longtable}[]{@{}ll@{}}
\toprule\noalign{}
Version & Release Date \\
\midrule\noalign{}
\endhead
\bottomrule\noalign{}
\endlastfoot
0.3.0 & October 10, 2024 \\
\href{https://typst.app/universe/package/bookletic/0.2.0/}{0.2.0} & May
23, 2024 \\
\href{https://typst.app/universe/package/bookletic/0.1.0/}{0.1.0} & May
8, 2024 \\
\end{longtable}

Typst GmbH did not create this package and cannot guarantee correct
functionality of this package or compatibility with any version of the
Typst compiler or app.
