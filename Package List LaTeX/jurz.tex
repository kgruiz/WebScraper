\title{typst.app/universe/package/jurz}

\phantomsection\label{banner}
\section{jurz}\label{jurz}

{ 0.1.0 }

Randziffern in Typst

\phantomsection\label{readme}
\href{https://de.wikipedia.org/w/index.php?title=Randnummer&oldid=231943223}{\emph{Randziffern}}
(also called \emph{Randnummern} ) are a way to reference text passages
in a document, independent of the page number or the section number.
They are used in many German legal texts, for example. This package
provides a way to create \emph{Randziffern} in Typst.

\subsection{Demo}\label{demo}

\begin{longtable}[]{@{}ll@{}}
\toprule\noalign{}
\endhead
\bottomrule\noalign{}
\endlastfoot
\pandocbounded{\includesvg[keepaspectratio]{https://github.com/typst/packages/raw/main/packages/preview/jurz/0.1.0/demo-2.svg}}
&
\pandocbounded{\includesvg[keepaspectratio]{https://github.com/typst/packages/raw/main/packages/preview/jurz/0.1.0/demo-3.svg}} \\
\end{longtable}

View source

\begin{Shaded}
\begin{Highlighting}[]
\NormalTok{\#show: init{-}jurz.with(}
\NormalTok{  gap: 1em,}
\NormalTok{  two{-}sided: true}
\NormalTok{)}

\NormalTok{\#rz \#lorem(50)}

\NormalTok{\#lorem(20)}

\NormalTok{\#rz\textless{}abc\textgreater{} \#lorem(30)}

\NormalTok{\#rz \#lorem(40)}

\NormalTok{\#rz \#lorem(50)}

\NormalTok{\#lorem(20)}

\NormalTok{\#rz \#lorem(24)}

\NormalTok{Fur further information, look at @abc.}
\end{Highlighting}
\end{Shaded}

\subsection{Reference}\label{reference}

\subsubsection{\texorpdfstring{\texttt{\ init-jurz\ }}{ init-jurz }}\label{init-jurz}

A show rule that initializes the \emph{Randziffern} for the document.
This rule should be placed at the beginning of the document. It also
allows customizing the behavior of the \emph{Randziffern} .

\paragraph{Usage}\label{usage}

\begin{Shaded}
\begin{Highlighting}[]
\NormalTok{\#show: init{-}jurz.with(}
\NormalTok{ // parameters}
\NormalTok{ // two{-}sided: true,}
\NormalTok{ // gap: 1em,}
\NormalTok{ // supplement: "Rz.",}
\NormalTok{ // reset{-}level: 0,}
\NormalTok{)}
\end{Highlighting}
\end{Shaded}

\paragraph{Parameters}\label{parameters}

\begin{itemize}
\tightlist
\item
  \texttt{\ two-sided\ } (optional): If \texttt{\ true\ } , the
  \emph{Randziffern} are placed on the outer margin of the page. If
  \texttt{\ false\ } , they are placed on the left margin. Default is
  \texttt{\ true\ } .
\item
  \texttt{\ gap\ } (optional): The distance between the
  \emph{Randziffer} and the text. Default is \texttt{\ 1em\ } .
\item
  \texttt{\ supplement\ } (optional): The text that is placed before the
  \emph{Randziffer} when referencing it. Default is \texttt{\ "Rz."\ } .
\item
  \texttt{\ reset-level\ } (optional): The heading level at which the
  \emph{Randziffern} are reset. If set to \texttt{\ 3\ } , for example,
  the numbering of the \emph{Randziffern} restarts after every heading
  of levels \texttt{\ 1\ } , \texttt{\ 2\ } , or \texttt{\ 3\ } .
  Default is \texttt{\ 0\ } .
\end{itemize}

\subsubsection{\texorpdfstring{\texttt{\ rz\ }}{ rz }}\label{rz}

Adds a \emph{Randziffer} to the text. The \emph{Randziffer} is a unique
identifier that can be referenced in the text.

You can add references the same way you can with headings. In fact, the
\emph{Randziffer} is treated as a heading of level \texttt{\ 99\ } under
the hood.

\paragraph{Usage}\label{usage-1}

\begin{Shaded}
\begin{Highlighting}[]
\NormalTok{\#rz \#lorem(100)}
\NormalTok{\#rz\textless{}abc\textgreater{} \#lorem(100)}

\NormalTok{See also @abc.}
\end{Highlighting}
\end{Shaded}

\subsection{License}\label{license}

This package is licensed under the MIT License.

\subsubsection{How to add}\label{how-to-add}

Copy this into your project and use the import as \texttt{\ jurz\ }

\begin{verbatim}
#import "@preview/jurz:0.1.0"
\end{verbatim}

\includesvg[width=0.16667in,height=0.16667in]{/assets/icons/16-copy.svg}

Check the docs for
\href{https://typst.app/docs/reference/scripting/\#packages}{more
information on how to import packages} .

\subsubsection{About}\label{about}

\begin{description}
\tightlist
\item[Author :]
\href{https://github.com/pklaschka}{Zuri Klaschka}
\item[License:]
MIT
\item[Current version:]
0.1.0
\item[Last updated:]
April 4, 2024
\item[First released:]
April 4, 2024
\item[Archive size:]
2.46 kB
\href{https://packages.typst.org/preview/jurz-0.1.0.tar.gz}{\pandocbounded{\includesvg[keepaspectratio]{/assets/icons/16-download.svg}}}
\item[Discipline :]
\begin{itemize}
\tightlist
\item[]
\item
  \href{https://typst.app/universe/search/?discipline=law}{Law}
\end{itemize}
\item[Categor ies :]
\begin{itemize}
\tightlist
\item[]
\item
  \pandocbounded{\includesvg[keepaspectratio]{/assets/icons/16-envelope.svg}}
  \href{https://typst.app/universe/search/?category=office}{Office}
\item
  \pandocbounded{\includesvg[keepaspectratio]{/assets/icons/16-package.svg}}
  \href{https://typst.app/universe/search/?category=components}{Components}
\item
  \pandocbounded{\includesvg[keepaspectratio]{/assets/icons/16-layout.svg}}
  \href{https://typst.app/universe/search/?category=layout}{Layout}
\end{itemize}
\end{description}

\subsubsection{Where to report issues?}\label{where-to-report-issues}

This package is a project of Zuri Klaschka . You can also try to ask for
help with this package on the \href{https://forum.typst.app}{Forum} .

Please report this package to the Typst team using the
\href{https://typst.app/contact}{contact form} if you believe it is a
safety hazard or infringes upon your rights.

\phantomsection\label{versions}
\subsubsection{Version history}\label{version-history}

\begin{longtable}[]{@{}ll@{}}
\toprule\noalign{}
Version & Release Date \\
\midrule\noalign{}
\endhead
\bottomrule\noalign{}
\endlastfoot
0.1.0 & April 4, 2024 \\
\end{longtable}

Typst GmbH did not create this package and cannot guarantee correct
functionality of this package or compatibility with any version of the
Typst compiler or app.
