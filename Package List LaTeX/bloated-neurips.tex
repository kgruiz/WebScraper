\title{typst.app/universe/package/bloated-neurips}

\phantomsection\label{banner}
\phantomsection\label{template-thumbnail}
\pandocbounded{\includegraphics[keepaspectratio]{https://packages.typst.org/preview/thumbnails/bloated-neurips-0.5.1-small.webp}}

\section{bloated-neurips}\label{bloated-neurips}

{ 0.5.1 }

NeurIPS-style paper template to publish at the Conference and Workshop
on Neural Information Processing Systems

\href{/app?template=bloated-neurips&version=0.5.1}{Create project in
app}

\phantomsection\label{readme}
\subsection{Usage}\label{usage}

You can use this template in the Typst web app by clicking \emph{Start
from template} on the dashboard and searching for
\texttt{\ bloated-neurips\ } .

Alternatively, you can use the CLI to kick this project off using the
command

\begin{Shaded}
\begin{Highlighting}[]
\NormalTok{typst init @preview/bloated{-}neurips}
\end{Highlighting}
\end{Shaded}

Typst will create a new directory with all the files needed to get you
started.

\subsection{Configuration}\label{configuration}

This template exports the \texttt{\ neurips2023\ } and
\texttt{\ neurips2024\ } function with the following named arguments.

\begin{itemize}
\tightlist
\item
  \texttt{\ title\ } : The paper’s title as content.
\item
  \texttt{\ authors\ } : An array of author dictionaries. Each of the
  author dictionaries must have a name key and can have the keys
  department, organization, location, and email.
\item
  \texttt{\ abstract\ } : The content of a brief summary of the paper or
  none. Appears at the top under the title.
\item
  \texttt{\ bibliography\ } : The result of a call to the bibliography
  function or none. The function also accepts a single, positional
  argument for the body of the paper.
\item
  \texttt{\ appendix\ } : A content which is placed after bibliography.
\item
  \texttt{\ accepted\ } : If this is set to \texttt{\ false\ } then
  anonymized ready for submission document is produced;
  \texttt{\ accepted:\ true\ } produces camera-redy version. If the
  argument is set to \texttt{\ none\ } then preprint version is produced
  (can be uploaded to arXiv).
\end{itemize}

The template will initialize your package with a sample call to the
\texttt{\ neurips2024\ } function in a show rule. If you want to change
an existing project to use this template, you can add a show rule at the
top of your file as follows.

\begin{Shaded}
\begin{Highlighting}[]
\NormalTok{\#import "@preview/bloated{-}neurips:0.5.1": neurips2024}

\NormalTok{\#show: neurips2024.with(}
\NormalTok{  title: [Formatting Instructions For NeurIPS 2024],}
\NormalTok{  authors: (authors, affls),}
\NormalTok{  keywords: ("Machine Learning", "NeurIPS"),}
\NormalTok{  abstract: [}
\NormalTok{    The abstract paragraph should be indented ½ inch (3 picas) on both the}
\NormalTok{    left{-} and right{-}hand margins. Use 10 point type, with a vertical spacing}
\NormalTok{    (leading) of 11 points. The word *Abstract* must be centered, bold, and in}
\NormalTok{    point size 12. Two line spaces precede the abstract. The abstract must be}
\NormalTok{    limited to one paragraph.}
\NormalTok{  ],}
\NormalTok{  bibliography: bibliography("main.bib"),}
\NormalTok{  appendix: [}
\NormalTok{    \#include "appendix.typ"}
\NormalTok{    \#include "checklist.typ"}
\NormalTok{  ],}
\NormalTok{  accepted: false,}
\NormalTok{)}

\NormalTok{\#lorem(42)}
\end{Highlighting}
\end{Shaded}

With template of version v0.5.1 or newer, one can override some parts.
Specifically, \texttt{\ get-notice\ } entry of \texttt{\ aux\ }
directory parameter of show rule allows to adjust the NeurIPS 2024
template to Science4DL workshop as follows.

\begin{Shaded}
\begin{Highlighting}[]
\NormalTok{\#import "@preview/bloated{-}neurips:0.5.1": neurips}

\NormalTok{\#let get{-}notice(accepted) = if accepted == none \{}
\NormalTok{  return [Preprint. Under review.]}
\NormalTok{\} else if accepted \{}
\NormalTok{  return [}
\NormalTok{    Workshop on Scientific Methods for Understanding Deep Learning, NeurIPS}
\NormalTok{    2024.}
\NormalTok{  ]}
\NormalTok{\} else \{}
\NormalTok{  return [}
\NormalTok{    Submitted to Workshop on Scientific Methods for Understanding Deep}
\NormalTok{    Learning, NeurIPS 2024.}
\NormalTok{  ]}
\NormalTok{\}}

\NormalTok{\#let science4dl2024(}
\NormalTok{  title: [], authors: (), keywords: (), date: auto, abstract: none,}
\NormalTok{  bibliography: none, appendix: none, accepted: false, body,}
\NormalTok{) = \{}
\NormalTok{  show: neurips.with(}
\NormalTok{    title: title,}
\NormalTok{    authors: authors,}
\NormalTok{    keywords: keywords,}
\NormalTok{    date: date,}
\NormalTok{    abstract: abstract,}
\NormalTok{    accepted: false,}
\NormalTok{    aux: (get{-}notice: get{-}notice),}
\NormalTok{  )}
\NormalTok{  body}
\NormalTok{\}}
\end{Highlighting}
\end{Shaded}

\subsection{Issues}\label{issues}

\begin{itemize}
\item
  The biggest and the most important issues is related to line ruler. We
  are not aware of universal method for numbering lines in the main body
  of a paper. Fortunately, line numbering support has been implemented
  at \href{https://github.com/typst/typst/pull/4516}{typst/typst\#4516}
  . Let’s wait for the next major release v0.12.0!
\item
  There is an issue in Typst with spacing between figures and between
  figure with floating placement. The issue is that there is no way to
  specify gap between subsequent figures. In order to have behaviour
  similar to original LaTeX template, one should consider direct spacing
  adjacemnt with \texttt{\ v(-1em)\ } as follows.

\begin{Shaded}
\begin{Highlighting}[]
\NormalTok{\#figure(}
\NormalTok{  rect(width: 4.25cm, height: 4.25cm, stroke: 0.4pt),}
\NormalTok{  caption: [Sample figure caption.\#v({-}1em)],}
\NormalTok{  placement: top,}
\NormalTok{)}
\NormalTok{\#figure(}
\NormalTok{  rect(width: 4.25cm, height: 4.25cm, stroke: 0.4pt),}
\NormalTok{  caption: [Sample figure caption.],}
\NormalTok{  placement: top,}
\NormalTok{)}
\end{Highlighting}
\end{Shaded}
\item
  Another issue is related to Typst’s inablity to produce colored
  annotation. In order to mitigte the issue, we add a script which
  modifies annotations and make them colored.

\begin{Shaded}
\begin{Highlighting}[]
\NormalTok{../colorize{-}annotations.py \textbackslash{}}
\NormalTok{    example{-}paper.typst.pdf example{-}paper{-}colored.typst.pdf}
\end{Highlighting}
\end{Shaded}

  See
  \href{https://github.com/daskol/typst-templates/\#colored-annotations}{README.md}
  for details.
\item
  NeurIPS 2023/2024 instructions discuss bibliography in vague terms.
  Namely, there is not specific requirements. Thus we stick to
  \texttt{\ ieee\ } bibliography style since we found it in several
  accepted papers and it is similar to that in the example paper.
\item
  It is unclear how to render notice in the bottom of the title page in
  case of final ( \texttt{\ accepted:\ true\ } ) or preprint (
  \texttt{\ accepted:\ none\ } ) submission.
\end{itemize}

\href{/app?template=bloated-neurips&version=0.5.1}{Create project in
app}

\subsubsection{How to use}\label{how-to-use}

Click the button above to create a new project using this template in
the Typst app.

You can also use the Typst CLI to start a new project on your computer
using this command:

\begin{verbatim}
typst init @preview/bloated-neurips:0.5.1
\end{verbatim}

\includesvg[width=0.16667in,height=0.16667in]{/assets/icons/16-copy.svg}

\subsubsection{About}\label{about}

\begin{description}
\tightlist
\item[Author :]
\href{mailto:daniel.bershatsky2@skoltech.ru}{Daniel Bershatsky}
\item[License:]
MIT
\item[Current version:]
0.5.1
\item[Last updated:]
October 8, 2024
\item[First released:]
March 19, 2024
\item[Minimum Typst version:]
0.11.1
\item[Archive size:]
21.2 kB
\href{https://packages.typst.org/preview/bloated-neurips-0.5.1.tar.gz}{\pandocbounded{\includesvg[keepaspectratio]{/assets/icons/16-download.svg}}}
\item[Repository:]
\href{https://github.com/daskol/typst-templates}{GitHub}
\item[Discipline s :]
\begin{itemize}
\tightlist
\item[]
\item
  \href{https://typst.app/universe/search/?discipline=computer-science}{Computer
  Science}
\item
  \href{https://typst.app/universe/search/?discipline=mathematics}{Mathematics}
\end{itemize}
\item[Categor y :]
\begin{itemize}
\tightlist
\item[]
\item
  \pandocbounded{\includesvg[keepaspectratio]{/assets/icons/16-atom.svg}}
  \href{https://typst.app/universe/search/?category=paper}{Paper}
\end{itemize}
\end{description}

\subsubsection{Where to report issues?}\label{where-to-report-issues}

This template is a project of Daniel Bershatsky . Report issues on
\href{https://github.com/daskol/typst-templates}{their repository} . You
can also try to ask for help with this template on the
\href{https://forum.typst.app}{Forum} .

Please report this template to the Typst team using the
\href{https://typst.app/contact}{contact form} if you believe it is a
safety hazard or infringes upon your rights.

\phantomsection\label{versions}
\subsubsection{Version history}\label{version-history}

\begin{longtable}[]{@{}ll@{}}
\toprule\noalign{}
Version & Release Date \\
\midrule\noalign{}
\endhead
\bottomrule\noalign{}
\endlastfoot
0.5.1 & October 8, 2024 \\
\href{https://typst.app/universe/package/bloated-neurips/0.5.0/}{0.5.0}
& September 22, 2024 \\
\href{https://typst.app/universe/package/bloated-neurips/0.2.1/}{0.2.1}
& March 19, 2024 \\
\end{longtable}

Typst GmbH did not create this template and cannot guarantee correct
functionality of this template or compatibility with any version of the
Typst compiler or app.
