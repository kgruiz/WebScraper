\title{typst.app/universe/package/gloss-awe}

\phantomsection\label{banner}
\section{gloss-awe}\label{gloss-awe}

{ 0.0.5 }

Awesome Glossary for Typst.

\phantomsection\label{readme}
Automatically create a glossary in \href{https://typst.app/}{typst} .

This typst component creates a glossary page from a given pool of
potential glossary entries using only those entries, that are marked
with the \texttt{\ gls\ } or \texttt{\ gls-add\ } functions in the
document.

âš~ï¸? Typst is in beta and evolving, and this package evolves with it.
As a result, no backward compatibility is guaranteed yet. Also, the
package itself is under development and fine-tuning.

\subsection{Table of Contents}\label{table-of-contents}

\begin{itemize}
\tightlist
\item
  \href{https://github.com/typst/packages/raw/main/packages/preview/gloss-awe/0.0.5/\#usage}{Usage}

  \begin{itemize}
  \tightlist
  \item
    \href{https://github.com/typst/packages/raw/main/packages/preview/gloss-awe/0.0.5/\#marking-the-entries}{Marking
    the Entries}
  \item
    \href{https://github.com/typst/packages/raw/main/packages/preview/gloss-awe/0.0.5/\#controlling-the-show}{Controlling
    the Show}
  \item
    \href{https://github.com/typst/packages/raw/main/packages/preview/gloss-awe/0.0.5/\#hiding-entries-from-the-glossary-page}{Hiding
    Entries from the Glossary Page}
  \item
    \href{https://github.com/typst/packages/raw/main/packages/preview/gloss-awe/0.0.5/\#pool-of-entries}{Pool
    of Entries}
  \item
    \href{https://github.com/typst/packages/raw/main/packages/preview/gloss-awe/0.0.5/\#unknown-entries}{Unknown
    Entries}
  \item
    \href{https://github.com/typst/packages/raw/main/packages/preview/gloss-awe/0.0.5/\#creating-the-glossary-page}{Creating
    the glossary page}
  \end{itemize}
\item
  \href{https://github.com/typst/packages/raw/main/packages/preview/gloss-awe/0.0.5/\#changelog}{Changelog}

  \begin{itemize}
  \tightlist
  \item
    \href{https://github.com/typst/packages/raw/main/packages/preview/gloss-awe/0.0.5/\#v005}{v0.0.5}
  \item
    \href{https://github.com/typst/packages/raw/main/packages/preview/gloss-awe/0.0.5/\#v004}{v0.0.4}
  \item
    \href{https://github.com/typst/packages/raw/main/packages/preview/gloss-awe/0.0.5/\#v003}{v0.0.3}
  \item
    \href{https://github.com/typst/packages/raw/main/packages/preview/gloss-awe/0.0.5/\#v002}{v0.0.2}
  \end{itemize}
\end{itemize}

\subsection{Usage}\label{usage}

\subsubsection{Marking the Entries}\label{marking-the-entries}

To include a term into the glossary, it can be marked with the
\texttt{\ gls\ } function. The simplest form is like this:

\begin{Shaded}
\begin{Highlighting}[]
\NormalTok{This is how to mark something for the glossary in \#gls[Typst].}
\end{Highlighting}
\end{Shaded}

The function will render as defined with the specified show rule (see
below!).

\subsubsection{Controlling the Show}\label{controlling-the-show}

To control, how the markers are rendered in the document, a show rule
has to be defined. It usually looks like the following:

\begin{Shaded}
\begin{Highlighting}[]
\NormalTok{// marker display : this rule makes the glossary marker in the document visible.}
\NormalTok{\#show figure.where(kind: "jkrb\_glossary"): it =\textgreater{} \{it.body\}}
\end{Highlighting}
\end{Shaded}

\subsubsection{Hiding Entries from the Glossary
Page}\label{hiding-entries-from-the-glossary-page}

It is also possible to hide entries (temporarily) from the generated
glossary page without removing any markers for them from the document.

The following sample will hide the entries for “Amaranth� and
“Butterscotch� from the glossary, even if it is marked with
\texttt{\ gls{[}...{]}\ } or \texttt{\ gls-add{[}...{]}\ } somewhere in
the document.

\begin{Shaded}
\begin{Highlighting}[]
\NormalTok{    \#let hidden{-}entries = (}
\NormalTok{        "Amaranth",}
\NormalTok{        "Butterscotch"}
\NormalTok{    )}

\NormalTok{    \#make{-}glossary(glossary{-}pool, excluded: hidden{-}entries)}
\end{Highlighting}
\end{Shaded}

\subsubsection{Pool of Entries}\label{pool-of-entries}

A “pool of entries� is a typst dictionary. It can be read from a
file or may be the result of other operations.

One or more pool(s) are then given to the \texttt{\ make-glossary()\ }
function. This will create a glossary page of all referenced entries. If
given more than one pool, the order pools are searched in the order they
are given to the method. The first match wins. This can be used to have
a global pool to be used in different documents, and another one for
local usage only.

The pool consists of a dictionary of definition entries. The key of an
entry is the term. Note, that it is case-sensitive. Each entry itself is
also a dictionary, and its main key is \texttt{\ description\ } . This
is the content for the term. There may be other keys in an entry in the
future. For now, it supports:

\begin{itemize}
\tightlist
\item
  description
\item
  link
\end{itemize}

An entry in the pool for the term “Engine� file may look like this:

\begin{Shaded}
\begin{Highlighting}[]
\NormalTok{    Engine: (}
\NormalTok{        description: [}

\NormalTok{            In the context of software, an engine...}

\NormalTok{        ],}
\NormalTok{        link: [}

\NormalTok{            (1) Software engine {-} Wikipedia.}
\NormalTok{            https://en.wikipedia.org/wiki/Software\_engine}
\NormalTok{            (13.5.2023).}

\NormalTok{        ]}
\NormalTok{    ),}
\end{Highlighting}
\end{Shaded}

\subsubsection{Unknown Entries}\label{unknown-entries}

If the document marks a term that is not contained in the pool, an entry
will be generated anyway, but it will be visually marked as missing.
This is convenient for the workflow, where one can mark the desired
entries while writing the document, and provide missing entries later.

There is a parameter for the \texttt{\ make-glossary()\ } function named
\texttt{\ missing\ } , where a function can be provided to format or
even suppress the missing entries.

\subsubsection{Creating the Glossary
Page}\label{creating-the-glossary-page}

To create the glossary page, provide the pool of potential entries to
the make-glossary function. The following listing shows a complete
sample document with a glossary. The sample glossary pool is contained
in the main document as well:

\begin{Shaded}
\begin{Highlighting}[]
\NormalTok{    \#import "@preview/gloss{-}awe:0.0.5": *}

\NormalTok{    // Text settings}
\NormalTok{    \#set text(font: ("Arial", "Trebuchet MS"), size: 12pt)}

\NormalTok{    // Defining the Glossary Pool with definitions.}
\NormalTok{    \#let glossary{-}pool = (}
\NormalTok{        Cloud: (}
\NormalTok{            description: [}

\NormalTok{                Cloud computing is a model where computer resources are made available}
\NormalTok{                over the internet. Such resources can be assigned on demand in a very short}
\NormalTok{                time, and only as long as they are required by the user.}

\NormalTok{            ]}
\NormalTok{        ),}

\NormalTok{        Marker: (}
\NormalTok{            description: [}

\NormalTok{                A Marker in \textasciigrave{}gloss{-}awe\textasciigrave{} is a typst function to mark a word or phrase to appear}
\NormalTok{                in the documents glossary. The marker is also linked to the glossary section}
\NormalTok{                by referencing the label \textasciigrave{}\textless{}Glossary\textgreater{}\textasciigrave{}.}

\NormalTok{            ]}
\NormalTok{        ),}

\NormalTok{        Glossary: (}
\NormalTok{            description: [}

\NormalTok{                A glossary is a list of terms and their definitions that are specific to a}
\NormalTok{                particular subject or field. It is used to define the intended meaning of}
\NormalTok{                terms used in a document and to agree on a common definition of those terms. A}
\NormalTok{                well{-}defined glossary can be very helpful in documents where very specific}
\NormalTok{                meanings of certain terms are used.}

\NormalTok{            ]}
\NormalTok{        ),}

\NormalTok{        "Glossary Pool": (}
\NormalTok{            description: [}

\NormalTok{                A glossary pool is a collection of glossary entries. An automated tool can}
\NormalTok{                pull needed definitions from this pool to create the glossary pages for a}
\NormalTok{                specific context.}

\NormalTok{            ]}
\NormalTok{        ),}

\NormalTok{        REST: (}
\NormalTok{            description: [}

\NormalTok{                Representational State Transfer (abgekürzt REST) ist ein Paradigma für die}
\NormalTok{                Softwarearchitektur von verteilten Systemen, insbesondere für Webservices.}
\NormalTok{                REST ist eine Abstraktion der Struktur und des Verhaltens des World Wide}
\NormalTok{                Web. REST hat das Ziel, einen Architekturstil zu schaffen, der den}
\NormalTok{                Anforderungen des modernen Web besser genügt.}

\NormalTok{            ]}
\NormalTok{        ),}

\NormalTok{        XML: (}
\NormalTok{            description: [}

\NormalTok{                XML stands for \textasciigrave{}\textquotesingle{}eXtensible Markup Language\textquotesingle{}\textasciigrave{}.}

\NormalTok{            ],}
\NormalTok{            link: [https://www.w3.org/XML]}
\NormalTok{        ),}
\NormalTok{    )}

\NormalTok{    // Defining, how marked glossary entries in the document appear}
\NormalTok{    \#show figure.where(kind: "jkrb\_glossary"): it =\textgreater{} \{it.body\}}

\NormalTok{    // This alternate rule, creates links to the glossary for marked entries.}
\NormalTok{    // \#show figure.where(kind: "jkrb\_glossary"): it =\textgreater{} [\#link(\textless{}Glossar\textgreater{})[\#it.body]]}

\NormalTok{    = My Sample Document with \textasciigrave{}gloss{-}awe\textasciigrave{}}

\NormalTok{    In this document the usage of the \textasciigrave{}gloss{-}awe\textasciigrave{} package is demonstrated to create a glossary}
\NormalTok{    with the help of a simple and small sample glossary pool. We have defined the pool in a}
\NormalTok{    dictionary named \#gls[Glossary Pool] above. It contains the definitions the \textasciigrave{}gloss{-}awe\textasciigrave{}}
\NormalTok{    can use to build the glossary in the \#gls[Glossary] section of this document. The pool}
\NormalTok{    could also come from external files, like \#gls[JSON] or \#gls[XML] or other sources. Only}
\NormalTok{    those definitions are shown in the glossary, that are marked in this document with one of}
\NormalTok{    the \#gls(entry: "Marker")[marker] functions \textasciigrave{}gloss{-}awe\textasciigrave{} provides.}

\NormalTok{    If a Word is marked, that is not in the Glossary Pool, it gets a special markup in the}
\NormalTok{    resulting glossary, making it easy for the Author to spot the missing information an}
\NormalTok{    providing a definition. In this sample, there is no definition for "JSON" provided,}
\NormalTok{    resulting in an Entry with a red remark "\#text(fill: red)[No\textasciitilde{}glossary\textasciitilde{}entry]".}

\NormalTok{    There is also a way to include Entries in the glossary for Words that are not contained in}
\NormalTok{    the documents text:}

\NormalTok{    \#gls{-}add("Cloud")}
\NormalTok{    \#gls{-}add("REST")}


\NormalTok{    = Glossary \textless{}Glossary\textgreater{}}

\NormalTok{    This section contains the generated Glossary, in a nice two{-}column{-}layout. It also bears a}
\NormalTok{    label, to enable the linking from marked words to the glossary.}

\NormalTok{    \#line(length: 100\%)}
\NormalTok{    \#set text(font: ("Arial", "Trebuchet MS"), size: 10pt)}
\NormalTok{    \#columns(2)[}
\NormalTok{        \#make{-}glossary(glossary{-}pool)}
\NormalTok{    ]}
\end{Highlighting}
\end{Shaded}

More usage samples are shown in the document
\texttt{\ sample-usage.typ\ } on
\href{https://github.com/typst/packages/raw/main/packages/preview/gloss-awe/0.0.5/\%5BTitle\%5D(https://github.com/RolfBremer/typst-glossary)}{gloss-awe´s
GitHub} .

A more complex sample PDF is available there as well.

\subsection{Changelog}\label{changelog}

\subsubsection{v0.0.5}\label{v0.0.5}

\begin{itemize}
\tightlist
\item
  Address change in \texttt{\ figure.caption\ } in typst (commit:
  976abdf ).
\end{itemize}

\subsubsection{v0.0.4}\label{v0.0.4}

\begin{itemize}
\tightlist
\item
  Breaking: Renamed the main file from \texttt{\ glossary.typ\ } to
  \texttt{\ gloss-awe.typ\ } to match package.
\item
  Added support for hidden glossary entries.
\item
  Added a Changelog to this readme.
\end{itemize}

\subsubsection{v0.0.3}\label{v0.0.3}

\begin{itemize}
\tightlist
\item
  Added support for package manager in Typst.
\item
  Add \texttt{\ gls-add{[}...{]}\ } function for entries that are not in
  the document.
\end{itemize}

\subsubsection{v.0.0.2}\label{v.0.0.2}

\begin{itemize}
\tightlist
\item
  Moved version to Github.
\end{itemize}

\subsubsection{How to add}\label{how-to-add}

Copy this into your project and use the import as \texttt{\ gloss-awe\ }

\begin{verbatim}
#import "@preview/gloss-awe:0.0.5"
\end{verbatim}

\includesvg[width=0.16667in,height=0.16667in]{/assets/icons/16-copy.svg}

Check the docs for
\href{https://typst.app/docs/reference/scripting/\#packages}{more
information on how to import packages} .

\subsubsection{About}\label{about}

\begin{description}
\tightlist
\item[Author :]
\href{https://github.com/RolfBremer}{JKRB}
\item[License:]
Apache-2.0
\item[Current version:]
0.0.5
\item[Last updated:]
September 13, 2023
\item[First released:]
July 3, 2023
\item[Archive size:]
8.39 kB
\href{https://packages.typst.org/preview/gloss-awe-0.0.5.tar.gz}{\pandocbounded{\includesvg[keepaspectratio]{/assets/icons/16-download.svg}}}
\item[Repository:]
\href{https://github.com/RolfBremer/gloss-awe}{GitHub}
\end{description}

\subsubsection{Where to report issues?}\label{where-to-report-issues}

This package is a project of JKRB . Report issues on
\href{https://github.com/RolfBremer/gloss-awe}{their repository} . You
can also try to ask for help with this package on the
\href{https://forum.typst.app}{Forum} .

Please report this package to the Typst team using the
\href{https://typst.app/contact}{contact form} if you believe it is a
safety hazard or infringes upon your rights.

\phantomsection\label{versions}
\subsubsection{Version history}\label{version-history}

\begin{longtable}[]{@{}ll@{}}
\toprule\noalign{}
Version & Release Date \\
\midrule\noalign{}
\endhead
\bottomrule\noalign{}
\endlastfoot
0.0.5 & September 13, 2023 \\
\href{https://typst.app/universe/package/gloss-awe/0.0.4/}{0.0.4} & July
6, 2023 \\
\href{https://typst.app/universe/package/gloss-awe/0.0.3/}{0.0.3} & July
3, 2023 \\
\end{longtable}

Typst GmbH did not create this package and cannot guarantee correct
functionality of this package or compatibility with any version of the
Typst compiler or app.
