\title{typst.app/universe/package/codly}

\phantomsection\label{banner}
\section{codly}\label{codly}

{ 1.0.0 }

Codly is a beautiful code presentation template with many features like
smart indentation, line numbering, highlighting, etc.

{ } Featured Package

\phantomsection\label{readme}
\href{https://github.com/Dherse/codly/blob/main/docs.pdf}{\pandocbounded{\includegraphics[keepaspectratio]{https://img.shields.io/website?down_message=offline&label=manual&up_color=007aff&up_message=online&url=https\%3A\%2F\%2Fgithub.com\%2FDherse\%2Fcodly\%2Fblob\%2Fmain\%2Fdocs.pdf}}}
\href{https://github.com/Dherse/codly/blob/main/LICENSE}{\pandocbounded{\includegraphics[keepaspectratio]{https://img.shields.io/badge/license-MIT-brightgreen}}}
\pandocbounded{\includesvg[keepaspectratio]{https://github.com/Dherse/codly/actions/workflows/test.yml/badge.svg}}

Codly is a package that lets you easily create \textbf{beautiful} code
blocks for your Typst documents. It uses the newly added
\href{https://typst.app/docs/reference/text/raw/\#definitions-line}{\texttt{\ raw.line\ }}
function to work across all languages easily. You can customize the
icons, colors, and more to suit your document’s theme. By default it
has zebra striping, line numbers, for ease of reading.

A full set of documentation can be found
\href{https://raw.githubusercontent.com/Dherse/codly/main/docs.pdf}{in
the repo} .

\pandocbounded{\includegraphics[keepaspectratio]{https://github.com/typst/packages/raw/main/packages/preview/codly/1.0.0/demo.png}}

\begin{Shaded}
\begin{Highlighting}[]
\NormalTok{\#import "@preview/codly:1.0.0": *}
\NormalTok{\#show: codly{-}init.with()}

\NormalTok{\#codly(}
\NormalTok{  languages: (}
\NormalTok{    rust: (}
\NormalTok{      name: "Rust",}
\NormalTok{      icon: text(font: "tabler{-}icons", "\textbackslash{}u\{fa53\}),}
\NormalTok{      color: rgb("\#CE412B")}
\NormalTok{    ),}
\NormalTok{  )}
\NormalTok{)}

\NormalTok{\textasciigrave{}\textasciigrave{}\textasciigrave{}rust}
\NormalTok{pub fn main() \{}
\NormalTok{    println!("Hello, world!");}
\NormalTok{\}}
\NormalTok{\textasciigrave{}\textasciigrave{}\textasciigrave{}}
\end{Highlighting}
\end{Shaded}

\subsubsection{Setup}\label{setup}

To start using codly, you need to initialize codly using a show rule:

\begin{Shaded}
\begin{Highlighting}[]
\NormalTok{\#show: codly{-}init.with()}
\end{Highlighting}
\end{Shaded}

\begin{quote}
{[}!TIP{]} You only need to do this once at the top of your document!
\end{quote}

Then you \emph{can} to configure codly with your parameters:

\begin{Shaded}
\begin{Highlighting}[]
\NormalTok{\#codly(}
\NormalTok{  languages: (}
\NormalTok{    rust: (name: "Rust", icon: "\textbackslash{}u\{fa53\}", color: rgb("\#CE412B")),}
\NormalTok{  )}
\NormalTok{)}
\end{Highlighting}
\end{Shaded}

\begin{quote}
{[}!IMPORTANT{]} Any parameter that you leave blank will use the
previous values (or the default value if never set) similar to a
\texttt{\ set\ } rule in regular typst. But the changes are always
global unless you use the provided \texttt{\ codly.local\ } function. To
get a full list of all settings, see the
\href{https://raw.githubusercontent.com/Dherse/codly/main/docs.pdf}{documentation}
.
\end{quote}

Then you just need to add a code block and it will be automatically
displayed correctly:

\begin{verbatim}
```rust
pub fn main() {
    println!("Hello, world!");
}
```
\end{verbatim}

\subsubsection{Disabling}\label{disabling}

To locally disable codly, you can just do the following, you can then
later re-enable it using the \texttt{\ codly\ } configuration function.

\begin{Shaded}
\begin{Highlighting}[]
\NormalTok{\#disable{-}codly()}
\end{Highlighting}
\end{Shaded}

Alternatively, you can use the \texttt{\ no-codly\ } function to achieve
the same effect locally:

\begin{Shaded}
\begin{Highlighting}[]
\NormalTok{\#no{-}codly[}
\NormalTok{  \textasciigrave{}\textasciigrave{}\textasciigrave{}typ}
\NormalTok{  I will be displayed using the normal raw blocks.}
\NormalTok{  \textasciigrave{}\textasciigrave{}\textasciigrave{}}
\NormalTok{]}
\end{Highlighting}
\end{Shaded}

\subsubsection{Setting an offset}\label{setting-an-offset}

If you wish to add an offset to your code block, but without selecting a
subset of lines, you can use the \texttt{\ codly-offset\ } function:

\begin{Shaded}
\begin{Highlighting}[]
\NormalTok{// Sets a 5 line offset}
\NormalTok{\#codly{-}offset(5)}
\end{Highlighting}
\end{Shaded}

\subsubsection{Selecting a subset of
lines}\label{selecting-a-subset-of-lines}

If you wish to select a subset of lines, you can use the
\texttt{\ codly-range\ } function. By setting the start to 1 and the end
to \texttt{\ none\ } you can select all lines from the start to the end
of the code block.

\begin{Shaded}
\begin{Highlighting}[]
\NormalTok{\#codly{-}range(start: 5, end: 10)}
\end{Highlighting}
\end{Shaded}

\subsubsection{Adding a “skip�}\label{adding-a-uxe2ux153skipuxe2}

You can add a “fake� skip between lines using the \texttt{\ skips\ }
parameters:

\begin{Shaded}
\begin{Highlighting}[]
\NormalTok{// Before the 5th line (indexing start at 0), insert a 32 line jump.}
\NormalTok{\#codly(skips: ((4, 32), ))}
\end{Highlighting}
\end{Shaded}

This can be customized using the \texttt{\ skip-line\ } and
\texttt{\ skip-number\ } to customize what it looks like.

\subsubsection{Adding annotations}\label{adding-annotations}

\begin{quote}
{[}!IMPORTANT{]} This is a Beta feature and has a few quirks, refer to
\href{https://raw.githubusercontent.com/Dherse/codly/main/docs.pdf}{the
documentation} for those
\end{quote}

You can annotate a line/group of lines using the
\texttt{\ annotations\ } parameters :

\begin{Shaded}
\begin{Highlighting}[]
\NormalTok{// Add an annotation from the second line (0 indexing) to the 5th line included.}
\NormalTok{\#codly(}
\NormalTok{  annotations: (}
\NormalTok{    (}
\NormalTok{      start: 1,}
\NormalTok{      end: 4,}
\NormalTok{      content: block(}
\NormalTok{        width: 2em,}
\NormalTok{        // Rotate the element to make it look nice}
\NormalTok{        rotate(}
\NormalTok{          {-}90deg,}
\NormalTok{          align(center, box(width: 100pt)[Function body])}
\NormalTok{        )}
\NormalTok{      )}
\NormalTok{    ), }
\NormalTok{  )}
\NormalTok{)}
\end{Highlighting}
\end{Shaded}

\subsubsection{Disabling line numbers}\label{disabling-line-numbers}

You can configure this with the \texttt{\ codly\ } function:

\begin{Shaded}
\begin{Highlighting}[]
\NormalTok{\#codly(number{-}format: none)}
\end{Highlighting}
\end{Shaded}

\subsubsection{Disabling zebra striping}\label{disabling-zebra-striping}

You disable zebra striping by setting the \texttt{\ zebra-fill\ } to
white.

\begin{Shaded}
\begin{Highlighting}[]
\NormalTok{\#codly(zebra{-}fill: none)}
\end{Highlighting}
\end{Shaded}

\subsubsection{Customize the stroke}\label{customize-the-stroke}

You can customize the stroke surrounding the figure using the
\texttt{\ stroke\ } parameter of the \texttt{\ codly\ } function:

\begin{Shaded}
\begin{Highlighting}[]
\NormalTok{\#codly(stroke: 1pt + red)}
\end{Highlighting}
\end{Shaded}

\subsubsection{Misc}\label{misc}

You can also disable the icon, by setting the \texttt{\ display-icon\ }
parameter to \texttt{\ false\ } :

\begin{Shaded}
\begin{Highlighting}[]
\NormalTok{\#codly(display{-}icon: false)}
\end{Highlighting}
\end{Shaded}

Same with the name, whether the block is breakable, the radius, the
padding, and the width of the numbers columns, and so many more
\href{https://raw.githubusercontent.com/Dherse/codly/main/docs.pdf}{documentation}
.

\subsubsection{How to add}\label{how-to-add}

Copy this into your project and use the import as \texttt{\ codly\ }

\begin{verbatim}
#import "@preview/codly:1.0.0"
\end{verbatim}

\includesvg[width=0.16667in,height=0.16667in]{/assets/icons/16-copy.svg}

Check the docs for
\href{https://typst.app/docs/reference/scripting/\#packages}{more
information on how to import packages} .

\subsubsection{About}\label{about}

\begin{description}
\tightlist
\item[Author :]
\href{https://github.com/Dherse}{Dherse}
\item[License:]
MIT
\item[Current version:]
1.0.0
\item[Last updated:]
July 17, 2024
\item[First released:]
November 6, 2023
\item[Minimum Typst version:]
0.11.0
\item[Archive size:]
14.3 kB
\href{https://packages.typst.org/preview/codly-1.0.0.tar.gz}{\pandocbounded{\includesvg[keepaspectratio]{/assets/icons/16-download.svg}}}
\item[Repository:]
\href{https://github.com/Dherse/codly}{GitHub}
\end{description}

\subsubsection{Where to report issues?}\label{where-to-report-issues}

This package is a project of Dherse . Report issues on
\href{https://github.com/Dherse/codly}{their repository} . You can also
try to ask for help with this package on the
\href{https://forum.typst.app}{Forum} .

Please report this package to the Typst team using the
\href{https://typst.app/contact}{contact form} if you believe it is a
safety hazard or infringes upon your rights.

\phantomsection\label{versions}
\subsubsection{Version history}\label{version-history}

\begin{longtable}[]{@{}ll@{}}
\toprule\noalign{}
Version & Release Date \\
\midrule\noalign{}
\endhead
\bottomrule\noalign{}
\endlastfoot
1.0.0 & July 17, 2024 \\
\href{https://typst.app/universe/package/codly/0.2.1/}{0.2.1} & April 1,
2024 \\
\href{https://typst.app/universe/package/codly/0.2.0/}{0.2.0} & January
1, 2024 \\
\href{https://typst.app/universe/package/codly/0.1.0/}{0.1.0} & November
6, 2023 \\
\end{longtable}

Typst GmbH did not create this package and cannot guarantee correct
functionality of this package or compatibility with any version of the
Typst compiler or app.
