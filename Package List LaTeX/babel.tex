\title{typst.app/universe/package/babel}

\phantomsection\label{banner}
\section{babel}\label{babel}

{ 0.1.1 }

Redact text by replacing it with random characters

\phantomsection\label{readme}
\href{https://typst.app/universe/package/babel}{\pandocbounded{\includegraphics[keepaspectratio]{https://img.shields.io/badge/Typst_Universe-fdfdfd?logo=typst}}}
\href{https://codeberg.org/afiaith/babel}{\pandocbounded{\includegraphics[keepaspectratio]{https://img.shields.io/badge/Git_repo-fdfdfd?logo=codeberg}}}
\href{https://github.com/typst/packages/raw/main/packages/preview/babel/0.1.1/docs/manual.pdf}{\pandocbounded{\includegraphics[keepaspectratio]{https://img.shields.io/badge/\%F0\%9F\%93\%96\%20manual-.pdf-239dad?labelColor=fdfdfd}}}
\href{https://github.com/typst/packages/raw/main/packages/preview/babel/0.1.1/LICENSE}{\pandocbounded{\includegraphics[keepaspectratio]{https://img.shields.io/badge/licence-MIT0-239dad?labelColor=fdfdfd}}}
\href{https://codeberg.org/afiaith/babel/releases/}{\pandocbounded{\includegraphics[keepaspectratio]{https://img.shields.io/gitea/v/release/afiaith/babel?gitea_url=https\%3A\%2F\%2Fcodeberg.org&labelColor=fdfdfd&color=239dad}}}
\href{https://codeberg.org/afiaith/babel/stars}{\pandocbounded{\includegraphics[keepaspectratio]{https://img.shields.io/gitea/stars/afiaith/babel?gitea_url=https\%3A\%2F\%2Fcodeberg.org&labelColor=fdfdfd&color=fdfdfd&logo=codeberg}}}

This package provides functions that replace actual text with random
characters, which is useful for redacting confidential information or
sharing the design and structure of an existing document without
disclosing the content itself. A variety of ready-made sets of
characters for replacement are available (75 in total; termed
\emph{alphabets} ), representing diverse writing systems, codes,
notations and symbols. Some of these are more conservative (such as
emulating redaction using a wide black pen) and many are more whimsical,
as demonstrated by the following example:

\begin{Shaded}
\begin{Highlighting}[]
\NormalTok{\#baffle(alphabet: "welsh")[Hello]. My \#tippex[name] is \#baffle(alphabet: "underscore")[Inigo Montoya]. You \#baffle(alphabet: "alchemy")[killed] my \#baffle(alphabet: "shavian")[father]. Prepare to \#redact[die].}

\NormalTok{Using show rules strings, regular expressions and other selectors can be redacted automatically:}

\NormalTok{\#show "jan Maja": baffle.with(alphabet: "sitelen{-}pona")}
\NormalTok{\#show regex("[a{-}zA{-}Z0{-}9.!\#$\%\&’*+/=?\^{}\_\textasciigrave{}\{|\}\textasciitilde{}{-}]+@[a{-}zA{-}Z0{-}9{-}]+(?:\textbackslash{}.[a{-}zA{-}Z0{-}9{-}]+)*"): baffle.with(alphabet: "maze{-}3") }

\NormalTok{I’m jan Maja, and my email is \textasciigrave{}foo@digitalwords.net\textasciigrave{}.}
\end{Highlighting}
\end{Shaded}

\pandocbounded{\includegraphics[keepaspectratio]{https://github.com/typst/packages/raw/main/packages/preview/babel/0.1.1/assets/example.webp}}

\subsection{ðŸ``-- The manual}\label{uxf0uxff-the-manual}

Using { Babel } is quite straightforward. A
\href{https://github.com/typst/packages/raw/main/packages/preview/babel/0.1.1/docs/manual.pdf}{\textbf{comprehensive
manual}} covers:

\begin{itemize}
\tightlist
\item
  Introductory background.
\item
  How to use the provided functions ( \texttt{\ baffle()\ } ,
  \texttt{\ redact()\ } and \texttt{\ tippex()\ } ).
\item
  A list of the provided alphabets, each demonstrated by a line of
  random text.
\end{itemize}

If the version of the precompiled manual doesn’t match the version of
the package, it means no difference between the two versions is
reflected in the manual.

\subsection{\texorpdfstring{ðŸ---¼ The Tower of { Babel
}}{ðŸ---¼ The Tower of  Babel }}\label{uxf0uxffuxbc-the-tower-of-babel}

A poster demonstrating the provided alphabets:

\href{https://github.com/typst/packages/raw/main/packages/preview/babel/0.1.1/assets/poster.webp}{\pandocbounded{\includegraphics[keepaspectratio]{https://github.com/typst/packages/raw/main/packages/preview/babel/0.1.1/assets/poster.webp}}}

\subsection{ðŸ''¨ Complementary
tools}\label{uxf0uxff-complementary-tools}

If you wish to share the Typst source files of your document, not just
the precompiled output, a tool called
\href{https://github.com/frozolotl/typst-mutilate}{\emph{Typst
Mutilate}} might be useful for you. Unlike { Babel } , it is not a Typst
package but an external tool, written in Rust. It replaces the content
of a Typst document with random words selected from a wordlist or random
characters (similarly to { Babel } ), changing the document in place (so
make sure to run it on a \emph{copy} !). As a package for Typst, { Babel
} cannot change your source files.

\subsubsection{How to add}\label{how-to-add}

Copy this into your project and use the import as \texttt{\ babel\ }

\begin{verbatim}
#import "@preview/babel:0.1.1"
\end{verbatim}

\includesvg[width=0.16667in,height=0.16667in]{/assets/icons/16-copy.svg}

Check the docs for
\href{https://typst.app/docs/reference/scripting/\#packages}{more
information on how to import packages} .

\subsubsection{About}\label{about}

\begin{description}
\tightlist
\item[Author :]
\href{https://me.digitalwords.net}{Maja Abramski-Kronenberg}
\item[License:]
MIT-0
\item[Current version:]
0.1.1
\item[Last updated:]
October 3, 2024
\item[First released:]
October 3, 2024
\item[Minimum Typst version:]
0.11.0
\item[Archive size:]
46.9 kB
\href{https://packages.typst.org/preview/babel-0.1.1.tar.gz}{\pandocbounded{\includesvg[keepaspectratio]{/assets/icons/16-download.svg}}}
\item[Repository:]
\href{https://codeberg.org/afiaith/babel}{Codeberg}
\item[Categor ies :]
\begin{itemize}
\tightlist
\item[]
\item
  \pandocbounded{\includesvg[keepaspectratio]{/assets/icons/16-world.svg}}
  \href{https://typst.app/universe/search/?category=languages}{Languages}
\item
  \pandocbounded{\includesvg[keepaspectratio]{/assets/icons/16-text.svg}}
  \href{https://typst.app/universe/search/?category=text}{Text}
\item
  \pandocbounded{\includesvg[keepaspectratio]{/assets/icons/16-smile.svg}}
  \href{https://typst.app/universe/search/?category=fun}{Fun}
\end{itemize}
\end{description}

\subsubsection{Where to report issues?}\label{where-to-report-issues}

This package is a project of Maja Abramski-Kronenberg . Report issues on
\href{https://codeberg.org/afiaith/babel}{their repository} . You can
also try to ask for help with this package on the
\href{https://forum.typst.app}{Forum} .

Please report this package to the Typst team using the
\href{https://typst.app/contact}{contact form} if you believe it is a
safety hazard or infringes upon your rights.

\phantomsection\label{versions}
\subsubsection{Version history}\label{version-history}

\begin{longtable}[]{@{}ll@{}}
\toprule\noalign{}
Version & Release Date \\
\midrule\noalign{}
\endhead
\bottomrule\noalign{}
\endlastfoot
0.1.1 & October 3, 2024 \\
\end{longtable}

Typst GmbH did not create this package and cannot guarantee correct
functionality of this package or compatibility with any version of the
Typst compiler or app.
