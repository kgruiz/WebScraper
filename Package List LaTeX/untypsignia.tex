\title{typst.app/universe/package/untypsignia}

\phantomsection\label{banner}
\section{untypsignia}\label{untypsignia}

{ 0.1.1 }

Unofficial typesetter\textquotesingle s insignia emulations

\phantomsection\label{readme}
The \texttt{\ untypsignia\ } is a 3rd-party, unofficial, unendorsed
Typst package that exposes commands for rendering, as
\texttt{\ content\ } texts, some typesetters names in a stylized
fashion, emulating their respective \emph{insignia} , i.e., marks by
which they are known.

\subsection{Name}\label{name}

The package name is a blend of:

\begin{itemize}
\tightlist
\item
  “un�, from “unofficial�,
\item
  “typ�, from “Typst�, and
\item
  “signia�, from “insignia�, which means marks by which anything
  is known.
\end{itemize}

\subsection{Description}\label{description}

The typical use case of \texttt{\ untypsignia\ } in Typst is to emulate
a given typesetting system’s mark, if available, when referring to
them, in sentences like: “This document is typeset in \texttt{\ XYZ\ }
�, as traditionally done in \texttt{\ TeX\ } systems and derivatives
thereof.

Currently available insignia emulations include:

\begin{itemize}
\tightlist
\item
  \texttt{\ TeX\ } ,
\item
  \texttt{\ LaTeX\ } , and
\item
  \texttt{\ Typst\ } (see below)
\end{itemize}

Despite there’s no such a thing as a Typst “official� typography,
according to this post on
\href{https://discord.com/channels/1054443721975922748/1054443722592497796/1107039477714665522}{Discord}
, it can be typeset with “whatever font� the surrounding text is
being typeset. Moreover, Typst
\href{https://typst.app/legal/brand/}{branding page} requires
capitalization of the initial “T� whenever the name is used in
prose. Therefore, the “Typst� support in this package is a mere,
still unofficial, implementation of the capitalization of “Typst� in
the currently used font.

\subsection{Font Requirements}\label{font-requirements}

For the \texttt{\ TeX\ } system and it’s derivatives, the
\texttt{\ "New\ Computer\ Modern"\ } font is required.

\subsection{Usage}\label{usage}

The package exposes the following few, parameterless, functions:

\begin{itemize}
\tightlist
\item
  \texttt{\ \#texmark()\ } ,
\item
  \texttt{\ \#latexmark()\ } , and
\item
  \texttt{\ \#typstmark()\ } .
\end{itemize}

Except for the \texttt{\ \#typstmark()\ } , each such command outputs
their respective namesake signus emulation, in the document’s current
\texttt{\ text\ } settings, with the exception of font â€'' meaning text
size, color, etc… will apply to the signus emulation.

Aditionally, the signus emulation is produced, as \texttt{\ contexts\ }
text inside a \texttt{\ box\ } â€'' hence not images â€'' so as to avoid
hyphenation to take place. This also applies to the
\texttt{\ \#typstmark()\ } function, for lack of specific guidance, and
also because “Typst� is a short word.

\subsection{Example}\label{example}

\begin{Shaded}
\begin{Highlighting}[]
\NormalTok{\#set page(width: auto, height: auto, margin: 12pt, fill: rgb("19181f"))}
\NormalTok{\#set par(leading: 1.5em)}
\NormalTok{\#set text(font: "Rouge Script", fill: rgb("80f4b6"))}

\NormalTok{\#import "@preview/untypsignia:0.1.1": *}

\NormalTok{\#let say() = [I prefer \#typstmark() over \#texmark() or \#latexmark().]}

\NormalTok{\#for sz in (20, 16, 14, 12, 10, 8) \{}
\NormalTok{  set text(size: sz * 1pt)}
\NormalTok{  say()}
\NormalTok{  linebreak()}
\NormalTok{\}}
\end{Highlighting}
\end{Shaded}

This example results in a 1-page document like this one:

\pandocbounded{\includegraphics[keepaspectratio]{https://raw.githubusercontent.com/cnaak/untypsignia.typ/86b221379931edcbfa91b50159a4ff930ecbec47/thumbnail.png}}

\subsection{Citing}\label{citing}

This package can be cited with the following bibliography database
entry:

\begin{Shaded}
\begin{Highlighting}[]
\FunctionTok{untypsignia{-}package}\KeywordTok{:}
\AttributeTok{  }\FunctionTok{type}\KeywordTok{:}\AttributeTok{ Web}
\AttributeTok{  }\FunctionTok{author}\KeywordTok{:}\AttributeTok{ Naaktgeboren, C.}
\AttributeTok{  }\FunctionTok{title}\KeywordTok{:}
\AttributeTok{    }\FunctionTok{value}\KeywordTok{:}\AttributeTok{ }\StringTok{"untypsignia: unofficial typesetter\textquotesingle{}s insignia emulations"}
\AttributeTok{  }\FunctionTok{url}\KeywordTok{:}\AttributeTok{ https://github.com/cnaak/untypsignia.typ}
\AttributeTok{  }\FunctionTok{version}\KeywordTok{:}\AttributeTok{ }\FloatTok{0.1.1}
\AttributeTok{  }\FunctionTok{date}\KeywordTok{:}\AttributeTok{ 2024{-}08}
\end{Highlighting}
\end{Shaded}

\subsubsection{How to add}\label{how-to-add}

Copy this into your project and use the import as
\texttt{\ untypsignia\ }

\begin{verbatim}
#import "@preview/untypsignia:0.1.1"
\end{verbatim}

\includesvg[width=0.16667in,height=0.16667in]{/assets/icons/16-copy.svg}

Check the docs for
\href{https://typst.app/docs/reference/scripting/\#packages}{more
information on how to import packages} .

\subsubsection{About}\label{about}

\begin{description}
\tightlist
\item[Author :]
Naaktgeboren, C.
\item[License:]
MIT
\item[Current version:]
0.1.1
\item[Last updated:]
August 21, 2024
\item[First released:]
August 14, 2024
\item[Minimum Typst version:]
0.11.1
\item[Archive size:]
2.13 kB
\href{https://packages.typst.org/preview/untypsignia-0.1.1.tar.gz}{\pandocbounded{\includesvg[keepaspectratio]{/assets/icons/16-download.svg}}}
\item[Discipline :]
\begin{itemize}
\tightlist
\item[]
\item
  \href{https://typst.app/universe/search/?discipline=computer-science}{Computer
  Science}
\end{itemize}
\item[Categor ies :]
\begin{itemize}
\tightlist
\item[]
\item
  \pandocbounded{\includesvg[keepaspectratio]{/assets/icons/16-chart.svg}}
  \href{https://typst.app/universe/search/?category=visualization}{Visualization}
\item
  \pandocbounded{\includesvg[keepaspectratio]{/assets/icons/16-hammer.svg}}
  \href{https://typst.app/universe/search/?category=utility}{Utility}
\item
  \pandocbounded{\includesvg[keepaspectratio]{/assets/icons/16-smile.svg}}
  \href{https://typst.app/universe/search/?category=fun}{Fun}
\end{itemize}
\end{description}

\subsubsection{Where to report issues?}\label{where-to-report-issues}

This package is a project of Naaktgeboren, C. . You can also try to ask
for help with this package on the \href{https://forum.typst.app}{Forum}
.

Please report this package to the Typst team using the
\href{https://typst.app/contact}{contact form} if you believe it is a
safety hazard or infringes upon your rights.

\phantomsection\label{versions}
\subsubsection{Version history}\label{version-history}

\begin{longtable}[]{@{}ll@{}}
\toprule\noalign{}
Version & Release Date \\
\midrule\noalign{}
\endhead
\bottomrule\noalign{}
\endlastfoot
0.1.1 & August 21, 2024 \\
\href{https://typst.app/universe/package/untypsignia/0.1.0/}{0.1.0} &
August 14, 2024 \\
\end{longtable}

Typst GmbH did not create this package and cannot guarantee correct
functionality of this package or compatibility with any version of the
Typst compiler or app.
