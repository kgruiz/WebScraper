\title{typst.app/universe/package/oasis-align}

\phantomsection\label{banner}
\section{oasis-align}\label{oasis-align}

{ 0.1.0 }

Cleanly place content side by side with equal heights using automatic
content sizing.

\phantomsection\label{readme}
\texttt{\ oasis-align\ } is a package that automatically sizes your
content so that their heights are equal, allowing you to cleanly place
content side by side.

To use \texttt{\ oasis-align\ } in your document, start by importing the
package like this:

\begin{Shaded}
\begin{Highlighting}[]
\NormalTok{\#import "@preview/oasis{-}align:0.1.0": *}
\end{Highlighting}
\end{Shaded}

This will give you access to the two functions found under
\href{https://github.com/typst/packages/raw/main/packages/preview/oasis-align/0.1.0/\#configuration}{configurations}
.

\subsection{Image with Text}\label{image-with-text}

\pandocbounded{\includegraphics[keepaspectratio]{https://github.com/typst/packages/raw/main/packages/preview/oasis-align/0.1.0/examples/image-with-text.gif}}

\subsection{Image with Image}\label{image-with-image}

\pandocbounded{\includegraphics[keepaspectratio]{https://github.com/typst/packages/raw/main/packages/preview/oasis-align/0.1.0/examples/image-with-image.gif}}

\subsection{Text with Text}\label{text-with-text}

\pandocbounded{\includegraphics[keepaspectratio]{https://github.com/typst/packages/raw/main/packages/preview/oasis-align/0.1.0/examples/text-with-text.gif}}

There are two functions associated with this package. The first is
specifically targeted at
\href{https://github.com/typst/packages/raw/main/packages/preview/oasis-align/0.1.0/\#oasis-align-images}{aligning
images} , and the second is targeted at
\href{https://github.com/typst/packages/raw/main/packages/preview/oasis-align/0.1.0/\#oasis-align-1}{content
in general} .

\begin{quote}
{[}!important{]} To change the size of the gutter in both functions, use
\texttt{\ \#set\ grid(column-gutter:\ length)\ } . This is necessary to
allow for fixed rules that aren’t possible with user-defined
functions.
\end{quote}

\subsection{\texorpdfstring{\texttt{\ oasis-align-images\ }}{ oasis-align-images }}\label{oasis-align-images}

Use this function to align two images.

\begin{Shaded}
\begin{Highlighting}[]
\NormalTok{\#oasis{-}align{-}images(}
\NormalTok{    "path/to/image1",}
\NormalTok{    "path/to/image2"}
\NormalTok{)}
\end{Highlighting}
\end{Shaded}

\begin{quote}
{[}!tip{]} Whenever aligning \textbf{only} images, it’s best to use
this function instead of the default \texttt{\ oasis-align\ } . \emph{To
learn more about why, check out
\href{https://github.com/typst/packages/raw/main/packages/preview/oasis-align/0.1.0/\#how-it-works}{how
it works} .}
\end{quote}

\subsection{\texorpdfstring{\texttt{\ oasis-align\ }}{ oasis-align }}\label{oasis-align-1}

Use this function to align content like text with other content like
images or figures.

\begin{quote}
{[}!tip{]} The parameters with defined values are the defaults and do
not need to be included unless desired.
\end{quote}

\begin{Shaded}
\begin{Highlighting}[]
\NormalTok{\#oasis{-}align(}
\NormalTok{  int{-}frac: 0.5,        // decimal between 0 and 1}
\NormalTok{  tolerance: 0.001pt,   // length}
\NormalTok{  max{-}iterations: 50,   // integer greater than 0}
\NormalTok{  int{-}dir: 1,           // 1 or {-}1}
\NormalTok{  debug: false          // boolean}
\NormalTok{  item1,                // content}
\NormalTok{  item2,                // content}
\NormalTok{)}
\end{Highlighting}
\end{Shaded}

\subsubsection{\texorpdfstring{\texttt{\ int-frac\ }}{ int-frac }}\label{int-frac}

The starting point of the search process. Changing this value may reduce
the total number of iterations of the function or find an
\href{https://github.com/typst/packages/raw/main/packages/preview/oasis-align/0.1.0/\#oasis-align-2}{alternate
solution} .

\subsubsection{\texorpdfstring{\texttt{\ tolerance\ }}{ tolerance }}\label{tolerance}

The allowable difference in heights between \texttt{\ item1\ } and
\texttt{\ item2\ } . The function will run until it has reached either
this \texttt{\ tolerance\ } or \texttt{\ max-iterations\ } . Making
\texttt{\ tolerance\ } larger may reduce the total number of iterations
but result in a larger height difference between pieces of content.

\begin{quote}
{[}!note{]} Two pieces of content may not always be able to achieve the
desired \texttt{\ tolerance\ } . In that case, the function sizes the
content to the iteration that had the least difference in height.
\emph{Check out
\href{https://github.com/typst/packages/raw/main/packages/preview/oasis-align/0.1.0/\#oasis-align-2}{how
it works} to understand why the function may not be able achieve the
desired \texttt{\ tolerance\ } .}
\end{quote}

\subsubsection{\texorpdfstring{\texttt{\ max-iterations\ }}{ max-iterations }}\label{max-iterations}

The maximum number of iterations the function is allowed to attempt
before terminating. Increasing this number may allow you to achieve a
smaller \texttt{\ tolerance\ } .

\subsubsection{\texorpdfstring{\texttt{\ int-dir\ }}{ int-dir }}\label{int-dir}

The initial direction that the dividing fraction is moved. Changing this
value will change the initial direction.

\begin{quote}
{[}!note{]} The program is hardcoded to switch directions if a solution
is not found in the initial direction. This parameter mainly serves to
let you easily choose between
\href{https://github.com/typst/packages/raw/main/packages/preview/oasis-align/0.1.0/\#oasis-align-2}{multiple
solutions} .
\end{quote}

\subsubsection{\texorpdfstring{\texttt{\ debug\ }}{ debug }}\label{debug}

A toggle that lets you look inside the function to see what is
happening. This is useful if you would like to understand why certain
content may be incompatible and which of the parameters above could be
changed to resolve the issue.

\subsection{\texorpdfstring{\texttt{\ oasis-align-images\ }}{ oasis-align-images }}\label{oasis-align-images-1}

The function begins by determining the width and height of the selected
images. These values can then be used to solve a set of linear
equations, the first of which states that the sum of the widths of the
images (plus the gutter) should be equal to the available horizontal
space, and the second which states that their heights should be equal.

If \$w\_1\$ and \$h\_1\$ are the width and height of \texttt{\ image1\ }
and \$w\_2\$ and \$h\_2\$ are the width and height of
\texttt{\ image2\ } , then the final width \$w\_1’\$ of
\texttt{\ image1\ } and the final width \$w\_2’\$ of
\texttt{\ image2\ } are

\$\$w\_1’ = \textbackslash left(\textbackslash frac\{h\_1 w\_2\}\{w\_1
h\_2\} + 1 \textbackslash right)\^{}\{-1\} \textbackslash qquad w\_2’
= \textbackslash left(\textbackslash frac\{w\_1 h\_2\}\{h\_1 w\_2\} + 1
\textbackslash right)\^{}\{-1\}\$\$

\subsection{\texorpdfstring{\texttt{\ oasis-align\ }}{ oasis-align }}\label{oasis-align-2}

Originally designed to allow for an image to be placed side-by-side with
text, this function takes an iterative approach to aligning the content.
When changing the width of a block of text, the height does not scale
linearly, but instead behaves as a step function that follows an
exponential trend (the graph below has a simplified visualization of
this). This prevents the use of an analytical methodology similar to the
one used in \texttt{\ oasis-align-images\ } , and thus must be solved
using an iterative approach.

The function starts by taking the available space and then splitting it
using the \texttt{\ int-frac\ } . The content is then placed in a block
with the width as determined using the split from \texttt{\ int-frac\ }
before measuring its height. Based on the \texttt{\ int-dir\ } , the
split will be moved left or right using the bisection method until a
solution within the \texttt{\ tolerance\ } has been found. In the case
that a solution within the \texttt{\ tolerance\ } is not found with the
\texttt{\ max-iterations\ } , the program terminates and uses the
container width fraction that had the smallest difference in height.

\pandocbounded{\includesvg[keepaspectratio]{https://github.com/typst/packages/raw/main/packages/preview/oasis-align/0.1.0/examples/graph-visualization.svg}}

\subsubsection{Multiple Solutions (1st
Graph)}\label{multiple-solutions-1st-graph}

Depending on the type of content, the function may find multiple
solutions. The parameters \texttt{\ int-dir\ } and \texttt{\ int-frac\ }
will allow you to choose between them by changing the direction in which
it iterates and changing the starting container width fraction
respectively.

\subsubsection{No Solutions (2nd Graph)}\label{no-solutions-2nd-graph}

There are cases in which the text size is incompatible with an image.
This can be because there is not enough or too much text, and regardless
of how the content is resized, their heights do not match.

\subsubsection{Tolerance Not Reached (3rd
Graph)}\label{tolerance-not-reached-3rd-graph}

In the case of having texts of different sizes (as seen in
\href{https://github.com/typst/packages/raw/main/packages/preview/oasis-align/0.1.0/\#text-with-text}{the
examples} ), the spacing between lines prevents the function from
finding a solution that meets the \texttt{\ tolerance\ } , and thus the
closest solution is used.

If you end up using this package, please feel free to share how you used
it under “Discussions� on the
\href{https://github.com/jdpieck/oasis-align}{GitHub Repository} or on
Discord with \texttt{\ @jdpieck\ } .

\subsubsection{How to add}\label{how-to-add}

Copy this into your project and use the import as
\texttt{\ oasis-align\ }

\begin{verbatim}
#import "@preview/oasis-align:0.1.0"
\end{verbatim}

\includesvg[width=0.16667in,height=0.16667in]{/assets/icons/16-copy.svg}

Check the docs for
\href{https://typst.app/docs/reference/scripting/\#packages}{more
information on how to import packages} .

\subsubsection{About}\label{about}

\begin{description}
\tightlist
\item[Author :]
@jdpieck
\item[License:]
MIT
\item[Current version:]
0.1.0
\item[Last updated:]
September 2, 2024
\item[First released:]
September 2, 2024
\item[Archive size:]
4.96 kB
\href{https://packages.typst.org/preview/oasis-align-0.1.0.tar.gz}{\pandocbounded{\includesvg[keepaspectratio]{/assets/icons/16-download.svg}}}
\item[Repository:]
\href{https://github.com/jdpieck/oasis-align}{GitHub}
\item[Categor y :]
\begin{itemize}
\tightlist
\item[]
\item
  \pandocbounded{\includesvg[keepaspectratio]{/assets/icons/16-layout.svg}}
  \href{https://typst.app/universe/search/?category=layout}{Layout}
\end{itemize}
\end{description}

\subsubsection{Where to report issues?}\label{where-to-report-issues}

This package is a project of @jdpieck . Report issues on
\href{https://github.com/jdpieck/oasis-align}{their repository} . You
can also try to ask for help with this package on the
\href{https://forum.typst.app}{Forum} .

Please report this package to the Typst team using the
\href{https://typst.app/contact}{contact form} if you believe it is a
safety hazard or infringes upon your rights.

\phantomsection\label{versions}
\subsubsection{Version history}\label{version-history}

\begin{longtable}[]{@{}ll@{}}
\toprule\noalign{}
Version & Release Date \\
\midrule\noalign{}
\endhead
\bottomrule\noalign{}
\endlastfoot
0.1.0 & September 2, 2024 \\
\end{longtable}

Typst GmbH did not create this package and cannot guarantee correct
functionality of this package or compatibility with any version of the
Typst compiler or app.
