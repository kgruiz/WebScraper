\title{typst.app/universe/package/suiji}

\phantomsection\label{banner}
\section{suiji}\label{suiji}

{ 0.3.0 }

A highly efficient random number generator for Typst

{ } Featured Package

\phantomsection\label{readme}
\href{https://github.com/liuguangxi/suiji}{Suiji} (�机 in Chinese,
/suíjī/, meaning random) is a high efficient random number generator in
Typst. Partial algorithm is inherited from
\href{https://www.gnu.org/software/gsl}{GSL} and most APIs are similar
to
\href{https://numpy.org/doc/stable/reference/random/generator.html}{NumPy
Random Generator} . It provides pure function implementation and does
not rely on any global state variables, resulting in better performance
and independency.

\subsection{Features}\label{features}

\begin{itemize}
\tightlist
\item
  All functions are immutable, which means results of random are
  completely deterministic.
\item
  Core random engine chooses “Maximally equidistributed combined
  Tausworthe generator� and “LCG�.
\item
  Generate random integers or floats from various distribution.
\item
  Randomly shuffle an array of objects.
\item
  Randomly sample from an array of objects.
\item
  Generate blind text of Simplified Chinese.
\end{itemize}

\subsection{Examples}\label{examples}

The example below uses \texttt{\ suiji\ } and \texttt{\ cetz\ } packages
to create a trajectory of a random walk.

\begin{Shaded}
\begin{Highlighting}[]
\NormalTok{\#import "@preview/suiji:0.3.0": *}
\NormalTok{\#import "@preview/cetz:0.2.2"}

\NormalTok{\#set page(width: auto, height: auto, margin: 0.5cm)}

\NormalTok{\#cetz.canvas(length: 5pt, \{}
\NormalTok{  import cetz.draw: *}

\NormalTok{  let n = 2000}
\NormalTok{  let (x, y) = (0, 0)}
\NormalTok{  let (x{-}new, y{-}new) = (0, 0)}
\NormalTok{  let rng = gen{-}rng(42)}
\NormalTok{  let v = ()}

\NormalTok{  for i in range(n) \{}
\NormalTok{    (rng, v) = uniform(rng, low: {-}2.0, high: 2.0, size: 2)}
\NormalTok{    (x{-}new, y{-}new) = (x {-} v.at(1), y {-} v.at(0))}
\NormalTok{    let col = color.mix((blue.transparentize(20\%), 1{-}i/n), (green.transparentize(20\%), i/n))}
\NormalTok{    line(stroke: (paint: col, cap: "round", thickness: 2pt),}
\NormalTok{      (x, y), (x{-}new, y{-}new)}
\NormalTok{    )}
\NormalTok{    (x, y) = (x{-}new, y{-}new)}
\NormalTok{  \}}
\NormalTok{\})}
\end{Highlighting}
\end{Shaded}

\pandocbounded{\includegraphics[keepaspectratio]{https://github.com/typst/packages/raw/main/packages/preview/suiji/0.3.0/examples/random-walk.png}}

Another example is drawing the the famous \textbf{Matrix} rain effect of
falling green characters in a terminal.

\begin{Shaded}
\begin{Highlighting}[]
\NormalTok{\#import "@preview/suiji:0.3.0": *}
\NormalTok{\#import "@preview/cetz:0.2.2"}

\NormalTok{\#set page(width: auto, height: auto, margin: 0pt)}

\NormalTok{\#cetz.canvas(length: 1pt, \{}
\NormalTok{  import cetz.draw: *}

\NormalTok{  let font{-}size = 10}
\NormalTok{  let num{-}col = 80}
\NormalTok{  let num{-}row = 32}
\NormalTok{  let text{-}len = 16}
\NormalTok{  let seq = "abcdefghijklmnopqrstuvwxyz!@\#$\%\^{}\&*".split("").slice(1, 35).map(it =\textgreater{} raw(it))}
\NormalTok{  let rng = gen{-}rng(42)}
\NormalTok{  let num{-}cnt = 0}
\NormalTok{  let val = 0}
\NormalTok{  let chars = ()}

\NormalTok{  rect(({-}10, {-}10), (font{-}size * (num{-}col {-} 1) * 0.6 + 10, font{-}size * (num{-}row {-} 1) + 10), fill: black)}

\NormalTok{  for c in range(num{-}col) \{}
\NormalTok{    (rng, num{-}cnt) = integers(rng, low: 1, high: 3)}
\NormalTok{    for cnt in range(num{-}cnt) \{}
\NormalTok{      (rng, val) = integers(rng, low: {-}10, high: num{-}row {-} 2)}
\NormalTok{      (rng, chars) = choice(rng, seq, size: text{-}len)}
\NormalTok{      for i in range(text{-}len) \{}
\NormalTok{        let y = i + val}
\NormalTok{        if y \textgreater{}= 0 and y \textless{} num{-}row \{}
\NormalTok{          let col = green.transparentize((i / text{-}len) * 100\%)}
\NormalTok{          content(}
\NormalTok{            (c * font{-}size * 0.6, y * font{-}size),}
\NormalTok{            text(size: font{-}size * 1pt, fill:col, stroke: (text{-}len {-} i) * 0.04pt + col, chars.at(i))}
\NormalTok{          )}
\NormalTok{        \}}
\NormalTok{      \}}
\NormalTok{    \}}
\NormalTok{  \}}
\NormalTok{\})}
\end{Highlighting}
\end{Shaded}

\pandocbounded{\includegraphics[keepaspectratio]{https://github.com/typst/packages/raw/main/packages/preview/suiji/0.3.0/examples/matrix-rain.png}}

\subsection{Usage}\label{usage}

Import \texttt{\ suiji\ } module first before use any random functions
from it.

\begin{Shaded}
\begin{Highlighting}[]
\NormalTok{\#import "@preview/suiji:0.3.0": *}
\end{Highlighting}
\end{Shaded}

For functions that generate various random numbers or randomly shuffle,
a random number generator object ( \textbf{rng} ) is required as both
input and output arguments. And the original \textbf{rng} should be
created by function \texttt{\ gen-rng\ } , with an integer as the
argument of seed. This calling style seems to be a little inconvenient,
as it is limited by the programming paradigm. For function
\texttt{\ discrete\ } , the given probalilities of the discrete events
should be preprocessed by function \texttt{\ discrete-preproc\ } , whose
output serves as an input argument of \texttt{\ discrete\ } .

Another set of functions with the same functionality provides higher
performance (about 3 times faster) and has the suffix \texttt{\ -f\ } in
their names. For example, \texttt{\ gen-rng-f\ } and
\texttt{\ integers-f\ } are the fast versions of \texttt{\ gen-rng\ }
and \texttt{\ integers\ } , respectively.

The function \texttt{\ rand-sc\ } creates blind text of Simplified
Chinese. This function yields a Chinese-like Lorem Ipsum blind text with
the given number of words, where punctuations are optional.

The code below generates several random permutations of 0 to 9. Each
time after function \texttt{\ shuffle-f\ } is called, the value of
variable \texttt{\ rng\ } is updated, so generated permutations are
different.

\begin{Shaded}
\begin{Highlighting}[]
\NormalTok{\#\{}
\NormalTok{  let rng = gen{-}rng{-}f(42)}
\NormalTok{  let a = ()}
\NormalTok{  for i in range(5) \{}
\NormalTok{    (rng, a) = shuffle{-}f(rng, range(10))}
\NormalTok{    [\#(a.map(it =\textgreater{} str(it)).join("  ")) \textbackslash{} ]}
\NormalTok{  \}}
\NormalTok{\}}
\end{Highlighting}
\end{Shaded}

\pandocbounded{\includegraphics[keepaspectratio]{https://github.com/typst/packages/raw/main/packages/preview/suiji/0.3.0/examples/random-permutation.png}}

For more codes with these functions see
\href{https://github.com/typst/packages/raw/main/packages/preview/suiji/0.3.0/tests}{tests}
.

\subsection{Reference}\label{reference}

\subsubsection{\texorpdfstring{\texttt{\ gen-rng\ } /
\texttt{\ gen-rng-f\ }}{ gen-rng  /  gen-rng-f }}\label{gen-rng-gen-rng-f}

Construct a new random number generator with a seed.

\begin{Shaded}
\begin{Highlighting}[]
\NormalTok{\#let gen{-}rng(seed) = \{...\}}
\end{Highlighting}
\end{Shaded}

\begin{itemize}
\item
  \textbf{Input Arguments}

  \begin{itemize}
  \tightlist
  \item
    \texttt{\ seed\ } : {[} \texttt{\ int\ } {]} value of seed.
  \end{itemize}
\item
  \textbf{Output Arguments}

  \begin{itemize}
  \tightlist
  \item
    \texttt{\ rng\ } : {[} \texttt{\ object\ } {]} generated object of
    random number generator.
  \end{itemize}
\end{itemize}

\subsubsection{\texorpdfstring{\texttt{\ randi-f\ }}{ randi-f }}\label{randi-f}

Return a raw random integer from {[}0, 2\^{}31).

\begin{Shaded}
\begin{Highlighting}[]
\NormalTok{\#let randi{-}f(rng) = \{...\}}
\end{Highlighting}
\end{Shaded}

\begin{itemize}
\item
  \textbf{Input Arguments}

  \begin{itemize}
  \tightlist
  \item
    \texttt{\ rng\ } : {[} \texttt{\ object\ } \textbar{}
    \texttt{\ int\ } {]} object of random number generator (generated by
    function \texttt{\ *-f\ } ).
  \end{itemize}
\item
  \textbf{Output Arguments}

  \begin{itemize}
  \tightlist
  \item
    \texttt{\ rng-out\ } : {[} \texttt{\ object\ } \textbar{}
    \texttt{\ int\ } {]} updated object of random number generator
    (random integer from the interval {[}0, 2\^{}31-1{]}).
  \end{itemize}
\end{itemize}

\subsubsection{\texorpdfstring{\texttt{\ integers\ } /
\texttt{\ integers-f\ }}{ integers  /  integers-f }}\label{integers-integers-f}

Return random integers from \texttt{\ low\ } (inclusive) to
\texttt{\ high\ } (exclusive).

\begin{Shaded}
\begin{Highlighting}[]
\NormalTok{\#let integers(rng, low: 0, high: 100, size: none, endpoint: false) = \{...\}}
\end{Highlighting}
\end{Shaded}

\begin{itemize}
\item
  \textbf{Input Arguments}

  \begin{itemize}
  \tightlist
  \item
    \texttt{\ rng\ } : {[} \texttt{\ object\ } {]} object of random
    number generator.
  \item
    \texttt{\ low\ } : {[} \texttt{\ int\ } {]} lowest (signed) integers
    to be drawn from the distribution, optional.
  \item
    \texttt{\ high\ } : {[} \texttt{\ int\ } {]} one above the largest
    (signed) integer to be drawn from the distribution, optional.
  \item
    \texttt{\ size\ } : {[} \texttt{\ none\ } or \texttt{\ int\ } {]}
    returned array size, must be none or non-negative integer, optional.
  \item
    \texttt{\ endpoint\ } : {[} \texttt{\ bool\ } {]} if true, sample
    from the interval {[} \texttt{\ low\ } , \texttt{\ high\ } {]}
    instead of the default {[} \texttt{\ low\ } , \texttt{\ high\ } ),
    optional.
  \end{itemize}
\item
  \textbf{Output Arguments}

  \begin{itemize}
  \tightlist
  \item
    {[} \texttt{\ array\ } {]} : ( \texttt{\ rng-out\ } ,
    \texttt{\ arr-out\ } )

    \begin{itemize}
    \tightlist
    \item
      \texttt{\ rng-out\ } : {[} \texttt{\ object\ } {]} updated object
      of random number generator.
    \item
      \texttt{\ arr-out\ } : {[} \texttt{\ int\ } \textbar{}
      \texttt{\ array\ } of \texttt{\ int\ } {]} array of random
      numbers.
    \end{itemize}
  \end{itemize}
\end{itemize}

\subsubsection{\texorpdfstring{\texttt{\ random\ } /
\texttt{\ random-f\ }}{ random  /  random-f }}\label{random-random-f}

Return random floats in the half-open interval {[}0.0, 1.0).

\begin{Shaded}
\begin{Highlighting}[]
\NormalTok{\#let random(rng, size: none) = \{...\}}
\end{Highlighting}
\end{Shaded}

\begin{itemize}
\item
  \textbf{Input Arguments}

  \begin{itemize}
  \tightlist
  \item
    \texttt{\ rng\ } : {[} \texttt{\ object\ } {]} object of random
    number generator.
  \item
    \texttt{\ size\ } : {[} \texttt{\ none\ } or \texttt{\ int\ } {]}
    returned array size, must be none or non-negative integer, optional.
  \end{itemize}
\item
  \textbf{Output Arguments}

  \begin{itemize}
  \tightlist
  \item
    {[} \texttt{\ array\ } {]} : ( \texttt{\ rng-out\ } ,
    \texttt{\ arr-out\ } )

    \begin{itemize}
    \tightlist
    \item
      \texttt{\ rng-out\ } : {[} \texttt{\ object\ } {]} updated object
      of random number generator.
    \item
      \texttt{\ arr-out\ } : {[} \texttt{\ float\ } \textbar{}
      \texttt{\ array\ } of \texttt{\ float\ } {]} array of random
      numbers.
    \end{itemize}
  \end{itemize}
\end{itemize}

\subsubsection{\texorpdfstring{\texttt{\ uniform\ } /
\texttt{\ uniform-f\ }}{ uniform  /  uniform-f }}\label{uniform-uniform-f}

Draw samples from a uniform distribution. Samples are uniformly
distributed over the half-open interval {[} \texttt{\ low\ } ,
\texttt{\ high\ } ) (includes \texttt{\ low\ } , but excludes
\texttt{\ high\ } ).

\begin{Shaded}
\begin{Highlighting}[]
\NormalTok{\#let uniform(rng, low: 0.0, high: 1.0, size: none) = \{...\}}
\end{Highlighting}
\end{Shaded}

\begin{itemize}
\item
  \textbf{Input Arguments}

  \begin{itemize}
  \tightlist
  \item
    \texttt{\ rng\ } : {[} \texttt{\ object\ } {]} object of random
    number generator.
  \item
    \texttt{\ low\ } : {[} \texttt{\ float\ } {]} lower boundary of the
    output interval, optional.
  \item
    \texttt{\ high\ } : {[} \texttt{\ float\ } {]} upper boundary of the
    output interval, optional.
  \item
    \texttt{\ size\ } : {[} \texttt{\ none\ } or \texttt{\ int\ } {]}
    returned array size, must be none or non-negative integer, optional.
  \end{itemize}
\item
  \textbf{Output Arguments}

  \begin{itemize}
  \tightlist
  \item
    {[} \texttt{\ array\ } {]} : ( \texttt{\ rng-out\ } ,
    \texttt{\ arr-out\ } )

    \begin{itemize}
    \tightlist
    \item
      \texttt{\ rng-out\ } : {[} \texttt{\ object\ } {]} updated object
      of random number generator.
    \item
      \texttt{\ arr-out\ } : {[} \texttt{\ float\ } \textbar{}
      \texttt{\ array\ } of \texttt{\ float\ } {]} array of random
      numbers.
    \end{itemize}
  \end{itemize}
\end{itemize}

\subsubsection{\texorpdfstring{\texttt{\ normal\ } /
\texttt{\ normal-f\ }}{ normal  /  normal-f }}\label{normal-normal-f}

Draw random samples from a normal (Gaussian) distribution.

\begin{Shaded}
\begin{Highlighting}[]
\NormalTok{\#let normal(rng, loc: 0.0, scale: 1.0, size: none) = \{...\}}
\end{Highlighting}
\end{Shaded}

\begin{itemize}
\item
  \textbf{Input Arguments}

  \begin{itemize}
  \tightlist
  \item
    \texttt{\ rng\ } : {[} \texttt{\ object\ } {]} object of random
    number generator.
  \item
    \texttt{\ loc\ } : {[} \texttt{\ float\ } {]} mean (centre) of the
    distribution, optional.
  \item
    \texttt{\ scale\ } : {[} \texttt{\ float\ } {]} standard deviation
    (spread or width) of the distribution, must be non-negative,
    optional.
  \item
    \texttt{\ size\ } : {[} \texttt{\ none\ } or \texttt{\ int\ } {]}
    returned array size, must be none or non-negative integer, optional.
  \end{itemize}
\item
  \textbf{Output Arguments}

  \begin{itemize}
  \tightlist
  \item
    {[} \texttt{\ array\ } {]} : ( \texttt{\ rng-out\ } ,
    \texttt{\ arr-out\ } )

    \begin{itemize}
    \tightlist
    \item
      \texttt{\ rng-out\ } : {[} \texttt{\ object\ } {]} updated object
      of random number generator.
    \item
      \texttt{\ arr-out\ } : {[} \texttt{\ float\ } \textbar{}
      \texttt{\ array\ } of \texttt{\ float\ } {]} array of random
      numbers.
    \end{itemize}
  \end{itemize}
\end{itemize}

\subsubsection{\texorpdfstring{\texttt{\ discrete-preproc\ } and
\texttt{\ discrete\ } / \texttt{\ discrete-preproc-f\ } and
\texttt{\ discrete-f\ }}{ discrete-preproc  and  discrete  /  discrete-preproc-f  and  discrete-f }}\label{discrete-preproc-and-discrete-discrete-preproc-f-and-discrete-f}

Return random indices from the given probalilities of the discrete
events.

\begin{Shaded}
\begin{Highlighting}[]
\NormalTok{\#let discrete{-}preproc(p) = \{...\}}
\end{Highlighting}
\end{Shaded}

\begin{itemize}
\item
  \textbf{Input Arguments}

  \begin{itemize}
  \tightlist
  \item
    \texttt{\ p\ } : {[} \texttt{\ array\ } of \texttt{\ int\ } or
    \texttt{\ float\ } {]} the array of probalilities of the discrete
    events, probalilities must be non-negative.
  \end{itemize}
\item
  \textbf{Output Arguments}

  \begin{itemize}
  \tightlist
  \item
    \texttt{\ g\ } : {[} \texttt{\ object\ } {]} generated object that
    contains the lookup table.
  \end{itemize}
\end{itemize}

\begin{Shaded}
\begin{Highlighting}[]
\NormalTok{\#let discrete(rng, g, size: none) = \{...\}}
\end{Highlighting}
\end{Shaded}

\begin{itemize}
\item
  \textbf{Input Arguments}

  \begin{itemize}
  \tightlist
  \item
    \texttt{\ rng\ } : {[} \texttt{\ object\ } {]} object of random
    number generator.
  \item
    \texttt{\ g\ } : {[} \texttt{\ object\ } {]} generated object that
    contains the lookup table by \texttt{\ discrete-preproc\ } function.
  \item
    \texttt{\ size\ } : {[} \texttt{\ none\ } or \texttt{\ int\ } {]}
    returned array size, must be none or non-negative integer, optional.
  \end{itemize}
\item
  \textbf{Output Arguments}

  \begin{itemize}
  \tightlist
  \item
    {[} \texttt{\ array\ } {]} : ( \texttt{\ rng-out\ } ,
    \texttt{\ arr-out\ } )

    \begin{itemize}
    \tightlist
    \item
      \texttt{\ rng-out\ } : {[} \texttt{\ object\ } {]} updated object
      of random number generator.
    \item
      \texttt{\ arr-out\ } : {[} \texttt{\ int\ } \textbar{}
      \texttt{\ array\ } of \texttt{\ int\ } {]} array of random
      indices.
    \end{itemize}
  \end{itemize}
\end{itemize}

\subsubsection{\texorpdfstring{\texttt{\ shuffle\ } /
\texttt{\ shuffle-f\ }}{ shuffle  /  shuffle-f }}\label{shuffle-shuffle-f}

Randomly shuffle a given array.

\begin{Shaded}
\begin{Highlighting}[]
\NormalTok{\#let shuffle(rng, arr) = \{...\}}
\end{Highlighting}
\end{Shaded}

\begin{itemize}
\item
  \textbf{Input Arguments}

  \begin{itemize}
  \tightlist
  \item
    \texttt{\ rng\ } : {[} \texttt{\ object\ } {]} object of random
    number generator.
  \item
    \texttt{\ arr\ } : {[} \texttt{\ array\ } {]} the array to be
    shuffled.
  \end{itemize}
\item
  \textbf{Output Arguments}

  \begin{itemize}
  \tightlist
  \item
    {[} \texttt{\ array\ } {]} : ( \texttt{\ rng-out\ } ,
    \texttt{\ arr-out\ } )

    \begin{itemize}
    \tightlist
    \item
      \texttt{\ rng-out\ } : {[} \texttt{\ object\ } {]} updated object
      of random number generator.
    \item
      \texttt{\ arr-out\ } : {[} \texttt{\ array\ } {]} shuffled array.
    \end{itemize}
  \end{itemize}
\end{itemize}

\subsubsection{\texorpdfstring{\texttt{\ choice\ } /
\texttt{\ choice-f\ }}{ choice  /  choice-f }}\label{choice-choice-f}

Generate random samples from a given array.

\begin{Shaded}
\begin{Highlighting}[]
\NormalTok{\#let choice(rng, arr, size: none, replacement: true, permutation: true) = \{...\}}
\end{Highlighting}
\end{Shaded}

\begin{itemize}
\item
  \textbf{Input Arguments}

  \begin{itemize}
  \tightlist
  \item
    \texttt{\ rng\ } : {[} \texttt{\ object\ } {]} object of random
    number generator.
  \item
    \texttt{\ arr\ } : {[} \texttt{\ array\ } {]} the array to be
    sampled.
  \item
    \texttt{\ size\ } : {[} \texttt{\ none\ } or \texttt{\ int\ } {]}
    returned array size, must be none or non-negative integer, optional.
  \item
    \texttt{\ replacement\ } : {[} \texttt{\ bool\ } {]} whether the
    sample is with or without replacement, optional; default is true,
    meaning that a value of \texttt{\ arr\ } can be selected multiple
    times.
  \item
    \texttt{\ permutation\ } : {[} \texttt{\ bool\ } {]} whether the
    sample is permuted when sampling without replacement, optional;
    default is true, false provides a speedup.
  \end{itemize}
\item
  \textbf{Output Arguments}

  \begin{itemize}
  \tightlist
  \item
    {[} \texttt{\ array\ } {]} : ( \texttt{\ rng-out\ } ,
    \texttt{\ arr-out\ } )

    \begin{itemize}
    \tightlist
    \item
      \texttt{\ rng-out\ } : {[} \texttt{\ object\ } {]} updated object
      of random number generator.
    \item
      \texttt{\ arr-out\ } : {[} \texttt{\ array\ } {]} generated random
      samples.
    \end{itemize}
  \end{itemize}
\end{itemize}

\subsubsection{\texorpdfstring{\texttt{\ rand-sc\ }}{ rand-sc }}\label{rand-sc}

Generate blind text of Simplified Chinese.

\begin{Shaded}
\begin{Highlighting}[]
\NormalTok{\#let rand{-}sc(words, seed: 42, punctuation: false, gap: 10) = \{...\}}
\end{Highlighting}
\end{Shaded}

\begin{itemize}
\item
  \textbf{Input Arguments}

  \begin{itemize}
  \tightlist
  \item
    \texttt{\ words\ } : {[} \texttt{\ int\ } {]} the length of the
    blind text in pure words.
  \item
    \texttt{\ seed\ } : {[} \texttt{\ int\ } {]} value of seed,
    optional.
  \item
    \texttt{\ punctuation\ } : {[} \texttt{\ bool\ } {]} if true, insert
    punctuations in generated words, optional.
  \item
    \texttt{\ gap\ } : {[} \texttt{\ int\ } {]} average gap between
    punctuations, optional.
  \end{itemize}
\item
  \textbf{Output Arguments}

  \begin{itemize}
  \tightlist
  \item
    {[} \texttt{\ str\ } {]} : generated blind text of Simplified
    Chinese.
  \end{itemize}
\end{itemize}

\subsubsection{How to add}\label{how-to-add}

Copy this into your project and use the import as \texttt{\ suiji\ }

\begin{verbatim}
#import "@preview/suiji:0.3.0"
\end{verbatim}

\includesvg[width=0.16667in,height=0.16667in]{/assets/icons/16-copy.svg}

Check the docs for
\href{https://typst.app/docs/reference/scripting/\#packages}{more
information on how to import packages} .

\subsubsection{About}\label{about}

\begin{description}
\tightlist
\item[Author :]
\href{https://github.com/liuguangxi}{Guangxi Liu}
\item[License:]
MIT
\item[Current version:]
0.3.0
\item[Last updated:]
April 16, 2024
\item[First released:]
March 19, 2024
\item[Minimum Typst version:]
0.11.0
\item[Archive size:]
17.4 kB
\href{https://packages.typst.org/preview/suiji-0.3.0.tar.gz}{\pandocbounded{\includesvg[keepaspectratio]{/assets/icons/16-download.svg}}}
\item[Repository:]
\href{https://github.com/liuguangxi/suiji}{GitHub}
\item[Categor y :]
\begin{itemize}
\tightlist
\item[]
\item
  \pandocbounded{\includesvg[keepaspectratio]{/assets/icons/16-hammer.svg}}
  \href{https://typst.app/universe/search/?category=utility}{Utility}
\end{itemize}
\end{description}

\subsubsection{Where to report issues?}\label{where-to-report-issues}

This package is a project of Guangxi Liu . Report issues on
\href{https://github.com/liuguangxi/suiji}{their repository} . You can
also try to ask for help with this package on the
\href{https://forum.typst.app}{Forum} .

Please report this package to the Typst team using the
\href{https://typst.app/contact}{contact form} if you believe it is a
safety hazard or infringes upon your rights.

\phantomsection\label{versions}
\subsubsection{Version history}\label{version-history}

\begin{longtable}[]{@{}ll@{}}
\toprule\noalign{}
Version & Release Date \\
\midrule\noalign{}
\endhead
\bottomrule\noalign{}
\endlastfoot
0.3.0 & April 16, 2024 \\
\href{https://typst.app/universe/package/suiji/0.2.2/}{0.2.2} & April 9,
2024 \\
\href{https://typst.app/universe/package/suiji/0.2.1/}{0.2.1} & March
28, 2024 \\
\href{https://typst.app/universe/package/suiji/0.2.0/}{0.2.0} & March
22, 2024 \\
\href{https://typst.app/universe/package/suiji/0.1.0/}{0.1.0} & March
19, 2024 \\
\end{longtable}

Typst GmbH did not create this package and cannot guarantee correct
functionality of this package or compatibility with any version of the
Typst compiler or app.
