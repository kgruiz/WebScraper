\title{typst.app/universe/package/touying-unistra-pristine}

\phantomsection\label{banner}
\phantomsection\label{template-thumbnail}
\pandocbounded{\includegraphics[keepaspectratio]{https://packages.typst.org/preview/thumbnails/touying-unistra-pristine-1.2.0-small.webp}}

\section{touying-unistra-pristine}\label{touying-unistra-pristine}

{ 1.2.0 }

Touying theme adhering to the core principles of the style guide of the
University of Strasbourg, France

\href{/app?template=touying-unistra-pristine&version=1.2.0}{Create
project in app}

\phantomsection\label{readme}
\begin{quote}
{[}!WARNING{]} This theme is \textbf{NOT} affiliated with the University
of Strasbourg. The logo and the fonts are the property of the University
of Strasbourg.
\end{quote}

\textbf{touying-unistra-pristine} is a
\href{https://github.com/touying-typ/touying}{Touying} theme for
creating presentation slides in
\href{https://github.com/typst/typst}{Typst} , adhering to the core
principles of the \href{https://langagevisuel.unistra.fr/}{style guide
of the University of Strasbourg, France} (French). It is an
\textbf{unofficial} theme and it is \textbf{NOT} affiliated with the
University of Strasbourg.

This theme was partly created using components from
\href{https://github.com/typst-tud/tud-slides}{tud-slides} and
\href{https://github.com/piepert/grape-suite}{grape-suite} .

\begin{itemize}
\tightlist
\item
  \textbf{Focus Slides} , with predefined themes and custom colors
  support.
\item
  \textbf{Hero Slides} .
\item
  \textbf{Gallery Slides} .
\item
  \textbf{Admonitions} (with localization and plural support).
\item
  \textbf{Universally Toggleable Header/Footer} (see
  \href{https://github.com/typst/packages/raw/main/packages/preview/touying-unistra-pristine/1.2.0/\#Configuration}{Configuration}
  ).
\item
  Subset of predefined colors taken from the
  \href{https://langagevisuel.unistra.fr/index.php?id=396}{style guide
  of the University of Strasbourg} (see
  \href{https://github.com/typst/packages/raw/main/packages/preview/touying-unistra-pristine/1.2.0/colors.typ}{colors.typ}
  ).
\end{itemize}

See
\href{https://github.com/typst/packages/raw/main/packages/preview/touying-unistra-pristine/1.2.0/example/example.pdf}{example/example.pdf}
for an example PDF output, and
\href{https://github.com/typst/packages/raw/main/packages/preview/touying-unistra-pristine/1.2.0/example/example.typ}{example/example.typ}
for the corresponding Typst file.

These steps assume that you already have
\href{https://typst.app/}{Typst} installed and/or running.

\subsection{Import from Typst
Universe}\label{import-from-typst-universe}

\begin{Shaded}
\begin{Highlighting}[]
\NormalTok{\#import "@preview/touying:0.5.3": *}
\NormalTok{\#import "@preview/touying{-}unistra{-}pristine:1.2.0": *}

\NormalTok{\#show: unistra{-}theme.with(}
\NormalTok{  aspect{-}ratio: "16{-}9",}
\NormalTok{  config{-}info(}
\NormalTok{    title: [Title],}
\NormalTok{    subtitle: [\_Subtitle\_],}
\NormalTok{    author: [Author],}
\NormalTok{    date: datetime.today().display("[month repr:long] [day], [year repr:full]"),}
\NormalTok{  ),}
\NormalTok{)}

\NormalTok{\#title{-}slide[]}

\NormalTok{= Example Section Title}

\NormalTok{== Example Slide}

\NormalTok{A slide with *important information*.}

\NormalTok{\#lorem(50)}
\end{Highlighting}
\end{Shaded}

\subsection{Local installation}\label{local-installation}

\subsubsection{1. Clone the project}\label{clone-the-project}

\texttt{\ git\ clone\ https://github.com/spidersouris/touying-unistra-pristine\ }

\subsubsection{2. Import Touying and
touying-unistra-pristine}\label{import-touying-and-touying-unistra-pristine}

See
\href{https://github.com/typst/packages/raw/main/packages/preview/touying-unistra-pristine/1.2.0/example/example.typ}{example/example.typ}
for a complete example with configuration.

\begin{Shaded}
\begin{Highlighting}[]
\NormalTok{\#import "@preview/touying:0.5.3": *}
\NormalTok{\#import "src/unistra.typ": *}
\NormalTok{\#import "src/colors.typ": *}
\NormalTok{\#import "src/admonition.typ": *}

\NormalTok{\#show: unistra{-}theme.with(}
\NormalTok{  aspect{-}ratio: "16{-}9",}
\NormalTok{  config{-}info(}
\NormalTok{    title: [Title],}
\NormalTok{    subtitle: [\_Subtitle\_],}
\NormalTok{    author: [Author],}
\NormalTok{    date: datetime.today().display("[month repr:long] [day], [year repr:full]"),}
\NormalTok{  ),}
\NormalTok{)}

\NormalTok{\#title{-}slide[]}

\NormalTok{= Example Section Title}

\NormalTok{== Example Slide}

\NormalTok{A slide with *important information*.}

\NormalTok{\#lorem(50)}
\end{Highlighting}
\end{Shaded}

\begin{quote}
{[}!NOTE{]} The default font used by touying-unistra-pristine is
“Unistra A�, a font that can only be downloaded by students and
staff from the University of Strasbourg. If the font is not installed on
your computer, Segoe UI or Roboto will be used as a fallback, in that
specific order. You can change that behavior in the
\href{https://github.com/typst/packages/raw/main/packages/preview/touying-unistra-pristine/1.2.0/\#Configuration}{settings}
.
\end{quote}

The theme can be configured to your liking by adding the
\texttt{\ config-store()\ } object when initializing
\texttt{\ unistra-theme\ } . An example with the \texttt{\ quotes\ }
setting can be found in
\href{https://github.com/typst/packages/raw/main/packages/preview/touying-unistra-pristine/1.2.0/example/example.typ}{example/example.typ}
.

A complete list of settings can be found in the
\texttt{\ config-store\ } object in
\href{https://github.com/typst/packages/raw/main/packages/preview/touying-unistra-pristine/1.2.0/src/unistra.typ}{src/unistra.typ}
.

\href{/app?template=touying-unistra-pristine&version=1.2.0}{Create
project in app}

\subsubsection{How to use}\label{how-to-use}

Click the button above to create a new project using this template in
the Typst app.

You can also use the Typst CLI to start a new project on your computer
using this command:

\begin{verbatim}
typst init @preview/touying-unistra-pristine:1.2.0
\end{verbatim}

\includesvg[width=0.16667in,height=0.16667in]{/assets/icons/16-copy.svg}

\subsubsection{About}\label{about}

\begin{description}
\tightlist
\item[Author :]
\href{https://edoyen.com/}{Enzo Doyen}
\item[License:]
MIT
\item[Current version:]
1.2.0
\item[Last updated:]
November 22, 2024
\item[First released:]
September 11, 2024
\item[Minimum Typst version:]
0.12.0
\item[Archive size:]
19.7 kB
\href{https://packages.typst.org/preview/touying-unistra-pristine-1.2.0.tar.gz}{\pandocbounded{\includesvg[keepaspectratio]{/assets/icons/16-download.svg}}}
\item[Repository:]
\href{https://github.com/spidersouris/touying-unistra-pristine}{GitHub}
\item[Categor y :]
\begin{itemize}
\tightlist
\item[]
\item
  \pandocbounded{\includesvg[keepaspectratio]{/assets/icons/16-presentation.svg}}
  \href{https://typst.app/universe/search/?category=presentation}{Presentation}
\end{itemize}
\end{description}

\subsubsection{Where to report issues?}\label{where-to-report-issues}

This template is a project of Enzo Doyen . Report issues on
\href{https://github.com/spidersouris/touying-unistra-pristine}{their
repository} . You can also try to ask for help with this template on the
\href{https://forum.typst.app}{Forum} .

Please report this template to the Typst team using the
\href{https://typst.app/contact}{contact form} if you believe it is a
safety hazard or infringes upon your rights.

\phantomsection\label{versions}
\subsubsection{Version history}\label{version-history}

\begin{longtable}[]{@{}ll@{}}
\toprule\noalign{}
Version & Release Date \\
\midrule\noalign{}
\endhead
\bottomrule\noalign{}
\endlastfoot
1.2.0 & November 22, 2024 \\
\href{https://typst.app/universe/package/touying-unistra-pristine/1.1.0/}{1.1.0}
& October 17, 2024 \\
\href{https://typst.app/universe/package/touying-unistra-pristine/1.0.0/}{1.0.0}
& September 11, 2024 \\
\end{longtable}

Typst GmbH did not create this template and cannot guarantee correct
functionality of this template or compatibility with any version of the
Typst compiler or app.
