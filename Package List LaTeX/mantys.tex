\title{typst.app/universe/package/mantys}

\phantomsection\label{banner}
\phantomsection\label{template-thumbnail}
\pandocbounded{\includegraphics[keepaspectratio]{https://packages.typst.org/preview/thumbnails/mantys-0.1.4-small.webp}}

\section{mantys}\label{mantys}

{ 0.1.4 }

Helpers to build manuals for Typst packages.

\href{/app?template=mantys&version=0.1.4}{Create project in app}

\phantomsection\label{readme}
\begin{quote}
\textbf{MAN} uals for \textbf{TY} p \textbf{S} t
\end{quote}

Template for documenting \href{https://github.com/typst/typst}{typst}
packages and templates.

\subsection{Usage}\label{usage}

Just import the package at the beginning of your manual:

\begin{Shaded}
\begin{Highlighting}[]
\NormalTok{\#import "@preview/mantys:0.1.4": *}
\end{Highlighting}
\end{Shaded}

Mantys supports \textbf{Typst 0.11.0} and newer.

\subsection{Writing basics}\label{writing-basics}

A basic template for a manual could look like this:

\begin{Shaded}
\begin{Highlighting}[]
\NormalTok{\#import "@local/mantys:0.1.4": *}

\NormalTok{\#import "your{-}package.typ"}

\NormalTok{\#show: mantys.with(}
\NormalTok{    name:        "your{-}package{-}name",}
\NormalTok{    title:       [A title for the manual],}
\NormalTok{    subtitle:    [A subtitle for the manual],}
\NormalTok{    info:        [A short descriptive text for the package.],}
\NormalTok{    authors: "Your Name",}
\NormalTok{    url:     "https://github.com/repository/url",}
\NormalTok{    version: "0.0.1",}
\NormalTok{    date:        "date{-}of{-}release",}
\NormalTok{    abstract:    [}
\NormalTok{        A few paragraphs of text to describe the package.}
\NormalTok{    ],}

\NormalTok{    example{-}imports: (your{-}package: your{-}package)}
\NormalTok{)}

\NormalTok{// end of preamble}

\NormalTok{\# About}
\NormalTok{\#lorem(50)}

\NormalTok{\# Usage}
\NormalTok{\#lorem(50)}

\NormalTok{\# Available commands}
\NormalTok{\#lorem(50)}
\end{Highlighting}
\end{Shaded}

Use \texttt{\ \#command(name,\ ..args){[}description{]}\ } to describe
commands and \texttt{\ \#argument(name,\ ...){[}description{]}\ } for
arguments:

\begin{Shaded}
\begin{Highlighting}[]
\NormalTok{\#command("headline", arg[color], arg(size:1.8em), sarg[other{-}args], barg[body])[}
\NormalTok{    Renders a prominent headline using \#doc("meta/heading").}

\NormalTok{    \#argument("color", type:"color")[}
\NormalTok{    The color of the headline will be used as the background of a \#doc("layout/block") element containing the headline.}
\NormalTok{  ]}
\NormalTok{  \#argument("size", default:1.8em)[}
\NormalTok{    The text size for the headline.}
\NormalTok{  ]}
\NormalTok{  \#argument("sarg", is{-}sink:true)[}
\NormalTok{    Other options will get passed directly to \#doc("meta/heading").}
\NormalTok{  ]}
\NormalTok{  \#argument("body", type:"content")[}
\NormalTok{    The text for the headline.}
\NormalTok{  ]}

\NormalTok{  The headline is shown as a prominent colored block to highlight important news articles in the newsletter:}

\NormalTok{  \#example[\textasciigrave{}\textasciigrave{}\textasciigrave{}}
\NormalTok{  \#headline(blue, size: 2em, level: 3)[}
\NormalTok{    \#lorem(8)}
\NormalTok{  ]}
\NormalTok{  \textasciigrave{}\textasciigrave{}\textasciigrave{}]}
\NormalTok{]}
\end{Highlighting}
\end{Shaded}

The result might look something like this:

\pandocbounded{\includegraphics[keepaspectratio]{https://github.com/typst/packages/raw/main/packages/preview/mantys/0.1.4/docs/assets/headline-example.png}}

For a full reference of available commands read
\href{https://github.com/typst/packages/raw/main/packages/preview/mantys/0.1.4/docs/mantys-manual.pdf}{the
manual} .

\subsection{Changelog}\label{changelog}

\subsubsection{Version 0.1.4}\label{version-0.1.4}

\begin{itemize}
\tightlist
\item
  Fix missing links in outline (@tingerrr).
\item
  Fixed problem when evaluating default values with Tidy.
\end{itemize}

\subsubsection{Version 0.1.3}\label{version-0.1.3}

\begin{itemize}
\tightlist
\item
  Fix for some datatypes not being displayed properly (thanks to
  @tingerrr).
\item
  Fix for imbalanced outline columns (thanks again to @tingerrr).
\end{itemize}

\subsubsection{Version 0.1.2}\label{version-0.1.2}

\begin{itemize}
\tightlist
\item
  Added \href{https://typst.app/universe/package/hydra}{hydra} for
  better detection of headings in page headers (thanks to @tingerrr for
  the suggestion).
\item
  Fixed problem with multiple quotes around default string values in
  tidy docs.
\item
  Fixed datatypes linking to wrong documentation urls.
\end{itemize}

\subsubsection{Version 0.1.1}\label{version-0.1.1}

\begin{itemize}
\tightlist
\item
  Added template files for submission to \emph{Typst Universe} .
\end{itemize}

\subsubsection{Version 0.1.0}\label{version-0.1.0}

\begin{itemize}
\tightlist
\item
  Refactorings and some style changes
\item
  Updated manual.
\item
  Restructuring of package repository.
\end{itemize}

\subsubsection{Version 0.0.4}\label{version-0.0.4}

\begin{itemize}
\tightlist
\item
  Added integration with \href{https://github.com/Mc-Zen/tidy}{tidy} .
\item
  Fixed issue with types in argument boxes.
\item
  \texttt{\ \#lambda\ } now uses \texttt{\ \#dtype\ }
\end{itemize}

\paragraph{Breaking changes}\label{breaking-changes}

\begin{itemize}
\tightlist
\item
  Adapted \texttt{\ scope\ } argument for \texttt{\ eval\ } in examples.

  \begin{itemize}
  \tightlist
  \item
    \texttt{\ \#example()\ } , \texttt{\ \#side-by-side()\ } and
    \texttt{\ \#shortex()\ } now support the \texttt{\ scope\ } and
    \texttt{\ mode\ } argument.
  \item
    The option \texttt{\ example-imports\ } was replaced by
    \texttt{\ examples-scope\ } .
  \end{itemize}
\end{itemize}

\subsubsection{Version 0.0.3}\label{version-0.0.3}

\begin{itemize}
\item
  It is now possible to load a packages’ \texttt{\ typst.toml\ } file
  directly into \texttt{\ \#mantys\ } :

\begin{Shaded}
\begin{Highlighting}[]
\NormalTok{\#show: mantys.with( ..toml("typst.toml") )}
\end{Highlighting}
\end{Shaded}
\item
  Added some dependencies:

  \begin{itemize}
  \tightlist
  \item
    \href{https://github.com/jneug/typst-tools4typst}{jneug/typst-tools4typst}
    for some common utilities,
  \item
    \href{https://github.com/jneug/typst-codelst}{jneug/typst-codelst}
    for rendering examples and source code,
  \item
    \href{https://github.com/Pablo-Gonzalez-Calderon/showybox-package}{Pablo-Gonzalez-Calderon/showybox-package}
    for adding frames to different areas of a manual (like examples).
  \end{itemize}
\item
  Redesign of some elements:

  \begin{itemize}
  \tightlist
  \item
    Argument display in command descriptions,
  \item
    Alert boxes.
  \end{itemize}
\item
  Added \texttt{\ \#version(since:(),\ until:())\ } command to add
  version markers to commands.
\item
  Styles moved to a separate \texttt{\ theme.typ\ } file to allow easy
  customization of colors and styles.
\item
  Added \texttt{\ \#func()\ } , \texttt{\ \#lambda()\ } and
  \texttt{\ \#symbol()\ } commands, to handle special cases for values.
\item
  Fixes and code improvements.
\end{itemize}

\subsubsection{Version 0.0.2}\label{version-0.0.2}

\begin{itemize}
\tightlist
\item
  Some major updates to the core commands and styles.
\end{itemize}

\subsubsection{Version 0.0.1}\label{version-0.0.1}

\begin{itemize}
\tightlist
\item
  Initial release.
\end{itemize}

\href{/app?template=mantys&version=0.1.4}{Create project in app}

\subsubsection{How to use}\label{how-to-use}

Click the button above to create a new project using this template in
the Typst app.

You can also use the Typst CLI to start a new project on your computer
using this command:

\begin{verbatim}
typst init @preview/mantys:0.1.4
\end{verbatim}

\includesvg[width=0.16667in,height=0.16667in]{/assets/icons/16-copy.svg}

\subsubsection{About}\label{about}

\begin{description}
\tightlist
\item[Author :]
Jonas Neugebauer
\item[License:]
MIT
\item[Current version:]
0.1.4
\item[Last updated:]
May 23, 2024
\item[First released:]
March 21, 2024
\item[Minimum Typst version:]
0.11.0
\item[Archive size:]
19.7 kB
\href{https://packages.typst.org/preview/mantys-0.1.4.tar.gz}{\pandocbounded{\includesvg[keepaspectratio]{/assets/icons/16-download.svg}}}
\item[Repository:]
\href{https://github.com/jneug/typst-mantys}{GitHub}
\item[Categor ies :]
\begin{itemize}
\tightlist
\item[]
\item
  \pandocbounded{\includesvg[keepaspectratio]{/assets/icons/16-layout.svg}}
  \href{https://typst.app/universe/search/?category=layout}{Layout}
\item
  \pandocbounded{\includesvg[keepaspectratio]{/assets/icons/16-list-unordered.svg}}
  \href{https://typst.app/universe/search/?category=model}{Model}
\item
  \pandocbounded{\includesvg[keepaspectratio]{/assets/icons/16-hammer.svg}}
  \href{https://typst.app/universe/search/?category=utility}{Utility}
\end{itemize}
\end{description}

\subsubsection{Where to report issues?}\label{where-to-report-issues}

This template is a project of Jonas Neugebauer . Report issues on
\href{https://github.com/jneug/typst-mantys}{their repository} . You can
also try to ask for help with this template on the
\href{https://forum.typst.app}{Forum} .

Please report this template to the Typst team using the
\href{https://typst.app/contact}{contact form} if you believe it is a
safety hazard or infringes upon your rights.

\phantomsection\label{versions}
\subsubsection{Version history}\label{version-history}

\begin{longtable}[]{@{}ll@{}}
\toprule\noalign{}
Version & Release Date \\
\midrule\noalign{}
\endhead
\bottomrule\noalign{}
\endlastfoot
0.1.4 & May 23, 2024 \\
\href{https://typst.app/universe/package/mantys/0.1.3/}{0.1.3} & April
29, 2024 \\
\href{https://typst.app/universe/package/mantys/0.1.1/}{0.1.1} & March
21, 2024 \\
\end{longtable}

Typst GmbH did not create this template and cannot guarantee correct
functionality of this template or compatibility with any version of the
Typst compiler or app.
