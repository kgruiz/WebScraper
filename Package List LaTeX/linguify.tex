\title{typst.app/universe/package/linguify}

\phantomsection\label{banner}
\section{linguify}\label{linguify}

{ 0.4.1 }

Load strings for different languages easily

\phantomsection\label{readme}
Load strings for different languages easily. This can be useful if you
create a package or template for multilingual usage.

\subsection{Usage}\label{usage}

The usage depends if you are using it inside a package or a template or
in your own document.

\subsubsection{For end users and own
templates}\label{for-end-users-and-own-templates}

You can use linguify global database.

Example:

\begin{Shaded}
\begin{Highlighting}[]
\NormalTok{\#import "@preview/linguify:0.4.0": *}

\NormalTok{\#let lang\_data = toml("lang.toml")}
\NormalTok{\#set{-}database(lang\_data);}

\NormalTok{\#set text(lang: "de")}

\NormalTok{\#linguify("abstract")  // Shows "Zusammenfassung" in the document.}
\end{Highlighting}
\end{Shaded}

The \texttt{\ lang.toml\ } musst look like this:

\begin{Shaded}
\begin{Highlighting}[]
\KeywordTok{[conf]}
\DataTypeTok{default{-}lang} \OperatorTok{=} \StringTok{"en"}

\KeywordTok{[en]}
\DataTypeTok{title} \OperatorTok{=} \StringTok{"A simple linguify example"}
\DataTypeTok{abstract} \OperatorTok{=} \StringTok{"Abstract"}

\KeywordTok{[de]}
\DataTypeTok{title} \OperatorTok{=} \StringTok{"Ein einfaches Linguify Beispiel"}
\DataTypeTok{abstract} \OperatorTok{=} \StringTok{"Zusammenfassung"}
\end{Highlighting}
\end{Shaded}

\subsubsection{Inside a package}\label{inside-a-package}

So that multiple packages can use linguify simultaneously, they should
contain their own database. A linguify database is just a dictionary
with a certain structure. (See database structure.)

Recommend is to store the database in a separate file like
\texttt{\ lang.toml\ } and load it inside the document. And specify it
in each \texttt{\ linguify()\ } function call.

Example:

\begin{Shaded}
\begin{Highlighting}[]
\NormalTok{\#import "@preview/linguify:0.4.0": *}

\NormalTok{\#let database = toml("lang.toml")}

\NormalTok{\#linguify("key", from: database, default: "key")}
\end{Highlighting}
\end{Shaded}

\subsection{Features}\label{features}

\begin{itemize}
\tightlist
\item
  Use a \texttt{\ toml\ } or other file to load strings for different
  languages. You need to pass a typst dictionary which follows the
  structure of the shown toml file.
\item
  Specify a \textbf{default-lang} . If none is specified it will default
  to \texttt{\ en\ }
\item
  \textbf{Fallback} to the default-lang if a key is not found for a
  certain language.
\item
  \href{https://projectfluent.org/}{Fluent} support
\end{itemize}

\subsubsection{How to add}\label{how-to-add}

Copy this into your project and use the import as \texttt{\ linguify\ }

\begin{verbatim}
#import "@preview/linguify:0.4.1"
\end{verbatim}

\includesvg[width=0.16667in,height=0.16667in]{/assets/icons/16-copy.svg}

Check the docs for
\href{https://typst.app/docs/reference/scripting/\#packages}{more
information on how to import packages} .

\subsubsection{About}\label{about}

\begin{description}
\tightlist
\item[Author :]
\href{https://github.com/jomaway}{Jomaway}
\item[License:]
MIT
\item[Current version:]
0.4.1
\item[Last updated:]
April 29, 2024
\item[First released:]
January 31, 2024
\item[Minimum Typst version:]
0.11.0
\item[Archive size:]
470 kB
\href{https://packages.typst.org/preview/linguify-0.4.1.tar.gz}{\pandocbounded{\includesvg[keepaspectratio]{/assets/icons/16-download.svg}}}
\item[Repository:]
\href{https://github.com/jomaway/typst-linguify}{GitHub}
\item[Categor ies :]
\begin{itemize}
\tightlist
\item[]
\item
  \pandocbounded{\includesvg[keepaspectratio]{/assets/icons/16-world.svg}}
  \href{https://typst.app/universe/search/?category=languages}{Languages}
\item
  \pandocbounded{\includesvg[keepaspectratio]{/assets/icons/16-hammer.svg}}
  \href{https://typst.app/universe/search/?category=utility}{Utility}
\end{itemize}
\end{description}

\subsubsection{Where to report issues?}\label{where-to-report-issues}

This package is a project of Jomaway . Report issues on
\href{https://github.com/jomaway/typst-linguify}{their repository} . You
can also try to ask for help with this package on the
\href{https://forum.typst.app}{Forum} .

Please report this package to the Typst team using the
\href{https://typst.app/contact}{contact form} if you believe it is a
safety hazard or infringes upon your rights.

\phantomsection\label{versions}
\subsubsection{Version history}\label{version-history}

\begin{longtable}[]{@{}ll@{}}
\toprule\noalign{}
Version & Release Date \\
\midrule\noalign{}
\endhead
\bottomrule\noalign{}
\endlastfoot
0.4.1 & April 29, 2024 \\
\href{https://typst.app/universe/package/linguify/0.4.0/}{0.4.0} & April
2, 2024 \\
\href{https://typst.app/universe/package/linguify/0.3.1/}{0.3.1} & March
26, 2024 \\
\href{https://typst.app/universe/package/linguify/0.3.0/}{0.3.0} & March
18, 2024 \\
\href{https://typst.app/universe/package/linguify/0.2.0/}{0.2.0} & March
16, 2024 \\
\href{https://typst.app/universe/package/linguify/0.1.0/}{0.1.0} &
January 31, 2024 \\
\end{longtable}

Typst GmbH did not create this package and cannot guarantee correct
functionality of this package or compatibility with any version of the
Typst compiler or app.
