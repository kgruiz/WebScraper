\title{typst.app/universe/package/game-theoryst}

\phantomsection\label{banner}
\section{game-theoryst}\label{game-theoryst}

{ 0.1.0 }

A package for typesetting games in Typst.

{ } Featured Package

\phantomsection\label{readme}
A package for typesetting games in Typst.

Full manual available
\href{https://github.com/typst/packages/raw/main/packages/preview/game-theoryst/0.1.0/doc/gtheoryst-manual.pdf}{here}

Work in progress â€`` \emph{coming soon!}

\subsection{Overview}\label{overview}

\paragraph{Simple Example}\label{simple-example}

The main function to make strategic (or \textbf{normal} ) form games is
\texttt{\ nfg\ } . For a basic 2x2 game, you can do

\begin{Shaded}
\begin{Highlighting}[]
\NormalTok{\#nfg(}
\NormalTok{  players: ("Jack", "Diane"),}
\NormalTok{  s1: ($C$, $D$),}
\NormalTok{  s2: ($C$, $D$),}
\NormalTok{  [$10, 10$], [$2, 20$], }
\NormalTok{  [$20, 2$], [$5, 5$],}
\NormalTok{)}
\end{Highlighting}
\end{Shaded}

\includegraphics[width=4.16667in,height=\textheight,keepaspectratio]{https://github.com/typst/packages/raw/main/packages/preview/game-theoryst/0.1.0/doc/gallery/simple-example.png}

\subsubsection{Importing}\label{importing}

Simply insert the following into your Typst code:

\begin{Shaded}
\begin{Highlighting}[]
\NormalTok{\#import "@preview/game{-}theoryst:0.1.0": *}
\end{Highlighting}
\end{Shaded}

This imports the \texttt{\ nfg()\ } function as well as the underlining
methods. If you want to tweak the helper functions for generating an
\texttt{\ nfg\ } , import them explicitly through the
\texttt{\ utils/\ } directory.

\paragraph{Full Example}\label{full-example}

\begin{Shaded}
\begin{Highlighting}[]
\NormalTok{\#nfg(}
\NormalTok{  players: ([A\textbackslash{} Joe], [Bas Pro]),}
\NormalTok{  s1: ([$x$], [a]),}
\NormalTok{  s2: ("x", "aaaa", [$a$]),}
\NormalTok{  pad: ("x": 12pt, "y": 10pt),}
\NormalTok{  eliminations: ("s11", "s21", "s22"),}
\NormalTok{  ejust: (}
\NormalTok{    s11: (x: (0pt, 36pt), y: ({-}3pt, {-}3.5pt)),}
\NormalTok{    s22: (x: ({-}10pt, {-}12pt), y: ({-}10pt, 10pt)),}
\NormalTok{    s21: (x: ({-}3pt, {-}9pt), y: ({-}10pt, 10pt)),}
\NormalTok{  ),}
\NormalTok{  mixings: (hmix: ($p$, $1{-}p$), vmix: ($q$, $r$, $1{-}q{-}r$)),}
\NormalTok{  custom{-}fills: (hp: maroon, vp: navy, hm: purple, vm: fuchsia, he: gray, ve: gray),}
\NormalTok{  [$0,vul(100000000)$], [$0,1$], [$0,0$],}
\NormalTok{  [$hul(1),1$], [$0, {-}1$], table.cell(fill: yellow.lighten(30\%), [$hful(0),vful(0)$])}
\NormalTok{)}
\end{Highlighting}
\end{Shaded}

\includegraphics[width=5.46875in,height=\textheight,keepaspectratio]{https://github.com/typst/packages/raw/main/packages/preview/game-theoryst/0.1.0/doc/gallery/full-example.png}

\subsubsection{Color}\label{color}

By default, player names, mixed-strategy parameters (called
\emph{mixings} ), and elimination lines are shown in color. These colors
can be turned off at the method-level by passing \texttt{\ bw:\ true\ }
, or at the document level by running the state helper-function
\texttt{\ \#colorless()\ } .

\texttt{\ nfg\ } accepts custom colors for all of the aforementioned
parameters by passing a \texttt{\ dictionary\ } of colors to the
\texttt{\ custom-fills\ } arg. The keys for this dictionary are as
follows ( \texttt{\ \textless{}defaults\textgreater{}\ } ):

\begin{itemize}
\tightlist
\item
  \texttt{\ hp\ } â€`` “horizontal playerâ€? (red)
\item
  \texttt{\ vp\ } â€`` “vertical playerâ€? (blue)
\item
  \texttt{\ hm\ } â€`` “hor. mixingâ€? (\#e64173)
\item
  \texttt{\ vm\ } â€`` “ver. mixingâ€? (eastern)
\item
  \texttt{\ he\ } â€`` “hor. eliminationâ€? line (orange)
\item
  \texttt{\ ve\ } â€`` “ver. eliminationâ€? line (olive)
\end{itemize}

\subsection{Cell Customization}\label{cell-customization}

Since the payoffs are implemented as argument sinks (
\texttt{\ ..args\ } ) which are passed directly to Typst’s
\texttt{\ \#table()\ } , underlining of non-math can be accomplished via
the standard \texttt{\ \#underline()\ } command. Similarly, any of the
payoff cells can be customized by using \texttt{\ table.cell()\ }
directly. For instance,
\texttt{\ table.cell(fill:\ yellow.lighten(30\%),\ {[}\$1,\ 1\${]})\ }
can be used to highlight a specific cell.

\subsubsection{Padding}\label{padding}

There are edge cases where the default padding may be off. These can be
mended by passing the optional \texttt{\ pad\ } argument to
\texttt{\ nfg()\ } . This should represent how much
\textbf{\emph{additional}} padding you want. The \texttt{\ pad\ } arg.
is interpreted as follows:

\begin{itemize}
\tightlist
\item
  If a \texttt{\ length\ } is provided, it assumes you want that much
  length added to all cell walls
\item
  If an array of the form \texttt{\ (L1,\ L2)\ } is provided, it assumes
  you want padding a horizontal ( \texttt{\ x\ } ) padding of
  \texttt{\ L1\ } and a vertical padding ( \texttt{\ y\ } ) of
  \texttt{\ L2\ }
\item
  If a \texttt{\ dictionary\ } is provided, it operates identically to
  that of the array, but you must specify the \texttt{\ x\ } /
  \texttt{\ y\ } keys yourself
\end{itemize}

\subsubsection{AUtomatic Cell Sizing}\label{automatic-cell-sizing}

Cell are automatically sized to equal heights/widths according to the
longest/tallest content. If you want to avoid this behavior, pass
\texttt{\ lazy-cells:\ true\ } to \texttt{\ nfg\ } . This behavior can
be combined with the custom \texttt{\ padding\ } argument.

\subsection{Semantic Game Theory
Features}\label{semantic-game-theory-features}

\subsubsection{Underlining}\label{underlining}

The package imports a small set of underlining utility functions.

The primary functions for underlining are

\begin{itemize}
\tightlist
\item
  \texttt{\ hul()\ } â€`` \emph{Horizontal Underline}
\item
  \texttt{\ vul()\ } â€`` \emph{Vertical Underline}
\item
  \texttt{\ bul()\ } â€`` \emph{Black Underline} These can be wrapped
  around values in math-mode ( \texttt{\ \$..\$\ } ) within the payoff
  matrix. The underlines for \texttt{\ hul\ } and \texttt{\ vul\ } are
  colored by default according to the default colors for names, but they
  accept an optional \texttt{\ col\ } parameter for changing the color
  of the underline. \texttt{\ bul()\ } produces a black underline.
\end{itemize}

\begin{Shaded}
\begin{Highlighting}[]
\NormalTok{\#nfg(}
\NormalTok{  players: ("Jack", "Diane"),}
\NormalTok{  s2: ($x$, $y$, $z$),}
\NormalTok{  s1: ($a$, $b$),}
\NormalTok{  [$hul(0),vul(0)$], [$1,1$], [$2,2$],}
\NormalTok{  [$3,3$], [$4,4$], [$5,5$],}
\NormalTok{)}
\end{Highlighting}
\end{Shaded}

By default, these commands leave the numbers themselves black, but
boldfaces them. \emph{Full Color} versions of \texttt{\ hul\ } and
\texttt{\ vul\ } , which color the numbers and under-lines identically,
are available via \texttt{\ hful()\ } and \texttt{\ vful()\ } . Like
their counterparts, they accept an optional \texttt{\ col\ } command for
the color.

Both of the colors can be modified individually via the general
\texttt{\ cul()\ } command, which takes in content ( \texttt{\ cont\ }
), an underline color ( \texttt{\ ucol\ } ), and the color for the text
value ( \texttt{\ tcol\ } ). For instance,

\begin{Shaded}
\begin{Highlighting}[]
\NormalTok{\#let new{-}ul(cont, col: olive, tcol: fuchsia) = \{ cul(cont, col, tcol) \}}
\end{Highlighting}
\end{Shaded}

will define a new command which underlines in olive and sets the text
(math) color to fuchsia.

\subsubsection{Mixed Strategies}\label{mixed-strategies}

You can optionally mark mixed strategies that a player will in a
\texttt{\ nfg\ } using the \texttt{\ mixing\ } argument. This can be a
\texttt{\ dictionary\ } with \texttt{\ hmix\ } and \texttt{\ vmix\ }
keys, or an \texttt{\ array\ } , interpreted as a dictionary with the
aforementioned keys in the \texttt{\ (hmix,\ vmix)\ } order. The
values/entries here should be arrays which mimic \texttt{\ s1\ } and
\texttt{\ s2\ } in size, with some parameter denoting the proportion of
time the relevant player uses that strategy. If you would like to omit a
strategy from this markup, pass \texttt{\ {[}{]}\ } in it’s place.

For example:

\begin{Shaded}
\begin{Highlighting}[]
\NormalTok{\#nfg(}
\NormalTok{  players: ("Chet", "North"),}
\NormalTok{  s1: ([$F$], [$G$], [$H$]),}
\NormalTok{  s2: ([$X$], [$Y$]),}
\NormalTok{  mixings: (}
\NormalTok{    hmix: ($p$, $1{-}p$), }
\NormalTok{    vmix: ($q$, [], $1{-}q$)),}
\NormalTok{  [$7,3$], [$2,4$], }
\NormalTok{  [$5,2$], [$6,1$], }
\NormalTok{  [$6,1$], [$5,4$]}
\NormalTok{)}
\end{Highlighting}
\end{Shaded}

\includegraphics[width=4.16667in,height=\textheight,keepaspectratio]{https://github.com/typst/packages/raw/main/packages/preview/game-theoryst/0.1.0/doc/gallery/mix-ex.png}

\subsubsection{Iterated Deletion (Elimination) of Dominated
Strategies}\label{iterated-deletion-elimination-of-dominated-strategies}

You can use the \texttt{\ pinit\ } package to cross out lines,
semantically eliminating strategies. \texttt{\ pinit\ } comes
pre-imported with \texttt{\ game-theoryst\ } by default.

You can tell \texttt{\ nfg\ } which strategies to eliminate with the
\texttt{\ eliminations\ } argument and the corresponding
\texttt{\ ejust\ } helper-argument. The \texttt{\ eliminations\ }
argument is simply an \texttt{\ array\ } of \texttt{\ strings\ } of the
form
\texttt{\ "s\textless{}i\textgreater{}\textless{}j\textgreater{}"\ } ,
where \texttt{\ \textless{}i\textgreater{}\ } is the player â€`` 1 or 2
â€`` and \texttt{\ \textless{}j\textgreater{}\ } is player
\texttt{\ i\ } ’s \texttt{\ \textless{}j\textgreater{}\ } th strategy,
in left-to-right / top-to-bottom order \emph{starting at 1} . These
strategy strings represent the rows/columns which you want to eliminate.
For instance, \texttt{\ ("s12",\ "s21")\ } denotes an elimination of
player 1’s second strategy as well as player 2’s first strategy.

Due to \texttt{\ context\ } dependence, the lines typically need manual
adjustments, which can be done via the \texttt{\ ejust\ } arg.
\texttt{\ ejust\ } needs to be a dictionary with keys of matching those
strings present in \texttt{\ eliminations\ } ( \texttt{\ s11\ } ,
\texttt{\ s21\ } , etc.). The values of one of these dictionary entries
is itself a dictionary: one with \texttt{\ x\ } and \texttt{\ y\ } keys.
Each of these keys needs an array consisting of 2 lengths, corresponding
to the starting/ending \texttt{\ dx/dy\ } adjustments from
\texttt{\ pinit-line\ } .

For example, one such \texttt{\ ejust\ } argument could be
\texttt{\ ("s12":\ (x:\ (5pt,\ -5pt),\ y:\ (-10pt,\ 3pt)))\ } . This
says to adjust the “s12� elimination line by \texttt{\ 5pt\ } in the
x direction and \texttt{\ -10pt\ } in the y direction for the starting
(strategy-) side of the line, and adjust by \texttt{\ -5pt\ } in x and
\texttt{\ 3pt\ } in y on the ending (payoff-) side of the line.

\begin{Shaded}
\begin{Highlighting}[]
\NormalTok{\#let just{-}arr = (}
\NormalTok{    "s12": (x: (0pt, 10pt), y: ({-}3pt, {-}3pt)),}
\NormalTok{    "s13": (x: (0pt, 10pt), y: ({-}3pt, {-}3pt)),}
\NormalTok{    "s14": (x: (0pt, 10pt), y: ({-}3pt, {-}3pt)),}
\NormalTok{    "s21": (x: ({-}6pt, {-}8pt), y: (3pt, 8pt)),}
\NormalTok{    "s22": (x: ({-}4pt, {-}8pt), y: (3pt, 8pt)),}
\NormalTok{    "s23": (x: ({-}4pt, {-}8pt), y: (3pt, 8pt)),}
\NormalTok{)}

\NormalTok{\#nfg(}
\NormalTok{  players: ("A", "B"),}
\NormalTok{  s1: ([$N$], [$S$], [$E$], [$W$] ),}
\NormalTok{  s2: ([$W$], [$E$], [$F$], [$A$]),}
\NormalTok{  eliminations: ("s12", "s13", "s14", "s21", "s22", "s23"),}
\NormalTok{  ejust: just{-}arr,}
\NormalTok{  [$6,4$], [$7,3$], [$5,5$], [$6,6$],}
\NormalTok{  [$7,3$], [$2,7$], [$4,6$], [$5,5$],}
\NormalTok{  [$8,2$], [$6,4$], [$3,7$], [$2,8$],}
\NormalTok{  [$3,7$], [$5,5$], [$4,6$], [$5,5$],}
\NormalTok{)}
\end{Highlighting}
\end{Shaded}

\includegraphics[width=4.16667in,height=\textheight,keepaspectratio]{https://github.com/typst/packages/raw/main/packages/preview/game-theoryst/0.1.0/doc/gallery/elim-ex.png}

\subsubsection{Debugging}\label{debugging}

If you want to see all of the lines for the table, including the ones
for a players, strategies, and mixings, set the following at the top of
your document.

\begin{Shaded}
\begin{Highlighting}[]
\NormalTok{\#set table.cell(stroke: (thickness: auto))}
\end{Highlighting}
\end{Shaded}

Note that cells are always present for mixings, they just have 0
width/height when no mixings of a specific variety are provided.

\subsection{License}\label{license}

game-theoryst Copyright © 2024 Connor T. Wiegand

This program is free software: you can redistribute it and/or modify it
under the terms of the GNU Affero General Public License as published by
the Free Software Foundation, either version 3 of the License, or (at
your option) any later version.

This program is distributed in the hope that it will be useful, but
WITHOUT ANY WARRANTY; without even the implied warranty of
MERCHANTABILITY or FITNESS FOR A PARTICULAR PURPOSE. See the GNU Affero
General Public License for more details.

You should have received a copy of the GNU Affero General Public License
along with this program. If not, see
\href{http://www.gnu.org/licenses/}{http:www.gnu.org/licenses/} .

\subsubsection{How to add}\label{how-to-add}

Copy this into your project and use the import as
\texttt{\ game-theoryst\ }

\begin{verbatim}
#import "@preview/game-theoryst:0.1.0"
\end{verbatim}

\includesvg[width=0.16667in,height=0.16667in]{/assets/icons/16-copy.svg}

Check the docs for
\href{https://typst.app/docs/reference/scripting/\#packages}{more
information on how to import packages} .

\subsubsection{About}\label{about}

\begin{description}
\tightlist
\item[Author :]
\href{https://github.com/connortwiegand}{Connor T. Wiegand}
\item[License:]
AGPL-3.0-only
\item[Current version:]
0.1.0
\item[Last updated:]
August 14, 2024
\item[First released:]
August 14, 2024
\item[Archive size:]
18.9 kB
\href{https://packages.typst.org/preview/game-theoryst-0.1.0.tar.gz}{\pandocbounded{\includesvg[keepaspectratio]{/assets/icons/16-download.svg}}}
\item[Repository:]
\href{https://github.com/connortwiegand/game-theoryst}{GitHub}
\item[Discipline s :]
\begin{itemize}
\tightlist
\item[]
\item
  \href{https://typst.app/universe/search/?discipline=economics}{Economics}
\item
  \href{https://typst.app/universe/search/?discipline=education}{Education}
\item
  \href{https://typst.app/universe/search/?discipline=mathematics}{Mathematics}
\end{itemize}
\end{description}

\subsubsection{Where to report issues?}\label{where-to-report-issues}

This package is a project of Connor T. Wiegand . Report issues on
\href{https://github.com/connortwiegand/game-theoryst}{their repository}
. You can also try to ask for help with this package on the
\href{https://forum.typst.app}{Forum} .

Please report this package to the Typst team using the
\href{https://typst.app/contact}{contact form} if you believe it is a
safety hazard or infringes upon your rights.

\phantomsection\label{versions}
\subsubsection{Version history}\label{version-history}

\begin{longtable}[]{@{}ll@{}}
\toprule\noalign{}
Version & Release Date \\
\midrule\noalign{}
\endhead
\bottomrule\noalign{}
\endlastfoot
0.1.0 & August 14, 2024 \\
\end{longtable}

Typst GmbH did not create this package and cannot guarantee correct
functionality of this package or compatibility with any version of the
Typst compiler or app.
