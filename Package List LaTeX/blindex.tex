\title{typst.app/universe/package/blindex}

\phantomsection\label{banner}
\section{blindex}\label{blindex}

{ 0.1.0 }

Index-making of Biblical literature citations in Typst

\phantomsection\label{readme}
Blindex ( \texttt{\ blindex:0.1.0\ } ) is a Typst package specifically
designed for the generation of indices of Biblical literature citations
in documents. Target audience includes theologians and authors of
documents that frequently cite biblical literature.

\subsection{Index Sorting Options}\label{index-sorting-options}

The generated indices are gathered and sorted by Biblical literature
books, which can be ordered according to various Biblical literature
book ordering conventions, including:

\begin{itemize}
\tightlist
\item
  \texttt{\ "LXX"\ } â€`` The Septuagint;
\item
  \texttt{\ "Greek-Bible"\ } â€`` Septuagint + New Testament (King
  James);
\item
  \texttt{\ "Hebrew-Tanakh"\ } â€`` The Hebrew (Torah + Neviim +
  Ketuvim);
\item
  \texttt{\ "Hebrew-Bible"\ } â€`` The Hebrew Tanakh + New Testament
  (King James);
\item
  \texttt{\ "Protestant-Bible"\ } â€`` The Protestant Old + New
  Testaments;
\item
  \texttt{\ "Catholic-Bible"\ } â€`` The Catholic Old + New Testaments;
\item
  \texttt{\ "Orthodox-Bible"\ } â€`` The Orthodox Old + New Testaments;
\item
  \texttt{\ "Oecumenic-Bible"\ } â€`` The Jewish Tanakh + Old Testament
  Deuterocanonical + New Testament;
\item
  \texttt{\ "code"\ } â€`` All registered Biblical literature books: All
  Protestant + All Apocripha.
\end{itemize}

\subsection{Biblical Literature
Abbrevations}\label{biblical-literature-abbrevations}

It is common practice among theologians to refer to biblical literature
books by their abbreviations. Practice shows that abbreviation
conventions are language- and tradition- dependent. Therefore,
\texttt{\ blindex\ } usage reflects this fact, while offering a way to
input arbitrary language-tradition abbreviations, in the
\texttt{\ lang.typ\ } source file.

\subsubsection{Language and Traditions
(Variants)}\label{language-and-traditions-variants}

The \texttt{\ blindex\ } implementation generalizes the concept of
\textbf{tradition} (in the context of biblical literature book
abbreviation bookkeeping) as language \textbf{variants} , since the
software can have things such as a “default� of “n-char�
variants.

As of the current release, supported languages include the following few
ones:

\begin{longtable}[]{@{}llll@{}}
\toprule\noalign{}
Language & Variant & Description & Name \\
\midrule\noalign{}
\endhead
\bottomrule\noalign{}
\endlastfoot
English & 3-char & A 3-char abbreviations & \texttt{\ en-3\ } \\
English & Logos & Used in \texttt{\ logos.com\ } &
\texttt{\ en-logos\ } \\
Portuguese (BR) & Protestant & Protestant for Brazil &
\texttt{\ br-pro\ } \\
Portuguese (BR) & Catholic & Catholic for Brazil &
\texttt{\ br-cat\ } \\
\end{longtable}

Additional language-variations can be added to the \texttt{\ lang.typ\ }
source file by the author and/or by pull requests to the
\texttt{\ dev\ } branch of the (UNFORKED!) development repository
\texttt{\ https://github.com/cnaak/blindex.typ\ } .

\subsection{Low-Level Indexing
Command}\label{low-level-indexing-command}

The \texttt{\ blindex\ } library has a low-level, index entry marking
function \texttt{\ \#blindex(abrv,\ lang,\ entry)\ } , whose arguments
are (abbreviation, language, entry), as in:

\begin{Shaded}
\begin{Highlighting}[]
\NormalTok{"the citation..." \#blindex("1Thess", "en", [1.1{-}{-}3]) citation\textquotesingle{}s typesetting...}
\end{Highlighting}
\end{Shaded}

Following the usual index making strategy in Typst, this user
\texttt{\ \#blindex\ } command only adds the index-marking
\texttt{\ \#metadata\ } in the document, without producing any visible
typeset output.

Biblical literature index listings can be generated (typeset) in
arbitrary amounts and locations throughout the document, just by calling
the user \texttt{\ \#mkIndex\ } command:

\begin{Shaded}
\begin{Highlighting}[]
\NormalTok{\#mkIndex()}
\end{Highlighting}
\end{Shaded}

Optional arguments control style and sorting convention parameters, as
exemplified below.

\subsection{Higher-Level Quoting-Indexing
Commands}\label{higher-level-quoting-indexing-commands}

The library also offers higher-level functions to assemble the entire
(i) citation typesetting, (ii) index entry, (iii) citation typesetting,
and (iv) bibliography entrying (with some typesetting (styling)
options), of the passage. Such commands are \texttt{\ \#iQuot(...)\ }
and \texttt{\ \#bQuot(...)\ } , respectively for \textbf{inline} and
\textbf{block} quoting of Biblical literature, with automatic indexing
and bibliography citation. Mandatory arguments are identical for either
command:

\begin{Shaded}
\begin{Highlighting}[]
\NormalTok{paragraph text...}
\NormalTok{\#iQuot(body, abrv, lang, pssg, version, cited)}
\NormalTok{more text...}

\NormalTok{// Displayed block quote of Biblical literature:}
\NormalTok{\#bQuot(body, abrv, lang, pssg, version, cited)}
\end{Highlighting}
\end{Shaded}

In which:

\begin{itemize}
\tightlist
\item
  \texttt{\ body\ } ( \texttt{\ content\ } ) is the quoted biblical
  literature text;
\item
  \texttt{\ abrv\ } ( \texttt{\ string\ } ) is the book abbreviation
  according to the
\item
  \texttt{\ lang\ } ( \texttt{\ string\ } ) language-variant (see
  above);
\item
  \texttt{\ pssg\ } ( \texttt{\ content\ } ) is the quoted text passage
  â€'' usually chapter and verses â€'' as they will appear in the text
  and in the biblical literature index;
\item
  \texttt{\ version\ } ( \texttt{\ string\ } ) is a translation
  identifier, such as \texttt{\ "LXX"\ } , or \texttt{\ "KJV"\ } ; and
\item
  \texttt{\ cited\ } ( \texttt{\ label\ } ) is the corresponding
  bibliography entry label, which can be constructed through:
\end{itemize}

\texttt{\ label("bib-key")\ } , where \texttt{\ bib-key\ } is the
bibliographic entry key, in the bibliography database â€'' whether
\texttt{\ bibTeX\ } or \texttt{\ Hayagriva\ } .

\subsection{Higher-Level Example}\label{higher-level-example}

\begin{Shaded}
\begin{Highlighting}[]
\NormalTok{\#set page(paper: "a7", fill: rgb("\#eec"))}
\NormalTok{\#import "@preview/blindex:0.1.0": *}

\NormalTok{The Septuagint (LXX) starts with \#iQuot([ΕΝ ἀρχῇ ἐποίησεν ὁ Θεὸς τὸν οὐρανὸν καὶ τὴν γῆν.],}
\NormalTok{"Gen", "en", [1.1], "LXX", label("2012{-}LXX{-}SBB")).}

\NormalTok{\#pagebreak()}

\NormalTok{Moreover, the book of Odes begins with: \#iQuot([ᾠδὴ Μωυσέως ἐν τῇ ἐξόδῳ], "Ode", "en", [1.0],}
\NormalTok{"LXX", label("2012{-}LXX{-}SBB")).}

\NormalTok{\#pagebreak()}

\NormalTok{= Biblical Citations}
\NormalTok{Books are sorted following the LXX ordering.}

\NormalTok{\#mkIndex(cols: 1, sorting{-}tradition: "LXX")}

\NormalTok{\#pagebreak()}

\NormalTok{\#bibliography("test{-}01{-}readme.yml", title: "References", style: "ieee")}
\end{Highlighting}
\end{Shaded}

The listing of the bibliography file, \texttt{\ test-01-readme.yml\ } ,
as shown in the example, is:

\begin{Shaded}
\begin{Highlighting}[]
\FunctionTok{2012{-}LXX{-}SBB}\KeywordTok{:}
\AttributeTok{  }\FunctionTok{type}\KeywordTok{:}\AttributeTok{ book}
\AttributeTok{  }\FunctionTok{title}\KeywordTok{:}
\AttributeTok{    }\FunctionTok{value}\KeywordTok{:}\AttributeTok{ }\StringTok{"Septuaginta: Edição Acadêmica Capa dura – Edição de luxo"}
\AttributeTok{    }\FunctionTok{sentence{-}case}\KeywordTok{:}\AttributeTok{ }\StringTok{"Septuaginta: edição acadêmica capa dura – edição de luxo"}
\AttributeTok{    }\FunctionTok{short}\KeywordTok{:}\AttributeTok{ Septuaginta}
\AttributeTok{  }\FunctionTok{publisher}\KeywordTok{:}\AttributeTok{ Sociedade Bíblica do Brasil, SBB}
\AttributeTok{  }\FunctionTok{editor}\KeywordTok{:}\AttributeTok{ Rahlfs, Alfred}
\AttributeTok{  }\FunctionTok{affiliated}\KeywordTok{:}
\AttributeTok{    }\KeywordTok{{-}}\AttributeTok{ }\FunctionTok{role}\KeywordTok{:}\AttributeTok{ collaborator}
\AttributeTok{      }\FunctionTok{names}\KeywordTok{:}\AttributeTok{ }\KeywordTok{[}\AttributeTok{ }\StringTok{"Hanhart, Robert"}\KeywordTok{,}\AttributeTok{ }\KeywordTok{]}
\AttributeTok{  }\FunctionTok{pages}\KeywordTok{:}\AttributeTok{ }\DecValTok{2240}
\AttributeTok{  }\FunctionTok{date}\KeywordTok{:}\AttributeTok{ 2012{-}01{-}11}
\AttributeTok{  }\FunctionTok{edition}\KeywordTok{:}\AttributeTok{ }\DecValTok{1}
\AttributeTok{  }\FunctionTok{ISBN}\KeywordTok{:}\AttributeTok{ 978{-}3438052278}
\AttributeTok{  }\FunctionTok{language}\KeywordTok{:}\AttributeTok{ el}
\end{Highlighting}
\end{Shaded}

This example results in a 4-page document like this one:

\pandocbounded{\includegraphics[keepaspectratio]{https://raw.githubusercontent.com/cnaak/blindex.typ/55d275e4fdab1f47c13e1fe01cbb2b397de5e0fb/thumbnail.png}}

\subsection{Citing}\label{citing}

This package can be cited with the following bibliography database
entry:

\begin{Shaded}
\begin{Highlighting}[]
\FunctionTok{blindex{-}package}\KeywordTok{:}
\AttributeTok{  }\FunctionTok{type}\KeywordTok{:}\AttributeTok{ Web}
\AttributeTok{  }\FunctionTok{author}\KeywordTok{:}\AttributeTok{ Naaktgeboren, C.}
\AttributeTok{  }\FunctionTok{title}\KeywordTok{:}
\AttributeTok{    }\FunctionTok{value}\KeywordTok{:}\AttributeTok{ }\StringTok{"Blindex: Index{-}making of Biblical literature citations in Typst"}
\AttributeTok{    }\FunctionTok{short}\KeywordTok{:}\AttributeTok{ }\StringTok{"Blindex: Index{-}making in Typst"}
\AttributeTok{  }\FunctionTok{url}\KeywordTok{:}\AttributeTok{ https://github.com/cnaak/blindex.typ}
\AttributeTok{  }\FunctionTok{version}\KeywordTok{:}\AttributeTok{ }\FloatTok{0.1.0}
\AttributeTok{  }\FunctionTok{date}\KeywordTok{:}\AttributeTok{ 2024{-}08}
\end{Highlighting}
\end{Shaded}

\subsubsection{How to add}\label{how-to-add}

Copy this into your project and use the import as \texttt{\ blindex\ }

\begin{verbatim}
#import "@preview/blindex:0.1.0"
\end{verbatim}

\includesvg[width=0.16667in,height=0.16667in]{/assets/icons/16-copy.svg}

Check the docs for
\href{https://typst.app/docs/reference/scripting/\#packages}{more
information on how to import packages} .

\subsubsection{About}\label{about}

\begin{description}
\tightlist
\item[Author :]
Naaktgeboren, C.
\item[License:]
MIT
\item[Current version:]
0.1.0
\item[Last updated:]
August 14, 2024
\item[First released:]
August 14, 2024
\item[Minimum Typst version:]
0.11.1
\item[Archive size:]
11.1 kB
\href{https://packages.typst.org/preview/blindex-0.1.0.tar.gz}{\pandocbounded{\includesvg[keepaspectratio]{/assets/icons/16-download.svg}}}
\item[Discipline :]
\begin{itemize}
\tightlist
\item[]
\item
  \href{https://typst.app/universe/search/?discipline=theology}{Theology}
\end{itemize}
\item[Categor ies :]
\begin{itemize}
\tightlist
\item[]
\item
  \pandocbounded{\includesvg[keepaspectratio]{/assets/icons/16-list-unordered.svg}}
  \href{https://typst.app/universe/search/?category=model}{Model}
\item
  \pandocbounded{\includesvg[keepaspectratio]{/assets/icons/16-code.svg}}
  \href{https://typst.app/universe/search/?category=scripting}{Scripting}
\end{itemize}
\end{description}

\subsubsection{Where to report issues?}\label{where-to-report-issues}

This package is a project of Naaktgeboren, C. . You can also try to ask
for help with this package on the \href{https://forum.typst.app}{Forum}
.

Please report this package to the Typst team using the
\href{https://typst.app/contact}{contact form} if you believe it is a
safety hazard or infringes upon your rights.

\phantomsection\label{versions}
\subsubsection{Version history}\label{version-history}

\begin{longtable}[]{@{}ll@{}}
\toprule\noalign{}
Version & Release Date \\
\midrule\noalign{}
\endhead
\bottomrule\noalign{}
\endlastfoot
0.1.0 & August 14, 2024 \\
\end{longtable}

Typst GmbH did not create this package and cannot guarantee correct
functionality of this package or compatibility with any version of the
Typst compiler or app.
