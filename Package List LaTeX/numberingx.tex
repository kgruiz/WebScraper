\title{typst.app/universe/package/numberingx}

\phantomsection\label{banner}
\section{numberingx}\label{numberingx}

{ 0.0.1 }

Extended numbering patterns using the CSS Counter Styles spec

\phantomsection\label{readme}
\emph{Extended numbering patterns using the
\href{https://www.w3.org/TR/css-counter-styles-3/}{CSS Counter Styles}
specification, along with a number of
\href{https://www.w3.org/TR/predefined-counter-styles/}{Ready-made
Counter Styles} .}

\subsection{Usage}\label{usage}

\begin{Shaded}
\begin{Highlighting}[]
\CommentTok{// numberingx is expected to be imported with the syntax creating a named module}
\NormalTok{\#}\ImportTok{import} \StringTok{"@preview/numberingx:0.0.1"}

\CommentTok{// Use full{-}width roman numerals for titles, and lowercase ukrainian letters}
\NormalTok{\#set }\FunctionTok{heading}\NormalTok{(numbering}\OperatorTok{:}\NormalTok{ numberingx}\OperatorTok{.}\FunctionTok{formatter}\NormalTok{(}
  \StringTok{"\{fullwidth{-}upper{-}roman\}.\{fullwidth{-}lower{-}roman\}.\{lower{-}ukrainian\}"}
\NormalTok{))}
\end{Highlighting}
\end{Shaded}

\subsubsection{Patterns}\label{patterns}

numberingx’s patterns are similiar to typst’s
\href{https://typst.app/docs/reference/meta/numbering/}{numbering
patterns} and use the same notion of fragments with a prefix and a final
suffix. The main difference is that it doesn’t use special characters
and all numbering styles must be written within braces. To insert a
literal brace, you can double it.

A list of patterns can be found in the
\href{https://www.w3.org/TR/predefined-counter-styles/}{Ready-made
Counter Styles} document. Additionally, numberingx allows typst’s
numbering characters to be used in patterns. This way,
\texttt{\ "\{upper-roman\}.\{decimal\})"\ } can be shortened to
\texttt{\ "\{I\}.\{1\})"\ } .

\subsubsection{API}\label{api}

numberingx exposes two functions, \texttt{\ format\ } and
\texttt{\ formatter\ } .

\paragraph{\texorpdfstring{\texttt{\ format(fmt,\ styles:\ (:),\ ..nums)\ }}{ format(fmt, styles: (:), ..nums) }}\label{formatfmt-styles-..nums}

This function uses the same api as typst’s \texttt{\ numbering()\ }
and takes the pattern string as its first positional argument, and
numbers as trailing arguments. An optional \texttt{\ styles\ } argument
allows for
\href{https://github.com/typst/packages/raw/main/packages/preview/numberingx/0.0.1/\#user-defined-styles}{user-defined
styles} .

\paragraph{\texorpdfstring{\texttt{\ formatter(fmt,\ styles:\ (:))\ }}{ formatter(fmt, styles: (:)) }}\label{formatterfmt-styles}

This function is little more than a shorter version of
\texttt{\ format.with(..)\ } . It takes a pattern string and an optional
\texttt{\ styles\ } argument, and return the matching numbering
functions. This is mainly intended to be used for \texttt{\ \#set\ }
rules.

\subsection{User-defined styles}\label{user-defined-styles}

Custom styles can be defined according to the
\href{https://www.w3.org/TR/css-counter-styles-3/}{CSS Counter Styles}
spec and passed through a \texttt{\ styles\ } named argument to
\texttt{\ format\ } and \texttt{\ formatter\ } . It must be a dictionary
mapping style names to style descriptions.

Note that the \texttt{\ prefix\ } , \texttt{\ suffix\ } ,
\texttt{\ pad\ } , and \texttt{\ speak-as\ } descriptors are not
supported, nor is the \texttt{\ extends\ } system.

\subsection{License}\label{license}

This repository is licensed under
\href{https://spdx.org/licenses/MIT-0.html}{MIT-0} , which is the
closest I’m legally allowed to public domain while being OSI approved.

\subsubsection{How to add}\label{how-to-add}

Copy this into your project and use the import as
\texttt{\ numberingx\ }

\begin{verbatim}
#import "@preview/numberingx:0.0.1"
\end{verbatim}

\includesvg[width=0.16667in,height=0.16667in]{/assets/icons/16-copy.svg}

Check the docs for
\href{https://typst.app/docs/reference/scripting/\#packages}{more
information on how to import packages} .

\subsubsection{About}\label{about}

\begin{description}
\tightlist
\item[Author :]
Edhebi
\item[License:]
MIT-0
\item[Current version:]
0.0.1
\item[Last updated:]
July 21, 2023
\item[First released:]
July 21, 2023
\item[Archive size:]
13.9 kB
\href{https://packages.typst.org/preview/numberingx-0.0.1.tar.gz}{\pandocbounded{\includesvg[keepaspectratio]{/assets/icons/16-download.svg}}}
\item[Repository:]
\href{https://github.com/edhebi/numberingx}{GitHub}
\end{description}

\subsubsection{Where to report issues?}\label{where-to-report-issues}

This package is a project of Edhebi . Report issues on
\href{https://github.com/edhebi/numberingx}{their repository} . You can
also try to ask for help with this package on the
\href{https://forum.typst.app}{Forum} .

Please report this package to the Typst team using the
\href{https://typst.app/contact}{contact form} if you believe it is a
safety hazard or infringes upon your rights.

\phantomsection\label{versions}
\subsubsection{Version history}\label{version-history}

\begin{longtable}[]{@{}ll@{}}
\toprule\noalign{}
Version & Release Date \\
\midrule\noalign{}
\endhead
\bottomrule\noalign{}
\endlastfoot
0.0.1 & July 21, 2023 \\
\end{longtable}

Typst GmbH did not create this package and cannot guarantee correct
functionality of this package or compatibility with any version of the
Typst compiler or app.
