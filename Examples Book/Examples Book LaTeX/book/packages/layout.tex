\title{sitandr.github.io/typst-examples-book/book/packages/layout}

\section{\texorpdfstring{\hyperref[layouting]{Layouting}}{Layouting}}\label{layouting}

General useful things.

\subsection{\texorpdfstring{\hyperref[pinit-relative-place-by-pins]{Pinit:
relative place by
pins}}{Pinit: relative place by pins}}\label{pinit-relative-place-by-pins}

The idea of \href{https://github.com/OrangeX4/typst-pinit}{pinit} is
pinning pins on the normal flow of the text, and then placing the
content relative to pins.

\begin{verbatim}
#import "@preview/pinit:0.1.3": *
#set page(height: 6em, width: 20em)

#set text(size: 24pt)

A simple #pin(1)highlighted text#pin(2).

#pinit-highlight(1, 2)

#pinit-point-from(2)[It is simple.]
\end{verbatim}

\pandocbounded{\includesvg[keepaspectratio]{typst-img/b0a3a289ec65a00a9b39e0689578c9c139a65d1d9f379fa1593ba8ea9268af25-1.svg}}

More complex example:

\begin{verbatim}
#import "@preview/pinit:0.1.3": *

// Pages
#set page(paper: "presentation-4-3")
#set text(size: 20pt)
#show heading: set text(weight: "regular")
#show heading: set block(above: 1.4em, below: 1em)
#show heading.where(level: 1): set text(size: 1.5em)

// Useful functions
#let crimson = rgb("#c00000")
#let greybox(..args, body) = rect(fill: luma(95%), stroke: 0.5pt, inset: 0pt, outset: 10pt, ..args, body)
#let redbold(body) = {
  set text(fill: crimson, weight: "bold")
  body
}
#let blueit(body) = {
  set text(fill: blue)
  body
}

// Main body
#block[
  = Asymptotic Notation: $O$

  Use #pin("h1")asymptotic notations#pin("h2") to describe asymptotic efficiency of algorithms.
  (Ignore constant coefficients and lower-order terms.)

  #greybox[
    Given a function $g(n)$, we denote by $O(g(n))$ the following *set of functions*:
    #redbold(${f(n): "exists" c > 0 "and" n_0 > 0, "such that" f(n) <= c dot g(n) "for all" n >= n_0}$)
  ]

  #pinit-highlight("h1", "h2")

  $f(n) = O(g(n))$: #pin(1)$f(n)$ is *asymptotically smaller* than $g(n)$.#pin(2)

  $f(n) redbold(in) O(g(n))$: $f(n)$ is *asymptotically* #redbold[at most] $g(n)$.

  #pinit-line(stroke: 3pt + crimson, start-dy: -0.25em, end-dy: -0.25em, 1, 2)

  #block[Insertion Sort as an #pin("r1")example#pin("r2"):]

  - Best Case: $T(n) approx c n + c' n - c''$ #pin(3)
  - Worst case: $T(n) approx c n + (c' \/ 2) n^2 - c''$ #pin(4)

  #pinit-rect("r1", "r2")

  #pinit-place(3, dx: 15pt, dy: -15pt)[#redbold[$T(n) = O(n)$]]
  #pinit-place(4, dx: 15pt, dy: -15pt)[#redbold[$T(n) = O(n)$]]

  #blueit[Q: Is $n^(3) = O(n^2)$#pin("que")? How to prove your answer#pin("ans")?]

  #pinit-point-to("que", fill: crimson, redbold[No.])
  #pinit-point-from("ans", body-dx: -150pt)[
    Show that the equation $(3/2)^n >= c$ \
    has infinitely many solutions for $n$.
  ]
]
\end{verbatim}

\pandocbounded{\includesvg[keepaspectratio]{typst-img/4cc4ac1de81450b49f618408d35cd551858a4fcee317859f7f2a5d84482a9612-1.svg}}

\subsection{\texorpdfstring{\hyperref[margin-notes]{Margin
notes}}{Margin notes}}\label{margin-notes}

\begin{verbatim}
#import "@preview/drafting:0.1.1": *

#let (l-margin, r-margin) = (1in, 2in)
#set page(
  margin: (left: l-margin, right: r-margin, rest: 0.1in),
)
#set-page-properties(margin-left: l-margin, margin-right: r-margin)

= Margin Notes
== Setup
Unfortunately `typst` doesn't expose margins to calling functions, so you'll need to set them explicitly. This is done using `set-page-properties` *before you place any content*:

// At the top of your source file
// Of course, you can substitute any margin numbers you prefer
// provided the page margins match what you pass to `set-page-properties`

== The basics
#lorem(20)
#margin-note(side: left)[Hello, world!]
#lorem(10)
#margin-note[Hello from the other side]

#lorem(25)
#margin-note[When notes are about to overlap, they're automatically shifted]
#margin-note(stroke: aqua + 3pt)[To avoid collision]
#lorem(25)

#let caution-rect = rect.with(inset: 1em, radius: 0.5em, fill: orange.lighten(80%))
#inline-note(rect: caution-rect)[
  Be aware that notes will stop automatically avoiding collisions if 4 or more notes
  overlap. This is because `typst` warns when the layout doesn't resolve after 5 attempts
  (initial layout + adjustment for each note)
]
\end{verbatim}

\pandocbounded{\includesvg[keepaspectratio]{typst-img/80c65cf70b8da549afe447ce97f6dc71087cc0654dd85cd4f5e95bea388e3179-1.svg}}

\begin{verbatim}
#import "@preview/drafting:0.1.1": *

#let (l-margin, r-margin) = (1in, 2in)
#set page(
  margin: (left: l-margin, right: r-margin, rest: 0.1in),
)
#set-page-properties(margin-left: l-margin, margin-right: r-margin)

== Adjusting the default style
All function defaults are customizable through updating the module state:

#lorem(4) #margin-note(dy: -2em)[Default style]
#set-margin-note-defaults(stroke: orange, side: left)
#lorem(4) #margin-note[Updated style]


Even deeper customization is possible by overriding the default `rect`:

#import "@preview/colorful-boxes:1.1.0": stickybox

#let default-rect(stroke: none, fill: none, width: 0pt, content) = {
  stickybox(rotation: 30deg, width: width/1.5, content)
}
#set-margin-note-defaults(rect: default-rect, stroke: none, side: right)

#lorem(20)
#margin-note(dy: -25pt)[Why not use sticky notes in the margin?]

// Undo changes from last example
#set-margin-note-defaults(rect: rect, stroke: red)

== Multiple document reviewers
#let reviewer-a = margin-note.with(stroke: blue)
#let reviewer-b = margin-note.with(stroke: purple)
#lorem(20)
#reviewer-a[Comment from reviewer A]
#lorem(15)
#reviewer-b(side: left)[Comment from reviewer B]

== Inline Notes
#lorem(10)
#inline-note[The default inline note will split the paragraph at its location]
#lorem(10)
/*
// Should work, but doesn't? Created an issue in repo.
#inline-note(par-break: false, stroke: (paint: orange, dash: "dashed"))[
  But you can specify `par-break: false` to prevent this
]
*/
#lorem(10)
\end{verbatim}

\pandocbounded{\includesvg[keepaspectratio]{typst-img/282de769e728a8bdb9c78c665664b382ecbf59fd7d3c915fab67aae7055e2acb-1.svg}}

\begin{verbatim}
#import "@preview/drafting:0.1.1": *

#let (l-margin, r-margin) = (1in, 2in)
#set page(
  margin: (left: l-margin, right: r-margin, rest: 0.1in),
)
#set-page-properties(margin-left: l-margin, margin-right: r-margin)

== Hiding notes for print preview
#set-margin-note-defaults(hidden: true)

#lorem(20)
#margin-note[This will respect the global "hidden" state]
#margin-note(hidden: false, dy: -2.5em)[This note will never be hidden]

= Positioning
== Precise placement: rule grid
Need to measure space for fine-tuned positioning? You can use `rule-grid` to cross-hatch
the page with rule lines:

#rule-grid(width: 10cm, height: 3cm, spacing: 20pt)
#place(
  dx: 180pt,
  dy: 40pt,
  rect(fill: white, stroke: red, width: 1in, "This will originate at (180pt, 40pt)")
)

// Optionally specify divisions of the smallest dimension to automatically calculate
// spacing
#rule-grid(dx: 10cm + 3em, width: 3cm, height: 1.2cm, divisions: 5, square: true,  stroke: green)

// The rule grid doesn't take up space, so add it explicitly
#v(3cm + 1em)

== Absolute positioning
What about absolutely positioning something regardless of margin and relative location? `absolute-place` is your friend. You can put content anywhere:

#context {
  let (dx, dy) = (25%, here().position().y)
  let content-str = (
    "This absolutely-placed box will originate at (" + repr(dx) + ", " + repr(dy) + ")"
    + " in page coordinates"
  )
  absolute-place(
    dx: dx, dy: dy,
    rect(
      fill: green.lighten(60%),
      radius: 0.5em,
      width: 2.5in,
      height: 0.5in,
      [#align(center + horizon, content-str)]
    )
  )
}
#v(1in)

The "rule-grid" also supports absolute placement at the top-left of the page by passing `relative: false`. This is helpful for "rule"-ing the whole page.
\end{verbatim}

\pandocbounded{\includesvg[keepaspectratio]{typst-img/212dfc0f37bc9749e459085bb305f46a1db7ab3fbb22760f62ec58e349837d9e-1.svg}}

\subsection{\texorpdfstring{\hyperref[dropped-capitals]{Dropped
capitals}}{Dropped capitals}}\label{dropped-capitals}

\begin{quote}
Get more info
\href{https://github.com/EpicEricEE/typst-plugins/tree/master/droplet}{here}
\end{quote}

\subsubsection{\texorpdfstring{\hyperref[basic-usage]{Basic
usage}}{Basic usage}}\label{basic-usage}

\begin{verbatim}
#import "@preview/droplet:0.1.0": dropcap

#dropcap(gap: -2pt, hanging-indent: 8pt)[
  #lorem(42)
]
\end{verbatim}

\pandocbounded{\includesvg[keepaspectratio]{typst-img/a9c411d628d90aa8313aa9f0829bfdf43122c4532ad0d9d323a64b989a049d64-1.svg}}

\subsubsection{\texorpdfstring{\hyperref[extended-styling]{Extended
styling}}{Extended styling}}\label{extended-styling}

\begin{verbatim}
#import "@preview/droplet:0.1.0": dropcap

#dropcap(
  height: 2,
  justify: true,
  gap: 6pt,
  transform: letter => style(styles => {
    let height = measure(letter, styles).height

    grid(columns: 2, gutter: 6pt,
      align(center + horizon, text(blue, letter)),
      // Use "place" to ignore the line's height when
      // the font size is calculated later on.
      place(horizon, line(
        angle: 90deg,
        length: height + 6pt,
        stroke: blue.lighten(40%) + 1pt
      )),
    )
  })
)[
  #lorem(42)
]
\end{verbatim}

\pandocbounded{\includesvg[keepaspectratio]{typst-img/50d7ee4ffb1e61856535409373b040d579ab05734f3f304a4dc15f23361fd710-1.svg}}

\subsection{\texorpdfstring{\hyperref[headings-for-actual-current-chapter]{Headings
for actual current
chapter}}{Headings for actual current chapter}}\label{headings-for-actual-current-chapter}

\begin{quote}
See \href{https://github.com/tingerrr/hydra}{hydra}
\end{quote}

\begin{verbatim}
#import "@preview/hydra:0.2.0": hydra

#set page(header: hydra() + line(length: 100%))
#set heading(numbering: "1.1")
#show heading.where(level: 1): it => pagebreak(weak: true) + it

= Introduction
#lorem(750)

= Content
== First Section
#lorem(500)
== Second Section
#lorem(250)
== Third Section
#lorem(500)

= Annex
#lorem(10)
\end{verbatim}

\pandocbounded{\includesvg[keepaspectratio]{typst-img/ab3a07e72c5e19f28936f6d2249c9c5bcd102a27f4af177db63d7715c5c64f33-1.svg}}

\pandocbounded{\includesvg[keepaspectratio]{typst-img/ab3a07e72c5e19f28936f6d2249c9c5bcd102a27f4af177db63d7715c5c64f33-2.svg}}

\pandocbounded{\includesvg[keepaspectratio]{typst-img/ab3a07e72c5e19f28936f6d2249c9c5bcd102a27f4af177db63d7715c5c64f33-3.svg}}

\pandocbounded{\includesvg[keepaspectratio]{typst-img/ab3a07e72c5e19f28936f6d2249c9c5bcd102a27f4af177db63d7715c5c64f33-4.svg}}

\pandocbounded{\includesvg[keepaspectratio]{typst-img/ab3a07e72c5e19f28936f6d2249c9c5bcd102a27f4af177db63d7715c5c64f33-5.svg}}
