\title{sitandr.github.io/typst-examples-book/book/print}

\section{\texorpdfstring{\hyperref[typst-examples-book]{Typst Examples
Book}}{Typst Examples Book}}\label{typst-examples-book}

This book provides an extended \emph{tutorial} and lots of
\href{https://github.com/typst/typst}{Typst} snippets that can help you
to write better Typst code.

This is an unofficial book. Some snippets \& suggestions here may be
outdated or useless (please let me know if you find some).

However, \emph{all of them should compile on last version of Typst
\textsuperscript{\hyperref[1]{1}}} .

\textbf{CAUTION:} the book is (probably forever) a \textbf{WIP} , so
don\textquotesingle t rely on it.

If you like it, consider
\href{https://github.com/sitandr/typst-examples-book}{giving a star on
GitHub} !

This will help me to stay motivated and continue working on this book.

\subsection{\texorpdfstring{\hyperref[navigation]{Navigation}}{Navigation}}\label{navigation}

The book consists of several chapters, each with its own goal:

\begin{enumerate}
\tightlist
\item
  \href{./basics/index.html}{Typst Basics}
\item
  \href{./snippets/index.html}{Typst Snippets}
\item
  \href{./packages/index.html}{Typst Packages}
\item
  \href{./typstonomicon/index.html}{Typstonomicon}
\end{enumerate}

\subsection{\texorpdfstring{\hyperref[contributions]{Contributions}}{Contributions}}\label{contributions}

Any contributions are very welcome! If you have a good code snippet that
you want to share, feel free to submit an issue with snippet or make a
PR in the
\href{https://github.com/sitandr/typst-examples-book}{repository} .

I will especially appreciate submissions of active community members and
compiler contributors!

However, I will also really appreciate feedback from beginners to make
the book as comprehensible as possible!

\subsection{\texorpdfstring{\hyperref[acknowledgements]{Acknowledgements}}{Acknowledgements}}\label{acknowledgements}

Thanks to everyone in the community who published their code snippets!

If someone doesn\textquotesingle t like their code and/or name being
published, please contact me.

\phantomsection\label{1}
\textsuperscript{1}

When a new version launches, it may take some time to update the book,
feel free to tag me to speed up the process.

\section{\texorpdfstring{\hyperref[typst-basics]{Typst
Basics}}{Typst Basics}}\label{typst-basics}

This is a chapter that consistently introduces you to the most things
you need to know when writing with Typst.

It show much more things than official tutorial, so maybe it will be
interesting to read for some of the experienced users too.

Some examples are taken from
\href{https://typst.app/docs/tutorial/}{Official Tutorial} and
\href{https://typst.app/docs/reference/}{Official Reference} . Most are
created and edited specially for this book.

\begin{quote}
\emph{Important:} in most cases there will be used "clipped" examples of
your rendered documents (no margins, smaller width and so on).

To set up the spacing as you want, see
\href{https://typst.app/docs/guides/page-setup-guide/}{Official Page
Setup Guide} .
\end{quote}

\section{\texorpdfstring{\hyperref[tutorial-by-examples]{Tutorial by
Examples}}{Tutorial by Examples}}\label{tutorial-by-examples}

The first section of Typst Basics is very similar to
\href{https://typst.app/docs/tutorial/}{Official Tutorial} , with more
specialized examples and less words. It is \emph{highly recommended to
read the official tutorial anyway} .

\section{\texorpdfstring{\hyperref[markup-language]{Markup
language}}{Markup language}}\label{markup-language}

\subsection{\texorpdfstring{\hyperref[starting]{Starting}}{Starting}}\label{starting}

\begin{verbatim}
Starting typing in Typst is easy.
You don't need packages or other weird things for most of things.

Blank line will move text to a new paragraph.

Btw, you can use any language and unicode symbols
without any problems as long as the font supports it: ßçœ̃ɛ̃ø∀αβёыა😆…
\end{verbatim}

\pandocbounded{\includesvg[keepaspectratio]{basics/tutorial/typst-img/ee9f64251c99c7aeaaf6fa1d5bc7e907c2d51a34aa38126544d515ca197ca2a8-1.svg}}

\subsection{\texorpdfstring{\hyperref[markup]{Markup}}{Markup}}\label{markup}

\begin{verbatim}
= Markup

This was a heading. Number of `=` in front of name corresponds to heading level.

== Second-level heading

Okay, let's move to _emphasis_ and *bold* text.

Markup syntax is generally similar to
`AsciiDoc` (this was `raw` for monospace text!)
\end{verbatim}

\pandocbounded{\includesvg[keepaspectratio]{basics/tutorial/typst-img/fa8b95f9b15083387a29c11d17efca9873b8e778643b1b5079aa137891d01c8d-1.svg}}

\subsection{\texorpdfstring{\hyperref[new-lines--escaping]{New lines \&
Escaping}}{New lines \& Escaping}}\label{new-lines--escaping}

\begin{verbatim}
You can break \
line anywhere you \
want using "\\" symbol.

Also you can use that symbol to
escape \_all the symbols you want\_,
if you don't want it to be interpreted as markup
or other special symbols.
\end{verbatim}

\pandocbounded{\includesvg[keepaspectratio]{basics/tutorial/typst-img/4dabdee2a61e7d10773d51772dba3665271a09d4d5df4a8f66dd80589f0bcd7a-1.svg}}

\subsection{\texorpdfstring{\hyperref[comments--codeblocks]{Comments \&
codeblocks}}{Comments \& codeblocks}}\label{comments--codeblocks}

\begin{verbatim}
You can write comments with `//` and `/* comment */`:
// Like this
/* Or even like
this */

```typ
Just in case you didn't read source,
this is how it is written:

// Like this
/* Or even like
this */

By the way, I'm writing it all in a _fenced code block_ with *syntax highlighting*!
```
\end{verbatim}

\pandocbounded{\includesvg[keepaspectratio]{basics/tutorial/typst-img/a481d12b3ed0bbe2d9db6cc4b4a1237cba9936de83333254dfce8702832db125-1.svg}}

\subsection{\texorpdfstring{\hyperref[smart-quotes]{Smart
quotes}}{Smart quotes}}\label{smart-quotes}

\begin{verbatim}
== What else?

There are not much things in basic "markup" syntax,
but we will see much more interesting things very soon!
I hope you noticed auto-matched "smart quotes" there.
\end{verbatim}

\pandocbounded{\includesvg[keepaspectratio]{basics/tutorial/typst-img/89114a6e9af45c2eb9db2ef44d0e5ba41e31bf816e72803bd1a9a02120e69fc3-1.svg}}

\subsection{\texorpdfstring{\hyperref[lists]{Lists}}{Lists}}\label{lists}

\begin{verbatim}
- Writing lists in a simple way is great.
- Nothing complex, start your points with `-`
  and this will become a list.
  - Indented lists are created via indentation.

+ Numbered lists start with `+` instead of `-`.
+ There is no alternative markup syntax for lists
+ So just remember `-` and `+`, all other symbols
  wouldn't work in an unintended way.
  + That is a general property of Typst's markup.
  + Unlike Markdown, there is only one way
    to write something with it.
\end{verbatim}

\pandocbounded{\includesvg[keepaspectratio]{basics/tutorial/typst-img/ad4e424e067a4362e9f145c0c4ba4b7c1b65e17e7d0e7631b6836841607ef85e-1.svg}}

\textbf{Notice:}

\begin{verbatim}
Typst numbered lists differ from markdown-like syntax for lists. If you write them by hand, numbering is preserved:

1. Apple
1. Orange
1. Peach
\end{verbatim}

\pandocbounded{\includesvg[keepaspectratio]{basics/tutorial/typst-img/477695c86becc136dceb144e90c0acd2b75faa2a49743f8673d09974b71da324-1.svg}}

\subsection{\texorpdfstring{\hyperref[math]{Math}}{Math}}\label{math}

\begin{verbatim}
I will just mention math ($a + b/c = sum_i x^i$)
is possible and quite pretty there:

$
7.32 beta +
  sum_(i=0)^nabla
    (Q_i (a_i - epsilon)) / 2
$

To learn more about math, see corresponding chapter.
\end{verbatim}

\pandocbounded{\includesvg[keepaspectratio]{basics/tutorial/typst-img/12cc318c8438cd8e91706013bbd53fee5ee004620a63348cfe2d7dcc3b8a19d4-1.svg}}

\section{\texorpdfstring{\hyperref[functions]{Functions}}{Functions}}\label{functions}

\subsection{\texorpdfstring{\hyperref[functions-1]{Functions}}{Functions}}\label{functions-1}

\begin{verbatim}
Okay, let's now move to more complex things.

First of all, there are *lots of magic* in Typst.
And it major part of it is called "scripting".

To go to scripting mode, type `#` and *some function name*
after that. We will start with _something dull_:

#lorem(50)

_That *function* just generated 50 "Lorem Ipsum" words!_
\end{verbatim}

\pandocbounded{\includesvg[keepaspectratio]{basics/tutorial/typst-img/036fce36d10e06e8e41be8e77d7d5672f5dfc82c57e7c3ba9b8060d0822ca115-1.svg}}

\subsection{\texorpdfstring{\hyperref[more-functions]{More
functions}}{More functions}}\label{more-functions}

\begin{verbatim}
#underline[functions can do everything!]

#text(orange)[L]ike #text(size: 0.8em)[Really] #sub[E]verything!

#figure(
  caption: [
    This is a screenshot from one of first theses written in Typst. \
    _All these things are written with #text(blue)[custom functions] too._
  ],
  image("../boxes.png", width: 80%)
)

In fact, you can #strong[forget] about markup
and #emph[just write] functions everywhere!

#list[
  All that markup is just a #emph[syntax sugar] over functions!
]
\end{verbatim}

\pandocbounded{\includesvg[keepaspectratio]{basics/tutorial/typst-img/455e15e83c25259f932178d68517cc012432cb17d072e60c659169470fe191ce-1.svg}}

\subsection{\texorpdfstring{\hyperref[how-to-call-functions]{How to call
functions}}{How to call functions}}\label{how-to-call-functions}

\begin{verbatim}
First, start with `#`. Then write the name.
Finally, write some parentheses and maybe something inside.

You can navigate lots of built-in functions
in #link("https://typst.app/docs/reference/")[Official Reference].

#quote(block: true, attribution: "Typst Examples Book")[
  That's right, links, quotes and lots of
  other document elements are created with functions.
]
\end{verbatim}

\pandocbounded{\includesvg[keepaspectratio]{basics/tutorial/typst-img/4c63fde73bb1ad0afe1332ab68c5b540ec786c6352a76860f4398fec32034cf0-1.svg}}

\subsection{\texorpdfstring{\hyperref[function-arguments]{Function
arguments}}{Function arguments}}\label{function-arguments}

\begin{verbatim}
There are _two types_ of function arguments:

+ *Positional.* Like `50` in `lorem(50)`.
  Just write them in parentheses and it will be okay.
  If you have many, use commas.
+ *Named.* Like in `#quote(attribution: "Whoever")`.
  Write the value after a name and a colon.

If argument is named, it has some _default value_.
To find out what it is, see
#link("https://typst.app/docs/reference/")[Official Typst Reference].
\end{verbatim}

\pandocbounded{\includesvg[keepaspectratio]{basics/tutorial/typst-img/d66fb474260490595a207f06c687efcc85808701c39c2a6e8b686bc22ffde279-1.svg}}

\subsection{\texorpdfstring{\hyperref[content]{Content}}{Content}}\label{content}

\begin{verbatim}
The most "universal" type in Typst language is *content*.
Everything you write in the document becomes content.

#[
  But you can explicitly create it with
  _scripting mode_ and *square brackets*.

  There, in square brackets, you can use any markup
  functions or whatever you want.
]
\end{verbatim}

\pandocbounded{\includesvg[keepaspectratio]{basics/tutorial/typst-img/faf9d7cddd55e68f84d212013a52a724c2ad763f18d83221a99bbd380410d7d1-1.svg}}

\subsection{\texorpdfstring{\hyperref[markup-and-code-modes]{Markup and
code modes}}{Markup and code modes}}\label{markup-and-code-modes}

\begin{verbatim}
When you use `#`, you are "switching" to code mode.
When you use `[]`, you turn back:

// +-- going from markup (the default mode) to scripting for that function
// |                 +-- scripting mode: calling `text`, the last argument is markup
// |     first arg   |
// v     vvvvvvvvv   vvvv
   #rect(width: 5cm, text(red)[hello *world*])
//  ^^^^                       ^^^^^^^^^^^^^ just a markup argument for `text`
//  |
//  +-- calling `rect` in scripting mode, with two arguments: width and other content
\end{verbatim}

\pandocbounded{\includesvg[keepaspectratio]{basics/tutorial/typst-img/0cabe3da1eb49f805535fb1d7e34a0d6eb1a6c49227b0be98634c6965e892185-1.svg}}

\subsection{\texorpdfstring{\hyperref[passing-content-into-functions]{Passing
content into
functions}}{Passing content into functions}}\label{passing-content-into-functions}

\begin{verbatim}
So what are these square brackets after functions?

If you *write content right after
function, it will be passed as positional argument there*.

#quote(block: true)[
  So #text(red)[_that_] allows me to write
  _literally anything in things
  I pass to #underline[functions]!_
]
\end{verbatim}

\pandocbounded{\includesvg[keepaspectratio]{basics/tutorial/typst-img/686d2b2a361a60244452ce53bd37ebef0699e92cf962c477bfb62bafdc0f7241-1.svg}}

\subsection{\texorpdfstring{\hyperref[passing-content-part-ii]{Passing
content, part
II}}{Passing content, part II}}\label{passing-content-part-ii}

\begin{verbatim}
So, just to make it clear, when I write

```typ
- #text(red)[red text]
- #text([red text], red)
- #text("red text", red)
//      ^        ^
// Quotes there mean a plain string, not a content!
// This is just text.
```

It all will result in a #text([red text], red).
\end{verbatim}

\pandocbounded{\includesvg[keepaspectratio]{basics/tutorial/typst-img/4686939b6d0932f1ebebac4111d8f02919dbc16446def7855c521d8dbf293689-1.svg}}

\section{\texorpdfstring{\hyperref[basic-styling]{Basic
styling}}{Basic styling}}\label{basic-styling}

\subsection{\texorpdfstring{\hyperref[set-rule]{\texttt{\ }{\texttt{\ Set\ }}\texttt{\ }
rule}}{  Set   rule}}\label{set-rule}

\begin{verbatim}
#set page(width: 15cm, margin: (left: 4cm, right: 4cm))

That was great, but using functions everywhere, especially
with many arguments every time is awfully cumbersome.

That's why Typst has _rules_. No, not for you, for the document.

#set par(justify: true)

And the first rule we will consider there is `set` rule.
As you see, I've just used it on `par` (which is short from paragraph)
and now all paragraphs became _justified_.

It will apply to all paragraphs after the rule,
but will work only in it's _scope_ (we will discuss them later).

#par(justify: false)[
  Of course, you can override a `set` rule.
  This rule just sets the _default value_
  of an argument of an element.
]

By the way, at first line of this snippet
I've reduced page size to make justifying more visible,
also increasing margins to add blank space on left and right.
\end{verbatim}

\pandocbounded{\includesvg[keepaspectratio]{basics/tutorial/typst-img/cee42a8b1274afa36891438d4b1611eb55b2cd8bb4546df47128a7d3eb66653b-1.svg}}

\subsection{\texorpdfstring{\hyperref[a-bit-about-length-units]{A bit
about length
units}}{A bit about length units}}\label{a-bit-about-length-units}

\begin{verbatim}
Before we continue with rules, we should talk about length. There are several absolute length units in Typst:

#set rect(height: 1em)

#table(
  columns: 2,
  [Points], rect(width: 72pt),
  [Millimeters], rect(width: 25.4mm),
  [Centimeters], rect(width: 2.54cm),
  [Inches], rect(width: 1in),
  [Relative to font size], rect(width: 6.5em)
)

`1 em` = current font size. \
It is a very convenient unit,
so we are going to use it a lot
\end{verbatim}

\pandocbounded{\includesvg[keepaspectratio]{basics/tutorial/typst-img/5f8abc94a3d9df0e16f78c258e7f487d5698b4c96491300b3a48ad8e685534bc-1.svg}}

\subsection{\texorpdfstring{\hyperref[setting-something-else]{Setting
something else}}{Setting something else}}\label{setting-something-else}

Of course, you can use \texttt{\ }{\texttt{\ set\ }}\texttt{\ } rule
with all built-in functions and all their named arguments to make some
argument "default".

For example, let\textquotesingle s make all quotes in this snippet
authored by the book:

\begin{verbatim}
#set quote(block: true, attribution: [Typst Examples Book])

#quote[
  Typst is great!
]

#quote[
  The problem with quotes on the internet is
  that it is hard to verify their authenticity.
]
\end{verbatim}

\pandocbounded{\includesvg[keepaspectratio]{basics/tutorial/typst-img/c34c25cad05b7c20b6e0f146002886a1de65b61f48666cfec3d3494bd694a641-1.svg}}

\subsection{\texorpdfstring{\hyperref[opinionated-defaults]{Opinionated
defaults}}{Opinionated defaults}}\label{opinionated-defaults}

That allows you to set Typst default styling as you want it:

\begin{verbatim}
#set par(justify: true)
#set list(indent: 1em)
#set enum(indent: 1em)
#set page(numbering: "1")

- List item
- List item

+ Enum item
+ Enum item
\end{verbatim}

\pandocbounded{\includesvg[keepaspectratio]{basics/tutorial/typst-img/773d68bc55eb89f119ad07b882eae5fd31868d8a1bb3d4963573ec80fb1c7466-1.svg}}

Don\textquotesingle t complain about bad defaults!
\texttt{\ }{\texttt{\ Set\ }}\texttt{\ } your own.

\subsection{\texorpdfstring{\hyperref[numbering]{Numbering}}{Numbering}}\label{numbering}

\begin{verbatim}
= Numbering

Some of elements have a property called "numbering".
They accept so-called "numbering patterns" and
are very useful with set rules. Let's see what I mean.

#set heading(numbering: "I.1:")

= This is first level
= Another first
== Second
== Another second
=== Now third
== And second again
= Now returning to first
= These are actual romanian numerals
\end{verbatim}

\pandocbounded{\includesvg[keepaspectratio]{basics/tutorial/typst-img/39fb958032888b1e41da849152fed716b6f590eed3ea975b051ab786fac4ce5c-1.svg}}

Of course, there are lots of other cool properties that can be
\emph{set} , so feel free to dive into
\href{https://typst.app/docs/reference/}{Official Reference} and explore
them!

And now we are moving into something much more interesting\ldots{}

\section{\texorpdfstring{\hyperref[advanced-styling]{Advanced
styling}}{Advanced styling}}\label{advanced-styling}

\subsection{\texorpdfstring{\hyperref[the-show-rule]{The
\texttt{\ }{\texttt{\ show\ }}\texttt{\ }
rule}}{The   show   rule}}\label{the-show-rule}

\begin{verbatim}
Advanced styling comes with another rule. The _`show` rule_.

Now please compare the source code and the output.

#show "Be careful": strong[Play]

This is a very powerful thing, sometimes even too powerful.
Be careful with it.

#show "it is holding me hostage": text(green)[I'm fine]

Wait, what? I told you "Be careful!", not "Play!".

Help, it is holding me hostage.
\end{verbatim}

\pandocbounded{\includesvg[keepaspectratio]{basics/tutorial/typst-img/8a9ac38769d4ac7b42a2755047d0cd5a6404ad26e9e7f5b72b6984fa67abadf9-1.svg}}

\subsection{\texorpdfstring{\hyperref[now-a-bit-more-serious]{Now a bit
more serious}}{Now a bit more serious}}\label{now-a-bit-more-serious}

\begin{verbatim}
Show rule is a powerful thing that takes a _selector_
and what to apply to it. After that it will apply to
all elements it can find.

It may be extremely useful like that:

#show emph: set text(blue)

Now if I want to _emphasize_ something,
it will be both _emphasized_ and _blue_.
Isn't that cool?
\end{verbatim}

\pandocbounded{\includesvg[keepaspectratio]{basics/tutorial/typst-img/657acaf5c4ca684408bbc6fe0dec4c74b9fa58d24805ec975be1382aa7bf959c-1.svg}}

\subsection{\texorpdfstring{\hyperref[about-syntax]{About
syntax}}{About syntax}}\label{about-syntax}

\begin{verbatim}
Sometimes show rules may be confusing. They may seem very diverse, but in fact they all are quite the same! So

// actually, this is the same as
// redify = text.with(red)
// `with` creates a new function with this argument already set
#let redify(string) = text(red, string)

// and this is the same as
// framify = rect.with(stroke: orange)
#let framify(object) = rect(object, stroke: orange)

// set default color of text blue for all following text
#show: set text(blue)

Blue text.

// wrap everything into a frame
#show: framify

Framed text.

// it's the same, just creating new function that calls framify
#show: a => framify(a)

Double-framed.

// apply function to `the`
#show "the": redify
// set text color for all the headings
#show heading: set text(purple)

= Conclusion

All these rules do basically the same!
\end{verbatim}

\pandocbounded{\includesvg[keepaspectratio]{basics/tutorial/typst-img/2dfcde68345d3fa276b99a1f308343118c6eeae09fd106389a8fc488d7244ebb-1.svg}}

\subsection{\texorpdfstring{\hyperref[blocks]{Blocks}}{Blocks}}\label{blocks}

One of the most important usages is that you can set up all spacing
using blocks. Like every element with text contains text that can be set
up, every \emph{block element} contains blocks:

\begin{verbatim}
Text before
= Heading
Text after

#show heading: set block(spacing: 0.5em)

Text before
= Heading
Text after
\end{verbatim}

\pandocbounded{\includesvg[keepaspectratio]{basics/tutorial/typst-img/7891207932d0918c88b5804b3a7ee051ce5dda93081f8999eb0f7ebaee48400a-1.svg}}

\subsection{\texorpdfstring{\hyperref[selector]{Selector}}{Selector}}\label{selector}

\begin{verbatim}
So show rule can accept _selectors_.

There are lots of different selector types,
for example

- element functions
- strings
- regular expressions
- field filters

Let's see example of the latter:

#show heading.where(level: 1): set align(center)

= Title
== Small title

Of course, you can set align by hand,
no need to use show rules
(but they are very handy!):

#align(center)[== Centered small title]
\end{verbatim}

\pandocbounded{\includesvg[keepaspectratio]{basics/tutorial/typst-img/f41f337dd75b55211dd8d16e2682132c1ffb1ef19f774ba6cafc94cae090ec75-1.svg}}

\subsection{\texorpdfstring{\hyperref[custom-formatting]{Custom
formatting}}{Custom formatting}}\label{custom-formatting}

\begin{verbatim}
Let's try now writing custom functions.
It is very easy, see yourself:

// "it" is a heading, we take it and output things in braces
#show heading: it => {
  // center it
  set align(center)
  // set size and weight
  set text(12pt, weight: "regular")
  // see more about blocks and boxes
  // in corresponding chapter
  block(smallcaps(it.body))
}

= Smallcaps heading
\end{verbatim}

\pandocbounded{\includesvg[keepaspectratio]{basics/tutorial/typst-img/a5c37bce3cf9a077a4eb62a4d95f89584b5ef8acee279b81de6019d0e5768ba0-1.svg}}

\subsection{\texorpdfstring{\hyperref[setting-spacing]{Setting
spacing}}{Setting spacing}}\label{setting-spacing}

TODO: explain block spacing for common elements

\subsection{\texorpdfstring{\hyperref[formatting-to-get-an-article-look]{Formatting
to get an "article
look"}}{Formatting to get an "article look"}}\label{formatting-to-get-an-article-look}

\begin{verbatim}
#set page(
  // Header is that small thing on top
  header: align(
    right + horizon,
    [Some header there]
  ),
  height: 12cm
)

#align(center, text(17pt)[
  *Important title*
])

#grid(
  columns: (1fr, 1fr),
  align(center)[
    Some author \
    Some Institute \
    #link("mailto:some@mail.edu")
  ],
  align(center)[
    Another author \
    Another Institute \
    #link("mailto:another@mail.edu")
  ]
)

Now let's split text into two columns:

#show: rest => columns(2, rest)

#show heading.where(
  level: 1
): it => block(width: 100%)[
  #set align(center)
  #set text(12pt, weight: "regular")
  #smallcaps(it.body)
]

#show heading.where(
  level: 2
): it => text(
  size: 11pt,
  weight: "regular",
  style: "italic",
  it.body + [.],
)

// Now let's fill it with words:

= Heading
== Small heading
#lorem(10)
== Second subchapter
#lorem(10)
= Second heading
#lorem(40)

== Second subchapter
#lorem(40)
\end{verbatim}

\pandocbounded{\includesvg[keepaspectratio]{basics/tutorial/typst-img/76ee0cca809299df178ec9d94371c01031d1808a700b39deac5245dd6b83157f-1.svg}}

\section{\texorpdfstring{\hyperref[templates]{Templates}}{Templates}}\label{templates}

\subsection{\texorpdfstring{\hyperref[templates-1]{Templates}}{Templates}}\label{templates-1}

If you want to reuse styling in other files, you can use the
\emph{template} idiom. Because \texttt{\ }{\texttt{\ set\ }}\texttt{\ }
and \texttt{\ }{\texttt{\ show\ }}\texttt{\ } rules are only active in
their current scope, they will not affect content in a file you imported
your file into. But functions can circumvent this in a predictable way:

\begin{verbatim}
// define a function that:
// - takes content
// - applies styling to it
// - returns the styled content
#let apply-template(body) = [
  #show heading.where(level: 1): emph
  #set heading(numbering: "1.1")
  // ...
  #body
]
\end{verbatim}

This is equivalent to:

\begin{verbatim}
// we can reduce the number of hashes needed here by using scripting mode
// same as above but we exchanged `[...]` for `{...}` to switch from markup
// into scripting mode
#let apply-template(body) = {
  show heading.where(level: 1): emph
  set heading(numbering: "1.1")
  // ...
  body
}
\end{verbatim}

Then in your main file:

\begin{verbatim}
#import "template.typ": apply-template
#show: apply-template
\end{verbatim}

\emph{This will apply a "template" function to the rest of your
document!}

\subsubsection{\texorpdfstring{\hyperref[passing-arguments]{Passing
arguments}}{Passing arguments}}\label{passing-arguments}

\begin{verbatim}
// add optional named arguments
#let apply-template(body, name: "My document") = {
  show heading.where(level: 1): emph
  set heading(numbering: "1.1")

  align(center, text(name, size: 2em))

  body
}
\end{verbatim}

Then, in template file:

\begin{verbatim}
#import "template.typ": apply-template

// `func.with(..)` applies the arguments to the function and returns the new
// function with those defaults applied
#show: apply-template.with(name: "Report")

// it is functionally the same as this
#let new-template(..args) = apply-template(name: "Report", ..args)
#show: new-template
\end{verbatim}

Writing templates is fairly easy if you understand
\href{basics/tutorial/../scripting/index.html}{scripting} .

See more information about writing templates in
\href{https://typst.app/docs/tutorial/making-a-template/}{Official
Tutorial} .

There is no official repository for templates yet, but there are a
plenty community ones in
\href{https://github.com/qjcg/awesome-typst?ysclid=lj8pur1am7431908794\#general}{awesome-typst}
.

\section{\texorpdfstring{\hyperref[must-know]{Must-know}}{Must-know}}\label{must-know}

This section contains things, that are not general enough to be part of
"tutorial", but still are very important to know for proper typesetting.

Feel free to skip through things you are sure you will not use.

\section{\texorpdfstring{\hyperref[boxing--blocking]{Boxing \&
Blocking}}{Boxing \& Blocking}}\label{boxing--blocking}

\begin{verbatim}
You can use boxes to wrap anything
into text: #box(image("../tiger.jpg", height: 2em)).

Blocks will always be "separate paragraphs".
They will not fit into a text: #block(image("../tiger.jpg", height: 2em))
\end{verbatim}

\pandocbounded{\includesvg[keepaspectratio]{basics/must_know/typst-img/8e3bd89485b00259666bd636cf28586f92db9c3c3922f0adcdad765ee66a06b1-1.svg}}

Both have similar useful properties:

\begin{verbatim}
#box(stroke: red, inset: 1em)[Box text]
#block(stroke: red, inset: 1em)[Block text]
\end{verbatim}

\pandocbounded{\includesvg[keepaspectratio]{basics/must_know/typst-img/9e3562619cb8a31b3d2311f53c3815a214f081e033a564e63dc003dfbc50d68d-1.svg}}

\subsection{\texorpdfstring{\hyperref[rect]{\texttt{\ }{\texttt{\ rect\ }}\texttt{\ }}}{  rect  }}\label{rect}

There is also \texttt{\ }{\texttt{\ rect\ }}\texttt{\ } that works like
\texttt{\ }{\texttt{\ block\ }}\texttt{\ } , but has useful default
inset and stroke:

\begin{verbatim}
#rect[Block text]
\end{verbatim}

\pandocbounded{\includesvg[keepaspectratio]{basics/must_know/typst-img/c778d1e94a3663a4f258985368c02e294a1333554c550b6cfe0465275a2eef0f-1.svg}}

\subsection{\texorpdfstring{\hyperref[figures]{Figures}}{Figures}}\label{figures}

For the purposes of adding a \emph{figure} to your document, use
\texttt{\ }{\texttt{\ figure\ }}\texttt{\ } function.
Don\textquotesingle t try to use boxes or blocks there.

Figures are that things like centered images (probably with captions),
tables, even code.

\begin{verbatim}
@tiger shows a tiger. Tigers
are animals.

#figure(
  image("../tiger.jpg", width: 80%),
  caption: [A tiger.],
) <tiger>
\end{verbatim}

\pandocbounded{\includesvg[keepaspectratio]{basics/must_know/typst-img/09a8b5b3c3bfffd81be7f34c31cc93ca5f8341b2594d022b2b92ac285aeb959d-1.svg}}

In fact, you can put there anything you want:

\begin{verbatim}
They told me to write a letter to you. Here it is:

#figure(
  text(size: 5em)[I],
  caption: [I'm cool, right?],
) 
\end{verbatim}

\pandocbounded{\includesvg[keepaspectratio]{basics/must_know/typst-img/e009534c4572064346490dfac659ff94a5a11d7f46af7a2b46c2136d206088c6-1.svg}}

\section{\texorpdfstring{\hyperref[using-spacing]{Using
spacing}}{Using spacing}}\label{using-spacing}

Most time you will pass spacing into functions. There are special
function fields that take only \emph{size} . They are usually called
like
\texttt{\ }{\texttt{\ width,\ length,\ in(out)set,\ spacing\ }}\texttt{\ }
and so on.

Like in CSS, one of the ways to set up spacing in Typst is setting
margins and padding of elements. However, you can also insert spacing
directly using functions \texttt{\ }{\texttt{\ h\ }}\texttt{\ }
(horizontal spacing) and \texttt{\ }{\texttt{\ v\ }}\texttt{\ }
(vertical spacing).

\begin{quote}
Links to reference: \href{https://typst.app/docs/reference/layout/h/}{h}
, \href{https://typst.app/docs/reference/layout/v/}{v} .
\end{quote}

\begin{verbatim}
Horizontal #h(1cm) spacing.
#v(1cm)
And some vertical too!
\end{verbatim}

\pandocbounded{\includesvg[keepaspectratio]{basics/must_know/typst-img/47b3ea7d16575780e489790177df9a624ad3c6c669594baa4127c1db516ebc94-1.svg}}

\section{\texorpdfstring{\hyperref[absolute-length-units]{Absolute
length units}}{Absolute length units}}\label{absolute-length-units}

\begin{quote}
Link to
\href{https://typst.app/docs/reference/layout/length/}{reference}
\end{quote}

Absolute length (aka just "length") units are not affected by outer
content and size of parent.

\begin{verbatim}
#set rect(height: 1em)
#table(
  columns: 2,
  [Points], rect(width: 72pt),
  [Millimeters], rect(width: 25.4mm),
  [Centimeters], rect(width: 2.54cm),
  [Inches], rect(width: 1in),
)
\end{verbatim}

\pandocbounded{\includesvg[keepaspectratio]{basics/must_know/typst-img/073ad26fe313743ab62dca82f30208dbf2d57ff354d5c37f0b6d4c063dc37d76-1.svg}}

\subsection{\texorpdfstring{\hyperref[relative-to-current-font-size]{Relative
to current font
size}}{Relative to current font size}}\label{relative-to-current-font-size}

\texttt{\ }{\texttt{\ 1em\ =\ 1\ current\ font\ size\ }}\texttt{\ } :

\begin{verbatim}
#set rect(height: 1em)
#table(
  columns: 2,
  [Centimeters], rect(width: 2.54cm),
  [Relative to font size], rect(width: 6.5em)
)

Double font size: #box(stroke: red, baseline: 40%, height: 2em, width: 2em)
\end{verbatim}

\pandocbounded{\includesvg[keepaspectratio]{basics/must_know/typst-img/7d62c9e2540f8bce40d8a3fc65a2779b161eb6b5b5682cf87247fee7f14145c2-1.svg}}

It is a very convenient unit, so it is used a lot in Typst.

\subsection{\texorpdfstring{\hyperref[combined]{Combined}}{Combined}}\label{combined}

\begin{verbatim}
Combined: #box(rect(height: 5pt + 1em))

#(5pt + 1em).abs
#(5pt + 1em).em
\end{verbatim}

\pandocbounded{\includesvg[keepaspectratio]{basics/must_know/typst-img/c8a0cae6047f35c85c41ac44ff2a6b0d28a28d0e097ca61b367202f9a361136e-1.svg}}

\section{\texorpdfstring{\hyperref[ratio-length]{Ratio
length}}{Ratio length}}\label{ratio-length}

\begin{quote}
Link to \href{https://typst.app/docs/reference/layout/ratio/}{reference}
\end{quote}

\texttt{\ }{\texttt{\ 1\%\ =\ 1\%\ from\ parent\ size\ in\ that\ dimension\ }}\texttt{\ }

\begin{verbatim}
This line width is 50% of available page size (without margins):

#line(length: 50%)

This line width is 50% of the box width: #box(stroke: red, width: 4em, inset: (y: 0.5em), line(length: 50%))
\end{verbatim}

\pandocbounded{\includesvg[keepaspectratio]{basics/must_know/typst-img/d478cb8be0a049380479b634cae709dc1e1ed406d323ecb1edbca1e582d7eafe-1.svg}}

\section{\texorpdfstring{\hyperref[relative-length]{Relative
length}}{Relative length}}\label{relative-length}

\begin{quote}
Link to
\href{https://typst.app/docs/reference/layout/relative/}{reference}
\end{quote}

You can \emph{combine} absolute and ratio lengths into \emph{relative
length} :

\begin{verbatim}
#rect(width: 100% - 50pt)

#(100% - 50pt).length \
#(100% - 50pt).ratio
\end{verbatim}

\pandocbounded{\includesvg[keepaspectratio]{basics/must_know/typst-img/6b72620a1972e758e55ef1ecf49d3e843095037399ed4dd2dfcd262ebbbe803f-1.svg}}

\section{\texorpdfstring{\hyperref[fractional-length]{Fractional
length}}{Fractional length}}\label{fractional-length}

\begin{quote}
Link to
\href{https://typst.app/docs/reference/layout/fraction/}{reference}
\end{quote}

Single fraction length just takes \emph{maximum size possible} to fill
the parent:

\begin{verbatim}
Left #h(1fr) Right

#rect(height: 1em)[
  #h(1fr)
]
\end{verbatim}

\pandocbounded{\includesvg[keepaspectratio]{basics/must_know/typst-img/b9c91f53b684699fff70c6889c8a47fccc57c5c540d7629b93c51a797eb2ef3c-1.svg}}

There are not many places you can use fractions, mainly those are
\texttt{\ }{\texttt{\ h\ }}\texttt{\ } and
\texttt{\ }{\texttt{\ v\ }}\texttt{\ } .

\subsection{\texorpdfstring{\hyperref[several-fractions]{Several
fractions}}{Several fractions}}\label{several-fractions}

If you use several fractions inside one parent, they will take all
remaining space \emph{proportional to their number} :

\begin{verbatim}
Left #h(1fr) Left-ish #h(2fr) Right
\end{verbatim}

\pandocbounded{\includesvg[keepaspectratio]{basics/must_know/typst-img/45182cbcecf395256d133af78fccacd9d48e29073672317744cb17340d0bafd8-1.svg}}

\subsection{\texorpdfstring{\hyperref[nested-layout]{Nested
layout}}{Nested layout}}\label{nested-layout}

Remember that fractions work in parent only, don\textquotesingle t
\emph{rely on them in nested layout} :

\begin{verbatim}
Word: #h(1fr) #box(height: 1em, stroke: red)[
  #h(2fr)
]
\end{verbatim}

\pandocbounded{\includesvg[keepaspectratio]{basics/must_know/typst-img/0c7ed8b25ea7e39a0907b1105b82027a0fb8b921b28978f30692f6c693bea5f7-1.svg}}

\section{\texorpdfstring{\hyperref[placing-moving-scale--hide]{Placing,
Moving, Scale \&
Hide}}{Placing, Moving, Scale \& Hide}}\label{placing-moving-scale--hide}

This is \textbf{a very important section} if you want to do arbitrary
things with layout, create custom elements and hacking a way around
current Typst limitations.

TODO: WIP, add text and better examples

\section{\texorpdfstring{\hyperref[place]{Place}}{Place}}\label{place}

\emph{Ignore layout} , just put some object somehow relative to parent
and current position. The placed object \emph{will not} affect layouting

\begin{quote}
Link to \href{https://typst.app/docs/reference/layout/place/}{reference}
\end{quote}

\begin{verbatim}
#set page(height: 60pt)
Hello, world!

#place(
  top + right, // place at the page right and top
  square(
    width: 20pt,
    stroke: 2pt + blue
  ),
)
\end{verbatim}

\pandocbounded{\includesvg[keepaspectratio]{basics/must_know/typst-img/e0d4c250d0f288e1a110ebddcb06149e0acd11b626a0ccb0ca9feb1c1d7be359-1.svg}}

\subsubsection{\texorpdfstring{\hyperref[basic-floating-with-place]{Basic
floating with
place}}{Basic floating with place}}\label{basic-floating-with-place}

\begin{verbatim}
#set page(height: 150pt)
#let note(where, body) = place(
  center + where,
  float: true,
  clearance: 6pt,
  rect(body),
)

#lorem(10)
#note(bottom)[Bottom 1]
#note(bottom)[Bottom 2]
#lorem(40)
#note(top)[Top]
#lorem(10)
\end{verbatim}

\pandocbounded{\includesvg[keepaspectratio]{basics/must_know/typst-img/b770cfef024690b5fc7ab82458797d6cfab0c5cc8f52078ecf2d61be17c13acc-1.svg}}

\pandocbounded{\includesvg[keepaspectratio]{basics/must_know/typst-img/b770cfef024690b5fc7ab82458797d6cfab0c5cc8f52078ecf2d61be17c13acc-2.svg}}

\subsubsection{\texorpdfstring{\hyperref[dx-dy]{dx,
dy}}{dx, dy}}\label{dx-dy}

Manually change position by
\texttt{\ }{\texttt{\ (dx,\ dy)\ }}\texttt{\ } relative to intended.

\begin{verbatim}
#set page(height: 100pt)
#for i in range(16) {
  let amount = i * 4pt
  place(center, dx: amount - 32pt, dy: amount)[A]
}
\end{verbatim}

\pandocbounded{\includesvg[keepaspectratio]{basics/must_know/typst-img/12464f1a2cfe81fb04623033345f3f88ff598af5dc77de378b9d7cf88fc1d5b3-1.svg}}

\section{\texorpdfstring{\hyperref[move]{Move}}{Move}}\label{move}

\begin{quote}
Link to \href{https://typst.app/docs/reference/layout/move/}{reference}
\end{quote}

\begin{verbatim}
#rect(inset: 0pt, move(
  dx: 6pt, dy: 6pt,
  rect(
    inset: 8pt,
    fill: white,
    stroke: black,
    [Abra cadabra]
  )
))
\end{verbatim}

\pandocbounded{\includesvg[keepaspectratio]{basics/must_know/typst-img/3292aebf7b633a2d9574027f50867d723d80850e046a101b9df5ab5143eb8a8d-1.svg}}

\section{\texorpdfstring{\hyperref[scale]{Scale}}{Scale}}\label{scale}

Scale content \emph{without affecting the layout} .

\begin{quote}
Link to \href{https://typst.app/docs/reference/layout/scale/}{reference}
\end{quote}

\begin{verbatim}
#scale(x: -100%)[This is mirrored.]
\end{verbatim}

\pandocbounded{\includesvg[keepaspectratio]{basics/must_know/typst-img/401c8cd6f306771a3b12432c3c51e097a3ec1d12656c131c0043a12c4c1c3a0e-1.svg}}

\begin{verbatim}
A#box(scale(75%)[A])A \
B#box(scale(75%, origin: bottom + left)[B])B
\end{verbatim}

\pandocbounded{\includesvg[keepaspectratio]{basics/must_know/typst-img/204b55690645eb6cc623c8d2d74b5521d72e4ba38d58ea40ea5e2d4354a01836-1.svg}}

\section{\texorpdfstring{\hyperref[hide]{Hide}}{Hide}}\label{hide}

Don\textquotesingle t show content, but leave empty space there.

\begin{quote}
Link to \href{https://typst.app/docs/reference/layout/hide/}{reference}
\end{quote}

\begin{verbatim}
Hello Jane \
#hide[Hello] Joe
\end{verbatim}

\pandocbounded{\includesvg[keepaspectratio]{basics/must_know/typst-img/610672d5e43baa3ce94fe61f8d6dd0307e405c785639359c6a9e84bdd66884ad-1.svg}}

\section{\texorpdfstring{\hyperref[tables-and-grids]{Tables and
grids}}{Tables and grids}}\label{tables-and-grids}

While tables are not that necessary to know if you don\textquotesingle t
plan to use them in your documents, grids may be very useful for
\emph{document layout} . We will use both of them them in the book
later.

Let\textquotesingle s not bother with copying examples from official
documentation. Just make sure to skim through it, okay?

\subsection{\texorpdfstring{\hyperref[basic-snippets]{Basic
snippets}}{Basic snippets}}\label{basic-snippets}

\subsubsection{\texorpdfstring{\hyperref[spreading]{Spreading}}{Spreading}}\label{spreading}

Spreading operators (see
\href{basics/must_know/../scripting/arguments.html}{there} ) may be
especially useful for the tables:

\begin{verbatim}
#set text(size: 9pt)

#let yield_cells(n) = {
  for i in range(0, n + 1) {
    for j in range(0, n + 1) {
      let product = if i * j != 0 {
        // math is used for the better look 
        if j <= i { $#{ j * i }$ } 
        else {
          // upper part of the table
          text(gray.darken(50%), str(i * j))
        }
      } else {
        if i == j {
          // the top right corner 
          $times$
        } else {
          // on of them is zero, we are at top/left
          $#{i + j}$
        }
      }
      // this is an array, for loops merge them together
      // into one large array of cells
      (
        table.cell(
          fill: if i == j and j == 0 { orange } // top right corner
          else if i == j { yellow } // the diagonal
          else if i * j == 0 { blue.lighten(50%) }, // multipliers
          product,),
      )
    }
  }
}

#let n = 10
#table(
  columns: (0.6cm,) * (n + 1), rows: (0.6cm,) * (n + 1), align: center + horizon, inset: 3pt, ..yield_cells(n),
)
\end{verbatim}

\pandocbounded{\includesvg[keepaspectratio]{basics/must_know/typst-img/0640c1d0e5f79bdcb5e60f7675ff1b1eb18810078f5bbbdfaf1c5648b987706e-1.svg}}

\subsubsection{\texorpdfstring{\hyperref[highlighting-table-row]{Highlighting
table row}}{Highlighting table row}}\label{highlighting-table-row}

\begin{verbatim}
#table(
  columns: 2,
  fill: (x, y) => if y == 2 { highlight.fill },
  [A], [B],
  [C], [D],
  [E], [F],
  [G], [H],
)
\end{verbatim}

\pandocbounded{\includesvg[keepaspectratio]{basics/must_know/typst-img/4ff8cbb75f85dbab08a336be31115bcb4cb8ca505799641534d937d444e88082-1.svg}}

For individual cells, use

\begin{verbatim}
#table(
  columns: 2,
  [A], [B],
  table.cell(fill: yellow)[C], table.cell(fill: yellow)[D],
  [E], [F],
  [G], [H],
)
\end{verbatim}

\pandocbounded{\includesvg[keepaspectratio]{basics/must_know/typst-img/07676a86d4643ff83988c0907aa17995b3d1f8fa7b5be4f11959551afd674bc9-1.svg}}

\subsubsection{\texorpdfstring{\hyperref[splitting-tables]{Splitting
tables}}{Splitting tables}}\label{splitting-tables}

Tables are split between pages automatically.

\begin{verbatim}
#set page(height: 8em)
#(
table(
  columns: 5,
  [Aligner], [publication], [Indexing], [Pairwise alignment], [Max. read length  (bp)],
  [BWA], [2009], [BWT-FM], [Semi-Global], [125],
  [Bowtie], [2009], [BWT-FM], [HD], [76],
  [CloudBurst], [2009], [Hashing], [Landau-Vishkin], [36],
  [GNUMAP], [2009], [Hashing], [NW], [36]
  )
)
\end{verbatim}

\pandocbounded{\includesvg[keepaspectratio]{basics/must_know/typst-img/34794c27fefc5c307a1dfdc9ad7958c1dcca0ff8fb64962047051c6a216e0ff7-1.svg}}

\pandocbounded{\includesvg[keepaspectratio]{basics/must_know/typst-img/34794c27fefc5c307a1dfdc9ad7958c1dcca0ff8fb64962047051c6a216e0ff7-2.svg}}

However, if you want to make it breakable inside other element,
you\textquotesingle ll have to make that element breakable too:

\begin{verbatim}
#set page(height: 8em)
// Without this, the table fails to split upon several pages
#show figure: set block(breakable: true)
#figure(
table(
  columns: 5,
  [Aligner], [publication], [Indexing], [Pairwise alignment], [Max. read length  (bp)],
  [BWA], [2009], [BWT-FM], [Semi-Global], [125],
  [Bowtie], [2009], [BWT-FM], [HD], [76],
  [CloudBurst], [2009], [Hashing], [Landau-Vishkin], [36],
  [GNUMAP], [2009], [Hashing], [NW], [36]
  )
)
\end{verbatim}

\pandocbounded{\includesvg[keepaspectratio]{basics/must_know/typst-img/5be04bf8770a33256599791fb50751bcb24fa5108c13d0e5e2807b675fed00fb-1.svg}}

\pandocbounded{\includesvg[keepaspectratio]{basics/must_know/typst-img/5be04bf8770a33256599791fb50751bcb24fa5108c13d0e5e2807b675fed00fb-2.svg}}

\section{\texorpdfstring{\hyperref[project-structure]{Project
structure}}{Project structure}}\label{project-structure}

\subsection{\texorpdfstring{\hyperref[large-document]{Large
document}}{Large document}}\label{large-document}

Once the document becomes large enough, it becomes harder to navigate
it. If you haven\textquotesingle t reached that size yet, you can ignore
that section.

For managing that I would recommend splitting your document into
\emph{chapters} . It is just a way to work with this, but once you
understand how it works, you can do anything you want.

Let\textquotesingle s say you have two chapters, then the recommended
structure will look like this:

\begin{verbatim}
#import "@preview/treet:0.1.1": *

#show list: tree-list
#set par(leading: 0.8em)
#show list: set text(font: "DejaVu Sans Mono", size: 0.8em)
- chapters/
  - chapter_1.typ
  - chapter_2.typ
- main.typ 👁 #text(gray)[← document entry point]
- template.typ
\end{verbatim}

\pandocbounded{\includesvg[keepaspectratio]{basics/must_know/typst-img/291489e71b40beea77872ad05adb609349872e9a11fc3a9c3f2008c88e37c9d5-1.svg}}

The exact file names are up to you.

Let\textquotesingle s see what to put in each of these files.

\subsubsection{\texorpdfstring{\hyperref[template]{Template}}{Template}}\label{template}

In the "template" file goes \emph{all useful functions and variables}
you will use across the chapters. If you have your own template or want
to write one, you can write it there.

\begin{verbatim}
// template.typ

#let template = doc => {
    set page(header: "My super document")
    show "physics": "magic"
    doc
}

#let info-block = block.with(stroke: blue, fill: blue.lighten(70%))
#let author = "@sitandr"
\end{verbatim}

\subsubsection{\texorpdfstring{\hyperref[main]{Main}}{Main}}\label{main}

\textbf{This file should be compiled} to get the whole compiled
document.

\begin{verbatim}
// main.typ

#import "template.typ": *
// if you have a template
#show: template

= This is the document title

// some additional formatting

#show emph: set text(blue)

// but don't define functions or variables there!
// chapters will not see it

// Now the chapters themselves as some Typst content
#include("chapters/chapter_1.typ")
#include("chapters/chapter_1.typ")
\end{verbatim}

\subsubsection{\texorpdfstring{\hyperref[chapter]{Chapter}}{Chapter}}\label{chapter}

\begin{verbatim}
// chapter_1.typ

#import "../template.typ": *

That's just content with _styling_ and blocks:

#infoblock[Some information].

// just any content you want to include in the document
\end{verbatim}

\subsection{\texorpdfstring{\hyperref[notes]{Notes}}{Notes}}\label{notes}

Note that modules in Typst can see only what they created themselves or
imported. Anything else is invisible for them. That\textquotesingle s
why you need \texttt{\ }{\texttt{\ template.typ\ }}\texttt{\ } file to
define all functions within.

That means chapters \emph{don\textquotesingle t see each other either} ,
only what is in the template.

\subsection{\texorpdfstring{\hyperref[cyclic-imports]{Cyclic
imports}}{Cyclic imports}}\label{cyclic-imports}

\textbf{Important:} Typst \emph{forbids} cyclic imports. That means you
can\textquotesingle t import
\texttt{\ }{\texttt{\ chapter\_1\ }}\texttt{\ } from
\texttt{\ }{\texttt{\ chapter\_2\ }}\texttt{\ } and
\texttt{\ }{\texttt{\ chapter\_2\ }}\texttt{\ } from
\texttt{\ }{\texttt{\ chapter\_1\ }}\texttt{\ } at the same time!

But the good news is that you can always create some other file to
import variable from.

\section{\texorpdfstring{\hyperref[scripting]{Scripting}}{Scripting}}\label{scripting}

\textbf{Typst} has a complete interpreted language inside. One of key
aspects of working with your document in a nicer way

\section{\texorpdfstring{\hyperref[basics]{Basics}}{Basics}}\label{basics}

\subsection{\texorpdfstring{\hyperref[variables-i]{Variables
I}}{Variables I}}\label{variables-i}

Let\textquotesingle s start with \emph{variables} .

The concept is very simple, just some value you can reuse:

\begin{verbatim}
#let author = "John Doe"

This is a book by #author. #author is a great guy.

#quote(block: true, attribution: author)[
  \<Some quote\>
]
\end{verbatim}

\pandocbounded{\includesvg[keepaspectratio]{basics/scripting/typst-img/c311c1612cafa802f16f0d4ca2d6f1ecca59f545ed1f6ee99d3c4ae06ee2bff4-1.svg}}

\subsection{\texorpdfstring{\hyperref[variables-ii]{Variables
II}}{Variables II}}\label{variables-ii}

You can store \emph{any} Typst value in variable:

\begin{verbatim}
#let block_text = block(stroke: red, inset: 1em)[Text]

#block_text

#figure(caption: "The block", block_text)
\end{verbatim}

\pandocbounded{\includesvg[keepaspectratio]{basics/scripting/typst-img/c6290389652d1771d5149c9393af8eb32bd37e4b2bfb2c11764f9f22c294f84b-1.svg}}

\subsection{\texorpdfstring{\hyperref[functions-2]{Functions}}{Functions}}\label{functions-2}

We have already seen some "custom" functions in
\href{basics/scripting/../tutorial/advanced_styling.html}{Advanced
Styling} chapter.

Functions are values that take some values and output some values:

\begin{verbatim}
// This is a syntax that we have seen earlier
#let f = (name) => "Hello, " + name

#f("world!")
\end{verbatim}

\pandocbounded{\includesvg[keepaspectratio]{basics/scripting/typst-img/23fba8e9081a8b32b16d7deb54018bb73a8ac910adbfb1a0ca577eb3520a73b4-1.svg}}

\subsubsection{\texorpdfstring{\hyperref[alternative-syntax]{Alternative
syntax}}{Alternative syntax}}\label{alternative-syntax}

You can write the same shorter:

\begin{verbatim}
// The following syntaxes are equivalent
#let f = (name) => "Hello, " + name
#let f(name) = "Hello, " + name

#f("world!")
\end{verbatim}

\pandocbounded{\includesvg[keepaspectratio]{basics/scripting/typst-img/e6e4bd179a38f1b3af96f3e7c6308be6f9494f41f43daa26ebabf7a77fc54780-1.svg}}

\section{\texorpdfstring{\hyperref[braces-brackets-and-default]{Braces,
brackets and
default}}{Braces, brackets and default}}\label{braces-brackets-and-default}

\subsection{\texorpdfstring{\hyperref[square-brackets]{Square
brackets}}{Square brackets}}\label{square-brackets}

You may remember that square brackets convert everything inside to
\emph{content} .

\begin{verbatim}
#let v = [Some text, _markup_ and other #strong[functions]]
#v
\end{verbatim}

\pandocbounded{\includesvg[keepaspectratio]{basics/scripting/typst-img/5ba617daa8d4c166d96a0abbba02d6502fe7fde1ded460afa78682993295142d-1.svg}}

We may use same for functions bodies:

\begin{verbatim}
#let f(name) = [Hello, #name]
#f[World] // also don't forget we can use it to pass content!
\end{verbatim}

\pandocbounded{\includesvg[keepaspectratio]{basics/scripting/typst-img/4545deeee45655ee6666feb4773416cd075fe7522cbfd80d0847c615c6c5f30a-1.svg}}

\textbf{Important:} It is very hard to convert \emph{content} to
\emph{plain text} , as \emph{content} may contain \emph{anything} ! Sp
be careful when passing and storing content in variables.

\subsection{\texorpdfstring{\hyperref[braces]{Braces}}{Braces}}\label{braces}

However, we often want to use code inside functions.
That\textquotesingle s when we use
\texttt{\ }{\texttt{\ \{\}\ }}\texttt{\ } :

\begin{verbatim}
#let f(name) = {
  // this is code mode

  // First part of our output
  "Hello, "

  // we check if name is empty, and if it is,
  // insert placeholder
  if name == "" {
      "anonym"
  } else {
      name
  }

  // finish sentence
  "!"
}

#f("")
#f("Joe")
#f("world")
\end{verbatim}

\pandocbounded{\includesvg[keepaspectratio]{basics/scripting/typst-img/f2bc6aebef06f213c9a8e740266a96e424318d953c09cffba6c5811375e91395-1.svg}}

\subsection{\texorpdfstring{\hyperref[scopes]{Scopes}}{Scopes}}\label{scopes}

\textbf{This is a very important thing to remember} .

\emph{You can\textquotesingle t use variables outside of scopes they are
defined (unless it is file root, then you can import them)} . \emph{Set
and show rules affect things in their scope only.}

\begin{verbatim}
#{
  let a = 3;
}
// can't use "a" there.

#[
  #show "true": "false"

  This is true.
]

This is true.
\end{verbatim}

\pandocbounded{\includesvg[keepaspectratio]{basics/scripting/typst-img/c25d356831eeea19bb243b87c0f32d062c7086a55b4ee432e41b388d626f875b-1.svg}}

\subsection{\texorpdfstring{\hyperref[return]{Return}}{Return}}\label{return}

\textbf{Important} : by default braces return anything that "returns"
into them. For example,

\begin{verbatim}
#let change_world() = {
  // some code there changing everything in the world
  str(4e7)
  // another code changing the world
}

#let g() = {
  "Hahaha, I will change the world now! "
  change_world()
  " So here is my long evil monologue..."
}

#g()
\end{verbatim}

\pandocbounded{\includesvg[keepaspectratio]{basics/scripting/typst-img/160d9672bd7abc64ea61943d1bfcbd1b06dc70f87be5e5cf9c411fe4ee6d2a44-1.svg}}

To avoid returning everything, return only what you want explicitly,
otherwise everything will be joined:

\begin{verbatim}
#let f() = {
  "Some long text"
  // Crazy numbers
  "2e7"
  return none
}

// Returns nothing
#f()
\end{verbatim}

\pandocbounded{\includesvg[keepaspectratio]{basics/scripting/typst-img/14c935733a8c91165ee4ebe8246efb841207feeaa0309e36a1cde2888acffb10-1.svg}}

\subsection{\texorpdfstring{\hyperref[default-values]{Default
values}}{Default values}}\label{default-values}

What we made just now was inventing "default values".

They are very common in styling, so there is a special syntax for them:

\begin{verbatim}
#let f(name: "anonym") = [Hello, #name!]

#f()
#f(name: "Joe")
#f(name: "world")
\end{verbatim}

\pandocbounded{\includesvg[keepaspectratio]{basics/scripting/typst-img/e9730d0d1f30ec9f2404179560ae4a4b19dd788b1afc2f6b956fb32268439cb6-1.svg}}

You may have noticed that the argument became \emph{named} now. In
Typst, named argument is an argument \emph{that has default value} .

\section{\texorpdfstring{\hyperref[types-part-i]{Types, part
I}}{Types, part I}}\label{types-part-i}

Each value in Typst has a type. You don\textquotesingle t have to
specify it, but it is important.

\subsection{\texorpdfstring{\hyperref[content-content]{Content (
\texttt{\ }{\texttt{\ content\ }}\texttt{\ }
)}}{Content (   content   )}}\label{content-content}

\begin{quote}
\href{https://typst.app/docs/reference/foundations/content/}{Link to
Reference} .
\end{quote}

We have already seen it. A type that represents what is displayed in
document.

\begin{verbatim}
#let c = [It is _content_!]

// Check type of c
#(type(c) == content)

#c

// repr gives an "inner representation" of value
#repr(c)
\end{verbatim}

\pandocbounded{\includesvg[keepaspectratio]{basics/scripting/typst-img/21fd80460de8e8a377a9ef2046a27232ad88924070509ccf8647c9135c9c2fe3-1.svg}}

\textbf{Important:} It is very hard to convert \emph{content} to
\emph{plain text} , as \emph{content} may contain \emph{anything} ! So
be careful when passing and storing content in variables.

\subsection{\texorpdfstring{\hyperref[none-none]{None (
\texttt{\ }{\texttt{\ none\ }}\texttt{\ }
)}}{None (   none   )}}\label{none-none}

Nothing. Also known as \texttt{\ }{\texttt{\ null\ }}\texttt{\ } in
other languages. It isn\textquotesingle t displayed, converts to empty
content.

\begin{verbatim}
#none
#repr(none)
\end{verbatim}

\pandocbounded{\includesvg[keepaspectratio]{basics/scripting/typst-img/c4100c1d1df8fc0a51bd99945d9bac3c5aa67de19b8f872fd33fd9068bb2507b-1.svg}}

\subsection{\texorpdfstring{\hyperref[string-str]{String (
\texttt{\ }{\texttt{\ str\ }}\texttt{\ }
)}}{String (   str   )}}\label{string-str}

\begin{quote}
\href{https://typst.app/docs/reference/foundations/str/}{Link to
Reference} .
\end{quote}

String contains only plain text and no formatting. Just some chars. That
allows us to work with chars:

\begin{verbatim}
#let s = "Some large string. There could be escape sentences: \n,
 line breaks, and even unicode codes: \u{1251}"
#s \
#type(s) \
`repr`: #repr(s)

#let s = "another small string"
#s.replace("a", sym.alpha) \
#s.split(" ") // split by space
\end{verbatim}

\pandocbounded{\includesvg[keepaspectratio]{basics/scripting/typst-img/b797f9c4a540fcf1429bec801d0b334e7d88dc9ccd10e3b7b859f451e269f30f-1.svg}}

You can convert other types to their string representation using this
type\textquotesingle s constructor (e.g. convert number to string):

\begin{verbatim}
#str(5) // string, can be worked with as string
\end{verbatim}

\pandocbounded{\includesvg[keepaspectratio]{basics/scripting/typst-img/ab4d4a5d93533525f7f9b2cc8378b79f1561904f3c5d5f6d2ec4bdc448669cb5-1.svg}}

\subsection{\texorpdfstring{\hyperref[boolean-bool]{Boolean (
\texttt{\ }{\texttt{\ bool\ }}\texttt{\ }
)}}{Boolean (   bool   )}}\label{boolean-bool}

\begin{quote}
\href{https://typst.app/docs/reference/foundations/bool/}{Link to
Reference} .
\end{quote}

true/false. Used in \texttt{\ }{\texttt{\ if\ }}\texttt{\ } and many
others

\begin{verbatim}
#let b = false
#b \
#repr(b) \
#(true and not true or true) = #((true and (not true)) or true) \
#if (4 > 3) {
  "4 is more than 3"
}
\end{verbatim}

\pandocbounded{\includesvg[keepaspectratio]{basics/scripting/typst-img/e848d78e220ca8cf3b6c323a99d5d963e529aad36857f0e6753c56c02984a616-1.svg}}

\subsection{\texorpdfstring{\hyperref[integer-int]{Integer (
\texttt{\ }{\texttt{\ int\ }}\texttt{\ }
)}}{Integer (   int   )}}\label{integer-int}

\begin{quote}
\href{https://typst.app/docs/reference/foundations/int/}{Link to
Reference} .
\end{quote}

A whole number.

The number can also be specified as hexadecimal, octal, or binary by
starting it with a zero followed by either x, o, or b.

\begin{verbatim}
#let n = 5
#n \
#(n += 1) \
#n \
#calc.pow(2, n) \
#type(n) \
#repr(n)
\end{verbatim}

\pandocbounded{\includesvg[keepaspectratio]{basics/scripting/typst-img/6f1c9e02393e14aa23add33d0e6dc2b596ee97a0d425cd3edb3e2b912c6ef6b0-1.svg}}

\begin{verbatim}
#(1 + 2) \
#(2 - 5) \
#(3 + 4 < 8)
\end{verbatim}

\pandocbounded{\includesvg[keepaspectratio]{basics/scripting/typst-img/e610f15659cb6b64c3516be48740b54e6caf3d933919004157ba64b757389ba5-1.svg}}

\begin{verbatim}
#0xff \
#0o10 \
#0b1001
\end{verbatim}

\pandocbounded{\includesvg[keepaspectratio]{basics/scripting/typst-img/1446dba05ee6f8006884c280ff32e31ede8425d4847445e97cae5dfcde1efe7f-1.svg}}

You can convert a value to an integer with this type\textquotesingle s
constructor (e.g. convert string to int).

\begin{verbatim}
#int(false) \
#int(true) \
#int(2.7) \
#(int("27") + int("4"))
\end{verbatim}

\pandocbounded{\includesvg[keepaspectratio]{basics/scripting/typst-img/b44779a87fd984d317ec4d1aed732c0ebdc6220fd4764e407f77fedd139c0d8c-1.svg}}

\subsection{\texorpdfstring{\hyperref[float-float]{Float (
\texttt{\ }{\texttt{\ float\ }}\texttt{\ }
)}}{Float (   float   )}}\label{float-float}

\begin{quote}
\href{https://typst.app/docs/reference/foundations/float/}{Link to
Reference} .
\end{quote}

Works the same way as integer, but can store floating point numbers.
However, precision may be lost.

\begin{verbatim}
#let n = 5.0

// You can mix floats and integers, 
// they will be implicitly converted
#(n += 1) \
#calc.pow(2, n) \
#(0.2 + 0.1) \
#type(n) 
\end{verbatim}

\pandocbounded{\includesvg[keepaspectratio]{basics/scripting/typst-img/21cafe751ec803dd9598c871b283a29bc3c6b2e302f0f9bd78edc17330b45616-1.svg}}

\begin{verbatim}
#3.14 \
#1e4 \
#(10 / 4)
\end{verbatim}

\pandocbounded{\includesvg[keepaspectratio]{basics/scripting/typst-img/05bd400096c1df5a954fda0897f3c1756c9f99f73503d32d992b3222667a45cd-1.svg}}

You can convert a value to a float with this type\textquotesingle s
constructor (e.g. convert string to float).

\begin{verbatim}
#float(40%) \
#float("2.7") \
#float("1e5")
\end{verbatim}

\pandocbounded{\includesvg[keepaspectratio]{basics/scripting/typst-img/f50a22cbea42fded97ab8340f0939e786e5c1cdb5ea531cd4b35b1f732947b7f-1.svg}}

\section{\texorpdfstring{\hyperref[types-part-ii]{Types, part
II}}{Types, part II}}\label{types-part-ii}

In Typst, most of things are \textbf{immutable} . You
can\textquotesingle t change content, you can just create new using this
one (for example, using addition).

Immutability is very important for Typst since it tries to be \emph{as
pure language as possible} . Functions do nothing outside of returning
some value.

However, purity is partly "broken" by these types. They are
\emph{super-useful} and not adding them would make Typst much pain.

However, using them adds complexity.

\subsection{\texorpdfstring{\hyperref[arrays-array]{Arrays (
\texttt{\ }{\texttt{\ array\ }}\texttt{\ }
)}}{Arrays (   array   )}}\label{arrays-array}

\begin{quote}
\href{https://typst.app/docs/reference/foundations/array/}{Link to
Reference} .
\end{quote}

Mutable object that stores data with their indices.

\subsubsection{\texorpdfstring{\hyperref[working-with-indices]{Working
with indices}}{Working with indices}}\label{working-with-indices}

\begin{verbatim}
#let values = (1, 7, 4, -3, 2)

// take value at index 0
#values.at(0) \
// set value at 0 to 3
#(values.at(0) = 3)
// negative index => start from the back
#values.at(-1) \
// add index of something that is even
#values.find(calc.even)
\end{verbatim}

\pandocbounded{\includesvg[keepaspectratio]{basics/scripting/typst-img/0374c20b28fbf2b2d15bc32e5428f7f5121ea9d673d96de3274a0c6d988d5fb5-1.svg}}

\subsubsection{\texorpdfstring{\hyperref[iterating-methods]{Iterating
methods}}{Iterating methods}}\label{iterating-methods}

\begin{verbatim}
#let values = (1, 7, 4, -3, 2)

// leave only what is odd
#values.filter(calc.odd) \
// create new list of absolute values of list values
#values.map(calc.abs) \
// reverse
#values.rev() \
// convert array of arrays to flat array
#(1, (2, 3)).flatten() \
// join array of string to string
#(("A", "B", "C")
 .join(", ", last: " and "))
\end{verbatim}

\pandocbounded{\includesvg[keepaspectratio]{basics/scripting/typst-img/684400186916f8f16a2d7edb151b7f5023c7e4c010b23a2c6566f0bd7a224061-1.svg}}

\subsubsection{\texorpdfstring{\hyperref[list-operations]{List
operations}}{List operations}}\label{list-operations}

\begin{verbatim}
// sum of lists:
#((1, 2, 3) + (4, 5, 6))

// list product:
#((1, 2, 3) * 4)
\end{verbatim}

\pandocbounded{\includesvg[keepaspectratio]{basics/scripting/typst-img/abe2d311638b351e0938be0e432f10265ca81a69a9ed7d2e6f88f656c60dfc65-1.svg}}

\subsubsection{\texorpdfstring{\hyperref[empty-list]{Empty
list}}{Empty list}}\label{empty-list}

\begin{verbatim}
#() \ // this is an empty list
#(1,) \  // this is a list with one element
BAD: #(1) // this is just an element, not a list!
\end{verbatim}

\pandocbounded{\includesvg[keepaspectratio]{basics/scripting/typst-img/da4f77f8784462ca5c4f73862e58420695916064d56921e4adef7a7e37d5a532-1.svg}}

\subsection{\texorpdfstring{\hyperref[dictionaries-dict]{Dictionaries (
\texttt{\ }{\texttt{\ dict\ }}\texttt{\ }
)}}{Dictionaries (   dict   )}}\label{dictionaries-dict}

\begin{quote}
\href{https://typst.app/docs/reference/foundations/dictionary/}{Link to
Reference} .
\end{quote}

Dictionaries are objects that store a string "key" and a value,
associated with that key.

\begin{verbatim}
#let dict = (
  name: "Typst",
  born: 2019,
)

#dict.name \
#(dict.launch = 20)
#dict.len() \
#dict.keys() \
#dict.values() \
#dict.at("born") \
#dict.insert("city", "Berlin ")
#("name" in dict)
\end{verbatim}

\pandocbounded{\includesvg[keepaspectratio]{basics/scripting/typst-img/638ada64eb36af0b1891def1b2c0a2cc97a14d87987df8c16f5f3872244553d6-1.svg}}

\subsubsection{\texorpdfstring{\hyperref[empty-dictionary]{Empty
dictionary}}{Empty dictionary}}\label{empty-dictionary}

\begin{verbatim}
This is an empty list: #() \
This is an empty dict: #(:)
\end{verbatim}

\pandocbounded{\includesvg[keepaspectratio]{basics/scripting/typst-img/6ef41801d46f0b7256bb6913482fde054c811a1850ecae3a446660eb6d1c8850-1.svg}}

\section{\texorpdfstring{\hyperref[conditions--loops]{Conditions \&
loops}}{Conditions \& loops}}\label{conditions--loops}

\subsection{\texorpdfstring{\hyperref[conditions]{Conditions}}{Conditions}}\label{conditions}

\begin{quote}
See
\href{https://typst.app/docs/reference/scripting/\#conditionals}{official
documentation} .
\end{quote}

In Typst, you can use \texttt{\ }{\texttt{\ if-else\ }}\texttt{\ }
statements. This is especially useful inside function bodies to vary
behavior depending on arguments types or many other things.

\begin{verbatim}
#if 1 < 2 [
  This is shown
] else [
  This is not.
]
\end{verbatim}

\pandocbounded{\includesvg[keepaspectratio]{basics/scripting/typst-img/2e914defa3353d6fd42ed58c37a97aedcc2237cfe20228f0cc0d223dfff4619a-1.svg}}

Of course, \texttt{\ }{\texttt{\ else\ }}\texttt{\ } is unnecessary:

\begin{verbatim}
#let a = 3

#if a < 4 {
  a = 5
}

#a
\end{verbatim}

\pandocbounded{\includesvg[keepaspectratio]{basics/scripting/typst-img/a7264774be154606a44d829d31edae18bf686262ccea66de9ed97fa20c720bd8-1.svg}}

You can also use \texttt{\ }{\texttt{\ else\ if\ }}\texttt{\ } statement
(known as \texttt{\ }{\texttt{\ elif\ }}\texttt{\ } in Python):

\begin{verbatim}
#let a = 5

#if a < 4 {
  a = 5
} else if a < 6 {
  a = -3
}

#a
\end{verbatim}

\pandocbounded{\includesvg[keepaspectratio]{basics/scripting/typst-img/9f65678fc26af2d197d979e1b0a5295ed64037ee00c30fa28c9c417a2c7dc308-1.svg}}

\subsubsection{\texorpdfstring{\hyperref[booleans]{Booleans}}{Booleans}}\label{booleans}

\texttt{\ }{\texttt{\ if,\ else\ if,\ else\ }}\texttt{\ } accept
\emph{only boolean} values as a switch. You can combine booleans as
described in \href{basics/scripting/./types.html\#boolean-bool}{types
section} :

\begin{verbatim}
#let a = 5

#if (a > 1 and a <= 4) or a == 5 [
    `a` matches the condition
]
\end{verbatim}

\pandocbounded{\includesvg[keepaspectratio]{basics/scripting/typst-img/21d3a48404d4e0c59bc0fccb114fdeac7384189db0020247796f44b0e9a7c362-1.svg}}

\subsection{\texorpdfstring{\hyperref[loops]{Loops}}{Loops}}\label{loops}

\begin{quote}
See \href{https://typst.app/docs/reference/scripting/\#loops}{official
documentation} .
\end{quote}

There are two kinds of loops: \texttt{\ }{\texttt{\ while\ }}\texttt{\ }
and \texttt{\ }{\texttt{\ for\ }}\texttt{\ } . While repeats body while
the condition is met:

\begin{verbatim}
#let a = 3

#while a < 100 {
    a *= 2
    str(a)
    " "
}
\end{verbatim}

\pandocbounded{\includesvg[keepaspectratio]{basics/scripting/typst-img/ece06c012663616cac05b0f365bd02ea5607dcddfaa0249963088ceff797c100-1.svg}}

\texttt{\ }{\texttt{\ for\ }}\texttt{\ } iterates over all elements of
sequence. The sequence may be an
\texttt{\ }{\texttt{\ array\ }}\texttt{\ } ,
\texttt{\ }{\texttt{\ string\ }}\texttt{\ } or
\texttt{\ }{\texttt{\ dictionary\ }}\texttt{\ } (
\texttt{\ }{\texttt{\ for\ }}\texttt{\ } iterates over its
\emph{key-value pairs} ).

\begin{verbatim}
#for c in "ABC" [
  #c is a letter.
]
\end{verbatim}

\pandocbounded{\includesvg[keepaspectratio]{basics/scripting/typst-img/9e70091e4c1f276d548f8200329298bf6b98946c331ca4630fec8313d5a91eff-1.svg}}

To iterate to all numbers from \texttt{\ }{\texttt{\ a\ }}\texttt{\ } to
\texttt{\ }{\texttt{\ b\ }}\texttt{\ } , use
\texttt{\ }{\texttt{\ range(a,\ b+1)\ }}\texttt{\ } :

\begin{verbatim}
#let s = 0

#for i in range(3, 6) {
    s += i
    [Number #i is added to sum. Now sum is #s.]
}
\end{verbatim}

\pandocbounded{\includesvg[keepaspectratio]{basics/scripting/typst-img/1e3d95ee79d7bc6989e40ff1e27c0ef6e3b152a1e5f8a0df5b2819621e0e299f-1.svg}}

Because range is end-exclusive this is equal to

\begin{verbatim}
#let s = 0

#for i in (3, 4, 5) {
    s += i
    [Number #i is added to sum. Now sum is #s.]
}
\end{verbatim}

\pandocbounded{\includesvg[keepaspectratio]{basics/scripting/typst-img/6158d29261339f8f285d592deff8992ca129ce32264abcdcf6734ac44cf558a4-1.svg}}

\begin{verbatim}
#let people = (Alice: 3, Bob: 5)

#for (name, value) in people [
    #name has #value apples.
]
\end{verbatim}

\pandocbounded{\includesvg[keepaspectratio]{basics/scripting/typst-img/50ff0963afe8c9ec5a0562d518431b63d5dd3810525f55f084f812452b11eb21-1.svg}}

\subsubsection{\texorpdfstring{\hyperref[break-and-continue]{Break and
continue}}{Break and continue}}\label{break-and-continue}

Inside loops can be used \texttt{\ }{\texttt{\ break\ }}\texttt{\ } and
\texttt{\ }{\texttt{\ continue\ }}\texttt{\ } commands.
\texttt{\ }{\texttt{\ break\ }}\texttt{\ } breaks loop, jumping outside.
\texttt{\ }{\texttt{\ continue\ }}\texttt{\ } jumps to next loop
iteration.

See the difference on these examples:

\begin{verbatim}
#for letter in "abc nope" {
  if letter == " " {
    // stop when there is space
    break
  }

  letter
}
\end{verbatim}

\pandocbounded{\includesvg[keepaspectratio]{basics/scripting/typst-img/a744551cab635d3ab70d9bf4258bb5fc26fe384f8e9f487ad0b8eee986ffe581-1.svg}}

\begin{verbatim}
#for letter in "abc nope" {
  if letter == " " {
    // skip the space
    continue
  }

  letter
}
\end{verbatim}

\pandocbounded{\includesvg[keepaspectratio]{basics/scripting/typst-img/bbb719820f986e52fbf64306536766ecbfd7264d29429a5c62d1bd648a4754c5-1.svg}}

\section{\texorpdfstring{\hyperref[advanced-arguments]{Advanced
arguments}}{Advanced arguments}}\label{advanced-arguments}

\subsection{\texorpdfstring{\hyperref[spreading-arguments-from-list]{Spreading
arguments from
list}}{Spreading arguments from list}}\label{spreading-arguments-from-list}

Spreading operator allows you to "unpack" the list of values into
arguments of function:

\begin{verbatim}
#let func(a, b, c, d, e) = [#a #b #c #d #e]
#func(..(([hi],) * 5))
\end{verbatim}

\pandocbounded{\includesvg[keepaspectratio]{basics/scripting/typst-img/0586f1f7eb73effd507824b57f7282f12fe2612119d64413f72e6518aba01513-1.svg}}

This may be super useful in tables:

\begin{verbatim}
#let a = ("hi", "b", "c")

#table(columns: 3,
  [test], [x], [hello],
  ..a
)
\end{verbatim}

\pandocbounded{\includesvg[keepaspectratio]{basics/scripting/typst-img/eb669f70df63815adcbe764fdb8635eecab33651c7eef55ea4de6cd63c96d9de-1.svg}}

\subsection{\texorpdfstring{\hyperref[key-arguments]{Key
arguments}}{Key arguments}}\label{key-arguments}

The same idea works with key arguments:

\begin{verbatim}
#let text-params = (fill: blue, size: 0.8em)

Some #text(..text-params)[text].
\end{verbatim}

\pandocbounded{\includesvg[keepaspectratio]{basics/scripting/typst-img/e56483e8f4285f8fed8cd6867e720b9a1c9f62ef0bffea28d124159f8a61648d-1.svg}}

\section{\texorpdfstring{\hyperref[managing-arbitrary-arguments]{Managing
arbitrary
arguments}}{Managing arbitrary arguments}}\label{managing-arbitrary-arguments}

Typst allows taking as many arbitrary positional and key arguments as
you want.

In that case function is given special
\texttt{\ }{\texttt{\ arguments\ }}\texttt{\ } object that stores in it
positional and named arguments.

\begin{quote}
Link to
\href{https://typst.app/docs/reference/foundations/arguments/}{reference}
\end{quote}

\begin{verbatim}
#let f(..args) = [
  #args.pos()\
  #args.named()
]

#f(1, "a", width: 50%, block: false)
\end{verbatim}

\pandocbounded{\includesvg[keepaspectratio]{basics/scripting/typst-img/2fc64c8521734ea689368ec83fe54025eb94b016a8ed1f6d6a9880ac6c94edf5-1.svg}}

You can combine them with other arguments. Spreading operator will "eat"
all remaining arguments:

\begin{verbatim}
#let format(title, ..authors) = {
  let by = authors
    .pos()
    .join(", ", last: " and ")

  [*#title* \ _Written by #by;_]
}

#format("ArtosFlow", "Jane", "Joe")
\end{verbatim}

\pandocbounded{\includesvg[keepaspectratio]{basics/scripting/typst-img/4ba76c5176e0b93c6c2b03c38d55f88702546a5183717ed8c3567865c0d1bf5d-1.svg}}

\subsection{\texorpdfstring{\hyperref[optional-argument]{Optional
argument}}{Optional argument}}\label{optional-argument}

\emph{Currently the only way in Typst to create optional positional
arguments is using \texttt{\ }{\texttt{\ arguments\ }}\texttt{\ }
object:}

TODO

\section{\texorpdfstring{\hyperref[tips]{Tips}}{Tips}}\label{tips}

There are lots of elements in Typst scripting that are not obvious, but
important. All the book is designated to show them, but some of them

\subsection{\texorpdfstring{\hyperref[equality]{Equality}}{Equality}}\label{equality}

Equality doesn\textquotesingle t mean objects are really the same, like
in many other objects:

\begin{verbatim}
#let a = 7
#let b = 7.0
#(a == b)
#(type(a) == type(b))
\end{verbatim}

\pandocbounded{\includesvg[keepaspectratio]{basics/scripting/typst-img/3632e0202f7aae6ed6e2958b7bc6360a6cba31aa3d1aaf169a133ef987c839de-1.svg}}

That may be less obvious for dictionaries. In dictionaries \textbf{the
order may matter} , so equality doesn\textquotesingle t mean they behave
exactly the same way:

\begin{verbatim}
#let a = (x: 1, y: 2)
#let b = (y: 2, x: 1)
#(a == b)
#(a.pairs() == b.pairs())
\end{verbatim}

\pandocbounded{\includesvg[keepaspectratio]{basics/scripting/typst-img/f7277d7cc170d7cc2ae1de5436b534fb113cda82d8e7829a0fc92e950b78238f-1.svg}}

\subsection{\texorpdfstring{\hyperref[check-key-is-in-dictionary]{Check
key is in
dictionary}}{Check key is in dictionary}}\label{check-key-is-in-dictionary}

Use the keyword \texttt{\ }{\texttt{\ in\ }}\texttt{\ } , like in
\texttt{\ }{\texttt{\ Python\ }}\texttt{\ } :

\begin{verbatim}
#let dict = (a: 1, b: 2)

#("a" in dict)
// gives the same as
#(dict.keys().contains("a"))
\end{verbatim}

\pandocbounded{\includesvg[keepaspectratio]{basics/scripting/typst-img/c4ae77418e54911af371f203d2bd3d5badb7269496bb8f07a2e3010e15f18922-1.svg}}

Note it works for lists too:

\begin{verbatim}
#("a" in ("b", "c", "a"))
#(("b", "c", "a").contains("a"))
\end{verbatim}

\pandocbounded{\includesvg[keepaspectratio]{basics/scripting/typst-img/0fc3ff7d44bbb5bcacd38e921f199699d2ea43ce0a142e79f67314d4f24386a7-1.svg}}

\section{\texorpdfstring{\hyperref[states--query]{States \&
Query}}{States \& Query}}\label{states--query}

This section is outdated. It may be still useful, but it is strongly
recommended to study new context system (using the reference).

Typst tries to be a \emph{pure language} as much as possible.

That means, a function can\textquotesingle t change anything outside of
it. That also means, if you call function, the result should be always
the same.

Unfortunately, our world (and therefore our documents)
isn\textquotesingle t pure. If you create a heading №2, you want the
next number to be three.

That section will guide you to using impure Typst. Don\textquotesingle t
overuse it, as this knowledge comes close to the Dark Arts of Typst!

\section{\texorpdfstring{\hyperref[states]{States}}{States}}\label{states}

This section is outdated. It may be still useful, but it is strongly
recommended to study new context system (using the reference).

Before we start something practical, it is important to understand
states in general.

Here is a good explanation of why do we \emph{need} them:
\href{https://typst.app/docs/reference/meta/state/}{Official Reference
about states} . It is highly recommended to read it first.

So instead of

\begin{verbatim}
#let x = 0
#let compute(expr) = {
  // eval evaluates string as Typst code
  // to calculate new x value
  x = eval(
    expr.replace("x", str(x))
  )
  [New value is #x.]
}

#compute("10") \
#compute("x + 3") \
#compute("x * 2") \
#compute("x - 5")
\end{verbatim}

\textbf{THIS DOES NOT COMPILE:} Variables from outside the function are
read-only and cannot be modified

Instead, you should write

\begin{verbatim}
#let s = state("x", 0)
#let compute(expr) = [
  // updates x current state with this function
  #s.update(x =>
    eval(expr.replace("x", str(x)))
  )
  // and displays it
  New value is #context s.get().
]

#compute("10") \
#compute("x + 3") \
#compute("x * 2") \
#compute("x - 5")

The computations will be made _in order_ they are _located_ in the document. So if you create computations first, but put them in the document later... See yourself:

#let more = [
  #compute("x * 2") \
  #compute("x - 5")
]

#compute("10") \
#compute("x + 3") \
#more
\end{verbatim}

\pandocbounded{\includesvg[keepaspectratio]{basics/states/typst-img/9a88397d1a9b5a44b1a3a218894595121bd4c5ec875df2b960638f2925060334-1.svg}}

\subsection{\texorpdfstring{\hyperref[context-magic]{Context
magic}}{Context magic}}\label{context-magic}

So what does this magic
\texttt{\ }{\texttt{\ context\ s.get()\ }}\texttt{\ } mean?

\begin{quote}
\href{https://typst.app/docs/reference/context/}{Context in Reference}
\end{quote}

In short, it specifies what part of your code (or markup) can
\emph{depend on states outside} . This context-expression is packed then
as one object, and it is evaluated on layout stage.

That means it is impossible to look from "normal" code at whatever is
inside the \texttt{\ }{\texttt{\ context\ }}\texttt{\ } . This is a
black box that would be known \emph{only after putting it into the
document} .

We will discuss \texttt{\ }{\texttt{\ context\ }}\texttt{\ } features
later.

\subsection{\texorpdfstring{\hyperref[operations-with-states]{Operations
with states}}{Operations with states}}\label{operations-with-states}

\subsubsection{\texorpdfstring{\hyperref[creating-new-state]{Creating
new state}}{Creating new state}}\label{creating-new-state}

\begin{verbatim}
#let x = state("state-id")
#let y = state("state-id", 2)

#x, #y

State is #context x.get() \ // the same as
#context [State is #y.get()] \ // the same as
#context {"State is" + str(y.get())}
\end{verbatim}

\pandocbounded{\includesvg[keepaspectratio]{basics/states/typst-img/4a52375bdeea2b7ca31dc51740563d01b3678f817dd6bc8c349d0714c2ac503f-1.svg}}

\subsubsection{\texorpdfstring{\hyperref[update]{Update}}{Update}}\label{update}

Updating is \emph{a content} that is an instruction. That instruction
tells compiler that in this place of document the state \emph{should be
updated} .

\begin{verbatim}
#let x = state("x", 0)
#context x.get() \
#let _ = x.update(3)
// nothing happens, we don't put `update` into the document flow
#context x.get()

#repr(x.update(3)) // this is how that content looks \

#context x.update(3)
#context x.get() // Finally!
\end{verbatim}

\pandocbounded{\includesvg[keepaspectratio]{basics/states/typst-img/3732a9c7bca8c4faedf9b024e09e647a65222c8244e9f3235a6057dfebc0a511-1.svg}}

Here we can see one of \emph{important
\texttt{\ }{\texttt{\ context\ }}\texttt{\ } traits} : it "sees" states
from outside, but can\textquotesingle t see how they change inside it:

\begin{verbatim}
#let x = state("x", 0)

#context {
  x.update(3)
  str(x.get())
}
\end{verbatim}

\pandocbounded{\includesvg[keepaspectratio]{basics/states/typst-img/78e500b80cb85e086a81302e2ce3dad88cb4304d4685b88e3f59111bc71f6748-1.svg}}

\subsubsection{\texorpdfstring{\hyperref[id-collision]{ID
collision}}{ID collision}}\label{id-collision}

\emph{TLDR; \textbf{Never allow colliding states.}}

States are described by their id-s, if they are the same, the code will
break.

So, if you write functions or loops that are used several times,
\emph{be careful} !

\begin{verbatim}
#let f(x) = {
  // return new state…
  // …but their id-s are the same!
  // so it will always be the same state!
  let y = state("x", 0)
  y.update(y => y + x)
  context y.get()
}

#let a = f(2)
#let b = f(3)

#a, #b \
#raw(repr(a) + "\n" + repr(b))
\end{verbatim}

\pandocbounded{\includesvg[keepaspectratio]{basics/states/typst-img/31a3e88747ed09ae6078bd3caf986f0e6ba744e055d0889d92bfa23941e7e451-1.svg}}

However, this \emph{may seem} okay:

\begin{verbatim}
// locations in code are different!
#let x = state("state-id")
#let y = state("state-id", 2)

#x, #y
\end{verbatim}

\pandocbounded{\includesvg[keepaspectratio]{basics/states/typst-img/1901e1449942d821c66f53bd6bc5fda10d63591aa45346fdf88bcbc3f2ab3425-1.svg}}

But in fact, it \emph{isn\textquotesingle t} :

\begin{verbatim}
#let x = state("state-id")
#let y = state("state-id", 2)

#context [#x.get(); #y.get()]

#x.update(3)

#context [#x.get(); #y.get()]
\end{verbatim}

\pandocbounded{\includesvg[keepaspectratio]{basics/states/typst-img/9185a298f9bcf8c519fa85481b9272e6ef3a00c117a0904d0509920a6abef8b2-1.svg}}

\section{\texorpdfstring{\hyperref[counters]{Counters}}{Counters}}\label{counters}

This section is outdated. It may be still useful, but it is strongly
recommended to study new context system (using the reference).

Counters are special states that \emph{count} elements of some type. As
with states, you can create your own with identifier strings.

\emph{Important:} to initiate counters of elements, you need to
\emph{set numbering for them} .

\subsection{\texorpdfstring{\hyperref[states-methods]{States
methods}}{States methods}}\label{states-methods}

Counters are states, so they can do all things states can do.

\begin{verbatim}
#set heading(numbering: "1.")

= Background
#counter(heading).update(3)
#counter(heading).update(n => n * 2)

== Analysis
Current heading number: #counter(heading).display().
\end{verbatim}

\pandocbounded{\includesvg[keepaspectratio]{basics/states/typst-img/c57c9907a5f238f0b5eee74f8c23c57a5e2d5b0c9cbf7ebd1befdfcbd33289df-1.svg}}

\begin{verbatim}
#let mine = counter("mycounter")
#mine.display()

#mine.step()
#mine.display()

#mine.update(c => c * 3)
#mine.display()
\end{verbatim}

\pandocbounded{\includesvg[keepaspectratio]{basics/states/typst-img/876103777c9564f0bb524f83a988a6d444c4e889baed31ee960548d90f3233e2-1.svg}}

\subsection{\texorpdfstring{\hyperref[displaying-counters]{Displaying
counters}}{Displaying counters}}\label{displaying-counters}

\begin{verbatim}
#set heading(numbering: "1.")

= Introduction
Some text here.

= Background
The current value is:
#counter(heading).display()

Or in roman numerals:
#counter(heading).display("I")
\end{verbatim}

\pandocbounded{\includesvg[keepaspectratio]{basics/states/typst-img/1ac65f4be42131b3cca1d7c56c6c60c3932a703e5e499c1c5cb874458028abea-1.svg}}

Counters also support displaying \emph{both current and final values}
out-of-box:

\begin{verbatim}
#set heading(numbering: "1.")

= Introduction
Some text here.

#counter(heading).display(both: true) \
#counter(heading).display("1 of 1", both: true) \
#counter(heading).display(
  (num, max) => [#num of #max],
   both: true
)

= Background
The current value is: #counter(heading).display()
\end{verbatim}

\pandocbounded{\includesvg[keepaspectratio]{basics/states/typst-img/af9d0da905bbb2215461b07b39653ef3890ff11a364afe018dae4ce4216f4961-1.svg}}

\subsection{\texorpdfstring{\hyperref[step]{Step}}{Step}}\label{step}

That\textquotesingle s quite easy, for counters you can increment value
using \texttt{\ }{\texttt{\ step\ }}\texttt{\ } . It works the same way
as \texttt{\ }{\texttt{\ update\ }}\texttt{\ } .

\begin{verbatim}
#set heading(numbering: "1.")

= Introduction
#counter(heading).step()

= Analysis
Let's skip 3.1.
#counter(heading).step(level: 2)

== Analysis
At #counter(heading).display().
\end{verbatim}

\pandocbounded{\includesvg[keepaspectratio]{basics/states/typst-img/12446a2258e9862d8df8b6b250ff14efbb9c35da165a2a04e8c4aa12c9b68cdf-1.svg}}

\subsection{\texorpdfstring{\hyperref[you-can-use-counters-in-your-functions]{You
can use counters in your
functions:}}{You can use counters in your functions:}}\label{you-can-use-counters-in-your-functions}

\begin{verbatim}
#let c = counter("theorem")
#let theorem(it) = block[
  #c.step()
  *Theorem #c.display():*
  #it
]

#theorem[$1 = 1$]
#theorem[$2 < 3$]
\end{verbatim}

\pandocbounded{\includesvg[keepaspectratio]{basics/states/typst-img/0f178f614e49a7400d646926705364a92ca3d4d888423b2693f071f83ce09e7d-1.svg}}

\section{\texorpdfstring{\hyperref[measure-layout]{Measure,
Layout}}{Measure, Layout}}\label{measure-layout}

This section is outdated. It may be still useful, but it is strongly
recommended to study new context system (using the reference).

\subsection{\texorpdfstring{\hyperref[style--measure]{Style \&
Measure}}{Style \& Measure}}\label{style--measure}

\begin{quote}
Style
\href{https://typst.app/docs/reference/foundations/style/}{documentation}
.
\end{quote}

\begin{quote}
Measure
\href{https://typst.app/docs/reference/layout/measure/}{documentation} .
\end{quote}

\texttt{\ }{\texttt{\ measure\ }}\texttt{\ } returns \emph{the element
size} . This command is extremely helpful when doing custom layout with
\texttt{\ }{\texttt{\ place\ }}\texttt{\ } .

However, there is a catch. Element size depends on styles, applied to
this element.

\begin{verbatim}
#let content = [Hello!]
#content
#set text(14pt)
#content
\end{verbatim}

\pandocbounded{\includesvg[keepaspectratio]{basics/typst-img/00a6cbbc650947c03f34564786b0645eee60396f288d26333c591ff9059cc369-1.svg}}

So if we will set the big text size for some part of our text, to
measure the element\textquotesingle s size, we have to know \emph{where
the element is located} . Without knowing it, we can\textquotesingle t
tell what styles should be applied.

So we need a scheme similar to
\texttt{\ }{\texttt{\ locate\ }}\texttt{\ } .

This is what \texttt{\ }{\texttt{\ styles\ }}\texttt{\ } function is
used for. It is \emph{a content} , which, when located in document,
calls a function inside on \emph{current styles} .

Now, when we got fixed \texttt{\ }{\texttt{\ styles\ }}\texttt{\ } , we
can get the element\textquotesingle s size using
\texttt{\ }{\texttt{\ measure\ }}\texttt{\ } :

\begin{verbatim}
#let thing(body) = style(styles => {
  let size = measure(body, styles)
  [Width of "#body" is #size.width]
})

#thing[Hey] \
#thing[Welcome]
\end{verbatim}

\pandocbounded{\includesvg[keepaspectratio]{basics/typst-img/5afe1855072b4ee8e343e5b5aa79affae5b17bc89738ffbe93dac245576cdd04-1.svg}}

\section{\texorpdfstring{\hyperref[layout]{Layout}}{Layout}}\label{layout}

Layout is similar to \texttt{\ }{\texttt{\ measure\ }}\texttt{\ } , but
it returns current scope \textbf{parent size} .

If you are putting elements in block, that will be
block\textquotesingle s size. If you are just putting right on the page,
that will be page\textquotesingle s size.

As parent\textquotesingle s size depends on it\textquotesingle s place
in document, it uses the similar scheme to
\texttt{\ }{\texttt{\ locate\ }}\texttt{\ } and
\texttt{\ }{\texttt{\ style\ }}\texttt{\ } :

\begin{verbatim}
#layout(size => {
  let half = 50% * size.width
  [Half a page is #half wide.]
})
\end{verbatim}

\pandocbounded{\includesvg[keepaspectratio]{basics/typst-img/c68a166f6e6b1b3229fd56478ae302dbeb39c882e229c69d4c6ebb6c9c528985-1.svg}}

It may be extremely useful to combine
\texttt{\ }{\texttt{\ layout\ }}\texttt{\ } with
\texttt{\ }{\texttt{\ measure\ }}\texttt{\ } , to get width of things
that depend on parent\textquotesingle s size:

\begin{verbatim}
#let text = lorem(30)
#layout(size => style(styles => [
  #let (height,) = measure(
    block(width: size.width, text),
    styles,
  )
  This text is #height high with
  the current page width: \
  #text
]))
\end{verbatim}

\pandocbounded{\includesvg[keepaspectratio]{basics/typst-img/93167dc0b22b02fe27aa92c6b03c2281665b4352624364a19c63f61a488aa75a-1.svg}}

\section{\texorpdfstring{\hyperref[query]{Query}}{Query}}\label{query}

This section is outdated. It may be still useful, but it is strongly
recommended to study new context system (using the reference).

\begin{quote}
Link \href{https://typst.app/docs/reference/meta/query/}{there}
\end{quote}

Query is a thing that allows you getting location by \emph{selector}
(this is the same thing we used in show rules).

That enables "time travel", getting information about document from its
parts and so on. \emph{That is a way to violate Typst\textquotesingle s
purity.}

It is currently one of the \emph{the darkest magics currently existing
in Typst} . It gives you great powers, but with great power comes great
responsibility.

\subsection{\texorpdfstring{\hyperref[time-travel]{Time
travel}}{Time travel}}\label{time-travel}

\begin{verbatim}
#let s = state("x", 0)
#let compute(expr) = [
  #s.update(x =>
    eval(expr.replace("x", str(x)))
  )
  New value is #s.display().
]

Value at `<here>` is
#context s.at(
  query(<here>)
    .first()
    .location()
)

#compute("10") \
#compute("x + 3") \
*Here.* <here> \
#compute("x * 2") \
#compute("x - 5")
\end{verbatim}

\pandocbounded{\includesvg[keepaspectratio]{basics/states/typst-img/130940aa5ae2ceb3364ef655c84cf8e7d2178210851b8fb20e6c0c3345c3ace7-1.svg}}

\subsection{\texorpdfstring{\hyperref[getting-nearest-chapter]{Getting
nearest
chapter}}{Getting nearest chapter}}\label{getting-nearest-chapter}

\begin{verbatim}
#set page(header: context {
  let elems = query(
    selector(heading).before(here()),
    here(),
  )
  let academy = smallcaps[
    Typst Academy
  ]
  if elems == () {
    align(right, academy)
  } else {
    let body = elems.last().body
    academy + h(1fr) + emph(body)
  }
})

= Introduction
#lorem(23)

= Background
#lorem(30)

= Analysis
#lorem(15)
\end{verbatim}

\pandocbounded{\includesvg[keepaspectratio]{basics/states/typst-img/b4d0562911dd308b0d9cbc36ad20ba6ed91fc3c3da5b6259eb6721f3a53a18e3-1.svg}}

\section{\texorpdfstring{\hyperref[metadata]{Metadata}}{Metadata}}\label{metadata}

Metadata is invisible content that can be extracted using query or other
content. This may be very useful with
\texttt{\ }{\texttt{\ typst\ query\ }}\texttt{\ } to pass values to
external tools.

\begin{verbatim}
// Put metadata somewhere.
#metadata("This is a note") <note>

// And find it from anywhere else.
#context {
  query(<note>).first().value
}
\end{verbatim}

\pandocbounded{\includesvg[keepaspectratio]{basics/states/typst-img/ef1c7d9faf74901f6c5266d48ae006167003a22754408a70ae9f9d1088b5fe24-1.svg}}

\section{\texorpdfstring{\hyperref[math-1]{Math}}{Math}}\label{math-1}

Math is a special environment that has special features related to...
math.

\subsection{\texorpdfstring{\hyperref[syntax]{Syntax}}{Syntax}}\label{syntax}

To start math environment, \texttt{\ }{\texttt{\ \$\ }}\texttt{\ } . The
spacing around \texttt{\ }{\texttt{\ \$\ }}\texttt{\ } will make it
either \emph{inline} math (smaller, used in text) or \emph{display} math
(used on math equations on their own).

\begin{verbatim}
// This is inline math
Let $a$, $b$, and $c$ be the side
lengths of right-angled triangle.
Then, we know that:

// This is display math
$ a^2 + b^2 = c^2 $

Prove by induction:

// You can use new lines as spacing too!
$
sum_(k=1)^n k = (n(n+1)) / 2
$
\end{verbatim}

\pandocbounded{\includesvg[keepaspectratio]{basics/math/typst-img/068db3a521a38c3acede771ebb6342807cca4fd98baf5b2b508184a6854ea8ff-1.svg}}

\subsection{\texorpdfstring{\hyperref[mathequation]{Math.equation}}{Math.equation}}\label{mathequation}

The element that math is displayed in is called
\texttt{\ }{\texttt{\ math.equation\ }}\texttt{\ } . You can use it for
set/show rules:

\begin{verbatim}
#show math.equation: set text(red)

$
integral_0^oo (f(t) + g(t))/2
$
\end{verbatim}

\pandocbounded{\includesvg[keepaspectratio]{basics/math/typst-img/94e0532dd7224d08e966cb82834283efd8889d7f117b04116e721a788bfcc16c-1.svg}}

Any symbol/command that is available in math, \emph{is also available}
in code mode using \texttt{\ }{\texttt{\ math.command\ }}\texttt{\ } :

\begin{verbatim}
#math.integral, #math.underbrace([a + b], [c])
\end{verbatim}

\pandocbounded{\includesvg[keepaspectratio]{basics/math/typst-img/b4ca12d7f34ed342f3cb3fba2ee1f5b58faa8fceecb74671baacd9166fcbb5aa-1.svg}}

\subsection{\texorpdfstring{\hyperref[letters-and-commands]{Letters and
commands}}{Letters and commands}}\label{letters-and-commands}

Typst aims to have as simple and effective syntax for math as possible.
That means no special symbols, just using commands.

To make it short, Typst uses several simple rules:

\begin{itemize}
\item
  All single-letter words \emph{turn into variables} . That includes any
  \emph{unicode symbols} too!
\item
  All multi-letter words \emph{turn into commands} . They may be
  built-in commands (available with math.something outside of math
  environment). Or they \textbf{may be user-defined variables/functions}
  . If the command \textbf{isn\textquotesingle t defined} , there will
  be \textbf{compilation error} .

  If you use kebab-case or snake\_case for variables you want to use in
  math, you will have to refer to them as \#snake-case-variable.
\item
  To write simple text, use quotes:

\begin{verbatim}
$a "equals to" 2$
\end{verbatim}

  \pandocbounded{\includesvg[keepaspectratio]{basics/math/typst-img/811f30ede68d08bec254f184c1be319958c3e11f9f9d58c40b2f460bba037e3d-1.svg}}

  Spacing matters there!

\begin{verbatim}
$a "is" 2$, $a"is"2$
\end{verbatim}

  \pandocbounded{\includesvg[keepaspectratio]{basics/math/typst-img/9cc2d263c76646c623e1e6b73756e1fe1e2c56d7fe0324ee945652107e6456ba-1.svg}}
\item
  You can turn it into multi-letter variables using
  \texttt{\ }{\texttt{\ italic\ }}\texttt{\ } :

\begin{verbatim}
$(italic("mass") v^2)/2$
\end{verbatim}

  \pandocbounded{\includesvg[keepaspectratio]{basics/math/typst-img/141d8a3b9beb3559387411170f7378078c80cb2ff80d8d5f5345c3231f55df9c-1.svg}}
\end{itemize}

Commands see
\href{https://typst.app/docs/reference/math/\#definitions}{there} (go to
the links to see the commands).

All symbols see
\href{https://typst.app/docs/reference/symbols/sym/}{there} .

\subsection{\texorpdfstring{\hyperref[multiline-equations]{Multiline
equations}}{Multiline equations}}\label{multiline-equations}

To create multiline \emph{display equation} , use the same symbol as in
markup mode: \texttt{\ }{\texttt{\ \textbackslash{}\ }}\texttt{\ } :

\begin{verbatim}
$
a = b\
a = c
$
\end{verbatim}

\pandocbounded{\includesvg[keepaspectratio]{basics/math/typst-img/2f16d9e64e38ff22ca27a09b0d8eaef1b020e4eccd7d2ce1380e10a0efcea163-1.svg}}

\subsection{\texorpdfstring{\hyperref[escaping]{Escaping}}{Escaping}}\label{escaping}

Any symbol that is used may be escaped with
\texttt{\ }{\texttt{\ \textbackslash{}\ }}\texttt{\ } , like in markup
mode. For example, you can disable fraction:

\begin{verbatim}
$
a  / b \
a \/ b
$
\end{verbatim}

\pandocbounded{\includesvg[keepaspectratio]{basics/math/typst-img/e7931e55a2772ee992446af8506d8d25b96167e3bb71d5c63ed8ca156530f2d9-1.svg}}

The same way it works with any other syntax.

\subsection{\texorpdfstring{\hyperref[wrapping-inline-math]{Wrapping
inline math}}{Wrapping inline math}}\label{wrapping-inline-math}

Sometimes, when you write large math, it may be too close to text
(especially for some long letter tails).

\begin{verbatim}
#lorem(17) $display(1)/display(1+x^n)$ #lorem(20)
\end{verbatim}

\pandocbounded{\includesvg[keepaspectratio]{basics/math/typst-img/a9cce2b851a01939a0abfc02e8cd994d20c465d2800cf64c5c6051ead5bc4e9a-1.svg}}

You may easily increase the distance it by wrapping into box:

\begin{verbatim}
#lorem(17) #box($display(1)/display(1+x^n)$, inset: 0.2em) #lorem(20)
\end{verbatim}

\pandocbounded{\includesvg[keepaspectratio]{basics/math/typst-img/ee9fc5a3ec529a9f3e811a70724c1585c294d82454c22ee9343235556f572792-1.svg}}

\section{\texorpdfstring{\hyperref[symbols]{Symbols}}{Symbols}}\label{symbols}

Multiletter words in math refer either to local variables, functions,
text operators, spacing or \emph{special symbols} . The latter are very
important for advanced math.

\begin{verbatim}
$
forall v, w in V, alpha in KK: alpha dot (v + w) = alpha v + alpha w
$
\end{verbatim}

\pandocbounded{\includesvg[keepaspectratio]{basics/math/typst-img/60a6e3e08582c87ec082b6714a45a90a914dd1299f788e2bb21b0cc5adc80e6a-1.svg}}

You can write the same with unicode:

\begin{verbatim}
$
∀ v, w ∈ V, α ∈ 𝕂: α ⋅ (v + w) = α v + α w
$
\end{verbatim}

\pandocbounded{\includesvg[keepaspectratio]{basics/math/typst-img/d37776c21d5c4d692e4ebbe7e5ce7e7cdf5e2c0777a88a47abe0c0c5992cf41a-1.svg}}

\subsection{\texorpdfstring{\hyperref[symbols-naming]{Symbols
naming}}{Symbols naming}}\label{symbols-naming}

\begin{quote}
See all available symbols list
\href{https://typst.app/docs/reference/symbols/sym/}{there} .
\end{quote}

\subsubsection{\texorpdfstring{\hyperref[general-idea]{General
idea}}{General idea}}\label{general-idea}

Typst wants to define some "basic" symbols with small easy-to-remember
words, and build complex ones using combinations. For example,

\begin{verbatim}
$
// cont — contour
integral, integral.cont, integral.double, integral.square, sum.integral\

// lt — less than, gt — greater than
lt, lt.circle, lt.eq, lt.not, lt.eq.not, lt.tri, lt.tri.eq, lt.tri.eq.not, gt, lt.gt.eq, lt.gt.not
$
\end{verbatim}

\pandocbounded{\includesvg[keepaspectratio]{basics/math/typst-img/a0ee196d2bf305ca6c2d812008f9955e5ae526de0b0ac0b83ca016a66bdc00f1-1.svg}}

I highly recommend using WebApp/Typst LSP when writing math with lots of
complex symbols. That helps you to quickly choose the right symbol
within all combinations.

Sometimes the names are not obvious, for example, sometimes it is used
prefix \texttt{\ }{\texttt{\ n-\ }}\texttt{\ } instead of
\texttt{\ }{\texttt{\ not\ }}\texttt{\ } :

\begin{verbatim}
$
gt.nequiv, gt.napprox, gt.ntilde, gt.tilde.not
$
\end{verbatim}

\pandocbounded{\includesvg[keepaspectratio]{basics/math/typst-img/e4d0ef024efaf9f4334ebf04a2ac4e015fc5ec76617be8b6d7aad2f4429e3317-1.svg}}

\subsubsection{\texorpdfstring{\hyperref[common-modifiers]{Common
modifiers}}{Common modifiers}}\label{common-modifiers}

\begin{itemize}
\item
  \texttt{\ }{\texttt{\ .b,\ .t,\ .l,\ .r\ }}\texttt{\ } : bottom, top,
  left, right. Change direction of symbol.

\begin{verbatim}
$arrow.b, triangle.r, angle.l$
\end{verbatim}

  \pandocbounded{\includesvg[keepaspectratio]{basics/math/typst-img/8ab0fa590b7a39023b1467e7a376a4810f997f720dd5d221ad83d7e741943b55-1.svg}}
\end{itemize}
