\title{sitandr.github.io/typst-examples-book/book/basics/states/index}

\section{\texorpdfstring{\hyperref[states--query]{States \&
Query}}{States \& Query}}\label{states--query}

This section is outdated. It may be still useful, but it is strongly
recommended to study new context system (using the reference).

Typst tries to be a \emph{pure language} as much as possible.

That means, a function can\textquotesingle t change anything outside of
it. That also means, if you call function, the result should be always
the same.

Unfortunately, our world (and therefore our documents)
isn\textquotesingle t pure. If you create a heading №2, you want the
next number to be three.

That section will guide you to using impure Typst. Don\textquotesingle t
overuse it, as this knowledge comes close to the Dark Arts of Typst!
