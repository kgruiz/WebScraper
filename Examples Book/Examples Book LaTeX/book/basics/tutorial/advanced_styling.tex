\title{sitandr.github.io/typst-examples-book/book/basics/tutorial/advanced_styling}

\section{\texorpdfstring{\hyperref[advanced-styling]{Advanced
styling}}{Advanced styling}}\label{advanced-styling}

\subsection{\texorpdfstring{\hyperref[the-show-rule]{The
\texttt{\ }{\texttt{\ show\ }}\texttt{\ }
rule}}{The   show   rule}}\label{the-show-rule}

\begin{verbatim}
Advanced styling comes with another rule. The _`show` rule_.

Now please compare the source code and the output.

#show "Be careful": strong[Play]

This is a very powerful thing, sometimes even too powerful.
Be careful with it.

#show "it is holding me hostage": text(green)[I'm fine]

Wait, what? I told you "Be careful!", not "Play!".

Help, it is holding me hostage.
\end{verbatim}

\pandocbounded{\includesvg[keepaspectratio]{typst-img/8a9ac38769d4ac7b42a2755047d0cd5a6404ad26e9e7f5b72b6984fa67abadf9-1.svg}}

\subsection{\texorpdfstring{\hyperref[now-a-bit-more-serious]{Now a bit
more serious}}{Now a bit more serious}}\label{now-a-bit-more-serious}

\begin{verbatim}
Show rule is a powerful thing that takes a _selector_
and what to apply to it. After that it will apply to
all elements it can find.

It may be extremely useful like that:

#show emph: set text(blue)

Now if I want to _emphasize_ something,
it will be both _emphasized_ and _blue_.
Isn't that cool?
\end{verbatim}

\pandocbounded{\includesvg[keepaspectratio]{typst-img/657acaf5c4ca684408bbc6fe0dec4c74b9fa58d24805ec975be1382aa7bf959c-1.svg}}

\subsection{\texorpdfstring{\hyperref[about-syntax]{About
syntax}}{About syntax}}\label{about-syntax}

\begin{verbatim}
Sometimes show rules may be confusing. They may seem very diverse, but in fact they all are quite the same! So

// actually, this is the same as
// redify = text.with(red)
// `with` creates a new function with this argument already set
#let redify(string) = text(red, string)

// and this is the same as
// framify = rect.with(stroke: orange)
#let framify(object) = rect(object, stroke: orange)

// set default color of text blue for all following text
#show: set text(blue)

Blue text.

// wrap everything into a frame
#show: framify

Framed text.

// it's the same, just creating new function that calls framify
#show: a => framify(a)

Double-framed.

// apply function to `the`
#show "the": redify
// set text color for all the headings
#show heading: set text(purple)

= Conclusion

All these rules do basically the same!
\end{verbatim}

\pandocbounded{\includesvg[keepaspectratio]{typst-img/2dfcde68345d3fa276b99a1f308343118c6eeae09fd106389a8fc488d7244ebb-1.svg}}

\subsection{\texorpdfstring{\hyperref[blocks]{Blocks}}{Blocks}}\label{blocks}

One of the most important usages is that you can set up all spacing
using blocks. Like every element with text contains text that can be set
up, every \emph{block element} contains blocks:

\begin{verbatim}
Text before
= Heading
Text after

#show heading: set block(spacing: 0.5em)

Text before
= Heading
Text after
\end{verbatim}

\pandocbounded{\includesvg[keepaspectratio]{typst-img/7891207932d0918c88b5804b3a7ee051ce5dda93081f8999eb0f7ebaee48400a-1.svg}}

\subsection{\texorpdfstring{\hyperref[selector]{Selector}}{Selector}}\label{selector}

\begin{verbatim}
So show rule can accept _selectors_.

There are lots of different selector types,
for example

- element functions
- strings
- regular expressions
- field filters

Let's see example of the latter:

#show heading.where(level: 1): set align(center)

= Title
== Small title

Of course, you can set align by hand,
no need to use show rules
(but they are very handy!):

#align(center)[== Centered small title]
\end{verbatim}

\pandocbounded{\includesvg[keepaspectratio]{typst-img/f41f337dd75b55211dd8d16e2682132c1ffb1ef19f774ba6cafc94cae090ec75-1.svg}}

\subsection{\texorpdfstring{\hyperref[custom-formatting]{Custom
formatting}}{Custom formatting}}\label{custom-formatting}

\begin{verbatim}
Let's try now writing custom functions.
It is very easy, see yourself:

// "it" is a heading, we take it and output things in braces
#show heading: it => {
  // center it
  set align(center)
  // set size and weight
  set text(12pt, weight: "regular")
  // see more about blocks and boxes
  // in corresponding chapter
  block(smallcaps(it.body))
}

= Smallcaps heading
\end{verbatim}

\pandocbounded{\includesvg[keepaspectratio]{typst-img/a5c37bce3cf9a077a4eb62a4d95f89584b5ef8acee279b81de6019d0e5768ba0-1.svg}}

\subsection{\texorpdfstring{\hyperref[setting-spacing]{Setting
spacing}}{Setting spacing}}\label{setting-spacing}

TODO: explain block spacing for common elements

\subsection{\texorpdfstring{\hyperref[formatting-to-get-an-article-look]{Formatting
to get an "article
look"}}{Formatting to get an "article look"}}\label{formatting-to-get-an-article-look}

\begin{verbatim}
#set page(
  // Header is that small thing on top
  header: align(
    right + horizon,
    [Some header there]
  ),
  height: 12cm
)

#align(center, text(17pt)[
  *Important title*
])

#grid(
  columns: (1fr, 1fr),
  align(center)[
    Some author \
    Some Institute \
    #link("mailto:some@mail.edu")
  ],
  align(center)[
    Another author \
    Another Institute \
    #link("mailto:another@mail.edu")
  ]
)

Now let's split text into two columns:

#show: rest => columns(2, rest)

#show heading.where(
  level: 1
): it => block(width: 100%)[
  #set align(center)
  #set text(12pt, weight: "regular")
  #smallcaps(it.body)
]

#show heading.where(
  level: 2
): it => text(
  size: 11pt,
  weight: "regular",
  style: "italic",
  it.body + [.],
)

// Now let's fill it with words:

= Heading
== Small heading
#lorem(10)
== Second subchapter
#lorem(10)
= Second heading
#lorem(40)

== Second subchapter
#lorem(40)
\end{verbatim}

\pandocbounded{\includesvg[keepaspectratio]{typst-img/76ee0cca809299df178ec9d94371c01031d1808a700b39deac5245dd6b83157f-1.svg}}
