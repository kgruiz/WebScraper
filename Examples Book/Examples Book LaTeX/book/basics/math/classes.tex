\title{sitandr.github.io/typst-examples-book/book/basics/math/classes}

\section{\texorpdfstring{\hyperref[classes]{Classes}}{Classes}}\label{classes}

\begin{quote}
See \href{https://typst.app/docs/reference/math/class/}{official
documentation}
\end{quote}

Each math symbol has its own "class", the way it behaves.
That\textquotesingle s one of the main reasons why they are layouted
differently.

\subsection{\texorpdfstring{\hyperref[classes-1]{Classes}}{Classes}}\label{classes-1}

\begin{verbatim}
$
a b c\
a class("normal", b) c\
a class("punctuation", b) c\
a class("opening", b) c\
a lr(b c]) c\
a lr(class("opening", b) c ]) c // notice it is moved vertically \
a class("closing", b) c\
a class("fence", b) c\
a class("large", b) c\
a class("relation", b) c\
a class("unary", b) c\
a class("binary", b) c\
a class("vary", b) c\
$
\end{verbatim}

\pandocbounded{\includesvg[keepaspectratio]{typst-img/5d4604274229b2f53ee04b88ff0e73d9aa8365643c5e60052fcca1298d4f5a23-1.svg}}

\subsection{\texorpdfstring{\hyperref[setting-class-for-symbol]{Setting
class for
symbol}}{Setting class for symbol}}\label{setting-class-for-symbol}

\begin{verbatim}
Default:

$square circle square$

With `#h(0)`:

$square #h(0pt) circle #h(0pt) square$

With `math.class`:

#show math.circle: math.class.with("normal")
$square circle square$
\end{verbatim}

\pandocbounded{\includesvg[keepaspectratio]{typst-img/86a709c6189649b79005752253a842631eed4722b350e4197116e0be19094035-1.svg}}
