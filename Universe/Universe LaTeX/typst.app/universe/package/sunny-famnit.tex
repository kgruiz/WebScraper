\title{typst.app/universe/package/sunny-famnit}

\phantomsection\label{banner}
\phantomsection\label{template-thumbnail}
\pandocbounded{\includegraphics[keepaspectratio]{https://packages.typst.org/preview/thumbnails/sunny-famnit-0.2.0-small.webp}}

\section{sunny-famnit}\label{sunny-famnit}

{ 0.2.0 }

Thesis template for University of Primorska, FAMNIT

\href{/app?template=sunny-famnit&version=0.2.0}{Create project in app}

\phantomsection\label{readme}
\pandocbounded{\includegraphics[keepaspectratio]{https://img.shields.io/github/v/release/Tiggax/famnit_typst_template}}
\pandocbounded{\includegraphics[keepaspectratio]{https://img.shields.io/github/stars/Tiggax/famnit_typst_template}}

\pandocbounded{\includegraphics[keepaspectratio]{https://www.famnit.upr.si/img/UP_FAMNIT.png}}

\emph{University of Primorska,}

\emph{Faculty of Mathematics, Natural Sciences and Information
Technologies}

\begin{center}\rule{0.5\linewidth}{0.5pt}\end{center}

This is a Typst template for FAMNIT final work.

\begin{center}\rule{0.5\linewidth}{0.5pt}\end{center}

\subsection{configuration example}\label{configuration-example}

\begin{Shaded}
\begin{Highlighting}[]
\NormalTok{\#import "@preview/sunny{-}famnit:0.2.0": project}

\NormalTok{\#show project.with(}
\NormalTok{    date: datetime(day: 1, month: 1, year: 2024), // you could also do \textasciigrave{}datetime.today()\textasciigrave{}}
\NormalTok{    text\_lang: "en" // the language that the thesis is gonna be written in.}
    
\NormalTok{    author: "your name"}
\NormalTok{    studij: "your course",}
\NormalTok{    mentor: (}
\NormalTok{    name: "his name", }
\NormalTok{    en: ("prepends","postpends"), // you can prepend or postpend any titles}
\NormalTok{    sl: ("predstavki","postavki"),// you can prepend or postpend any titles}
\NormalTok{    ),}
\NormalTok{    somentor: none, // if you have a co{-}mentor write him here the same way as mentor, else you can just remove the line.}
\NormalTok{    work\_mentor: none, // if you have a work mentor, the same as above}

\NormalTok{    naslov: "your title in slovene",}
\NormalTok{    title: "your title",}

\NormalTok{    izvleček: [}
\NormalTok{        your abstract in slovene.}
\NormalTok{    ],}
\NormalTok{    abstract: [}
\NormalTok{        your abstract}
\NormalTok{    ],}

\NormalTok{    ključne\_besede: ("Typst", "je", "super!"),}
\NormalTok{    key\_words: ("Typst", "is", "Awesome!"),}

\NormalTok{    kratice: (}
\NormalTok{        "Famnit": "Fakulteta za matematiko naravoslovje in informacijske tehnologije",}
\NormalTok{        "PDF": "Portable document format",}
\NormalTok{    ),}

\NormalTok{    priloge: (), // you can add attachments as a dict of a title and content like \textasciigrave{}"name": [content],\textasciigrave{}}

\NormalTok{    zahvala: [}
\NormalTok{        you can add an acknowlegment.}
\NormalTok{    ],}

\NormalTok{  bib\_file: bibliography(}
\NormalTok{    "my\_references.bib",}
\NormalTok{    style: "ieee",}
\NormalTok{    title: [Bibliography],}
\NormalTok{  ),}

\NormalTok{    /* Additional content and their defaults}
\NormalTok{    kraj: "Koper",}
\NormalTok{    */}
\NormalTok{)}

\NormalTok{// Your content goes below.}
\end{Highlighting}
\end{Shaded}

\subsection{Abbreviations (kratice)}\label{abbreviations-kratice}

You can specify Abbreviations at the start as an attribute
\texttt{\ kratice\ } and pass it a dictionary of the abbriviation and
it’s explanation. Then you can reference them in text using
\texttt{\ @\textless{}short\ name\textgreater{}\ } to create a link to
it.

\subsection{Attachments}\label{attachments}

Some thesis need Attachments that are shown at the end of the file. To
add these attachments add them in your project under
\texttt{\ priloge\ } as a dictionary of the attachment name and its
content. I suggest having a seperate \texttt{\ attachments.typ\ } file,
from where you can reference them in the main project.

\subsection{Language}\label{language}

The writing of the thesis can be achieved in two languages; Slovene and
English. They have some differences between them in the way the template
is generated, as the thesis needs to be different for each one. you can
specify the language with the \texttt{\ text\_lang\ } attribute.

\begin{center}\rule{0.5\linewidth}{0.5pt}\end{center}

If you have any questions, suggestion or improvements open an issue or a
pull request
\href{https://github.com/Tiggax/famnit_typst_template}{here}

\href{/app?template=sunny-famnit&version=0.2.0}{Create project in app}

\subsubsection{How to use}\label{how-to-use}

Click the button above to create a new project using this template in
the Typst app.

You can also use the Typst CLI to start a new project on your computer
using this command:

\begin{verbatim}
typst init @preview/sunny-famnit:0.2.0
\end{verbatim}

\includesvg[width=0.16667in,height=0.16667in]{/assets/icons/16-copy.svg}

\subsubsection{About}\label{about}

\begin{description}
\tightlist
\item[Author :]
Tilen Gimpelj {[}@Tiggax{]}
\item[License:]
MIT
\item[Current version:]
0.2.0
\item[Last updated:]
July 19, 2024
\item[First released:]
March 18, 2024
\item[Minimum Typst version:]
0.11.0
\item[Archive size:]
5.38 kB
\href{https://packages.typst.org/preview/sunny-famnit-0.2.0.tar.gz}{\pandocbounded{\includesvg[keepaspectratio]{/assets/icons/16-download.svg}}}
\item[Repository:]
\href{https://github.com/Tiggax/famnit_typst_template}{GitHub}
\item[Discipline s :]
\begin{itemize}
\tightlist
\item[]
\item
  \href{https://typst.app/universe/search/?discipline=computer-science}{Computer
  Science}
\item
  \href{https://typst.app/universe/search/?discipline=biology}{Biology}
\item
  \href{https://typst.app/universe/search/?discipline=mathematics}{Mathematics}
\end{itemize}
\item[Categor y :]
\begin{itemize}
\tightlist
\item[]
\item
  \pandocbounded{\includesvg[keepaspectratio]{/assets/icons/16-mortarboard.svg}}
  \href{https://typst.app/universe/search/?category=thesis}{Thesis}
\end{itemize}
\end{description}

\subsubsection{Where to report issues?}\label{where-to-report-issues}

This template is a project of Tilen Gimpelj {[}@Tiggax{]} . Report
issues on \href{https://github.com/Tiggax/famnit_typst_template}{their
repository} . You can also try to ask for help with this template on the
\href{https://forum.typst.app}{Forum} .

Please report this template to the Typst team using the
\href{https://typst.app/contact}{contact form} if you believe it is a
safety hazard or infringes upon your rights.

\phantomsection\label{versions}
\subsubsection{Version history}\label{version-history}

\begin{longtable}[]{@{}ll@{}}
\toprule\noalign{}
Version & Release Date \\
\midrule\noalign{}
\endhead
\bottomrule\noalign{}
\endlastfoot
0.2.0 & July 19, 2024 \\
\href{https://typst.app/universe/package/sunny-famnit/0.1.0/}{0.1.0} &
March 18, 2024 \\
\end{longtable}

Typst GmbH did not create this template and cannot guarantee correct
functionality of this template or compatibility with any version of the
Typst compiler or app.
