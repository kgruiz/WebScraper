\title{typst.app/universe/package/jlyfish}

\phantomsection\label{banner}
\section{jlyfish}\label{jlyfish}

{ 0.1.0 }

Julia code evaluation inside your Typst document

\phantomsection\label{readme}
\pandocbounded{\includesvg[keepaspectratio]{https://github.com/typst/packages/raw/main/packages/preview/jlyfish/0.1.0/assets/logo.svg}}

Jlyfish is a package for Julia and Typst that allows you to integrate
Julia computations in your Typst document.

\href{https://github.com/andreasKroepelin/TypstJlyfish.jl/wiki}{\pandocbounded{\includegraphics[keepaspectratio]{https://img.shields.io/badge/docs-wiki-blue}}}
\pandocbounded{\includegraphics[keepaspectratio]{https://img.shields.io/github/license/andreasKroepelin/TypstJlyfish.jl}}
\pandocbounded{\includegraphics[keepaspectratio]{https://img.shields.io/github/v/release/andreasKroepelin/TypstJlyfish.jl}}
\href{https://github.com/andreasKroepelin/TypstJlyfish.jl}{\pandocbounded{\includegraphics[keepaspectratio]{https://img.shields.io/github/stars/andreasKroepelin/TypstJlyfish.jl}}}

You should use Jlyfish if you want to write a Typst document and have
some of the content automatically produced by Julia code but want the
source code for that within your document source. It fills a similar
role as \href{https://github.com/gpoore/pythontex}{PythonTeX} does for
Python and LaTeX. Note that this is different from tools like
\href{https://quarto.org/}{Quarto} where you write documents in
Markdown, also integrate some Julia code, but then might use Typst only
as a backend to produce the final document.

See below for a quick introduction or read the
\href{https://github.com/andreasKroepelin/TypstJlyfish.jl/wiki}{wiki}
for an in depth explanation.

Since Jlyfish builds a bridge between Julia and Typst, we also have to
get two things running. First, install the Julia package
\texttt{\ TypstJlyfish\ } from the general registry by executing

\begin{Shaded}
\begin{Highlighting}[]
\NormalTok{julia\textgreater{} ]}

\NormalTok{(@v1.10) pkg\textgreater{} add TypstJlyfish}
\end{Highlighting}
\end{Shaded}

You only have to do this once. (It is like installing and using the
Pluto notebook system, if you are familiar with that.)

When you want to use Jlyfish in a Typst document (say,
\texttt{\ your-document.typ\ } ), add the following line at the top:

\begin{Shaded}
\begin{Highlighting}[]
\NormalTok{\#import "@preview/jlyfish:0.1.0": *}
\end{Highlighting}
\end{Shaded}

Then, open a Julia REPL and run

\begin{Shaded}
\begin{Highlighting}[]
\NormalTok{julia\textgreater{} import TypstJlyfish}

\NormalTok{julia\textgreater{} TypstJlyfish.watch("your{-}document.typ")}
\end{Highlighting}
\end{Shaded}

Jlyfish facilitates the communication between Julia and Typst via a JSON
file. By default, Jlyfish uses the name of your document and adds a
\texttt{\ -jlyfish.json\ } , so \texttt{\ your-document.typ\ } would
become \texttt{\ your-document-jlyfish.json\ } . This can be configured,
of course.

To let Typst know of the computed data in the JSON file, add the
following line to your document:

\begin{Shaded}
\begin{Highlighting}[]
\NormalTok{\#read{-}julia{-}output(json("your{-}document{-}jlyfish.json"))}
\end{Highlighting}
\end{Shaded}

You can then place some Julia code in your Typst source using the
\texttt{\ \#jl\ } function:

\begin{Shaded}
\begin{Highlighting}[]
\NormalTok{What is the sum of the whole numbers from one to a hundred? \#jl(\textasciigrave{}sum(1:100)\textasciigrave{})}
\end{Highlighting}
\end{Shaded}

Head over to the
\href{https://github.com/andreasKroepelin/TypstJlyfish.jl/wiki}{wiki} to
learn more!

Just to show what is possible with Jlyfish:

\pandocbounded{\includesvg[keepaspectratio]{https://github.com/typst/packages/raw/main/packages/preview/jlyfish/0.1.0/examples/demo.svg}}

\begin{Shaded}
\begin{Highlighting}[]
\NormalTok{\#import "@preview/jlyfish:0.1.0": *}

\NormalTok{\#set page(width: auto, height: auto, margin: 1em)}
\NormalTok{\#set text(font: "Alegreya Sans")}
\NormalTok{\#let note = text.with(size: .7em, fill: luma(100), style: "italic")}

\NormalTok{\#read{-}julia{-}output(json("demo{-}jlyfish.json"))}
\NormalTok{\#jl{-}pkg("Colors", "Typstry", "Makie", "CairoMakie")}

\NormalTok{\#grid(}
\NormalTok{  columns: 2,}
\NormalTok{  gutter: 1em,}
\NormalTok{  align: top,}
\NormalTok{  [}
\NormalTok{    \#note[Generate Typst code in Julia:]}

\NormalTok{    \#set text(size: 4em)}
\NormalTok{    \#jl(\textasciigrave{}\textasciigrave{}\textasciigrave{}julia}
\NormalTok{      using Typstry, Colors}

\NormalTok{      parts = map([:red, :green, :purple], ["Ju", "li", "a"]) do name, text}
\NormalTok{        color = hex(Colors.JULIA\_LOGO\_COLORS[name])}
\NormalTok{        "\#text(fill: rgb(\textbackslash{}"$color\textbackslash{}"))[$text]"}
\NormalTok{      end}
\NormalTok{      TypstText(join(parts))}
\NormalTok{    \textasciigrave{}\textasciigrave{}\textasciigrave{})}
\NormalTok{  ],}
\NormalTok{  [}
\NormalTok{    \#note[Produce images in Julia:]}

\NormalTok{    \#set image(width: 10em)}
\NormalTok{    \#jl(recompute: false, \textasciigrave{}\textasciigrave{}\textasciigrave{}}
\NormalTok{      using Makie, CairoMakie}

\NormalTok{      as = {-}2.2:.01:.7}
\NormalTok{      bs = {-}1.5:.01:1.5}
\NormalTok{      C = [a + b * im for a in as, b in bs]}
\NormalTok{      function mandelbrot(c)}
\NormalTok{        z = c}
\NormalTok{        i = 1}
\NormalTok{        while i \textless{} 100 \&\& abs2(z) \textless{} 4}
\NormalTok{          z = z\^{}2 + c}
\NormalTok{          i += 1}
\NormalTok{        end}
\NormalTok{        i}
\NormalTok{      end}

\NormalTok{      contour(as, bs, mandelbrot.(C), axis = (;aspect = DataAspect()))}
\NormalTok{    \textasciigrave{}\textasciigrave{}\textasciigrave{})}
\NormalTok{  ],}
\NormalTok{  [}
\NormalTok{    \#note[Hand over raw data from Julia to Typst:]}
\NormalTok{    \#let barchart(counts) = \{}
\NormalTok{      set align(bottom)}
\NormalTok{      let bars = counts.map(count =\textgreater{} rect(}
\NormalTok{        width: .3em,}
\NormalTok{        height: count * 9em,}
\NormalTok{        stroke: white,}
\NormalTok{        fill: blue,}
\NormalTok{      ))}
\NormalTok{      stack(dir: ltr, ..bars)}
\NormalTok{    \}}

\NormalTok{    \#jl{-}raw(fn: it =\textgreater{} barchart(it.result.data), \textasciigrave{}\textasciigrave{}\textasciigrave{}julia}
\NormalTok{      p = .5}
\NormalTok{      n = 40}
\NormalTok{      counts = zeros(n + 1)}
\NormalTok{      for \_ in 1:10\_000}
\NormalTok{        count = 0}
\NormalTok{        for \_ in 1:n}
\NormalTok{          if rand() \textless{} p}
\NormalTok{            count += 1}
\NormalTok{          end}
\NormalTok{        end}
\NormalTok{        counts[count + 1] += 1}
\NormalTok{      end}

\NormalTok{      counts ./= maximum(counts)}
\NormalTok{      lo, hi = findfirst(\textgreater{}(1e{-}3), counts), findlast(\textgreater{}(1e{-}3), counts)}
\NormalTok{      counts[lo:hi]}
\NormalTok{    \textasciigrave{}\textasciigrave{}\textasciigrave{})}
\NormalTok{  ],}
\NormalTok{  [}
\NormalTok{    \#note[See errors, stdout, and logs:]}

\NormalTok{    \#jl(\textasciigrave{}\textasciigrave{}\textasciigrave{}julia}
\NormalTok{      println("Hello from stdout!")}
\NormalTok{      @info "Something to note" n p}
\NormalTok{      @warn "You should read this!"}
\NormalTok{      this\_does\_not\_exist}
\NormalTok{    \textasciigrave{}\textasciigrave{}\textasciigrave{})}
\NormalTok{  ]}
\NormalTok{)}
\end{Highlighting}
\end{Shaded}

\subsubsection{How to add}\label{how-to-add}

Copy this into your project and use the import as \texttt{\ jlyfish\ }

\begin{verbatim}
#import "@preview/jlyfish:0.1.0"
\end{verbatim}

\includesvg[width=0.16667in,height=0.16667in]{/assets/icons/16-copy.svg}

Check the docs for
\href{https://typst.app/docs/reference/scripting/\#packages}{more
information on how to import packages} .

\subsubsection{About}\label{about}

\begin{description}
\tightlist
\item[Author :]
Andreas Kröpelin
\item[License:]
MIT
\item[Current version:]
0.1.0
\item[Last updated:]
July 8, 2024
\item[First released:]
July 8, 2024
\item[Archive size:]
2.75 kB
\href{https://packages.typst.org/preview/jlyfish-0.1.0.tar.gz}{\pandocbounded{\includesvg[keepaspectratio]{/assets/icons/16-download.svg}}}
\item[Repository:]
\href{https://github.com/andreasKroepelin/TypstJlyfish.jl}{GitHub}
\item[Categor ies :]
\begin{itemize}
\tightlist
\item[]
\item
  \pandocbounded{\includesvg[keepaspectratio]{/assets/icons/16-code.svg}}
  \href{https://typst.app/universe/search/?category=scripting}{Scripting}
\item
  \pandocbounded{\includesvg[keepaspectratio]{/assets/icons/16-hammer.svg}}
  \href{https://typst.app/universe/search/?category=utility}{Utility}
\item
  \pandocbounded{\includesvg[keepaspectratio]{/assets/icons/16-integration.svg}}
  \href{https://typst.app/universe/search/?category=integration}{Integration}
\end{itemize}
\end{description}

\subsubsection{Where to report issues?}\label{where-to-report-issues}

This package is a project of Andreas Kröpelin . Report issues on
\href{https://github.com/andreasKroepelin/TypstJlyfish.jl}{their
repository} . You can also try to ask for help with this package on the
\href{https://forum.typst.app}{Forum} .

Please report this package to the Typst team using the
\href{https://typst.app/contact}{contact form} if you believe it is a
safety hazard or infringes upon your rights.

\phantomsection\label{versions}
\subsubsection{Version history}\label{version-history}

\begin{longtable}[]{@{}ll@{}}
\toprule\noalign{}
Version & Release Date \\
\midrule\noalign{}
\endhead
\bottomrule\noalign{}
\endlastfoot
0.1.0 & July 8, 2024 \\
\end{longtable}

Typst GmbH did not create this package and cannot guarantee correct
functionality of this package or compatibility with any version of the
Typst compiler or app.
