\title{typst.app/universe/package/a2c-nums}

\phantomsection\label{banner}
\section{a2c-nums}\label{a2c-nums}

{ 0.0.1 }

Convert a number to Chinese

\phantomsection\label{readme}
Convert Arabic numbers to Chinese characters.

\subsection{usage}\label{usage}

\begin{Shaded}
\begin{Highlighting}[]
\NormalTok{\#import "@preview/a2c{-}nums:0.0.1": int{-}to{-}cn{-}num, int{-}to{-}cn{-}ancient{-}num, int{-}to{-}cn{-}simple{-}num, num{-}to{-}cn{-}currency}

\NormalTok{\#int{-}to{-}cn{-}num(1234567890)}

\NormalTok{\#int{-}to{-}cn{-}ancient{-}num(1234567890)}

\NormalTok{\#int{-}to{-}cn{-}simple{-}num(2024)}

\NormalTok{\#num{-}to{-}cn{-}currency(1234567890.12)}
\end{Highlighting}
\end{Shaded}

\subsection{Functions}\label{functions}

\subsubsection{int-to-cn-num}\label{int-to-cn-num}

Convert an integer to Chinese number. ex:
\texttt{\ \#int-to-cn-num(123)\ } will be \texttt{\ 一百二å??三\ }

\subsubsection{int-to-cn-ancient-num}\label{int-to-cn-ancient-num}

Convert an integer to ancient Chinese number. ex:
\texttt{\ \#int-to-cn-ancient-num(123)\ } will be
\texttt{\ 壹佰贰拾å??\ }

\subsubsection{int-to-cn-simple-num}\label{int-to-cn-simple-num}

Convert an integer to simpple Chinese number. ex:
\texttt{\ \#int-to-cn-simple-num(2024)\ } will be
\texttt{\ 二〇二四\ }

\subsubsection{num-to-cn-currency}\label{num-to-cn-currency}

Convert a number to Chinese currency. ex:
\texttt{\ \#int-to-cn-simple-num(1234.56)\ } will be
\texttt{\ 壹仟贰佰å??拾肆元ä¼?角陆分\ }

\subsubsection{more details}\label{more-details}

Reference
\href{https://github.com/typst/packages/raw/main/packages/preview/a2c-nums/0.0.1/demo.typ}{demo.typ}
for more details please.

\subsubsection{How to add}\label{how-to-add}

Copy this into your project and use the import as \texttt{\ a2c-nums\ }

\begin{verbatim}
#import "@preview/a2c-nums:0.0.1"
\end{verbatim}

\includesvg[width=0.16667in,height=0.16667in]{/assets/icons/16-copy.svg}

Check the docs for
\href{https://typst.app/docs/reference/scripting/\#packages}{more
information on how to import packages} .

\subsubsection{About}\label{about}

\begin{description}
\tightlist
\item[Author :]
\href{mailto:soarowl@yeah.net}{Zhuo Nengwen}
\item[License:]
MIT
\item[Current version:]
0.0.1
\item[Last updated:]
January 8, 2024
\item[First released:]
January 8, 2024
\item[Minimum Typst version:]
0.10.0
\item[Archive size:]
2.54 kB
\href{https://packages.typst.org/preview/a2c-nums-0.0.1.tar.gz}{\pandocbounded{\includesvg[keepaspectratio]{/assets/icons/16-download.svg}}}
\item[Repository:]
\href{https://github.com/soarowl/a2c-nums.git}{GitHub}
\end{description}

\subsubsection{Where to report issues?}\label{where-to-report-issues}

This package is a project of Zhuo Nengwen . Report issues on
\href{https://github.com/soarowl/a2c-nums.git}{their repository} . You
can also try to ask for help with this package on the
\href{https://forum.typst.app}{Forum} .

Please report this package to the Typst team using the
\href{https://typst.app/contact}{contact form} if you believe it is a
safety hazard or infringes upon your rights.

\phantomsection\label{versions}
\subsubsection{Version history}\label{version-history}

\begin{longtable}[]{@{}ll@{}}
\toprule\noalign{}
Version & Release Date \\
\midrule\noalign{}
\endhead
\bottomrule\noalign{}
\endlastfoot
0.0.1 & January 8, 2024 \\
\end{longtable}

Typst GmbH did not create this package and cannot guarantee correct
functionality of this package or compatibility with any version of the
Typst compiler or app.
