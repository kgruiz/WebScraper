\title{typst.app/universe/package/bytefield}

\phantomsection\label{banner}
\section{bytefield}\label{bytefield}

{ 0.0.6 }

A package to create network protocol headers, memory map, register
definitions and more.

\phantomsection\label{readme}
A simple way to create network protocol headers, memory maps, register
definitions and more in typst.

âš~ï¸? Warning. As this package is still in an early stage, things might
break with the next version.

ℹ� If you find a bug or a feature which is missing, please open an
issue and/or send an PR.

\subsection{Example}\label{example}

\pandocbounded{\includegraphics[keepaspectratio]{https://github.com/typst/packages/raw/main/packages/preview/bytefield/0.0.6/docs/bytefield_example.png}}

\begin{Shaded}
\begin{Highlighting}[]
\NormalTok{\#import "@preview/bytefield:0.0.6": *}

\NormalTok{\#bytefield(}
\NormalTok{// Config the header}
\NormalTok{bitheader(}
\NormalTok{"bytes",}
\NormalTok{// adds every multiple of 8 to the header.}
\NormalTok{0, [start], // number with label}
\NormalTok{5,}
\NormalTok{// number without label}
\NormalTok{12, [\#text(14pt, fill: red, "test")],}
\NormalTok{23, [end\_test],}
\NormalTok{24, [start\_break],}
\NormalTok{36, [Fix], // will not be shown}
\NormalTok{angle: {-}50deg, // angle (default: {-}60deg)}
\NormalTok{text{-}size: 8pt, // length (default: global header\_font\_size or 9pt)}
\NormalTok{),}
\NormalTok{// Add data fields (bit, bits, byte, bytes) and notes}
\NormalTok{// A note always aligns on the same row as the start of the next data field.}
\NormalTok{note(left)[\#text(16pt, fill: blue, font: "Consolas", "Testing")],}
\NormalTok{bytes(3,fill: red.lighten(30\%))[Test],}
\NormalTok{note(right)[\#set text(9pt); \#sym.arrow.l This field \textbackslash{} breaks into 2 rows.],}
\NormalTok{bytes(2)[Break],}
\NormalTok{note(left)[\#set text(9pt); and continues \textbackslash{} here \#sym.arrow],}
\NormalTok{bits(24,fill: green.lighten(30\%))[Fill],}
\NormalTok{group(right,3)[spanning 3 rows],}
\NormalTok{bytes(12)[\#set text(20pt); *Multi* Row],}
\NormalTok{note(left, bracket: true)[Flags],}
\NormalTok{bits(4)[\#text(8pt)[reserved]],}
\NormalTok{flag[\#text(8pt)[SYN]],}
\NormalTok{flag(fill: orange.lighten(60\%))[\#text(8pt)[ACK]],}
\NormalTok{flag[\#text(8pt)[BOB]],}
\NormalTok{bits(25, fill: purple.lighten(60\%))[Padding],}
\NormalTok{// padding(fill: purple.lighten(40\%))[Padding],}
\NormalTok{bytes(2)[Next],}
\NormalTok{bytes(8, fill: yellow.lighten(60\%))[Multi break],}
\NormalTok{note(right)[\#emoji.checkmark Finish],}
\NormalTok{bytes(2)[\_End\_],}
\NormalTok{)}
\end{Highlighting}
\end{Shaded}

\subsection{Usage}\label{usage}

To use this library through the Typst package manager import bytefield
with \texttt{\ \#import\ "@preview/bytefield:0.0.6":\ *\ } at the top of
your file.

The package contains some of the most common network protocol headers
which are available under: \texttt{\ common.ipv4\ } ,
\texttt{\ common.ipv6\ } , \texttt{\ common.icmp\ } ,
\texttt{\ common.icmpv6\ } , \texttt{\ common.dns\ } ,
\texttt{\ common.tcp\ } , \texttt{\ common.udp\ } .

\subsection{Features}\label{features}

Here is a unsorted list of features which is possible right now.

\begin{itemize}
\tightlist
\item
  Adding fields with \texttt{\ bit\ } , \texttt{\ bits\ } ,
  \texttt{\ byte\ } or \texttt{\ bytes\ } function.

  \begin{itemize}
  \tightlist
  \item
    Fields can be colored
  \item
    Multirow and breaking fields are supported.
  \end{itemize}
\item
  Adding notes to the left or right with \texttt{\ note\ } or
  \texttt{\ group\ } function.
\item
  Config the header with the \texttt{\ bitheader\ } function. !Only one
  header per bytefield is processed currently.

  \begin{itemize}
  \tightlist
  \item
    Show numbers
  \item
    Show numbers and labels
  \item
    Show only labels
  \end{itemize}
\item
  Change the bit order in the header with \texttt{\ msb:left\ } or
  \texttt{\ msb:right\ } (default)
\end{itemize}

See
\href{https://github.com/typst/packages/raw/main/packages/preview/bytefield/0.0.6/example.typ}{example.typ}
for more information.

See
\href{https://github.com/typst/packages/raw/main/packages/preview/bytefield/0.0.6/CHANGELOG.md}{CHANGELOG.md}

\subsubsection{How to add}\label{how-to-add}

Copy this into your project and use the import as \texttt{\ bytefield\ }

\begin{verbatim}
#import "@preview/bytefield:0.0.6"
\end{verbatim}

\includesvg[width=0.16667in,height=0.16667in]{/assets/icons/16-copy.svg}

Check the docs for
\href{https://typst.app/docs/reference/scripting/\#packages}{more
information on how to import packages} .

\subsubsection{About}\label{about}

\begin{description}
\tightlist
\item[Author :]
\href{https://github.com/jomaway}{Jomaway}
\item[License:]
MIT
\item[Current version:]
0.0.6
\item[Last updated:]
May 24, 2024
\item[First released:]
September 3, 2023
\item[Minimum Typst version:]
0.10.0
\item[Archive size:]
12.0 kB
\href{https://packages.typst.org/preview/bytefield-0.0.6.tar.gz}{\pandocbounded{\includesvg[keepaspectratio]{/assets/icons/16-download.svg}}}
\item[Repository:]
\href{https://github.com/jomaway/typst-bytefield}{GitHub}
\end{description}

\subsubsection{Where to report issues?}\label{where-to-report-issues}

This package is a project of Jomaway . Report issues on
\href{https://github.com/jomaway/typst-bytefield}{their repository} .
You can also try to ask for help with this package on the
\href{https://forum.typst.app}{Forum} .

Please report this package to the Typst team using the
\href{https://typst.app/contact}{contact form} if you believe it is a
safety hazard or infringes upon your rights.

\phantomsection\label{versions}
\subsubsection{Version history}\label{version-history}

\begin{longtable}[]{@{}ll@{}}
\toprule\noalign{}
Version & Release Date \\
\midrule\noalign{}
\endhead
\bottomrule\noalign{}
\endlastfoot
0.0.6 & May 24, 2024 \\
\href{https://typst.app/universe/package/bytefield/0.0.5/}{0.0.5} &
March 11, 2024 \\
\href{https://typst.app/universe/package/bytefield/0.0.4/}{0.0.4} &
February 21, 2024 \\
\href{https://typst.app/universe/package/bytefield/0.0.3/}{0.0.3} &
November 20, 2023 \\
\href{https://typst.app/universe/package/bytefield/0.0.2/}{0.0.2} &
October 27, 2023 \\
\href{https://typst.app/universe/package/bytefield/0.0.1/}{0.0.1} &
September 3, 2023 \\
\end{longtable}

Typst GmbH did not create this package and cannot guarantee correct
functionality of this package or compatibility with any version of the
Typst compiler or app.
