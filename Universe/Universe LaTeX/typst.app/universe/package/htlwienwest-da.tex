\title{typst.app/universe/package/htlwienwest-da}

\phantomsection\label{banner}
\phantomsection\label{template-thumbnail}
\pandocbounded{\includegraphics[keepaspectratio]{https://packages.typst.org/preview/thumbnails/htlwienwest-da-0.1.0-small.webp}}

\section{htlwienwest-da}\label{htlwienwest-da}

{ 0.1.0 }

The diploma thesis template for students of the HTL Wien West.

{ } Officially affiliated

\href{/app?template=htlwienwest-da&version=0.1.0}{Create project in app}

\phantomsection\label{readme}
This is a Typst diploma thesis template for students of the HTL Wien
West. It fulfils all the necessary requirements for the diploma thesis.

\subsection{Usage}\label{usage}

You can use this template in the Typst web app by clicking “Start from
template� on the dashboard and searching for
\texttt{\ htlwienwest-da\ } .

Alternatively, you can use the CLI to kick this project off using the
command

\begin{verbatim}
typst init @preview/htlwienwest-da
\end{verbatim}

Typst will create a new directory with all the files needed to get you
started.

\subsection{Configuration}\label{configuration}

This template exports the \texttt{\ diplomarbeit\ } function with the
following named arguments:

\begin{itemize}
\tightlist
\item
  \texttt{\ titel\ } : \texttt{\ string\ } - The title of the thesis
\item
  \texttt{\ schuljahr\ } : \texttt{\ string\ } - The current school year
\item
  \texttt{\ abteilung\ } : \texttt{\ string\ } - The student’s
  department
\item
  \texttt{\ unterschrifts-datum\ } : \texttt{\ string\ } - The
  submission date
\item
  \texttt{\ autoren\ } : \texttt{\ array(dict)\ } - An array of all
  authors, represented as dictionaries. Each of them has the following
  properties

  \begin{itemize}
  \tightlist
  \item
    \texttt{\ vorname\ } : \texttt{\ string\ } - Firstname of the
    student
  \item
    \texttt{\ nachname\ } : \texttt{\ string\ } - Lastname of the
    student
  \item
    \texttt{\ klasse\ } : \texttt{\ string\ } - School class of the
    student
  \item
    \texttt{\ betreuer\ } : \texttt{\ dict\ } - The student’s advisor
    as dictionary

    \begin{itemize}
    \tightlist
    \item
      \texttt{\ name\ } : \texttt{\ string\ \textbar{}\ content\ } - The
      advisor’s name
    \item
      \texttt{\ geschlecht\ } :
      \texttt{\ "male"\ \textbar{}\ "female"\ } - Gender of advisor for
      correct gendering
    \end{itemize}
  \item
    \texttt{\ aufgaben\ } : \texttt{\ content\ } - The student’s
    responsibilities
  \end{itemize}
\item
  \texttt{\ kurzfassung\ } : \texttt{\ content\ } - Abstract in german
  as content block
\item
  \texttt{\ abstract\ } : \texttt{\ content\ } - Abstract in english as
  content block
\item
  \texttt{\ vorwort\ } : \texttt{\ content\ } - The thesis’ preface
\item
  \texttt{\ danksagung\ } : \texttt{\ content\ } - Acknowledgement
\item
  \texttt{\ anhang\ } : \texttt{\ content\ \textbar{}\ none\ } -
  Appendix
\item
  \texttt{\ literaturverzeichnis\ } : \texttt{\ function\ } - The
  bibliography prefilled with the BibTex file path
\end{itemize}

The function also accepts a single, positional argument for the body of
the paper.

The template will initialize your package with a sample call to the
\texttt{\ diplomarbeit\ } function in a show rule. If you want to change
an existing project to use thistemplate, you can add a show rule like
this at the top of your file:

\begin{Shaded}
\begin{Highlighting}[]
\NormalTok{\#import "@preview/htlwienwest{-}da:0.1.0": *}

\NormalTok{\#show: diplomarbeit.with(}
\NormalTok{  titel: "Titel der Diplomarbeit",}
\NormalTok{  abteilung: "Informationstechnologie",}
\NormalTok{  schuljahr: "2023/24",}
\NormalTok{  unterschrifts{-}datum: "20.04.2024",}
\NormalTok{  autoren: (}
\NormalTok{   (}
\NormalTok{     vorname: "Hans", nachname: "Mustermann",}
\NormalTok{     klasse: "5AHITN",}
\NormalTok{     betreuer: (name: "Dr. Walter Turbo", geschlecht: "male"),}
\NormalTok{     aufgaben: [}
\NormalTok{       \#lorem(100)}
\NormalTok{     ]}
\NormalTok{   ),}
\NormalTok{   (}
\NormalTok{     vorname: "Herta", nachname: "Musterfrau",}
\NormalTok{     klasse: "5AHITN",}
\NormalTok{     betreuer: (name: "Dipl.{-}Ing Maria Kreisel", geschlecht: "female"),}
\NormalTok{     aufgaben: [}
\NormalTok{       \#lorem(100)}
\NormalTok{     ]}
\NormalTok{   ),}
\NormalTok{  kurzfassung: [}
\NormalTok{    Die Kurzfassung muss die folgenden Inhalte darlegen (§8, Absatz 5 Prüfungsordnung): Thema, Fragestellung, Problemformulierung, wesentliche Ergebnisse. Sie soll einen prägnanten Überblick über die Arbeit geben.}
\NormalTok{  ],}
\NormalTok{  abstract: [}
\NormalTok{    Englische Version der Kurzfassung (siehe \#link(\textless{}Kurzfassung\textgreater{})[\_Kurzfassung\_])}
\NormalTok{  ],}
\NormalTok{  vorwort: [}
\NormalTok{    Perönlicher Zugang zum Thema. Gründe für die Themenwahl.}
\NormalTok{  ],}
\NormalTok{  danksagung: [}
\NormalTok{    Dank an Personen, die bei der Erstellung der Arbeit unterstützt haben.}
\NormalTok{  ],}
\NormalTok{  anhang: include "anhang.typ", // entfernen falls nicht benötigt}
\NormalTok{  literaturverzeichnis: bibliography.with("literaturverzeichnis.bib")}
\NormalTok{)}

\NormalTok{// Your content goes below.}
\end{Highlighting}
\end{Shaded}

\subsection{Provided Functions}\label{provided-functions}

Beside the \texttt{\ diplomarbeit\ } function, the template also
provides the \texttt{\ autor\ } function that is used after a heading to
indicate the specific author of the current section.

\begin{verbatim}
== Some Heading
#autor[Your Name]
\end{verbatim}

This will render additional information to the section’s heading.

To install the template locally, you can use

\begin{Shaded}
\begin{Highlighting}[]
\ExtensionTok{just}\NormalTok{ install}
\end{Highlighting}
\end{Shaded}

which uses the \href{https://github.com/casey/just}{just} command
runner.

If you don’t want to install \texttt{\ just\ } , you can run

\begin{Shaded}
\begin{Highlighting}[]
\FunctionTok{bash}\NormalTok{ ./scripts/package @local}
\end{Highlighting}
\end{Shaded}

The installed version can be used via \texttt{\ @local\ } instead of
\texttt{\ @preview\ } . To create a new typst project from the template,
run

\begin{Shaded}
\begin{Highlighting}[]
\ExtensionTok{typst}\NormalTok{ init @local/htlwienwest{-}da:}
\end{Highlighting}
\end{Shaded}

\href{/app?template=htlwienwest-da&version=0.1.0}{Create project in app}

\subsubsection{How to use}\label{how-to-use}

Click the button above to create a new project using this template in
the Typst app.

You can also use the Typst CLI to start a new project on your computer
using this command:

\begin{verbatim}
typst init @preview/htlwienwest-da:0.1.0
\end{verbatim}

\includesvg[width=0.16667in,height=0.16667in]{/assets/icons/16-copy.svg}

\subsubsection{About}\label{about}

\begin{description}
\tightlist
\item[Author s :]
\href{https://github.com/jozott00}{Johannes Zottele} \&
\href{https://github.com/peterw16}{peterw16}
\item[License:]
MIT
\item[Current version:]
0.1.0
\item[Last updated:]
May 3, 2024
\item[First released:]
May 3, 2024
\item[Archive size:]
61.9 kB
\href{https://packages.typst.org/preview/htlwienwest-da-0.1.0.tar.gz}{\pandocbounded{\includesvg[keepaspectratio]{/assets/icons/16-download.svg}}}
\item[Verification:]
We verified that the author is affiliated with their institution
\pandocbounded{\includesvg[keepaspectratio]{/assets/icons/16-verified.svg}}
\item[Repository:]
\href{https://github.com/htlwienwest/da-vorlage-typst}{GitHub}
\item[Categor y :]
\begin{itemize}
\tightlist
\item[]
\item
  \pandocbounded{\includesvg[keepaspectratio]{/assets/icons/16-mortarboard.svg}}
  \href{https://typst.app/universe/search/?category=thesis}{Thesis}
\end{itemize}
\end{description}

\subsubsection{Where to report issues?}\label{where-to-report-issues}

This template is a project of Johannes Zottele and peterw16 . Report
issues on \href{https://github.com/htlwienwest/da-vorlage-typst}{their
repository} . You can also try to ask for help with this template on the
\href{https://forum.typst.app}{Forum} .

Please report this template to the Typst team using the
\href{https://typst.app/contact}{contact form} if you believe it is a
safety hazard or infringes upon your rights.

\phantomsection\label{versions}
\subsubsection{Version history}\label{version-history}

\begin{longtable}[]{@{}ll@{}}
\toprule\noalign{}
Version & Release Date \\
\midrule\noalign{}
\endhead
\bottomrule\noalign{}
\endlastfoot
0.1.0 & May 3, 2024 \\
\end{longtable}

Typst GmbH did not create this template and cannot guarantee correct
functionality of this template or compatibility with any version of the
Typst compiler or app.
