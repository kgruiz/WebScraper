\title{typst.app/universe/package/easy-pinyin}

\phantomsection\label{banner}
\section{easy-pinyin}\label{easy-pinyin}

{ 0.1.0 }

Write Chinese pinyin easily.

\phantomsection\label{readme}
Write Chinese pinyin easily.

\subsection{Usage}\label{usage}

Import the package:

\begin{Shaded}
\begin{Highlighting}[]
\NormalTok{\#import "@preview/easy{-}pinyin:0.1.0": pinyin, zhuyin}
\end{Highlighting}
\end{Shaded}

With the \texttt{\ pinyin\ } function, you can use \texttt{\ a2\ } to
write an \texttt{\ É‘Ì?\ } , \texttt{\ o3\ } to write an \texttt{\ Ç’\ }
, \texttt{\ v4\ } to represent \texttt{\ ǜ\ } , etc.

With \texttt{\ zhuyin\ } function,you can put pinyin above the text
easily, with parameters:

\begin{itemize}
\tightlist
\item
  positional parameters:

  \begin{itemize}
  \tightlist
  \item
    \texttt{\ doc:\ content\textbar{}string\ } : main characters
  \item
    \texttt{\ ruby:\ content\textbar{}string\ } : zhuyin characters
  \end{itemize}
\item
  named parameters:

  \begin{itemize}
  \tightlist
  \item
    \texttt{\ scale:\ number\ =\ 0.7\ } : font size scale of
    \texttt{\ ruby\ } , default \texttt{\ 0.7\ }
  \item
    \texttt{\ gutter:\ length\ =\ 0.3em\ } : spacing between
    \texttt{\ doc\ } and \texttt{\ ruby\ } , default \texttt{\ 0.3em\ }
  \item
    \texttt{\ delimiter:\ string\textbar{}none\ =\ none\ } : if not
    none, use this character to split \texttt{\ doc\ } and
    \texttt{\ ruby\ } into parts
  \item
    \texttt{\ spacing:\ length\textbar{}none\ =\ none\ } : spacing
    between each parts
  \end{itemize}
\end{itemize}

See example bellow.

\subsection{Example}\label{example}

\begin{Shaded}
\begin{Highlighting}[]
\NormalTok{汉(\#pinyin[ha4n])语(\#pinyin[yu3])拼(\#pinyin[pi1n])音(\#pinyin[yi1n])。}

\NormalTok{\#let per{-}char(f) = [\#f(delimiter: "|")[汉|语|拼|音][ha4n|yu3|pi1n|yi1n]]}
\NormalTok{\#let per{-}word(f) = [\#f(delimiter: "|")[汉语|拼音][ha4nyu3|pi1nyi1n]]}
\NormalTok{\#let all{-}in{-}one(f) = [\#f[汉语拼音][ha4nyu3pi1nyi1n]]}
\NormalTok{\#let example(f) = (per{-}char(f), per{-}word(f), all{-}in{-}one(f))}

\NormalTok{// argument of scale and spacing}
\NormalTok{\#let arguments = ((0.5, none), (0.7, none), (0.7, 0.1em), (1.0, none), (1.0, 0.2em))}

\NormalTok{\#table(}
\NormalTok{  columns: (auto, auto, auto, auto),}
\NormalTok{  align: (center + horizon, center, center, center),}
\NormalTok{  [arguments], [per char], [per word], [all in one],}
\NormalTok{  ..arguments.map(((scale, spacing)) =\textgreater{} (}
\NormalTok{    text(size: 0.7em)[\#scale,\#repr(spacing)], }
\NormalTok{    ..example(zhuyin.with(scale: scale, spacing: spacing))}
\NormalTok{  )).flatten(),}
\NormalTok{)}
\end{Highlighting}
\end{Shaded}

\pandocbounded{\includegraphics[keepaspectratio]{https://raw.githubusercontent.com/7sDream/typst-easy-pinyin/master/example.png?raw=true}}

\subsection{LICENSE}\label{license}

MIT, see License file.

\subsubsection{How to add}\label{how-to-add}

Copy this into your project and use the import as
\texttt{\ easy-pinyin\ }

\begin{verbatim}
#import "@preview/easy-pinyin:0.1.0"
\end{verbatim}

\includesvg[width=0.16667in,height=0.16667in]{/assets/icons/16-copy.svg}

Check the docs for
\href{https://typst.app/docs/reference/scripting/\#packages}{more
information on how to import packages} .

\subsubsection{About}\label{about}

\begin{description}
\tightlist
\item[Author s :]
7sDream \& Other open-source contributors
\item[License:]
MIT
\item[Current version:]
0.1.0
\item[Last updated:]
July 6, 2023
\item[First released:]
July 6, 2023
\item[Archive size:]
2.43 kB
\href{https://packages.typst.org/preview/easy-pinyin-0.1.0.tar.gz}{\pandocbounded{\includesvg[keepaspectratio]{/assets/icons/16-download.svg}}}
\item[Repository:]
\href{https://github.com/7sDream/typst-easy-pinyin}{GitHub}
\end{description}

\subsubsection{Where to report issues?}\label{where-to-report-issues}

This package is a project of 7sDream and Other open-source contributors
. Report issues on
\href{https://github.com/7sDream/typst-easy-pinyin}{their repository} .
You can also try to ask for help with this package on the
\href{https://forum.typst.app}{Forum} .

Please report this package to the Typst team using the
\href{https://typst.app/contact}{contact form} if you believe it is a
safety hazard or infringes upon your rights.

\phantomsection\label{versions}
\subsubsection{Version history}\label{version-history}

\begin{longtable}[]{@{}ll@{}}
\toprule\noalign{}
Version & Release Date \\
\midrule\noalign{}
\endhead
\bottomrule\noalign{}
\endlastfoot
0.1.0 & July 6, 2023 \\
\end{longtable}

Typst GmbH did not create this package and cannot guarantee correct
functionality of this package or compatibility with any version of the
Typst compiler or app.
