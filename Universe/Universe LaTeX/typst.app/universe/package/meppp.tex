\title{typst.app/universe/package/meppp}

\phantomsection\label{banner}
\phantomsection\label{template-thumbnail}
\pandocbounded{\includegraphics[keepaspectratio]{https://packages.typst.org/preview/thumbnails/meppp-0.2.1-small.webp}}

\section{meppp}\label{meppp}

{ 0.2.1 }

Template for modern physics experiment reports at the Physics School of
PKU.

\href{/app?template=meppp&version=0.2.1}{Create project in app}

\phantomsection\label{readme}
A simple template for modern physics experiments (MPE) courses at the
Physics School of PKU.

\subsection{meppp-lab-report}\label{meppp-lab-report}

The recommended report format of MPE course. Default arguments are shown
as below:

\begin{Shaded}
\begin{Highlighting}[]
\NormalTok{\#import "@preview/meppp:0.2.1": *}

\NormalTok{\#let meppp{-}lab{-}report(}
\NormalTok{  title: "",}
\NormalTok{  author: "",}
\NormalTok{  info: [],}
\NormalTok{  abstract: [],}
\NormalTok{  keywords: (),}
\NormalTok{  author{-}footnote: [],}
\NormalTok{  heading{-}numbering{-}array: ("I" ,"A", "1", "a"),}
\NormalTok{  heading{-}suffix: ". ",}
\NormalTok{  doc,}
\NormalTok{) = ...}
\end{Highlighting}
\end{Shaded}

\begin{itemize}
\tightlist
\item
  \texttt{\ title\ } is the title of the report.
\item
  \texttt{\ author\ } is the name of the author.
\item
  \texttt{\ info\ } is a line (or lines) of brief information of author
  and the report (e.g. student ID, school, experiment date…)
\item
  \texttt{\ abstract\ } is the abstract of the report, not shown when it
  is empty.
\item
  \texttt{\ keywords\ } are keywords of the report, only shown when the
  abstract is shown.
\item
  \texttt{\ author-footnote\ } is the phone number or the e-mail of the
  author, shown in the footnote.
\item
  \texttt{\ heading-numbering-array\ } is the heading numbering of each
  level. Only shows the numbering of the deepest level.
\item
  \texttt{\ heading-suffix\ } is the suffix of headings
\end{itemize}

It is recommended to use \texttt{\ \#show\ } to use the template:

\begin{Shaded}
\begin{Highlighting}[]
\NormalTok{\#show: meppp{-}lab{-}report.with(}
\NormalTok{    title: [Test title],}
\NormalTok{    ..args}
\NormalTok{)}
\NormalTok{...your report below.}
\end{Highlighting}
\end{Shaded}

\subsection{meppp-tl-table}\label{meppp-tl-table}

Modify your input \texttt{\ table\ } to a three-lined table (AIP style),
returned as a \texttt{\ figure\ } . Double-lines above and below the
table, and a single line below the header.

\begin{Shaded}
\begin{Highlighting}[]
\NormalTok{\#let meppp{-}tl{-}table(}
\NormalTok{  caption: none,}
\NormalTok{  supplement: auto,}
\NormalTok{  stroke: 0.5pt,}
\NormalTok{  tbl}
\NormalTok{) = ...}
\end{Highlighting}
\end{Shaded}

\begin{itemize}
\tightlist
\item
  \texttt{\ caption\ } is the caption above the table, center-aligned
\item
  \texttt{\ supplement\ } is same as the supplement in the figure.
\item
  \texttt{\ stroke\ } is the stroke used in the three lines (maybe five
  lines).
\item
  \texttt{\ tbl\ } is the input table, which must contains a
  \texttt{\ table.header\ }
\end{itemize}

Example:

\begin{Shaded}
\begin{Highlighting}[]
\NormalTok{\#meppp{-}tl{-}table(}
\NormalTok{  table(}
\NormalTok{    columns: 4,}
\NormalTok{    rows: 2,}
\NormalTok{    table.header([Item1], [Item2], [Item3], [Item4]),}
\NormalTok{    [Data1], [Data2], [Data3], [Data4],}
\NormalTok{  )}
\NormalTok{)}
\end{Highlighting}
\end{Shaded}

\subsection{subfigure}\label{subfigure}

Counts subfigures and displays in the figure, mostly used when inserting
multiple images.

\begin{Shaded}
\begin{Highlighting}[]
\NormalTok{\#let subfigure(}
\NormalTok{  body,}
\NormalTok{  caption: none,}
\NormalTok{  numbering: "(a)",}
\NormalTok{  inside: true,}
\NormalTok{  dx: 10pt,}
\NormalTok{  dy: 10pt,}
\NormalTok{  boxargs: (fill: white, inset: 5pt),}
\NormalTok{  alignment: top + left,}
\NormalTok{) = ...}
\end{Highlighting}
\end{Shaded}

\subsection{pku-logo}\label{pku-logo}

The logo of PKU, returned as a \texttt{\ image\ }

\begin{Shaded}
\begin{Highlighting}[]
\NormalTok{\#let pku{-}logo(..args) = image("pkulogo.png", ..args)}
\end{Highlighting}
\end{Shaded}

Example:

\begin{Shaded}
\begin{Highlighting}[]
\NormalTok{\#pku{-}logo(width: 50\%)}
\NormalTok{\#pku{-}logo()}
\end{Highlighting}
\end{Shaded}

\href{/app?template=meppp&version=0.2.1}{Create project in app}

\subsubsection{How to use}\label{how-to-use}

Click the button above to create a new project using this template in
the Typst app.

You can also use the Typst CLI to start a new project on your computer
using this command:

\begin{verbatim}
typst init @preview/meppp:0.2.1
\end{verbatim}

\includesvg[width=0.16667in,height=0.16667in]{/assets/icons/16-copy.svg}

\subsubsection{About}\label{about}

\begin{description}
\tightlist
\item[Author :]
\href{https://github.com/CL4R3T}{CL4R3T}
\item[License:]
MIT
\item[Current version:]
0.2.1
\item[Last updated:]
September 22, 2024
\item[First released:]
May 8, 2024
\item[Archive size:]
103 kB
\href{https://packages.typst.org/preview/meppp-0.2.1.tar.gz}{\pandocbounded{\includesvg[keepaspectratio]{/assets/icons/16-download.svg}}}
\item[Repository:]
\href{https://github.com/pku-typst/meppp}{GitHub}
\item[Categor y :]
\begin{itemize}
\tightlist
\item[]
\item
  \pandocbounded{\includesvg[keepaspectratio]{/assets/icons/16-speak.svg}}
  \href{https://typst.app/universe/search/?category=report}{Report}
\end{itemize}
\end{description}

\subsubsection{Where to report issues?}\label{where-to-report-issues}

This template is a project of CL4R3T . Report issues on
\href{https://github.com/pku-typst/meppp}{their repository} . You can
also try to ask for help with this template on the
\href{https://forum.typst.app}{Forum} .

Please report this template to the Typst team using the
\href{https://typst.app/contact}{contact form} if you believe it is a
safety hazard or infringes upon your rights.

\phantomsection\label{versions}
\subsubsection{Version history}\label{version-history}

\begin{longtable}[]{@{}ll@{}}
\toprule\noalign{}
Version & Release Date \\
\midrule\noalign{}
\endhead
\bottomrule\noalign{}
\endlastfoot
0.2.1 & September 22, 2024 \\
\href{https://typst.app/universe/package/meppp/0.2.0/}{0.2.0} &
September 14, 2024 \\
\href{https://typst.app/universe/package/meppp/0.1.0/}{0.1.0} & May 8,
2024 \\
\end{longtable}

Typst GmbH did not create this template and cannot guarantee correct
functionality of this template or compatibility with any version of the
Typst compiler or app.
