\title{typst.app/universe/package/koma-labeling}

\phantomsection\label{banner}
\section{koma-labeling}\label{koma-labeling}

{ 0.1.0 }

This package introduces a labeling feature to Typst, inspired by the
KOMA-Script\textquotesingle s labeling environment.

\phantomsection\label{readme}
Version 0.1.0

The koma-labeling package for Typst is inspired by the labeling
environment from the KOMA-Script bundle in LaTeX. It provides a
convenient way to create labeled lists with customizable label widths
and optional delimiters, making it perfect for creating structured
descriptions and lists in your Typst documents.

\subsection{Getting Started}\label{getting-started}

To get started with koma-labeling, simply import the package in your
Typst document and use the labeling environment to create your labeled
lists.

\begin{Shaded}
\begin{Highlighting}[]
\NormalTok{\#import "@preview/koma{-}labeling:0.1.0": labeling}

\NormalTok{\#labeling(}
\NormalTok{  (}
\NormalTok{    (lorem(1), lorem(10)),}
\NormalTok{    (lorem(2), lorem(20)),}
\NormalTok{    (lorem(3), lorem(30)),}
\NormalTok{  )}
\NormalTok{)}

\NormalTok{// or}

\NormalTok{\#labeling(}
\NormalTok{  (}
\NormalTok{    ([\#lorem(1)], [\#lorem(10)]),}
\NormalTok{    ([\#lorem(2)], [\#lorem(20)]),}
\NormalTok{    ([\#lorem(3)], [\#lorem(30)]),}
\NormalTok{  )}
\NormalTok{)}
\end{Highlighting}
\end{Shaded}

Output:

\pandocbounded{\includegraphics[keepaspectratio]{https://github.com/user-attachments/assets/bf382afe-f66d-4032-9055-f46c72a2e7dd}}

\textbf{Note:} Remember to terminate the list with a comma, even if only
one pair of items is passed.

\begin{Shaded}
\begin{Highlighting}[]
\NormalTok{\#import "@preview/koma{-}labeling:0.1.0": labeling}

\NormalTok{\#labeling(}
\NormalTok{  (}
\NormalTok{    (lorem(1), lorem(10)),  // Terminating the list with a comma is REQUIRED}
\NormalTok{  )}
\NormalTok{)}
\end{Highlighting}
\end{Shaded}

\subsection{Parameters}\label{parameters}

Although labeling is implemented using \texttt{\ tables\ } , its usage
is similar to \texttt{\ terms\ } , except that it lacks the
\texttt{\ tight\ } and \texttt{\ hanging-indent\ } parameters. If you
have any questions about the parameters for \texttt{\ labeling\ } , you
can refer to
\href{https://typst.app/docs/reference/model/terms/}{\texttt{\ terms\ }}
.

\begin{Shaded}
\begin{Highlighting}[]
\NormalTok{labeling(}
\NormalTok{  separator: content,}
\NormalTok{  indent: length,}
\NormalTok{  spacing: auto length}
\NormalTok{  pairs: ((content, content))}
\NormalTok{)}
\end{Highlighting}
\end{Shaded}

\subsubsection{separator}\label{separator}

The separator between the item and the description.

Default: \texttt{\ {[}:\#h(0.6em){]}\ }

\subsubsection{indent}\label{indent}

The indentation of each item.

Default: \texttt{\ 0pt\ }

\subsubsection{spacing}\label{spacing}

The spacing between the items of the term list.

Default: \texttt{\ auto\ }

\subsubsection{pairs}\label{pairs}

An array of \texttt{\ (item,\ description)\ } pairs.

Example:

\begin{Shaded}
\begin{Highlighting}[]
\NormalTok{\#labeling(}
\NormalTok{  (}
\NormalTok{    ([key 1],[description 1]),}
\NormalTok{    ([keyword 2],[description 2]),}
\NormalTok{  )}
\NormalTok{)}
\end{Highlighting}
\end{Shaded}

\subsection{Additional Documentation and
Acknowledgments}\label{additional-documentation-and-acknowledgments}

For more information on the koma-labeling package and its features, you
can refer to the following resources:

\begin{itemize}
\tightlist
\item
  Typst Documentation: \href{https://typst.app/docs}{Typst
  Documentation}
\item
  KOMA-Script Documentation:
  \href{https://ctan.org/pkg/koma-script}{KOMA-Script Documentation}
\end{itemize}

\subsubsection{How to add}\label{how-to-add}

Copy this into your project and use the import as
\texttt{\ koma-labeling\ }

\begin{verbatim}
#import "@preview/koma-labeling:0.1.0"
\end{verbatim}

\includesvg[width=0.16667in,height=0.16667in]{/assets/icons/16-copy.svg}

Check the docs for
\href{https://typst.app/docs/reference/scripting/\#packages}{more
information on how to import packages} .

\subsubsection{About}\label{about}

\begin{description}
\tightlist
\item[Author :]
Laniakea Kamasylvia
\item[License:]
MIT
\item[Current version:]
0.1.0
\item[Last updated:]
October 28, 2024
\item[First released:]
October 28, 2024
\item[Minimum Typst version:]
0.11.0
\item[Archive size:]
2.54 kB
\href{https://packages.typst.org/preview/koma-labeling-0.1.0.tar.gz}{\pandocbounded{\includesvg[keepaspectratio]{/assets/icons/16-download.svg}}}
\item[Categor y :]
\begin{itemize}
\tightlist
\item[]
\item
  \pandocbounded{\includesvg[keepaspectratio]{/assets/icons/16-code.svg}}
  \href{https://typst.app/universe/search/?category=scripting}{Scripting}
\end{itemize}
\end{description}

\subsubsection{Where to report issues?}\label{where-to-report-issues}

This package is a project of Laniakea Kamasylvia . You can also try to
ask for help with this package on the
\href{https://forum.typst.app}{Forum} .

Please report this package to the Typst team using the
\href{https://typst.app/contact}{contact form} if you believe it is a
safety hazard or infringes upon your rights.

\phantomsection\label{versions}
\subsubsection{Version history}\label{version-history}

\begin{longtable}[]{@{}ll@{}}
\toprule\noalign{}
Version & Release Date \\
\midrule\noalign{}
\endhead
\bottomrule\noalign{}
\endlastfoot
0.1.0 & October 28, 2024 \\
\end{longtable}

Typst GmbH did not create this package and cannot guarantee correct
functionality of this package or compatibility with any version of the
Typst compiler or app.
