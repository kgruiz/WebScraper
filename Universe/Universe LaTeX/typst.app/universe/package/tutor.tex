\title{typst.app/universe/package/tutor}

\phantomsection\label{banner}
\section{tutor}\label{tutor}

{ 0.7.0 }

Utilities to create exams.

\phantomsection\label{readme}
Utilities to write exams and exercises with integrated solutions. Set
the variable \texttt{\ \#(cfg.sol\ =\ true)\ } to display the solutions
of a document.

Currently the following features are supported:

\begin{itemize}
\tightlist
\item
  Automatic total point calculation through the \texttt{\ \#points()\ }
  and \texttt{\ \#totalpoints()\ } functions.
\item
  Checkboxes that are either blank or show the solution state using eg.
  \texttt{\ \#checkbox(cfg,\ true)\ } .
\item
  Display blank lines allowing students to write their answer using eg.
  \texttt{\ \#lines(cfg,\ 3)\ } .
\item
  A proposition for a project structure allowing self-contained
  exercises and a mechanism to show or hide the solutions of an
  exercise.
\end{itemize}

\subsection{Usage}\label{usage}

\subsubsection{Minimal Example}\label{minimal-example}

\begin{Shaded}
\begin{Highlighting}[]
\NormalTok{\#import "@local/tutor:0.6.1": points, totalpoints, lines, checkbox, default{-}config}

\NormalTok{\#let cfg = default{-}config()}
\NormalTok{// enable solution mode}
\NormalTok{\#(cfg.sol = true)}

\NormalTok{// display 3 lines (for hand written answer)}
\NormalTok{\#lines(cfg, 3)}
\NormalTok{// checkbox for multiple choice (indicates correct state)}
\NormalTok{\#checkbox(cfg, true)}

\NormalTok{// show achievable points}
\NormalTok{Max. points: \#points(2)}
\NormalTok{Max. points: \#points(3)}
\NormalTok{// show sum of all total achievable points (will show 5)}
\NormalTok{Total points: \#totalpoints(cfg)}
\end{Highlighting}
\end{Shaded}

\subsubsection{Practical Example}\label{practical-example}

Check
\href{https://github.com/rangerjo/tutor/tree/main/example}{example} for
a more practical example.

\texttt{\ tutor\ } is best used with the following directory and file
structure:

\begin{verbatim}
├── main.typ
├── src
│   ├── ex1
│   │   └── ex.typ
│   └── ex2
│       └── ex.typ
└── tutor.toml
\end{verbatim}

Every directory in \texttt{\ src\ } holds one self-contained exercise.
The exercises can be imported into \texttt{\ main.typ\ } :

\begin{Shaded}
\begin{Highlighting}[]
\NormalTok{\#import "@local/tutor:0.6.1": totalpoints, lines, default{-}config}

\NormalTok{\#import "src/ex1/ex.typ" as ex1}
\NormalTok{\#import "src/ex2/ex.typ" as ex2}


\NormalTok{\#let cfg = default{-}config()}
\NormalTok{\#ex1.exercise(cfg)}
\NormalTok{\#ex2.exercise(cfg)}
\end{Highlighting}
\end{Shaded}

Supporting self-contained exercises is one of \texttt{\ tutor\ } s
primary design goals. Each exercise lives within a folder and can easily
be copied or referenced in a new document.

An exercise is a folder that contains an \texttt{\ ex.typ\ } file along
with any other assets (images, source code aso). The following exercise
shows a practical usage of the \texttt{\ \#checkbox()\ } and
\texttt{\ \#points()\ } functions.

\texttt{\ src/ex1/ex.typ\ }

\begin{Shaded}
\begin{Highlighting}[]
\NormalTok{\#import "@local/tutor:0.6.1": points, checkbox}

\NormalTok{\#let exercise(cfg) = [}
\NormalTok{\#heading(level:cfg.lvl, [Abbreviation FHIR (\#points(1) point)])}

\NormalTok{What does FHIR stand for?}

\NormalTok{\#set list(marker: none)}
\NormalTok{{-} \#checkbox(cfg, false)  Finally He Is Real}
\NormalTok{{-} \#checkbox(cfg, true)   Fast Health Interoperability Resources}
\NormalTok{{-} \#checkbox(cfg, false)   First Health Inactivation Restriction}

\NormalTok{\#if cfg.sol \{}
\NormalTok{  [ Further explanation: FHIR is the new standard developed by HL7. ]}
\NormalTok{\}}
\NormalTok{]}
\end{Highlighting}
\end{Shaded}

Finally this second example shows the \texttt{\ \#lines()\ } function.
\texttt{\ src/ex2/ex.typ\ }

\begin{Shaded}
\begin{Highlighting}[]
\NormalTok{\#import "@local/tutor:0.6.1": points, lines }

\NormalTok{\#let exercise(cfg) = [}
\NormalTok{\#heading(level:cfg.lvl, [FHIR vs HL7v2 (\#points(4.5) points)])}

\NormalTok{List two differences between HL7v2 and FHIR:}

\NormalTok{+ \#if cfg.sol \{ [ HL7v2 uses a non{-}standard line format, where as FHIR uses XML or JSON] \} else \{ [ \#lines(cfg, 3) ] \}}
\NormalTok{+ \#if cfg.sol \{ [ FHIR specifies various resources that can be queried, where as HL7v2 has a number of fixed fields that are either filled in or not]\} else \{ [ \#lines(cfg, 3) ] \}}
\NormalTok{]}
\end{Highlighting}
\end{Shaded}

This would then give the following output in question mode (
\texttt{\ \#(cfg.sol=false)\ } ) and in solution mode (
\texttt{\ \#(cfg.sol=true)\ } ):
\pandocbounded{\includegraphics[keepaspectratio]{https://raw.githubusercontent.com/rangerjo/tutor/main/imgs/example_mod.png}}

\subsection{Utilities}\label{utilities}

\subsubsection{lines}\label{lines}

\texttt{\ \#lines(cfg,\ count)\ } prints \texttt{\ count\ } lines for
students to write their answer.

Configuration:

\begin{Shaded}
\begin{Highlighting}[]
\NormalTok{// Vertical line spacing between rows. }
\NormalTok{\#(cfg.utils.lines.spacing = 8mm)}
\end{Highlighting}
\end{Shaded}

\subsubsection{grid}\label{grid}

\texttt{\ \#grid(cfg,\ width,\ height)\ } prints a grid for students to
write their answer.

Configuration:

\begin{Shaded}
\begin{Highlighting}[]
\NormalTok{// Grid spacing. }
\NormalTok{\#(cfg.utils.grid.spacing = 4mm)}
\end{Highlighting}
\end{Shaded}

\subsubsection{checkbox}\label{checkbox}

\texttt{\ \#checkbox(cfg,\ answer)\ } shows a checkbox. In solution
mode, the checkbox is shown filled out.

Configuration:

\begin{Shaded}
\begin{Highlighting}[]
\NormalTok{// Symbol to show if answer is true }
\NormalTok{\#(cfg.utils.checkbox.sym\_true = "☒")}
\NormalTok{// Symbol to show if answer is false}
\NormalTok{\#(cfg.utils.checkbox.sym\_false = "☐")}
\NormalTok{// Symbol to show in question mode}
\NormalTok{\#(cfg.utils.checkbox.sym\_question = "☐")}
\end{Highlighting}
\end{Shaded}

\subsubsection{points}\label{points}

\texttt{\ \#points(cfg,\ num)\ } displays the given \texttt{\ num\ }
while adding its value to the total points counter.

Configuration: none

\subsubsection{totalpoints}\label{totalpoints}

\texttt{\ \#totalpoints(cfg)\ } shows the final value of the total
points counter.

Configuration:

\begin{Shaded}
\begin{Highlighting}[]
\NormalTok{// If points() is used in the outline, totalpoints value becomes doubled.}
\NormalTok{// By setting outline to true, totalpoints gets divided by half.}
\NormalTok{\#(cfg.utils.totalpoints.outline = false)}
\end{Highlighting}
\end{Shaded}

\subsection{Modes}\label{modes}

\texttt{\ tutor\ } comes with a solution and a test mode.

\subsubsection{solution mode}\label{solution-mode}

Solution mode controls wheter solutions are shown or not. This mode
controls eg. the utility \texttt{\ \#checkbox(cfg,\ answer)\ } .

\begin{enumerate}
\tightlist
\item
  \texttt{\ (cfg.sol\ =\ false)\ } : Solutions are hidden. This is used
  for the actual exam handed out to students.
\item
  \texttt{\ (cfg.sol\ =\ true)\ } : Solutions are shown. This is used to
  create the exam solutions.
\end{enumerate}

You can also use the following helper functions:

\begin{itemize}
\tightlist
\item
  \texttt{\ if-sol(cfg,{[}Content\ only\ shown\ in\ solution\ mode.{]})\ }
\item
  \texttt{\ if-sol-else(cfg,{[}Content\ only\ shown\ in\ solution\ mode.{]},\ {[}Content\ only\ shown\ in\ exam\ mode.{]})\ }
\end{itemize}

\subsubsection{test mode}\label{test-mode}

Test mode can be used to show or hide additional information. In test
mode, one might want

\begin{enumerate}
\item
  \texttt{\ (cfg.test\ =\ true)\ } : Test information are shown. Use
  this eg. to display \texttt{\ \#points(4)\ } . This is used in case
  the document is used as an exam/test.
\item
  \texttt{\ (cfg.test\ =\ false)\ } : Test information are hidden. This
  is used in case the document is used as an excerise.
\end{enumerate}

The following would show the points only in test mode.

\begin{Shaded}
\begin{Highlighting}[]
\NormalTok{\#if cfg.test \{}
\NormalTok{  \#points(4)}
\NormalTok{\}}
\end{Highlighting}
\end{Shaded}

Or you can use the following helper functions:

\begin{itemize}
\tightlist
\item
  \texttt{\ if-test(cfg,{[}Content\ only\ shown\ in\ test\ mode.{]})\ }
\item
  \texttt{\ if-test-else(cfg,{[}Content\ only\ shown\ in\ test\ mode.{]},\ {[}Content\ only\ shown\ in\ exercise\ mode.{]})\ }
\end{itemize}

\subsection{Configuration}\label{configuration}

\texttt{\ tutor\ } is designed to create exams and solutions with one
single document source. Furthermore, the individual utilities provided
by \texttt{\ tutor\ } can be configured. This can be done in one of
three ways:

\begin{enumerate}
\tightlist
\item
  Use the \texttt{\ \#default-config()\ } function and patch your
  configuration. The following example would configure the solution mode
  and basic line spacings to 8 millimeters:
\end{enumerate}

\begin{Shaded}
\begin{Highlighting}[]
\NormalTok{\#let cfg = default{-}config()}
\NormalTok{\#(cfg.sol = false)}
\NormalTok{\#(cfg.utils.lines.spacing = 8mm)}
\end{Highlighting}
\end{Shaded}

\begin{enumerate}
\setcounter{enumi}{1}
\tightlist
\item
  Use an external file to hold the configurations in your prefered
  format. See
  \href{https://github.com/rangerjo/tutor/blob/main/example/tutor.toml}{tutor.toml}
  for a configuration in TOML. Load the configuration into your main
  document using
\end{enumerate}

\begin{Shaded}
\begin{Highlighting}[]
\NormalTok{\#let cfg = toml("tutor.toml")}
\end{Highlighting}
\end{Shaded}

\begin{enumerate}
\setcounter{enumi}{2}
\tightlist
\item
  Use typst’s input feature added with compiler version 0.11.0. Add
  the following snippet to load the configuration, then overwrite it
  from the CLI like this:
  \texttt{\ typst\ compile\ -\/-input\ tutor\_sol=true\ main.typ\ }
\end{enumerate}

\begin{Shaded}
\begin{Highlighting}[]
\NormalTok{\#let cfg = toml("tutor.toml")}

\NormalTok{\#if sys.inputs.tutor\_sol == "true" \{}
\NormalTok{  (cfg.sol = true)}
\NormalTok{\} else if sys.inputs.tutor\_sol == "false" \{}
\NormalTok{  (cfg.sol = false)}
\NormalTok{\}}
\end{Highlighting}
\end{Shaded}

\subsubsection{How to add}\label{how-to-add}

Copy this into your project and use the import as \texttt{\ tutor\ }

\begin{verbatim}
#import "@preview/tutor:0.7.0"
\end{verbatim}

\includesvg[width=0.16667in,height=0.16667in]{/assets/icons/16-copy.svg}

Check the docs for
\href{https://typst.app/docs/reference/scripting/\#packages}{more
information on how to import packages} .

\subsubsection{About}\label{about}

\begin{description}
\tightlist
\item[Author :]
Jonas Amstutz
\item[License:]
MIT
\item[Current version:]
0.7.0
\item[Last updated:]
October 9, 2024
\item[First released:]
October 17, 2023
\item[Minimum Typst version:]
0.11.0
\item[Archive size:]
4.82 kB
\href{https://packages.typst.org/preview/tutor-0.7.0.tar.gz}{\pandocbounded{\includesvg[keepaspectratio]{/assets/icons/16-download.svg}}}
\item[Repository:]
\href{https://github.com/rangerjo/tutor}{GitHub}
\item[Discipline :]
\begin{itemize}
\tightlist
\item[]
\item
  \href{https://typst.app/universe/search/?discipline=education}{Education}
\end{itemize}
\item[Categor y :]
\begin{itemize}
\tightlist
\item[]
\item
  \pandocbounded{\includesvg[keepaspectratio]{/assets/icons/16-package.svg}}
  \href{https://typst.app/universe/search/?category=components}{Components}
\end{itemize}
\end{description}

\subsubsection{Where to report issues?}\label{where-to-report-issues}

This package is a project of Jonas Amstutz . Report issues on
\href{https://github.com/rangerjo/tutor}{their repository} . You can
also try to ask for help with this package on the
\href{https://forum.typst.app}{Forum} .

Please report this package to the Typst team using the
\href{https://typst.app/contact}{contact form} if you believe it is a
safety hazard or infringes upon your rights.

\phantomsection\label{versions}
\subsubsection{Version history}\label{version-history}

\begin{longtable}[]{@{}ll@{}}
\toprule\noalign{}
Version & Release Date \\
\midrule\noalign{}
\endhead
\bottomrule\noalign{}
\endlastfoot
0.7.0 & October 9, 2024 \\
\href{https://typst.app/universe/package/tutor/0.6.1/}{0.6.1} & October
9, 2024 \\
\href{https://typst.app/universe/package/tutor/0.4.0/}{0.4.0} & March
19, 2024 \\
\href{https://typst.app/universe/package/tutor/0.3.0/}{0.3.0} & October
17, 2023 \\
\end{longtable}

Typst GmbH did not create this package and cannot guarantee correct
functionality of this package or compatibility with any version of the
Typst compiler or app.
