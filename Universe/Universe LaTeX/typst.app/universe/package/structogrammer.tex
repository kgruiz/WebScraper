\title{typst.app/universe/package/structogrammer}

\phantomsection\label{banner}
\section{structogrammer}\label{structogrammer}

{ 0.1.1 }

Draw Nassi-Shneiderman diagrams (or structograms)

\phantomsection\label{readme}
Draw Nassi-Shneiderman diagrams, also called structograms, in Typst.

\subsection{Basic Usage}\label{basic-usage}

Import with:

\begin{Shaded}
\begin{Highlighting}[]
\NormalTok{\#import "@preview/structogrammer:0.1.0": structogram}
\end{Highlighting}
\end{Shaded}

You can then draw structograms, like so:

\begin{Shaded}
\begin{Highlighting}[]
\NormalTok{\#structogram(}
\NormalTok{  width: 30em,}
\NormalTok{  title: "merge\_sort(list)",}
\NormalTok{  (}
\NormalTok{    (If: "list empty", Then: (Break: "exit (return list)")),}
\NormalTok{    "left = []",}
\NormalTok{    "right = []",}
\NormalTok{    (For: "element with index i", In: "list", Do: (}
\NormalTok{      (If: "i \textless{} list.length / 2", Then: (}
\NormalTok{        "left.add(element)"}
\NormalTok{      ), Else: (}
\NormalTok{        "right.add(element)"}
\NormalTok{      ))}
\NormalTok{    )),}
\NormalTok{    "left = merge\_sort(left)",}
\NormalTok{    "right = merge\_sort(right)",}
\NormalTok{    (Break: "return with merge(left, right)")}
\NormalTok{  )}
\NormalTok{)}
\end{Highlighting}
\end{Shaded}

which yields:\\
\pandocbounded{\includesvg[keepaspectratio]{https://raw.githubusercontent.com/genericusername3/structogrammer/master/examples/merge-sort.svg}}

If \texttt{\ text.lang\ } is set to another language, this package will
try to match inserted text to it. Currently, only \texttt{\ "en"\ } and
\texttt{\ "de"\ } are supported

\subsection{Advanced usage}\label{advanced-usage}

\texttt{\ structogram()\ } takes the following named arguments:

\begin{itemize}
\tightlist
\item
  \texttt{\ columns\ } : If you already allocated wide and narrow
  columns, \texttt{\ to-elements\ } can use them. Useful for sub-specs,
  as you’d usually generate allocations first and then do another
  recursive pass to fill them.\\
  The default, \texttt{\ auto\ } does exactly this on the highest
  recursion level.
\item
  \texttt{\ stroke\ } : The stroke to use between cells, or for control
  blocks. Note: to avoid duplicate strokes, every cell only adds strokes
  to its top and left side. Put the resulting cells in a container with
  bottom and right strokes for a finished diagram. See
  \texttt{\ structogram()\ } .\\
  Default: \texttt{\ 0.5pt\ +\ black\ }
\item
  \texttt{\ inset\ } : How much to pad each cell.\\
  Default: \texttt{\ 0.5em\ }
\item
  \texttt{\ segment-height\ } : How high each row should be.\\
  Default: \texttt{\ 2em\ }
\item
  \texttt{\ narrow-width\ } : The width that narrow columns will be.
  Needed for diagonals in conditional blocks.\\
  Default: 1em
\end{itemize}

A \texttt{\ spec\ } (the positional argument to
\texttt{\ structogram()\ } ) can be one of the following:

\begin{itemize}
\tightlist
\item
  \texttt{\ none\ } or an emtpy
  \href{https://typst.app/docs/reference/foundations/array/}{\texttt{\ array\ }}
  \texttt{\ ()\ } : An empty cell, taking up at least a narrow column
\item
  a
  \href{https://typst.app/docs/reference/foundations/str/}{\texttt{\ string\ }}
  or
  \href{https://typst.app/docs/reference/foundations/content/}{\texttt{\ content\ }}
  : A cell containing that string or content, taking up at least a wide
  column
\item
  A
  \href{https://typst.app/docs/reference/foundations/dictionary/}{\texttt{\ dictionary\ }}
  : Control block (
  \href{https://github.com/typst/packages/raw/main/packages/preview/structogrammer/0.1.1/\#control-blocks}{see
  below} )
\item
  An
  \href{https://typst.app/docs/reference/foundations/array/}{\texttt{\ array\ }}
  of specs: The cells that each element produced, stacked on top of each
  other. Wide columns are aligned to wide columns of other element specs
  and narrow columns consumed as needed.
\end{itemize}

\subsubsection{Control blocks}\label{control-blocks}

Specs can contain the following control blocks, as dictionaries:

\paragraph{\texorpdfstring{1. \texttt{\ If\ } / \texttt{\ Then\ } /
\texttt{\ Else\ } :}{1.  If  /  Then  /  Else  :}}\label{if-then-else}

A conditional with the following keys:

\begin{itemize}
\tightlist
\item
  \texttt{\ If\ } : The condition on which to branch
\item
  \texttt{\ Then\ } : A diagram spec for the “yes�-branch
\item
  \texttt{\ Else\ } : A diagram spec for the “no�-branch
\end{itemize}

\texttt{\ Then\ } and \texttt{\ Else\ } are both optional, but at least
one must be present

Examples:

\begin{itemize}
\tightlist
\item
\item
  \texttt{\ (If:\ "debug\ mode",\ Then:\ ("print\ debug\ message"))\ }

  \pandocbounded{\includesvg[keepaspectratio]{https://raw.githubusercontent.com/genericusername3/structogrammer/master/examples/if-then.svg}}
\item
  \texttt{\ (If:\ "x\ \textgreater{}\ 5",\ Then:\ ("x\ =\ x\ -\ 1",\ "print\ x"),\ Else:\ "print\ x")\ }

  \pandocbounded{\includesvg[keepaspectratio]{https://raw.githubusercontent.com/genericusername3/structogrammer/master/examples/if-then-else.svg}}
\end{itemize}

Columns: Takes up columns according to its contents next to one another,
inserting narrow columns for empty branches

\paragraph{\texorpdfstring{2. \texttt{\ For\ } / \texttt{\ Do\ } ,
\texttt{\ For\ } / \texttt{\ To\ } / \texttt{\ Do\ } , \texttt{\ For\ }
/ \texttt{\ In\ } / \texttt{\ Do\ } , \texttt{\ While\ } /
\texttt{\ Do\ } , \texttt{\ Do\ } / \texttt{\ While\ }
:}{2.  For  /  Do  ,  For  /  To  /  Do  ,  For  /  In  /  Do  ,  While  /  Do  ,  Do  /  While  :}}\label{for-do-for-to-do-for-in-do-while-do-do-while}

A loop, with the loop control either at the top or bottom.

\begin{itemize}
\tightlist
\item
  \texttt{\ For\ } / \texttt{\ Do\ } formats the control as “For
  \$For�,
\item
  \texttt{\ For\ } / \texttt{\ To\ } / \texttt{\ Do\ } as “For \$For
  to \$To�,
\item
  \texttt{\ For\ } / \texttt{\ In\ } / \texttt{\ Do\ } as “For each
  \$For in \$In�,
\item
  \texttt{\ While\ } / \texttt{\ Do\ } and \texttt{\ Do\ } /
  \texttt{\ While\ } as “While \$While�.
\end{itemize}

Order of specified keys matters.

Examples:

\begin{itemize}
\tightlist
\item
\item
  \texttt{\ (While:\ "true",\ Do:\ "print\ \textbackslash{}"endless\ loop\textbackslash{}"")\ }

  \pandocbounded{\includesvg[keepaspectratio]{https://raw.githubusercontent.com/genericusername3/structogrammer/master/examples/while-do.svg}}
\item
  \texttt{\ (Do:\ "print\ \textbackslash{}"endless\ loop\textbackslash{}"",\ While:\ "true")\ }

  \pandocbounded{\includesvg[keepaspectratio]{https://raw.githubusercontent.com/genericusername3/structogrammer/master/examples/do-while.svg}}
\item
  \texttt{\ (For:\ "item",\ In:\ "Container",\ Do:\ "print\ item.name")\ }

  \pandocbounded{\includesvg[keepaspectratio]{https://raw.githubusercontent.com/genericusername3/structogrammer/master/examples/for-in.svg}}
\end{itemize}

Columns: Inserts a narrow column left to its content.

\paragraph{\texorpdfstring{3. Method call ( \texttt{\ Call\ }
)}{3. Method call (  Call  )}}\label{method-call-call}

A block indicating that a subroutine is executed here. Only accepts the
key \texttt{\ Call\ } , which is the string name

Example:

\begin{itemize}
\tightlist
\item
\item
  \texttt{\ (Call:\ "func()")\ }

  \pandocbounded{\includesvg[keepaspectratio]{https://raw.githubusercontent.com/genericusername3/structogrammer/master/examples/call.svg}}
\end{itemize}

Columns: One wide column

\paragraph{\texorpdfstring{4. Break/Return ( \texttt{\ Break\ }
)}{4. Break/Return (  Break  )}}\label{breakreturn-break}

A block indicating that a subroutine is executed here. Only accepts the
key \texttt{\ Break\ } , which is the target to break to

Examples:

\begin{itemize}
\tightlist
\item
\item
  \texttt{\ (Break:\ "")\ }

  \pandocbounded{\includesvg[keepaspectratio]{https://raw.githubusercontent.com/genericusername3/structogrammer/master/examples/break.svg}}
\item
  \texttt{\ (Break:\ "to\ enclosing\ loop")\ }

  \pandocbounded{\includesvg[keepaspectratio]{https://raw.githubusercontent.com/genericusername3/structogrammer/master/examples/break-to.svg}}
\end{itemize}

Columns: One wide column

\subsubsection{How to add}\label{how-to-add}

Copy this into your project and use the import as
\texttt{\ structogrammer\ }

\begin{verbatim}
#import "@preview/structogrammer:0.1.1"
\end{verbatim}

\includesvg[width=0.16667in,height=0.16667in]{/assets/icons/16-copy.svg}

Check the docs for
\href{https://typst.app/docs/reference/scripting/\#packages}{more
information on how to import packages} .

\subsubsection{About}\label{about}

\begin{description}
\tightlist
\item[Author :]
\href{https://cza.li}{Charlotte}
\item[License:]
MIT
\item[Current version:]
0.1.1
\item[Last updated:]
May 14, 2024
\item[First released:]
May 13, 2024
\item[Archive size:]
8.68 kB
\href{https://packages.typst.org/preview/structogrammer-0.1.1.tar.gz}{\pandocbounded{\includesvg[keepaspectratio]{/assets/icons/16-download.svg}}}
\end{description}

\subsubsection{Where to report issues?}\label{where-to-report-issues}

This package is a project of Charlotte . You can also try to ask for
help with this package on the \href{https://forum.typst.app}{Forum} .

Please report this package to the Typst team using the
\href{https://typst.app/contact}{contact form} if you believe it is a
safety hazard or infringes upon your rights.

\phantomsection\label{versions}
\subsubsection{Version history}\label{version-history}

\begin{longtable}[]{@{}ll@{}}
\toprule\noalign{}
Version & Release Date \\
\midrule\noalign{}
\endhead
\bottomrule\noalign{}
\endlastfoot
0.1.1 & May 14, 2024 \\
\href{https://typst.app/universe/package/structogrammer/0.1.0/}{0.1.0} &
May 13, 2024 \\
\end{longtable}

Typst GmbH did not create this package and cannot guarantee correct
functionality of this package or compatibility with any version of the
Typst compiler or app.
