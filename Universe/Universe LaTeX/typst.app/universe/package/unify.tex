\title{typst.app/universe/package/unify}

\phantomsection\label{banner}
\section{unify}\label{unify}

{ 0.7.0 }

Format numbers, units, and ranges correctly.

{ } Featured Package

\phantomsection\label{readme}
\texttt{\ unify\ } is a \href{https://github.com/typst/typst}{Typst}
package simplifying the typesetting of numbers, units, and ranges. It is
the equivalent to LaTeX’s \texttt{\ siunitx\ } , though not as mature.

\subsection{Overview}\label{overview}

\texttt{\ unify\ } allows flexible numbers and units, and still mostly
gets well typeset results.

\begin{Shaded}
\begin{Highlighting}[]
\NormalTok{\#import "@preview/unify:0.7.0": num,qty,numrange,qtyrange}

\NormalTok{$ num("{-}1.32865+{-}0.50273e{-}6") $}
\NormalTok{$ qty("1.3+1.2{-}0.3e3", "erg/cm\^{}2/s", space: "\#h(2mm)") $}
\NormalTok{$ numrange("1,1238e{-}2", "3,0868e5", thousandsep: "\textquotesingle{}") $}
\NormalTok{$ qtyrange("1e3", "2e3", "meter per second squared", per: "/", delimiter: "\textbackslash{}"to\textbackslash{}"") $}
\end{Highlighting}
\end{Shaded}

\includegraphics[width=3.125in,height=\textheight,keepaspectratio]{https://github.com/typst/packages/raw/main/packages/preview/unify/0.7.0/examples/overview.jpg}

Right now, physical, monetary, and binary units are supported. New
issues or pull requests for new units are welcome!

\subsection{Multilingual support}\label{multilingual-support}

The Unify package supports multiple languages. Currently, the supported
languages are English and Russian. The fallback is English. If you want
to add your language, you should add two files:
\texttt{\ prefixes-xx.csv\ } and \texttt{\ units-xx.csv\ } , and in the
\texttt{\ lib.typ\ } file you should fix the \texttt{\ lang-db\ } state
for your files.

\subsection{\texorpdfstring{\texttt{\ num\ }}{ num }}\label{num}

\texttt{\ num\ } uses string parsing in order to typeset numbers,
including separators between the thousands. They can have the following
form:

\begin{itemize}
\tightlist
\item
  \texttt{\ float\ } or \texttt{\ integer\ } number
\item
  either ( \texttt{\ \{\}\ } stands for a number)

  \begin{itemize}
  \tightlist
  \item
    symmetric uncertainties with \texttt{\ +-\{\}\ }
  \item
    asymmetric uncertainties with \texttt{\ +\{\}-\{\}\ }
  \end{itemize}
\item
  exponential notation \texttt{\ e\{\}\ }
\end{itemize}

Parentheses are automatically set as necessary. Use
\texttt{\ thousandsep\ } to change the separator between the thousands,
and \texttt{\ multiplier\ } to change the multiplication symbol between
the number and exponential.

\subsection{\texorpdfstring{\texttt{\ unit\ }}{ unit }}\label{unit}

\texttt{\ unit\ } takes the unit in words or in symbolic notation as its
first argument. The value of \texttt{\ space\ } will be inserted between
units if necessary. Setting \texttt{\ per\ } to \texttt{\ symbol\ } will
format the number with exponents (i.e. \texttt{\ \^{}(-1)\ } ),
\texttt{\ fraction\ } or \texttt{\ /\ } using fraction, and
\texttt{\ fraction-short\ } or
\texttt{\ \textbackslash{}\textbackslash{}/\ } using in-line
fractions.\\
Units in words have four possible parts:

\begin{itemize}
\tightlist
\item
  \texttt{\ per\ } forms the inverse of the following unit.
\item
  A written-out prefix in the sense of SI (e.g. \texttt{\ centi\ } ).
  This is added before the unit.
\item
  The unit itself written out (e.g. \texttt{\ gram\ } ).
\item
  A postfix like \texttt{\ squared\ } . This is added after the unit and
  takes \texttt{\ per\ } into account.
\end{itemize}

The shorthand notation also has four parts:

\begin{itemize}
\tightlist
\item
  \texttt{\ /\ } forms the inverse of the following unit.
\item
  A short prefix in the sense of SI (e.g. \texttt{\ k\ } ). This is
  added before the unit.
\item
  The short unit itself (e.g. \texttt{\ g\ } ).
\item
  An exponent like \texttt{\ \^{}2\ } . This is added after the unit and
  takes \texttt{\ /\ } into account.
\end{itemize}

Note: Use \texttt{\ u\ } for micro.

The possible values of the three latter parts are loaded at runtime from
\texttt{\ prefixes.csv\ } , \texttt{\ units.csv\ } , and
\texttt{\ postfixes.csv\ } (in the library directory). Your own units
etc. can be permanently added in these files. At runtime, they can be
added using \texttt{\ add-unit\ } and \texttt{\ add-prefix\ } ,
respectively. The formats for the pre- and postfixes are:

\begin{longtable}[]{@{}lll@{}}
\toprule\noalign{}
pre-/postfix & shorthand & symbol \\
\midrule\noalign{}
\endhead
\bottomrule\noalign{}
\endlastfoot
milli & m & upright(“m�) \\
\end{longtable}

and for units:

\begin{longtable}[]{@{}llll@{}}
\toprule\noalign{}
unit & shorthand & symbol & space \\
\midrule\noalign{}
\endhead
\bottomrule\noalign{}
\endlastfoot
meter & m & upright(“m�) & true \\
\end{longtable}

The first column specifies the written-out word, the second one the
shorthand. These should be unique. The third column represents the
string that will be inserted as the unit symbol. For units, the last
column describes whether there should be space before the unit (possible
values: \texttt{\ true\ } / \texttt{\ false\ } , \texttt{\ 1\ } ,
\texttt{\ 0\ } ). This is mostly the cases for degrees and other angle
units (e.g. arcseconds).\\
If you think there are units not included that are of interest for other
users, you can create an issue or PR.

\subsection{\texorpdfstring{\texttt{\ qty\ }}{ qty }}\label{qty}

\texttt{\ qty\ } allows a \texttt{\ num\ } as the first argument
following the same rules. The second argument is a unit. If
\texttt{\ rawunit\ } is set to true, its value will be passed on to the
result (note that the string passed on will be passed to
\texttt{\ eval\ } , so add escaped quotes \texttt{\ \textbackslash{}"\ }
if necessary). Otherwise, it follows the rules of \texttt{\ unit\ } .
The value of \texttt{\ space\ } will be inserted between units if
necessary, \texttt{\ thousandsep\ } between the thousands, and
\texttt{\ per\ } switches between exponents and fractions.

\subsection{\texorpdfstring{\texttt{\ numrange\ }}{ numrange }}\label{numrange}

\texttt{\ numrange\ } takes two \texttt{\ num\ } s as the first two
arguments. If they have the same exponent, it is automatically
factorized. The range symbol can be changed with \texttt{\ delimiter\ }
, and the space between the numbers and symbols with \texttt{\ space\ }
.

\subsection{\texorpdfstring{\texttt{\ qtyrange\ }}{ qtyrange }}\label{qtyrange}

\texttt{\ qtyrange\ } is just a combination of \texttt{\ unit\ } and
\texttt{\ range\ } .

\subsubsection{How to add}\label{how-to-add}

Copy this into your project and use the import as \texttt{\ unify\ }

\begin{verbatim}
#import "@preview/unify:0.7.0"
\end{verbatim}

\includesvg[width=0.16667in,height=0.16667in]{/assets/icons/16-copy.svg}

Check the docs for
\href{https://typst.app/docs/reference/scripting/\#packages}{more
information on how to import packages} .

\subsubsection{About}\label{about}

\begin{description}
\tightlist
\item[Author :]
Christopher Hecker
\item[License:]
MIT
\item[Current version:]
0.7.0
\item[Last updated:]
November 28, 2024
\item[First released:]
July 27, 2023
\item[Archive size:]
9.04 kB
\href{https://packages.typst.org/preview/unify-0.7.0.tar.gz}{\pandocbounded{\includesvg[keepaspectratio]{/assets/icons/16-download.svg}}}
\item[Repository:]
\href{https://github.com/ChHecker/unify}{GitHub}
\item[Discipline s :]
\begin{itemize}
\tightlist
\item[]
\item
  \href{https://typst.app/universe/search/?discipline=business}{Business}
\item
  \href{https://typst.app/universe/search/?discipline=chemistry}{Chemistry}
\item
  \href{https://typst.app/universe/search/?discipline=computer-science}{Computer
  Science}
\item
  \href{https://typst.app/universe/search/?discipline=economics}{Economics}
\item
  \href{https://typst.app/universe/search/?discipline=engineering}{Engineering}
\item
  \href{https://typst.app/universe/search/?discipline=mathematics}{Mathematics}
\item
  \href{https://typst.app/universe/search/?discipline=physics}{Physics}
\end{itemize}
\item[Categor y :]
\begin{itemize}
\tightlist
\item[]
\item
  \pandocbounded{\includesvg[keepaspectratio]{/assets/icons/16-text.svg}}
  \href{https://typst.app/universe/search/?category=text}{Text}
\end{itemize}
\end{description}

\subsubsection{Where to report issues?}\label{where-to-report-issues}

This package is a project of Christopher Hecker . Report issues on
\href{https://github.com/ChHecker/unify}{their repository} . You can
also try to ask for help with this package on the
\href{https://forum.typst.app}{Forum} .

Please report this package to the Typst team using the
\href{https://typst.app/contact}{contact form} if you believe it is a
safety hazard or infringes upon your rights.

\phantomsection\label{versions}
\subsubsection{Version history}\label{version-history}

\begin{longtable}[]{@{}ll@{}}
\toprule\noalign{}
Version & Release Date \\
\midrule\noalign{}
\endhead
\bottomrule\noalign{}
\endlastfoot
0.7.0 & November 28, 2024 \\
\href{https://typst.app/universe/package/unify/0.6.1/}{0.6.1} & November
18, 2024 \\
\href{https://typst.app/universe/package/unify/0.6.0/}{0.6.0} & May 23,
2024 \\
\href{https://typst.app/universe/package/unify/0.5.0/}{0.5.0} & March
26, 2024 \\
\href{https://typst.app/universe/package/unify/0.4.3/}{0.4.3} & October
22, 2023 \\
\href{https://typst.app/universe/package/unify/0.4.2/}{0.4.2} & October
9, 2023 \\
\href{https://typst.app/universe/package/unify/0.4.1/}{0.4.1} &
September 3, 2023 \\
\href{https://typst.app/universe/package/unify/0.4.0/}{0.4.0} & July 28,
2023 \\
\href{https://typst.app/universe/package/unify/0.1.0/}{0.1.0} & July 27,
2023 \\
\end{longtable}

Typst GmbH did not create this package and cannot guarantee correct
functionality of this package or compatibility with any version of the
Typst compiler or app.
