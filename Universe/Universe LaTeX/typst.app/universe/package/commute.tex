\title{typst.app/universe/package/commute}

\phantomsection\label{banner}
\section{commute}\label{commute}

{ 0.2.0 }

A proof of concept library for commutative diagrams.

{ } Featured Package

\phantomsection\label{readme}
Proof-of-concept commutative diagrams library for
\href{https://typst.app/home}{typst}

See {[}EricWay1024/tikzcd-editor{]}{[}
\url{https://github.com/EricWay1024/tikzcd-editor} {]} for a web-based
visual diagram editor for this library!

\begin{verbatim}
#import "@preview/commute:0.2.0": node, arr, commutative-diagram

#align(center)[#commutative-diagram(
  node((0, 0), $X$),
  node((0, 1), $Y$),
  node((1, 0), $X \/ "ker"(f)$, "quot"),
  arr($X$, $Y$, $f$),
  arr("quot", (0, 1), $tilde(f)$, label-pos: right, "dashed", "inj"),
  arr($X$, "quot", $pi$),
)]
\end{verbatim}

\pandocbounded{\includegraphics[keepaspectratio]{https://github.com/typst/packages/assets/20535498/71eb8d47-b6f9-43fa-a1fd-7ff58b8d0025}}

For more usage examples look at \texttt{\ example.typ\ }

The library provides 3 functions: \texttt{\ node\ } , \texttt{\ arr\ } ,
and \texttt{\ commutative-diagram\ } . You can clone this repo and
import \texttt{\ lib.typ\ } :

\begin{verbatim}
#import "path/to/commute/lib.typ": node, arr, commutative-diagram
\end{verbatim}

Or directly use the builtin package manager:

\begin{verbatim}
#import "@preview/commute:0.2.0": node, arr, commutative-diagram
\end{verbatim}

\subsection{\texorpdfstring{\texttt{\ commutative-diagram\ }}{ commutative-diagram }}\label{commutative-diagram}

\begin{verbatim}
commutative-diagram(
  node-padding: (70pt, 70pt),
  arr-clearance: 0.7em,
  padding: 1.5em,
  debug: false,
  ..entities
)
\end{verbatim}

\texttt{\ commutative-diagram\ } returns a rectangular region containing
the nodes and arrows. All the unnamed arguments passed to
\texttt{\ commutative-diagram\ } are treated as nodes or arrows of the
diagram. These can be constructed using the \texttt{\ node\ } and
\texttt{\ arr\ } functions explained below. The other arguments are as
follows:

\begin{itemize}
\tightlist
\item
  \texttt{\ node-padding\ } : \texttt{\ (length,\ length)\ } . The space
  to leave between adjacent nodes. It’s a tuple, \texttt{\ (h,\ v)\ }
  , containing the horizontal and vertical spacing respectively.
\item
  \texttt{\ arr-clearance\ } : \texttt{\ length\ } . The default space
  between arrows’ base/tip and the diagram’s nodes.
\item
  \texttt{\ padding\ } : \texttt{\ length\ } . The padding around the
  whole diagram
\item
  \texttt{\ debug\ } : \texttt{\ bool\ } . Whether or not to display
  debug information.
\end{itemize}

\subsection{\texorpdfstring{\texttt{\ node\ }}{ node }}\label{node}

\begin{verbatim}
node(
  pos,
  label,
  id: label,
)
\end{verbatim}

Creates a new diagram node. Has the following positional arguments:

\begin{itemize}
\tightlist
\item
  \texttt{\ pos\ } : \texttt{\ (integer,\ integer)\ } . The position of
  the node in \texttt{\ (row,\ column)\ } format. Must be integers, but
  can be negative, the only thing that counts is the difference between
  the coordinares of the variuos nodes in the diagram.
\item
  \texttt{\ label\ } : \texttt{\ content\ } . The node’s label.
\item
  \texttt{\ id\ } : \texttt{\ any\ } . The node’s id, defaults to its
  label if not specified.
\end{itemize}

\subsection{\texorpdfstring{\texttt{\ arr\ }}{ arr }}\label{arr}

\begin{verbatim}
arr(
  start,
  end,
  label,
  start-space: none,
  end-space: none,
  label-pos: left,
  curve: 0deg,
  stroke: 0.45pt,
  ..options
)
\end{verbatim}

Creates an arrow. Has the following arguments:

\begin{itemize}
\tightlist
\item
  \texttt{\ start\ } : \texttt{\ (integer,\ integer)\ } or
  \texttt{\ any\ } . The position of the node from which the arrow
  starts, in \texttt{\ (row,\ column)\ } format, or its id.
\item
  \texttt{\ end\ } : \texttt{\ (integer,\ integer)\ } or
  \texttt{\ any\ } . The position of the node where the arrow ends, in
  \texttt{\ (row,\ column)\ } format, or its id.
\item
  \texttt{\ label\ } : \texttt{\ content\ } . The label to put on the
  arrow.
\item
  \texttt{\ start-space\ } : \texttt{\ length\ } . The space between the
  start node and the beginning of the arrow. You can pass
  \texttt{\ none\ } to leave a sensible default, customizable using the
  \texttt{\ arr-clearance\ } parameter of the
  \texttt{\ commutative-diagram\ } function.
\item
  \texttt{\ end-space\ } : \texttt{\ length\ } . Similar to the above.
\item
  \texttt{\ label-pos\ } : \texttt{\ length\ } or \texttt{\ left\ } or
  \texttt{\ right\ } . Where to position the arrow’s label relative to
  the arrow. A positive length means that, when looking towards the tip
  of the arrow, the label is on the left. \texttt{\ left\ } and
  \texttt{\ right\ } measure the label to automatically get a reasonable
  length. If set to \texttt{\ 0\ } ( \texttt{\ 0\ } the number, which is
  different from \texttt{\ 0pt\ } or \texttt{\ 0em\ } ) then the label
  is placed on top of the arrow, with a white background to help with
  legibility.
\item
  \texttt{\ curve\ } : \texttt{\ angle\ } . The difference in
  orientation between the start and the end of the arrow. If positive,
  the arrow curves to the right, when looking towards the tip.
\item
  \texttt{\ stroke\ } : \texttt{\ stroke\ } . The thickness and color of
  the arrows. The default should probably be fine.
\item
  \texttt{\ options\ } : \texttt{\ string\ } s. After the mandatory
  positional arguments \texttt{\ start\ } , \texttt{\ end\ } and
  \texttt{\ label\ } , any remaining unnamed argument is treated as an
  extra option. Recognized options are:

  \begin{itemize}
  \tightlist
  \item
    \texttt{\ "inj"\ } , gives the arrow a hook at the start, used for
    injective functions
  \item
    \texttt{\ "surj"\ } , gives the arrow a double tip, used for
    surjective functions
  \item
    \texttt{\ "bij"\ } , gives the arrow a tip also at the start, used
    for bijective functions
  \item
    \texttt{\ "def"\ } , gives the arrow a bar at the start, used for
    function definitions
  \item
    \texttt{\ "nat"\ } , gives the arrow a double stem, used for natural
    transformations
  \item
    All the options supported by the \texttt{\ dash\ } parameter of
    Typst’s \texttt{\ stroke\ } type, such as \texttt{\ "dashed"\ } ,
    \texttt{\ "densely-dotted"\ } , etc. These change the appearance of
    the arrow’s stem
  \end{itemize}
\end{itemize}

\subsubsection{How to add}\label{how-to-add}

Copy this into your project and use the import as \texttt{\ commute\ }

\begin{verbatim}
#import "@preview/commute:0.2.0"
\end{verbatim}

\includesvg[width=0.16667in,height=0.16667in]{/assets/icons/16-copy.svg}

Check the docs for
\href{https://typst.app/docs/reference/scripting/\#packages}{more
information on how to import packages} .

\subsubsection{About}\label{about}

\begin{description}
\tightlist
\item[Author :]
\href{https://gitlab.com/giacomogallina}{giacomogallina}
\item[License:]
MIT
\item[Current version:]
0.2.0
\item[Last updated:]
November 1, 2023
\item[First released:]
July 21, 2023
\item[Archive size:]
6.15 kB
\href{https://packages.typst.org/preview/commute-0.2.0.tar.gz}{\pandocbounded{\includesvg[keepaspectratio]{/assets/icons/16-download.svg}}}
\item[Repository:]
\href{https://gitlab.com/giacomogallina/commute}{GitLab}
\end{description}

\subsubsection{Where to report issues?}\label{where-to-report-issues}

This package is a project of giacomogallina . Report issues on
\href{https://gitlab.com/giacomogallina/commute}{their repository} . You
can also try to ask for help with this package on the
\href{https://forum.typst.app}{Forum} .

Please report this package to the Typst team using the
\href{https://typst.app/contact}{contact form} if you believe it is a
safety hazard or infringes upon your rights.

\phantomsection\label{versions}
\subsubsection{Version history}\label{version-history}

\begin{longtable}[]{@{}ll@{}}
\toprule\noalign{}
Version & Release Date \\
\midrule\noalign{}
\endhead
\bottomrule\noalign{}
\endlastfoot
0.2.0 & November 1, 2023 \\
\href{https://typst.app/universe/package/commute/0.1.0/}{0.1.0} & July
21, 2023 \\
\end{longtable}

Typst GmbH did not create this package and cannot guarantee correct
functionality of this package or compatibility with any version of the
Typst compiler or app.
