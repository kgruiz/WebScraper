\title{typst.app/universe/package/ichigo}

\phantomsection\label{banner}
\phantomsection\label{template-thumbnail}
\pandocbounded{\includegraphics[keepaspectratio]{https://packages.typst.org/preview/thumbnails/ichigo-0.2.0-small.webp}}

\section{ichigo}\label{ichigo}

{ 0.2.0 }

A customizable Typst template for homework

\href{/app?template=ichigo&version=0.2.0}{Create project in app}

\phantomsection\label{readme}
Homework Template - 作业模�

\subsection{Usage -
使ç''¨æ--¹æ³•}\label{usage---uxe4uxbduxe7uxe6uxb9uxe6uxb3}

\begin{Shaded}
\begin{Highlighting}[]
\NormalTok{\#import "@preview/ichigo:0.2.0": config, prob}

\NormalTok{\#show: config.with(}
\NormalTok{  course{-}name: "Typst 使用小练习",}
\NormalTok{  serial{-}str: "第 1 次作业",}
\NormalTok{  author{-}info: [}
\NormalTok{    sjfhsjfh from PKU{-}Typst}
\NormalTok{  ],}
\NormalTok{  author{-}names: "sjfhsjfh",}
\NormalTok{)}

\NormalTok{\#prob[}
\NormalTok{  Calculate the 25th number in the Fibonacci sequence using Typst}
\NormalTok{][}
\NormalTok{  \#let f(n) = \{}
\NormalTok{    if n \textless{}= 2 \{}
\NormalTok{      return 1}
\NormalTok{    \}}
\NormalTok{    return f(n {-} 1) + f(n {-} 2)}
\NormalTok{  \}}
\NormalTok{  \#f(25)}
\NormalTok{]}
\end{Highlighting}
\end{Shaded}

\subsection{Documentation - æ--‡æ¡£}\label{documentation---uxe6uxe6}

\href{https://github.com/PKU-Typst/ichigo/releases/download/v0.2.0/documentation.pdf}{Click
to download - 点击下载}

\subsection{TODO - å¾\ldots 办}\label{todo---uxe5uxbeuxe5ux161ux17e}

\begin{itemize}
\tightlist
\item
  {[} {]} Theme list \& documentation
\end{itemize}

\href{/app?template=ichigo&version=0.2.0}{Create project in app}

\subsubsection{How to use}\label{how-to-use}

Click the button above to create a new project using this template in
the Typst app.

You can also use the Typst CLI to start a new project on your computer
using this command:

\begin{verbatim}
typst init @preview/ichigo:0.2.0
\end{verbatim}

\includesvg[width=0.16667in,height=0.16667in]{/assets/icons/16-copy.svg}

\subsubsection{About}\label{about}

\begin{description}
\tightlist
\item[Author :]
\href{https://github.com/PKU-Typst}{PKU-Typst}
\item[License:]
MIT
\item[Current version:]
0.2.0
\item[Last updated:]
November 18, 2024
\item[First released:]
October 3, 2024
\item[Archive size:]
17.1 kB
\href{https://packages.typst.org/preview/ichigo-0.2.0.tar.gz}{\pandocbounded{\includesvg[keepaspectratio]{/assets/icons/16-download.svg}}}
\item[Repository:]
\href{https://github.com/PKU-Typst/ichigo}{GitHub}
\item[Categor y :]
\begin{itemize}
\tightlist
\item[]
\item
  \pandocbounded{\includesvg[keepaspectratio]{/assets/icons/16-speak.svg}}
  \href{https://typst.app/universe/search/?category=report}{Report}
\end{itemize}
\end{description}

\subsubsection{Where to report issues?}\label{where-to-report-issues}

This template is a project of PKU-Typst . Report issues on
\href{https://github.com/PKU-Typst/ichigo}{their repository} . You can
also try to ask for help with this template on the
\href{https://forum.typst.app}{Forum} .

Please report this template to the Typst team using the
\href{https://typst.app/contact}{contact form} if you believe it is a
safety hazard or infringes upon your rights.

\phantomsection\label{versions}
\subsubsection{Version history}\label{version-history}

\begin{longtable}[]{@{}ll@{}}
\toprule\noalign{}
Version & Release Date \\
\midrule\noalign{}
\endhead
\bottomrule\noalign{}
\endlastfoot
0.2.0 & November 18, 2024 \\
\href{https://typst.app/universe/package/ichigo/0.1.0/}{0.1.0} & October
3, 2024 \\
\end{longtable}

Typst GmbH did not create this template and cannot guarantee correct
functionality of this template or compatibility with any version of the
Typst compiler or app.
