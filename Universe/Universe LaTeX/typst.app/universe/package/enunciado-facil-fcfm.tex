\title{typst.app/universe/package/enunciado-facil-fcfm}

\phantomsection\label{banner}
\phantomsection\label{template-thumbnail}
\pandocbounded{\includegraphics[keepaspectratio]{https://packages.typst.org/preview/thumbnails/enunciado-facil-fcfm-0.1.0-small.webp}}

\section{enunciado-facil-fcfm}\label{enunciado-facil-fcfm}

{ 0.1.0 }

Documentos de ejercicios (controles, auxiliares, tareas, pautas) para la
FCFM, UChile

\href{/app?template=enunciado-facil-fcfm&version=0.1.0}{Create project
in app}

\phantomsection\label{readme}
Template de Typst para documentos de la FCFM (auxiliares, controles,
pautas)

\subsection{Ejemplo de uso}\label{ejemplo-de-uso}

\subsubsection{\texorpdfstring{En
\href{https://typst.app/}{typst.app}}{En typst.app}}\label{en-typst.app}

Si utilizas la aplicación web oficial, puedes presionar “Start from
template� y buscar “enunciado-facil-fcfm� para crear un proyecto
ya inicializado con el template.

\subsubsection{En CLI}\label{en-cli}

Si usas Typst de manera local, puedes ejecutar:

\begin{Shaded}
\begin{Highlighting}[]
\ExtensionTok{typst}\NormalTok{ init @preview/enunciado{-}facil{-}fcfm:0.1.0}
\end{Highlighting}
\end{Shaded}

lo cual inicializará un proyecto usando el template en el directorio
actual.

\subsubsection{Manualmente}\label{manualmente}

Basta crear un archivo con el siguiente contenido para usar el template:

\begin{Shaded}
\begin{Highlighting}[]
\NormalTok{\#import "@preview/enunciado{-}facil{-}fcfm:0.1.0" as template}

\NormalTok{\#show: template.conf.with(}
\NormalTok{  titulo: "Auxiliar 1",}
\NormalTok{  subtitulo: "Typst",}
\NormalTok{  titulo{-}extra: (}
\NormalTok{    [*Profesora*: Ada Lovelace],}
\NormalTok{    [*Auxiliares*: Grace Hopper y Alan Turing],}
\NormalTok{  ),}
\NormalTok{  departamento: template.departamentos.dcc,}
\NormalTok{  curso: "CC4034 {-} Composición de documentos",}
\NormalTok{)}

\NormalTok{...el resto del documento comienza acá}
\end{Highlighting}
\end{Shaded}

Puedes ver un ejemplo más completo en
\href{https://github.com/typst/packages/raw/main/packages/preview/enunciado-facil-fcfm/0.1.0/template/main.typ}{main.typ}
. Para aprender la sintáxis de Typst existe la
\href{https://typst.app/docs}{documentación oficial} . Si vienes desde
LaTeX, te recomiendo la
\href{https://typst.app/docs/guides/guide-for-latex-users/}{guía para
usuarios de LaTeX} .

\subsection{Configuración}\label{configuraciuxe3uxb3n}

La función \texttt{\ conf\ } importada desde el template recibe los
siguientes parámetros:

\begin{longtable}[]{@{}ll@{}}
\toprule\noalign{}
Parámetro & Descripción \\
\midrule\noalign{}
\endhead
\bottomrule\noalign{}
\endlastfoot
\texttt{\ titulo\ } & Título del documento \\
\texttt{\ subtitulo\ } & Subtítulo del documento \\
\texttt{\ titulo-extra\ } & Arreglo con bloques de contenido adicionales
a agregar después del título. Útil para mostrar los nombres del equipo
docente. \\
\texttt{\ departamento\ } & Diccionario que contiene el nombre (
\texttt{\ string\ } ) y el logo del departamento ( \texttt{\ content\ }
). El template viene con uno ya creado para cada departamento bajo
\texttt{\ template.departamentos\ } . Valor por defecto:
\texttt{\ template.departamentos.dcc\ } \\
\texttt{\ curso\ } & Código y/o nombre del curso. \\
\texttt{\ page-conf\ } & Diccionario con parámetros adicionales
(tamaño de página, márgenes, etc) para pasarle a la función
\href{https://typst.app/docs/reference/layout/page/}{page} . \\
\end{longtable}

\subsection{FAQ}\label{faq}

\subsubsection{Cómo cambiar el logo del
departamento}\label{cuxe3uxb3mo-cambiar-el-logo-del-departamento}

El parámetro \texttt{\ departamento\ } solamente es un diccionario de
Typst con las llaves \texttt{\ nombre\ } y \texttt{\ logo\ } . Puedes
crear un diccionario con un logo personalizado y pasárselo al template:

\begin{Shaded}
\begin{Highlighting}[]
\NormalTok{\#import "@preview/enunciado{-}facil{-}fcfm:0.1.0" as template}

\NormalTok{\#let mi{-}departamento = (}
\NormalTok{  nombre: "Mi súper departamento personalizado",}
\NormalTok{  logo: image("mi{-}super{-}logo.png"),}
\NormalTok{)}

\NormalTok{\#show: template.conf.with(}
\NormalTok{  titulo: "Documento con logo personalizado",}
\NormalTok{  departamento: mi{-}departamento,}
\NormalTok{  curso: "CC4034 {-} Composición de documentos",}
\NormalTok{)}
\end{Highlighting}
\end{Shaded}

\subsubsection{Cómo cambiar márgenes, tamaño de página,
etcétera}\label{cuxe3uxb3mo-cambiar-muxe3rgenes-tamauxe3o-de-puxe3gina-etcuxe3tera}

Para cambiar la configuración de la página hay que interceptar la
\href{https://typst.app/docs/reference/styling/\#set-rules}{set rule}
que se hace sobre \texttt{\ page\ } . Para ello, el template expone el
parámetro \texttt{\ page-conf\ } que permit sobreescribir la
configuración de página del template. Por ejemplo, para cambiar el
tamaño del papel a A4:

\begin{Shaded}
\begin{Highlighting}[]
\NormalTok{\#import "@preview/enunciado{-}facil{-}fcfm:0.1.0" as template}

\NormalTok{\#show: template.conf.with(}
\NormalTok{  titulo: "Documento con tamaño A4",}
\NormalTok{  departamento: template.departamentos.dcc,}
\NormalTok{  curso: "CC4034 {-} Composición de documentos",}
\NormalTok{  page{-}conf: (paper: "a4")}
\NormalTok{)}
\end{Highlighting}
\end{Shaded}

\subsubsection{Cómo cambiar la fuente, headings,
etc}\label{cuxe3uxb3mo-cambiar-la-fuente-headings-etc}

Usando \href{https://typst.app/docs/reference/styling/}{show y set
rules} puedes personalizar mucho más el template. Por ejemplo, para
cambiar la fuente:

\begin{Shaded}
\begin{Highlighting}[]
\NormalTok{\#import "@preview/enunciado{-}facil{-}fcfm:0.1.0" as template}

\NormalTok{// En este caso hay que cambiar la fuente}
\NormalTok{// antes de que se configure el template}
\NormalTok{// para que se aplique en el título y encabezado}
\NormalTok{\#set text(font: "New Computer Modern")}

\NormalTok{\#show: template.conf.with(}
\NormalTok{  titulo: "Documento con la fuente de LaTeX",}
\NormalTok{  departamento: template.departamentos.dcc,}
\NormalTok{  curso: "CC4034 {-} Composición de documentos",}
\NormalTok{)}
\end{Highlighting}
\end{Shaded}

\href{/app?template=enunciado-facil-fcfm&version=0.1.0}{Create project
in app}

\subsubsection{How to use}\label{how-to-use}

Click the button above to create a new project using this template in
the Typst app.

You can also use the Typst CLI to start a new project on your computer
using this command:

\begin{verbatim}
typst init @preview/enunciado-facil-fcfm:0.1.0
\end{verbatim}

\includesvg[width=0.16667in,height=0.16667in]{/assets/icons/16-copy.svg}

\subsubsection{About}\label{about}

\begin{description}
\tightlist
\item[Author :]
\href{https://github.com/bkorecic}{Blaz Korecic}
\item[License:]
MIT
\item[Current version:]
0.1.0
\item[Last updated:]
October 9, 2024
\item[First released:]
October 9, 2024
\item[Archive size:]
264 kB
\href{https://packages.typst.org/preview/enunciado-facil-fcfm-0.1.0.tar.gz}{\pandocbounded{\includesvg[keepaspectratio]{/assets/icons/16-download.svg}}}
\item[Repository:]
\href{https://github.com/bkorecic/enunciado-facil-fcfm}{GitHub}
\item[Categor y :]
\begin{itemize}
\tightlist
\item[]
\item
  \pandocbounded{\includesvg[keepaspectratio]{/assets/icons/16-speak.svg}}
  \href{https://typst.app/universe/search/?category=report}{Report}
\end{itemize}
\end{description}

\subsubsection{Where to report issues?}\label{where-to-report-issues}

This template is a project of Blaz Korecic . Report issues on
\href{https://github.com/bkorecic/enunciado-facil-fcfm}{their
repository} . You can also try to ask for help with this template on the
\href{https://forum.typst.app}{Forum} .

Please report this template to the Typst team using the
\href{https://typst.app/contact}{contact form} if you believe it is a
safety hazard or infringes upon your rights.

\phantomsection\label{versions}
\subsubsection{Version history}\label{version-history}

\begin{longtable}[]{@{}ll@{}}
\toprule\noalign{}
Version & Release Date \\
\midrule\noalign{}
\endhead
\bottomrule\noalign{}
\endlastfoot
0.1.0 & October 9, 2024 \\
\end{longtable}

Typst GmbH did not create this template and cannot guarantee correct
functionality of this template or compatibility with any version of the
Typst compiler or app.
