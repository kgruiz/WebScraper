\title{typst.app/universe/package/outline-summaryst}

\phantomsection\label{banner}
\section{outline-summaryst}\label{outline-summaryst}

{ 0.1.0 }

A basic template for including a summary for each entry in the table of
contents. Useful for writing books.

\phantomsection\label{readme}
\subsection{Description}\label{description}

\texttt{\ outline-summaryst\ } is a basic package designed for including
a summary for each entry in the table of contents, particularly useful
for writing books. It provides a simple structure for organizing content
and generating formatted documents with summary sections.

\subsection{Features}\label{features}

\begin{itemize}
\tightlist
\item
  A template for the outline, which styles both the heading and their
  summaries.
\item
  A macro for creating new headings and a summary for each heading.
\end{itemize}

\subsection{Note:}\label{note}

Because of the way the project is implemented, only the headings created
with the provided \texttt{\ make-heading("heading\ name",\ "summary")\ }
are shown in in the outline. Headings created with the default
\texttt{\ =\ Heading\ } syntax will not show in said outline (though
they will show up in the document itself).

\subsection{Example Usage:}\label{example-usage}

\begin{verbatim}
#import "@preview/outline-summaryst:0.1.0": style-outline, make-heading


// you can set `outline-title: none` in order not to display any title
#show outline: style-outline.with(outline-title: "Table of Contents")

#outline()


#make-heading("Part One", "This is the summary for part one")
#lorem(500)

#make-heading("Chapter One", "Summary for chapter one in part one", level: 2)
#lorem(300)

#make-heading("Chapter Two", "This is the summary for chapter two in part one", level: 2)
#lorem(300)

#make-heading("Part Two", "And here we have the summary for part two")
#lorem(500)

#make-heading("Chapter One", "Summary for chapter one in part two", level: 2)
#lorem(300)

#make-heading("Chapter Two", "Summary for chapter two in part two", level: 2)
#lorem(300)
\end{verbatim}

\subsection{Known limitations}\label{known-limitations}

\begin{itemize}
\tightlist
\item
  Currently, we do not provide a way for styling the table of contents
  or headings
\end{itemize}

\subsection{License:}\label{license}

This project is licensed under the MIT License. See the LICENSE file for
details.

\subsection{Contribution:}\label{contribution}

Contributions are welcome! Feel free to open an issue or submit a pull
request on GitHub.

\subsubsection{How to add}\label{how-to-add}

Copy this into your project and use the import as
\texttt{\ outline-summaryst\ }

\begin{verbatim}
#import "@preview/outline-summaryst:0.1.0"
\end{verbatim}

\includesvg[width=0.16667in,height=0.16667in]{/assets/icons/16-copy.svg}

Check the docs for
\href{https://typst.app/docs/reference/scripting/\#packages}{more
information on how to import packages} .

\subsubsection{About}\label{about}

\begin{description}
\tightlist
\item[Author :]
@aarneng
\item[License:]
MIT
\item[Current version:]
0.1.0
\item[Last updated:]
April 15, 2024
\item[First released:]
April 15, 2024
\item[Archive size:]
2.97 kB
\href{https://packages.typst.org/preview/outline-summaryst-0.1.0.tar.gz}{\pandocbounded{\includesvg[keepaspectratio]{/assets/icons/16-download.svg}}}
\item[Repository:]
\href{https://github.com/aarneng/Outline-Summary}{GitHub}
\item[Discipline :]
\begin{itemize}
\tightlist
\item[]
\item
  \href{https://typst.app/universe/search/?discipline=literature}{Literature}
\end{itemize}
\item[Categor y :]
\begin{itemize}
\tightlist
\item[]
\item
  \pandocbounded{\includesvg[keepaspectratio]{/assets/icons/16-layout.svg}}
  \href{https://typst.app/universe/search/?category=layout}{Layout}
\end{itemize}
\end{description}

\subsubsection{Where to report issues?}\label{where-to-report-issues}

This package is a project of @aarneng . Report issues on
\href{https://github.com/aarneng/Outline-Summary}{their repository} .
You can also try to ask for help with this package on the
\href{https://forum.typst.app}{Forum} .

Please report this package to the Typst team using the
\href{https://typst.app/contact}{contact form} if you believe it is a
safety hazard or infringes upon your rights.

\phantomsection\label{versions}
\subsubsection{Version history}\label{version-history}

\begin{longtable}[]{@{}ll@{}}
\toprule\noalign{}
Version & Release Date \\
\midrule\noalign{}
\endhead
\bottomrule\noalign{}
\endlastfoot
0.1.0 & April 15, 2024 \\
\end{longtable}

Typst GmbH did not create this package and cannot guarantee correct
functionality of this package or compatibility with any version of the
Typst compiler or app.
