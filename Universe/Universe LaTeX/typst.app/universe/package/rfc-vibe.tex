\title{typst.app/universe/package/rfc-vibe}

\phantomsection\label{banner}
\section{rfc-vibe}\label{rfc-vibe}

{ 0.1.0 }

Bring RFC language into everyday docs

\phantomsection\label{readme}
Bring that RFC lingo to your everyday documents.

A \href{https://typst.app/}{Typst} package that makes it easy to use the
exact keywords, boilerplate, and definitions provided by BCP 14,
RFC2119, and RFC8174. See the end of this README for an example of the
output.

In the future, this package may include other RFC-related patterns which
are applicable to a wide variety of everyday documents.

\subsection{Usage}\label{usage}

Import the package in your Typst document:

\begin{Shaded}
\begin{Highlighting}[]
\NormalTok{\#import "@preview/rfc{-}vibe:0.1.0": *}
\end{Highlighting}
\end{Shaded}

\subsubsection{Keywords}\label{keywords}

Use the keywords according to these examples:

\begin{Shaded}
\begin{Highlighting}[]
\NormalTok{\#must              // renders as: MUST}
\NormalTok{\#must{-}not          // renders as: MUST NOT}
\NormalTok{\#required          // renders as: REQUIRED}
\NormalTok{\#shall             // renders as: SHALL}
\NormalTok{\#shall{-}not         // renders as: SHALL NOT}
\NormalTok{\#should            // renders as: SHOULD}
\NormalTok{\#should{-}not        // renders as: SHOULD NOT}
\NormalTok{\#recommended       // renders as: RECOMMENDED}
\NormalTok{\#not{-}recommended   // renders as: NOT RECOMMENDED}
\NormalTok{\#may               // renders as: MAY}
\NormalTok{\#optional          // renders as: OPTIONAL}
\end{Highlighting}
\end{Shaded}

For the rare situation when you want the keywords included in quotation
marks, use the \texttt{\ -quoted\ } versions:

\begin{Shaded}
\begin{Highlighting}[]
\NormalTok{\#must{-}quoted              // renders as: "MUST"}
\NormalTok{\#must{-}not{-}quoted          // renders as: "MUST NOT"}
\NormalTok{\#required{-}quoted          // renders as: "REQUIRED"}
\NormalTok{\#shall{-}quoted             // renders as: "SHALL"}
\NormalTok{\#shall{-}not{-}quoted         // renders as: "SHALL NOT"}
\NormalTok{\#should{-}quoted            // renders as: "SHOULD"}
\NormalTok{\#should{-}not{-}quoted        // renders as: "SHOULD NOT"}
\NormalTok{\#recommended{-}quoted       // renders as: "RECOMMENDED"}
\NormalTok{\#not{-}recommended{-}quoted   // renders as: "NOT RECOMMENDED"}
\NormalTok{\#may{-}quoted               // renders as: "MAY"}
\NormalTok{\#optional{-}quoted          // renders as: "OPTIONAL"}
\end{Highlighting}
\end{Shaded}

\subsubsection{Boilerplate}\label{boilerplate}

According to RFC8174, \emph{authors who follow these guidelines should
incorporate a specific phrase near the beginning of their document} .
Include this boilerplate text with:

\begin{Shaded}
\begin{Highlighting}[]
\NormalTok{\#rfc{-}keyword{-}boilerplate}
\end{Highlighting}
\end{Shaded}

This will render as:

\begin{Shaded}
\begin{Highlighting}[]
\NormalTok{The key words "MUST", "MUST NOT", "REQUIRED", "SHALL", "SHALL NOT", "SHOULD",}
\NormalTok{"SHOULD NOT", "RECOMMENDED", "NOT RECOMMENDED", "MAY", and "OPTIONAL" in this}
\NormalTok{document are to be interpreted as described in BCP 14 [RFC2119] [RFC8174] when,}
\NormalTok{and only when, they appear in all capitals, as shown here.}
\end{Highlighting}
\end{Shaded}

\subsubsection{Definitions}\label{definitions}

Although not required (and maybe discouraged), you can include
definitions of individual keywords in your document:

\begin{Shaded}
\begin{Highlighting}[]
\NormalTok{\#rfc{-}keyword{-}must{-}definition}
\NormalTok{\#rfc{-}keyword{-}must{-}not{-}definition}
\NormalTok{\#rfc{-}keyword{-}should{-}definition}
\NormalTok{\#rfc{-}keyword{-}should{-}not{-}definition}
\NormalTok{\#rfc{-}keyword{-}may{-}definition}
\end{Highlighting}
\end{Shaded}

Or include all keyword definitions at once with:

\begin{Shaded}
\begin{Highlighting}[]
\NormalTok{\#rfc{-}keyword{-}definitions}
\end{Highlighting}
\end{Shaded}

\subsection{Example Output}\label{example-output}

\pandocbounded{\includegraphics[keepaspectratio]{https://github.com/typst/packages/raw/main/packages/preview/rfc-vibe/0.1.0/thumbnail.png}}

\subsection{License}\label{license}

This project is licensed under the MIT License. See the
\href{https://github.com/typst/packages/raw/main/packages/preview/rfc-vibe/0.1.0/LICENSE}{LICENSE}
file for details.

\subsubsection{How to add}\label{how-to-add}

Copy this into your project and use the import as \texttt{\ rfc-vibe\ }

\begin{verbatim}
#import "@preview/rfc-vibe:0.1.0"
\end{verbatim}

\includesvg[width=0.16667in,height=0.16667in]{/assets/icons/16-copy.svg}

Check the docs for
\href{https://typst.app/docs/reference/scripting/\#packages}{more
information on how to import packages} .

\subsubsection{About}\label{about}

\begin{description}
\tightlist
\item[Author :]
\href{mailto:steve@waits.net}{Stephen Waits}
\item[License:]
MIT
\item[Current version:]
0.1.0
\item[Last updated:]
November 28, 2024
\item[First released:]
November 28, 2024
\item[Archive size:]
3.35 kB
\href{https://packages.typst.org/preview/rfc-vibe-0.1.0.tar.gz}{\pandocbounded{\includesvg[keepaspectratio]{/assets/icons/16-download.svg}}}
\item[Repository:]
\href{https://github.com/swaits/typst-collection}{GitHub}
\item[Categor y :]
\begin{itemize}
\tightlist
\item[]
\item
  \pandocbounded{\includesvg[keepaspectratio]{/assets/icons/16-hammer.svg}}
  \href{https://typst.app/universe/search/?category=utility}{Utility}
\end{itemize}
\end{description}

\subsubsection{Where to report issues?}\label{where-to-report-issues}

This package is a project of Stephen Waits . Report issues on
\href{https://github.com/swaits/typst-collection}{their repository} .
You can also try to ask for help with this package on the
\href{https://forum.typst.app}{Forum} .

Please report this package to the Typst team using the
\href{https://typst.app/contact}{contact form} if you believe it is a
safety hazard or infringes upon your rights.

\phantomsection\label{versions}
\subsubsection{Version history}\label{version-history}

\begin{longtable}[]{@{}ll@{}}
\toprule\noalign{}
Version & Release Date \\
\midrule\noalign{}
\endhead
\bottomrule\noalign{}
\endlastfoot
0.1.0 & November 28, 2024 \\
\end{longtable}

Typst GmbH did not create this package and cannot guarantee correct
functionality of this package or compatibility with any version of the
Typst compiler or app.
