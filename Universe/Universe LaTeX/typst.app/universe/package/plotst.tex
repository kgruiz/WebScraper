\title{typst.app/universe/package/plotst}

\phantomsection\label{banner}
\section{plotst}\label{plotst}

{ 0.2.0 }

A library to draw a variety of graphs and plots to use in your papers

\phantomsection\label{readme}
A Typst library for drawing graphs and plots. Made by Gewi413 and
Pegacraffft

\subsection{Currently supported
graphs}\label{currently-supported-graphs}

\begin{itemize}
\item
  Scatter plots
\item
  Graph charts
\item
  Histograms
\item
  Bar charts
\item
  Pie charts
\item
  Overlaying plots/charts

  (more to come)
\end{itemize}

\subsection{How to use}\label{how-to-use}

To use the package you can import it through this command
\texttt{\ import\ "@preview/plotst:0.2.0":\ *\ } . The documentation is
found in the
\href{https://github.com/Pegacraft/typst-plotting/blob/8d834689359b708ce75fe51be05eed45570e463e/docs/Docs.pdf}{Docs.pdf}
file. It contains all functions necessary to use this library. It also
includes a tutorial to create every available plot under their
respective render methods.

If you need some example code, check out
\href{https://github.com/Pegacraft/typst-plotting/blob/8d834689359b708ce75fe51be05eed45570e463e/example/main.typ}{main.typ}
. It also includes a
\href{https://github.com/Pegacraft/typst-plotting/blob/8d834689359b708ce75fe51be05eed45570e463e/example/Plotting.pdf}{compiled
version} .

\subsection{Examples:}\label{examples}

All these images were created using the
\href{https://github.com/Pegacraft/typst-plotting/blob/8d834689359b708ce75fe51be05eed45570e463e/example/main.typ}{main.typ}
.

\subsubsection{Scatter plots}\label{scatter-plots}

\begin{Shaded}
\begin{Highlighting}[]
\CommentTok{// Plot 1:}
\CommentTok{// The data to be displayed  }
\KeywordTok{let}\NormalTok{ gender\_data }\OperatorTok{=}\NormalTok{ (}
\NormalTok{  (}\StringTok{"w"}\OperatorTok{,} \DecValTok{1}\NormalTok{)}\OperatorTok{,}\NormalTok{ (}\StringTok{"w"}\OperatorTok{,} \DecValTok{3}\NormalTok{)}\OperatorTok{,}\NormalTok{ (}\StringTok{"w"}\OperatorTok{,} \DecValTok{5}\NormalTok{)}\OperatorTok{,}\NormalTok{ (}\StringTok{"w"}\OperatorTok{,} \DecValTok{4}\NormalTok{)}\OperatorTok{,}\NormalTok{ (}\StringTok{"m"}\OperatorTok{,} \DecValTok{2}\NormalTok{)}\OperatorTok{,}\NormalTok{ (}\StringTok{"m"}\OperatorTok{,} \DecValTok{2}\NormalTok{)}\OperatorTok{,}
\NormalTok{  (}\StringTok{"m"}\OperatorTok{,} \DecValTok{4}\NormalTok{)}\OperatorTok{,}\NormalTok{ (}\StringTok{"m"}\OperatorTok{,} \DecValTok{6}\NormalTok{)}\OperatorTok{,}\NormalTok{ (}\StringTok{"d"}\OperatorTok{,} \DecValTok{1}\NormalTok{)}\OperatorTok{,}\NormalTok{ (}\StringTok{"d"}\OperatorTok{,} \DecValTok{9}\NormalTok{)}\OperatorTok{,}\NormalTok{ (}\StringTok{"d"}\OperatorTok{,} \DecValTok{5}\NormalTok{)}\OperatorTok{,}\NormalTok{ (}\StringTok{"d"}\OperatorTok{,} \DecValTok{8}\NormalTok{)}\OperatorTok{,}
\NormalTok{  (}\StringTok{"d"}\OperatorTok{,} \DecValTok{3}\NormalTok{)}\OperatorTok{,}\NormalTok{ (}\StringTok{"d"}\OperatorTok{,} \DecValTok{1}\NormalTok{)}\OperatorTok{,}\NormalTok{ (}\DecValTok{0}\OperatorTok{,} \DecValTok{11}\NormalTok{)}
\NormalTok{)}

\CommentTok{// Create the axes used for the chart}
\KeywordTok{let}\NormalTok{ y\_axis }\OperatorTok{=} \FunctionTok{axis}\NormalTok{(min}\OperatorTok{:} \DecValTok{0}\OperatorTok{,}\NormalTok{ max}\OperatorTok{:} \DecValTok{11}\OperatorTok{,}\NormalTok{ step}\OperatorTok{:} \DecValTok{1}\OperatorTok{,}\NormalTok{ location}\OperatorTok{:} \StringTok{"left"}\OperatorTok{,}\NormalTok{ helper\_lines}\OperatorTok{:} \KeywordTok{true}\OperatorTok{,}\NormalTok{ invert\_markings}\OperatorTok{:} \KeywordTok{false}\OperatorTok{,}\NormalTok{ title}\OperatorTok{:} \StringTok{"foo"}\NormalTok{)}
\KeywordTok{let}\NormalTok{ x\_axis }\OperatorTok{=} \FunctionTok{axis}\NormalTok{(values}\OperatorTok{:}\NormalTok{ (}\StringTok{""}\OperatorTok{,} \StringTok{"m"}\OperatorTok{,} \StringTok{"w"}\OperatorTok{,} \StringTok{"d"}\NormalTok{)}\OperatorTok{,}\NormalTok{ location}\OperatorTok{:} \StringTok{"bottom"}\OperatorTok{,}\NormalTok{ helper\_lines}\OperatorTok{:} \KeywordTok{true}\OperatorTok{,}\NormalTok{ invert\_markings}\OperatorTok{:} \KeywordTok{false}\OperatorTok{,}\NormalTok{ title}\OperatorTok{:} \StringTok{"Gender"}\NormalTok{)}

\CommentTok{// Combine the axes and the data and feed it to the plot render function.}
\KeywordTok{let}\NormalTok{ pl }\OperatorTok{=} \FunctionTok{plot}\NormalTok{(data}\OperatorTok{:}\NormalTok{ gender\_data}\OperatorTok{,}\NormalTok{ axes}\OperatorTok{:}\NormalTok{ (x\_axis}\OperatorTok{,}\NormalTok{ y\_axis))}
\FunctionTok{scatter\_plot}\NormalTok{(pl}\OperatorTok{,}\NormalTok{ (}\DecValTok{100}\OperatorTok{\%,}\DecValTok{50}\OperatorTok{\%}\NormalTok{))}

\CommentTok{// Plot 2:}
\CommentTok{// Same as above}
\KeywordTok{let}\NormalTok{ data }\OperatorTok{=}\NormalTok{ (}
\NormalTok{  (}\DecValTok{0}\OperatorTok{,} \DecValTok{0}\NormalTok{)}\OperatorTok{,}\NormalTok{ (}\DecValTok{2}\OperatorTok{,} \DecValTok{2}\NormalTok{)}\OperatorTok{,}\NormalTok{ (}\DecValTok{3}\OperatorTok{,} \DecValTok{0}\NormalTok{)}\OperatorTok{,}\NormalTok{ (}\DecValTok{4}\OperatorTok{,} \DecValTok{4}\NormalTok{)}\OperatorTok{,}\NormalTok{ (}\DecValTok{5}\OperatorTok{,} \DecValTok{7}\NormalTok{)}\OperatorTok{,}\NormalTok{ (}\DecValTok{6}\OperatorTok{,} \DecValTok{6}\NormalTok{)}\OperatorTok{,}\NormalTok{ (}\DecValTok{7}\OperatorTok{,} \DecValTok{9}\NormalTok{)}\OperatorTok{,}\NormalTok{ (}\DecValTok{8}\OperatorTok{,} \DecValTok{5}\NormalTok{)}\OperatorTok{,}\NormalTok{ (}\DecValTok{9}\OperatorTok{,} \DecValTok{9}\NormalTok{)}\OperatorTok{,}\NormalTok{ (}\DecValTok{10}\OperatorTok{,} \DecValTok{1}\NormalTok{)}
\NormalTok{)}
\KeywordTok{let}\NormalTok{ x\_axis }\OperatorTok{=} \FunctionTok{axis}\NormalTok{(min}\OperatorTok{:} \DecValTok{0}\OperatorTok{,}\NormalTok{ max}\OperatorTok{:} \DecValTok{11}\OperatorTok{,}\NormalTok{ step}\OperatorTok{:} \DecValTok{2}\OperatorTok{,}\NormalTok{ location}\OperatorTok{:} \StringTok{"bottom"}\NormalTok{)}
\KeywordTok{let}\NormalTok{ y\_axis }\OperatorTok{=} \FunctionTok{axis}\NormalTok{(min}\OperatorTok{:} \DecValTok{0}\OperatorTok{,}\NormalTok{ max}\OperatorTok{:} \DecValTok{11}\OperatorTok{,}\NormalTok{ step}\OperatorTok{:} \DecValTok{2}\OperatorTok{,}\NormalTok{ location}\OperatorTok{:} \StringTok{"left"}\OperatorTok{,}\NormalTok{ helper\_lines}\OperatorTok{:} \KeywordTok{false}\NormalTok{)}
\KeywordTok{let}\NormalTok{ pl }\OperatorTok{=} \FunctionTok{plot}\NormalTok{(data}\OperatorTok{:}\NormalTok{ data}\OperatorTok{,}\NormalTok{ axes}\OperatorTok{:}\NormalTok{ (x\_axis}\OperatorTok{,}\NormalTok{ y\_axis))}
\FunctionTok{scatter\_plot}\NormalTok{(pl}\OperatorTok{,}\NormalTok{ (}\DecValTok{100}\OperatorTok{\%,} \DecValTok{25}\OperatorTok{\%}\NormalTok{))}
\end{Highlighting}
\end{Shaded}

\pandocbounded{\includegraphics[keepaspectratio]{https://raw.githubusercontent.com/Pegacraft/typst-plotting/8d834689359b708ce75fe51be05eed45570e463e/images/scatter.png}}

\subsubsection{Graph charts}\label{graph-charts}

\begin{Shaded}
\begin{Highlighting}[]
\CommentTok{// The data to be displayed}
\KeywordTok{let}\NormalTok{ data }\OperatorTok{=}\NormalTok{ (}
\NormalTok{  (}\DecValTok{0}\OperatorTok{,} \DecValTok{0}\NormalTok{)}\OperatorTok{,}\NormalTok{ (}\DecValTok{2}\OperatorTok{,} \DecValTok{2}\NormalTok{)}\OperatorTok{,}\NormalTok{ (}\DecValTok{3}\OperatorTok{,} \DecValTok{0}\NormalTok{)}\OperatorTok{,}\NormalTok{ (}\DecValTok{4}\OperatorTok{,} \DecValTok{4}\NormalTok{)}\OperatorTok{,}\NormalTok{ (}\DecValTok{5}\OperatorTok{,} \DecValTok{7}\NormalTok{)}\OperatorTok{,}\NormalTok{ (}\DecValTok{6}\OperatorTok{,} \DecValTok{6}\NormalTok{)}\OperatorTok{,}\NormalTok{ (}\DecValTok{7}\OperatorTok{,} \DecValTok{9}\NormalTok{)}\OperatorTok{,}\NormalTok{ (}\DecValTok{8}\OperatorTok{,} \DecValTok{5}\NormalTok{)}\OperatorTok{,}\NormalTok{ (}\DecValTok{9}\OperatorTok{,} \DecValTok{9}\NormalTok{)}\OperatorTok{,}\NormalTok{ (}\DecValTok{10}\OperatorTok{,} \DecValTok{1}\NormalTok{)}
\NormalTok{)}

\CommentTok{// Create the axes used for the chart }
\KeywordTok{let}\NormalTok{ x\_axis }\OperatorTok{=} \FunctionTok{axis}\NormalTok{(min}\OperatorTok{:} \DecValTok{0}\OperatorTok{,}\NormalTok{ max}\OperatorTok{:} \DecValTok{11}\OperatorTok{,}\NormalTok{ step}\OperatorTok{:} \DecValTok{2}\OperatorTok{,}\NormalTok{ location}\OperatorTok{:} \StringTok{"bottom"}\NormalTok{)}
\KeywordTok{let}\NormalTok{ y\_axis }\OperatorTok{=} \FunctionTok{axis}\NormalTok{(min}\OperatorTok{:} \DecValTok{0}\OperatorTok{,}\NormalTok{ max}\OperatorTok{:} \DecValTok{11}\OperatorTok{,}\NormalTok{ step}\OperatorTok{:} \DecValTok{2}\OperatorTok{,}\NormalTok{ location}\OperatorTok{:} \StringTok{"left"}\OperatorTok{,}\NormalTok{ helper\_lines}\OperatorTok{:} \KeywordTok{false}\NormalTok{)}

\CommentTok{// Combine the axes and the data and feed it to the plot render function.}
\KeywordTok{let}\NormalTok{ pl }\OperatorTok{=} \FunctionTok{plot}\NormalTok{(data}\OperatorTok{:}\NormalTok{ data}\OperatorTok{,}\NormalTok{ axes}\OperatorTok{:}\NormalTok{ (x\_axis}\OperatorTok{,}\NormalTok{ y\_axis))}
\FunctionTok{graph\_plot}\NormalTok{(pl}\OperatorTok{,}\NormalTok{ (}\DecValTok{100}\OperatorTok{\%,} \DecValTok{25}\OperatorTok{\%}\NormalTok{))}
\FunctionTok{graph\_plot}\NormalTok{(pl}\OperatorTok{,}\NormalTok{ (}\DecValTok{100}\OperatorTok{\%,} \DecValTok{25}\OperatorTok{\%}\NormalTok{)}\OperatorTok{,}\NormalTok{ rounding}\OperatorTok{:} \DecValTok{30}\OperatorTok{\%,}\NormalTok{ caption}\OperatorTok{:} \StringTok{"Graph Plot with caption and rounding"}\NormalTok{)}
\end{Highlighting}
\end{Shaded}

\pandocbounded{\includegraphics[keepaspectratio]{https://raw.githubusercontent.com/Pegacraft/typst-plotting/8d834689359b708ce75fe51be05eed45570e463e/images/graph.png}}

\subsubsection{Histograms}\label{histograms}

\begin{Shaded}
\begin{Highlighting}[]
\CommentTok{// Plot 1:}
\CommentTok{// The data to be displayed}
\KeywordTok{let}\NormalTok{ data }\OperatorTok{=}\NormalTok{ (}
  \DecValTok{18000}\OperatorTok{,} \DecValTok{18000}\OperatorTok{,} \DecValTok{18000}\OperatorTok{,} \DecValTok{18000}\OperatorTok{,} \DecValTok{18000}\OperatorTok{,} \DecValTok{18000}\OperatorTok{,} \DecValTok{18000}\OperatorTok{,} \DecValTok{18000}\OperatorTok{,}
  \DecValTok{18000}\OperatorTok{,} \DecValTok{18000}\OperatorTok{,} \DecValTok{28000}\OperatorTok{,} \DecValTok{28000}\OperatorTok{,} \DecValTok{28000}\OperatorTok{,} \DecValTok{28000}\OperatorTok{,} \DecValTok{28000}\OperatorTok{,} \DecValTok{28000}\OperatorTok{,}
  \DecValTok{28000}\OperatorTok{,} \DecValTok{28000}\OperatorTok{,} \DecValTok{28000}\OperatorTok{,} \DecValTok{28000}\OperatorTok{,} \DecValTok{28000}\OperatorTok{,} \DecValTok{28000}\OperatorTok{,} \DecValTok{28000}\OperatorTok{,} \DecValTok{28000}\OperatorTok{,}
  \DecValTok{28000}\OperatorTok{,} \DecValTok{28000}\OperatorTok{,} \DecValTok{28000}\OperatorTok{,} \DecValTok{28000}\OperatorTok{,} \DecValTok{28000}\OperatorTok{,} \DecValTok{28000}\OperatorTok{,} \DecValTok{28000}\OperatorTok{,} \DecValTok{28000}\OperatorTok{,}
  \DecValTok{35000}\OperatorTok{,} \DecValTok{46000}\OperatorTok{,} \DecValTok{75000}\OperatorTok{,} \DecValTok{95000}
\NormalTok{)}

\CommentTok{// Classify the data}
\KeywordTok{let}\NormalTok{ classes }\OperatorTok{=} \FunctionTok{class\_generator}\NormalTok{(}\DecValTok{10000}\OperatorTok{,} \DecValTok{50000}\OperatorTok{,} \DecValTok{4}\NormalTok{)}
\NormalTok{classes}\OperatorTok{.}\FunctionTok{push}\NormalTok{(}\KeywordTok{class}\NormalTok{(}\DecValTok{50000}\OperatorTok{,} \DecValTok{100000}\NormalTok{))}
\NormalTok{classes }\OperatorTok{=} \FunctionTok{classify}\NormalTok{(data}\OperatorTok{,}\NormalTok{ classes)}

\CommentTok{// Create the axes used for the chart }
\KeywordTok{let}\NormalTok{ x\_axis }\OperatorTok{=} \FunctionTok{axis}\NormalTok{(min}\OperatorTok{:} \DecValTok{0}\OperatorTok{,}\NormalTok{ max}\OperatorTok{:} \DecValTok{100000}\OperatorTok{,}\NormalTok{ step}\OperatorTok{:} \DecValTok{10000}\OperatorTok{,}\NormalTok{ location}\OperatorTok{:} \StringTok{"bottom"}\NormalTok{)}
\KeywordTok{let}\NormalTok{ y\_axis }\OperatorTok{=} \FunctionTok{axis}\NormalTok{(min}\OperatorTok{:} \DecValTok{0}\OperatorTok{,}\NormalTok{ max}\OperatorTok{:} \DecValTok{31}\OperatorTok{,}\NormalTok{ step}\OperatorTok{:} \DecValTok{5}\OperatorTok{,}\NormalTok{ location}\OperatorTok{:} \StringTok{"left"}\OperatorTok{,}\NormalTok{ helper\_lines}\OperatorTok{:} \KeywordTok{true}\NormalTok{)}

\CommentTok{// Combine the axes and the data and feed it to the plot render function.}
\KeywordTok{let}\NormalTok{ pl }\OperatorTok{=} \FunctionTok{plot}\NormalTok{(data}\OperatorTok{:}\NormalTok{ classes}\OperatorTok{,}\NormalTok{ axes}\OperatorTok{:}\NormalTok{ (x\_axis}\OperatorTok{,}\NormalTok{ y\_axis))}
\FunctionTok{histogram}\NormalTok{(pl}\OperatorTok{,}\NormalTok{ (}\DecValTok{100}\OperatorTok{\%,} \DecValTok{40}\OperatorTok{\%}\NormalTok{)}\OperatorTok{,}\NormalTok{ stroke}\OperatorTok{:}\NormalTok{ black}\OperatorTok{,}\NormalTok{ fill}\OperatorTok{:}\NormalTok{ (purple}\OperatorTok{,}\NormalTok{ blue}\OperatorTok{,}\NormalTok{ red}\OperatorTok{,}\NormalTok{ green}\OperatorTok{,}\NormalTok{ yellow))}

\CommentTok{// Plot 2:}
\CommentTok{// Create the different classes}
\KeywordTok{let}\NormalTok{ classes }\OperatorTok{=}\NormalTok{ ()}
\NormalTok{classes}\OperatorTok{.}\FunctionTok{push}\NormalTok{(}\KeywordTok{class}\NormalTok{(}\DecValTok{11}\OperatorTok{,} \DecValTok{13}\NormalTok{))}
\NormalTok{classes}\OperatorTok{.}\FunctionTok{push}\NormalTok{(}\KeywordTok{class}\NormalTok{(}\DecValTok{13}\OperatorTok{,} \DecValTok{15}\NormalTok{))}
\NormalTok{classes}\OperatorTok{.}\FunctionTok{push}\NormalTok{(}\KeywordTok{class}\NormalTok{(}\DecValTok{1}\OperatorTok{,} \DecValTok{6}\NormalTok{))}
\NormalTok{classes}\OperatorTok{.}\FunctionTok{push}\NormalTok{(}\KeywordTok{class}\NormalTok{(}\DecValTok{6}\OperatorTok{,} \DecValTok{11}\NormalTok{))}
\NormalTok{classes}\OperatorTok{.}\FunctionTok{push}\NormalTok{(}\KeywordTok{class}\NormalTok{(}\DecValTok{15}\OperatorTok{,} \DecValTok{30}\NormalTok{))}

\CommentTok{// Define the data to map}
\KeywordTok{let}\NormalTok{ data }\OperatorTok{=}\NormalTok{ ((}\DecValTok{20}\OperatorTok{,} \DecValTok{2}\NormalTok{)}\OperatorTok{,}\NormalTok{ (}\DecValTok{30}\OperatorTok{,} \DecValTok{7}\NormalTok{)}\OperatorTok{,}\NormalTok{ (}\DecValTok{16}\OperatorTok{,} \DecValTok{12}\NormalTok{)}\OperatorTok{,}\NormalTok{ (}\DecValTok{40}\OperatorTok{,} \DecValTok{13}\NormalTok{)}\OperatorTok{,}\NormalTok{ (}\DecValTok{5}\OperatorTok{,} \DecValTok{17}\NormalTok{))}

\CommentTok{// Create the axes}
\KeywordTok{let}\NormalTok{ x\_axis }\OperatorTok{=} \FunctionTok{axis}\NormalTok{(min}\OperatorTok{:} \DecValTok{0}\OperatorTok{,}\NormalTok{ max}\OperatorTok{:} \DecValTok{31}\OperatorTok{,}\NormalTok{ step}\OperatorTok{:} \DecValTok{1}\OperatorTok{,}\NormalTok{ location}\OperatorTok{:} \StringTok{"bottom"}\OperatorTok{,}\NormalTok{ show\_markings}\OperatorTok{:} \KeywordTok{false}\NormalTok{)}
\KeywordTok{let}\NormalTok{ y\_axis }\OperatorTok{=} \FunctionTok{axis}\NormalTok{(min}\OperatorTok{:} \DecValTok{0}\OperatorTok{,}\NormalTok{ max}\OperatorTok{:} \DecValTok{41}\OperatorTok{,}\NormalTok{ step}\OperatorTok{:} \DecValTok{5}\OperatorTok{,}\NormalTok{ location}\OperatorTok{:} \StringTok{"left"}\OperatorTok{,}\NormalTok{ helper\_lines}\OperatorTok{:} \KeywordTok{true}\NormalTok{)}

\CommentTok{// Classify the data}
\NormalTok{classes }\OperatorTok{=} \FunctionTok{classify}\NormalTok{(data}\OperatorTok{,}\NormalTok{ classes)}

\CommentTok{// Combine the axes and the data and feed it to the plot render function.}
\KeywordTok{let}\NormalTok{ pl }\OperatorTok{=} \FunctionTok{plot}\NormalTok{(axes}\OperatorTok{:}\NormalTok{ (x\_axis}\OperatorTok{,}\NormalTok{ y\_axis)}\OperatorTok{,}\NormalTok{ data}\OperatorTok{:}\NormalTok{ classes)}
\FunctionTok{histogram}\NormalTok{(pl}\OperatorTok{,}\NormalTok{ (}\DecValTok{100}\OperatorTok{\%,} \DecValTok{40}\OperatorTok{\%}\NormalTok{))}
\end{Highlighting}
\end{Shaded}

\pandocbounded{\includegraphics[keepaspectratio]{https://raw.githubusercontent.com/Pegacraft/typst-plotting/8d834689359b708ce75fe51be05eed45570e463e/images/histogram.png}}

\subsubsection{Bar charts}\label{bar-charts}

\begin{Shaded}
\begin{Highlighting}[]
\CommentTok{// Plot 1:}
\CommentTok{// The data to be displayed}
\KeywordTok{let}\NormalTok{ data }\OperatorTok{=}\NormalTok{ ((}\DecValTok{10}\OperatorTok{,} \StringTok{"Monday"}\NormalTok{)}\OperatorTok{,}\NormalTok{ (}\DecValTok{5}\OperatorTok{,} \StringTok{"Tuesday"}\NormalTok{)}\OperatorTok{,}\NormalTok{ (}\DecValTok{15}\OperatorTok{,} \StringTok{"Wednesday"}\NormalTok{)}\OperatorTok{,}\NormalTok{ (}\DecValTok{9}\OperatorTok{,} \StringTok{"Thursday"}\NormalTok{)}\OperatorTok{,}\NormalTok{ (}\DecValTok{11}\OperatorTok{,} \StringTok{"Friday"}\NormalTok{))}

\CommentTok{// Create the necessary axes}
\KeywordTok{let}\NormalTok{ y\_axis }\OperatorTok{=} \FunctionTok{axis}\NormalTok{(values}\OperatorTok{:}\NormalTok{ (}\StringTok{""}\OperatorTok{,} \StringTok{"Monday"}\OperatorTok{,} \StringTok{"Tuesday"}\OperatorTok{,} \StringTok{"Wednesday"}\OperatorTok{,} \StringTok{"Thursday"}\OperatorTok{,} \StringTok{"Friday"}\NormalTok{)}\OperatorTok{,}\NormalTok{ location}\OperatorTok{:} \StringTok{"left"}\OperatorTok{,}\NormalTok{ show\_markings}\OperatorTok{:} \KeywordTok{true}\NormalTok{)}
\KeywordTok{let}\NormalTok{ x\_axis }\OperatorTok{=} \FunctionTok{axis}\NormalTok{(min}\OperatorTok{:} \DecValTok{0}\OperatorTok{,}\NormalTok{ max}\OperatorTok{:} \DecValTok{20}\OperatorTok{,}\NormalTok{ step}\OperatorTok{:} \DecValTok{2}\OperatorTok{,}\NormalTok{ location}\OperatorTok{:} \StringTok{"bottom"}\OperatorTok{,}\NormalTok{ helper\_lines}\OperatorTok{:} \KeywordTok{true}\NormalTok{)}

\CommentTok{// Combine the axes and the data and feed it to the plot render function.}
\KeywordTok{let}\NormalTok{ pl }\OperatorTok{=} \FunctionTok{plot}\NormalTok{(axes}\OperatorTok{:}\NormalTok{ (x\_axis}\OperatorTok{,}\NormalTok{ y\_axis)}\OperatorTok{,}\NormalTok{ data}\OperatorTok{:}\NormalTok{ data)}
\FunctionTok{bar\_chart}\NormalTok{(pl}\OperatorTok{,}\NormalTok{ (}\DecValTok{100}\OperatorTok{\%,} \DecValTok{33}\OperatorTok{\%}\NormalTok{)}\OperatorTok{,}\NormalTok{ fill}\OperatorTok{:}\NormalTok{ (purple}\OperatorTok{,}\NormalTok{ blue}\OperatorTok{,}\NormalTok{ red}\OperatorTok{,}\NormalTok{ green}\OperatorTok{,}\NormalTok{ yellow)}\OperatorTok{,}\NormalTok{ bar\_width}\OperatorTok{:} \DecValTok{70}\OperatorTok{\%,}\NormalTok{ rotated}\OperatorTok{:} \KeywordTok{true}\NormalTok{)}

\CommentTok{// Plot 2:}
\CommentTok{// Same as above, but with numbers as data}
\KeywordTok{let}\NormalTok{ data\_2 }\OperatorTok{=}\NormalTok{ ((}\DecValTok{20}\OperatorTok{,} \DecValTok{2}\NormalTok{)}\OperatorTok{,}\NormalTok{ (}\DecValTok{30}\OperatorTok{,} \DecValTok{7}\NormalTok{)}\OperatorTok{,}\NormalTok{ (}\DecValTok{16}\OperatorTok{,} \DecValTok{12}\NormalTok{)}\OperatorTok{,}\NormalTok{ (}\DecValTok{40}\OperatorTok{,} \DecValTok{13}\NormalTok{)}\OperatorTok{,}\NormalTok{ (}\DecValTok{5}\OperatorTok{,} \DecValTok{17}\NormalTok{))}
\KeywordTok{let}\NormalTok{ y\_axis\_2 }\OperatorTok{=} \FunctionTok{axis}\NormalTok{(min}\OperatorTok{:} \DecValTok{0}\OperatorTok{,}\NormalTok{ max}\OperatorTok{:} \DecValTok{41}\OperatorTok{,}\NormalTok{ step}\OperatorTok{:} \DecValTok{5}\OperatorTok{,}\NormalTok{ location}\OperatorTok{:} \StringTok{"left"}\OperatorTok{,}\NormalTok{ show\_markings}\OperatorTok{:} \KeywordTok{true}\OperatorTok{,}\NormalTok{ helper\_lines}\OperatorTok{:} \KeywordTok{true}\NormalTok{)}
\KeywordTok{let}\NormalTok{ x\_axis\_2 }\OperatorTok{=} \FunctionTok{axis}\NormalTok{(min}\OperatorTok{:} \DecValTok{0}\OperatorTok{,}\NormalTok{ max}\OperatorTok{:} \DecValTok{21}\OperatorTok{,}\NormalTok{ step}\OperatorTok{:} \DecValTok{1}\OperatorTok{,}\NormalTok{ location}\OperatorTok{:} \StringTok{"bottom"}\NormalTok{)}
\KeywordTok{let}\NormalTok{ pl\_2 }\OperatorTok{=} \FunctionTok{plot}\NormalTok{(axes}\OperatorTok{:}\NormalTok{ (x\_axis\_2}\OperatorTok{,}\NormalTok{ y\_axis\_2)}\OperatorTok{,}\NormalTok{ data}\OperatorTok{:}\NormalTok{ data\_2)}
\FunctionTok{bar\_chart}\NormalTok{(pl\_2}\OperatorTok{,}\NormalTok{ (}\DecValTok{100}\OperatorTok{\%,} \DecValTok{60}\OperatorTok{\%}\NormalTok{)}\OperatorTok{,}\NormalTok{ bar\_width}\OperatorTok{:} \DecValTok{100}\OperatorTok{\%}\NormalTok{)}
\end{Highlighting}
\end{Shaded}

\pandocbounded{\includegraphics[keepaspectratio]{https://raw.githubusercontent.com/Pegacraft/typst-plotting/8d834689359b708ce75fe51be05eed45570e463e/images/bar.png}}

\subsubsection{Pie charts}\label{pie-charts}

\begin{Shaded}
\begin{Highlighting}[]
\NormalTok{show}\OperatorTok{:}\NormalTok{ r }\KeywordTok{=\textgreater{}} \FunctionTok{columns}\NormalTok{(}\DecValTok{2}\OperatorTok{,}\NormalTok{ r)}

\CommentTok{// create the sample data}
\KeywordTok{let}\NormalTok{ data }\OperatorTok{=}\NormalTok{ ((}\DecValTok{10}\OperatorTok{,} \StringTok{"Male"}\NormalTok{)}\OperatorTok{,}\NormalTok{ (}\DecValTok{20}\OperatorTok{,} \StringTok{"Female"}\NormalTok{)}\OperatorTok{,}\NormalTok{ (}\DecValTok{15}\OperatorTok{,} \StringTok{"Divers"}\NormalTok{)}\OperatorTok{,}\NormalTok{ (}\DecValTok{2}\OperatorTok{,} \StringTok{"Other"}\NormalTok{)}

\CommentTok{// Skip the axis step, as no axes are needed}

\CommentTok{// Put the data into a plot }
\KeywordTok{let}\NormalTok{ p }\OperatorTok{=} \FunctionTok{plot}\NormalTok{(data}\OperatorTok{:}\NormalTok{ data)}

\CommentTok{// Display the pie\_charts in all different display ways}
\FunctionTok{pie\_chart}\NormalTok{(p}\OperatorTok{,}\NormalTok{ (}\DecValTok{100}\OperatorTok{\%,} \DecValTok{20}\OperatorTok{\%}\NormalTok{)}\OperatorTok{,}\NormalTok{ display\_style}\OperatorTok{:} \StringTok{"legend{-}inside{-}chart"}\NormalTok{)}
\FunctionTok{pie\_chart}\NormalTok{(p}\OperatorTok{,}\NormalTok{ (}\DecValTok{100}\OperatorTok{\%,} \DecValTok{20}\OperatorTok{\%}\NormalTok{)}\OperatorTok{,}\NormalTok{ display\_style}\OperatorTok{:} \StringTok{"hor{-}chart{-}legend"}\NormalTok{)}
\FunctionTok{pie\_chart}\NormalTok{(p}\OperatorTok{,}\NormalTok{ (}\DecValTok{100}\OperatorTok{\%,} \DecValTok{20}\OperatorTok{\%}\NormalTok{)}\OperatorTok{,}\NormalTok{ display\_style}\OperatorTok{:} \StringTok{"hor{-}legend{-}chart"}\NormalTok{)}
\FunctionTok{pie\_chart}\NormalTok{(p}\OperatorTok{,}\NormalTok{ (}\DecValTok{100}\OperatorTok{\%,} \DecValTok{20}\OperatorTok{\%}\NormalTok{)}\OperatorTok{,}\NormalTok{ display\_style}\OperatorTok{:} \StringTok{"vert{-}chart{-}legend"}\NormalTok{)}
\FunctionTok{pie\_chart}\NormalTok{(p}\OperatorTok{,}\NormalTok{ (}\DecValTok{100}\OperatorTok{\%,} \DecValTok{20}\OperatorTok{\%}\NormalTok{)}\OperatorTok{,}\NormalTok{ display\_style}\OperatorTok{:} \StringTok{"vert{-}legend{-}chart"}\NormalTok{)}
\end{Highlighting}
\end{Shaded}

\pandocbounded{\includegraphics[keepaspectratio]{https://raw.githubusercontent.com/Pegacraft/typst-plotting/8d834689359b708ce75fe51be05eed45570e463e/images/pie.png}}

\textbf{Overlayed Graphs}

\begin{Shaded}
\begin{Highlighting}[]
\CommentTok{// Create the data for the two plots to overlay}
\KeywordTok{let}\NormalTok{ data\_scatter }\OperatorTok{=}\NormalTok{ (}
\NormalTok{  (}\DecValTok{0}\OperatorTok{,} \DecValTok{0}\NormalTok{)}\OperatorTok{,}\NormalTok{ (}\DecValTok{2}\OperatorTok{,} \DecValTok{2}\NormalTok{)}\OperatorTok{,}\NormalTok{ (}\DecValTok{3}\OperatorTok{,} \DecValTok{0}\NormalTok{)}\OperatorTok{,}\NormalTok{ (}\DecValTok{4}\OperatorTok{,} \DecValTok{4}\NormalTok{)}\OperatorTok{,}\NormalTok{ (}\DecValTok{5}\OperatorTok{,} \DecValTok{7}\NormalTok{)}\OperatorTok{,}\NormalTok{ (}\DecValTok{6}\OperatorTok{,} \DecValTok{6}\NormalTok{)}\OperatorTok{,}\NormalTok{ (}\DecValTok{7}\OperatorTok{,} \DecValTok{9}\NormalTok{)}\OperatorTok{,}\NormalTok{ (}\DecValTok{8}\OperatorTok{,} \DecValTok{5}\NormalTok{)}\OperatorTok{,}\NormalTok{ (}\DecValTok{9}\OperatorTok{,} \DecValTok{9}\NormalTok{)}\OperatorTok{,}\NormalTok{ (}\DecValTok{10}\OperatorTok{,} \DecValTok{1}\NormalTok{)}
\NormalTok{)}
\KeywordTok{let}\NormalTok{ data\_graph }\OperatorTok{=}\NormalTok{ (}
\NormalTok{  (}\DecValTok{0}\OperatorTok{,} \DecValTok{3}\NormalTok{)}\OperatorTok{,}\NormalTok{ (}\DecValTok{1}\OperatorTok{,} \DecValTok{5}\NormalTok{)}\OperatorTok{,}\NormalTok{ (}\DecValTok{2}\OperatorTok{,} \DecValTok{1}\NormalTok{)}\OperatorTok{,}\NormalTok{ (}\DecValTok{3}\OperatorTok{,} \DecValTok{7}\NormalTok{)}\OperatorTok{,}\NormalTok{ (}\DecValTok{4}\OperatorTok{,} \DecValTok{3}\NormalTok{)}\OperatorTok{,}\NormalTok{ (}\DecValTok{5}\OperatorTok{,} \DecValTok{5}\NormalTok{)}\OperatorTok{,}\NormalTok{ (}\DecValTok{6}\OperatorTok{,} \DecValTok{7}\NormalTok{)}\OperatorTok{,}\NormalTok{(}\DecValTok{7}\OperatorTok{,} \DecValTok{4}\NormalTok{)}\OperatorTok{,}\NormalTok{(}\DecValTok{11}\OperatorTok{,} \DecValTok{6}\NormalTok{)}
\NormalTok{)}

\CommentTok{// Create the axes for the overlay plot}
\KeywordTok{let}\NormalTok{ x\_axis }\OperatorTok{=} \FunctionTok{axis}\NormalTok{(min}\OperatorTok{:} \DecValTok{0}\OperatorTok{,}\NormalTok{ max}\OperatorTok{:} \DecValTok{11}\OperatorTok{,}\NormalTok{ step}\OperatorTok{:} \DecValTok{2}\OperatorTok{,}\NormalTok{ location}\OperatorTok{:} \StringTok{"bottom"}\NormalTok{)}
\KeywordTok{let}\NormalTok{ y\_axis }\OperatorTok{=} \FunctionTok{axis}\NormalTok{(min}\OperatorTok{:} \DecValTok{0}\OperatorTok{,}\NormalTok{ max}\OperatorTok{:} \DecValTok{11}\OperatorTok{,}\NormalTok{ step}\OperatorTok{:} \DecValTok{2}\OperatorTok{,}\NormalTok{ location}\OperatorTok{:} \StringTok{"left"}\OperatorTok{,}\NormalTok{ helper\_lines}\OperatorTok{:} \KeywordTok{false}\NormalTok{)}

\CommentTok{// create a plot for each individual plot type and save the render call}
\KeywordTok{let}\NormalTok{ pl\_scatter }\OperatorTok{=} \FunctionTok{plot}\NormalTok{(data}\OperatorTok{:}\NormalTok{ data\_scatter}\OperatorTok{,}\NormalTok{ axes}\OperatorTok{:}\NormalTok{ (x\_axis}\OperatorTok{,}\NormalTok{ y\_axis))}
\KeywordTok{let}\NormalTok{ scatter\_display }\OperatorTok{=} \FunctionTok{scatter\_plot}\NormalTok{(pl\_scatter}\OperatorTok{,}\NormalTok{ (}\DecValTok{100}\OperatorTok{\%,} \DecValTok{25}\OperatorTok{\%}\NormalTok{)}\OperatorTok{,}\NormalTok{ stroke}\OperatorTok{:}\NormalTok{ red)}
\KeywordTok{let}\NormalTok{ pl\_graph }\OperatorTok{=} \FunctionTok{plot}\NormalTok{(data}\OperatorTok{:}\NormalTok{ data\_graph}\OperatorTok{,}\NormalTok{ axes}\OperatorTok{:}\NormalTok{ (x\_axis}\OperatorTok{,}\NormalTok{ y\_axis))}
\KeywordTok{let}\NormalTok{ graph\_display }\OperatorTok{=} \FunctionTok{graph\_plot}\NormalTok{(pl\_graph}\OperatorTok{,}\NormalTok{ (}\DecValTok{100}\OperatorTok{\%,} \DecValTok{25}\OperatorTok{\%}\NormalTok{)}\OperatorTok{,}\NormalTok{ stroke}\OperatorTok{:}\NormalTok{ blue)}

\CommentTok{// overlay the plots using the overlay function}
\FunctionTok{overlay}\NormalTok{((scatter\_display}\OperatorTok{,}\NormalTok{ graph\_display)}\OperatorTok{,}\NormalTok{ (}\DecValTok{100}\OperatorTok{\%,} \DecValTok{25}\OperatorTok{\%}\NormalTok{))}
\end{Highlighting}
\end{Shaded}

\pandocbounded{\includegraphics[keepaspectratio]{https://raw.githubusercontent.com/Pegacraft/typst-plotting/8d834689359b708ce75fe51be05eed45570e463e/images/overlay.png}}

\subsubsection{How to add}\label{how-to-add}

Copy this into your project and use the import as \texttt{\ plotst\ }

\begin{verbatim}
#import "@preview/plotst:0.2.0"
\end{verbatim}

\includesvg[width=0.16667in,height=0.16667in]{/assets/icons/16-copy.svg}

Check the docs for
\href{https://typst.app/docs/reference/scripting/\#packages}{more
information on how to import packages} .

\subsubsection{About}\label{about}

\begin{description}
\tightlist
\item[Author s :]
Pegacraft \& Gewi413
\item[License:]
MIT
\item[Current version:]
0.2.0
\item[Last updated:]
October 28, 2023
\item[First released:]
July 2, 2023
\item[Archive size:]
15.2 kB
\href{https://packages.typst.org/preview/plotst-0.2.0.tar.gz}{\pandocbounded{\includesvg[keepaspectratio]{/assets/icons/16-download.svg}}}
\item[Repository:]
\href{https://github.com/Pegacraft/typst-plotting}{GitHub}
\end{description}

\subsubsection{Where to report issues?}\label{where-to-report-issues}

This package is a project of Pegacraft and Gewi413 . Report issues on
\href{https://github.com/Pegacraft/typst-plotting}{their repository} .
You can also try to ask for help with this package on the
\href{https://forum.typst.app}{Forum} .

Please report this package to the Typst team using the
\href{https://typst.app/contact}{contact form} if you believe it is a
safety hazard or infringes upon your rights.

\phantomsection\label{versions}
\subsubsection{Version history}\label{version-history}

\begin{longtable}[]{@{}ll@{}}
\toprule\noalign{}
Version & Release Date \\
\midrule\noalign{}
\endhead
\bottomrule\noalign{}
\endlastfoot
0.2.0 & October 28, 2023 \\
\href{https://typst.app/universe/package/plotst/0.1.0/}{0.1.0} & July 2,
2023 \\
\end{longtable}

Typst GmbH did not create this package and cannot guarantee correct
functionality of this package or compatibility with any version of the
Typst compiler or app.
