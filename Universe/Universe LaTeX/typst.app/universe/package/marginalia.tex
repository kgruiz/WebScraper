\title{typst.app/universe/package/marginalia}

\phantomsection\label{banner}
\section{marginalia}\label{marginalia}

{ 0.1.1 }

Configurable margin-notes and matching wide blocks.

\phantomsection\label{readme}
\subsection{Setup}\label{setup}

Put something akin to the following at the start of your
\texttt{\ .typ\ } file:

\begin{Shaded}
\begin{Highlighting}[]
\NormalTok{\#import "@preview/marginalia:0.1.1" as marginalia: note, wideblock}
\NormalTok{\#let config = (}
\NormalTok{  // inner: ( far: 5mm, width: 15mm, sep: 5mm ),}
\NormalTok{  // outer: ( far: 5mm, width: 15mm, sep: 5mm ),}
\NormalTok{  // top: 2.5cm,}
\NormalTok{  // bottom: 2.5cm,}
\NormalTok{  // book: false,}
\NormalTok{  // clearance: 8pt,}
\NormalTok{  // flush{-}numbers: false,}
\NormalTok{  // numbering: /* numbering{-}function */,}
\NormalTok{)}
\NormalTok{\#marginalia.configure(..config)}
\NormalTok{\#set page(}
\NormalTok{  // setup margins:}
\NormalTok{  ..marginalia.page{-}setup(..config),}
\NormalTok{  /* other page setup */}
\NormalTok{)}
\end{Highlighting}
\end{Shaded}

If \texttt{\ book\ } is \texttt{\ false\ } , \texttt{\ inner\ } and
\texttt{\ outer\ } correspond to the left and right margins
respectively. If book is true, the margins swap sides on even and odd
pages. Notes are placed in the outside margin by default.

Where you can then customize \texttt{\ config\ } to your preferences.
Shown here (as comments) are the default values taken if the
corresponding keys are unset.
\href{https://github.com/nleanba/typst-marginalia/blob/v0.1.1/Marginalia.pdf}{Please
refer to the PDF documentation for more details on the configuration and
the provided commands.}

\subsection{Margin-Notes}\label{margin-notes}

Provided via the \texttt{\ \#note{[}...{]}\ } command.

\begin{itemize}
\tightlist
\item
  \texttt{\ \#note(reverse:\ true){[}...{]}\ } to put it on the inside
  margin.
\item
  \texttt{\ \#note(numbered:\ false){[}...{]}\ } to remove the marker.
\end{itemize}

Note: it is recommended to reset the counter for the markers regularly,
e.g. by putting \texttt{\ marginalia.notecounter.update(0)\ } into the
code for your header.

\subsection{Wide Blocks}\label{wide-blocks}

Provided via the \texttt{\ \#wideblock{[}...{]}\ } command.

\begin{itemize}
\tightlist
\item
  \texttt{\ \#wideblock(reverse:\ true){[}...{]}\ } to extend into the
  inside margin instead.
\item
  \texttt{\ \#wideblock(double:\ true){[}...{]}\ } to extend into both
  margins.
\end{itemize}

Note: \texttt{\ reverse\ } and \texttt{\ double\ } are mutually
exclusive.

Note: Wideblocks do not handle pagebreaks in \texttt{\ book:\ true\ }
documents well.

\subsection{Figures}\label{figures}

You can use figures as normal, also within wideblocks. To get captions
on the side, use

\begin{Shaded}
\begin{Highlighting}[]
\NormalTok{\#set figure(gap: 0pt)}
\NormalTok{\#set figure.caption(position: top)}
\NormalTok{\#show figure.caption.where(position: top): note.with(numbered: false, dy: 1em)}
\end{Highlighting}
\end{Shaded}

For small figures, the package also provides a \texttt{\ notefigure\ }
command which places the figure in the margin.

\begin{Shaded}
\begin{Highlighting}[]
\NormalTok{\#marginalia.notefigure(}
\NormalTok{  rect(width: 100\%),}
\NormalTok{  label: \textless{}aaa\textgreater{},}
\NormalTok{  caption: [A notefigure.],}
\NormalTok{)}
\end{Highlighting}
\end{Shaded}

\begin{center}\rule{0.5\linewidth}{0.5pt}\end{center}

\href{https://github.com/nleanba/typst-marginalia/blob/v0.1.1/Marginalia.pdf}{\pandocbounded{\includesvg[keepaspectratio]{https://raw.githubusercontent.com/nleanba/typst-marginalia/refs/tags/v0.1.1/preview.svg}}}

\subsubsection{How to add}\label{how-to-add}

Copy this into your project and use the import as
\texttt{\ marginalia\ }

\begin{verbatim}
#import "@preview/marginalia:0.1.1"
\end{verbatim}

\includesvg[width=0.16667in,height=0.16667in]{/assets/icons/16-copy.svg}

Check the docs for
\href{https://typst.app/docs/reference/scripting/\#packages}{more
information on how to import packages} .

\subsubsection{About}\label{about}

\begin{description}
\tightlist
\item[Author :]
\href{https://github.com/nleanba}{nleanba}
\item[License:]
Unlicense
\item[Current version:]
0.1.1
\item[Last updated:]
November 25, 2024
\item[First released:]
November 19, 2024
\item[Minimum Typst version:]
0.12.0
\item[Archive size:]
6.17 kB
\href{https://packages.typst.org/preview/marginalia-0.1.1.tar.gz}{\pandocbounded{\includesvg[keepaspectratio]{/assets/icons/16-download.svg}}}
\item[Repository:]
\href{https://github.com/nleanba/typst-marginalia}{GitHub}
\item[Categor ies :]
\begin{itemize}
\tightlist
\item[]
\item
  \pandocbounded{\includesvg[keepaspectratio]{/assets/icons/16-layout.svg}}
  \href{https://typst.app/universe/search/?category=layout}{Layout}
\item
  \pandocbounded{\includesvg[keepaspectratio]{/assets/icons/16-hammer.svg}}
  \href{https://typst.app/universe/search/?category=utility}{Utility}
\end{itemize}
\end{description}

\subsubsection{Where to report issues?}\label{where-to-report-issues}

This package is a project of nleanba . Report issues on
\href{https://github.com/nleanba/typst-marginalia}{their repository} .
You can also try to ask for help with this package on the
\href{https://forum.typst.app}{Forum} .

Please report this package to the Typst team using the
\href{https://typst.app/contact}{contact form} if you believe it is a
safety hazard or infringes upon your rights.

\phantomsection\label{versions}
\subsubsection{Version history}\label{version-history}

\begin{longtable}[]{@{}ll@{}}
\toprule\noalign{}
Version & Release Date \\
\midrule\noalign{}
\endhead
\bottomrule\noalign{}
\endlastfoot
0.1.1 & November 25, 2024 \\
\href{https://typst.app/universe/package/marginalia/0.1.0/}{0.1.0} &
November 19, 2024 \\
\end{longtable}

Typst GmbH did not create this package and cannot guarantee correct
functionality of this package or compatibility with any version of the
Typst compiler or app.
