\title{typst.app/universe/package/t4t}

\phantomsection\label{banner}
\section{t4t}\label{t4t}

{ 0.3.2 }

A utility package for typst package authors

\phantomsection\label{readme}
\begin{quote}
A utility package for typst package authors.
\end{quote}

\textbf{Tools for Typst} ( \texttt{\ t4t\ } in short) is a utility
package for
\href{https://github.com/typst/packages/raw/main/packages/preview/t4t/0.3.2/typst/typst}{Typst}
package and template authors. It provides solutions to some recurring
tasks in package development.

The package can be imported or any useful parts of it copied into a
project. It is perfectly fine to treat \texttt{\ t4t\ } as a snippet
collection and to pick and choose only some useful functions. For this
reason, most functions are implemented without further dependencies.

Hopefully, this collection will grow over time with \emph{Typst} to
provide solutions for common problems.

\subsection{Usage}\label{usage}

Either import the package from the Typst preview repository:

\begin{Shaded}
\begin{Highlighting}[]
\NormalTok{\#import }\StringTok{"@preview/t4t:0.3.2"}\OperatorTok{:} \OperatorTok{*}
\end{Highlighting}
\end{Shaded}

If only a few functions from \texttt{\ t4t\ } are needed, simply copy
the necessary code to the beginning of the document.

\subsection{Reference}\label{reference}

\begin{center}\rule{0.5\linewidth}{0.5pt}\end{center}

\textbf{Note:} This reference might be out of date. Please refer to the
manual for a complete overview of all functions.

\begin{center}\rule{0.5\linewidth}{0.5pt}\end{center}

The functions are categorized into different submodules that can be
imported separately.

The modules are:

\begin{itemize}
\tightlist
\item
  \texttt{\ is\ }
\item
  \texttt{\ def\ }
\item
  \texttt{\ assert\ }
\item
  \texttt{\ alias\ }
\item
  \texttt{\ math\ }
\item
  \texttt{\ get\ }
\end{itemize}

Any or all modules can be imported the usual way:

\begin{Shaded}
\begin{Highlighting}[]
\CommentTok{// Import as "t4t"}
\NormalTok{\#import }\StringTok{"@preview/t4t:0.3.2"}
\CommentTok{// Import all modules}
\NormalTok{\#import }\StringTok{"@preview/t4t:0.3.2"}\OperatorTok{:} \OperatorTok{*}
\CommentTok{// Import specific modules}
\NormalTok{\#import }\StringTok{"@preview/t4t:0.3.2"}\OperatorTok{:}\NormalTok{ is}\OperatorTok{,}\NormalTok{ def}
\end{Highlighting}
\end{Shaded}

In general, the main value is passed last to the utility functions.
\texttt{\ \#def.if-none()\ } , for example, takes the default value
first and the value to test second. This is somewhat counterintuitive at
first, but allows the use of \texttt{\ .with()\ } to generate derivative
functions:

\begin{Shaded}
\begin{Highlighting}[]
\NormalTok{\#let is}\OperatorTok{{-}}\NormalTok{foo }\OperatorTok{=}\NormalTok{ eq}\OperatorTok{.}\FunctionTok{with}\NormalTok{(}\StringTok{"foo"}\NormalTok{)}
\end{Highlighting}
\end{Shaded}

\subsubsection{Test functions}\label{test-functions}

\begin{Shaded}
\begin{Highlighting}[]
\NormalTok{\#import }\StringTok{"@preview/t4t:0.3.2"}\OperatorTok{:}\NormalTok{ is}
\end{Highlighting}
\end{Shaded}

These functions provide shortcuts to common tests like
\texttt{\ \#is.eq()\ } . Some of these are not shorter as writing pure
Typst code (e.g. \texttt{\ a\ ==\ b\ } ), but can easily be used in
\texttt{\ .any()\ } or \texttt{\ .find()\ } calls:

\begin{Shaded}
\begin{Highlighting}[]
\CommentTok{// check all values for none}
\ControlFlowTok{if}\NormalTok{ some}\OperatorTok{{-}}\NormalTok{array}\OperatorTok{.}\FunctionTok{any}\NormalTok{(is}\OperatorTok{{-}}\NormalTok{none) \{}
    \OperatorTok{...}
\NormalTok{\}}

\CommentTok{// find first not none value}
\KeywordTok{let}\NormalTok{ x }\OperatorTok{=}\NormalTok{ (none}\OperatorTok{,}\NormalTok{ none}\OperatorTok{,} \DecValTok{5}\OperatorTok{,}\NormalTok{ none)}\OperatorTok{.}\FunctionTok{find}\NormalTok{(is}\OperatorTok{.}\AttributeTok{not}\OperatorTok{{-}}\NormalTok{none)}

\CommentTok{// find position of a value}
\KeywordTok{let}\NormalTok{ pos}\OperatorTok{{-}}\NormalTok{bar }\OperatorTok{=}\NormalTok{ args}\OperatorTok{.}\FunctionTok{pos}\NormalTok{()}\OperatorTok{.}\FunctionTok{position}\NormalTok{(is}\OperatorTok{.}\AttributeTok{eq}\OperatorTok{.}\FunctionTok{with}\NormalTok{(}\StringTok{"|"}\NormalTok{))}
\end{Highlighting}
\end{Shaded}

There are two exceptions: \texttt{\ is-none\ } and \texttt{\ is-auto\ }
. Since keywords can’t be used as function names, the \texttt{\ is\ }
module can’t define a function to do \texttt{\ is.none()\ } .
Therefore the methods \texttt{\ is-none\ } and \texttt{\ is-auto\ } are
provided in the base module of \texttt{\ t4t\ } :

\begin{Shaded}
\begin{Highlighting}[]
\NormalTok{\#import }\StringTok{"@preview/t4t:0.3.2"}\OperatorTok{:}\NormalTok{ is}\OperatorTok{{-}}\NormalTok{none}\OperatorTok{,}\NormalTok{ is}\OperatorTok{{-}}\NormalTok{auto}
\end{Highlighting}
\end{Shaded}

The \texttt{\ is\ } submodule still has these tests, but under different
names ( \texttt{\ is.n\ } and \texttt{\ is.non\ } for \texttt{\ none\ }
and \texttt{\ is.a\ } and \texttt{\ is.aut\ } for \texttt{\ auto\ } ).

\begin{itemize}
\item
  \texttt{\ \#is.neq(\ test\ )\ } : Creates a new test function that is
  \texttt{\ true\ } when \texttt{\ test\ } is \texttt{\ false\ } . Can
  be used to create negations of tests like
  \texttt{\ \#let\ not-raw\ =\ is.neg(is.raw)\ } .
\item
  \texttt{\ \#is.eq(\ a,\ b\ )\ } : Tests if values \texttt{\ a\ } and
  \texttt{\ b\ } are equal.
\item
  \texttt{\ \#is.neq(\ a,\ b\ )\ } : Tests if values \texttt{\ a\ } and
  \texttt{\ b\ } are not equal.
\item
  \texttt{\ \#is.n(\ ..values\ )\ } : Tests if any of the passed
  \texttt{\ values\ } is \texttt{\ none\ } .
\item
  \texttt{\ \#is.non(\ ..values\ )\ } : Alias for \texttt{\ is.n\ } .
\item
  \texttt{\ \#is.not-none(\ ..values\ )\ } : Tests if all of the passed
  \texttt{\ values\ } are not \texttt{\ none\ } .
\item
  \texttt{\ \#is.not-n(\ ..values\ )\ } : Alias for
  \texttt{\ is.not-none\ } .
\item
  \texttt{\ \#is.a(\ ..values\ )\ } : Tests if any of the passed
  \texttt{\ values\ } is \texttt{\ auto\ } .
\item
  \texttt{\ \#is.aut(\ ..values\ )\ } : Alias for \texttt{\ is-a\ } .
\item
  \texttt{\ \#is.not-auto(\ ..values\ )\ } : Tests if all of the passed
  \texttt{\ values\ } are not \texttt{\ auto\ } .
\item
  \texttt{\ \#is.not-a(\ ..values\ )\ } : Alias for
  \texttt{\ is.not-auto\ } .
\item
  \texttt{\ \#is.empty(\ value\ )\ } : Tests if \texttt{\ value\ } is
  \emph{empty} . A value is considered \emph{empty} if it is an empty
  array, dictionary or string or \texttt{\ none\ } otherwise.
\item
  \texttt{\ \#is.not-empty(\ value\ )\ } : Tests if \texttt{\ value\ }
  is not empty.
\item
  \texttt{\ \#is.any(\ ..compare,\ value\ )\ } : Tests if
  \texttt{\ value\ } is equal to any one of the other passed-in values.
\item
  \texttt{\ \#is.not-any(\ ..compare,\ value)\ } : Tests if
  \texttt{\ value\ } is not equal to any one of the other passed-in
  values.
\item
  \texttt{\ \#is.has(\ ..keys,\ value\ )\ } : Tests if
  \texttt{\ value\ } contains all the passed \texttt{\ keys\ } . Either
  as keys in a dictionary or elements in an array. If \texttt{\ value\ }
  is neither of those types, \texttt{\ false\ } is returned.
\item
  \texttt{\ \#is.type(\ t,\ value\ )\ } : Tests if \texttt{\ value\ } is
  of type \texttt{\ t\ } .
\item
  \texttt{\ \#is.dict(\ value\ )\ } : Tests if \texttt{\ value\ } is a
  dictionary.
\item
  \texttt{\ \#is.arr(\ value\ )\ } : Tests if \texttt{\ value\ } is an
  array.
\item
  \texttt{\ \#is.content(\ value\ )\ } : Tests if \texttt{\ value\ } is
  of type content.
\item
  \texttt{\ \#is.color(\ value\ )\ } : Tests if \texttt{\ value\ } is a
  color.
\item
  \texttt{\ \#is.stroke(\ value\ )\ } : Tests if \texttt{\ value\ } is a
  stroke.
\item
  \texttt{\ \#is.loc(\ value\ )\ } : Tests if \texttt{\ value\ } is a
  location.
\item
  \texttt{\ \#is.bool(\ value\ )\ } : Tests if \texttt{\ value\ } is a
  boolean.
\item
  \texttt{\ \#is.str(\ value\ )\ } : Tests if \texttt{\ value\ } is a
  string.
\item
  \texttt{\ \#is.int(\ value\ )\ } : Tests if \texttt{\ value\ } is an
  integer.
\item
  \texttt{\ \#is.float(\ value\ )\ } : Tests if \texttt{\ value\ } is a
  float.
\item
  \texttt{\ \#is.num(\ value\ )\ } : Tests if \texttt{\ value\ } is a
  numeric value ( \texttt{\ integer\ } or \texttt{\ float\ } ).
\item
  \texttt{\ \#is.frac(\ value\ )\ } : Tests if \texttt{\ value\ } is a
  fraction.
\item
  \texttt{\ \#is.length(\ value\ )\ } : Tests if \texttt{\ value\ } is a
  length.
\item
  \texttt{\ \#is.rlength(\ value\ )\ } : Tests if \texttt{\ value\ } is
  a relative length.
\item
  \texttt{\ \#is.ratio(\ value\ )\ } : Tests if \texttt{\ value\ } is a
  ratio.
\item
  \texttt{\ \#is.align(\ value\ )\ } : Tests if \texttt{\ value\ } is an
  alignment.
\item
  \texttt{\ \#is.align2d(\ value\ )\ } : Tests if \texttt{\ value\ } is
  a 2d alignment.
\item
  \texttt{\ \#is.func(\ value\ )\ } : Tests if \texttt{\ value\ } is a
  function.
\item
  \texttt{\ \#is.any-type(\ ..types,\ value\ )\ } : Tests if
  \texttt{\ value\ } has any of the passed-in types.
\item
  \texttt{\ \#is.same-type(\ ..values\ )\ } : Tests if all passed-in
  values have the same type.
\item
  \texttt{\ \#is.all-of-type(\ t,\ ..values\ )\ } : Tests if all of the
  passed-in values have the type \texttt{\ t\ } .
\item
  \texttt{\ \#is.none-of-type(\ t,\ ..values\ )\ } : Tests if none of
  the passed-in values has the type \texttt{\ t\ } .
\item
  \texttt{\ \#is.one-not-none(\ ..values\ )\ } : Checks, if at least one
  value in \texttt{\ values\ } is not equal to \texttt{\ none\ } .
  Useful for checking multiple optional arguments for a valid value:
  \texttt{\ \#if\ is.one-not-none(..args.pos())\ {[}\ \#args.pos().find(is.not-none)\ {]}\ }
\item
  \texttt{\ \#is.elem(\ func,\ value\ )\ } : Tests if \texttt{\ value\ }
  is a content element with \texttt{\ value.func()\ ==\ func\ } . If
  \texttt{\ func\ } is a string, \texttt{\ value\ } will be compared to
  \texttt{\ repr(value.func())\ } instead.

  Both of these effectively do the same:

\begin{Shaded}
\begin{Highlighting}[]
\NormalTok{\#is}\OperatorTok{.}\FunctionTok{elem}\NormalTok{(raw}\OperatorTok{,}\NormalTok{ some\_content)}
\NormalTok{\#is}\OperatorTok{.}\FunctionTok{elem}\NormalTok{(}\StringTok{"raw"}\OperatorTok{,}\NormalTok{ some\_content)}
\end{Highlighting}
\end{Shaded}
\item
  \texttt{\ \#is.sequence(\ value\ )\ } : Tests if \texttt{\ value\ } is
  a sequence of content.
\item
  \texttt{\ \#is.raw(\ value\ )\ } : Tests if \texttt{\ value\ } is a
  raw element.
\item
  \texttt{\ \#is.table(\ value\ )\ } : Tests if \texttt{\ value\ } is a
  table element.
\item
  \texttt{\ \#is.list(\ value\ )\ } : Tests if \texttt{\ value\ } is a
  list element.
\item
  \texttt{\ \#is.enum(\ value\ )\ } : Tests if \texttt{\ value\ } is an
  enum element.
\item
  \texttt{\ \#is.terms(\ value\ )\ } : Tests if \texttt{\ value\ } is a
  terms element.
\item
  \texttt{\ \#is.cols(\ value\ )\ } : Tests if \texttt{\ value\ } is a
  columns element.
\item
  \texttt{\ \#is.grid(\ value\ )\ } : Tests if \texttt{\ value\ } is a
  grid element.
\item
  \texttt{\ \#is.stack(\ value\ )\ } : Tests if \texttt{\ value\ } is a
  stack element.
\item
  \texttt{\ \#is.label(\ value\ )\ } : Tests if \texttt{\ value\ } is of
  type \texttt{\ label\ } .
\end{itemize}

\subsubsection{Default values}\label{default-values}

\begin{Shaded}
\begin{Highlighting}[]
\NormalTok{\#import }\StringTok{"@preview/t4t:0.3.2"}\OperatorTok{:}\NormalTok{ def}
\end{Highlighting}
\end{Shaded}

These functions perform a test to decide if a given \texttt{\ value\ }
is \emph{invalid} . If the test \emph{passes} , the \texttt{\ default\ }
is returned, the \texttt{\ value\ } otherwise.

Almost all functions support an optional \texttt{\ do\ } argument, to be
set to a function of one argument, that will be applied to the value if
the test fails. For example:

\begin{Shaded}
\begin{Highlighting}[]
\CommentTok{// Sets date to a datetime from an optional}
\CommentTok{// string argument in the format "YYYY{-}MM{-}DD"}
\NormalTok{\#let date }\OperatorTok{=}\NormalTok{ def}\OperatorTok{.}\AttributeTok{if}\OperatorTok{{-}}\FunctionTok{none}\NormalTok{(}
\NormalTok{    datetime}\OperatorTok{.}\FunctionTok{today}\NormalTok{()}\OperatorTok{,}   \CommentTok{// default}
\NormalTok{    passed\_date}\OperatorTok{,}        \CommentTok{// passed{-}in argument}
    \ControlFlowTok{do}\OperatorTok{:}\NormalTok{ (d) }\OperatorTok{\textgreater{}=}\NormalTok{ \{     }\CommentTok{// post{-}processor}
\NormalTok{        d }\OperatorTok{=}\NormalTok{ d}\OperatorTok{.}\FunctionTok{split}\NormalTok{(}\StringTok{"{-}"}\NormalTok{)}
        \FunctionTok{datetime}\NormalTok{(year}\OperatorTok{=}\NormalTok{d[}\DecValTok{0}\NormalTok{]}\OperatorTok{,}\NormalTok{ month}\OperatorTok{=}\NormalTok{d[}\DecValTok{1}\NormalTok{]}\OperatorTok{,}\NormalTok{ day}\OperatorTok{=}\NormalTok{d[}\DecValTok{2}\NormalTok{])}
\NormalTok{    \}}
\NormalTok{)}
\end{Highlighting}
\end{Shaded}

\begin{itemize}
\tightlist
\item
  \texttt{\ \#def.if-true(\ test,\ default,\ do:none,\ value\ )\ } :
  Returns \texttt{\ default\ } if \texttt{\ test\ } is \texttt{\ true\ }
  , \texttt{\ value\ } otherwise.
\item
  \texttt{\ \#def.if-false(\ test,\ default,\ do:none,\ value\ )\ } :
  Returns \texttt{\ default\ } if \texttt{\ test\ } is
  \texttt{\ false\ } , \texttt{\ value\ } otherwise.
\item
  \texttt{\ \#def.if-none(\ default,\ do:none,\ value\ )\ } : Returns
  \texttt{\ default\ } if \texttt{\ value\ } is \texttt{\ none\ } ,
  \texttt{\ value\ } otherwise.
\item
  \texttt{\ \#def.if-auto(\ default,\ do:none,\ value\ )\ } : Returns
  \texttt{\ default\ } if \texttt{\ value\ } is \texttt{\ auto\ } ,
  \texttt{\ value\ } otherwise.
\item
  \texttt{\ \#def.if-any(\ ..compare,\ default,\ do:none,\ value\ )\ } :
  Returns \texttt{\ default\ } if \texttt{\ value\ } is equal to any of
  the passed-in values, \texttt{\ value\ } otherwise. (
  \texttt{\ \#def.if-any(none,\ auto,\ 1pt,\ width)\ } )
\item
  \texttt{\ \#def.if-not-any(\ ..compare,\ default,\ do:none,\ value\ )\ }
  : Returns \texttt{\ default\ } if \texttt{\ value\ } is not equal to
  any of the passed-in values, \texttt{\ value\ } otherwise. (
  \texttt{\ \#def.if-not-any(left,\ right,\ top,\ bottom,\ position)\ }
  )
\item
  \texttt{\ \#def.if-empty(\ default,\ do:none,\ value\ )\ } : Returns
  \texttt{\ default\ } if \texttt{\ value\ } is \emph{empty} ,
  \texttt{\ value\ } otherwise.
\item
  \texttt{\ \#def.as-arr(\ ..values\ )\ } : Always returns an array
  containing all \texttt{\ values\ } . Any arrays in \texttt{\ values\ }
  will be flattened into the result. This is useful for arguments, that
  can have one element or an array of elements:
  \texttt{\ \#def.as-arr(author).join(",\ ")\ } .
\end{itemize}

\subsubsection{Assertions}\label{assertions}

\begin{Shaded}
\begin{Highlighting}[]
\NormalTok{\#import }\StringTok{"@preview/t4t:0.3.2"}\OperatorTok{:}\NormalTok{ assert}
\end{Highlighting}
\end{Shaded}

This submodule overloads the default \texttt{\ assert\ } function and
provides more asserts to quickly check if given values are valid. All
functions use \texttt{\ assert\ } in the background.

Since a module in Typst is not callable, the \texttt{\ assert\ }
function is now available as \texttt{\ assert.that()\ } .
\texttt{\ assert.eq\ } and \texttt{\ assert.ne\ } work as expected.

All assert functions take an optional argument \texttt{\ message\ } to
set the error message shown if the assert fails.

\begin{itemize}
\item
  \texttt{\ \#assert.that(\ test\ )\ } : Asserts that the passed
  \texttt{\ test\ } is \texttt{\ true\ } .
\item
  \texttt{\ \#assert.that-not(\ test\ )\ } : Asserts that the passed
  \texttt{\ test\ } is \texttt{\ false\ } .
\item
  \texttt{\ \#assert.eq(\ a,\ b\ )\ } : Asserts that \texttt{\ a\ } is
  equal to \texttt{\ b\ } .
\item
  \texttt{\ \#assert.ne(\ a,\ b\ )\ } : Asserts that \texttt{\ a\ } is
  not equal to \texttt{\ b\ } .
\item
  \texttt{\ \#assert.neq(\ a,\ b\ )\ } : Alias for
  \texttt{\ assert.ne\ } .
\item
  \texttt{\ \#assert.not-none(\ value\ )\ } : Asserts that
  \texttt{\ value\ } is not equal to \texttt{\ none\ } .
\item
  \texttt{\ \#assert.any(\ ..values,\ value\ )\ } : Asserts that
  \texttt{\ value\ } is one of the passed \texttt{\ values\ } .
\item
  \texttt{\ \#assert.not-any(\ ..values,\ value\ )\ } : Asserts that
  \texttt{\ value\ } is not one of the passed \texttt{\ values\ } .
\item
  \texttt{\ \#assert.any-type(\ ..types,\ value\ )\ } : Asserts that the
  type of \texttt{\ value\ } is one of the passed \texttt{\ types\ } .
\item
  \texttt{\ \#assert.not-any-type(\ ..types,\ value\ )\ } : Asserts that
  the type of \texttt{\ value\ } is not one of the passed
  \texttt{\ types\ } .
\item
  \texttt{\ \#assert.all-of-type(\ t,\ ..values\ )\ } : Asserts that the
  type of all passed \texttt{\ values\ } is equal to \texttt{\ t\ } .
\item
  \texttt{\ \#assert.none-of-type(\ t,\ ..values\ )\ } : Asserts that
  the type of all passed \texttt{\ values\ } is not equal to
  \texttt{\ t\ } .
\item
  \texttt{\ \#assert.not-empty(\ value\ )\ } : Asserts that
  \texttt{\ value\ } is not \emph{empty} .
\item
  \texttt{\ \#assert.new(\ test\ )\ } : Creates a new assert function
  that uses the passed \texttt{\ test\ } . \texttt{\ test\ } is a
  function with signature \texttt{\ (any)\ =\textgreater{}\ boolean\ } .
  This is a quick way to create an assertion from any of the
  \texttt{\ is\ } functions:

\begin{Shaded}
\begin{Highlighting}[]
\NormalTok{\#let assert}\OperatorTok{{-}}\NormalTok{foo }\OperatorTok{=}\NormalTok{ assert}\OperatorTok{.}\FunctionTok{new}\NormalTok{(is}\OperatorTok{.}\AttributeTok{eq}\OperatorTok{.}\FunctionTok{with}\NormalTok{(}\StringTok{"foo"}\NormalTok{))}

\NormalTok{\#let assert}\OperatorTok{{-}}\NormalTok{length }\OperatorTok{=}\NormalTok{ assert}\OperatorTok{.}\FunctionTok{new}\NormalTok{(is}\OperatorTok{.}\AttributeTok{length}\NormalTok{)}
\end{Highlighting}
\end{Shaded}
\end{itemize}

\subsection{Element helpers}\label{element-helpers}

\begin{Shaded}
\begin{Highlighting}[]
\NormalTok{\#import }\StringTok{"@preview/t4t:0.3.2"}\OperatorTok{:} \KeywordTok{get}
\end{Highlighting}
\end{Shaded}

This submodule is a collection of functions, that mostly deal with
content elements and \emph{get} some information from them. Though some
handle other types like dictionaries.

\begin{itemize}
\item
  \texttt{\ \#get.dict(\ ..values\ )\ } : Create a new dictionary from
  the passed \texttt{\ values\ } . All named arguments are stored in the
  new dictionary as is. All positional arguments are grouped in
  key-value pairs and inserted into the dictionary:

\begin{Shaded}
\begin{Highlighting}[]
\NormalTok{\#get}\OperatorTok{.}\FunctionTok{dict}\NormalTok{(}\StringTok{"a"}\OperatorTok{,} \DecValTok{1}\OperatorTok{,} \StringTok{"b"}\OperatorTok{,} \DecValTok{2}\OperatorTok{,} \StringTok{"c"}\OperatorTok{,}\NormalTok{ d}\OperatorTok{:}\DecValTok{4}\OperatorTok{,}\NormalTok{ e}\OperatorTok{:}\DecValTok{5}\NormalTok{)}

\CommentTok{// (a:1, b:2, c:none, d:4, e:5)}
\end{Highlighting}
\end{Shaded}
\item
  \texttt{\ \#get.dict-merge(\ ..dicts\ )\ } : Recursively merges the
  passed-in dictionaries:

\begin{Shaded}
\begin{Highlighting}[]
\NormalTok{\#get}\OperatorTok{.}\AttributeTok{dict}\OperatorTok{{-}}\FunctionTok{merge}\NormalTok{(}
\NormalTok{    (a}\OperatorTok{:} \DecValTok{1}\NormalTok{)}\OperatorTok{,}
\NormalTok{    (a}\OperatorTok{:}\NormalTok{ (one}\OperatorTok{:} \DecValTok{1}\OperatorTok{,}\NormalTok{ two}\OperatorTok{:}\DecValTok{2}\NormalTok{))}\OperatorTok{,}
\NormalTok{    (a}\OperatorTok{:}\NormalTok{ (two}\OperatorTok{:} \DecValTok{4}\OperatorTok{,}\NormalTok{ three}\OperatorTok{:}\DecValTok{3}\NormalTok{))}
\NormalTok{)}

\CommentTok{// (a:(one:1, two:4, three:3))}
\end{Highlighting}
\end{Shaded}
\item
  \texttt{\ \#get.args(\ args,\ prefix:\ ""\ )\ } : Creates a function
  to extract values from an argument sink \texttt{\ args\ } .

  The resulting function takes any number of positional and named
  arguments and creates a dictionary with values from
  \texttt{\ args.named()\ } . Positional arguments are present in the
  result if they are present in \texttt{\ args.named()\ } . Named
  arguments are always present, either with their value from
  \texttt{\ args.named()\ } or with the provided value.

  A \texttt{\ prefix\ } can be specified, to extract only specific
  arguments. The resulting dictionary will have all keys with the prefix
  removed, though.

\begin{Shaded}
\begin{Highlighting}[]
\NormalTok{\#let my}\OperatorTok{{-}}\FunctionTok{func}\NormalTok{( }\OperatorTok{..}\AttributeTok{options}\OperatorTok{,}\NormalTok{ title ) }\OperatorTok{=} \FunctionTok{block}\NormalTok{(}
    \OperatorTok{..}\AttributeTok{get}\OperatorTok{.}\FunctionTok{args}\NormalTok{(options)(}
        \StringTok{"spacing"}\OperatorTok{,} \StringTok{"above"}\OperatorTok{,} \StringTok{"below"}\OperatorTok{,}
\NormalTok{        width}\OperatorTok{:}\DecValTok{100}\OperatorTok{\%}
\NormalTok{    )}
\NormalTok{)[}
\NormalTok{    \#}\FunctionTok{text}\NormalTok{(}\OperatorTok{..}\AttributeTok{get}\OperatorTok{.}\FunctionTok{args}\NormalTok{(options}\OperatorTok{,}\NormalTok{ prefix}\OperatorTok{:}\StringTok{"text{-}"}\NormalTok{)(}
\NormalTok{        fill}\OperatorTok{:}\NormalTok{black}\OperatorTok{,}\NormalTok{ size}\OperatorTok{:}\FloatTok{0.8}\NormalTok{em}
\NormalTok{    )}\OperatorTok{,}\NormalTok{ title)}
\NormalTok{]}

\NormalTok{\#my}\OperatorTok{{-}}\FunctionTok{func}\NormalTok{(}
\NormalTok{    width}\OperatorTok{:} \DecValTok{50}\OperatorTok{\%,}
\NormalTok{    text}\OperatorTok{{-}}\NormalTok{fill}\OperatorTok{:}\NormalTok{ red}\OperatorTok{,}\NormalTok{ text}\OperatorTok{{-}}\NormalTok{size}\OperatorTok{:} \FloatTok{1.2}\NormalTok{em}
\NormalTok{)[\#}\FunctionTok{lorem}\NormalTok{(}\DecValTok{5}\NormalTok{)]}
\end{Highlighting}
\end{Shaded}
\item
  \texttt{\ \#get.text(\ element,\ sep:\ ""\ )\ } : Recursively extracts
  the text content of a content element. If present, all child elements
  are converted to text and joined with \texttt{\ sep\ } .
\item
  \texttt{\ \#get.stroke-paint(\ stroke,\ default:\ black\ )\ } :
  Returns the color of \texttt{\ stroke\ } . If no color information is
  available, \texttt{\ default\ } is used. (Deprecated, use
  \texttt{\ stroke.paint\ } instead.)
\item
  \texttt{\ \#get.stroke-thickness(\ stroke,\ default:\ 1pt\ )\ } :
  Returns the thickness of \texttt{\ stroke\ } . If no thickness
  information is available, \texttt{\ default\ } is used. (Deprecated,
  use \texttt{\ stroke.thickness\ } instead.)
\item
  \texttt{\ \#get.stroke-dict(\ stroke,\ ..overrides\ )\ } : Creates a
  dictionary with the keys necessary for a stroke. The returned
  dictionary is guaranteed to have the keys \texttt{\ paint\ } ,
  \texttt{\ thickness\ } , \texttt{\ dash\ } , \texttt{\ cap\ } and
  \texttt{\ join\ } .

  If \texttt{\ stroke\ } is a dictionary itself, all key-value pairs are
  copied to the resulting stroke. Any named arguments in
  \texttt{\ overrides\ } will override the previous value.
\item
  \texttt{\ \#get.inset-at(\ direction,\ inset,\ default:\ 0pt\ )\ } :
  Returns the inset (or outset) in a given \texttt{\ direction\ } ,
  ascertained from \texttt{\ inset\ } .
\item
  \texttt{\ \#get.inset-dict(\ inset,\ ..overrides\ )\ } : Creates a
  dictionary usable as an inset (or outset) argument.

  The resulting dictionary is guaranteed to have the keys
  \texttt{\ top\ } , \texttt{\ left\ } , \texttt{\ bottom\ } and
  \texttt{\ right\ } .

  If \texttt{\ inset\ } is a dictionary itself, all key-value pairs are
  copied to the resulting stroke. Any named arguments in
  \texttt{\ overrides\ } will override the previous value.
\item
  \texttt{\ \#get.x-align(\ align,\ default:left\ )\ } : Returns the
  alignment along the x-axis from the passed-in \texttt{\ align\ }
  value. If none is present, \texttt{\ default\ } is returned.
  (Deprecated, use \texttt{\ align.x\ } instead.)

\begin{Shaded}
\begin{Highlighting}[]
\NormalTok{\#get}\OperatorTok{.}\AttributeTok{x}\OperatorTok{{-}}\FunctionTok{align}\NormalTok{(top }\OperatorTok{+}\NormalTok{ center) }\CommentTok{// center}
\end{Highlighting}
\end{Shaded}
\item
  \texttt{\ \#get.y-align(\ align,\ default:top\ )\ } : Returns the
  alignment along the y-axis from the passed-in \texttt{\ align\ }
  value. If none is present, \texttt{\ default\ } is returned.
  (Deprecated, use \texttt{\ align.y\ } instead.)
\end{itemize}

\subsection{Math functions}\label{math-functions}

\begin{Shaded}
\begin{Highlighting}[]
\NormalTok{\#import }\StringTok{"@preview/t4t:0.3.2"}\OperatorTok{:}\NormalTok{ math}
\end{Highlighting}
\end{Shaded}

Some functions to complement the native \texttt{\ calc\ } module.

\begin{itemize}
\item
  \texttt{\ \#math.minmax(\ a,\ b\ )\ } : Returns an array with the
  minimum of \texttt{\ a\ } and \texttt{\ b\ } as the first element and
  the maximum as the second:

\begin{Shaded}
\begin{Highlighting}[]
\NormalTok{\#}\FunctionTok{let}\NormalTok{ (min}\OperatorTok{,}\NormalTok{ max) }\OperatorTok{=}\NormalTok{ math}\OperatorTok{.}\FunctionTok{minmax}\NormalTok{(a}\OperatorTok{,}\NormalTok{ b)}
\end{Highlighting}
\end{Shaded}
\item
  \texttt{\ \#math.clamp(\ min,\ max,\ value\ )\ } : Clamps a value
  between \texttt{\ min\ } and \texttt{\ max\ } . In contrast to
  \texttt{\ calc.clamp()\ } this function works for other values than
  numbers, as long as they are comparable.

\begin{Shaded}
\begin{Highlighting}[]
\NormalTok{text}\OperatorTok{{-}}\NormalTok{size }\OperatorTok{=}\NormalTok{ math}\OperatorTok{.}\FunctionTok{clamp}\NormalTok{(}\FloatTok{0.8}\NormalTok{em}\OperatorTok{,} \FloatTok{1.2}\NormalTok{em}\OperatorTok{,}\NormalTok{ text}\OperatorTok{{-}}\NormalTok{size)}
\end{Highlighting}
\end{Shaded}
\item
  \texttt{\ \#lerp(\ min,\ max,\ t\ )\ } : Calculates the linear
  interpolation of \texttt{\ t\ } between \texttt{\ min\ } and
  \texttt{\ max\ } .

\begin{Shaded}
\begin{Highlighting}[]
\NormalTok{\#let width }\OperatorTok{=}\NormalTok{ math}\OperatorTok{.}\FunctionTok{lerp}\NormalTok{(}\DecValTok{0}\OperatorTok{\%,} \DecValTok{100}\OperatorTok{\%,}\NormalTok{ x)}
\end{Highlighting}
\end{Shaded}

  \texttt{\ t\ } should be a value between 0 and 1, but the
  interpolation works with other values, too. To constrain the result
  into the given interval, use \texttt{\ math.clamp\ } :

\begin{Shaded}
\begin{Highlighting}[]
\NormalTok{\#let width }\OperatorTok{=}\NormalTok{ math}\OperatorTok{.}\FunctionTok{lerp}\NormalTok{(}\DecValTok{0}\OperatorTok{\%,} \DecValTok{100}\OperatorTok{\%,}\NormalTok{ math}\OperatorTok{.}\FunctionTok{clamp}\NormalTok{(}\DecValTok{0}\OperatorTok{,} \DecValTok{1}\OperatorTok{,}\NormalTok{ x))}
\end{Highlighting}
\end{Shaded}
\item
  \texttt{\ \#map(\ min,\ max,\ range-min,\ range-max,\ value\ )\ } :
  Maps a \texttt{\ value\ } from the interval
  \texttt{\ {[}min,\ max{]}\ } into the interval
  \texttt{\ {[}range-min,\ range-max{]}\ } :

\begin{Shaded}
\begin{Highlighting}[]
\NormalTok{\#let text}\OperatorTok{{-}}\NormalTok{weight }\OperatorTok{=} \FunctionTok{int}\NormalTok{(math}\OperatorTok{.}\FunctionTok{map}\NormalTok{(}\DecValTok{8}\ErrorTok{pt}\OperatorTok{,} \DecValTok{16}\ErrorTok{pt}\OperatorTok{,} \DecValTok{400}\OperatorTok{,} \DecValTok{800}\OperatorTok{,}\NormalTok{ text}\OperatorTok{{-}}\NormalTok{size))}
\end{Highlighting}
\end{Shaded}
\end{itemize}

\subsection{Alias functions}\label{alias-functions}

\begin{Shaded}
\begin{Highlighting}[]
\NormalTok{\#import }\StringTok{"@preview/t4t:0.3.2"}\OperatorTok{:}\NormalTok{ alias}
\end{Highlighting}
\end{Shaded}

Some of the native Typst function as aliases, to prevent collisions with
some common argument names.

For example using \texttt{\ numbering\ } as an argument is not possible
if the value is supposed to be passed to the \texttt{\ numbering()\ }
function. To still allow argument names that are in line with the common
Typst names (like \texttt{\ numbering\ } , \texttt{\ align\ } …),
these alias functions can be used:

\begin{Shaded}
\begin{Highlighting}[]
\NormalTok{\#let }\FunctionTok{excercise}\NormalTok{( no}\OperatorTok{,}\NormalTok{ numbering}\OperatorTok{:} \StringTok{"1)"}\NormalTok{ ) }\OperatorTok{=}\NormalTok{ [}
\NormalTok{    Exercise \#alias}\OperatorTok{.}\FunctionTok{numbering}\NormalTok{(numbering}\OperatorTok{,}\NormalTok{ no)}
\NormalTok{]}
\end{Highlighting}
\end{Shaded}

The following functions have aliases right now:

\begin{itemize}
\tightlist
\item
  \texttt{\ numbering\ }
\item
  \texttt{\ align\ }
\item
  \texttt{\ type\ }
\item
  \texttt{\ label\ }
\item
  \texttt{\ text\ }
\item
  \texttt{\ raw\ }
\item
  \texttt{\ table\ }
\item
  \texttt{\ list\ }
\item
  \texttt{\ enum\ }
\item
  \texttt{\ terms\ }
\item
  \texttt{\ grid\ }
\item
  \texttt{\ stack\ }
\item
  \texttt{\ columns\ }
\end{itemize}

\subsection{Changelog}\label{changelog}

\subsubsection{Version 0.3.2}\label{version-0.3.2}

\begin{itemize}
\tightlist
\item
  Fixed issue with the new \texttt{\ \#type\ } function in Typst 0.8.0.
\end{itemize}

\subsubsection{Version 0.3.1}\label{version-0.3.1}

\begin{itemize}
\tightlist
\item
  Some fixes for message evaluation in \texttt{\ assert\ } module.
\end{itemize}

\subsubsection{Version 0.3.0}\label{version-0.3.0}

\begin{itemize}
\tightlist
\item
  Added a manual (build with \href{https://github.com/Mc-Zen/tidy}{tidy}
  and \href{https://github.com/jneug/typst-mantys}{Mantys} ).
\item
  Added simple tests for all functions.
\item
  Fixed bug in \texttt{\ is.elem\ } (see \#2).
\item
  Added \texttt{\ assert.has-pos\ } , \texttt{\ assert.no-pos\ } ,
  \texttt{\ assert.has-named\ } and \texttt{\ assert.no-named\ } .
\item
  Added meaningful messages to asserts.

  \begin{itemize}
  \tightlist
  \item
    Asserts now support functions as message arguments that can
    dynamically build an error message from the arguments.
  \end{itemize}
\item
  Improved spelling. (Thanks to @cherryblossom000 for proofreading.)
\end{itemize}

\subsubsection{Version 0.2.0}\label{version-0.2.0}

\begin{itemize}
\tightlist
\item
  Added \texttt{\ is.neg\ } function to negate a test function.
\item
  Added \texttt{\ alias.label\ } .
\item
  Added \texttt{\ is.label\ } test.
\item
  Added \texttt{\ def.as-arr\ } to create an array of the supplied
  values. Useful if an argument can be both a single value or an array.
\item
  Added \texttt{\ assert.that-not\ } for negated assertions.
\item
  Added \texttt{\ is.one-not-none\ } test to check multiple values, if
  at least one is not none.
\item
  Added \texttt{\ do\ } argument to all functions in \texttt{\ def\ } .
\item
  Allowed strings in \texttt{\ is.elem\ } (see \#1).

  \begin{itemize}
  \tightlist
  \item
    Added \texttt{\ is.sequence\ } test.
  \end{itemize}
\item
  Deprecated \texttt{\ get.stroke-paint\ } ,
  \texttt{\ get.stroke-thickness\ } , \texttt{\ get.x-align\ } and
  \texttt{\ get.y-align\ } in favor of new Typst 0.7.0 features.
\end{itemize}

\subsubsection{Version 0.1.0}\label{version-0.1.0}

\begin{itemize}
\tightlist
\item
  Initial release
\end{itemize}

\subsubsection{How to add}\label{how-to-add}

Copy this into your project and use the import as \texttt{\ t4t\ }

\begin{verbatim}
#import "@preview/t4t:0.3.2"
\end{verbatim}

\includesvg[width=0.16667in,height=0.16667in]{/assets/icons/16-copy.svg}

Check the docs for
\href{https://typst.app/docs/reference/scripting/\#packages}{more
information on how to import packages} .

\subsubsection{About}\label{about}

\begin{description}
\tightlist
\item[Author :]
Jonas Neugebauer
\item[License:]
MIT
\item[Current version:]
0.3.2
\item[Last updated:]
September 15, 2023
\item[First released:]
July 31, 2023
\item[Archive size:]
15.5 kB
\href{https://packages.typst.org/preview/t4t-0.3.2.tar.gz}{\pandocbounded{\includesvg[keepaspectratio]{/assets/icons/16-download.svg}}}
\item[Repository:]
\href{https://github.com/jneug/typst-tools4typst}{GitHub}
\end{description}

\subsubsection{Where to report issues?}\label{where-to-report-issues}

This package is a project of Jonas Neugebauer . Report issues on
\href{https://github.com/jneug/typst-tools4typst}{their repository} .
You can also try to ask for help with this package on the
\href{https://forum.typst.app}{Forum} .

Please report this package to the Typst team using the
\href{https://typst.app/contact}{contact form} if you believe it is a
safety hazard or infringes upon your rights.

\phantomsection\label{versions}
\subsubsection{Version history}\label{version-history}

\begin{longtable}[]{@{}ll@{}}
\toprule\noalign{}
Version & Release Date \\
\midrule\noalign{}
\endhead
\bottomrule\noalign{}
\endlastfoot
0.3.2 & September 15, 2023 \\
\href{https://typst.app/universe/package/t4t/0.3.1/}{0.3.1} & September
7, 2023 \\
\href{https://typst.app/universe/package/t4t/0.3.0/}{0.3.0} & August 24,
2023 \\
\href{https://typst.app/universe/package/t4t/0.2.0/}{0.2.0} & August 8,
2023 \\
\href{https://typst.app/universe/package/t4t/0.1.0/}{0.1.0} & July 31,
2023 \\
\end{longtable}

Typst GmbH did not create this package and cannot guarantee correct
functionality of this package or compatibility with any version of the
Typst compiler or app.
