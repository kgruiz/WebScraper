\title{typst.app/universe/package/minimal-presentation}

\phantomsection\label{banner}
\phantomsection\label{template-thumbnail}
\pandocbounded{\includegraphics[keepaspectratio]{https://packages.typst.org/preview/thumbnails/minimal-presentation-0.3.0-small.webp}}

\section{minimal-presentation}\label{minimal-presentation}

{ 0.3.0 }

A modern minimalistic presentation template ready to use

\href{/app?template=minimal-presentation&version=0.3.0}{Create project
in app}

\phantomsection\label{readme}
A modern minimalistic presentation template ready to use.

\subsection{Usage}\label{usage}

You can use this template in the Typst web app by clicking “Start from
template� on the dashboard and searching for
\texttt{\ minimal-presentation\ } .

Alternatively, you can use the CLI to kick this project off using the
command

\begin{verbatim}
typst init @preview/minimal-presentation
\end{verbatim}

Typst will create a new directory with all the files needed to get you
started.

\subsection{Configuration}\label{configuration}

This template exports the \texttt{\ project\ } function with the
following named arguments:

\begin{itemize}
\tightlist
\item
  \texttt{\ title\ } : The book’s title as content.
\item
  \texttt{\ sub-title\ } : The book’s subtitle as content.
\item
  \texttt{\ author\ } : Content or an array of content to specify the
  author.
\item
  \texttt{\ aspect-ratio\ } : Defaults to \texttt{\ 16-9\ } . Can be
  also \texttt{\ 4-3\ } .
\end{itemize}

The function also accepts a single, positional argument for the body of
the book.

The template will initialize your package with a sample call to the
\texttt{\ project\ } function in a show rule. If you, however, want to
change an existing project to use this template, you can add a show rule
like this at the top of your file:

\begin{Shaded}
\begin{Highlighting}[]
\NormalTok{\#import "@preview/minimal{-}presentation:0.1.0": *}

\NormalTok{\#set text(font: "Lato")}
\NormalTok{\#show math.equation: set text(font: "Lato Math")}
\NormalTok{\#show raw: set text(font: "Fira Code")}

\NormalTok{\#show: project.with(}
\NormalTok{  title: "Minimalist presentation template",}
\NormalTok{  sub{-}title: "This is where your presentation begins",}
\NormalTok{  author: "Flavio Barisi",}
\NormalTok{  date: "10/08/2023",}
\NormalTok{  index{-}title: "Contents",}
\NormalTok{  logo: image("./logo.svg"),}
\NormalTok{  logo{-}light: image("./logo\_light.svg"),}
\NormalTok{  cover: image("./image\_3.jpg")}
\NormalTok{)}

\NormalTok{= This is a section}

\NormalTok{== This is a slide title}

\NormalTok{\#lorem(10)}

\NormalTok{{-} \#lorem(10)}
\NormalTok{  {-} \#lorem(10)}
\NormalTok{  {-} \#lorem(10)}
\NormalTok{  {-} \#lorem(10)}

\NormalTok{== One column image}

\NormalTok{\#figure(}
\NormalTok{  image("image\_1.jpg", height: 10.5cm),}
\NormalTok{  caption: [An image],}
\NormalTok{) \textless{}image\_label\textgreater{}}

\NormalTok{== Two columns image}

\NormalTok{\#columns{-}content()[}
\NormalTok{  \#figure(}
\NormalTok{    image("image\_1.jpg", width: 100\%),}
\NormalTok{    caption: [An image],}
\NormalTok{  ) \textless{}image\_label\_1\textgreater{}}
\NormalTok{][}
\NormalTok{  \#figure(}
\NormalTok{    image("image\_1.jpg", width: 100\%),}
\NormalTok{    caption: [An image],}
\NormalTok{  ) \textless{}image\_label\_2\textgreater{}}
\NormalTok{]}

\NormalTok{== Two columns}

\NormalTok{\#columns{-}content()[}
\NormalTok{  {-} \#lorem(10)}
\NormalTok{  {-} \#lorem(10)}
\NormalTok{  {-} \#lorem(10)}
\NormalTok{][}
\NormalTok{  \#figure(}
\NormalTok{    image("image\_3.jpg", width: 100\%),}
\NormalTok{    caption: [An image],}
\NormalTok{  ) \textless{}image\_label\_3\textgreater{}}
\NormalTok{]}

\NormalTok{= This is a section}

\NormalTok{== This is a slide title}

\NormalTok{\#lorem(10)}

\NormalTok{= This is a section}

\NormalTok{== This is a slide title}

\NormalTok{\#lorem(10)}

\NormalTok{= This is a section}

\NormalTok{== This is a slide title}

\NormalTok{\#lorem(10)}

\NormalTok{= This is a very v v v v v v v v v v v v v v v v v v v v  long section}

\NormalTok{== This is a very v v v v v v v v v v v v v v v v v v v v  long slide title}

\NormalTok{= sub{-}title test}

\NormalTok{== Slide title}

\NormalTok{\#lorem(50)}

\NormalTok{=== Slide sub{-}title 1}

\NormalTok{\#lorem(50)}

\NormalTok{=== Slide sub{-}title 2}

\NormalTok{\#lorem(50)}

\end{Highlighting}
\end{Shaded}

\subsection{Fonts}\label{fonts}

You can use the font selected by the author of this plugin, by download
theme at the following link:

\url{https://github.com/flavio20002/typst-presentation-minimal-template/tree/main/fonts}

You can then import thme in your system, import them in the typst web
app or just put them in a folder and launch the compilation with the
following argoument:

\begin{verbatim}
typst watch main.typ --root . --font-path fonts
\end{verbatim}

\href{/app?template=minimal-presentation&version=0.3.0}{Create project
in app}

\subsubsection{How to use}\label{how-to-use}

Click the button above to create a new project using this template in
the Typst app.

You can also use the Typst CLI to start a new project on your computer
using this command:

\begin{verbatim}
typst init @preview/minimal-presentation:0.3.0
\end{verbatim}

\includesvg[width=0.16667in,height=0.16667in]{/assets/icons/16-copy.svg}

\subsubsection{About}\label{about}

\begin{description}
\tightlist
\item[Author :]
Flavio Barisi
\item[License:]
MIT-0
\item[Current version:]
0.3.0
\item[Last updated:]
November 18, 2024
\item[First released:]
September 2, 2024
\item[Minimum Typst version:]
0.12.0
\item[Archive size:]
755 kB
\href{https://packages.typst.org/preview/minimal-presentation-0.3.0.tar.gz}{\pandocbounded{\includesvg[keepaspectratio]{/assets/icons/16-download.svg}}}
\item[Repository:]
\href{https://github.com/flavio20002/typst-presentation-minimal-template}{GitHub}
\item[Categor y :]
\begin{itemize}
\tightlist
\item[]
\item
  \pandocbounded{\includesvg[keepaspectratio]{/assets/icons/16-presentation.svg}}
  \href{https://typst.app/universe/search/?category=presentation}{Presentation}
\end{itemize}
\end{description}

\subsubsection{Where to report issues?}\label{where-to-report-issues}

This template is a project of Flavio Barisi . Report issues on
\href{https://github.com/flavio20002/typst-presentation-minimal-template}{their
repository} . You can also try to ask for help with this template on the
\href{https://forum.typst.app}{Forum} .

Please report this template to the Typst team using the
\href{https://typst.app/contact}{contact form} if you believe it is a
safety hazard or infringes upon your rights.

\phantomsection\label{versions}
\subsubsection{Version history}\label{version-history}

\begin{longtable}[]{@{}ll@{}}
\toprule\noalign{}
Version & Release Date \\
\midrule\noalign{}
\endhead
\bottomrule\noalign{}
\endlastfoot
0.3.0 & November 18, 2024 \\
\href{https://typst.app/universe/package/minimal-presentation/0.2.0/}{0.2.0}
& October 23, 2024 \\
\href{https://typst.app/universe/package/minimal-presentation/0.1.0/}{0.1.0}
& September 2, 2024 \\
\end{longtable}

Typst GmbH did not create this template and cannot guarantee correct
functionality of this template or compatibility with any version of the
Typst compiler or app.
