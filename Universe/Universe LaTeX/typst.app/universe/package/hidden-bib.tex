\title{typst.app/universe/package/hidden-bib}

\phantomsection\label{banner}
\section{hidden-bib}\label{hidden-bib}

{ 0.1.1 }

Create hidden bibliographies or bibliographies with unmentioned (hidden)
citations.

\phantomsection\label{readme}
\href{https://github.com/pklaschka/typst-hidden-bib}{GitHub Repository
including Examples}

A Typst package to create hidden bibliographies or bibliographies with
unmentioned (hidden) citations.

\subsection{Use Cases}\label{use-cases}

\subsubsection{Hidden Bibliographies}\label{hidden-bibliographies}

In some documents, such as a letter, you may want to cite a reference
without printing a bibliography.

This can easily be achieved by wrapping your
\texttt{\ bibliography(...)\ } with the \texttt{\ hidden-bibliography\ }
function after importing the \texttt{\ hidden-bib\ } package.

The code then looks like this:

\begin{Shaded}
\begin{Highlighting}[]
\NormalTok{\#import "@preview/hidden{-}bib:0.1.0": hidden{-}bibliography}

\NormalTok{\#lorem(20) @example1}
\NormalTok{\#lorem(40) @example2[p. 2]}

\NormalTok{\#hidden{-}bibliography(}
\NormalTok{  bibliography("/refs.yml")}
\NormalTok{)}
\end{Highlighting}
\end{Shaded}

\emph{Note that this automatically sets the \texttt{\ style\ } option to
\texttt{\ "chicago-notes"\ } unless you specify a different style.}

\subsubsection{Hidden Citations}\label{hidden-citations}

In some documents, it may be necessary to include items in your
bibliography which weren’t explicitly cited at any specific point in
your document.

The code then looks like this:

\begin{Shaded}
\begin{Highlighting}[]
\NormalTok{\#import "@preview/hidden{-}bib:0.1.0": hidden{-}cite}

\NormalTok{\#hidden{-}cite("example1")}
\end{Highlighting}
\end{Shaded}

\subsubsection{Multiple Hidden
Citations}\label{multiple-hidden-citations}

If you want to include a large number of items in your bibliography
without having to use \texttt{\ hidden-cite\ } (to still get
autocompletion in the web editor), you can use the
\texttt{\ hidden-citations\ } environment.

The code then looks like this:

\begin{Shaded}
\begin{Highlighting}[]
\NormalTok{\#import "@preview/hidden{-}bib:0.1.0": hidden{-}citations}

\NormalTok{\#hidden{-}citations[}
\NormalTok{  @example1}
\NormalTok{  @example2}
\NormalTok{]}
\end{Highlighting}
\end{Shaded}

\subsection{FAQ}\label{faq}

\subsubsection{Why would I want to have hidden citations and a hidden
bibliography?}\label{why-would-i-want-to-have-hidden-citations-and-a-hidden-bibliography}

You don’t. While this package solves both (related) problems, you
should only use one of them at a time. Otherwise, you’ll simply see
nothing at all.

\subsubsection{Why would I want to have hidden
citations?}\label{why-would-i-want-to-have-hidden-citations}

That’s for you to decide. It essentially enables you to include
“uncited references�, similar to LaTeX’s
\texttt{\ \textbackslash{}nocite\{\}\ } command.

\subsection{License}\label{license}

This package is licensed under the MIT license. See the
\href{https://github.com/typst/packages/raw/main/packages/preview/hidden-bib/0.1.1/LICENSE}{LICENSE}
file for details.

\subsubsection{How to add}\label{how-to-add}

Copy this into your project and use the import as
\texttt{\ hidden-bib\ }

\begin{verbatim}
#import "@preview/hidden-bib:0.1.1"
\end{verbatim}

\includesvg[width=0.16667in,height=0.16667in]{/assets/icons/16-copy.svg}

Check the docs for
\href{https://typst.app/docs/reference/scripting/\#packages}{more
information on how to import packages} .

\subsubsection{About}\label{about}

\begin{description}
\tightlist
\item[Author :]
Zuri Klaschka
\item[License:]
MIT
\item[Current version:]
0.1.1
\item[Last updated:]
October 10, 2023
\item[First released:]
October 2, 2023
\item[Archive size:]
2.12 kB
\href{https://packages.typst.org/preview/hidden-bib-0.1.1.tar.gz}{\pandocbounded{\includesvg[keepaspectratio]{/assets/icons/16-download.svg}}}
\item[Repository:]
\href{https://github.com/pklaschka/typst-hidden-bib}{GitHub}
\end{description}

\subsubsection{Where to report issues?}\label{where-to-report-issues}

This package is a project of Zuri Klaschka . Report issues on
\href{https://github.com/pklaschka/typst-hidden-bib}{their repository} .
You can also try to ask for help with this package on the
\href{https://forum.typst.app}{Forum} .

Please report this package to the Typst team using the
\href{https://typst.app/contact}{contact form} if you believe it is a
safety hazard or infringes upon your rights.

\phantomsection\label{versions}
\subsubsection{Version history}\label{version-history}

\begin{longtable}[]{@{}ll@{}}
\toprule\noalign{}
Version & Release Date \\
\midrule\noalign{}
\endhead
\bottomrule\noalign{}
\endlastfoot
0.1.1 & October 10, 2023 \\
\href{https://typst.app/universe/package/hidden-bib/0.1.0/}{0.1.0} &
October 2, 2023 \\
\end{longtable}

Typst GmbH did not create this package and cannot guarantee correct
functionality of this package or compatibility with any version of the
Typst compiler or app.
