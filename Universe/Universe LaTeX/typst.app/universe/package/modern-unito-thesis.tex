\title{typst.app/universe/package/modern-unito-thesis}

\phantomsection\label{banner}
\phantomsection\label{template-thumbnail}
\pandocbounded{\includegraphics[keepaspectratio]{https://packages.typst.org/preview/thumbnails/modern-unito-thesis-0.1.0-small.webp}}

\section{modern-unito-thesis}\label{modern-unito-thesis}

{ 0.1.0 }

A thesis template of the University of Turin

\href{/app?template=modern-unito-thesis&version=0.1.0}{Create project in
app}

\phantomsection\label{readme}
This is a thesis template for the University of Turin (UniTO) based on
\href{https://github.com/eduardz1/Bachelor-Thesis}{my thesis} , since
there are no strict templates (notable mention to
\href{https://github.com/esenes/Unito-thesis-template}{Eugenio’s LaTeX
template though} ) take my choices with a grain of salt, different
supervisors may ask you to customize the template differently. My
choices are loosely based on this document:
\href{https://elearning.unito.it/sme/pluginfile.php/29485/mod_folder/content/0/format_TESI_2011-2012.pdf}{Indicazioni
per il Format della Tesi} .

If you find errors or ways to improve the template please open an issue
or contribute directly with a PR.

\subsection{Usage}\label{usage}

In the Typst web app simply click “Start from template� on the
dashboard and search for \texttt{\ modern-unito-thesis\ } .

From the CLI you can initialize the project with the command

\begin{Shaded}
\begin{Highlighting}[]
\ExtensionTok{typst}\NormalTok{ init @preview/modern{-}unito{-}thesis}
\end{Highlighting}
\end{Shaded}

A new directory with all the files needed to get started will be
created.

\subsection{Configuration}\label{configuration}

This template exports the \texttt{\ template\ } function with the
following named arguments:

\begin{itemize}
\tightlist
\item
  \texttt{\ title\ } : the title of the thesis
\item
  \texttt{\ academic-year\ } : the academic year (e.g. 2023/2024)
\item
  \texttt{\ subtitle\ } : e.g. “Bachelor’s Thesis�
\item
  \texttt{\ paper-size\ } (default \texttt{\ a4\ } ): the paper format
\item
  \texttt{\ candidate\ } : your name, surname and matricola (student id)
\item
  \texttt{\ supervisor\ } (relatore): your supervisor’s name and
  surname
\item
  \texttt{\ co-supervisor\ } (correlatore): an array of your
  co-supervisors’ names and surnames
\item
  \texttt{\ affiliation\ } : a dictionary that specifies
  \texttt{\ university\ } , \texttt{\ school\ } and \texttt{\ degree\ }
  keywords
\item
  \texttt{\ lang\ } : configurable between \texttt{\ en\ } for English
  and \texttt{\ it\ } for Italian
\item
  \texttt{\ bibliography-path\ } : the path to your bibliography file
  (e.g. \texttt{\ works.bib\ } )
\item
  \texttt{\ logo\ } (already set to UniTO’s logo by default): the path
  to your university’s logo
\item
  \texttt{\ abstract\ } : your thesis’ abstract, can be set to
  \texttt{\ none\ } if not needed
\item
  \texttt{\ acknowledgments\ } : your thesis’ acknowledgments, can be
  set to \texttt{\ none\ } if not needed
\item
  \texttt{\ keywords\ } : a list of keywords for the thesis, can be set
  to \texttt{\ none\ } if not needed
\end{itemize}

The template will initialize an example project with sensible defaults.

The template divides the level 1 headings in chapters under the
\texttt{\ chapters\ } directory, I suggest using this structure to keep
the project organized.

If you want to change an existing project to use this template, you can
add a show rule like this at the top of your file:

\begin{Shaded}
\begin{Highlighting}[]
\NormalTok{\#import "@preview/modern{-}unito{-}thesis:0.1.0": template}

\NormalTok{\#show: template.with(}
\NormalTok{  title: "My Beautiful Thesis",}
\NormalTok{  academic{-}year: [2023/2024],}
\NormalTok{  subtitle: "Bachelor\textquotesingle{}s Thesis",}
\NormalTok{  logo: image("path/to/your/logo.png"),}
\NormalTok{  candidate: (}
\NormalTok{    name: "Eduard Antonovic Occhipinti",}
\NormalTok{    matricola: 947847}
\NormalTok{  ),}
\NormalTok{  supervisor: (}
\NormalTok{    "Prof. Luigi Paperino"}
\NormalTok{  ),}
\NormalTok{  co{-}supervisor: (}
\NormalTok{    "Dott. Pluto Mario",}
\NormalTok{    "Dott. Minni Topolino"}
\NormalTok{  ),}
\NormalTok{  affiliation: (}
\NormalTok{    university: "Università degli Studi di Torino",}
\NormalTok{    school: "Scuola di Scienze della Natura",}
\NormalTok{    degree: "Corso di Laurea Triennale in Informatica",}
\NormalTok{  ),}
\NormalTok{  bibliography: bibliography("works.yml"),}
\NormalTok{  abstract: [Your abstract goes here],}
\NormalTok{  acknowledgments: [Your acknowledgments go here],}
\NormalTok{  keywords: [keyword1, keyword2, keyword3]}
\NormalTok{)}

\NormalTok{// Your content goes here}
\end{Highlighting}
\end{Shaded}

\subsection{Compile}\label{compile}

To compile the project from the CLI you just need to run

\begin{Shaded}
\begin{Highlighting}[]
\ExtensionTok{typst}\NormalTok{ compile main.typ}
\end{Highlighting}
\end{Shaded}

or if you want to watch for changes (recommended)

\begin{Shaded}
\begin{Highlighting}[]
\ExtensionTok{typst}\NormalTok{ watch main.typ}
\end{Highlighting}
\end{Shaded}

\subsection{Bibliography}\label{bibliography}

I integrated the bibliography as a
\href{https://github.com/typst/hayagriva}{Hayagriva} \texttt{\ yaml\ }
file under
\href{https://github.com/typst/packages/raw/main/packages/preview/modern-unito-thesis/0.1.0/template/works.yml}{works.yml}
, nonetheless using the more common \texttt{\ bib\ } format for your
bibliography management is as simple as passing a BibTex file to the
template \texttt{\ bibliography\ } parameter. Given that our university
is not strict in this regard I suggest using Hayagriva though :).

\href{/app?template=modern-unito-thesis&version=0.1.0}{Create project in
app}

\subsubsection{How to use}\label{how-to-use}

Click the button above to create a new project using this template in
the Typst app.

You can also use the Typst CLI to start a new project on your computer
using this command:

\begin{verbatim}
typst init @preview/modern-unito-thesis:0.1.0
\end{verbatim}

\includesvg[width=0.16667in,height=0.16667in]{/assets/icons/16-copy.svg}

\subsubsection{About}\label{about}

\begin{description}
\tightlist
\item[Author :]
Eduard Antonovic Occhipinti
\item[License:]
MIT
\item[Current version:]
0.1.0
\item[Last updated:]
March 20, 2024
\item[First released:]
March 20, 2024
\item[Minimum Typst version:]
0.11.0
\item[Archive size:]
19.6 kB
\href{https://packages.typst.org/preview/modern-unito-thesis-0.1.0.tar.gz}{\pandocbounded{\includesvg[keepaspectratio]{/assets/icons/16-download.svg}}}
\item[Repository:]
\href{https://github.com/eduardz1/unito-typst-template}{GitHub}
\item[Categor y :]
\begin{itemize}
\tightlist
\item[]
\item
  \pandocbounded{\includesvg[keepaspectratio]{/assets/icons/16-mortarboard.svg}}
  \href{https://typst.app/universe/search/?category=thesis}{Thesis}
\end{itemize}
\end{description}

\subsubsection{Where to report issues?}\label{where-to-report-issues}

This template is a project of Eduard Antonovic Occhipinti . Report
issues on \href{https://github.com/eduardz1/unito-typst-template}{their
repository} . You can also try to ask for help with this template on the
\href{https://forum.typst.app}{Forum} .

Please report this template to the Typst team using the
\href{https://typst.app/contact}{contact form} if you believe it is a
safety hazard or infringes upon your rights.

\phantomsection\label{versions}
\subsubsection{Version history}\label{version-history}

\begin{longtable}[]{@{}ll@{}}
\toprule\noalign{}
Version & Release Date \\
\midrule\noalign{}
\endhead
\bottomrule\noalign{}
\endlastfoot
0.1.0 & March 20, 2024 \\
\end{longtable}

Typst GmbH did not create this template and cannot guarantee correct
functionality of this template or compatibility with any version of the
Typst compiler or app.
