\title{typst.app/universe/package/use-academicons}

\phantomsection\label{banner}
\section{use-academicons}\label{use-academicons}

{ 0.1.0 }

A Typst library for Academicons the desktop fonts.

\phantomsection\label{readme}
A Typst library for Academicons through the desktop fonts.

This is based on the code from \texttt{\ duskmoon314\ } and the package
for
\href{https://github.com/duskmoon314/typst-fontawesome}{\textbf{typst-fontawesome}}
.

p.s. The library is based on the Academicons desktop fonts (v1.9.4)

\subsection{Usage}\label{usage}

\subsubsection{Install the fonts}\label{install-the-fonts}

You can download the fonts from the
\href{https://jpswalsh.github.io/academicons/}{official website}

After downloading the zip file, you can install the fonts depending on
your OS.

\paragraph{Typst web app}\label{typst-web-app}

You can simply upload the \texttt{\ ttf\ } files to the web app and use
them with this package.

\paragraph{Mac}\label{mac}

You can double click the \texttt{\ ttf\ } files to install them.

\paragraph{Windows}\label{windows}

You can right-click the \texttt{\ ttf\ } files and select
\texttt{\ Install\ } .

\subsubsection{Import the library}\label{import-the-library}

\paragraph{Using the typst packages}\label{using-the-typst-packages}

You can install the library using the typst packages:

\texttt{\ \#import\ "@preview/use-academicons:0.1.0":\ *\ }

\paragraph{Manually install}\label{manually-install}

Copy all files start with \texttt{\ lib\ } to your project and import
the library:

\texttt{\ \#import\ "lib.typ":\ *\ }

There are three files:

\begin{itemize}
\tightlist
\item
  \texttt{\ lib.typ\ } : The main entrypoint of the library.
\item
  \texttt{\ lib-impl.typ\ } : The implementation of \texttt{\ ai-icon\ }
  .
\item
  \texttt{\ lib-gen.typ\ } : The generated icon map and functions.
\end{itemize}

I recommend renaming these files to avoid conflicts with other
libraries.

\subsubsection{Use the icons}\label{use-the-icons}

You can use the \texttt{\ ai-icon\ } function to create an icon with its
name:

\texttt{\ \#ai-icon("lattes")\ }

Or you can use the \texttt{\ ai-\ } prefix to create an icon with its
name:

\texttt{\ \#ai-lattes()\ } (This is equivalent to
\texttt{\ \#ai-icon().with("lattes")\ } )

\paragraph{Full list of icons}\label{full-list-of-icons}

You can find all icons on the
\href{https://jpswalsh.github.io/academicons/}{official website}

\paragraph{Customization}\label{customization}

The \texttt{\ ai-icon\ } function passes args to \texttt{\ text\ } , so
you can customize the icon by passing parameters to it:

\texttt{\ \#ai-icon("lattes",\ fill:\ blue)\ }

\paragraph{Stacking icons}\label{stacking-icons}

The \texttt{\ ai-stack\ } function can be used to create stacked icons:

\texttt{\ \#ai-stack(ai-icon-args:\ (fill:\ black),\ "doi",\ ("cv",\ (fill:\ blue,\ size:\ 20pt)))\ }

Declaration is
\texttt{\ ai-stack(box-args:\ (:),\ grid-args:\ (:),\ ai-icon-args:\ (:),\ ..icons)\ }

\begin{itemize}
\tightlist
\item
  The order of the icons is from the bottom to the top.
\item
  \texttt{\ ai-icon-args\ } is used to set the default args for all
  icons.
\item
  You can also control the internal \texttt{\ box\ } and
  \texttt{\ grid\ } by passing the \texttt{\ box-args\ } and
  \texttt{\ grid-args\ } to the \texttt{\ ai-stack\ } function.
\item
  Currently, four types of icons are supported. The first three types
  leverage the \texttt{\ ai-icon\ } function, and the last type is just
  a content you want to put in the stack.

  \begin{itemize}
  \tightlist
  \item
    \texttt{\ str\ } , e.g., \texttt{\ "lattes"\ }
  \item
    \texttt{\ array\ } , e.g.,
    \texttt{\ ("lattes",\ (fill:\ white,\ size:\ 5.5pt))\ }
  \item
    \texttt{\ arguments\ } , e.g.
    \texttt{\ arguments("lattes",\ fill:\ white)\ }
  \item
    \texttt{\ content\ } , e.g. \texttt{\ ai-lattes(fill:\ white)\ }
  \end{itemize}
\end{itemize}

\subsection{Example}\label{example}

See the
\href{https://typst.app/project/rsgOFC4YkwpN7OqtRyiXP3}{\texttt{\ use-academicons.typ\ }}
file for a complete example.

\subsection{Contribution}\label{contribution}

Feel free to open an issue or a pull request if you find any problems or
have any suggestions.

\subsubsection{R helper}\label{r-helper}

The \texttt{\ helper.R\ } script is used to get unicodes for icons and
generate typst code.

\subsubsection{Repo structure}\label{repo-structure}

\begin{itemize}
\tightlist
\item
  \texttt{\ helper.R\ } : The helper script to get unicodes and generate
  typst code.
\item
  \texttt{\ lib.typ\ } : The main entrypoint of the library.
\item
  \texttt{\ lib-impl.typ\ } : The implementation of \texttt{\ ai-icon\ }
  .
\item
  \texttt{\ lib-gen.typ\ } : The generated functions of icons.
\item
  \texttt{\ example.typ\ } : An example file to show how to use the
  library.
\item
  \texttt{\ gallery.typ\ } : The generated gallery of icons. It is used
  in the example file.
\end{itemize}

\subsection{License}\label{license}

This library is licensed under the MIT license. Feel free to use it in
your project.

\subsubsection{How to add}\label{how-to-add}

Copy this into your project and use the import as
\texttt{\ use-academicons\ }

\begin{verbatim}
#import "@preview/use-academicons:0.1.0"
\end{verbatim}

\includesvg[width=0.16667in,height=0.16667in]{/assets/icons/16-copy.svg}

Check the docs for
\href{https://typst.app/docs/reference/scripting/\#packages}{more
information on how to import packages} .

\subsubsection{About}\label{about}

\begin{description}
\tightlist
\item[Author s :]
\href{mailto:kp.campbell.he@duskmoon314.com}{duskmoon (Campbell He)} \&
\href{mailto:philipp.kleer@posteo.com}{bpkleer (Philipp Kleer)}
\item[License:]
MIT
\item[Current version:]
0.1.0
\item[Last updated:]
August 8, 2024
\item[First released:]
August 8, 2024
\item[Archive size:]
5.41 kB
\href{https://packages.typst.org/preview/use-academicons-0.1.0.tar.gz}{\pandocbounded{\includesvg[keepaspectratio]{/assets/icons/16-download.svg}}}
\item[Repository:]
\href{https://github.com/bpkleer/typst-academicons}{GitHub}
\end{description}

\subsubsection{Where to report issues?}\label{where-to-report-issues}

This package is a project of duskmoon (Campbell He) and bpkleer (Philipp
Kleer) . Report issues on
\href{https://github.com/bpkleer/typst-academicons}{their repository} .
You can also try to ask for help with this package on the
\href{https://forum.typst.app}{Forum} .

Please report this package to the Typst team using the
\href{https://typst.app/contact}{contact form} if you believe it is a
safety hazard or infringes upon your rights.

\phantomsection\label{versions}
\subsubsection{Version history}\label{version-history}

\begin{longtable}[]{@{}ll@{}}
\toprule\noalign{}
Version & Release Date \\
\midrule\noalign{}
\endhead
\bottomrule\noalign{}
\endlastfoot
0.1.0 & August 8, 2024 \\
\end{longtable}

Typst GmbH did not create this package and cannot guarantee correct
functionality of this package or compatibility with any version of the
Typst compiler or app.
