\title{typst.app/universe/package/based}

\phantomsection\label{banner}
\section{based}\label{based}

{ 0.2.0 }

Encoder and decoder for base64, base32, and base16.

\phantomsection\label{readme}
A package for encoding and decoding in base64, base32, and base16.

\subsection{Usage}\label{usage}

The package comes with three submodules: \texttt{\ base64\ } ,
\texttt{\ base32\ } , and \texttt{\ base16\ } . All of them have an
\texttt{\ encode\ } and \texttt{\ decode\ } function. The package also
provides the function aliases

\begin{itemize}
\tightlist
\item
  \texttt{\ encode64\ } / \texttt{\ decode64\ } ,
\item
  \texttt{\ encode32\ } / \texttt{\ decode32\ } , and
\item
  \texttt{\ encode16\ } / \texttt{\ decode16\ } .
\end{itemize}

Both base64 and base32 allow you to choose whether to use padding for
encoding via the \texttt{\ pad\ } parameter, which is enabled by
default. Base64 also allows you to encode with the URL-safe alphabet by
enabling the \texttt{\ url\ } parameter, while base32 allows you to
encode or decode with the “extended hex� alphabet via the
\texttt{\ hex\ } parameter. Both options are disabled by default. The
base16 encoder uses lowercase letters, the decoder is case-insensitive.

You can encode strings, arrays and bytes. The \texttt{\ encode\ }
function will return a string, while the \texttt{\ decode\ } function
will return bytes.

\subsection{Example}\label{example}

\begin{Shaded}
\begin{Highlighting}[]
\NormalTok{\#import "@preview/based:0.2.0": base64, base32, base16}

\NormalTok{\#table(}
\NormalTok{  columns: 3,}
  
\NormalTok{  table.header[*Base64*][*Base32*][*Base16*],}

\NormalTok{  raw(base64.encode("Hello world!")),}
\NormalTok{  raw(base32.encode("Hello world!")),}
\NormalTok{  raw(base16.encode("Hello world!")),}

\NormalTok{  str(base64.decode("SGVsbG8gd29ybGQh")),}
\NormalTok{  str(base32.decode("JBSWY3DPEB3W64TMMQQQ====")),}
\NormalTok{  str(base16.decode("48656C6C6F20776F726C6421"))}
\NormalTok{)}
\end{Highlighting}
\end{Shaded}

\pandocbounded{\includesvg[keepaspectratio]{https://github.com/typst/packages/raw/main/packages/preview/based/0.2.0/assets/example.svg}}

\subsubsection{How to add}\label{how-to-add}

Copy this into your project and use the import as \texttt{\ based\ }

\begin{verbatim}
#import "@preview/based:0.2.0"
\end{verbatim}

\includesvg[width=0.16667in,height=0.16667in]{/assets/icons/16-copy.svg}

Check the docs for
\href{https://typst.app/docs/reference/scripting/\#packages}{more
information on how to import packages} .

\subsubsection{About}\label{about}

\begin{description}
\tightlist
\item[Author :]
Eric Biedert
\item[License:]
MIT
\item[Current version:]
0.2.0
\item[Last updated:]
November 6, 2024
\item[First released:]
July 5, 2024
\item[Archive size:]
21.1 kB
\href{https://packages.typst.org/preview/based-0.2.0.tar.gz}{\pandocbounded{\includesvg[keepaspectratio]{/assets/icons/16-download.svg}}}
\item[Repository:]
\href{https://github.com/EpicEricEE/typst-based}{GitHub}
\item[Categor y :]
\begin{itemize}
\tightlist
\item[]
\item
  \pandocbounded{\includesvg[keepaspectratio]{/assets/icons/16-code.svg}}
  \href{https://typst.app/universe/search/?category=scripting}{Scripting}
\end{itemize}
\end{description}

\subsubsection{Where to report issues?}\label{where-to-report-issues}

This package is a project of Eric Biedert . Report issues on
\href{https://github.com/EpicEricEE/typst-based}{their repository} . You
can also try to ask for help with this package on the
\href{https://forum.typst.app}{Forum} .

Please report this package to the Typst team using the
\href{https://typst.app/contact}{contact form} if you believe it is a
safety hazard or infringes upon your rights.

\phantomsection\label{versions}
\subsubsection{Version history}\label{version-history}

\begin{longtable}[]{@{}ll@{}}
\toprule\noalign{}
Version & Release Date \\
\midrule\noalign{}
\endhead
\bottomrule\noalign{}
\endlastfoot
0.2.0 & November 6, 2024 \\
\href{https://typst.app/universe/package/based/0.1.0/}{0.1.0} & July 5,
2024 \\
\end{longtable}

Typst GmbH did not create this package and cannot guarantee correct
functionality of this package or compatibility with any version of the
Typst compiler or app.
