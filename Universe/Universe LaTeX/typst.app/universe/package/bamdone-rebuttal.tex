\title{typst.app/universe/package/bamdone-rebuttal}

\phantomsection\label{banner}
\phantomsection\label{template-thumbnail}
\pandocbounded{\includegraphics[keepaspectratio]{https://packages.typst.org/preview/thumbnails/bamdone-rebuttal-0.1.1-small.webp}}

\section{bamdone-rebuttal}\label{bamdone-rebuttal}

{ 0.1.1 }

Rebuttal/response letter template that allows authors to respond to
feedback given by reviewers in a peer-review process on a point-by-point
basis.

{ } Featured Template

\href{/app?template=bamdone-rebuttal&version=0.1.1}{Create project in
app}

\phantomsection\label{readme}
This is a Typst template for a rebuttal/response letter. It allows
authors to respond to feedback given by reviewers in a peer-review
process on a point-by-point basis. This template is based heavily on the
LaTeX template from Zenke Lab (see
\href{https://zenkelab.org/resources/latex-rebuttal-response-to-reviewers-template/}{here}
).

\subsection{Usage}\label{usage}

You can use this template in the Typst web app by clicking “Start from
template� on the dashboard and searching for
\texttt{\ bamdone-rebuttal\ } .

Alternatively, you can use the CLI to kick this project off using the
command

\begin{verbatim}
typst init @preview/bamdone-rebuttal
\end{verbatim}

Typst will create a new directory with all the files needed to get you
started.

\subsection{Configuration}\label{configuration}

This template exports the \texttt{\ rebuttal\ } function with the
following named arguments:

\begin{itemize}
\tightlist
\item
  \texttt{\ title\ } : (content), something like “Response Letter�
  (the default) or “Rebuttal�.
\item
  \texttt{\ authors\ } : (content), list of author names the top of the
  first column in boldface.
\item
  \texttt{\ date\ } : (content), defaults to
  \texttt{\ datetime.today().display()\ }
\item
  \texttt{\ paper-size\ } : Defaults to \texttt{\ us-letter\ } . Specify
  a
  \href{https://typst.app/docs/reference/layout/page/\#parameters-paper}{paper
  size string} to change the page format. Specifying this will configure
  numeric, IEEE-style citations.
\end{itemize}

The function also accepts a single, positional argument for the body of
the letter.

In addition, the template exports the \texttt{\ configure\ } function
which accepts the following named arguments corresponding to the text
color of various pieces of the letter:

\begin{itemize}
\tightlist
\item
  \texttt{\ point-color\ } : defaults to \texttt{\ blue.darken(30\%)\ }
  , the text color for reviewers’ points
\item
  \texttt{\ response-color\ } : defaults to \texttt{\ black\ } , the
  text color for the authors’ responses
\item
  \texttt{\ new-color\ } : defaults to \texttt{\ green.darken(30\%)\ } ,
  the text color for changes/additions to the manuscript (i.e., within a
  \texttt{\ quote\ } block to show what’s changed from the initial
  submission)
\end{itemize}

The template will initialize your package with a sample call to the
\texttt{\ rebuttal\ } function in a show rule.

\begin{Shaded}
\begin{Highlighting}[]
\NormalTok{// Optional color configuration}
\NormalTok{\#let (point, response, new) = configure(}
\NormalTok{  point{-}color: blue.darken(30\%),}
\NormalTok{  response{-}color: black,}
\NormalTok{  new{-}color: green.darken(30\%)}
\NormalTok{)}

\NormalTok{// Setup the rebuttal}
\NormalTok{\#show: rebuttal.with(}
\NormalTok{  authors: [First A. Author and Second B. Author],}
\NormalTok{  // date: ,}
\NormalTok{  // paper{-}size: ,}
\NormalTok{)}

\NormalTok{// Your content goes below}
\NormalTok{We thank the reviewers...}
\end{Highlighting}
\end{Shaded}

\href{/app?template=bamdone-rebuttal&version=0.1.1}{Create project in
app}

\subsubsection{How to use}\label{how-to-use}

Click the button above to create a new project using this template in
the Typst app.

You can also use the Typst CLI to start a new project on your computer
using this command:

\begin{verbatim}
typst init @preview/bamdone-rebuttal:0.1.1
\end{verbatim}

\includesvg[width=0.16667in,height=0.16667in]{/assets/icons/16-copy.svg}

\subsubsection{About}\label{about}

\begin{description}
\tightlist
\item[Author s :]
\href{https://avonmoll.github.io}{Alexander Von Moll} \&
\href{https://wwww.isaacew.com}{Isaac Weintraub}
\item[License:]
MIT-0
\item[Current version:]
0.1.1
\item[Last updated:]
November 12, 2024
\item[First released:]
May 16, 2024
\item[Minimum Typst version:]
0.12.0
\item[Archive size:]
4.26 kB
\href{https://packages.typst.org/preview/bamdone-rebuttal-0.1.1.tar.gz}{\pandocbounded{\includesvg[keepaspectratio]{/assets/icons/16-download.svg}}}
\item[Repository:]
\href{https://github.com/avonmoll/bamdone-rebuttal}{GitHub}
\item[Categor ies :]
\begin{itemize}
\tightlist
\item[]
\item
  \pandocbounded{\includesvg[keepaspectratio]{/assets/icons/16-envelope.svg}}
  \href{https://typst.app/universe/search/?category=office}{Office}
\item
  \pandocbounded{\includesvg[keepaspectratio]{/assets/icons/16-speak.svg}}
  \href{https://typst.app/universe/search/?category=report}{Report}
\end{itemize}
\end{description}

\subsubsection{Where to report issues?}\label{where-to-report-issues}

This template is a project of Alexander Von Moll and Isaac Weintraub .
Report issues on
\href{https://github.com/avonmoll/bamdone-rebuttal}{their repository} .
You can also try to ask for help with this template on the
\href{https://forum.typst.app}{Forum} .

Please report this template to the Typst team using the
\href{https://typst.app/contact}{contact form} if you believe it is a
safety hazard or infringes upon your rights.

\phantomsection\label{versions}
\subsubsection{Version history}\label{version-history}

\begin{longtable}[]{@{}ll@{}}
\toprule\noalign{}
Version & Release Date \\
\midrule\noalign{}
\endhead
\bottomrule\noalign{}
\endlastfoot
0.1.1 & November 12, 2024 \\
\href{https://typst.app/universe/package/bamdone-rebuttal/0.1.0/}{0.1.0}
& May 16, 2024 \\
\end{longtable}

Typst GmbH did not create this template and cannot guarantee correct
functionality of this template or compatibility with any version of the
Typst compiler or app.
