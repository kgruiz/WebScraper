\title{typst.app/universe/package/i-figured}

\phantomsection\label{banner}
\section{i-figured}\label{i-figured}

{ 0.2.4 }

Configurable figure and equation numbering per section.

\phantomsection\label{readme}
Configurable figure numbering per section.

\subsection{Examples}\label{examples}

\subsubsection{Basic}\label{basic}

Have a look at the source
\href{https://github.com/typst/packages/raw/main/packages/preview/i-figured/0.2.4/examples/basic.typ}{here}
.

\pandocbounded{\includegraphics[keepaspectratio]{https://github.com/typst/packages/raw/main/packages/preview/i-figured/0.2.4/examples/basic.png}}

\subsubsection{Two levels deep}\label{two-levels-deep}

Have a look at the source
\href{https://github.com/typst/packages/raw/main/packages/preview/i-figured/0.2.4/examples/level-two.typ}{here}
.

\pandocbounded{\includegraphics[keepaspectratio]{https://github.com/typst/packages/raw/main/packages/preview/i-figured/0.2.4/examples/level-two.png}}

\subsection{Usage}\label{usage}

The package mainly consists of two customizable show rules, which set up
all the numbering. There is also an additional function to make showing
an outline of figures easier.

Because the
\href{https://github.com/typst/packages/raw/main/packages/preview/i-figured/0.2.4/\#show-figure}{\texttt{\ show-figure()\ }}
function must internally create another figure element, attached labels
cannot directly be used for references. To circumvent this, a new label
is attached to the internal figure, with the same name but prefixed with
\texttt{\ fig:\ } , \texttt{\ tbl:\ } , or \texttt{\ lst:\ } for images
(and all other types of generic figures), tables, and raw code figures
(aka listings) respectively. These new labels can be used for
referencing without problems.

\begin{Shaded}
\begin{Highlighting}[]
\NormalTok{// import the package}
\NormalTok{\#import "@preview/i{-}figured:0.2.4"}

\NormalTok{// make sure you have some heading numbering set}
\NormalTok{\#set heading(numbering: "1.")}

\NormalTok{// apply the show rules (these can be customized)}
\NormalTok{\#show heading: i{-}figured.reset{-}counters}
\NormalTok{\#show figure: i{-}figured.show{-}figure}

\NormalTok{// show an outline}
\NormalTok{\#i{-}figured.outline()}

\NormalTok{= Hello World}

\NormalTok{\#figure([hi], caption: [Bye World.]) \textless{}bye\textgreater{}}

\NormalTok{// when referencing, the label names must be prefixed with \textasciigrave{}fig:\textasciigrave{}, \textasciigrave{}tbl:\textasciigrave{},}
\NormalTok{// or \textasciigrave{}lst:\textasciigrave{} depending on the figure kind.}
\NormalTok{@fig:bye displays the text "hi".}
\end{Highlighting}
\end{Shaded}

\subsection{Reference}\label{reference}

\subsubsection{\texorpdfstring{\texttt{\ reset-counters\ }}{ reset-counters }}\label{reset-counters}

Reset all figure counters. To be used in a heading show rule like
\texttt{\ \#show\ heading:\ i-figured.reset-counters\ } .

\begin{Shaded}
\begin{Highlighting}[]
\NormalTok{\#let reset{-}counters(}
\NormalTok{  it,}
\NormalTok{  level: 1,}
\NormalTok{  extra{-}kinds: (),}
\NormalTok{  equations: true,}
\NormalTok{  return{-}orig{-}heading: true,}
\NormalTok{) = \{ .. \}}
\end{Highlighting}
\end{Shaded}

\textbf{Arguments:}

\begin{itemize}
\tightlist
\item
  \texttt{\ it\ } :
  \href{https://typst.app/docs/reference/foundations/content/}{\texttt{\ content\ }}
  â€'' The heading element from the show rule.
\item
  \texttt{\ level\ } :
  \href{https://typst.app/docs/reference/foundations/int/}{\texttt{\ int\ }}
  â€'' At which heading level to reset the counters. A value of
  \texttt{\ 2\ } will cause the counters to be reset at level two
  \emph{and} level one headings.
\item
  \texttt{\ extra-kinds\ } :
  \href{https://typst.app/docs/reference/foundations/array/}{\texttt{\ array\ }}
  of (
  \href{https://typst.app/docs/reference/foundations/str/}{\texttt{\ str\ }}
  or
  \href{https://typst.app/docs/reference/foundations/function/}{\texttt{\ function\ }}
  ) â€'' Additional custom figure kinds. If you have any figures with a
  \texttt{\ kind\ } other than \texttt{\ image\ } , \texttt{\ table\ } ,
  or \texttt{\ raw\ } , you must add the \texttt{\ kind\ } here for its
  counter to be reset.
\item
  \texttt{\ equations\ } :
  \href{https://typst.app/docs/reference/foundations/bool/}{\texttt{\ bool\ }}
  â€'' Whether the counter for math equations should be reset.
\item
  \texttt{\ return-orig-heading\ } :
  \href{https://typst.app/docs/reference/foundations/bool/}{\texttt{\ bool\ }}
  â€'' Whether the original heading element should be included in the
  returned content. Set this to false if you manually want to construct
  a heading instead of using the default.
\end{itemize}

\textbf{Returns:}
\href{https://typst.app/docs/reference/foundations/content/}{\texttt{\ content\ }}
â€'' The unmodified heading.

\subsubsection{\texorpdfstring{\texttt{\ show-figure\ }}{ show-figure }}\label{show-figure}

Show a figure with per-section numbering. To be used in a figure show
rule like \texttt{\ \#show\ figure:\ i-figured.show-figure\ } .

\begin{Shaded}
\begin{Highlighting}[]
\NormalTok{\#let show{-}figure(}
\NormalTok{  it,}
\NormalTok{  level: 1,}
\NormalTok{  zero{-}fill: true,}
\NormalTok{  leading{-}zero: true,}
\NormalTok{  numbering: "1.1",}
\NormalTok{  extra{-}prefixes: (:),}
\NormalTok{  fallback{-}prefix: "fig:",}
\NormalTok{) = \{ .. \}}
\end{Highlighting}
\end{Shaded}

\textbf{Arguments:}

\begin{itemize}
\tightlist
\item
  \texttt{\ it\ } :
  \href{https://typst.app/docs/reference/foundations/content/}{\texttt{\ content\ }}
  â€'' The figure element from the show rule.
\item
  \texttt{\ level\ } :
  \href{https://typst.app/docs/reference/foundations/int/}{\texttt{\ int\ }}
  â€'' How many levels of the current heading counter should be added in
  front. Note that you can control this individually from the
  \texttt{\ level\ } parameter on
  \href{https://github.com/typst/packages/raw/main/packages/preview/i-figured/0.2.4/\#reset-counters}{\texttt{\ reset-counters()\ }}
  .
\item
  \texttt{\ zero-fill\ } :
  \href{https://typst.app/docs/reference/foundations/bool/}{\texttt{\ bool\ }}
  â€'' If \texttt{\ true\ } and assuming a \texttt{\ level\ } of
  \texttt{\ 2\ } , a figure after a \texttt{\ 1.\ } heading but before a
  \texttt{\ 1.1.\ } heading will show \texttt{\ 1.0.1\ } as numbering,
  else the middle zero is excluded. Note that if set to
  \texttt{\ false\ } , not all figure numberings are guaranteed to have
  the same length.
\item
  \texttt{\ leading-zero\ } :
  \href{https://typst.app/docs/reference/foundations/bool/}{\texttt{\ bool\ }}
  â€'' Whether figures before the first top-level heading should have a
  leading \texttt{\ 0\ } . Note that if set to \texttt{\ false\ } , not
  all figure numberings are guaranteed to have the same length.
\item
  \texttt{\ numbering\ } :
  \href{https://typst.app/docs/reference/foundations/str/}{\texttt{\ str\ }}
  or
  \href{https://typst.app/docs/reference/foundations/function/}{\texttt{\ function\ }}
  â€'' The actual numbering pattern to use for the figures.
\item
  \texttt{\ extra-prefixes\ } :
  \href{https://typst.app/docs/reference/foundations/dictionary/}{\texttt{\ dictionary\ }}
  of
  \href{https://typst.app/docs/reference/foundations/str/}{\texttt{\ str\ }}
  to
  \href{https://typst.app/docs/reference/foundations/str/}{\texttt{\ str\ }}
  pairs â€'' Additional label prefixes. This can optionally be used to
  specify prefixes for custom figure kinds, otherwise they will also use
  the fallback prefix.
\item
  \texttt{\ fallback-prefix\ } :
  \href{https://typst.app/docs/reference/foundations/str/}{\texttt{\ str\ }}
  â€'' The label prefix to use for figure kinds which don’t have
  another prefix set.
\end{itemize}

\textbf{Returns:}
\href{https://typst.app/docs/reference/foundations/content/}{\texttt{\ content\ }}
â€'' The modified figure.

\subsubsection{\texorpdfstring{\texttt{\ show-equation\ }}{ show-equation }}\label{show-equation}

Show a math equation with per-section numbering. To be used in a show
rule like \texttt{\ \#show\ math.equation:\ i-figured.show-equation\ } .

\begin{Shaded}
\begin{Highlighting}[]
\NormalTok{\#let show{-}equation(}
\NormalTok{  it,}
\NormalTok{  level: 1,}
\NormalTok{  zero{-}fill: true,}
\NormalTok{  leading{-}zero: true,}
\NormalTok{  numbering: "(1.1)",}
\NormalTok{  prefix: "eqt:",}
\NormalTok{  only{-}labeled: false,}
\NormalTok{  unnumbered{-}label: "{-}",}
\NormalTok{) = \{ .. \}}
\end{Highlighting}
\end{Shaded}

\textbf{Arguments:}

For the arguments \texttt{\ level\ } , \texttt{\ zero-fill\ } ,
\texttt{\ leading-zero\ } , and \texttt{\ numbering\ } refer to
\href{https://github.com/typst/packages/raw/main/packages/preview/i-figured/0.2.4/\#show-figure}{\texttt{\ show-figure()\ }}
.

\begin{itemize}
\tightlist
\item
  \texttt{\ it\ } :
  \href{https://typst.app/docs/reference/foundations/content/}{\texttt{\ content\ }}
  â€'' The equation element from the show rule.
\item
  \texttt{\ prefix\ } :
  \href{https://typst.app/docs/reference/foundations/str/}{\texttt{\ str\ }}
  â€'' The label prefix to use for all equations.
\item
  \texttt{\ only-labeled\ } :
  \href{https://typst.app/docs/reference/foundations/bool/}{\texttt{\ bool\ }}
  â€'' Whether only equations with labels should be numbered.
\item
  \texttt{\ unnumbered-label\ } :
  \href{https://typst.app/docs/reference/foundations/str/}{\texttt{\ str\ }}
  â€'' A label to explicitly disable numbering for an equation.
\end{itemize}

\textbf{Returns:}
\href{https://typst.app/docs/reference/foundations/content/}{\texttt{\ content\ }}
â€'' The modified equation.

\subsubsection{\texorpdfstring{\texttt{\ outline\ }}{ outline }}\label{outline}

Show the outline for a kind of figure. This is just the same as calling
\texttt{\ outline(target:\ figure.where(kind:\ i-figured.\_prefix\ +\ repr(target-kind)),\ ..)\ }
, the function just exists for convenience and clarity.

\begin{Shaded}
\begin{Highlighting}[]
\NormalTok{\#let outline(target{-}kind: image, title: [List of Figures], ..args) = \{ .. \}}
\end{Highlighting}
\end{Shaded}

\textbf{Arguments:}

\begin{itemize}
\tightlist
\item
  \texttt{\ target-kind\ } :
  \href{https://typst.app/docs/reference/foundations/str/}{\texttt{\ str\ }}
  or
  \href{https://typst.app/docs/reference/foundations/function/}{\texttt{\ function\ }}
  â€'' Which kind of figure to list.
\item
  \texttt{\ title\ } :
  \href{https://typst.app/docs/reference/foundations/content/}{\texttt{\ content\ }}
  or \texttt{\ none\ } â€'' The title of the outline.
\item
  \texttt{\ ..args\ } â€'' Other arguments to pass to the underlying
  \href{https://typst.app/docs/reference/meta/outline/}{\texttt{\ outline()\ }}
  call.
\end{itemize}

\textbf{Returns:}
\href{https://typst.app/docs/reference/foundations/content/}{\texttt{\ content\ }}
â€'' The outline element.

\subsection{Acknowledgements}\label{acknowledgements}

The core code is based off code from
\href{https://github.com/PgBiel}{@PgBiel} ( \texttt{\ @PgSuper\ } on
Discord) and \href{https://github.com/aagolovanov}{@aagolovanov} (
\texttt{\ @aag.\ } on Discord). Specifically from
\href{https://discord.com/channels/1054443721975922748/1088371919725793360/1158534418760224809}{this
message} and the conversation around
\href{https://discord.com/channels/1054443721975922748/1088371919725793360/1159172567282749561}{here}
.

\subsubsection{How to add}\label{how-to-add}

Copy this into your project and use the import as \texttt{\ i-figured\ }

\begin{verbatim}
#import "@preview/i-figured:0.2.4"
\end{verbatim}

\includesvg[width=0.16667in,height=0.16667in]{/assets/icons/16-copy.svg}

Check the docs for
\href{https://typst.app/docs/reference/scripting/\#packages}{more
information on how to import packages} .

\subsubsection{About}\label{about}

\begin{description}
\tightlist
\item[Author :]
RubixDev
\item[License:]
MIT
\item[Current version:]
0.2.4
\item[Last updated:]
January 26, 2024
\item[First released:]
October 9, 2023
\item[Archive size:]
1.97 kB
\href{https://packages.typst.org/preview/i-figured-0.2.4.tar.gz}{\pandocbounded{\includesvg[keepaspectratio]{/assets/icons/16-download.svg}}}
\item[Repository:]
\href{https://github.com/RubixDev/typst-i-figured}{GitHub}
\end{description}

\subsubsection{Where to report issues?}\label{where-to-report-issues}

This package is a project of RubixDev . Report issues on
\href{https://github.com/RubixDev/typst-i-figured}{their repository} .
You can also try to ask for help with this package on the
\href{https://forum.typst.app}{Forum} .

Please report this package to the Typst team using the
\href{https://typst.app/contact}{contact form} if you believe it is a
safety hazard or infringes upon your rights.

\phantomsection\label{versions}
\subsubsection{Version history}\label{version-history}

\begin{longtable}[]{@{}ll@{}}
\toprule\noalign{}
Version & Release Date \\
\midrule\noalign{}
\endhead
\bottomrule\noalign{}
\endlastfoot
0.2.4 & January 26, 2024 \\
\href{https://typst.app/universe/package/i-figured/0.2.3/}{0.2.3} &
December 11, 2023 \\
\href{https://typst.app/universe/package/i-figured/0.2.2/}{0.2.2} &
December 7, 2023 \\
\href{https://typst.app/universe/package/i-figured/0.2.1/}{0.2.1} &
November 19, 2023 \\
\href{https://typst.app/universe/package/i-figured/0.2.0/}{0.2.0} &
November 16, 2023 \\
\href{https://typst.app/universe/package/i-figured/0.1.0/}{0.1.0} &
October 9, 2023 \\
\end{longtable}

Typst GmbH did not create this package and cannot guarantee correct
functionality of this package or compatibility with any version of the
Typst compiler or app.
