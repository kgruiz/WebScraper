\title{typst.app/universe/package/big-rati}

\phantomsection\label{banner}
\section{big-rati}\label{big-rati}

{ 0.1.0 }

Utilities to work with big rational numbers in Typst

\phantomsection\label{readme}
\texttt{\ big-rati\ } is a package to work with rational numbers in
Typst

\subsection{Usage}\label{usage}

\begin{Shaded}
\begin{Highlighting}[]
\NormalTok{\#import "@preview/big{-}rati:0.1.0"}

\NormalTok{\#let a = 2      // 2/1}
\NormalTok{\#let b = (1, 2) // 1/2}

\NormalTok{\#let sum = big{-}rati.add(a, b) // 5/2}

\NormalTok{\#let c = ("4", 6)}
\NormalTok{\#let prod = big{-}rati.mul(c, sum) // 5/3}

\NormalTok{$\#big{-}rati.repr(prod)$}
\end{Highlighting}
\end{Shaded}

Functions, exported by the package are:

\begin{Shaded}
\begin{Highlighting}[]
\NormalTok{// Converts \textasciigrave{}x\textasciigrave{} to bytes, representing the rational number,}
\NormalTok{// that can be used in the functions below.}
\NormalTok{// \textasciigrave{}x\textasciigrave{} might be an integer or a big integer string.}
\NormalTok{// If \textasciigrave{}x\textasciigrave{} is an array of length two, which elements are integers}
\NormalTok{// or big integer strings, then it is converted to the array of all}
\NormalTok{// big integer strings, and then into the bytes representation.}
\NormalTok{\#let rational(x)}

\NormalTok{// Functions below work with "rational numbers", integers or big integer strings}

\NormalTok{// Returns \textasciigrave{}a + b\textasciigrave{}}
\NormalTok{\#let add(a, b)}

\NormalTok{// Returns \textasciigrave{}a {-} b\textasciigrave{}}
\NormalTok{\#let sub(a, b)}

\NormalTok{// Returns \textasciigrave{}a / b\textasciigrave{}}
\NormalTok{\#let div(a, b)}

\NormalTok{// Returns \textasciigrave{}a * b\textasciigrave{}}
\NormalTok{\#let mul(a, b)}

\NormalTok{// Returns \textasciigrave{}a \% b\textasciigrave{}}
\NormalTok{\#let rem(a, b)}

\NormalTok{// Returns \textasciigrave{}|a {-} b|\textasciigrave{}}
\NormalTok{\#let abs{-}diff(a, b)}

\NormalTok{// Returns \textasciigrave{}{-}1\textasciigrave{} if \textasciigrave{}a \textless{} b\textasciigrave{}, \textasciigrave{}0\textasciigrave{} if \textasciigrave{}a == b\textasciigrave{}, \textasciigrave{}1\textasciigrave{} if \textasciigrave{}a \textgreater{} b\textasciigrave{}}
\NormalTok{\#let cmp(a, b)}

\NormalTok{// Returns \textasciigrave{}{-}x\textasciigrave{}}
\NormalTok{\#let neg(x)}

\NormalTok{// Returns \textasciigrave{}|x|\textasciigrave{}}
\NormalTok{\#let abs(x)}

\NormalTok{// Rounds towards plus infinity}
\NormalTok{\#let ceil(x)}

\NormalTok{// Rounds towards minus infinity}
\NormalTok{\#let floor(x)}

\NormalTok{// Rounds to the nearest integer. Rounds half{-}way cases away from zero.}
\NormalTok{\#let round(x)}

\NormalTok{// Rounds towards zero.}
\NormalTok{\#let trunc(x)}

\NormalTok{// Returns the fractional part of a number, with division rounded towards zero.}
\NormalTok{// Satisfies \textasciigrave{}number == add(trunc(number), fract(number))\textasciigrave{}.}
\NormalTok{\#let fract(number)}

\NormalTok{// Returns the reciprocal.}
\NormalTok{// Panics if the number is zero.}
\NormalTok{\#let recip(x)}

\NormalTok{// Returns \textasciigrave{}x\^{}y\textasciigrave{}. \textasciigrave{}y\textasciigrave{} must be an \textasciigrave{}int\textasciigrave{}, in range of \textasciigrave{}{-}2\^{}32\textasciigrave{} to \textasciigrave{}2\^{}32 {-} 1\textasciigrave{}}
\NormalTok{\#let pow(x, y)}

\NormalTok{// Restrict a value to a certain interval.}
\NormalTok{//}
\NormalTok{// Returns \textasciigrave{}max\textasciigrave{} if \textasciigrave{}number\textasciigrave{} is greater than \textasciigrave{}max\textasciigrave{},}
\NormalTok{// and \textasciigrave{}min\textasciigrave{} if \textasciigrave{}number\textasciigrave{} is less than \textasciigrave{}min\textasciigrave{}.}
\NormalTok{// Otherwise returns \textasciigrave{}number\textasciigrave{}.}
\NormalTok{//}
\NormalTok{// Returns error if \textasciigrave{}min\textasciigrave{} is greater than \textasciigrave{}max\textasciigrave{}.}
\NormalTok{\#let clamp(number, min, max)}

\NormalTok{// Compares and returns the minimum of two values.}
\NormalTok{\#let min(a, b)}

\NormalTok{// Compares and returns the maximum of two values.}
\NormalTok{\#let max(a, b)}

\NormalTok{// Returns a value of type \textasciigrave{}content\textasciigrave{}, representing the rational number.}
\NormalTok{// If \textasciigrave{}is{-}mixed\textasciigrave{} is true, then the result is a mixed fraction,}
\NormalTok{// otherwise, it is a simple fraction.}
\NormalTok{\#let repr(x, is{-}mixed: true)}

\NormalTok{// Returns a string, representing the rational number}
\NormalTok{\#let to{-}decimal{-}str(x, precision: 8)}

\NormalTok{// Returns a floating{-}point number, representing the rational number}
\NormalTok{\#let to{-}float(x, precision: 8)}

\NormalTok{// Returns a decimal number, representing the rational number}
\NormalTok{\#let to{-}decimal(x, precision: 8)}
\end{Highlighting}
\end{Shaded}

\subsubsection{How to add}\label{how-to-add}

Copy this into your project and use the import as \texttt{\ big-rati\ }

\begin{verbatim}
#import "@preview/big-rati:0.1.0"
\end{verbatim}

\includesvg[width=0.16667in,height=0.16667in]{/assets/icons/16-copy.svg}

Check the docs for
\href{https://typst.app/docs/reference/scripting/\#packages}{more
information on how to import packages} .

\subsubsection{About}\label{about}

\begin{description}
\tightlist
\item[Author :]
Danik Vitek
\item[License:]
MIT
\item[Current version:]
0.1.0
\item[Last updated:]
October 29, 2024
\item[First released:]
October 29, 2024
\item[Archive size:]
33.9 kB
\href{https://packages.typst.org/preview/big-rati-0.1.0.tar.gz}{\pandocbounded{\includesvg[keepaspectratio]{/assets/icons/16-download.svg}}}
\item[Repository:]
\href{https://github.com/DanikVitek/typst-plugin-bigrational}{GitHub}
\item[Discipline :]
\begin{itemize}
\tightlist
\item[]
\item
  \href{https://typst.app/universe/search/?discipline=mathematics}{Mathematics}
\end{itemize}
\item[Categor y :]
\begin{itemize}
\tightlist
\item[]
\item
  \pandocbounded{\includesvg[keepaspectratio]{/assets/icons/16-code.svg}}
  \href{https://typst.app/universe/search/?category=scripting}{Scripting}
\end{itemize}
\end{description}

\subsubsection{Where to report issues?}\label{where-to-report-issues}

This package is a project of Danik Vitek . Report issues on
\href{https://github.com/DanikVitek/typst-plugin-bigrational}{their
repository} . You can also try to ask for help with this package on the
\href{https://forum.typst.app}{Forum} .

Please report this package to the Typst team using the
\href{https://typst.app/contact}{contact form} if you believe it is a
safety hazard or infringes upon your rights.

\phantomsection\label{versions}
\subsubsection{Version history}\label{version-history}

\begin{longtable}[]{@{}ll@{}}
\toprule\noalign{}
Version & Release Date \\
\midrule\noalign{}
\endhead
\bottomrule\noalign{}
\endlastfoot
0.1.0 & October 29, 2024 \\
\end{longtable}

Typst GmbH did not create this package and cannot guarantee correct
functionality of this package or compatibility with any version of the
Typst compiler or app.
