\title{typst.app/universe/package/uo-pup-thesis-manuscript}

\phantomsection\label{banner}
\phantomsection\label{template-thumbnail}
\pandocbounded{\includegraphics[keepaspectratio]{https://packages.typst.org/preview/thumbnails/uo-pup-thesis-manuscript-0.1.0-small.webp}}

\section{uo-pup-thesis-manuscript}\label{uo-pup-thesis-manuscript}

{ 0.1.0 }

Unofficial Typst template for PUP (Polytechnic University of the
Philippines) undergraduate thesis manuscript

\href{/app?template=uo-pup-thesis-manuscript&version=0.1.0}{Create
project in app}

\phantomsection\label{readme}
Unofficial \href{https://typst.app/}{typst} template for undergraduate
thesis manuscript for PUP (Polytechnic University of the Philippines).
This template adheres to the University’s Thesis and Dissertation
Manual as of 2017 (ISBN: 978-971â€``95208-8-7 (Online)). An example
manuscript is also provided (see \texttt{\ ./thesis.pdf\ } ).

\subsection{Setup}\label{setup}

Using
\href{https://github.com/typst/typst?tab=readme-ov-file\#installation}{Typst
CLI} :

\begin{Shaded}
\begin{Highlighting}[]
\ExtensionTok{typst}\NormalTok{ init @preview/uo{-}pup{-}thesis{-}manuscript my{-}thesis}
\BuiltInTok{cd}\NormalTok{ my{-}thesis}
\ExtensionTok{typst}\NormalTok{ compile thesis.typ  }\CommentTok{\# to compile to PDF}
\end{Highlighting}
\end{Shaded}

or run

\begin{Shaded}
\begin{Highlighting}[]
\ExtensionTok{typst}\NormalTok{ watch thesis.typ  }\CommentTok{\# to automatically compiles PDF on save}
\end{Highlighting}
\end{Shaded}

\subsection{Usage}\label{usage}

The template already provided an example structure and some guides. But
to start from nothing, make an entrypoint file with a basic structure
like this:

\begin{Shaded}
\begin{Highlighting}[]
\NormalTok{// thesis.typ}
\NormalTok{\#import "@preview/uo{-}pup{-}thesis{-}manuscript:0.1.0": *}


\NormalTok{\#show: template.with(}
\NormalTok{  [\textless{}your thesis title here\textgreater{}],}
\NormalTok{  ("Author 1", "Author 2", ..., "Author N"),}
\NormalTok{  "name of your college here",}
\NormalTok{  "name of your deg. program here",}
\NormalTok{  "Month YYYY"}
\NormalTok{)}


\NormalTok{// Main content starts here}

\NormalTok{// This provides a customized heading for}
\NormalTok{// chapters that follows the manual}
\NormalTok{\#chapter(1, "Chapter 1 Title") }

\NormalTok{// Since \#chapter() provides a heading level 1,}
\NormalTok{// start each headings under chapters with level 2}
\NormalTok{// to avoid messing up the generated Table of Contents}
\NormalTok{== Introduction}

\NormalTok{...}

\NormalTok{\#chapter(2, "Chapter 2 Title")}

\NormalTok{== Topic A}

\NormalTok{...}

\NormalTok{// End of main content}


\NormalTok{// Bibliography formatting setup}
\NormalTok{\#set par(first{-}line{-}indent: 0pt, hanging{-}indent: 0.5in)}
\NormalTok{\#set page(header: context [\#h(1fr) \#counter(page).get().first()])}
\NormalTok{\#align(center)[ \#heading("REFERENCES") ]}
\NormalTok{\#set par(spacing: 1.5em)}

\NormalTok{// Get the apa.csl file from \textasciigrave{}template/\textasciigrave{} folder}
\NormalTok{\#bibliography(title: none, style: "./apa.csl", "path/to/your/bibtex/file.bib")}


\NormalTok{// Appendices}
\NormalTok{\#show: appendices{-}section}

\NormalTok{\#appendix(1, "Appendix Title")}

\NormalTok{...}

\NormalTok{\#pagebreak()}

\NormalTok{\#appendix(2, "Appendix Title")}

\NormalTok{...}
\end{Highlighting}
\end{Shaded}

There are also provided utilities for some parts that have a specific
way of formatting.

For example, in \texttt{\ Definition\ of\ Terms\ } and
\texttt{\ Significance\ of\ the\ Study\ } sections, use
\texttt{\ \#description\ } function:

\begin{Shaded}
\begin{Highlighting}[]
\NormalTok{== Significance of the Study}
\NormalTok{\#description(}
\NormalTok{  (}
\NormalTok{    (term: [Topic A], desc: [\#lorem(30)]),}
\NormalTok{    (term: [Topic B], desc: [\#lorem(30)]),}
\NormalTok{    (term: [Topic C], desc: [\#lorem(30)]),}
\NormalTok{  )}
\NormalTok{)}

\NormalTok{...}

\NormalTok{== Definition of Terms}
\NormalTok{\#description(}
\NormalTok{  (}
\NormalTok{    (term: [Topic A], desc: [\#lorem(30)]),}
\NormalTok{    (term: [Topic B], desc: [\#lorem(30)]),}
\NormalTok{  )}
\NormalTok{)}
\end{Highlighting}
\end{Shaded}

\begin{center}\rule{0.5\linewidth}{0.5pt}\end{center}

If there’s any mistakes, wrong formatting (e.g., not actually
following the manual), etc., file an issue or a pull request.

\begin{center}\rule{0.5\linewidth}{0.5pt}\end{center}

\subsection{TODO}\label{todo}

\begin{itemize}
\tightlist
\item
  {[} {]} Chapter 4
\item
  {[} {]} Chapter 5
\item
  {[} {]} Abstract
\item
  {[} {]} Acknowledgement
\item
  {[} {]} Copyright
\item
  If possible:

  \begin{itemize}
  \tightlist
  \item
    Approval Sheet
  \item
    Certificate of Originality
  \end{itemize}
\end{itemize}

\href{/app?template=uo-pup-thesis-manuscript&version=0.1.0}{Create
project in app}

\subsubsection{How to use}\label{how-to-use}

Click the button above to create a new project using this template in
the Typst app.

You can also use the Typst CLI to start a new project on your computer
using this command:

\begin{verbatim}
typst init @preview/uo-pup-thesis-manuscript:0.1.0
\end{verbatim}

\includesvg[width=0.16667in,height=0.16667in]{/assets/icons/16-copy.svg}

\subsubsection{About}\label{about}

\begin{description}
\tightlist
\item[Author :]
\href{https://github.com/datsudo}{Datsudo}
\item[License:]
MIT
\item[Current version:]
0.1.0
\item[Last updated:]
November 4, 2024
\item[First released:]
November 4, 2024
\item[Archive size:]
21.6 kB
\href{https://packages.typst.org/preview/uo-pup-thesis-manuscript-0.1.0.tar.gz}{\pandocbounded{\includesvg[keepaspectratio]{/assets/icons/16-download.svg}}}
\item[Repository:]
\href{https://gitlab.com/datsudo/uo-pup-thesis-manuscript}{GitLab}
\item[Categor y :]
\begin{itemize}
\tightlist
\item[]
\item
  \pandocbounded{\includesvg[keepaspectratio]{/assets/icons/16-mortarboard.svg}}
  \href{https://typst.app/universe/search/?category=thesis}{Thesis}
\end{itemize}
\end{description}

\subsubsection{Where to report issues?}\label{where-to-report-issues}

This template is a project of Datsudo . Report issues on
\href{https://gitlab.com/datsudo/uo-pup-thesis-manuscript}{their
repository} . You can also try to ask for help with this template on the
\href{https://forum.typst.app}{Forum} .

Please report this template to the Typst team using the
\href{https://typst.app/contact}{contact form} if you believe it is a
safety hazard or infringes upon your rights.

\phantomsection\label{versions}
\subsubsection{Version history}\label{version-history}

\begin{longtable}[]{@{}ll@{}}
\toprule\noalign{}
Version & Release Date \\
\midrule\noalign{}
\endhead
\bottomrule\noalign{}
\endlastfoot
0.1.0 & November 4, 2024 \\
\end{longtable}

Typst GmbH did not create this template and cannot guarantee correct
functionality of this template or compatibility with any version of the
Typst compiler or app.
