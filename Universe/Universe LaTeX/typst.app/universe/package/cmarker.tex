\title{typst.app/universe/package/cmarker}

\phantomsection\label{banner}
\section{cmarker}\label{cmarker}

{ 0.1.1 }

Transpile CommonMark Markdown to Typst, from within Typst!

\phantomsection\label{readme}
\#set document(title: "cmarker.typ") \#set page(numbering: "1",
number-align: center) \#set text(lang: "en") \#align(center,
text(weight: 700, 1.75em){[}cmarker.typ{]}) \#set heading(numbering:
"1.") \#show link: c =\textgreater{} text(underline(c), fill: blue)
\#set image(height: 2em) \#show par: set block(above: 1.2em, below:
1.2em) \#align(center){[}https://github.com/SabrinaJewson/cmarker.typ{]}
\#"

This package enables you to write CommonMark Markdown, and import it
directly into Typst.

\begin{longtable}[]{@{}
  >{\raggedright\arraybackslash}p{(\linewidth - 2\tabcolsep) * \real{0.5000}}
  >{\raggedright\arraybackslash}p{(\linewidth - 2\tabcolsep) * \real{0.5000}}@{}}
\toprule\noalign{}
\endhead
\bottomrule\noalign{}
\endlastfoot
\texttt{\ simple.typ\ } & \texttt{\ simple.md\ } \\
\begin{minipage}[t]{\linewidth}\raggedright
\begin{Shaded}
\begin{Highlighting}[]
\NormalTok{\#import "@preview/cmarker:0.1.1"}

\NormalTok{\#cmarker.render(read("simple.md"))}
\end{Highlighting}
\end{Shaded}
\end{minipage} & \begin{minipage}[t]{\linewidth}\raggedright
\begin{Shaded}
\begin{Highlighting}[]
\FunctionTok{\# We can write Markdown!}

\NormalTok{*Using* \_\_lots\_\_ \textasciitilde{}of\textasciitilde{} }\InformationTok{\textasciigrave{}fancy\textasciigrave{}} \CommentTok{[}\OtherTok{features}\CommentTok{](https://example.org/)}\NormalTok{.}
\end{Highlighting}
\end{Shaded}
\end{minipage} \\
\end{longtable}

\begin{longtable}[]{@{}l@{}}
\toprule\noalign{}
\endhead
\bottomrule\noalign{}
\endlastfoot
\texttt{\ simple.pdf\ } \\
\pandocbounded{\includegraphics[keepaspectratio]{https://github.com/typst/packages/raw/main/packages/preview/cmarker/0.1.1/examples/simple.png}} \\
\end{longtable}

This document is available in
\href{https://github.com/SabrinaJewson/cmarker.typ/tree/main\#cmarker}{Markdown}
and
\href{https://github.com/SabrinaJewson/cmarker.typ/blob/main/README.pdf}{rendered
PDF} formats.

\subsection{API}\label{api}

We offer a single function:

\begin{Shaded}
\begin{Highlighting}[]
\NormalTok{render(}
\NormalTok{  markdown,}
\NormalTok{  smart{-}punctuation: true,}
\NormalTok{  blockquote: none,}
\NormalTok{  math: none,}
\NormalTok{  h1{-}level: 1,}
\NormalTok{  raw{-}typst: true,}
\NormalTok{  scope: (:),}
\NormalTok{  show{-}source: false,}
\NormalTok{) {-}\textgreater{} content}
\end{Highlighting}
\end{Shaded}

The parameters are as follows:

\begin{itemize}
\item
  \texttt{\ markdown\ } : The
  \href{https://spec.commonmark.org/0.30/}{CommonMark} Markdown string
  to be processed. Parsed with the
  \href{https://docs.rs/pulldown-cmark}{pulldown-cmark} Rust library.
  You can set this to \texttt{\ read("somefile.md")\ } to import an
  external markdown file; see the
  \href{https://typst.app/docs/reference/data-loading/read/}{documentation
  for the read function} .

  \begin{itemize}
  \tightlist
  \item
    Accepted values: Strings and raw text code blocks.
  \item
    Required.
  \end{itemize}
\item
  \texttt{\ smart-punctuation\ } : Automatically convert ASCII
  punctuation to Unicode equivalents:

  \begin{itemize}
  \tightlist
  \item
    nondirectional quotations (" and \textquotesingle) become
    directional (“� and ‘’);
  \item
    three consecutive full stops (…) become ellipses (…);
  \item
    two and three consecutive hypen-minus signs (-\/- and â€'') become
    en and em dashes (â€`` and â€'').
  \end{itemize}

  Note that although Typst also offers this functionality, this
  conversion is done through the Markdown parser rather than Typst.

  \begin{itemize}
  \tightlist
  \item
    Accepted values: Booleans.
  \item
    Default value: \texttt{\ true\ } .
  \end{itemize}
\item
  \texttt{\ blockquote\ } : A callback to be used when a blockquote is
  encountered in the Markdown, or \texttt{\ none\ } if blockquotes
  should be treated as normal text. Because Typst does not support
  blockquotes natively, the user must configure this.

  \begin{itemize}
  \tightlist
  \item
    Accepted values: Functions accepting content and returning content,
    or \texttt{\ none\ } .
  \item
    Default value: \texttt{\ none\ } .
  \end{itemize}

  For example, to display a black border to the left of the text one can
  use:

\begin{Shaded}
\begin{Highlighting}[]
\NormalTok{box.with(stroke: (left: 1pt + black), inset: (left: 5pt, y: 6pt))}
\end{Highlighting}
\end{Shaded}
\item
  \texttt{\ math\ } : A callback to be used when equations are
  encountered in the Markdown, or \texttt{\ none\ } if it should be
  treated as normal text. Because Typst does not support LaTeX equations
  natively, the user must configure this.

  \begin{itemize}
  \tightlist
  \item
    Accepted values: Functions that take a boolean argument named
    \texttt{\ block\ } and a positional string argument (often, the
    \texttt{\ mitex\ } function from
    \href{https://typst.app/universe/package/mitex}{the mitex package}
    ), or \texttt{\ none\ } .
  \item
    Default value: \texttt{\ none\ } .
  \end{itemize}

  For example, to render math equation as a Typst math block, one can
  use:

\begin{Shaded}
\begin{Highlighting}[]
\NormalTok{\#import "@preview/mitex:0.2.4": mitex}
\NormalTok{\#cmarker.render(\textasciigrave{}$\textbackslash{}int\_1\^{}2 x \textbackslash{}mathrm\{d\} x$\textasciigrave{}, math: mitex)}
\end{Highlighting}
\end{Shaded}
\item
  \texttt{\ h1-level\ } : The level that top-level headings in Markdown
  should get in Typst. When set to zero, top-level headings are treated
  as text, \texttt{\ \#\#\ } headings become \texttt{\ =\ } headings,
  \texttt{\ \#\#\#\ } headings become \texttt{\ ==\ } headings, et
  cetera; when set to \texttt{\ 2\ } , \texttt{\ \#\ } headings become
  \texttt{\ ==\ } headings, \texttt{\ \#\#\ } headings become
  \texttt{\ ===\ } headings, et cetera.

  \begin{itemize}
  \tightlist
  \item
    Accepted values: Integers in the range {[}0, 255{]}.
  \item
    Default value: 1.
  \end{itemize}
\item
  \texttt{\ raw-typst\ } : Whether to allow raw Typst code to be
  injected into the document via HTML comments. If disabled, the
  comments will act as regular HTML comments.

  \begin{itemize}
  \tightlist
  \item
    Accepted values: Booleans.
  \item
    Default value: \texttt{\ true\ } .
  \end{itemize}

  For example, when this is enabled,
  \texttt{\ \textless{}!-\/-raw-typst\ \#circle(radius:\ 10pt)\ -\/-\textgreater{}\ }
  will result in a circle in the document (but only when rendered
  through Typst). See also
  \texttt{\ \textless{}!-\/-typst-begin-exclude-\/-\textgreater{}\ } and
  \texttt{\ \textless{}!-\/-typst-end-exclude-\/-\textgreater{}\ } ,
  which is the inverse of this.
\item
  \texttt{\ scope\ } : When \texttt{\ raw-typst\ } is enabled, this is a
  dictionary providing the context in which the evaluated Typst code
  runs. It is useful to pass values in to code inside
  \texttt{\ \textless{}!-\/-raw-typst-\/-\textgreater{}\ } blocks.

  \begin{itemize}
  \tightlist
  \item
    Accepted values: Any dictionary.
  \item
    Default value: \texttt{\ (:)\ } .
  \end{itemize}
\item
  \texttt{\ show-source\ } : A debugging tool. When set to
  \texttt{\ true\ } , the Typst code that would otherwise have been
  displayed will be instead rendered in a code block.

  \begin{itemize}
  \tightlist
  \item
    Accepted values: Booleans.
  \item
    Default value: \texttt{\ false\ } .
  \end{itemize}
\end{itemize}

This function returns the rendered \texttt{\ content\ } .

\subsection{Supported Markdown Syntax}\label{supported-markdown-syntax}

We support CommonMark with a couple extensions.

\begin{itemize}
\item
  Paragraph breaks: Two newlines, i.e. one blank line.
\item
  Hard line breaks (used more in poetry than prose): Put two spaces at
  the end of the line.
\item
  \texttt{\ *emphasis*\ } or \texttt{\ \_emphasis\_\ } : \emph{emphasis}
\item
  \texttt{\ **strong**\ } or \texttt{\ \_\_strong\_\_\ } :
  \textbf{strong}
\item
  \texttt{\ \textasciitilde{}strikethrough\textasciitilde{}\ } :
  \textasciitilde strikethrough\textasciitilde{}
\item
  \texttt{\ {[}links{]}(https://example.org)\ } :
  \href{https://example.org/}{links}
\item
  \texttt{\ \#\#\#\ Headings\ } , where \texttt{\ \#\ } is a top-level
  heading, \texttt{\ \#\#\ } a subheading, \texttt{\ \#\#\#\ } a
  sub-subheading, etc
\item
  \texttt{\ \textasciigrave{}inline\ code\ blocks\textasciigrave{}\ } :
  \texttt{\ inline\ code\ blocks\ }
\item
\begin{verbatim}
```
out of line code blocks
```
\end{verbatim}

  Syntax highlighting can be achieved by specifying a language after the
  opening backticks:

\begin{verbatim}
```rust
let x = 5;
```
\end{verbatim}

  giving:

\begin{Shaded}
\begin{Highlighting}[]
\KeywordTok{let}\NormalTok{ x }\OperatorTok{=} \DecValTok{5}\OperatorTok{;}
\end{Highlighting}
\end{Shaded}
\item
  \texttt{\ -\/-\/-\ } , making a horizontal rule:
\end{itemize}

\begin{center}\rule{0.5\linewidth}{0.5pt}\end{center}

\begin{itemize}
\item
\begin{Shaded}
\begin{Highlighting}[]
\SpecialStringTok{{-} }\NormalTok{Unordered}
\SpecialStringTok{{-} }\NormalTok{lists}
\end{Highlighting}
\end{Shaded}

  \begin{itemize}
  \tightlist
  \item
    Unordered
  \item
    Lists
  \end{itemize}
\item
\begin{Shaded}
\begin{Highlighting}[]
\SpecialStringTok{1. }\NormalTok{Ordered}
\SpecialStringTok{1. }\NormalTok{Lists}
\end{Highlighting}
\end{Shaded}

  \begin{enumerate}
  \tightlist
  \item
    Ordered
  \item
    Lists
  \end{enumerate}
\item
  \texttt{\ \$x\ +\ y\$\ } or \texttt{\ \$\$x\ +\ y\$\$\ } : math
  equations, if the \texttt{\ math\ } parameter is set.
\item
  \texttt{\ \textgreater{}\ blockquotes\ } , if the
  \texttt{\ blockquote\ } parameter is set.
\item
  Images:
  \texttt{\ !{[}Some\ tiled\ hexagons{]}(examples/hexagons.png)\ } ,
  giving
  \pandocbounded{\includegraphics[keepaspectratio]{https://github.com/typst/packages/raw/main/packages/preview/cmarker/0.1.1/examples/hexagons.png}}
\end{itemize}

\subsection{Interleaving Markdown and
Typst}\label{interleaving-markdown-and-typst}

Sometimes, you might want to render a certain section of the document
only when viewed as Markdown, or only when viewed through Typst. To
achieve the former, you can simply wrap the section in
\texttt{\ \textless{}!-\/-typst-begin-exclude-\/-\textgreater{}\ } and
\texttt{\ \textless{}!-\/-typst-end-exclude-\/-\textgreater{}\ } :

\begin{Shaded}
\begin{Highlighting}[]
\NormalTok{Hello from not Typst!}
\end{Highlighting}
\end{Shaded}

Most Markdown parsers support HTML comments, so from their perspective
this is no different to just writing out the Markdown directly; but
\texttt{\ cmarker.typ\ } knows to search for those comments and avoid
rendering the content in between.

Note that when the opening comment is followed by the end of an element,
\texttt{\ cmarker.typ\ } will close the block for you. For example:

\begin{Shaded}
\begin{Highlighting}[]
\AttributeTok{\textgreater{} }
\AttributeTok{\textgreater{} One}

\NormalTok{Two}
\end{Highlighting}
\end{Shaded}

In this code, “Two� will be given no matter where the document is
rendered. This is done to prevent us from generating invalid Typst code.

Conversely, one can put Typst code inside a HTML comment of the form
\texttt{\ \textless{}!-\/-raw-typst\ {[}…{]}-\/-\textgreater{}\ } to
have it evaluated directly as Typst code (but only if the
\texttt{\ raw-typst\ } option to \texttt{\ render\ } is set to
\texttt{\ true\ } , otherwise it will just be seen as a regular comment
and removed):

\begin{Shaded}
\begin{Highlighting}[]

\end{Highlighting}
\end{Shaded}

\subsection{Markdownâ€``Typst
Polyglots}\label{markdownuxe2typst-polyglots}

This project has a manual as a PDF and a README as a Markdown document,
but by the power of this library they are in fact the same thing!
Furthermore, one can read the \texttt{\ README.md\ } in a markdown
viewer and it will display correctly, but one can \emph{also} run
\texttt{\ typst\ compile\ README.md\ } to generate the Typst-typeset
\texttt{\ README.pdf\ } .

How does this work? We just have to be clever about how we write the
README:

\begin{Shaded}
\begin{Highlighting}[]
\NormalTok{(Typst preamble content)}
\NormalTok{\#"}


\NormalTok{Regular Markdown goes here…}

\end{Highlighting}
\end{Shaded}

The same code but syntax-highlighted as Typst code helps to illuminate
it:

\begin{Shaded}
\begin{Highlighting}[]
\NormalTok{\textless{}picture\textgreater{}}
\NormalTok{(Typst preamble content)}
\NormalTok{\#"\textless{}/picture\textgreater{}}
\NormalTok{\textless{}!{-}{-}".slice(0,0)}
\NormalTok{\#import "@preview/cmarker:0.1.1"}
\NormalTok{\#let markdown = read("README.md")}
\NormalTok{\#markdown = markdown.slice(markdown.position("\textless{}/picture\textgreater{}") + "\textless{}/picture\textgreater{}".len())}
\NormalTok{\#cmarker.render(markdown, h1{-}level: 0)}
\NormalTok{/*{-}{-}\textgreater{}}

\NormalTok{Regular Markdown goes here…}

\NormalTok{\textless{}!{-}{-}*///{-}{-}\textgreater{}}
\end{Highlighting}
\end{Shaded}

\subsection{Limitations}\label{limitations}

\begin{itemize}
\tightlist
\item
  We do not currently support HTML tags, and they will be stripped from
  the output; for example, GitHub supports writing
  \texttt{\ \textless{}sub\textgreater{}text\textless{}/sub\textgreater{}\ }
  to get subscript text, but we will render that as simply “text�.
  In future it would be nice to support a subset of HTML tags.
\item
  We do not currently support Markdown tables and footnotes. In future
  it would be good to.
\item
  Although I tried my best to escape everything correctly, I won’t
  provide a hard guarantee that everything is fully sandboxed even if
  you set \texttt{\ raw-typst:\ false\ } . That said, Typst itself is
  well-sandboxed anyway.
\end{itemize}

\subsection{Development}\label{development}

\begin{itemize}
\tightlist
\item
  Build the plugin with \texttt{\ ./build.sh\ } , which produces the
  \texttt{\ plugin.wasm\ } necessary to use this.
\item
  Compile examples with
  \texttt{\ typst\ compile\ examples/\{name\}.typ\ -\/-root\ .\ } .
\item
  Compile this README to PDF with \texttt{\ typst\ compile\ README.md\ }
  .
\end{itemize}

\subsubsection{How to add}\label{how-to-add}

Copy this into your project and use the import as \texttt{\ cmarker\ }

\begin{verbatim}
#import "@preview/cmarker:0.1.1"
\end{verbatim}

\includesvg[width=0.16667in,height=0.16667in]{/assets/icons/16-copy.svg}

Check the docs for
\href{https://typst.app/docs/reference/scripting/\#packages}{more
information on how to import packages} .

\subsubsection{About}\label{about}

\begin{description}
\tightlist
\item[Author :]
Sabrina Jewson
\item[License:]
MIT
\item[Current version:]
0.1.1
\item[Last updated:]
September 11, 2024
\item[First released:]
October 23, 2023
\item[Minimum Typst version:]
0.8.0
\item[Archive size:]
94.8 kB
\href{https://packages.typst.org/preview/cmarker-0.1.1.tar.gz}{\pandocbounded{\includesvg[keepaspectratio]{/assets/icons/16-download.svg}}}
\item[Repository:]
\href{https://github.com/SabrinaJewson/cmarker.typ}{GitHub}
\end{description}

\subsubsection{Where to report issues?}\label{where-to-report-issues}

This package is a project of Sabrina Jewson . Report issues on
\href{https://github.com/SabrinaJewson/cmarker.typ}{their repository} .
You can also try to ask for help with this package on the
\href{https://forum.typst.app}{Forum} .

Please report this package to the Typst team using the
\href{https://typst.app/contact}{contact form} if you believe it is a
safety hazard or infringes upon your rights.

\phantomsection\label{versions}
\subsubsection{Version history}\label{version-history}

\begin{longtable}[]{@{}ll@{}}
\toprule\noalign{}
Version & Release Date \\
\midrule\noalign{}
\endhead
\bottomrule\noalign{}
\endlastfoot
0.1.1 & September 11, 2024 \\
\href{https://typst.app/universe/package/cmarker/0.1.0/}{0.1.0} &
October 23, 2023 \\
\end{longtable}

Typst GmbH did not create this package and cannot guarantee correct
functionality of this package or compatibility with any version of the
Typst compiler or app.
