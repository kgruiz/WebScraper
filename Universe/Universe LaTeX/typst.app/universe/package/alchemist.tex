\title{typst.app/universe/package/alchemist}

\phantomsection\label{banner}
\section{alchemist}\label{alchemist}

{ 0.1.2 }

A package to render skeletal formulas using cetz

{ } Featured Package

\phantomsection\label{readme}
Alchemist is a typst package to draw skeletal formulae. It is based on
the \href{https://ctan.org/pkg/chemfig}{chemfig} package. The main goal
of alchemist is not to reproduce one-to-one chemfig. Instead, it aims to
provide an interface to achieve the same results in Typst.

\begin{Shaded}
\begin{Highlighting}[]
\NormalTok{\#skeletize(\{}
\NormalTok{  molecule(name: "A", "A")}
\NormalTok{  single()}
\NormalTok{  molecule("B")}
\NormalTok{  branch(\{}
\NormalTok{    single(angle: 1)}
\NormalTok{    molecule(}
\NormalTok{      "W",}
\NormalTok{      links: (}
\NormalTok{        "A": double(stroke: red),}
\NormalTok{      ),}
\NormalTok{    )}
\NormalTok{    single()}
\NormalTok{    molecule(name: "X", "X")}
\NormalTok{  \})}
\NormalTok{  branch(\{}
\NormalTok{    single(angle: {-}1)}
\NormalTok{    molecule("Y")}
\NormalTok{    single()}
\NormalTok{    molecule(}
\NormalTok{      name: "Z",}
\NormalTok{      "Z",}
\NormalTok{      links: (}
\NormalTok{        "X": single(stroke: black + 3pt),}
\NormalTok{      ),}
\NormalTok{    )}
\NormalTok{  \})}
\NormalTok{  single()}
\NormalTok{  molecule(}
\NormalTok{    "C",}
\NormalTok{    links: (}
\NormalTok{      "X": cram{-}filled{-}left(fill: blue),}
\NormalTok{      "Z": single(),}
\NormalTok{    ),}
\NormalTok{  )}
\NormalTok{\})}
\end{Highlighting}
\end{Shaded}

\pandocbounded{\includegraphics[keepaspectratio]{https://raw.githubusercontent.com/Robotechnic/alchemist/master/images/links1.png}}

Alchemist uses cetz to draw the molecules. This means that you can draw
cetz shapes in the same canvas as the molecules. Like this:

\begin{Shaded}
\begin{Highlighting}[]
\NormalTok{\#skeletize(\{}
\NormalTok{  import cetz.draw: *}
\NormalTok{  double(absolute: 30deg, name: "l1")}
\NormalTok{  single(absolute: {-}30deg, name: "l2")}
\NormalTok{  molecule("X", name: "X")}
\NormalTok{  hobby(}
\NormalTok{    "l1.50\%",}
\NormalTok{    ("l1.start", 0.5, 90deg, "l1.end"),}
\NormalTok{    "l1.start",}
\NormalTok{    stroke: (paint: red, dash: "dashed"),}
\NormalTok{    mark: (end: "\textgreater{}"),}
\NormalTok{  )}
\NormalTok{  hobby(}
\NormalTok{    (to: "X.north", rel: (0, 1pt)),}
\NormalTok{    ("l2.end", 0.4, {-}90deg, "l2.start"),}
\NormalTok{    "l2.50\%",}
\NormalTok{    mark: (end: "\textgreater{}"),}
\NormalTok{  )}
\NormalTok{\})}
\end{Highlighting}
\end{Shaded}

\pandocbounded{\includegraphics[keepaspectratio]{https://raw.githubusercontent.com/Robotechnic/alchemist/master/images/cetz1.png}}

\subsection{Usage}\label{usage}

To start using alchemist, just use the following code:

\begin{Shaded}
\begin{Highlighting}[]
\NormalTok{\#import "@preview/alchemist:0.1.2": *}

\NormalTok{\#skeletize(\{}
\NormalTok{  // Your molecule here}
\NormalTok{\})}
\end{Highlighting}
\end{Shaded}

For more information, check the
\href{https://raw.githubusercontent.com/Robotechnic/alchemist/master/doc/manual.pdf}{manual}
.

\subsection{Changelog}\label{changelog}

\subsubsection{0.1.2}\label{section}

\begin{itemize}
\tightlist
\item
  Added default values for link style properties.
\item
  Updated \texttt{\ cetz\ } to version 0.3.1.
\item
  Added a \texttt{\ tip-lenght\ } argument to dashed cram links.
\end{itemize}

\subsubsection{0.1.1}\label{section-1}

\begin{itemize}
\tightlist
\item
  Exposed the \texttt{\ draw-skeleton\ } function. This allows to draw
  in a cetz canvas directly.
\item
  Fixed multiples bugs that causes overdraws of links.
\end{itemize}

\subsubsection{0.1.0}\label{section-2}

\begin{itemize}
\tightlist
\item
  Initial release
\end{itemize}

\subsubsection{How to add}\label{how-to-add}

Copy this into your project and use the import as \texttt{\ alchemist\ }

\begin{verbatim}
#import "@preview/alchemist:0.1.2"
\end{verbatim}

\includesvg[width=0.16667in,height=0.16667in]{/assets/icons/16-copy.svg}

Check the docs for
\href{https://typst.app/docs/reference/scripting/\#packages}{more
information on how to import packages} .

\subsubsection{About}\label{about}

\begin{description}
\tightlist
\item[Author :]
\href{https://github.com/Robotechnic}{Robotechnic}
\item[License:]
MIT
\item[Current version:]
0.1.2
\item[Last updated:]
November 13, 2024
\item[First released:]
August 14, 2024
\item[Minimum Typst version:]
0.11.0
\item[Archive size:]
11.5 kB
\href{https://packages.typst.org/preview/alchemist-0.1.2.tar.gz}{\pandocbounded{\includesvg[keepaspectratio]{/assets/icons/16-download.svg}}}
\item[Repository:]
\href{https://github.com/Robotechnic/alchemist}{GitHub}
\item[Discipline s :]
\begin{itemize}
\tightlist
\item[]
\item
  \href{https://typst.app/universe/search/?discipline=chemistry}{Chemistry}
\item
  \href{https://typst.app/universe/search/?discipline=biology}{Biology}
\end{itemize}
\item[Categor y :]
\begin{itemize}
\tightlist
\item[]
\item
  \pandocbounded{\includesvg[keepaspectratio]{/assets/icons/16-chart.svg}}
  \href{https://typst.app/universe/search/?category=visualization}{Visualization}
\end{itemize}
\end{description}

\subsubsection{Where to report issues?}\label{where-to-report-issues}

This package is a project of Robotechnic . Report issues on
\href{https://github.com/Robotechnic/alchemist}{their repository} . You
can also try to ask for help with this package on the
\href{https://forum.typst.app}{Forum} .

Please report this package to the Typst team using the
\href{https://typst.app/contact}{contact form} if you believe it is a
safety hazard or infringes upon your rights.

\phantomsection\label{versions}
\subsubsection{Version history}\label{version-history}

\begin{longtable}[]{@{}ll@{}}
\toprule\noalign{}
Version & Release Date \\
\midrule\noalign{}
\endhead
\bottomrule\noalign{}
\endlastfoot
0.1.2 & November 13, 2024 \\
\href{https://typst.app/universe/package/alchemist/0.1.1/}{0.1.1} &
August 19, 2024 \\
\href{https://typst.app/universe/package/alchemist/0.1.0/}{0.1.0} &
August 14, 2024 \\
\end{longtable}

Typst GmbH did not create this package and cannot guarantee correct
functionality of this package or compatibility with any version of the
Typst compiler or app.
