\title{typst.app/universe/package/appreciated-letter}

\phantomsection\label{banner}
\phantomsection\label{template-thumbnail}
\pandocbounded{\includegraphics[keepaspectratio]{https://packages.typst.org/preview/thumbnails/appreciated-letter-0.1.0-small.webp}}

\section{appreciated-letter}\label{appreciated-letter}

{ 0.1.0 }

Correspond with business associates and your friends via mail

\href{/app?template=appreciated-letter&version=0.1.0}{Create project in
app}

\phantomsection\label{readme}
A basic letter with sender and recipient address. The letter is ready
for a DIN DL windowed envelope.

\subsection{Usage}\label{usage}

You can use this template in the Typst web app by clicking “Start from
template� on the dashboard and searching for
\texttt{\ appreciated-letter\ } .

Alternatively, you can use the CLI to kick this project off using the
command

\begin{verbatim}
typst init @preview/appreciated-letter
\end{verbatim}

Typst will create a new directory with all the files needed to get you
started.

\subsection{Configuration}\label{configuration}

This template exports the \texttt{\ letter\ } function with the
following named arguments:

\begin{itemize}
\tightlist
\item
  \texttt{\ sender\ } : The letter’s sender as content. This is
  displayed at the top of the page.
\item
  \texttt{\ recipient\ } : The address of the letter’s recipient as
  content. This is displayed near the top of the page.
\item
  \texttt{\ date\ } : The date, and possibly place, the letter was
  written at as content. Flushed to the right after the address.
\item
  \texttt{\ subject\ } : The subject line for the letter as content.
\item
  \texttt{\ name\ } : The name the letter closes with as content.
\end{itemize}

The function also accepts a single, positional argument for the body of
the letter.

The template will initialize your package with a sample call to the
\texttt{\ letter\ } function in a show rule. If you, however, want to
change an existing project to use this template, you can add a show rule
like this at the top of your file:

\begin{Shaded}
\begin{Highlighting}[]
\NormalTok{\#import "@preview/appreciated{-}letter:0.1.0": letter}

\NormalTok{\#show: letter.with(}
\NormalTok{  sender: [}
\NormalTok{    Jane Smith, Universal Exports, 1 Heavy Plaza, Morristown, NJ 07964}
\NormalTok{  ],}
\NormalTok{  recipient: [}
\NormalTok{    Mr. John Doe \textbackslash{}}
\NormalTok{    Acme Corp. \textbackslash{}}
\NormalTok{    123 Glennwood Ave \textbackslash{}}
\NormalTok{    Quarto Creek, VA 22438}
\NormalTok{  ],}
\NormalTok{  date: [Morristown, June 9th, 2023],}
\NormalTok{  subject: [Revision of our Producrement Contract],}
\NormalTok{  name: [Jane Smith \textbackslash{} Regional Director],}
\NormalTok{)}

\NormalTok{Dear Joe,}

\NormalTok{\#lorem(99)}

\NormalTok{Best,}
\end{Highlighting}
\end{Shaded}

\href{/app?template=appreciated-letter&version=0.1.0}{Create project in
app}

\subsubsection{How to use}\label{how-to-use}

Click the button above to create a new project using this template in
the Typst app.

You can also use the Typst CLI to start a new project on your computer
using this command:

\begin{verbatim}
typst init @preview/appreciated-letter:0.1.0
\end{verbatim}

\includesvg[width=0.16667in,height=0.16667in]{/assets/icons/16-copy.svg}

\subsubsection{About}\label{about}

\begin{description}
\tightlist
\item[Author :]
\href{https://typst.app}{Typst GmbH}
\item[License:]
MIT-0
\item[Current version:]
0.1.0
\item[Last updated:]
March 6, 2024
\item[First released:]
March 6, 2024
\item[Minimum Typst version:]
0.6.0
\item[Archive size:]
2.33 kB
\href{https://packages.typst.org/preview/appreciated-letter-0.1.0.tar.gz}{\pandocbounded{\includesvg[keepaspectratio]{/assets/icons/16-download.svg}}}
\item[Repository:]
\href{https://github.com/typst/templates}{GitHub}
\item[Categor y :]
\begin{itemize}
\tightlist
\item[]
\item
  \pandocbounded{\includesvg[keepaspectratio]{/assets/icons/16-envelope.svg}}
  \href{https://typst.app/universe/search/?category=office}{Office}
\end{itemize}
\end{description}

\subsubsection{Where to report issues?}\label{where-to-report-issues}

This template is a project of Typst GmbH . Report issues on
\href{https://github.com/typst/templates}{their repository} . You can
also try to ask for help with this template on the
\href{https://forum.typst.app}{Forum} .

\phantomsection\label{versions}
\subsubsection{Version history}\label{version-history}

\begin{longtable}[]{@{}ll@{}}
\toprule\noalign{}
Version & Release Date \\
\midrule\noalign{}
\endhead
\bottomrule\noalign{}
\endlastfoot
0.1.0 & March 6, 2024 \\
\end{longtable}
