\title{typst.app/docs/tutorial/advanced-styling}

\begin{itemize}
\tightlist
\item
  \href{/docs}{\includesvg[width=0.16667in,height=0.16667in]{/assets/icons/16-docs-dark.svg}}
\item
  \includesvg[width=0.16667in,height=0.16667in]{/assets/icons/16-arrow-right.svg}
\item
  \href{/docs/tutorial/}{Tutorial}
\item
  \includesvg[width=0.16667in,height=0.16667in]{/assets/icons/16-arrow-right.svg}
\item
  \href{/docs/tutorial/advanced-styling/}{Advanced Styling}
\end{itemize}

\section{Advanced Styling}\label{advanced-styling}

In the previous two chapters of this tutorial, you have learned how to
write a document in Typst and how to change its formatting. The report
you wrote throughout the last two chapters got a straight A and your
supervisor wants to base a conference paper on it! The report will of
course have to comply with the conference\textquotesingle s style guide.
Let\textquotesingle s see how we can achieve that.

Before we start, let\textquotesingle s create a team, invite your
supervisor and add them to the team. You can do this by going back to
the app dashboard with the back icon in the top left corner of the
editor. Then, choose the plus icon in the left toolbar and create a
team. Finally, click on the new team and go to its settings by clicking
\textquotesingle manage team\textquotesingle{} next to the team name.
Now you can invite your supervisor by email.

\pandocbounded{\includegraphics[keepaspectratio]{/assets/docs/3-advanced-team-settings.png}}

Next, move your project into the team: Open it, going to its settings by
choosing the gear icon in the left toolbar and selecting your new team
from the owners dropdown. Don\textquotesingle t forget to save your
changes!

Now, your supervisor can also edit the project and you can both see the
changes in real time. You can join our
\href{https://discord.gg/2uDybryKPe}{Discord server} to find other users
and try teams with them!

\subsection{The conference guidelines}\label{guidelines}

The layout guidelines are available on the conference website.
Let\textquotesingle s take a look at them:

\begin{itemize}
\tightlist
\item
  The font should be an 11pt serif font
\item
  The title should be in 17pt and bold
\item
  The paper contains a single-column abstract and two-column main text
\item
  The abstract should be centered
\item
  The main text should be justified
\item
  First level section headings should be 13pt, centered, and rendered in
  small capitals
\item
  Second level headings are run-ins, italicized and have the same size
  as the body text
\item
  Finally, the pages should be US letter sized, numbered in the center
  of the footer and the top right corner of each page should contain the
  title of the paper
\end{itemize}

We already know how to do many of these things, but for some of them,
we\textquotesingle ll need to learn some new tricks.

\subsection{Writing the right set rules}\label{set-rules}

Let\textquotesingle s start by writing some set rules for the document.

\begin{verbatim}
#set page(
  paper: "us-letter",
  header: align(right)[
    A fluid dynamic model for
    glacier flow
  ],
  numbering: "1",
)
#set par(justify: true)
#set text(
  font: "Libertinus Serif",
  size: 11pt,
)

#lorem(600)
\end{verbatim}

\includegraphics[width=12.75in,height=\textheight,keepaspectratio]{/assets/docs/p6Vtj1ockTIzscSwa5_kewAAAAAAAAAA.png}

You are already familiar with most of what is going on here. We set the
text size to \texttt{\ }{\texttt{\ 11pt\ }}\texttt{\ } and the font to
Libertinus Serif. We also enable paragraph justification and set the
page size to US letter.

The \texttt{\ header\ } argument is new: With it, we can provide content
to fill the top margin of every page. In the header, we specify our
paper\textquotesingle s title as requested by the conference style
guide. We use the \texttt{\ align\ } function to align the text to the
right.

Last but not least is the \texttt{\ numbering\ } argument. Here, we can
provide a \href{/docs/reference/model/numbering/}{numbering pattern}
that defines how to number the pages. By setting into to
\texttt{\ }{\texttt{\ "1"\ }}\texttt{\ } , Typst only displays the bare
page number. Setting it to \texttt{\ }{\texttt{\ "(1/1)"\ }}\texttt{\ }
would have displayed the current page and total number of pages
surrounded by parentheses. And we could even have provided a completely
custom function here to format things to our liking.

\subsection{Creating a title and abstract}\label{title-and-abstract}

Now, let\textquotesingle s add a title and an abstract.
We\textquotesingle ll start with the title. We center align it and
increase its font weight by enclosing it in
\texttt{\ }{\texttt{\ *stars*\ }}\texttt{\ } .

\begin{verbatim}
#align(center, text(17pt)[
  *A fluid dynamic model
  for glacier flow*
])
\end{verbatim}

\includegraphics[width=6.25in,height=\textheight,keepaspectratio]{/assets/docs/EYkMw9AAwHWkDqrGblkKBgAAAAAAAAAA.png}

This looks right. We used the \texttt{\ text\ } function to override the
previous text set rule locally, increasing the size to 17pt for the
function\textquotesingle s argument. Let\textquotesingle s also add the
author list: Since we are writing this paper together with our
supervisor, we\textquotesingle ll add our own and their name.

\begin{verbatim}
#grid(
  columns: (1fr, 1fr),
  align(center)[
    Therese Tungsten \
    Artos Institute \
    #link("mailto:tung@artos.edu")
  ],
  align(center)[
    Dr. John Doe \
    Artos Institute \
    #link("mailto:doe@artos.edu")
  ]
)
\end{verbatim}

\includegraphics[width=6.25in,height=\textheight,keepaspectratio]{/assets/docs/Iwl_3LT7ijX6dcpL71YOWAAAAAAAAAAA.png}

The two author blocks are laid out next to each other. We use the
\href{/docs/reference/layout/grid/}{\texttt{\ grid\ }} function to
create this layout. With a grid, we can control exactly how large each
column is and which content goes into which cell. The
\texttt{\ columns\ } argument takes an array of
\href{/docs/reference/layout/relative/}{relative lengths} or
\href{/docs/reference/layout/fraction/}{fractions} . In this case, we
passed it two equal fractional sizes, telling it to split the available
space into two equal columns. We then passed two content arguments to
the grid function. The first with our own details, and the second with
our supervisors\textquotesingle. We again use the \texttt{\ align\ }
function to center the content within the column. The grid takes an
arbitrary number of content arguments specifying the cells. Rows are
added automatically, but they can also be manually sized with the
\texttt{\ rows\ } argument.

Now, let\textquotesingle s add the abstract. Remember that the
conference wants the abstract to be set ragged and centered.

\begin{verbatim}
...

#align(center)[
  #set par(justify: false)
  *Abstract* \
  #lorem(80)
]
\end{verbatim}

\includegraphics[width=12.75in,height=\textheight,keepaspectratio]{/assets/docs/4IdrVTeq86rbgvB-RNog6gAAAAAAAAAA.png}

Well done! One notable thing is that we used a set rule within the
content argument of \texttt{\ align\ } to turn off justification for the
abstract. This does not affect the remainder of the document even though
it was specified after the first set rule because content blocks
\emph{scope} styling. Anything set within a content block will only
affect the content within that block.

Another tweak could be to save the paper title in a variable, so that we
do not have to type it twice, for header and title. We can do that with
the \texttt{\ }{\texttt{\ let\ }}\texttt{\ } keyword:

\begin{verbatim}
#let title = [
  A fluid dynamic model
  for glacier flow
]

...

#set page(
  header: align(
    right + horizon,
    title
  ),
  ...
)

#align(center, text(17pt)[
  *#title*
])

...
\end{verbatim}

\includegraphics[width=12.75in,height=\textheight,keepaspectratio]{/assets/docs/ZKsZ2Eei-RUgPTLeJrMEYgAAAAAAAAAA.png}

After we bound the content to the \texttt{\ title\ } variable, we can
use it in functions and also within markup (prefixed by \texttt{\ \#\ }
, like functions). This way, if we decide on another title, we can
easily change it in one place.

\subsection{Adding columns and headings}\label{columns-and-headings}

The paper above unfortunately looks like a wall of lead. To fix that,
let\textquotesingle s add some headings and switch our paper to a
two-column layout. Fortunately, that\textquotesingle s easy to do: We
just need to amend our \texttt{\ page\ } set rule with the
\texttt{\ columns\ } argument.

By adding \texttt{\ columns:\ }{\texttt{\ 2\ }}\texttt{\ } to the
argument list, we have wrapped the whole document in two columns.
However, that would also affect the title and authors overview. To keep
them spanning the whole page, we can wrap them in a function call to
\href{/docs/reference/layout/place/}{\texttt{\ place\ }} . Place expects
an alignment and the content it should place as positional arguments.
Using the named \texttt{\ scope\ } argument, we can decide if the items
should be placed relative to the current column or its parent (the
page). There is one more thing to configure: If no other arguments are
provided, \texttt{\ place\ } takes its content out of the flow of the
document and positions it over the other content without affecting the
layout of other content in its container:

\begin{verbatim}
#place(
  top + center,
  rect(fill: black),
)
#lorem(30)
\end{verbatim}

\includegraphics[width=5in,height=\textheight,keepaspectratio]{/assets/docs/30s7cU9X36lW286rJXE3RwAAAAAAAAAA.png}

If we hadn\textquotesingle t used \texttt{\ place\ } here, the square
would be in its own line, but here it overlaps the few lines of text
following it. Likewise, that text acts like as if there was no square.
To change this behavior, we can pass the argument
\texttt{\ float:\ }{\texttt{\ true\ }}\texttt{\ } to ensure that the
space taken up by the placed item at the top or bottom of the page is
not occupied by any other content.

\begin{verbatim}
#set page(
  paper: "us-letter",
  header: align(
    right + horizon,
    title
  ),
  numbering: "1",
  columns: 2,
)

#place(
  top + center,
  float: true,
  scope: "parent",
  clearance: 2em,
)[
  ...

  #par(justify: false)[
    *Abstract* \
    #lorem(80)
  ]
]

= Introduction
#lorem(300)

= Related Work
#lorem(200)
\end{verbatim}

\includegraphics[width=12.75in,height=\textheight,keepaspectratio]{/assets/docs/dAJVP8paZmMvnK23cMA_0AAAAAAAAAAA.png}

In this example, we also used the \texttt{\ clearance\ } argument of the
\texttt{\ place\ } function to provide the space between it and the body
instead of using the \href{/docs/reference/layout/v/}{\texttt{\ v\ }}
function. We can also remove the explicit
\texttt{\ }{\texttt{\ align\ }}\texttt{\ }{\texttt{\ (\ }}\texttt{\ center\ }{\texttt{\ ,\ }}\texttt{\ }{\texttt{\ ..\ }}\texttt{\ }{\texttt{\ )\ }}\texttt{\ }
calls around the various parts since they inherit the center alignment
from the placement.

Now there is only one thing left to do: Style our headings. We need to
make them centered and use small capitals. Because the
\texttt{\ heading\ } function does not offer a way to set any of that,
we need to write our own heading show rule.

\begin{verbatim}
#show heading: it => [
  #set align(center)
  #set text(13pt, weight: "regular")
  #block(smallcaps(it.body))
]

...
\end{verbatim}

\includegraphics[width=5.52083in,height=\textheight,keepaspectratio]{/assets/docs/ZJxJWdUySZNKlj1_Vn1NWgAAAAAAAAAA.png}

This looks great! We used a show rule that applies to all headings. We
give it a function that gets passed the heading as a parameter. That
parameter can be used as content but it also has some fields like
\texttt{\ title\ } , \texttt{\ numbers\ } , and \texttt{\ level\ } from
which we can compose a custom look. Here, we are center-aligning,
setting the font weight to
\texttt{\ }{\texttt{\ "regular"\ }}\texttt{\ } because headings are bold
by default, and use the
\href{/docs/reference/text/smallcaps/}{\texttt{\ smallcaps\ }} function
to render the heading\textquotesingle s title in small capitals.

The only remaining problem is that all headings look the same now. The
"Motivation" and "Problem Statement" subsections ought to be italic run
in headers, but right now, they look indistinguishable from the section
headings. We can fix that by using a \texttt{\ where\ } selector on our
set rule: This is a \href{/docs/reference/scripting/\#methods}{method}
we can call on headings (and other elements) that allows us to filter
them by their level. We can use it to differentiate between section and
subsection headings:

\begin{verbatim}
#show heading.where(
  level: 1
): it => block(width: 100%)[
  #set align(center)
  #set text(13pt, weight: "regular")
  #smallcaps(it.body)
]

#show heading.where(
  level: 2
): it => text(
  size: 11pt,
  weight: "regular",
  style: "italic",
  it.body + [.],
)
\end{verbatim}

\includegraphics[width=5.52083in,height=\textheight,keepaspectratio]{/assets/docs/eBNymJDskGFkYVAXkF9cuAAAAAAAAAAA.png}

This looks great! We wrote two show rules that each selectively apply to
the first and second level headings. We used a \texttt{\ where\ }
selector to filter the headings by their level. We then rendered the
subsection headings as run-ins. We also automatically add a period to
the end of the subsection headings.

Let\textquotesingle s review the conference\textquotesingle s style
guide:

\begin{itemize}
\tightlist
\item
  The font should be an 11pt serif font âœ``
\item
  The title should be in 17pt and bold âœ``
\item
  The paper contains a single-column abstract and two-column main text
  âœ``
\item
  The abstract should be centered âœ``
\item
  The main text should be justified âœ``
\item
  First level section headings should be centered, rendered in small
  caps and in 13pt âœ``
\item
  Second level headings are run-ins, italicized and have the same size
  as the body text âœ``
\item
  Finally, the pages should be US letter sized, numbered in the center
  and the top right corner of each page should contain the title of the
  paper âœ``
\end{itemize}

We are now in compliance with all of these styles and can submit the
paper to the conference! The finished paper looks like this:

\pandocbounded{\includegraphics[keepaspectratio]{/assets/docs/3-advanced-paper.png}}

\subsection{Review}\label{review}

You have now learned how to create headers and footers, how to use
functions and scopes to locally override styles, how to create more
complex layouts with the
\href{/docs/reference/layout/grid/}{\texttt{\ grid\ }} function and how
to write show rules for individual functions, and the whole document.
You also learned how to use the
\href{/docs/reference/styling/\#show-rules}{\texttt{\ where\ } selector}
to filter the headings by their level.

The paper was a great success! You\textquotesingle ve met a lot of
like-minded researchers at the conference and are planning a project
which you hope to publish at the same venue next year.
You\textquotesingle ll need to write a new paper using the same style
guide though, so maybe now you want to create a time-saving template for
you and your team?

In the next section, we will learn how to create templates that can be
reused in multiple documents. This is a more advanced topic, so feel
free to come back to it later if you don\textquotesingle t feel up to it
right now.

\href{/docs/tutorial/formatting/}{\pandocbounded{\includesvg[keepaspectratio]{/assets/icons/16-arrow-right.svg}}}

{ Formatting } { Previous page }

\href{/docs/tutorial/making-a-template/}{\pandocbounded{\includesvg[keepaspectratio]{/assets/icons/16-arrow-right.svg}}}

{ Making a Template } { Next page }

\textbf{On this page}

\begin{itemize}
\tightlist
\item
  \hyperref[guidelines]{Guidelines}
\item
  \hyperref[set-rules]{Set Rules}
\item
  \hyperref[title-and-abstract]{Title And Abstract}
\item
  \hyperref[columns-and-headings]{Columns And Headings}
\item
  \hyperref[review]{Review}
\end{itemize}

\begin{itemize}
\tightlist
\item
  \href{/}{Home}
\item
  \href{/pricing/}{Pricing}
\item
  \href{/docs/}{Documentation}
\item
  \href{/universe/}{Universe}
\item
  \href{/about/}{About Us}
\item
  \href{/contact/}{Contact Us}
\item
  \href{/privacy/}{Privacy}
\item
  \href{https://typst.app/terms}{Terms and Conditions}
\item
  \href{/legal/}{Legal (Impressum)}
\end{itemize}

\begin{itemize}
\tightlist
\item
  \href{https://forum.typst.app}{Forum}
\item
  \href{/tools/}{Tools}
\item
  \href{/blog/}{Blog}
\item
  \href{https://github.com/typst/}{GitHub}
\item
  \href{https://discord.gg/2uDybryKPe}{Discord}
\item
  \href{https://mastodon.social/@typst}{Mastodon}
\item
  \href{https://bsky.app/profile/typst.app}{Bluesky}
\item
  \href{https://www.linkedin.com/company/typst/}{LinkedIn}
\item
  \href{https://instagram.com/typstapp/}{Instagram}
\end{itemize}

Made in Berlin
