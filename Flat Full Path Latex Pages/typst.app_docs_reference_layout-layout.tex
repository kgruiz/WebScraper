\title{typst.app/docs/reference/layout/layout}

\begin{itemize}
\tightlist
\item
  \href{/docs}{\includesvg[width=0.16667in,height=0.16667in]{/assets/icons/16-docs-dark.svg}}
\item
  \includesvg[width=0.16667in,height=0.16667in]{/assets/icons/16-arrow-right.svg}
\item
  \href{/docs/reference/}{Reference}
\item
  \includesvg[width=0.16667in,height=0.16667in]{/assets/icons/16-arrow-right.svg}
\item
  \href{/docs/reference/layout/}{Layout}
\item
  \includesvg[width=0.16667in,height=0.16667in]{/assets/icons/16-arrow-right.svg}
\item
  \href{/docs/reference/layout/layout/}{Layout}
\end{itemize}

\section{\texorpdfstring{\texttt{\ layout\ }}{ layout }}\label{summary}

Provides access to the current outer container\textquotesingle s (or
page\textquotesingle s, if none) dimensions (width and height).

Accepts a function that receives a single parameter, which is a
dictionary with keys \texttt{\ width\ } and \texttt{\ height\ } , both
of type \href{/docs/reference/layout/length/}{\texttt{\ length\ }} . The
function is provided \href{/docs/reference/context/}{context} , meaning
you don\textquotesingle t need to use it in combination with the
\texttt{\ context\ } keyword. This is why
\href{/docs/reference/layout/measure/}{\texttt{\ measure\ }} can be
called in the example below.

\begin{verbatim}
#let text = lorem(30)
#layout(size => [
  #let (height,) = measure(
    block(width: size.width, text),
  )
  This text is #height high with
  the current page width: \
  #text
])
\end{verbatim}

\includegraphics[width=5in,height=\textheight,keepaspectratio]{/assets/docs/SI9ZxtAftdvELQJYlwu_CgAAAAAAAAAA.png}

Note that the \texttt{\ layout\ } function forces its contents into a
\href{/docs/reference/layout/block/}{block} -level container, so
placement relative to the page or pagebreaks are not possible within it.

If the \texttt{\ layout\ } call is placed inside a box with a width of
\texttt{\ }{\texttt{\ 800pt\ }}\texttt{\ } and a height of
\texttt{\ }{\texttt{\ 400pt\ }}\texttt{\ } , then the specified function
will be given the argument
\texttt{\ }{\texttt{\ (\ }}\texttt{\ width\ }{\texttt{\ :\ }}\texttt{\ }{\texttt{\ 800pt\ }}\texttt{\ }{\texttt{\ ,\ }}\texttt{\ height\ }{\texttt{\ :\ }}\texttt{\ }{\texttt{\ 400pt\ }}\texttt{\ }{\texttt{\ )\ }}\texttt{\ }
. If it is placed directly into the page, it receives the
page\textquotesingle s dimensions minus its margins. This is mostly
useful in combination with
\href{/docs/reference/layout/measure/}{measurement} .

You can also use this function to resolve
\href{/docs/reference/layout/ratio/}{\texttt{\ ratio\ }} to fixed
lengths. This might come in handy if you\textquotesingle re building
your own layout abstractions.

\begin{verbatim}
#layout(size => {
  let half = 50% * size.width
  [Half a page is #half wide.]
})
\end{verbatim}

\includegraphics[width=5in,height=\textheight,keepaspectratio]{/assets/docs/1AoOPrEARH2i9ZcdcamicAAAAAAAAAAA.png}

Note that the width or height provided by \texttt{\ layout\ } will be
infinite if the corresponding page dimension is set to
\texttt{\ }{\texttt{\ auto\ }}\texttt{\ } .

\subsection{\texorpdfstring{{ Parameters
}}{ Parameters }}\label{parameters}

\phantomsection\label{parameters-tooltip}
Parameters are the inputs to a function. They are specified in
parentheses after the function name.

{ layout } (

{ \href{/docs/reference/foundations/function/}{function} }

) -\textgreater{} \href{/docs/reference/foundations/content/}{content}

\subsubsection{\texorpdfstring{\texttt{\ func\ }}{ func }}\label{parameters-func}

\href{/docs/reference/foundations/function/}{function}

{Required} {{ Positional }}

\phantomsection\label{parameters-func-positional-tooltip}
Positional parameters are specified in order, without names.

A function to call with the outer container\textquotesingle s size. Its
return value is displayed in the document.

The container\textquotesingle s size is given as a
\href{/docs/reference/foundations/dictionary/}{dictionary} with the keys
\texttt{\ width\ } and \texttt{\ height\ } .

This function is called once for each time the content returned by
\texttt{\ layout\ } appears in the document. This makes it possible to
generate content that depends on the dimensions of its container.

\href{/docs/reference/layout/hide/}{\pandocbounded{\includesvg[keepaspectratio]{/assets/icons/16-arrow-right.svg}}}

{ Hide } { Previous page }

\href{/docs/reference/layout/length/}{\pandocbounded{\includesvg[keepaspectratio]{/assets/icons/16-arrow-right.svg}}}

{ Length } { Next page }

\textbf{On this page}

\begin{itemize}
\tightlist
\item
  \hyperref[summary]{Summary}
\item
  \hyperref[parameters]{Parameters}

  \begin{itemize}
  \tightlist
  \item
    \hyperref[parameters-func]{func}
  \end{itemize}
\end{itemize}

\begin{itemize}
\tightlist
\item
  \href{/}{Home}
\item
  \href{/pricing/}{Pricing}
\item
  \href{/docs/}{Documentation}
\item
  \href{/universe/}{Universe}
\item
  \href{/about/}{About Us}
\item
  \href{/contact/}{Contact Us}
\item
  \href{/privacy/}{Privacy}
\item
  \href{https://typst.app/terms}{Terms and Conditions}
\item
  \href{/legal/}{Legal (Impressum)}
\end{itemize}

\begin{itemize}
\tightlist
\item
  \href{https://forum.typst.app}{Forum}
\item
  \href{/tools/}{Tools}
\item
  \href{/blog/}{Blog}
\item
  \href{https://github.com/typst/}{GitHub}
\item
  \href{https://discord.gg/2uDybryKPe}{Discord}
\item
  \href{https://mastodon.social/@typst}{Mastodon}
\item
  \href{https://bsky.app/profile/typst.app}{Bluesky}
\item
  \href{https://www.linkedin.com/company/typst/}{LinkedIn}
\item
  \href{https://instagram.com/typstapp/}{Instagram}
\end{itemize}

Made in Berlin
