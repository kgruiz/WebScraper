\title{typst.app/docs/reference/model/enum}

\begin{itemize}
\tightlist
\item
  \href{/docs}{\includesvg[width=0.16667in,height=0.16667in]{/assets/icons/16-docs-dark.svg}}
\item
  \includesvg[width=0.16667in,height=0.16667in]{/assets/icons/16-arrow-right.svg}
\item
  \href{/docs/reference/}{Reference}
\item
  \includesvg[width=0.16667in,height=0.16667in]{/assets/icons/16-arrow-right.svg}
\item
  \href{/docs/reference/model/}{Model}
\item
  \includesvg[width=0.16667in,height=0.16667in]{/assets/icons/16-arrow-right.svg}
\item
  \href{/docs/reference/model/enum/}{Numbered List}
\end{itemize}

\section{\texorpdfstring{\texttt{\ enum\ } {{ Element
}}}{ enum   Element }}\label{summary}

\phantomsection\label{element-tooltip}
Element functions can be customized with \texttt{\ set\ } and
\texttt{\ show\ } rules.

A numbered list.

Displays a sequence of items vertically and numbers them consecutively.

\subsection{Example}\label{example}

\begin{verbatim}
Automatically numbered:
+ Preparations
+ Analysis
+ Conclusions

Manually numbered:
2. What is the first step?
5. I am confused.
+  Moving on ...

Multiple lines:
+ This enum item has multiple
  lines because the next line
  is indented.

Function call.
#enum[First][Second]
\end{verbatim}

\includegraphics[width=5in,height=\textheight,keepaspectratio]{/assets/docs/HrnJ1mRKvbXNf6U4DmZCaAAAAAAAAAAA.png}

You can easily switch all your enumerations to a different numbering
style with a set rule.

\begin{verbatim}
#set enum(numbering: "a)")

+ Starting off ...
+ Don't forget step two
\end{verbatim}

\includegraphics[width=5in,height=\textheight,keepaspectratio]{/assets/docs/hFb68a8DC-Rvf_eMOYtVMwAAAAAAAAAA.png}

You can also use
\href{/docs/reference/model/enum/\#definitions-item}{\texttt{\ enum.item\ }}
to programmatically customize the number of each item in the
enumeration:

\begin{verbatim}
#enum(
  enum.item(1)[First step],
  enum.item(5)[Fifth step],
  enum.item(10)[Tenth step]
)
\end{verbatim}

\includegraphics[width=5in,height=\textheight,keepaspectratio]{/assets/docs/uRQbjXrkv7FwltBxluVMMAAAAAAAAAAA.png}

\subsection{Syntax}\label{syntax}

This functions also has dedicated syntax:

\begin{itemize}
\tightlist
\item
  Starting a line with a plus sign creates an automatically numbered
  enumeration item.
\item
  Starting a line with a number followed by a dot creates an explicitly
  numbered enumeration item.
\end{itemize}

Enumeration items can contain multiple paragraphs and other block-level
content. All content that is indented more than an
item\textquotesingle s marker becomes part of that item.

\subsection{\texorpdfstring{{ Parameters
}}{ Parameters }}\label{parameters}

\phantomsection\label{parameters-tooltip}
Parameters are the inputs to a function. They are specified in
parentheses after the function name.

{ enum } (

{ \hyperref[parameters-tight]{tight :}
\href{/docs/reference/foundations/bool/}{bool} , } {
\hyperref[parameters-numbering]{numbering :}
\href{/docs/reference/foundations/str/}{str}
\href{/docs/reference/foundations/function/}{function} , } {
\hyperref[parameters-start]{start :}
\href{/docs/reference/foundations/int/}{int} , } {
\hyperref[parameters-full]{full :}
\href{/docs/reference/foundations/bool/}{bool} , } {
\hyperref[parameters-indent]{indent :}
\href{/docs/reference/layout/length/}{length} , } {
\hyperref[parameters-body-indent]{body-indent :}
\href{/docs/reference/layout/length/}{length} , } {
\hyperref[parameters-spacing]{spacing :}
\href{/docs/reference/foundations/auto/}{auto}
\href{/docs/reference/layout/length/}{length} , } {
\hyperref[parameters-number-align]{number-align :}
\href{/docs/reference/layout/alignment/}{alignment} , } {
\hyperref[parameters-children]{..}
\href{/docs/reference/foundations/content/}{content}
\href{/docs/reference/foundations/array/}{array} , }

) -\textgreater{} \href{/docs/reference/foundations/content/}{content}

\subsubsection{\texorpdfstring{\texttt{\ tight\ }}{ tight }}\label{parameters-tight}

\href{/docs/reference/foundations/bool/}{bool}

{{ Settable }}

\phantomsection\label{parameters-tight-settable-tooltip}
Settable parameters can be customized for all following uses of the
function with a \texttt{\ set\ } rule.

Defines the default
\href{/docs/reference/model/enum/\#parameters-spacing}{spacing} of the
enumeration. If it is \texttt{\ }{\texttt{\ false\ }}\texttt{\ } , the
items are spaced apart with
\href{/docs/reference/model/par/\#parameters-spacing}{paragraph spacing}
. If it is \texttt{\ }{\texttt{\ true\ }}\texttt{\ } , they use
\href{/docs/reference/model/par/\#parameters-leading}{paragraph leading}
instead. This makes the list more compact, which can look better if the
items are short.

In markup mode, the value of this parameter is determined based on
whether items are separated with a blank line. If items directly follow
each other, this is set to \texttt{\ }{\texttt{\ true\ }}\texttt{\ } ;
if items are separated by a blank line, this is set to
\texttt{\ }{\texttt{\ false\ }}\texttt{\ } . The markup-defined
tightness cannot be overridden with set rules.

Default: \texttt{\ }{\texttt{\ true\ }}\texttt{\ }

\includesvg[width=0.16667in,height=0.16667in]{/assets/icons/16-arrow-right.svg}
View example

\begin{verbatim}
+ If an enum has a lot of text, and
  maybe other inline content, it
  should not be tight anymore.

+ To make an enum wide, simply
  insert a blank line between the
  items.
\end{verbatim}

\includegraphics[width=5in,height=\textheight,keepaspectratio]{/assets/docs/CGCi1WYCPLux25Xc9ZWwDQAAAAAAAAAA.png}

\subsubsection{\texorpdfstring{\texttt{\ numbering\ }}{ numbering }}\label{parameters-numbering}

\href{/docs/reference/foundations/str/}{str} {or}
\href{/docs/reference/foundations/function/}{function}

{{ Settable }}

\phantomsection\label{parameters-numbering-settable-tooltip}
Settable parameters can be customized for all following uses of the
function with a \texttt{\ set\ } rule.

How to number the enumeration. Accepts a
\href{/docs/reference/model/numbering/}{numbering pattern or function} .

If the numbering pattern contains multiple counting symbols, they apply
to nested enums. If given a function, the function receives one argument
if \texttt{\ full\ } is \texttt{\ }{\texttt{\ false\ }}\texttt{\ } and
multiple arguments if \texttt{\ full\ } is
\texttt{\ }{\texttt{\ true\ }}\texttt{\ } .

Default: \texttt{\ }{\texttt{\ "1."\ }}\texttt{\ }

\includesvg[width=0.16667in,height=0.16667in]{/assets/icons/16-arrow-right.svg}
View example

\begin{verbatim}
#set enum(numbering: "1.a)")
+ Different
+ Numbering
  + Nested
  + Items
+ Style

#set enum(numbering: n => super[#n])
+ Superscript
+ Numbering!
\end{verbatim}

\includegraphics[width=5in,height=\textheight,keepaspectratio]{/assets/docs/b_5poTPc-SH9hcwOp4TcbAAAAAAAAAAA.png}

\subsubsection{\texorpdfstring{\texttt{\ start\ }}{ start }}\label{parameters-start}

\href{/docs/reference/foundations/int/}{int}

{{ Settable }}

\phantomsection\label{parameters-start-settable-tooltip}
Settable parameters can be customized for all following uses of the
function with a \texttt{\ set\ } rule.

Which number to start the enumeration with.

Default: \texttt{\ }{\texttt{\ 1\ }}\texttt{\ }

\includesvg[width=0.16667in,height=0.16667in]{/assets/icons/16-arrow-right.svg}
View example

\begin{verbatim}
#enum(
  start: 3,
  [Skipping],
  [Ahead],
)
\end{verbatim}

\includegraphics[width=5in,height=\textheight,keepaspectratio]{/assets/docs/NqaMIUfLtrq2fhf9xChjagAAAAAAAAAA.png}

\subsubsection{\texorpdfstring{\texttt{\ full\ }}{ full }}\label{parameters-full}

\href{/docs/reference/foundations/bool/}{bool}

{{ Settable }}

\phantomsection\label{parameters-full-settable-tooltip}
Settable parameters can be customized for all following uses of the
function with a \texttt{\ set\ } rule.

Whether to display the full numbering, including the numbers of all
parent enumerations.

Default: \texttt{\ }{\texttt{\ false\ }}\texttt{\ }

\includesvg[width=0.16667in,height=0.16667in]{/assets/icons/16-arrow-right.svg}
View example

\begin{verbatim}
#set enum(numbering: "1.a)", full: true)
+ Cook
  + Heat water
  + Add ingredients
+ Eat
\end{verbatim}

\includegraphics[width=5in,height=\textheight,keepaspectratio]{/assets/docs/ecL0fn92ARx_6xbLZYFkVAAAAAAAAAAA.png}

\subsubsection{\texorpdfstring{\texttt{\ indent\ }}{ indent }}\label{parameters-indent}

\href{/docs/reference/layout/length/}{length}

{{ Settable }}

\phantomsection\label{parameters-indent-settable-tooltip}
Settable parameters can be customized for all following uses of the
function with a \texttt{\ set\ } rule.

The indentation of each item.

Default: \texttt{\ }{\texttt{\ 0pt\ }}\texttt{\ }

\subsubsection{\texorpdfstring{\texttt{\ body-indent\ }}{ body-indent }}\label{parameters-body-indent}

\href{/docs/reference/layout/length/}{length}

{{ Settable }}

\phantomsection\label{parameters-body-indent-settable-tooltip}
Settable parameters can be customized for all following uses of the
function with a \texttt{\ set\ } rule.

The space between the numbering and the body of each item.

Default: \texttt{\ }{\texttt{\ 0.5em\ }}\texttt{\ }

\subsubsection{\texorpdfstring{\texttt{\ spacing\ }}{ spacing }}\label{parameters-spacing}

\href{/docs/reference/foundations/auto/}{auto} {or}
\href{/docs/reference/layout/length/}{length}

{{ Settable }}

\phantomsection\label{parameters-spacing-settable-tooltip}
Settable parameters can be customized for all following uses of the
function with a \texttt{\ set\ } rule.

The spacing between the items of the enumeration.

If set to \texttt{\ }{\texttt{\ auto\ }}\texttt{\ } , uses paragraph
\href{/docs/reference/model/par/\#parameters-leading}{\texttt{\ leading\ }}
for tight enumerations and paragraph
\href{/docs/reference/model/par/\#parameters-spacing}{\texttt{\ spacing\ }}
for wide (non-tight) enumerations.

Default: \texttt{\ }{\texttt{\ auto\ }}\texttt{\ }

\subsubsection{\texorpdfstring{\texttt{\ number-align\ }}{ number-align }}\label{parameters-number-align}

\href{/docs/reference/layout/alignment/}{alignment}

{{ Settable }}

\phantomsection\label{parameters-number-align-settable-tooltip}
Settable parameters can be customized for all following uses of the
function with a \texttt{\ set\ } rule.

The alignment that enum numbers should have.

By default, this is set to
\texttt{\ end\ }{\texttt{\ +\ }}\texttt{\ top\ } , which aligns enum
numbers towards end of the current text direction (in left-to-right
script, for example, this is the same as \texttt{\ right\ } ) and at the
top of the line. The choice of \texttt{\ end\ } for horizontal alignment
of enum numbers is usually preferred over \texttt{\ start\ } , as
numbers then grow away from the text instead of towards it, avoiding
certain visual issues. This option lets you override this behaviour,
however. (Also to note is that the
\href{/docs/reference/model/list/}{unordered list} uses a different
method for this, by giving the \texttt{\ marker\ } content an alignment
directly.).

Default: \texttt{\ end\ }{\texttt{\ +\ }}\texttt{\ top\ }

\includesvg[width=0.16667in,height=0.16667in]{/assets/icons/16-arrow-right.svg}
View example

\begin{verbatim}
#set enum(number-align: start + bottom)

Here are some powers of two:
1. One
2. Two
4. Four
8. Eight
16. Sixteen
32. Thirty two
\end{verbatim}

\includegraphics[width=5in,height=\textheight,keepaspectratio]{/assets/docs/s-zUl9r9z6yKdW4VnsLi_AAAAAAAAAAA.png}

\subsubsection{\texorpdfstring{\texttt{\ children\ }}{ children }}\label{parameters-children}

\href{/docs/reference/foundations/content/}{content} {or}
\href{/docs/reference/foundations/array/}{array}

{Required} {{ Positional }}

\phantomsection\label{parameters-children-positional-tooltip}
Positional parameters are specified in order, without names.

{{ Variadic }}

\phantomsection\label{parameters-children-variadic-tooltip}
Variadic parameters can be specified multiple times.

The numbered list\textquotesingle s items.

When using the enum syntax, adjacent items are automatically collected
into enumerations, even through constructs like for loops.

\includesvg[width=0.16667in,height=0.16667in]{/assets/icons/16-arrow-right.svg}
View example

\begin{verbatim}
#for phase in (
   "Launch",
   "Orbit",
   "Descent",
) [+ #phase]
\end{verbatim}

\includegraphics[width=5in,height=\textheight,keepaspectratio]{/assets/docs/9haSHPkr8gDAx-1cEtmf8QAAAAAAAAAA.png}

\subsection{\texorpdfstring{{ Definitions
}}{ Definitions }}\label{definitions}

\phantomsection\label{definitions-tooltip}
Functions and types and can have associated definitions. These are
accessed by specifying the function or type, followed by a period, and
then the definition\textquotesingle s name.

\subsubsection{\texorpdfstring{\texttt{\ item\ } {{ Element
}}}{ item   Element }}\label{definitions-item}

\phantomsection\label{definitions-item-element-tooltip}
Element functions can be customized with \texttt{\ set\ } and
\texttt{\ show\ } rules.

An enumeration item.

enum { . } { item } (

{ \hyperref[definitions-item-parameters-number]{}
\href{/docs/reference/foundations/none/}{none}
\href{/docs/reference/foundations/int/}{int} , } {
\href{/docs/reference/foundations/content/}{content} , }

) -\textgreater{} \href{/docs/reference/foundations/content/}{content}

\paragraph{\texorpdfstring{\texttt{\ number\ }}{ number }}\label{definitions-item-number}

\href{/docs/reference/foundations/none/}{none} {or}
\href{/docs/reference/foundations/int/}{int}

{{ Positional }}

\phantomsection\label{definitions-item-number-positional-tooltip}
Positional parameters are specified in order, without names.

{{ Settable }}

\phantomsection\label{definitions-item-number-settable-tooltip}
Settable parameters can be customized for all following uses of the
function with a \texttt{\ set\ } rule.

The item\textquotesingle s number.

Default: \texttt{\ }{\texttt{\ none\ }}\texttt{\ }

\paragraph{\texorpdfstring{\texttt{\ body\ }}{ body }}\label{definitions-item-body}

\href{/docs/reference/foundations/content/}{content}

{Required} {{ Positional }}

\phantomsection\label{definitions-item-body-positional-tooltip}
Positional parameters are specified in order, without names.

The item\textquotesingle s body.

\href{/docs/reference/model/link/}{\pandocbounded{\includesvg[keepaspectratio]{/assets/icons/16-arrow-right.svg}}}

{ Link } { Previous page }

\href{/docs/reference/model/numbering/}{\pandocbounded{\includesvg[keepaspectratio]{/assets/icons/16-arrow-right.svg}}}

{ Numbering } { Next page }

\textbf{On this page}

\begin{itemize}
\tightlist
\item
  \hyperref[summary]{Summary}
\item
  \hyperref[example]{Example}
\item
  \hyperref[syntax]{Syntax}
\item
  \hyperref[parameters]{Parameters}

  \begin{itemize}
  \tightlist
  \item
    \hyperref[parameters-tight]{tight}
  \item
    \hyperref[parameters-numbering]{numbering}
  \item
    \hyperref[parameters-start]{start}
  \item
    \hyperref[parameters-full]{full}
  \item
    \hyperref[parameters-indent]{indent}
  \item
    \hyperref[parameters-body-indent]{body-indent}
  \item
    \hyperref[parameters-spacing]{spacing}
  \item
    \hyperref[parameters-number-align]{number-align}
  \item
    \hyperref[parameters-children]{children}
  \end{itemize}
\item
  \hyperref[definitions]{Definitions}

  \begin{itemize}
  \tightlist
  \item
    \hyperref[definitions-item]{Numbered List Item}

    \begin{itemize}
    \tightlist
    \item
      \hyperref[definitions-item-number]{number}
    \item
      \hyperref[definitions-item-body]{body}
    \end{itemize}
  \end{itemize}
\end{itemize}

\begin{itemize}
\tightlist
\item
  \href{/}{Home}
\item
  \href{/pricing/}{Pricing}
\item
  \href{/docs/}{Documentation}
\item
  \href{/universe/}{Universe}
\item
  \href{/about/}{About Us}
\item
  \href{/contact/}{Contact Us}
\item
  \href{/privacy/}{Privacy}
\item
  \href{https://typst.app/terms}{Terms and Conditions}
\item
  \href{/legal/}{Legal (Impressum)}
\end{itemize}

\begin{itemize}
\tightlist
\item
  \href{https://forum.typst.app}{Forum}
\item
  \href{/tools/}{Tools}
\item
  \href{/blog/}{Blog}
\item
  \href{https://github.com/typst/}{GitHub}
\item
  \href{https://discord.gg/2uDybryKPe}{Discord}
\item
  \href{https://mastodon.social/@typst}{Mastodon}
\item
  \href{https://bsky.app/profile/typst.app}{Bluesky}
\item
  \href{https://www.linkedin.com/company/typst/}{LinkedIn}
\item
  \href{https://instagram.com/typstapp/}{Instagram}
\end{itemize}

Made in Berlin
