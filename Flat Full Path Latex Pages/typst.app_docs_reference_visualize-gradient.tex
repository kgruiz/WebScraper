\title{typst.app/docs/reference/visualize/gradient}

\begin{itemize}
\tightlist
\item
  \href{/docs}{\includesvg[width=0.16667in,height=0.16667in]{/assets/icons/16-docs-dark.svg}}
\item
  \includesvg[width=0.16667in,height=0.16667in]{/assets/icons/16-arrow-right.svg}
\item
  \href{/docs/reference/}{Reference}
\item
  \includesvg[width=0.16667in,height=0.16667in]{/assets/icons/16-arrow-right.svg}
\item
  \href{/docs/reference/visualize/}{Visualize}
\item
  \includesvg[width=0.16667in,height=0.16667in]{/assets/icons/16-arrow-right.svg}
\item
  \href{/docs/reference/visualize/gradient/}{Gradient}
\end{itemize}

\section{\texorpdfstring{{ gradient }}{ gradient }}\label{summary}

A color gradient.

Typst supports linear gradients through the
\href{/docs/reference/visualize/gradient/\#definitions-linear}{\texttt{\ gradient.linear\ }
function} , radial gradients through the
\href{/docs/reference/visualize/gradient/\#definitions-radial}{\texttt{\ gradient.radial\ }
function} , and conic gradients through the
\href{/docs/reference/visualize/gradient/\#definitions-conic}{\texttt{\ gradient.conic\ }
function} .

A gradient can be used for the following purposes:

\begin{itemize}
\tightlist
\item
  As a fill to paint the interior of a shape:
  \texttt{\ }{\texttt{\ rect\ }}\texttt{\ }{\texttt{\ (\ }}\texttt{\ fill\ }{\texttt{\ :\ }}\texttt{\ gradient\ }{\texttt{\ .\ }}\texttt{\ }{\texttt{\ linear\ }}\texttt{\ }{\texttt{\ (\ }}\texttt{\ }{\texttt{\ ..\ }}\texttt{\ }{\texttt{\ )\ }}\texttt{\ }{\texttt{\ )\ }}\texttt{\ }
\item
  As a stroke to paint the outline of a shape:
  \texttt{\ }{\texttt{\ rect\ }}\texttt{\ }{\texttt{\ (\ }}\texttt{\ stroke\ }{\texttt{\ :\ }}\texttt{\ }{\texttt{\ 1pt\ }}\texttt{\ }{\texttt{\ +\ }}\texttt{\ gradient\ }{\texttt{\ .\ }}\texttt{\ }{\texttt{\ linear\ }}\texttt{\ }{\texttt{\ (\ }}\texttt{\ }{\texttt{\ ..\ }}\texttt{\ }{\texttt{\ )\ }}\texttt{\ }{\texttt{\ )\ }}\texttt{\ }
\item
  As the fill of text:
  \texttt{\ }{\texttt{\ set\ }}\texttt{\ }{\texttt{\ text\ }}\texttt{\ }{\texttt{\ (\ }}\texttt{\ fill\ }{\texttt{\ :\ }}\texttt{\ gradient\ }{\texttt{\ .\ }}\texttt{\ }{\texttt{\ linear\ }}\texttt{\ }{\texttt{\ (\ }}\texttt{\ }{\texttt{\ ..\ }}\texttt{\ }{\texttt{\ )\ }}\texttt{\ }{\texttt{\ )\ }}\texttt{\ }
\item
  As a color map you can
  \href{/docs/reference/visualize/gradient/\#definitions-sample}{sample}
  from:
  \texttt{\ gradient\ }{\texttt{\ .\ }}\texttt{\ }{\texttt{\ linear\ }}\texttt{\ }{\texttt{\ (\ }}\texttt{\ }{\texttt{\ ..\ }}\texttt{\ }{\texttt{\ )\ }}\texttt{\ }{\texttt{\ .\ }}\texttt{\ }{\texttt{\ sample\ }}\texttt{\ }{\texttt{\ (\ }}\texttt{\ }{\texttt{\ 50\%\ }}\texttt{\ }{\texttt{\ )\ }}\texttt{\ }
\end{itemize}

\subsection{Examples}\label{examples}

\begin{verbatim}
#stack(
  dir: ltr,
  spacing: 1fr,
  square(fill: gradient.linear(..color.map.rainbow)),
  square(fill: gradient.radial(..color.map.rainbow)),
  square(fill: gradient.conic(..color.map.rainbow)),
)
\end{verbatim}

\includegraphics[width=5in,height=\textheight,keepaspectratio]{/assets/docs/_ynuy5GKkV7ADtX87C9EiAAAAAAAAAAA.png}

Gradients are also supported on text, but only when setting the
\href{/docs/reference/visualize/gradient/\#definitions-relative}{relativeness}
to either \texttt{\ }{\texttt{\ auto\ }}\texttt{\ } (the default value)
or \texttt{\ }{\texttt{\ "parent"\ }}\texttt{\ } . To create
word-by-word or glyph-by-glyph gradients, you can wrap the words or
characters of your text in \href{/docs/reference/layout/box/}{boxes}
manually or through a \href{/docs/reference/styling/\#show-rules}{show
rule} .

\begin{verbatim}
#set text(fill: gradient.linear(red, blue))
#let rainbow(content) = {
  set text(fill: gradient.linear(..color.map.rainbow))
  box(content)
}

This is a gradient on text, but with a #rainbow[twist]!
\end{verbatim}

\includegraphics[width=4.85417in,height=\textheight,keepaspectratio]{/assets/docs/ch0LALUCwuQoVDnxrE_UZwAAAAAAAAAA.png}

\subsection{Stops}\label{stops}

A gradient is composed of a series of stops. Each of these stops has a
color and an offset. The offset is a
\href{/docs/reference/layout/ratio/}{ratio} between
\texttt{\ }{\texttt{\ 0\%\ }}\texttt{\ } and
\texttt{\ }{\texttt{\ 100\%\ }}\texttt{\ } or an angle between
\texttt{\ }{\texttt{\ 0deg\ }}\texttt{\ } and
\texttt{\ }{\texttt{\ 360deg\ }}\texttt{\ } . The offset is a relative
position that determines how far along the gradient the stop is located.
The stop\textquotesingle s color is the color of the gradient at that
position. You can choose to omit the offsets when defining a gradient.
In this case, Typst will space all stops evenly.

\subsection{Relativeness}\label{relativeness}

The location of the \texttt{\ }{\texttt{\ 0\%\ }}\texttt{\ } and
\texttt{\ }{\texttt{\ 100\%\ }}\texttt{\ } stops depends on the
dimensions of a container. This container can either be the shape that
it is being painted on, or the closest surrounding container. This is
controlled by the \texttt{\ relative\ } argument of a gradient
constructor. By default, gradients are relative to the shape they are
being painted on, unless the gradient is applied on text, in which case
they are relative to the closest ancestor container.

Typst determines the ancestor container as follows:

\begin{itemize}
\tightlist
\item
  For shapes that are placed at the root/top level of the document, the
  closest ancestor is the page itself.
\item
  For other shapes, the ancestor is the innermost
  \href{/docs/reference/layout/block/}{\texttt{\ block\ }} or
  \href{/docs/reference/layout/box/}{\texttt{\ box\ }} that contains the
  shape. This includes the boxes and blocks that are implicitly created
  by show rules and elements. For example, a
  \href{/docs/reference/layout/rotate/}{\texttt{\ rotate\ }} will not
  affect the parent of a gradient, but a
  \href{/docs/reference/layout/grid/}{\texttt{\ grid\ }} will.
\end{itemize}

\subsection{Color spaces and
interpolation}\label{color-spaces-and-interpolation}

Gradients can be interpolated in any color space. By default, gradients
are interpolated in the
\href{/docs/reference/visualize/color/\#definitions-oklab}{Oklab} color
space, which is a
\href{https://programmingdesignsystems.com/color/perceptually-uniform-color-spaces/index.html}{perceptually
uniform} color space. This means that the gradient will be perceived as
having a smooth progression of colors. This is particularly useful for
data visualization.

However, you can choose to interpolate the gradient in any supported
color space you want, but beware that some color spaces are not suitable
for perceptually interpolating between colors. Consult the table below
when choosing an interpolation space.

\begin{longtable}[]{@{}ll@{}}
\toprule\noalign{}
Color space & Perceptually uniform? \\
\midrule\noalign{}
\endhead
\bottomrule\noalign{}
\endlastfoot
\href{/docs/reference/visualize/color/\#definitions-oklab}{Oklab} &
\emph{Yes} \\
\href{/docs/reference/visualize/color/\#definitions-oklch}{Oklch} &
\emph{Yes} \\
\href{/docs/reference/visualize/color/\#definitions-rgb}{sRGB} &
\emph{No} \\
\href{/docs/reference/visualize/color/\#definitions-linear-rgb}{linear-RGB}
& \emph{Yes} \\
\href{/docs/reference/visualize/color/\#definitions-cmyk}{CMYK} &
\emph{No} \\
\href{/docs/reference/visualize/color/\#definitions-luma}{Grayscale} &
\emph{Yes} \\
\href{/docs/reference/visualize/color/\#definitions-hsl}{HSL} &
\emph{No} \\
\href{/docs/reference/visualize/color/\#definitions-hsv}{HSV} &
\emph{No} \\
\end{longtable}

\includegraphics[width=5in,height=\textheight,keepaspectratio]{/assets/docs/hDyhl3_sixunf7X8Ctx-hAAAAAAAAAAA.png}

\subsection{Direction}\label{direction}

Some gradients are sensitive to direction. For example, a linear
gradient has an angle that determines its direction. Typst uses a
clockwise angle, with 0° being from left to right, 90° from top to
bottom, 180° from right to left, and 270° from bottom to top.

\begin{verbatim}
#stack(
  dir: ltr,
  spacing: 1fr,
  square(fill: gradient.linear(red, blue, angle: 0deg)),
  square(fill: gradient.linear(red, blue, angle: 90deg)),
  square(fill: gradient.linear(red, blue, angle: 180deg)),
  square(fill: gradient.linear(red, blue, angle: 270deg)),
)
\end{verbatim}

\includegraphics[width=5in,height=\textheight,keepaspectratio]{/assets/docs/cXgxeaTP2ci7NL16a3rB_gAAAAAAAAAA.png}

\subsection{Presets}\label{presets}

Typst predefines color maps that you can use with your gradients. See
the
\href{/docs/reference/visualize/color/\#predefined-color-maps}{\texttt{\ color\ }}
documentation for more details.

\subsection{Note on file sizes}\label{note-on-file-sizes}

Gradients can be quite large, especially if they have many stops. This
is because gradients are stored as a list of colors and offsets, which
can take up a lot of space. If you are concerned about file sizes, you
should consider the following:

\begin{itemize}
\tightlist
\item
  SVG gradients are currently inefficiently encoded. This will be
  improved in the future.
\item
  PDF gradients in the
  \href{/docs/reference/visualize/color/\#definitions-oklab}{\texttt{\ color.oklab\ }}
  ,
  \href{/docs/reference/visualize/color/\#definitions-hsv}{\texttt{\ color.hsv\ }}
  ,
  \href{/docs/reference/visualize/color/\#definitions-hsl}{\texttt{\ color.hsl\ }}
  , and
  \href{/docs/reference/visualize/color/\#definitions-oklch}{\texttt{\ color.oklch\ }}
  color spaces are stored as a list of
  \href{/docs/reference/visualize/color/\#definitions-rgb}{\texttt{\ color.rgb\ }}
  colors with extra stops in between. This avoids needing to encode
  these color spaces in your PDF file, but it does add extra stops to
  your gradient, which can increase the file size.
\end{itemize}

\subsection{\texorpdfstring{{ Definitions
}}{ Definitions }}\label{definitions}

\phantomsection\label{definitions-tooltip}
Functions and types and can have associated definitions. These are
accessed by specifying the function or type, followed by a period, and
then the definition\textquotesingle s name.

\subsubsection{\texorpdfstring{\texttt{\ linear\ }}{ linear }}\label{definitions-linear}

Creates a new linear gradient, in which colors transition along a
straight line.

gradient { . } { linear } (

{ \hyperref[definitions-linear-parameters-stops]{..}
\href{/docs/reference/visualize/color/}{color}
\href{/docs/reference/foundations/array/}{array} , } {
\hyperref[definitions-linear-parameters-space]{space :} { any } , } {
\hyperref[definitions-linear-parameters-relative]{relative :}
\href{/docs/reference/foundations/auto/}{auto}
\href{/docs/reference/foundations/str/}{str} , } {
\href{/docs/reference/layout/direction/}{direction} , } {
\href{/docs/reference/layout/angle/}{angle} , }

) -\textgreater{} \href{/docs/reference/visualize/gradient/}{gradient}

\begin{verbatim}
#rect(
  width: 100%,
  height: 20pt,
  fill: gradient.linear(
    ..color.map.viridis,
  ),
)
\end{verbatim}

\includegraphics[width=5in,height=\textheight,keepaspectratio]{/assets/docs/3vCVaADmPcUqYOLma4-wcgAAAAAAAAAA.png}

\paragraph{\texorpdfstring{\texttt{\ stops\ }}{ stops }}\label{definitions-linear-stops}

\href{/docs/reference/visualize/color/}{color} {or}
\href{/docs/reference/foundations/array/}{array}

{Required} {{ Positional }}

\phantomsection\label{definitions-linear-stops-positional-tooltip}
Positional parameters are specified in order, without names.

{{ Variadic }}

\phantomsection\label{definitions-linear-stops-variadic-tooltip}
Variadic parameters can be specified multiple times.

The color \hyperref[stops]{stops} of the gradient.

\paragraph{\texorpdfstring{\texttt{\ space\ }}{ space }}\label{definitions-linear-space}

{ any }

The color space in which to interpolate the gradient.

Defaults to a perceptually uniform color space called
\href{/docs/reference/visualize/color/\#definitions-oklab}{Oklab} .

Default: \texttt{\ oklab\ }

\paragraph{\texorpdfstring{\texttt{\ relative\ }}{ relative }}\label{definitions-linear-relative}

\href{/docs/reference/foundations/auto/}{auto} {or}
\href{/docs/reference/foundations/str/}{str}

The \hyperref[relativeness]{relative placement} of the gradient.

For an element placed at the root/top level of the document, the parent
is the page itself. For other elements, the parent is the innermost
block, box, column, grid, or stack that contains the element.

\begin{longtable}[]{@{}ll@{}}
\toprule\noalign{}
Variant & Details \\
\midrule\noalign{}
\endhead
\bottomrule\noalign{}
\endlastfoot
\texttt{\ "\ self\ "\ } & The gradient is relative to itself (its own
bounding box). \\
\texttt{\ "\ parent\ "\ } & The gradient is relative to its parent (the
parent\textquotesingle s bounding box). \\
\end{longtable}

Default: \texttt{\ }{\texttt{\ auto\ }}\texttt{\ }

\paragraph{\texorpdfstring{\texttt{\ dir\ }}{ dir }}\label{definitions-linear-dir}

\href{/docs/reference/layout/direction/}{direction}

{{ Positional }}

\phantomsection\label{definitions-linear-dir-positional-tooltip}
Positional parameters are specified in order, without names.

The direction of the gradient.

Default: \texttt{\ ltr\ }

\paragraph{\texorpdfstring{\texttt{\ angle\ }}{ angle }}\label{definitions-linear-angle}

\href{/docs/reference/layout/angle/}{angle}

{Required} {{ Positional }}

\phantomsection\label{definitions-linear-angle-positional-tooltip}
Positional parameters are specified in order, without names.

The angle of the gradient.

\subsubsection{\texorpdfstring{\texttt{\ radial\ }}{ radial }}\label{definitions-radial}

Creates a new radial gradient, in which colors radiate away from an
origin.

The gradient is defined by two circles: the focal circle and the end
circle. The focal circle is a circle with center
\texttt{\ focal-center\ } and radius \texttt{\ focal-radius\ } , that
defines the points at which the gradient starts and has the color of the
first stop. The end circle is a circle with center \texttt{\ center\ }
and radius \texttt{\ radius\ } , that defines the points at which the
gradient ends and has the color of the last stop. The gradient is then
interpolated between these two circles.

Using these four values, also called the focal point for the starting
circle and the center and radius for the end circle, we can define a
gradient with more interesting properties than a basic radial gradient.

gradient { . } { radial } (

{ \hyperref[definitions-radial-parameters-stops]{..}
\href{/docs/reference/visualize/color/}{color}
\href{/docs/reference/foundations/array/}{array} , } {
\hyperref[definitions-radial-parameters-space]{space :} { any } , } {
\hyperref[definitions-radial-parameters-relative]{relative :}
\href{/docs/reference/foundations/auto/}{auto}
\href{/docs/reference/foundations/str/}{str} , } {
\hyperref[definitions-radial-parameters-center]{center :}
\href{/docs/reference/foundations/array/}{array} , } {
\hyperref[definitions-radial-parameters-radius]{radius :}
\href{/docs/reference/layout/ratio/}{ratio} , } {
\hyperref[definitions-radial-parameters-focal-center]{focal-center :}
\href{/docs/reference/foundations/auto/}{auto}
\href{/docs/reference/foundations/array/}{array} , } {
\hyperref[definitions-radial-parameters-focal-radius]{focal-radius :}
\href{/docs/reference/layout/ratio/}{ratio} , }

) -\textgreater{} \href{/docs/reference/visualize/gradient/}{gradient}

\begin{verbatim}
#stack(
  dir: ltr,
  spacing: 1fr,
  circle(fill: gradient.radial(
    ..color.map.viridis,
  )),
  circle(fill: gradient.radial(
    ..color.map.viridis,
    focal-center: (10%, 40%),
    focal-radius: 5%,
  )),
)
\end{verbatim}

\includegraphics[width=5in,height=\textheight,keepaspectratio]{/assets/docs/IfkE7bIcLhrH24l0nk0sIQAAAAAAAAAA.png}

\paragraph{\texorpdfstring{\texttt{\ stops\ }}{ stops }}\label{definitions-radial-stops}

\href{/docs/reference/visualize/color/}{color} {or}
\href{/docs/reference/foundations/array/}{array}

{Required} {{ Positional }}

\phantomsection\label{definitions-radial-stops-positional-tooltip}
Positional parameters are specified in order, without names.

{{ Variadic }}

\phantomsection\label{definitions-radial-stops-variadic-tooltip}
Variadic parameters can be specified multiple times.

The color \hyperref[stops]{stops} of the gradient.

\paragraph{\texorpdfstring{\texttt{\ space\ }}{ space }}\label{definitions-radial-space}

{ any }

The color space in which to interpolate the gradient.

Defaults to a perceptually uniform color space called
\href{/docs/reference/visualize/color/\#definitions-oklab}{Oklab} .

Default: \texttt{\ oklab\ }

\paragraph{\texorpdfstring{\texttt{\ relative\ }}{ relative }}\label{definitions-radial-relative}

\href{/docs/reference/foundations/auto/}{auto} {or}
\href{/docs/reference/foundations/str/}{str}

The \hyperref[relativeness]{relative placement} of the gradient.

For an element placed at the root/top level of the document, the parent
is the page itself. For other elements, the parent is the innermost
block, box, column, grid, or stack that contains the element.

\begin{longtable}[]{@{}ll@{}}
\toprule\noalign{}
Variant & Details \\
\midrule\noalign{}
\endhead
\bottomrule\noalign{}
\endlastfoot
\texttt{\ "\ self\ "\ } & The gradient is relative to itself (its own
bounding box). \\
\texttt{\ "\ parent\ "\ } & The gradient is relative to its parent (the
parent\textquotesingle s bounding box). \\
\end{longtable}

Default: \texttt{\ }{\texttt{\ auto\ }}\texttt{\ }

\paragraph{\texorpdfstring{\texttt{\ center\ }}{ center }}\label{definitions-radial-center}

\href{/docs/reference/foundations/array/}{array}

The center of the end circle of the gradient.

A value of
\texttt{\ }{\texttt{\ (\ }}\texttt{\ }{\texttt{\ 50\%\ }}\texttt{\ }{\texttt{\ ,\ }}\texttt{\ }{\texttt{\ 50\%\ }}\texttt{\ }{\texttt{\ )\ }}\texttt{\ }
means that the end circle is centered inside of its container.

Default:
\texttt{\ }{\texttt{\ (\ }}\texttt{\ }{\texttt{\ 50\%\ }}\texttt{\ }{\texttt{\ ,\ }}\texttt{\ }{\texttt{\ 50\%\ }}\texttt{\ }{\texttt{\ )\ }}\texttt{\ }

\paragraph{\texorpdfstring{\texttt{\ radius\ }}{ radius }}\label{definitions-radial-radius}

\href{/docs/reference/layout/ratio/}{ratio}

The radius of the end circle of the gradient.

By default, it is set to \texttt{\ }{\texttt{\ 50\%\ }}\texttt{\ } . The
ending radius must be bigger than the focal radius.

Default: \texttt{\ }{\texttt{\ 50\%\ }}\texttt{\ }

\paragraph{\texorpdfstring{\texttt{\ focal-center\ }}{ focal-center }}\label{definitions-radial-focal-center}

\href{/docs/reference/foundations/auto/}{auto} {or}
\href{/docs/reference/foundations/array/}{array}

The center of the focal circle of the gradient.

The focal center must be inside of the end circle.

A value of
\texttt{\ }{\texttt{\ (\ }}\texttt{\ }{\texttt{\ 50\%\ }}\texttt{\ }{\texttt{\ ,\ }}\texttt{\ }{\texttt{\ 50\%\ }}\texttt{\ }{\texttt{\ )\ }}\texttt{\ }
means that the focal circle is centered inside of its container.

By default it is set to the same as the center of the last circle.

Default: \texttt{\ }{\texttt{\ auto\ }}\texttt{\ }

\paragraph{\texorpdfstring{\texttt{\ focal-radius\ }}{ focal-radius }}\label{definitions-radial-focal-radius}

\href{/docs/reference/layout/ratio/}{ratio}

The radius of the focal circle of the gradient.

The focal center must be inside of the end circle.

By default, it is set to \texttt{\ }{\texttt{\ 0\%\ }}\texttt{\ } . The
focal radius must be smaller than the ending radius`.

Default: \texttt{\ }{\texttt{\ 0\%\ }}\texttt{\ }

\subsubsection{\texorpdfstring{\texttt{\ conic\ }}{ conic }}\label{definitions-conic}

Creates a new conic gradient, in which colors change radially around a
center point.

You can control the center point of the gradient by using the
\texttt{\ center\ } argument. By default, the center point is the center
of the shape.

gradient { . } { conic } (

{ \hyperref[definitions-conic-parameters-stops]{..}
\href{/docs/reference/visualize/color/}{color}
\href{/docs/reference/foundations/array/}{array} , } {
\hyperref[definitions-conic-parameters-angle]{angle :}
\href{/docs/reference/layout/angle/}{angle} , } {
\hyperref[definitions-conic-parameters-space]{space :} { any } , } {
\hyperref[definitions-conic-parameters-relative]{relative :}
\href{/docs/reference/foundations/auto/}{auto}
\href{/docs/reference/foundations/str/}{str} , } {
\hyperref[definitions-conic-parameters-center]{center :}
\href{/docs/reference/foundations/array/}{array} , }

) -\textgreater{} \href{/docs/reference/visualize/gradient/}{gradient}

\begin{verbatim}
#stack(
  dir: ltr,
  spacing: 1fr,
  circle(fill: gradient.conic(
    ..color.map.viridis,
  )),
  circle(fill: gradient.conic(
    ..color.map.viridis,
    center: (20%, 30%),
  )),
)
\end{verbatim}

\includegraphics[width=5in,height=\textheight,keepaspectratio]{/assets/docs/Mqmcewscuekk2Rsln7oKygAAAAAAAAAA.png}

\paragraph{\texorpdfstring{\texttt{\ stops\ }}{ stops }}\label{definitions-conic-stops}

\href{/docs/reference/visualize/color/}{color} {or}
\href{/docs/reference/foundations/array/}{array}

{Required} {{ Positional }}

\phantomsection\label{definitions-conic-stops-positional-tooltip}
Positional parameters are specified in order, without names.

{{ Variadic }}

\phantomsection\label{definitions-conic-stops-variadic-tooltip}
Variadic parameters can be specified multiple times.

The color \hyperref[stops]{stops} of the gradient.

\paragraph{\texorpdfstring{\texttt{\ angle\ }}{ angle }}\label{definitions-conic-angle}

\href{/docs/reference/layout/angle/}{angle}

The angle of the gradient.

Default: \texttt{\ }{\texttt{\ 0deg\ }}\texttt{\ }

\paragraph{\texorpdfstring{\texttt{\ space\ }}{ space }}\label{definitions-conic-space}

{ any }

The color space in which to interpolate the gradient.

Defaults to a perceptually uniform color space called
\href{/docs/reference/visualize/color/\#definitions-oklab}{Oklab} .

Default: \texttt{\ oklab\ }

\paragraph{\texorpdfstring{\texttt{\ relative\ }}{ relative }}\label{definitions-conic-relative}

\href{/docs/reference/foundations/auto/}{auto} {or}
\href{/docs/reference/foundations/str/}{str}

The \hyperref[relativeness]{relative placement} of the gradient.

For an element placed at the root/top level of the document, the parent
is the page itself. For other elements, the parent is the innermost
block, box, column, grid, or stack that contains the element.

\begin{longtable}[]{@{}ll@{}}
\toprule\noalign{}
Variant & Details \\
\midrule\noalign{}
\endhead
\bottomrule\noalign{}
\endlastfoot
\texttt{\ "\ self\ "\ } & The gradient is relative to itself (its own
bounding box). \\
\texttt{\ "\ parent\ "\ } & The gradient is relative to its parent (the
parent\textquotesingle s bounding box). \\
\end{longtable}

Default: \texttt{\ }{\texttt{\ auto\ }}\texttt{\ }

\paragraph{\texorpdfstring{\texttt{\ center\ }}{ center }}\label{definitions-conic-center}

\href{/docs/reference/foundations/array/}{array}

The center of the last circle of the gradient.

A value of
\texttt{\ }{\texttt{\ (\ }}\texttt{\ }{\texttt{\ 50\%\ }}\texttt{\ }{\texttt{\ ,\ }}\texttt{\ }{\texttt{\ 50\%\ }}\texttt{\ }{\texttt{\ )\ }}\texttt{\ }
means that the end circle is centered inside of its container.

Default:
\texttt{\ }{\texttt{\ (\ }}\texttt{\ }{\texttt{\ 50\%\ }}\texttt{\ }{\texttt{\ ,\ }}\texttt{\ }{\texttt{\ 50\%\ }}\texttt{\ }{\texttt{\ )\ }}\texttt{\ }

\subsubsection{\texorpdfstring{\texttt{\ sharp\ }}{ sharp }}\label{definitions-sharp}

Creates a sharp version of this gradient.

Sharp gradients have discrete jumps between colors, instead of a smooth
transition. They are particularly useful for creating color lists for a
preset gradient.

self { . } { sharp } (

{ \href{/docs/reference/foundations/int/}{int} , } {
\hyperref[definitions-sharp-parameters-smoothness]{smoothness :}
\href{/docs/reference/layout/ratio/}{ratio} , }

) -\textgreater{} \href{/docs/reference/visualize/gradient/}{gradient}

\begin{verbatim}
#set rect(width: 100%, height: 20pt)
#let grad = gradient.linear(..color.map.rainbow)
#rect(fill: grad)
#rect(fill: grad.sharp(5))
#rect(fill: grad.sharp(5, smoothness: 20%))
\end{verbatim}

\includegraphics[width=5in,height=\textheight,keepaspectratio]{/assets/docs/k1IrJvVHW9DTjXfwHdfh_QAAAAAAAAAA.png}

\paragraph{\texorpdfstring{\texttt{\ steps\ }}{ steps }}\label{definitions-sharp-steps}

\href{/docs/reference/foundations/int/}{int}

{Required} {{ Positional }}

\phantomsection\label{definitions-sharp-steps-positional-tooltip}
Positional parameters are specified in order, without names.

The number of stops in the gradient.

\paragraph{\texorpdfstring{\texttt{\ smoothness\ }}{ smoothness }}\label{definitions-sharp-smoothness}

\href{/docs/reference/layout/ratio/}{ratio}

How much to smooth the gradient.

Default: \texttt{\ }{\texttt{\ 0\%\ }}\texttt{\ }

\subsubsection{\texorpdfstring{\texttt{\ repeat\ }}{ repeat }}\label{definitions-repeat}

Repeats this gradient a given number of times, optionally mirroring it
at each repetition.

self { . } { repeat } (

{ \href{/docs/reference/foundations/int/}{int} , } {
\hyperref[definitions-repeat-parameters-mirror]{mirror :}
\href{/docs/reference/foundations/bool/}{bool} , }

) -\textgreater{} \href{/docs/reference/visualize/gradient/}{gradient}

\begin{verbatim}
#circle(
  radius: 40pt,
  fill: gradient
    .radial(aqua, white)
    .repeat(4),
)
\end{verbatim}

\includegraphics[width=5in,height=\textheight,keepaspectratio]{/assets/docs/ydbGAMwgwvGMCJpfMs1wAAAAAAAAAAAA.png}

\paragraph{\texorpdfstring{\texttt{\ repetitions\ }}{ repetitions }}\label{definitions-repeat-repetitions}

\href{/docs/reference/foundations/int/}{int}

{Required} {{ Positional }}

\phantomsection\label{definitions-repeat-repetitions-positional-tooltip}
Positional parameters are specified in order, without names.

The number of times to repeat the gradient.

\paragraph{\texorpdfstring{\texttt{\ mirror\ }}{ mirror }}\label{definitions-repeat-mirror}

\href{/docs/reference/foundations/bool/}{bool}

Whether to mirror the gradient at each repetition.

Default: \texttt{\ }{\texttt{\ false\ }}\texttt{\ }

\subsubsection{\texorpdfstring{\texttt{\ kind\ }}{ kind }}\label{definitions-kind}

Returns the kind of this gradient.

self { . } { kind } (

) -\textgreater{} \href{/docs/reference/foundations/function/}{function}

\subsubsection{\texorpdfstring{\texttt{\ stops\ }}{ stops }}\label{definitions-stops}

Returns the stops of this gradient.

self { . } { stops } (

) -\textgreater{} \href{/docs/reference/foundations/array/}{array}

\subsubsection{\texorpdfstring{\texttt{\ space\ }}{ space }}\label{definitions-space}

Returns the mixing space of this gradient.

self { . } { space } (

) -\textgreater{} { any }

\subsubsection{\texorpdfstring{\texttt{\ relative\ }}{ relative }}\label{definitions-relative}

Returns the relative placement of this gradient.

self { . } { relative } (

) -\textgreater{} \href{/docs/reference/foundations/auto/}{auto}

\subsubsection{\texorpdfstring{\texttt{\ angle\ }}{ angle }}\label{definitions-angle}

Returns the angle of this gradient.

self { . } { angle } (

) -\textgreater{} \href{/docs/reference/foundations/none/}{none}
\href{/docs/reference/layout/angle/}{angle}

\subsubsection{\texorpdfstring{\texttt{\ sample\ }}{ sample }}\label{definitions-sample}

Sample the gradient at a given position.

The position is either a position along the gradient (a
\href{/docs/reference/layout/ratio/}{ratio} between
\texttt{\ }{\texttt{\ 0\%\ }}\texttt{\ } and
\texttt{\ }{\texttt{\ 100\%\ }}\texttt{\ } ) or an
\href{/docs/reference/layout/angle/}{angle} . Any value outside of this
range will be clamped.

self { . } { sample } (

{ \href{/docs/reference/layout/angle/}{angle}
\href{/docs/reference/layout/ratio/}{ratio} }

) -\textgreater{} \href{/docs/reference/visualize/color/}{color}

\paragraph{\texorpdfstring{\texttt{\ t\ }}{ t }}\label{definitions-sample-t}

\href{/docs/reference/layout/angle/}{angle} {or}
\href{/docs/reference/layout/ratio/}{ratio}

{Required} {{ Positional }}

\phantomsection\label{definitions-sample-t-positional-tooltip}
Positional parameters are specified in order, without names.

The position at which to sample the gradient.

\subsubsection{\texorpdfstring{\texttt{\ samples\ }}{ samples }}\label{definitions-samples}

Samples the gradient at multiple positions at once and returns the
results as an array.

self { . } { samples } (

{ \hyperref[definitions-samples-parameters-ts]{..}
\href{/docs/reference/layout/angle/}{angle}
\href{/docs/reference/layout/ratio/}{ratio} }

) -\textgreater{} \href{/docs/reference/foundations/array/}{array}

\paragraph{\texorpdfstring{\texttt{\ ts\ }}{ ts }}\label{definitions-samples-ts}

\href{/docs/reference/layout/angle/}{angle} {or}
\href{/docs/reference/layout/ratio/}{ratio}

{Required} {{ Positional }}

\phantomsection\label{definitions-samples-ts-positional-tooltip}
Positional parameters are specified in order, without names.

{{ Variadic }}

\phantomsection\label{definitions-samples-ts-variadic-tooltip}
Variadic parameters can be specified multiple times.

The positions at which to sample the gradient.

\href{/docs/reference/visualize/ellipse/}{\pandocbounded{\includesvg[keepaspectratio]{/assets/icons/16-arrow-right.svg}}}

{ Ellipse } { Previous page }

\href{/docs/reference/visualize/image/}{\pandocbounded{\includesvg[keepaspectratio]{/assets/icons/16-arrow-right.svg}}}

{ Image } { Next page }

\textbf{On this page}

\begin{itemize}
\tightlist
\item
  \hyperref[summary]{Summary}
\item
  \hyperref[examples]{Examples}
\item
  \hyperref[stops]{Stops}
\item
  \hyperref[relativeness]{Relativeness}
\item
  \hyperref[color-spaces-and-interpolation]{Color Spaces And
  Interpolation}
\item
  \hyperref[direction]{Direction}
\item
  \hyperref[presets]{Presets}
\item
  \hyperref[note-on-file-sizes]{Note On File Sizes}
\item
  \hyperref[definitions]{Definitions}

  \begin{itemize}
  \tightlist
  \item
    \hyperref[definitions-linear]{Linear Gradient}

    \begin{itemize}
    \tightlist
    \item
      \hyperref[definitions-linear-stops]{stops}
    \item
      \hyperref[definitions-linear-space]{space}
    \item
      \hyperref[definitions-linear-relative]{relative}
    \item
      \hyperref[definitions-linear-dir]{dir}
    \item
      \hyperref[definitions-linear-angle]{angle}
    \end{itemize}
  \item
    \hyperref[definitions-radial]{Radial}

    \begin{itemize}
    \tightlist
    \item
      \hyperref[definitions-radial-stops]{stops}
    \item
      \hyperref[definitions-radial-space]{space}
    \item
      \hyperref[definitions-radial-relative]{relative}
    \item
      \hyperref[definitions-radial-center]{center}
    \item
      \hyperref[definitions-radial-radius]{radius}
    \item
      \hyperref[definitions-radial-focal-center]{focal-center}
    \item
      \hyperref[definitions-radial-focal-radius]{focal-radius}
    \end{itemize}
  \item
    \hyperref[definitions-conic]{Conic}

    \begin{itemize}
    \tightlist
    \item
      \hyperref[definitions-conic-stops]{stops}
    \item
      \hyperref[definitions-conic-angle]{angle}
    \item
      \hyperref[definitions-conic-space]{space}
    \item
      \hyperref[definitions-conic-relative]{relative}
    \item
      \hyperref[definitions-conic-center]{center}
    \end{itemize}
  \item
    \hyperref[definitions-sharp]{Sharp}

    \begin{itemize}
    \tightlist
    \item
      \hyperref[definitions-sharp-steps]{steps}
    \item
      \hyperref[definitions-sharp-smoothness]{smoothness}
    \end{itemize}
  \item
    \hyperref[definitions-repeat]{Repeat}

    \begin{itemize}
    \tightlist
    \item
      \hyperref[definitions-repeat-repetitions]{repetitions}
    \item
      \hyperref[definitions-repeat-mirror]{mirror}
    \end{itemize}
  \item
    \hyperref[definitions-kind]{Kind}
  \item
    \hyperref[definitions-stops]{Stops}
  \item
    \hyperref[definitions-space]{Space}
  \item
    \hyperref[definitions-relative]{Relative}
  \item
    \hyperref[definitions-angle]{Angle}
  \item
    \hyperref[definitions-sample]{Sample}

    \begin{itemize}
    \tightlist
    \item
      \hyperref[definitions-sample-t]{t}
    \end{itemize}
  \item
    \hyperref[definitions-samples]{Samples}

    \begin{itemize}
    \tightlist
    \item
      \hyperref[definitions-samples-ts]{ts}
    \end{itemize}
  \end{itemize}
\end{itemize}

\begin{itemize}
\tightlist
\item
  \href{/}{Home}
\item
  \href{/pricing/}{Pricing}
\item
  \href{/docs/}{Documentation}
\item
  \href{/universe/}{Universe}
\item
  \href{/about/}{About Us}
\item
  \href{/contact/}{Contact Us}
\item
  \href{/privacy/}{Privacy}
\item
  \href{https://typst.app/terms}{Terms and Conditions}
\item
  \href{/legal/}{Legal (Impressum)}
\end{itemize}

\begin{itemize}
\tightlist
\item
  \href{https://forum.typst.app}{Forum}
\item
  \href{/tools/}{Tools}
\item
  \href{/blog/}{Blog}
\item
  \href{https://github.com/typst/}{GitHub}
\item
  \href{https://discord.gg/2uDybryKPe}{Discord}
\item
  \href{https://mastodon.social/@typst}{Mastodon}
\item
  \href{https://bsky.app/profile/typst.app}{Bluesky}
\item
  \href{https://www.linkedin.com/company/typst/}{LinkedIn}
\item
  \href{https://instagram.com/typstapp/}{Instagram}
\end{itemize}

Made in Berlin
