\title{typst.app/docs/reference/model/terms}

\begin{itemize}
\tightlist
\item
  \href{/docs}{\includesvg[width=0.16667in,height=0.16667in]{/assets/icons/16-docs-dark.svg}}
\item
  \includesvg[width=0.16667in,height=0.16667in]{/assets/icons/16-arrow-right.svg}
\item
  \href{/docs/reference/}{Reference}
\item
  \includesvg[width=0.16667in,height=0.16667in]{/assets/icons/16-arrow-right.svg}
\item
  \href{/docs/reference/model/}{Model}
\item
  \includesvg[width=0.16667in,height=0.16667in]{/assets/icons/16-arrow-right.svg}
\item
  \href{/docs/reference/model/terms/}{Term List}
\end{itemize}

\section{\texorpdfstring{\texttt{\ terms\ } {{ Element
}}}{ terms   Element }}\label{summary}

\phantomsection\label{element-tooltip}
Element functions can be customized with \texttt{\ set\ } and
\texttt{\ show\ } rules.

A list of terms and their descriptions.

Displays a sequence of terms and their descriptions vertically. When the
descriptions span over multiple lines, they use hanging indent to
communicate the visual hierarchy.

\subsection{Example}\label{example}

\begin{verbatim}
/ Ligature: A merged glyph.
/ Kerning: A spacing adjustment
  between two adjacent letters.
\end{verbatim}

\includegraphics[width=5in,height=\textheight,keepaspectratio]{/assets/docs/qjdQTTJFa_RYtcfu42IiawAAAAAAAAAA.png}

\subsection{Syntax}\label{syntax}

This function also has dedicated syntax: Starting a line with a slash,
followed by a term, a colon and a description creates a term list item.

\subsection{\texorpdfstring{{ Parameters
}}{ Parameters }}\label{parameters}

\phantomsection\label{parameters-tooltip}
Parameters are the inputs to a function. They are specified in
parentheses after the function name.

{ terms } (

{ \hyperref[parameters-tight]{tight :}
\href{/docs/reference/foundations/bool/}{bool} , } {
\hyperref[parameters-separator]{separator :}
\href{/docs/reference/foundations/content/}{content} , } {
\hyperref[parameters-indent]{indent :}
\href{/docs/reference/layout/length/}{length} , } {
\hyperref[parameters-hanging-indent]{hanging-indent :}
\href{/docs/reference/layout/length/}{length} , } {
\hyperref[parameters-spacing]{spacing :}
\href{/docs/reference/foundations/auto/}{auto}
\href{/docs/reference/layout/length/}{length} , } {
\hyperref[parameters-children]{..}
\href{/docs/reference/foundations/content/}{content}
\href{/docs/reference/foundations/array/}{array} , }

) -\textgreater{} \href{/docs/reference/foundations/content/}{content}

\subsubsection{\texorpdfstring{\texttt{\ tight\ }}{ tight }}\label{parameters-tight}

\href{/docs/reference/foundations/bool/}{bool}

{{ Settable }}

\phantomsection\label{parameters-tight-settable-tooltip}
Settable parameters can be customized for all following uses of the
function with a \texttt{\ set\ } rule.

Defines the default
\href{/docs/reference/model/terms/\#parameters-spacing}{spacing} of the
term list. If it is \texttt{\ }{\texttt{\ false\ }}\texttt{\ } , the
items are spaced apart with
\href{/docs/reference/model/par/\#parameters-spacing}{paragraph spacing}
. If it is \texttt{\ }{\texttt{\ true\ }}\texttt{\ } , they use
\href{/docs/reference/model/par/\#parameters-leading}{paragraph leading}
instead. This makes the list more compact, which can look better if the
items are short.

In markup mode, the value of this parameter is determined based on
whether items are separated with a blank line. If items directly follow
each other, this is set to \texttt{\ }{\texttt{\ true\ }}\texttt{\ } ;
if items are separated by a blank line, this is set to
\texttt{\ }{\texttt{\ false\ }}\texttt{\ } . The markup-defined
tightness cannot be overridden with set rules.

Default: \texttt{\ }{\texttt{\ true\ }}\texttt{\ }

\includesvg[width=0.16667in,height=0.16667in]{/assets/icons/16-arrow-right.svg}
View example

\begin{verbatim}
/ Fact: If a term list has a lot
  of text, and maybe other inline
  content, it should not be tight
  anymore.

/ Tip: To make it wide, simply
  insert a blank line between the
  items.
\end{verbatim}

\includegraphics[width=5in,height=\textheight,keepaspectratio]{/assets/docs/skkuR2BgltlCHUy9cPpX7gAAAAAAAAAA.png}

\subsubsection{\texorpdfstring{\texttt{\ separator\ }}{ separator }}\label{parameters-separator}

\href{/docs/reference/foundations/content/}{content}

{{ Settable }}

\phantomsection\label{parameters-separator-settable-tooltip}
Settable parameters can be customized for all following uses of the
function with a \texttt{\ set\ } rule.

The separator between the item and the description.

If you want to just separate them with a certain amount of space, use
\texttt{\ }{\texttt{\ h\ }}\texttt{\ }{\texttt{\ (\ }}\texttt{\ }{\texttt{\ 2cm\ }}\texttt{\ }{\texttt{\ ,\ }}\texttt{\ weak\ }{\texttt{\ :\ }}\texttt{\ }{\texttt{\ true\ }}\texttt{\ }{\texttt{\ )\ }}\texttt{\ }
as the separator and replace \texttt{\ }{\texttt{\ 2cm\ }}\texttt{\ }
with your desired amount of space.

Default:
\texttt{\ }{\texttt{\ h\ }}\texttt{\ }{\texttt{\ (\ }}\texttt{\ amount\ }{\texttt{\ :\ }}\texttt{\ }{\texttt{\ 0.6em\ }}\texttt{\ }{\texttt{\ ,\ }}\texttt{\ weak\ }{\texttt{\ :\ }}\texttt{\ }{\texttt{\ true\ }}\texttt{\ }{\texttt{\ )\ }}\texttt{\ }

\includesvg[width=0.16667in,height=0.16667in]{/assets/icons/16-arrow-right.svg}
View example

\begin{verbatim}
#set terms(separator: [: ])

/ Colon: A nice separator symbol.
\end{verbatim}

\includegraphics[width=5in,height=\textheight,keepaspectratio]{/assets/docs/xyyblMI8l_99lTt1_p5kWgAAAAAAAAAA.png}

\subsubsection{\texorpdfstring{\texttt{\ indent\ }}{ indent }}\label{parameters-indent}

\href{/docs/reference/layout/length/}{length}

{{ Settable }}

\phantomsection\label{parameters-indent-settable-tooltip}
Settable parameters can be customized for all following uses of the
function with a \texttt{\ set\ } rule.

The indentation of each item.

Default: \texttt{\ }{\texttt{\ 0pt\ }}\texttt{\ }

\subsubsection{\texorpdfstring{\texttt{\ hanging-indent\ }}{ hanging-indent }}\label{parameters-hanging-indent}

\href{/docs/reference/layout/length/}{length}

{{ Settable }}

\phantomsection\label{parameters-hanging-indent-settable-tooltip}
Settable parameters can be customized for all following uses of the
function with a \texttt{\ set\ } rule.

The hanging indent of the description.

This is in addition to the whole item\textquotesingle s
\texttt{\ indent\ } .

Default: \texttt{\ }{\texttt{\ 2em\ }}\texttt{\ }

\includesvg[width=0.16667in,height=0.16667in]{/assets/icons/16-arrow-right.svg}
View example

\begin{verbatim}
#set terms(hanging-indent: 0pt)
/ Term: This term list does not
  make use of hanging indents.
\end{verbatim}

\includegraphics[width=5in,height=\textheight,keepaspectratio]{/assets/docs/6yYrKErT2JtRwBRmpS8r5wAAAAAAAAAA.png}

\subsubsection{\texorpdfstring{\texttt{\ spacing\ }}{ spacing }}\label{parameters-spacing}

\href{/docs/reference/foundations/auto/}{auto} {or}
\href{/docs/reference/layout/length/}{length}

{{ Settable }}

\phantomsection\label{parameters-spacing-settable-tooltip}
Settable parameters can be customized for all following uses of the
function with a \texttt{\ set\ } rule.

The spacing between the items of the term list.

If set to \texttt{\ }{\texttt{\ auto\ }}\texttt{\ } , uses paragraph
\href{/docs/reference/model/par/\#parameters-leading}{\texttt{\ leading\ }}
for tight term lists and paragraph
\href{/docs/reference/model/par/\#parameters-spacing}{\texttt{\ spacing\ }}
for wide (non-tight) term lists.

Default: \texttt{\ }{\texttt{\ auto\ }}\texttt{\ }

\subsubsection{\texorpdfstring{\texttt{\ children\ }}{ children }}\label{parameters-children}

\href{/docs/reference/foundations/content/}{content} {or}
\href{/docs/reference/foundations/array/}{array}

{Required} {{ Positional }}

\phantomsection\label{parameters-children-positional-tooltip}
Positional parameters are specified in order, without names.

{{ Variadic }}

\phantomsection\label{parameters-children-variadic-tooltip}
Variadic parameters can be specified multiple times.

The term list\textquotesingle s children.

When using the term list syntax, adjacent items are automatically
collected into term lists, even through constructs like for loops.

\includesvg[width=0.16667in,height=0.16667in]{/assets/icons/16-arrow-right.svg}
View example

\begin{verbatim}
#for (year, product) in (
  "1978": "TeX",
  "1984": "LaTeX",
  "2019": "Typst",
) [/ #product: Born in #year.]
\end{verbatim}

\includegraphics[width=5in,height=\textheight,keepaspectratio]{/assets/docs/wkvQM6jeTkSTRoaT9Y0lSQAAAAAAAAAA.png}

\subsection{\texorpdfstring{{ Definitions
}}{ Definitions }}\label{definitions}

\phantomsection\label{definitions-tooltip}
Functions and types and can have associated definitions. These are
accessed by specifying the function or type, followed by a period, and
then the definition\textquotesingle s name.

\subsubsection{\texorpdfstring{\texttt{\ item\ } {{ Element
}}}{ item   Element }}\label{definitions-item}

\phantomsection\label{definitions-item-element-tooltip}
Element functions can be customized with \texttt{\ set\ } and
\texttt{\ show\ } rules.

A term list item.

terms { . } { item } (

{ \href{/docs/reference/foundations/content/}{content} , } {
\href{/docs/reference/foundations/content/}{content} , }

) -\textgreater{} \href{/docs/reference/foundations/content/}{content}

\paragraph{\texorpdfstring{\texttt{\ term\ }}{ term }}\label{definitions-item-term}

\href{/docs/reference/foundations/content/}{content}

{Required} {{ Positional }}

\phantomsection\label{definitions-item-term-positional-tooltip}
Positional parameters are specified in order, without names.

The term described by the list item.

\paragraph{\texorpdfstring{\texttt{\ description\ }}{ description }}\label{definitions-item-description}

\href{/docs/reference/foundations/content/}{content}

{Required} {{ Positional }}

\phantomsection\label{definitions-item-description-positional-tooltip}
Positional parameters are specified in order, without names.

The description of the term.

\href{/docs/reference/model/table/}{\pandocbounded{\includesvg[keepaspectratio]{/assets/icons/16-arrow-right.svg}}}

{ Table } { Previous page }

\href{/docs/reference/text/}{\pandocbounded{\includesvg[keepaspectratio]{/assets/icons/16-arrow-right.svg}}}

{ Text } { Next page }

\textbf{On this page}

\begin{itemize}
\tightlist
\item
  \hyperref[summary]{Summary}
\item
  \hyperref[example]{Example}
\item
  \hyperref[syntax]{Syntax}
\item
  \hyperref[parameters]{Parameters}

  \begin{itemize}
  \tightlist
  \item
    \hyperref[parameters-tight]{tight}
  \item
    \hyperref[parameters-separator]{separator}
  \item
    \hyperref[parameters-indent]{indent}
  \item
    \hyperref[parameters-hanging-indent]{hanging-indent}
  \item
    \hyperref[parameters-spacing]{spacing}
  \item
    \hyperref[parameters-children]{children}
  \end{itemize}
\item
  \hyperref[definitions]{Definitions}

  \begin{itemize}
  \tightlist
  \item
    \hyperref[definitions-item]{Term List Item}

    \begin{itemize}
    \tightlist
    \item
      \hyperref[definitions-item-term]{term}
    \item
      \hyperref[definitions-item-description]{description}
    \end{itemize}
  \end{itemize}
\end{itemize}

\begin{itemize}
\tightlist
\item
  \href{/}{Home}
\item
  \href{/pricing/}{Pricing}
\item
  \href{/docs/}{Documentation}
\item
  \href{/universe/}{Universe}
\item
  \href{/about/}{About Us}
\item
  \href{/contact/}{Contact Us}
\item
  \href{/privacy/}{Privacy}
\item
  \href{https://typst.app/terms}{Terms and Conditions}
\item
  \href{/legal/}{Legal (Impressum)}
\end{itemize}

\begin{itemize}
\tightlist
\item
  \href{https://forum.typst.app}{Forum}
\item
  \href{/tools/}{Tools}
\item
  \href{/blog/}{Blog}
\item
  \href{https://github.com/typst/}{GitHub}
\item
  \href{https://discord.gg/2uDybryKPe}{Discord}
\item
  \href{https://mastodon.social/@typst}{Mastodon}
\item
  \href{https://bsky.app/profile/typst.app}{Bluesky}
\item
  \href{https://www.linkedin.com/company/typst/}{LinkedIn}
\item
  \href{https://instagram.com/typstapp/}{Instagram}
\end{itemize}

Made in Berlin
