\title{typst.app/docs/tutorial/formatting}

\begin{itemize}
\tightlist
\item
  \href{/docs}{\includesvg[width=0.16667in,height=0.16667in]{/assets/icons/16-docs-dark.svg}}
\item
  \includesvg[width=0.16667in,height=0.16667in]{/assets/icons/16-arrow-right.svg}
\item
  \href{/docs/tutorial/}{Tutorial}
\item
  \includesvg[width=0.16667in,height=0.16667in]{/assets/icons/16-arrow-right.svg}
\item
  \href{/docs/tutorial/formatting/}{Formatting}
\end{itemize}

\section{Formatting}\label{formatting}

So far, you have written a report with some text, a few equations and
images. However, it still looks very plain. Your teaching assistant does
not yet know that you are using a new typesetting system, and you want
your report to fit in with the other student\textquotesingle s
submissions. In this chapter, we will see how to format your report
using Typst\textquotesingle s styling system.

\subsection{Set rules}\label{set-rules}

As we have seen in the previous chapter, Typst has functions that
\emph{insert} content (e.g. the
\href{/docs/reference/visualize/image/}{\texttt{\ image\ }} function)
and others that \emph{manipulate} content that they received as
arguments (e.g. the
\href{/docs/reference/layout/align/}{\texttt{\ align\ }} function). The
first impulse you might have when you want, for example, to justify the
report, could be to look for a function that does that and wrap the
complete document in it.

\begin{verbatim}
#par(justify: true)[
  = Background
  In the case of glaciers, fluid
  dynamics principles can be used
  to understand how the movement
  and behaviour of the ice is
  influenced by factors such as
  temperature, pressure, and the
  presence of other fluids (such as
  water).
]
\end{verbatim}

\includegraphics[width=5in,height=\textheight,keepaspectratio]{/assets/docs/Dijg8l-irnssXE7n_oJpJQAAAAAAAAAA.png}

Wait, shouldn\textquotesingle t all arguments of a function be specified
within parentheses? Why is there a second set of square brackets with
content \emph{after} the parentheses? The answer is that, as passing
content to a function is such a common thing to do in Typst, there is
special syntax for it: Instead of putting the content inside of the
argument list, you can write it in square brackets directly after the
normal arguments, saving on punctuation.

As seen above, that works. The
\href{/docs/reference/model/par/}{\texttt{\ par\ }} function justifies
all paragraphs within it. However, wrapping the document in countless
functions and applying styles selectively and in-situ can quickly become
cumbersome.

Fortunately, Typst has a more elegant solution. With \emph{set rules,}
you can apply style properties to all occurrences of some kind of
content. You write a set rule by entering the
\texttt{\ }{\texttt{\ set\ }}\texttt{\ } keyword, followed by the name
of the function whose properties you want to set, and a list of
arguments in parentheses.

\begin{verbatim}
#set par(justify: true)

= Background
In the case of glaciers, fluid
dynamics principles can be used
to understand how the movement
and behaviour of the ice is
influenced by factors such as
temperature, pressure, and the
presence of other fluids (such as
water).
\end{verbatim}

\includegraphics[width=5in,height=\textheight,keepaspectratio]{/assets/docs/JHqbSYpLaF9kuNFQoo1lAgAAAAAAAAAA.png}

Want to know in more technical terms what is happening here?

Set rules can be conceptualized as setting default values for some of
the parameters of a function for all future uses of that function.

\subsection{The autocomplete panel}\label{autocomplete}

If you followed along and tried a few things in the app, you might have
noticed that always after you enter a \texttt{\ \#\ } character, a panel
pops up to show you the available functions, and, within an argument
list, the available parameters. That\textquotesingle s the autocomplete
panel. It can be very useful while you are writing your document: You
can apply its suggestions by hitting the Return key or navigate to the
desired completion with the arrow keys. The panel can be dismissed by
hitting the Escape key and opened again by typing \texttt{\ \#\ } or
hitting { Ctrl } + { Space } . Use the autocomplete panel to discover
the right arguments for functions. Most suggestions come with a small
description of what they do.

\pandocbounded{\includegraphics[keepaspectratio]{/assets/docs/2-formatting-autocomplete.png}}

\subsection{Set up the page}\label{page-setup}

Back to set rules: When writing a rule, you choose the function
depending on what type of element you want to style. Here is a list of
some functions that are commonly used in set rules:

\begin{itemize}
\tightlist
\item
  \href{/docs/reference/text/text/}{\texttt{\ text\ }} to set font
  family, size, color, and other properties of text
\item
  \href{/docs/reference/layout/page/}{\texttt{\ page\ }} to set the page
  size, margins, headers, enable columns, and footers
\item
  \href{/docs/reference/model/par/}{\texttt{\ par\ }} to justify
  paragraphs, set line spacing, and more
\item
  \href{/docs/reference/model/heading/}{\texttt{\ heading\ }} to set the
  appearance of headings and enable numbering
\item
  \href{/docs/reference/model/document/}{\texttt{\ document\ }} to set
  the metadata contained in the PDF output, such as title and author
\end{itemize}

Not all function parameters can be set. In general, only parameters that
tell a function \emph{how} to do something can be set, not those that
tell it \emph{what} to do it with. The function reference pages indicate
which parameters are settable.

Let\textquotesingle s add a few more styles to our document. We want
larger margins and a serif font. For the purposes of the example,
we\textquotesingle ll also set another page size.

\begin{verbatim}
#set page(
  paper: "a6",
  margin: (x: 1.8cm, y: 1.5cm),
)
#set text(
  font: "New Computer Modern",
  size: 10pt
)
#set par(
  justify: true,
  leading: 0.52em,
)

= Introduction
In this report, we will explore the
various factors that influence fluid
dynamics in glaciers and how they
contribute to the formation and
behaviour of these natural structures.

...

#align(center + bottom)[
  #image("glacier.jpg", width: 70%)

  *Glaciers form an important
  part of the earth's climate
  system.*
]
\end{verbatim}

\includegraphics[width=6.19792in,height=\textheight,keepaspectratio]{/assets/docs/vXvjGwfGgpk5eo7U4CVWMQAAAAAAAAAA.png}

There are a few things of note here.

First is the \href{/docs/reference/layout/page/}{\texttt{\ page\ }} set
rule. It receives two arguments: the page size and margins for the page.
The page size is a string. Typst accepts
\href{/docs/reference/layout/page/\#parameters-paper}{many standard page
sizes,} but you can also specify a custom page size. The margins are
specified as a
\href{/docs/reference/foundations/dictionary/}{dictionary.} Dictionaries
are a collection of key-value pairs. In this case, the keys are
\texttt{\ x\ } and \texttt{\ y\ } , and the values are the horizontal
and vertical margins, respectively. We could also have specified
separate margins for each side by passing a dictionary with the keys
\texttt{\ left\ } , \texttt{\ right\ } , \texttt{\ top\ } , and
\texttt{\ bottom\ } .

Next is the set \href{/docs/reference/text/text/}{\texttt{\ text\ }} set
rule. Here, we set the font size to
\texttt{\ }{\texttt{\ 10pt\ }}\texttt{\ } and font family to
\texttt{\ }{\texttt{\ "New\ Computer\ Modern"\ }}\texttt{\ } . The Typst
app comes with many fonts that you can try for your document. When you
are in the text function\textquotesingle s argument list, you can
discover the available fonts in the autocomplete panel.

We have also set the spacing between lines (a.k.a. leading): It is
specified as a \href{/docs/reference/layout/length/}{length} value, and
we used the \texttt{\ em\ } unit to specify the leading relative to the
size of the font: \texttt{\ }{\texttt{\ 1em\ }}\texttt{\ } is equivalent
to the current font size (which defaults to
\texttt{\ }{\texttt{\ 11pt\ }}\texttt{\ } ).

Finally, we have bottom aligned our image by adding a vertical alignment
to our center alignment. Vertical and horizontal alignments can be
combined with the \texttt{\ }{\texttt{\ +\ }}\texttt{\ } operator to
yield a 2D alignment.

\subsection{A hint of sophistication}\label{sophistication}

To structure our document more clearly, we now want to number our
headings. We can do this by setting the \texttt{\ numbering\ } parameter
of the \href{/docs/reference/model/heading/}{\texttt{\ heading\ }}
function.

\begin{verbatim}
#set heading(numbering: "1.")

= Introduction
#lorem(10)

== Background
#lorem(12)

== Methods
#lorem(15)
\end{verbatim}

\includegraphics[width=5in,height=\textheight,keepaspectratio]{/assets/docs/4WtF0u81AczurIYrwpRdcwAAAAAAAAAA.png}

We specified the string \texttt{\ }{\texttt{\ "1."\ }}\texttt{\ } as the
numbering parameter. This tells Typst to number the headings with arabic
numerals and to put a dot between the number of each level. We can also
use \href{/docs/reference/model/numbering/}{letters, roman numerals, and
symbols} for our headings:

\begin{verbatim}
#set heading(numbering: "1.a")

= Introduction
#lorem(10)

== Background
#lorem(12)

== Methods
#lorem(15)
\end{verbatim}

\includegraphics[width=5in,height=\textheight,keepaspectratio]{/assets/docs/Llv0DrZ6U-QKf1vju2wP4QAAAAAAAAAA.png}

This example also uses the
\href{/docs/reference/text/lorem/}{\texttt{\ lorem\ }} function to
generate some placeholder text. This function takes a number as an
argument and generates that many words of \emph{Lorem Ipsum} text.

Did you wonder why the headings and text set rules apply to all text and
headings, even if they are not produced with the respective functions?

Typst internally calls the \texttt{\ heading\ } function every time you
write \texttt{\ }{\texttt{\ =\ Conclusion\ }}\texttt{\ } . In fact, the
function call
\texttt{\ }{\texttt{\ \#\ }}\texttt{\ }{\texttt{\ heading\ }}\texttt{\ }{\texttt{\ {[}\ }}\texttt{\ Conclusion\ }{\texttt{\ {]}\ }}\texttt{\ }
is equivalent to the heading markup above. Other markup elements work
similarly, they are only \emph{syntax sugar} for the corresponding
function calls.

\subsection{Show rules}\label{show-rules}

You are already pretty happy with how this turned out. But one last
thing needs to be fixed: The report you are writing is intended for a
larger project and that project\textquotesingle s name should always be
accompanied by a logo, even in prose.

You consider your options. You could add an
\texttt{\ }{\texttt{\ \#\ }}\texttt{\ }{\texttt{\ image\ }}\texttt{\ }{\texttt{\ (\ }}\texttt{\ }{\texttt{\ "logo.svg"\ }}\texttt{\ }{\texttt{\ )\ }}\texttt{\ }
call before every instance of the logo using search and replace. That
sounds very tedious. Instead, you could maybe
\href{/docs/reference/foundations/function/\#defining-functions}{define
a custom function} that always yields the logo with its image. However,
there is an even easier way:

With show rules, you can redefine how Typst displays certain elements.
You specify which elements Typst should show differently and how they
should look. Show rules can be applied to instances of text, many
functions, and even the whole document.

\begin{verbatim}
#show "ArtosFlow": name => box[
  #box(image(
    "logo.svg",
    height: 0.7em,
  ))
  #name
]

This report is embedded in the
ArtosFlow project. ArtosFlow is a
project of the Artos Institute.
\end{verbatim}

\includegraphics[width=5in,height=\textheight,keepaspectratio]{/assets/docs/349_Itx4-rTeNxzJmNofvgAAAAAAAAAA.png}

There is a lot of new syntax in this example: We write the
\texttt{\ }{\texttt{\ show\ }}\texttt{\ } keyword, followed by a string
of text we want to show differently and a colon. Then, we write a
function that takes the content that shall be shown as an argument.
Here, we called that argument \texttt{\ name\ } . We can now use the
\texttt{\ name\ } variable in the function\textquotesingle s body to
print the ArtosFlow name. Our show rule adds the logo image in front of
the name and puts the result into a box to prevent linebreaks from
occurring between logo and name. The image is also put inside of a box,
so that it does not appear in its own paragraph.

The calls to the first box function and the image function did not
require a leading \texttt{\ \#\ } because they were not embedded
directly in markup. When Typst expects code instead of markup, the
leading \texttt{\ \#\ } is not needed to access functions, keywords, and
variables. This can be observed in parameter lists, function
definitions, and \href{/docs/reference/scripting/}{code blocks} .

\subsection{Review}\label{review}

You now know how to apply basic formatting to your Typst documents. You
learned how to set the font, justify your paragraphs, change the page
dimensions, and add numbering to your headings with set rules. You also
learned how to use a basic show rule to change how text appears
throughout your document.

You have handed in your report. Your supervisor was so happy with it
that they want to adapt it into a conference paper! In the next section,
we will learn how to format your document as a paper using more advanced
show rules and functions.

\href{/docs/tutorial/writing-in-typst/}{\pandocbounded{\includesvg[keepaspectratio]{/assets/icons/16-arrow-right.svg}}}

{ Writing in Typst } { Previous page }

\href{/docs/tutorial/advanced-styling/}{\pandocbounded{\includesvg[keepaspectratio]{/assets/icons/16-arrow-right.svg}}}

{ Advanced Styling } { Next page }

\textbf{On this page}

\begin{itemize}
\tightlist
\item
  \hyperref[set-rules]{Set Rules}
\item
  \hyperref[autocomplete]{Autocomplete}
\item
  \hyperref[page-setup]{Page Setup}
\item
  \hyperref[sophistication]{Sophistication}
\item
  \hyperref[show-rules]{Show Rules}
\item
  \hyperref[review]{Review}
\end{itemize}

\begin{itemize}
\tightlist
\item
  \href{/}{Home}
\item
  \href{/pricing/}{Pricing}
\item
  \href{/docs/}{Documentation}
\item
  \href{/universe/}{Universe}
\item
  \href{/about/}{About Us}
\item
  \href{/contact/}{Contact Us}
\item
  \href{/privacy/}{Privacy}
\item
  \href{https://typst.app/terms}{Terms and Conditions}
\item
  \href{/legal/}{Legal (Impressum)}
\end{itemize}

\begin{itemize}
\tightlist
\item
  \href{https://forum.typst.app}{Forum}
\item
  \href{/tools/}{Tools}
\item
  \href{/blog/}{Blog}
\item
  \href{https://github.com/typst/}{GitHub}
\item
  \href{https://discord.gg/2uDybryKPe}{Discord}
\item
  \href{https://mastodon.social/@typst}{Mastodon}
\item
  \href{https://bsky.app/profile/typst.app}{Bluesky}
\item
  \href{https://www.linkedin.com/company/typst/}{LinkedIn}
\item
  \href{https://instagram.com/typstapp/}{Instagram}
\end{itemize}

Made in Berlin
