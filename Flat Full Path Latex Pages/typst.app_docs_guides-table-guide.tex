\title{typst.app/docs/guides/table-guide}

\begin{itemize}
\tightlist
\item
  \href{/docs}{\includesvg[width=0.16667in,height=0.16667in]{/assets/icons/16-docs-dark.svg}}
\item
  \includesvg[width=0.16667in,height=0.16667in]{/assets/icons/16-arrow-right.svg}
\item
  \href{/docs/guides/}{Guides}
\item
  \includesvg[width=0.16667in,height=0.16667in]{/assets/icons/16-arrow-right.svg}
\item
  \href{/docs/guides/table-guide/}{Table guide}
\end{itemize}

\section{Table guide}\label{table-guide}

Tables are a great way to present data to your readers in an easily
readable, compact, and organized manner. They are not only used for
numerical values, but also survey responses, task planning, schedules,
and more. Because of this wide set of possible applications, there is no
single best way to lay out a table. Instead, think about the data you
want to highlight, your document\textquotesingle s overarching design,
and ultimately how your table can best serve your readers.

Typst can help you with your tables by automating styling, importing
data from other applications, and more! This guide takes you through a
few of the most common questions you may have when adding a table to
your document with Typst. Feel free to skip to the section most relevant
to you â€`` we designed this guide to be read out of order.

If you want to look up a detail of how tables work, you should also
\href{/docs/reference/model/table/}{check out their reference page} .
And if you are looking for a table of contents rather than a normal
table, the reference page of the
\href{/docs/reference/model/outline/}{\texttt{\ outline\ } function} is
the right place to learn more.

\subsection{How to create a basic table?}\label{basic-tables}

In order to create a table in Typst, use the
\href{/docs/reference/model/table/}{\texttt{\ table\ } function} . For a
basic table, you need to tell the table function two things:

\begin{itemize}
\tightlist
\item
  The number of columns
\item
  The content for each of the table cells
\end{itemize}

So, let\textquotesingle s say you want to create a table with two
columns describing the ingredients for a cookie recipe:

\begin{verbatim}
#table(
  columns: 2,
  [*Amount*], [*Ingredient*],
  [360g], [Baking flour],
  [250g], [Butter (room temp.)],
  [150g], [Brown sugar],
  [100g], [Cane sugar],
  [100g], [70% cocoa chocolate],
  [100g], [35-40% cocoa chocolate],
  [2], [Eggs],
  [Pinch], [Salt],
  [Drizzle], [Vanilla extract],
)
\end{verbatim}

\includegraphics[width=5in,height=\textheight,keepaspectratio]{/assets/docs/DrLInmCn8ozR2FQ7a7txfwAAAAAAAAAA.png}

This example shows how to call, configure, and populate a table. Both
the column count and cell contents are passed to the table as arguments.
The \href{/docs/reference/foundations/function/}{argument list} is
surrounded by round parentheses. In it, we first pass the column count
as a named argument. Then, we pass multiple
\href{/docs/reference/foundations/content/}{content blocks} as
positional arguments. Each content block contains the contents for a
single cell.

To make the example more legible, we have placed two content block
arguments on each line, mimicking how they would appear in the table.
You could also write each cell on its own line. Typst does not care on
which line you place the arguments. Instead, Typst will place the
content cells from left to right (or right to left, if that is the
writing direction of your language) and then from top to bottom. It will
automatically add enough rows to your table so that it fits all of your
content.

It is best to wrap the header row of your table in the
\href{/docs/reference/model/table/\#definitions-header}{\texttt{\ table.header\ }
function} . This clarifies your intent and will also allow future
versions of Typst to make the output more accessible to users with a
screen reader:

\begin{verbatim}
#table(
  columns: 2,
  table.header[*Amount*][*Ingredient*],
  [360g], [Baking flour],
 // ... the remaining cells
)
\end{verbatim}

\includegraphics[width=5in,height=\textheight,keepaspectratio]{/assets/docs/aiVJ7GWTsso2QP8oAOCqrgAAAAAAAAAA.png}

You could also write a show rule that automatically
\href{/docs/reference/model/strong/}{strongly emphasizes} the contents
of the first cells for all tables. This quickly becomes useful if your
document contains multiple tables!

\begin{verbatim}
#show table.cell.where(y: 0): strong

#table(
  columns: 2,
  table.header[Amount][Ingredient],
  [360g], [Baking flour],
 // ... the remaining cells
)
\end{verbatim}

\includegraphics[width=5in,height=\textheight,keepaspectratio]{/assets/docs/U7CxDwHdU4boWRgp53c3XgAAAAAAAAAA.png}

We are using a show rule with a selector for cell coordinates here
instead of applying our styles directly to \texttt{\ table.header\ } .
This is due to a current limitation of Typst that will be fixed in a
future release.

Congratulations, you have created your first table! Now you can proceed
to \hyperref[column-sizes]{change column sizes} ,
\hyperref[strokes]{adjust the strokes} , \hyperref[fills]{add striped
rows} , and more!

\subsection{How to change the column sizes?}\label{column-sizes}

If you create a table and specify the number of columns, Typst will make
each column large enough to fit its largest cell. Often, you want
something different, for example, to make a table span the whole width
of the page. You can provide a list, specifying how wide you want each
column to be, through the \texttt{\ columns\ } argument. There are a few
different ways to specify column widths:

\begin{itemize}
\tightlist
\item
  First, there is \texttt{\ }{\texttt{\ auto\ }}\texttt{\ } . This is
  the default behavior and tells Typst to grow the column to fit its
  contents. If there is not enough space, Typst will try its best to
  distribute the space among the
  \texttt{\ }{\texttt{\ auto\ }}\texttt{\ } -sized columns.
\item
  \href{/docs/reference/layout/length/}{Lengths} like
  \texttt{\ }{\texttt{\ 6cm\ }}\texttt{\ } ,
  \texttt{\ }{\texttt{\ 0.7in\ }}\texttt{\ } , or
  \texttt{\ }{\texttt{\ 120pt\ }}\texttt{\ } . As usual, you can also
  use the font-dependent \texttt{\ em\ } unit. This is a multiple of
  your current font size. It\textquotesingle s useful if you want to
  size your table so that it always fits about the same amount of text,
  independent of font size.
\item
  A \href{/docs/reference/layout/ratio/}{ratio in percent} such as
  \texttt{\ }{\texttt{\ 40\%\ }}\texttt{\ } . This will make the column
  take up 40\% of the total horizontal space available to the table, so
  either the inner width of the page or the table\textquotesingle s
  container. You can also mix ratios and lengths into
  \href{/docs/reference/layout/relative/}{relative lengths} . Be mindful
  that even if you specify a list of column widths that sum up to 100\%,
  your table could still become larger than its container. This is
  because there can be
  \href{/docs/reference/model/table/\#parameters-gutter}{gutter} between
  columns that is not included in the column widths. If you want to make
  a table fill the page, the next option is often very useful.
\item
  A \href{/docs/reference/layout/fraction/}{fractional part of the free
  space} using the \texttt{\ fr\ } unit, such as \texttt{\ 1fr\ } . This
  unit allows you to distribute the available space to columns. It works
  as follows: First, Typst sums up the lengths of all columns that do
  not use \texttt{\ fr\ } s. Then, it determines how much horizontal
  space is left. This horizontal space then gets distributed to all
  columns denominated in \texttt{\ fr\ } s. During this process, a
  \texttt{\ 2fr\ } column will become twice as wide as a
  \texttt{\ 1fr\ } column. This is where the name comes from: The width
  of the column is its fraction of the total fractionally sized columns.
\end{itemize}

Let\textquotesingle s put this to use with a table that contains the
dates, numbers, and descriptions of some routine checks. The first two
columns are \texttt{\ auto\ } -sized and the last column is
\texttt{\ 1fr\ } wide as to fill the whole page.

\begin{verbatim}
#table(
  columns: (auto, auto, 1fr),
  table.header[Date][°No][Description],
  [24/01/03], [813], [Filtered participant pool],
  [24/01/03], [477], [Transitioned to sec. regimen],
  [24/01/11], [051], [Cycled treatment substrate],
)
\end{verbatim}

\includegraphics[width=5in,height=\textheight,keepaspectratio]{/assets/docs/2U_qe89XvypLADel8g81vgAAAAAAAAAA.png}

Here, we have passed our list of column lengths as an
\href{/docs/reference/foundations/array/}{array} , enclosed in round
parentheses, with its elements separated by commas. The first two
columns are automatically sized, so that they take on the size of their
content and the third column is sized as
\texttt{\ }{\texttt{\ 1fr\ }}\texttt{\ } so that it fills up the
remainder of the space on the page. If you wanted to instead change the
second column to be a bit more spacious, you could replace its entry in
the \texttt{\ columns\ } array with a value like
\texttt{\ }{\texttt{\ 6em\ }}\texttt{\ } .

\subsection{How to caption and reference my
table?}\label{captions-and-references}

A table is just as valuable as the information your readers draw from
it. You can enhance the effectiveness of both your prose and your table
by making a clear connection between the two with a cross-reference.
Typst can help you with automatic
\href{/docs/reference/model/ref/}{references} and the
\href{/docs/reference/model/figure/}{\texttt{\ figure\ } function} .

Just like with images, wrapping a table in the \texttt{\ figure\ }
function allows you to add a caption and a label, so you can reference
the figure elsewhere. Wrapping your table in a figure also lets you use
the figure\textquotesingle s \texttt{\ placement\ } parameter to float
it to the top or bottom of a page.

Let\textquotesingle s take a look at a captioned table and how to
reference it in prose:

\begin{verbatim}
#show table.cell.where(y: 0): set text(weight: "bold")

#figure(
  table(
    columns: 4,
    stroke: none,

    table.header[Test Item][Specification][Test Result][Compliance],
    [Voltage], [220V ± 5%], [218V], [Pass],
    [Current], [5A ± 0.5A], [4.2A], [Fail],
  ),
  caption: [Probe results for design A],
) <probe-a>

The results from @probe-a show that the design is not yet optimal.
We will show how its performance can be improved in this section.
\end{verbatim}

\includegraphics[width=8.27083in,height=\textheight,keepaspectratio]{/assets/docs/q8w9c3xaiqkQD9Ab1PAe6AAAAAAAAAAA.png}

The example shows how to wrap a table in a figure, set a caption and a
label, and how to reference that label. We start by using the
\texttt{\ figure\ } function. It expects the contents of the figure as a
positional argument. We just put the table function call in its argument
list, omitting the \texttt{\ \#\ } character because it is only needed
when calling a function in markup mode. We also add the caption as a
named argument (above or below) the table.

After the figure call, we put a label in angle brackets (
\texttt{\ }{\texttt{\ \textless{}probe-a\textgreater{}\ }}\texttt{\ } ).
This tells Typst to remember this element and make it referenceable
under this name throughout your document. We can then reference it in
prose by using the at sign and the label name
\texttt{\ }{\texttt{\ @probe-a\ }}\texttt{\ } . Typst will print a
nicely formatted reference and automatically update the label if the
table\textquotesingle s number changes.

\subsection{How to get a striped table?}\label{fills}

Many tables use striped rows or columns instead of strokes to
differentiate between rows and columns. This effect is often called
\emph{zebra stripes.} Tables with zebra stripes are popular in Business
and commercial Data Analytics applications, while academic applications
tend to use strokes instead.

To add zebra stripes to a table, we use the \texttt{\ table\ }
function\textquotesingle s \texttt{\ fill\ } argument. It can take three
kinds of arguments:

\begin{itemize}
\tightlist
\item
  A single color (this can also be a gradient or a pattern) to fill all
  cells with. Because we want some cells to have another color, this is
  not useful if we want to build zebra tables.
\item
  An array with colors which Typst cycles through for each column. We
  can use an array with two elements to get striped columns.
\item
  A function that takes the horizontal coordinate \texttt{\ x\ } and the
  vertical coordinate \texttt{\ y\ } of a cell and returns its fill. We
  can use this to create horizontal stripes or
  \href{/docs/reference/layout/grid/\#definitions-cell}{checkerboard
  patterns} .
\end{itemize}

Let\textquotesingle s start with an example of a horizontally striped
table:

\begin{verbatim}
#set text(font: "IBM Plex Sans")

// Medium bold table header.
#show table.cell.where(y: 0): set text(weight: "medium")

// Bold titles.
#show table.cell.where(x: 1): set text(weight: "bold")

// See the strokes section for details on this!
#let frame(stroke) = (x, y) => (
  left: if x > 0 { 0pt } else { stroke },
  right: stroke,
  top: if y < 2 { stroke } else { 0pt },
  bottom: stroke,
)

#set table(
  fill: (rgb("EAF2F5"), none),
  stroke: frame(rgb("21222C")),
)

#table(
  columns: (0.4fr, 1fr, 1fr, 1fr),

  table.header[Month][Title][Author][Genre],
  [January], [The Great Gatsby], [F. Scott Fitzgerald], [Classic],
  [February], [To Kill a Mockingbird], [Harper Lee], [Drama],
  [March], [1984], [George Orwell], [Dystopian],
  [April], [The Catcher in the Rye], [J.D. Salinger], [Coming-of-Age],
)
\end{verbatim}

\includegraphics[width=9.44792in,height=\textheight,keepaspectratio]{/assets/docs/UJoXkTjV0r6grf2p2oGJMAAAAAAAAAAA.png}

This example shows a book club reading list. The line
\texttt{\ fill:\ }{\texttt{\ (\ }}\texttt{\ }{\texttt{\ rgb\ }}\texttt{\ }{\texttt{\ (\ }}\texttt{\ }{\texttt{\ "EAF2F5"\ }}\texttt{\ }{\texttt{\ )\ }}\texttt{\ }{\texttt{\ ,\ }}\texttt{\ }{\texttt{\ none\ }}\texttt{\ }{\texttt{\ )\ }}\texttt{\ }
in \texttt{\ table\ } \textquotesingle s set rule is all that is needed
to add striped columns. It tells Typst to alternate between coloring
columns with a light blue (in the
\href{/docs/reference/visualize/color/\#definitions-rgb}{\texttt{\ rgb\ }}
function call) and nothing ( \texttt{\ }{\texttt{\ none\ }}\texttt{\ }
). Note that we extracted all of our styling from the \texttt{\ table\ }
function call itself into set and show rules, so that we can
automatically reuse it for multiple tables.

Because setting the stripes itself is easy we also added some other
styles to make it look nice. The other code in the example provides a
dark blue \hyperref[stroke-functions]{stroke} around the table and below
the first line and emboldens the first row and the column with the book
title. See the \hyperref[strokes]{strokes} section for details on how we
achieved this stroke configuration.

Let\textquotesingle s next take a look at how we can change only the set
rule to achieve horizontal stripes instead:

\begin{verbatim}
#set table(
  fill: (_, y) => if calc.odd(y) { rgb("EAF2F5") },
  stroke: frame(rgb("21222C")),
)
\end{verbatim}

\includegraphics[width=9.44792in,height=\textheight,keepaspectratio]{/assets/docs/25lCWRlQDgNwC6P_ss5rsgAAAAAAAAAA.png}

We just need to replace the set rule from the previous example with this
one and get horizontal stripes instead. Here, we are passing a function
to \texttt{\ fill\ } . It discards the horizontal coordinate with an
underscore and then checks if the vertical coordinate \texttt{\ y\ } of
the cell is odd. If so, the cell gets a light blue fill, otherwise, no
fill is returned.

Of course, you can make this function arbitrarily complex. For example,
if you want to stripe the rows with a light and darker shade of blue,
you could do something like this:

\begin{verbatim}
#set table(
  fill: (_, y) => (none, rgb("EAF2F5"), rgb("DDEAEF")).at(calc.rem(y, 3)),
  stroke: frame(rgb("21222C")),
)
\end{verbatim}

\includegraphics[width=9.44792in,height=\textheight,keepaspectratio]{/assets/docs/cCvrBfZ8ZZjy8abtCE_cmgAAAAAAAAAA.png}

This example shows an alternative approach to write our fill function.
The function uses an array with three colors and then cycles between its
values for each row by indexing the array with the remainder of
\texttt{\ y\ } divided by 3.

Finally, here is a bonus example that uses the \emph{stroke} to achieve
striped rows:

\begin{verbatim}
#set table(
  stroke: (x, y) => (
    y: 1pt,
    left: if x > 0 { 0pt } else if calc.even(y) { 1pt },
    right: if calc.even(y) { 1pt },
  ),
)
\end{verbatim}

\includegraphics[width=9.44792in,height=\textheight,keepaspectratio]{/assets/docs/GD_bV9_znidgeZ0v5zIXUAAAAAAAAAAA.png}

\subsubsection{Manually overriding a cell\textquotesingle s fill
color}\label{fill-override}

Sometimes, the fill of a cell needs not to vary based on its position in
the table, but rather based on its contents. We can use the
\href{/docs/reference/model/table/\#definitions-cell}{\texttt{\ table.cell\ }
element} in the \texttt{\ table\ } \textquotesingle s parameter list to
wrap a cell\textquotesingle s content and override its fill.

For example, here is a list of all German presidents, with the cell
borders colored in the color of their party.

\begin{verbatim}
#set text(font: "Roboto")

#let cdu(name) = ([CDU], table.cell(fill: black, text(fill: white, name)))
#let spd(name) = ([SPD], table.cell(fill: red, text(fill: white, name)))
#let fdp(name) = ([FDP], table.cell(fill: yellow, name))

#table(
  columns: (auto, auto, 1fr),
  stroke: (x: none),

  table.header[Tenure][Party][President],
  [1949-1959], ..fdp[Theodor Heuss],
  [1959-1969], ..cdu[Heinrich Lübke],
  [1969-1974], ..spd[Gustav Heinemann],
  [1974-1979], ..fdp[Walter Scheel],
  [1979-1984], ..cdu[Karl Carstens],
  [1984-1994], ..cdu[Richard von Weizsäcker],
  [1994-1999], ..cdu[Roman Herzog],
  [1999-2004], ..spd[Johannes Rau],
  [2004-2010], ..cdu[Horst Köhler],
  [2010-2012], ..cdu[Christian Wulff],
  [2012-2017], [n/a], [Joachim Gauck],
  [2017-],     ..spd[Frank-Walter-Steinmeier],
)
\end{verbatim}

\includegraphics[width=5.90625in,height=\textheight,keepaspectratio]{/assets/docs/Kc5oSTV9kIbzwSd275OEwQAAAAAAAAAA.png}

In this example, we make use of variables because there only have been a
total of three parties whose members have become president (and one
unaffiliated president). Their colors will repeat multiple times, so we
store a function that produces an array with their
party\textquotesingle s name and a table cell with that
party\textquotesingle s color and the president\textquotesingle s name (
\texttt{\ cdu\ } , \texttt{\ spd\ } , and \texttt{\ fdp\ } ). We then
use these functions in the \texttt{\ table\ } argument list instead of
directly adding the name. We use the
\href{/docs/reference/foundations/arguments/\#spreading}{spread
operator} \texttt{\ ..\ } to turn the items of the arrays into single
cells. We could also write something like
\texttt{\ }{\texttt{\ {[}\ }}\texttt{\ FDP\ }{\texttt{\ {]}\ }}\texttt{\ ,\ table\ }{\texttt{\ .\ }}\texttt{\ }{\texttt{\ cell\ }}\texttt{\ }{\texttt{\ (\ }}\texttt{\ fill\ }{\texttt{\ :\ }}\texttt{\ yellow\ }{\texttt{\ )\ }}\texttt{\ }{\texttt{\ {[}\ }}\texttt{\ Theodor\ Heuss\ }{\texttt{\ {]}\ }}\texttt{\ }
for each cell directly in the \texttt{\ table\ } \textquotesingle s
argument list, but that becomes unreadable, especially for the parties
whose colors are dark so that they require white text. We also delete
vertical strokes and set the font to Roboto.

The party column and the cell color in this example communicate
redundant information on purpose: Communicating important data using
color only is a bad accessibility practice. It disadvantages users with
vision impairment and is in violation of universal access standards,
such as the
\href{https://www.w3.org/WAI/WCAG21/Understanding/use-of-color.html}{WCAG
2.1 Success Criterion 1.4.1} . To improve this table, we added a column
printing the party name. Alternatively, you could have made sure to
choose a color-blindness friendly palette and mark up your cells with an
additional label that screen readers can read out loud. The latter
feature is not currently supported by Typst, but will be added in a
future release. You can check how colors look for color-blind readers
with
\href{https://chromewebstore.google.com/detail/colorblindly/floniaahmccleoclneebhhmnjgdfijgg}{this
Chrome extension} ,
\href{https://helpx.adobe.com/photoshop/using/proofing-colors.html}{Photoshop}
, or
\href{https://docs.gimp.org/2.10/en/gimp-display-filter-dialog.html}{GIMP}
.

\subsection{How to adjust the lines in a table?}\label{strokes}

By default, Typst adds strokes between each row and column of a table.
You can adjust these strokes in a variety of ways. Which one is the most
practical, depends on the modification you want to make and your intent:

\begin{itemize}
\tightlist
\item
  Do you want to style all tables in your document, irrespective of
  their size and content? Use the \texttt{\ table\ }
  function\textquotesingle s
  \href{/docs/reference/model/table/\#parameters-stroke}{stroke}
  argument in a set rule.
\item
  Do you want to customize all lines in a single table? Use the
  \texttt{\ table\ } function\textquotesingle s
  \href{/docs/reference/model/table/\#parameters-stroke}{stroke}
  argument when calling the table function.
\item
  Do you want to change, add, or remove the stroke around a single cell?
  Use the \texttt{\ table.cell\ } element in the argument list of your
  table call.
\item
  Do you want to change, add, or remove a single horizontal or vertical
  stroke in a single table? Use the
  \href{/docs/reference/model/table/\#definitions-hline}{\texttt{\ table.hline\ }}
  and
  \href{/docs/reference/model/table/\#definitions-vline}{\texttt{\ table.vline\ }}
  elements in the argument list of your table call.
\end{itemize}

We will go over all of these options with examples next! First, we will
tackle the \texttt{\ table\ } function\textquotesingle s
\href{/docs/reference/model/table/\#parameters-stroke}{stroke} argument.
Here, you can adjust both how the table\textquotesingle s lines get
drawn and configure which lines are drawn at all.

Let\textquotesingle s start by modifying the color and thickness of the
stroke:

\begin{verbatim}
#table(
  columns: 4,
  stroke: 0.5pt + rgb("666675"),
  [*Monday*], [11.5], [13.0], [4.0],
  [*Tuesday*], [8.0], [14.5], [5.0],
  [*Wednesday*], [9.0], [18.5], [13.0],
)
\end{verbatim}

\includegraphics[width=5in,height=\textheight,keepaspectratio]{/assets/docs/y0uGf9aX-aUGWqCKMwHEJwAAAAAAAAAA.png}

This makes the table lines a bit less wide and uses a bluish gray. You
can see that we added a width in point to a color to achieve our
customized stroke. This addition yields a value of the
\href{/docs/reference/visualize/stroke/}{stroke type} . Alternatively,
you can use the dictionary representation for strokes which allows you
to access advanced features such as dashed lines.

The previous example showed how to use the stroke argument in the table
function\textquotesingle s invocation. Alternatively, you can specify
the stroke argument in the \texttt{\ table\ } \textquotesingle s set
rule. This will have exactly the same effect on all subsequent
\texttt{\ table\ } calls as if the stroke argument was specified in the
argument list. This is useful if you are writing a template or want to
style your whole document.

\begin{verbatim}
// Renders the exact same as the last example
#set table(stroke: 0.5pt + rgb("666675"))

#table(
  columns: 4,
  [*Monday*], [11.5], [13.0], [4.0],
  [*Tuesday*], [8.0], [14.5], [5.0],
  [*Wednesday*], [9.0], [18.5], [13.0],
)
\end{verbatim}

For small tables, you sometimes want to suppress all strokes because
they add too much visual noise. To do this, just set the stroke argument
to \texttt{\ }{\texttt{\ none\ }}\texttt{\ } :

\begin{verbatim}
#table(
  columns: 4,
  stroke: none,
  [*Monday*], [11.5], [13.0], [4.0],
  [*Tuesday*], [8.0], [14.5], [5.0],
  [*Wednesday*], [9.0], [18.5], [13.0],
)
\end{verbatim}

\includegraphics[width=5in,height=\textheight,keepaspectratio]{/assets/docs/vQmODupZkk6MbI3nByjkqwAAAAAAAAAA.png}

If you want more fine-grained control of where lines get placed in your
table, you can also pass a dictionary with the keys \texttt{\ top\ } ,
\texttt{\ left\ } , \texttt{\ right\ } , \texttt{\ bottom\ }
(controlling the respective cell sides), \texttt{\ x\ } , \texttt{\ y\ }
(controlling vertical and horizontal strokes), and \texttt{\ rest\ }
(covers all strokes not styled by other dictionary entries). All keys
are optional; omitted keys will be treated as if their value was the
default value. For example, to get a table with only horizontal lines,
you can do this:

\begin{verbatim}
#table(
  columns: 2,
  stroke: (x: none),
  align: horizon,
  [☒], [Close cabin door],
  [☐], [Start engines],
  [☐], [Radio tower],
  [☐], [Push back],
)
\end{verbatim}

\includegraphics[width=5in,height=\textheight,keepaspectratio]{/assets/docs/eb2IfSiEgTxcBIC79R6b-QAAAAAAAAAA.png}

This turns off all vertical strokes and leaves the horizontal strokes in
place. To achieve the reverse effect (only horizontal strokes), set the
stroke argument to
\texttt{\ }{\texttt{\ (\ }}\texttt{\ y\ }{\texttt{\ :\ }}\texttt{\ }{\texttt{\ none\ }}\texttt{\ }{\texttt{\ )\ }}\texttt{\ }
instead.

\hyperref[stroke-functions]{Further down in the guide} , we cover how to
use a function in the stroke argument to customize all strokes
individually. This is how you achieve more complex stroking patterns.

\subsubsection{Adding individual lines in the
table}\label{individual-lines}

If you want to add a single horizontal or vertical line in your table,
for example to separate a group of rows, you can use the
\href{/docs/reference/model/table/\#definitions-hline}{\texttt{\ table.hline\ }}
and
\href{/docs/reference/model/table/\#definitions-vline}{\texttt{\ table.vline\ }}
elements for horizontal and vertical lines, respectively. Add them to
the argument list of the \texttt{\ table\ } function just like you would
add individual cells and a header.

Let\textquotesingle s take a look at the following example from the
reference:

\begin{verbatim}
#set table.hline(stroke: 0.6pt)

#table(
  stroke: none,
  columns: (auto, 1fr),
  // Morning schedule abridged.
  [14:00], [Talk: Tracked Layout],
  [15:00], [Talk: Automations],
  [16:00], [Workshop: Tables],
  table.hline(),
  [19:00], [Day 1 Attendee Mixer],
)
\end{verbatim}

\includegraphics[width=5in,height=\textheight,keepaspectratio]{/assets/docs/3OGDUdhafmvnsbwhpEZnqAAAAAAAAAAA.png}

In this example, you can see that we have placed a call to
\texttt{\ table.hline\ } between the cells, producing a horizontal line
at that spot. We also used a set rule on the element to reduce its
stroke width to make it fit better with the weight of the font.

By default, Typst places horizontal and vertical lines after the current
row or column, depending on their position in the argument list. You can
also manually move them to a different position by adding the
\texttt{\ y\ } (for \texttt{\ hline\ } ) or \texttt{\ x\ } (for
\texttt{\ vline\ } ) argument. For example, the code below would produce
the same result:

\begin{verbatim}
#set table.hline(stroke: 0.6pt)

#table(
  stroke: none,
  columns: (auto, 1fr),
  // Morning schedule abridged.
  table.hline(y: 3),
  [14:00], [Talk: Tracked Layout],
  [15:00], [Talk: Automations],
  [16:00], [Workshop: Tables],
  [19:00], [Day 1 Attendee Mixer],
)
\end{verbatim}

Let\textquotesingle s imagine you are working with a template that shows
none of the table strokes except for one between the first and second
row. Now, since you have one table that also has labels in the first
column, you want to add an extra vertical line to it. However, you do
not want this vertical line to cross into the top row. You can achieve
this with the \texttt{\ start\ } argument:

\begin{verbatim}
// Base template already configured tables, but we need some
// extra configuration for this table.
#{
  set table(align: (x, _) => if x == 0 { left } else { right })
  show table.cell.where(x: 0): smallcaps
  table(
    columns: (auto, 1fr, 1fr, 1fr),
    table.vline(x: 1, start: 1),
    table.header[Trainset][Top Speed][Length][Weight],
    [TGV Réseau], [320 km/h], [200m], [383t],
    [ICE 403], [330 km/h], [201m], [409t],
    [Shinkansen N700], [300 km/h], [405m], [700t],
  )
}
\end{verbatim}

\includegraphics[width=7.08333in,height=\textheight,keepaspectratio]{/assets/docs/LINgvIDoMEPxmydMRnWXAgAAAAAAAAAA.png}

In this example, we have added \texttt{\ table.vline\ } at the start of
our positional argument list. But because the line is not supposed to go
to the left of the first column, we specified the \texttt{\ x\ }
argument as \texttt{\ }{\texttt{\ 1\ }}\texttt{\ } . We also set the
\texttt{\ start\ } argument to \texttt{\ }{\texttt{\ 1\ }}\texttt{\ } so
that the line does only start after the first row.

The example also contains two more things: We use the align argument
with a function to right-align the data in all but the first column and
use a show rule to make the first column of table cells appear in small
capitals. Because these styles are specific to this one table, we put
everything into a \href{/docs/reference/scripting/\#blocks}{code block}
, so that the styling does not affect any further tables.

\subsubsection{Overriding the strokes of a single
cell}\label{stroke-override}

Imagine you want to change the stroke around a single cell. Maybe your
cell is very important and needs highlighting! For this scenario, there
is the
\href{/docs/reference/model/table/\#definitions-cell}{\texttt{\ table.cell\ }
function} . Instead of adding your content directly in the argument list
of the table, you wrap it in a \texttt{\ table.cell\ } call. Now, you
can use \texttt{\ table.cell\ } \textquotesingle s argument list to
override the table properties, such as the stroke, for this cell only.

Here\textquotesingle s an example with a matrix of two of the Big Five
personality factors, with one intersection highlighted.

\begin{verbatim}
#table(
  columns: 3,
  stroke: (x: none),

  [], [*High Neuroticism*], [*Low Neuroticism*],

  [*High Agreeableness*],
  table.cell(stroke: orange + 2pt)[
    _Sensitive_ \ Prone to emotional distress but very empathetic.
  ],
  [_Compassionate_ \ Caring and stable, often seen as a supportive figure.],

  [*Low Agreeableness*],
  [_Contentious_ \ Competitive and easily agitated.],
  [_Detached_ \ Independent and calm, may appear aloof.],
)
\end{verbatim}

\includegraphics[width=9.44792in,height=\textheight,keepaspectratio]{/assets/docs/zT-rPhodrO99vfmjEknOKwAAAAAAAAAA.png}

Above, you can see that we used the \texttt{\ table.cell\ } element in
the table\textquotesingle s argument list and passed the cell content to
it. We have used its \texttt{\ stroke\ } argument to set a wider orange
stroke. Despite the fact that we disabled vertical strokes on the table,
the orange stroke appeared on all sides of the modified cell, showing
that the table\textquotesingle s stroke configuration is overwritten.

\subsubsection{Complex document-wide stroke
customization}\label{stroke-functions}

This section explains how to customize all lines at once in one or
multiple tables. This allows you to draw only the first horizontal line
or omit the outer lines, without knowing how many cells the table has.
This is achieved by providing a function to the table\textquotesingle s
\texttt{\ stroke\ } parameter. The function should return a stroke given
the zero-indexed x and y position of the current cell. You should only
need these functions if you are a template author, do not use a
template, or need to heavily customize your tables. Otherwise, your
template should set appropriate default table strokes.

For example, this is a set rule that draws all horizontal lines except
for the very first and last line.

\begin{verbatim}
#show table.cell.where(x: 0): set text(style: "italic")
#show table.cell.where(y: 0): set text(style: "normal", weight: "bold")
#set table(stroke: (_, y) => if y > 0 { (top: 0.8pt) })

#table(
  columns: 3,
  align: center + horizon,
  table.header[Technique][Advantage][Drawback],
  [Diegetic], [Immersive], [May be contrived],
  [Extradiegetic], [Breaks immersion], [Obtrusive],
  [Omitted], [Fosters engagement], [May fracture audience],
)
\end{verbatim}

\includegraphics[width=5in,height=\textheight,keepaspectratio]{/assets/docs/SKiPog79AfwUoglArT-17wAAAAAAAAAA.png}

In the set rule, we pass a function that receives two arguments,
assigning the vertical coordinate to \texttt{\ y\ } and discarding the
horizontal coordinate. It then returns a stroke dictionary with a
\texttt{\ }{\texttt{\ 0.8pt\ }}\texttt{\ } top stroke for all but the
first line. The cells in the first line instead implicitly receive
\texttt{\ }{\texttt{\ none\ }}\texttt{\ } as the return value. You can
easily modify this function to just draw the inner vertical lines
instead as
\texttt{\ }{\texttt{\ (\ }}\texttt{\ x\ }{\texttt{\ ,\ }}\texttt{\ \_\ }{\texttt{\ )\ }}\texttt{\ }{\texttt{\ =\textgreater{}\ }}\texttt{\ }{\texttt{\ if\ }}\texttt{\ x\ }{\texttt{\ \textgreater{}\ }}\texttt{\ }{\texttt{\ 0\ }}\texttt{\ }{\texttt{\ \{\ }}\texttt{\ }{\texttt{\ (\ }}\texttt{\ left\ }{\texttt{\ :\ }}\texttt{\ }{\texttt{\ 0.8pt\ }}\texttt{\ }{\texttt{\ )\ }}\texttt{\ }{\texttt{\ \}\ }}\texttt{\ }
.

Let\textquotesingle s try a few more stroking functions. The next
function will only draw a line below the first row:

\begin{verbatim}
#set table(stroke: (_, y) => if y == 0 { (bottom: 1pt) })

// Table as seen above
\end{verbatim}

\includegraphics[width=5in,height=\textheight,keepaspectratio]{/assets/docs/nyvsIK-tDvHuRwbnwhj4kQAAAAAAAAAA.png}

If you understood the first example, it becomes obvious what happens
here. We check if we are in the first row. If so, we return a bottom
stroke. Otherwise, we\textquotesingle ll return
\texttt{\ }{\texttt{\ none\ }}\texttt{\ } implicitly.

The next example shows how to draw all but the outer lines:

\begin{verbatim}
#set table(stroke: (x, y) => (
  left: if x > 0 { 0.8pt },
  top: if y > 0 { 0.8pt },
))

// Table as seen above
\end{verbatim}

\includegraphics[width=5in,height=\textheight,keepaspectratio]{/assets/docs/QRh4-eyPqD4Yz9VResk9dwAAAAAAAAAA.png}

This example uses both the \texttt{\ x\ } and \texttt{\ y\ }
coordinates. It omits the left stroke in the first column and the top
stroke in the first row. The right and bottom lines are not drawn.

Finally, here is a table that draws all lines except for the vertical
lines in the first row and horizontal lines in the table body. It looks
a bit like a calendar.

\begin{verbatim}
#set table(stroke: (x, y) => (
  left: if x == 0 or y > 0 { 1pt } else { 0pt },
  right: 1pt,
  top: if y <= 1 { 1pt } else { 0pt },
  bottom: 1pt,
))

// Table as seen above
\end{verbatim}

\includegraphics[width=5in,height=\textheight,keepaspectratio]{/assets/docs/qw8tKe_4tpVbtFAmFf0POQAAAAAAAAAA.png}

This example is a bit more complex. We start by drawing all the strokes
on the right of the cells. But this means that we have drawn strokes in
the top row, too, and we don\textquotesingle t need those! We use the
fact that \texttt{\ left\ } will override \texttt{\ right\ } and only
draw the left line if we are not in the first row or if we are in the
first column. In all other cases, we explicitly remove the left line.
Finally, we draw the horizontal lines by first setting the bottom line
and then for the first two rows with the \texttt{\ top\ } key,
suppressing all other top lines. The last line appears because there is
no \texttt{\ top\ } line that could suppress it.

\subsubsection{How to achieve a double line?}\label{double-stroke}

Typst does not yet have a native way to draw double strokes, but there
are multiple ways to emulate them, for example with
\href{/docs/reference/visualize/pattern/}{patterns} . We will show a
different workaround in this section: Table gutters.

Tables can space their cells apart using the \texttt{\ gutter\ }
argument. When a gutter is applied, a stroke is drawn on each of the now
separated cells. We can selectively add gutter between the rows or
columns for which we want to draw a double line. The
\texttt{\ row-gutter\ } and \texttt{\ column-gutter\ } arguments allow
us to do this. They accept arrays of gutter values.
Let\textquotesingle s take a look at an example:

\begin{verbatim}
#table(
  columns: 3,
  stroke: (x: none),
  row-gutter: (2.2pt, auto),
  table.header[Date][Exercise Type][Calories Burned],
  [2023-03-15], [Swimming], [400],
  [2023-03-17], [Weightlifting], [250],
  [2023-03-18], [Yoga], [200],
)
\end{verbatim}

\includegraphics[width=5in,height=\textheight,keepaspectratio]{/assets/docs/ketpWMF_8TQEKQlrbhiNRQAAAAAAAAAA.png}

We can see that we used an array for \texttt{\ row-gutter\ } that
specifies a \texttt{\ }{\texttt{\ 2.2pt\ }}\texttt{\ } gap between the
first and second row. It then continues with \texttt{\ auto\ } (which is
the default, in this case \texttt{\ }{\texttt{\ 0pt\ }}\texttt{\ }
gutter) which will be the gutter between all other rows, since it is the
last entry in the array.

\subsection{How to align the contents of the cells in my
table?}\label{alignment}

You can use multiple mechanisms to align the content in your table. You
can either use the \texttt{\ table\ } function\textquotesingle s
\texttt{\ align\ } argument to set the alignment for your whole table
(or use it in a set rule to set the alignment for tables throughout your
document) or the
\href{/docs/reference/layout/align/}{\texttt{\ align\ }} function (or
\texttt{\ table.cell\ } \textquotesingle s \texttt{\ align\ } argument)
to override the alignment of a single cell.

When using the \texttt{\ table\ } function\textquotesingle s align
argument, you can choose between three methods to specify an
\href{/docs/reference/layout/alignment/}{alignment} :

\begin{itemize}
\tightlist
\item
  Just specify a single alignment like \texttt{\ right\ } (aligns in the
  top-right corner) or \texttt{\ center\ +\ horizon\ } (centers all cell
  content). This changes the alignment of all cells.
\item
  Provide an array. Typst will cycle through this array for each column.
\item
  Provide a function that is passed the horizontal \texttt{\ x\ } and
  vertical \texttt{\ y\ } coordinate of a cell and returns an alignment.
\end{itemize}

For example, this travel itinerary right-aligns the day column and
left-aligns everything else by providing an array in the
\texttt{\ align\ } argument:

\begin{verbatim}
#set text(font: "IBM Plex Sans")
#show table.cell.where(y: 0): set text(weight: "bold")

#table(
  columns: 4,
  align: (right, left, left, left),
  fill: (_, y) => if calc.odd(y) { green.lighten(90%) },
  stroke: none,

  table.header[Day][Location][Hotel or Apartment][Activities],
  [1], [Paris, France], [Hotel de L'Europe], [Arrival, Evening River Cruise],
  [2], [Paris, France], [Hotel de L'Europe], [Louvre Museum, Eiffel Tower],
  [3], [Lyon, France], [Lyon City Hotel], [City Tour, Local Cuisine Tasting],
  [4], [Geneva, Switzerland], [Lakeview Inn], [Lake Geneva, Red Cross Museum],
  [5], [Zermatt, Switzerland], [Alpine Lodge], [Visit Matterhorn, Skiing],
)
\end{verbatim}

\includegraphics[width=7.08333in,height=\textheight,keepaspectratio]{/assets/docs/JDfAUmIJzQHzE6NL8LKAnwAAAAAAAAAA.png}

However, this example does not yet look perfect â€'' the header cells
should be bottom-aligned. Let\textquotesingle s use a function instead
to do so:

\begin{verbatim}
#set text(font: "IBM Plex Sans")
#show table.cell.where(y: 0): set text(weight: "bold")

#table(
  columns: 4,
  align: (x, y) =>
    if x == 0 { right } else { left } +
    if y == 0 { bottom } else { top },
  fill: (_, y) => if calc.odd(y) { green.lighten(90%) },
  stroke: none,

  table.header[Day][Location][Hotel or Apartment][Activities],
  [1], [Paris, France], [Hotel de L'Europe], [Arrival, Evening River Cruise],
  [2], [Paris, France], [Hotel de L'Europe], [Louvre Museum, Eiffel Tower],
 // ... remaining days omitted
)
\end{verbatim}

\includegraphics[width=7.08333in,height=\textheight,keepaspectratio]{/assets/docs/ENpsdZXkKtSK7dBT6e9dAgAAAAAAAAAA.png}

In the function, we calculate a horizontal and vertical alignment based
on whether we are in the first column (
\texttt{\ x\ }{\texttt{\ ==\ }}\texttt{\ }{\texttt{\ 0\ }}\texttt{\ } )
or the first row (
\texttt{\ y\ }{\texttt{\ ==\ }}\texttt{\ }{\texttt{\ 0\ }}\texttt{\ } ).
We then make use of the fact that we can add horizontal and vertical
alignments with \texttt{\ +\ } to receive a single, two-dimensional
alignment.

You can find an example of using \texttt{\ table.cell\ } to change a
single cell\textquotesingle s alignment on
\href{/docs/reference/model/table/\#definitions-cell}{its reference
page} .

\subsection{How to merge cells?}\label{merge-cells}

When a table contains logical groupings or the same data in multiple
adjacent cells, merging multiple cells into a single, larger cell can be
advantageous. Another use case for cell groups are table headers with
multiple rows: That way, you can group for example a sales data table by
quarter in the first row and by months in the second row.

A merged cell spans multiple rows and/or columns. You can achieve it
with the
\href{/docs/reference/model/table/\#definitions-cell}{\texttt{\ table.cell\ }}
function\textquotesingle s \texttt{\ rowspan\ } and \texttt{\ colspan\ }
arguments: Just specify how many rows or columns you want your cell to
span.

The example below contains an attendance calendar for an office with
in-person and remote days for each team member. To make the table more
glanceable, we merge adjacent cells with the same value:

\begin{verbatim}
#let ofi = [Office]
#let rem = [_Remote_]
#let lea = [*On leave*]

#show table.cell.where(y: 0): set text(
  fill: white,
  weight: "bold",
)

#table(
  columns: 6 * (1fr,),
  align: (x, y) => if x == 0 or y == 0 { left } else { center },
  stroke: (x, y) => (
    // Separate black cells with white strokes.
    left: if y == 0 and x > 0 { white } else { black },
    rest: black,
  ),
  fill: (_, y) => if y == 0 { black },

  table.header(
    [Team member],
    [Monday],
    [Tuesday],
    [Wednesday],
    [Thursday],
    [Friday]
  ),
  [Evelyn Archer],
    table.cell(colspan: 2, ofi),
    table.cell(colspan: 2, rem),
    ofi,
  [Lila Montgomery],
    table.cell(colspan: 5, lea),
  [Nolan Pearce],
    rem,
    table.cell(colspan: 2, ofi),
    rem,
    ofi,
)
\end{verbatim}

\includegraphics[width=12.98958in,height=\textheight,keepaspectratio]{/assets/docs/yyaKf-LLQbyy3QccVJXb2wAAAAAAAAAA.png}

In the example, we first define variables with "Office", "Remote", and
"On leave" so we don\textquotesingle t have to write these labels out
every time. We can then use these variables in the table body either
directly or in a \texttt{\ table.cell\ } call if the team member spends
multiple consecutive days in office, remote, or on leave.

The example also contains a black header (created with
\texttt{\ table\ } \textquotesingle s \texttt{\ fill\ } argument) with
white strokes ( \texttt{\ table\ } \textquotesingle s
\texttt{\ stroke\ } argument) and white text (set by the
\texttt{\ table.cell\ } set rule). Finally, we align all the content of
all table cells in the body in the center. If you want to know more
about the functions passed to \texttt{\ align\ } , \texttt{\ stroke\ } ,
and \texttt{\ fill\ } , you can check out the sections on
\href{/docs/reference/layout/alignment/}{alignment} ,
\hyperref[stroke-functions]{strokes} , and \hyperref[fills]{striped
tables} .

This table would be a great candidate for fully automated generation
from an external data source! Check out the
\hyperref[importing-data]{section about importing data} to learn more
about that.

\subsection{How to rotate a table?}\label{rotate-table}

When tables have many columns, a portrait paper orientation can quickly
get cramped. Hence, you\textquotesingle ll sometimes want to switch your
tables to landscape orientation. There are two ways to accomplish this
in Typst:

\begin{itemize}
\tightlist
\item
  If you want to rotate only the table but not the other content of the
  page and the page itself, use the
  \href{/docs/reference/layout/rotate/}{\texttt{\ rotate\ } function}
  with the \texttt{\ reflow\ } argument set to
  \texttt{\ }{\texttt{\ true\ }}\texttt{\ } .
\item
  If you want to rotate the whole page the table is on, you can use the
  \href{/docs/reference/layout/page/}{\texttt{\ page\ } function} with
  its \texttt{\ flipped\ } argument set to
  \texttt{\ }{\texttt{\ true\ }}\texttt{\ } . The header, footer, and
  page number will now also appear on the long edge of the page. This
  has the advantage that the table will appear right side up when read
  on a computer, but it also means that a page in your document has
  different dimensions than all the others, which can be jarring to your
  readers.
\end{itemize}

Below, we will demonstrate both techniques with a student grade book
table.

First, we will rotate the table on the page. The example also places
some text on the right of the table.

\begin{verbatim}
#set page("a5", columns: 2, numbering: "— 1 —")
#show table.cell.where(y: 0): set text(weight: "bold")

#rotate(
  -90deg,
  reflow: true,

  table(
    columns: (1fr,) + 5 * (auto,),
    inset: (x: 0.6em,),
    stroke: (_, y) => (
      x: 1pt,
      top: if y <= 1 { 1pt } else { 0pt },
      bottom: 1pt,
    ),
    align: (left, right, right, right, right, left),

    table.header(
      [Student Name],
      [Assignment 1], [Assignment 2],
      [Mid-term], [Final Exam],
      [Total Grade],
    ),
    [Jane Smith], [78%], [82%], [75%], [80%], [B],
    [Alex Johnson], [90%], [95%], [94%], [96%], [A+],
    [John Doe], [85%], [90%], [88%], [92%], [A],
    [Maria Garcia], [88%], [84%], [89%], [85%], [B+],
    [Zhang Wei], [93%], [89%], [90%], [91%], [A-],
    [Marina Musterfrau], [96%], [91%], [74%], [69%], [B-],
  ),
)

#lorem(80)
\end{verbatim}

\includegraphics[width=8.73958in,height=\textheight,keepaspectratio]{/assets/docs/t_mhLmSe89ZV_R--e5hFagAAAAAAAAAA.png}

What we have here is a two-column document on ISO A5 paper with page
numbers on the bottom. The table has six columns and contains a few
customizations to \hyperref[strokes]{stroke} , alignment and spacing.
But the most important part is that the table is wrapped in a call to
the \texttt{\ rotate\ } function with the \texttt{\ reflow\ } argument
being \texttt{\ }{\texttt{\ true\ }}\texttt{\ } . This will make the
table rotate 90 degrees counterclockwise. The reflow argument is needed
so that the table\textquotesingle s rotation affects the layout. If it
was omitted, Typst would lay out the page as if the table was not
rotated ( \texttt{\ }{\texttt{\ true\ }}\texttt{\ } might become the
default in the future).

The example also shows how to produce many columns of the same size: To
the initial \texttt{\ }{\texttt{\ 1fr\ }}\texttt{\ } column, we add an
array with five \texttt{\ }{\texttt{\ auto\ }}\texttt{\ } items that we
create by multiplying an array with one
\texttt{\ }{\texttt{\ auto\ }}\texttt{\ } item by five. Note that arrays
with just one item need a trailing comma to distinguish them from merely
parenthesized expressions.

The second example shows how to rotate the whole page, so that the table
stays upright:

\begin{verbatim}
#set page("a5", numbering: "— 1 —")
#show table.cell.where(y: 0): set text(weight: "bold")

#page(flipped: true)[
  #table(
    columns: (1fr,) + 5 * (auto,),
    inset: (x: 0.6em,),
    stroke: (_, y) => (
      x: 1pt,
      top: if y <= 1 { 1pt } else { 0pt },
      bottom: 1pt,
    ),
    align: (left, right, right, right, right, left),

    table.header(
      [Student Name],
      [Assignment 1], [Assignment 2],
      [Mid-term], [Final Exam],
      [Total Grade],
    ),
    [Jane Smith], [78%], [82%], [75%], [80%], [B],
    [Alex Johnson], [90%], [95%], [94%], [96%], [A+],
    [John Doe], [85%], [90%], [88%], [92%], [A],
    [Maria Garcia], [88%], [84%], [89%], [85%], [B+],
    [Zhang Wei], [93%], [89%], [90%], [91%], [A-],
    [Marina Musterfrau], [96%], [91%], [74%], [69%], [B-],
  )

  #pad(x: 15%, top: 1.5em)[
    = Winter 2023/24 results
    #lorem(80)
  ]
]
\end{verbatim}

\includegraphics[width=12.40625in,height=\textheight,keepaspectratio]{/assets/docs/9Goo6xTF0vprclhyFSb4zwAAAAAAAAAA.png}

Here, we take the same table and the other content we want to set with
it and put it into a call to the
\href{/docs/reference/layout/page/}{\texttt{\ page\ }} function while
supplying \texttt{\ }{\texttt{\ true\ }}\texttt{\ } to the
\texttt{\ flipped\ } argument. This will instruct Typst to create new
pages with width and height swapped and place the contents of the
function call onto a new page. Notice how the page number is also on the
long edge of the paper now. At the bottom of the page, we use the
\href{/docs/reference/layout/pad/}{\texttt{\ pad\ }} function to
constrain the width of the paragraph to achieve a nice and legible line
length.

\subsection{How to break a table across
pages?}\label{table-across-pages}

It is best to contain a table on a single page. However, some tables
just have many rows, so breaking them across pages becomes unavoidable.
Fortunately, Typst supports breaking tables across pages out of the box.
If you are using the
\href{/docs/reference/model/table/\#definitions-header}{\texttt{\ table.header\ }}
and
\href{/docs/reference/model/table/\#definitions-footer}{\texttt{\ table.footer\ }}
functions, their contents will be repeated on each page as the first and
last rows, respectively. If you want to disable this behavior, you can
set \texttt{\ repeat\ } to \texttt{\ }{\texttt{\ false\ }}\texttt{\ } on
either of them.

If you have placed your table inside of a
\href{/docs/reference/model/figure/}{figure} , it becomes unable to
break across pages by default. However, you can change this behavior.
Let\textquotesingle s take a look:

\begin{verbatim}
#set page(width: 9cm, height: 6cm)
#show table.cell.where(y: 0): set text(weight: "bold")
#show figure: set block(breakable: true)

#figure(
  caption: [Training regimen for Marathon],
  table(
    columns: 3,
    fill: (_, y) => if y == 0 { gray.lighten(75%) },

    table.header[Week][Distance (km)][Time (hh:mm:ss)],
    [1], [5],  [00:30:00],
    [2], [7],  [00:45:00],
    [3], [10], [01:00:00],
    [4], [12], [01:10:00],
    [5], [15], [01:25:00],
    [6], [18], [01:40:00],
    [7], [20], [01:50:00],
    [8], [22], [02:00:00],
    [...], [...], [...],
    table.footer[_Goal_][_42.195_][_02:45:00_],
  )
)
\end{verbatim}

\includegraphics[width=5.3125in,height=\textheight,keepaspectratio]{/assets/docs/LN81XNtn32FKL9Eh3VVjowAAAAAAAAAA.png}
\includegraphics[width=5.3125in,height=\textheight,keepaspectratio]{/assets/docs/LN81XNtn32FKL9Eh3VVjowAAAAAAAAAB.png}

A figure automatically produces a
\href{/docs/reference/layout/block/}{block} which cannot break by
default. However, we can reconfigure the block of the figure using a
show rule to make it \texttt{\ breakable\ } . Now, the figure spans
multiple pages with the headers and footers repeating.

\subsection{How to import data into a table?}\label{importing-data}

Often, you need to put data that you obtained elsewhere into a table.
Sometimes, this is from Microsoft Excel or Google Sheets, sometimes it
is from a dataset on the web or from your experiment. Fortunately, Typst
can load many \href{/docs/reference/data-loading/}{common file formats}
, so you can use scripting to include their data in a table.

The most common file format for tabular data is CSV. You can obtain a
CSV file from Excel by choosing "Save as" in the \emph{File} menu and
choosing the file format "CSV UTF-8 (Comma-delimited) (.csv)". Save the
file and, if you are using the web app, upload it to your project.

In our case, we will be building a table about Moore\textquotesingle s
Law. For this purpose, we are using a statistic with
\href{https://ourworldindata.org/grapher/transistors-per-microprocessor}{how
many transistors the average microprocessor consists of per year from
Our World in Data} . Let\textquotesingle s start by pressing the
"Download" button to get a CSV file with the raw data.

Be sure to move the file to your project or somewhere Typst can see it,
if you are using the CLI. Once you did that, we can open the file to see
how it is structured:

\begin{verbatim}
Entity,Code,Year,Transistors per microprocessor
World,OWID_WRL,1971,2308.2417
World,OWID_WRL,1972,3554.5222
World,OWID_WRL,1974,6097.5625
\end{verbatim}

The file starts with a header and contains four columns: Entity (which
is to whom the metric applies), Code, the year, and the number of
transistors per microprocessor. Only the last two columns change between
each row, so we can disregard "Entity" and "Code".

First, let\textquotesingle s start by loading this file with the
\href{/docs/reference/data-loading/csv/}{\texttt{\ csv\ }} function. It
accepts the file name of the file we want to load as a string argument:

\begin{verbatim}
#let moore = csv("moore.csv")
\end{verbatim}

We have loaded our file (assuming we named it \texttt{\ moore.csv\ } )
and \href{/docs/reference/scripting/\#bindings}{bound it} to the new
variable \texttt{\ moore\ } . This will not produce any output, so
there\textquotesingle s nothing to see yet. If we want to examine what
Typst loaded, we can either hover the name of the variable in the web
app or print some items from the array:

\begin{verbatim}
#let moore = csv("moore.csv")

#moore.slice(0, 3)
\end{verbatim}

\includegraphics[width=5in,height=\textheight,keepaspectratio]{/assets/docs/4Wrq2xnRY1_FUBMwiwPizQAAAAAAAAAA.png}

With the arguments
\texttt{\ }{\texttt{\ (\ }}\texttt{\ }{\texttt{\ 0\ }}\texttt{\ }{\texttt{\ ,\ }}\texttt{\ }{\texttt{\ 3\ }}\texttt{\ }{\texttt{\ )\ }}\texttt{\ }
, the
\href{/docs/reference/foundations/array/\#definitions-slice}{\texttt{\ slice\ }}
method returns the first three items in the array (with the indices 0,
1, and 2). We can see that each row is its own array with one item per
cell.

Now, let\textquotesingle s write a loop that will transform this data
into an array of cells that we can use with the table function.

\begin{verbatim}
#let moore = csv("moore.csv")

#table(
  columns: 2,
  ..for (.., year, count) in moore {
    (year, count)
  }
)
\end{verbatim}

\includegraphics[width=5in,height=\textheight,keepaspectratio]{/assets/docs/1JHLF5egGPQD3C2V0wZONgAAAAAAAAAA.png}

The example above uses a for loop that iterates over the rows in our CSV
file and returns an array for each iteration. We use the for
loop\textquotesingle s
\href{/docs/reference/scripting/\#bindings}{destructuring} capability to
discard all but the last two items of each row. We then create a new
array with just these two. Because Typst will concatenate the array
results of all the loop iterations, we get a one-dimensional array in
which the year column and the number of transistors alternate. We can
then insert the array as cells. For this we use the
\href{/docs/reference/foundations/arguments/\#spreading}{spread
operator} ( \texttt{\ ..\ } ). By prefixing an array, or, in our case an
expression that yields an array, with two dots, we tell Typst that the
array\textquotesingle s items should be used as positional arguments.

Alternatively, we can also use the
\href{/docs/reference/foundations/array/\#definitions-map}{\texttt{\ map\ }}
,
\href{/docs/reference/foundations/array/\#definitions-slice}{\texttt{\ slice\ }}
, and
\href{/docs/reference/foundations/array/\#definitions-flatten}{\texttt{\ flatten\ }}
array methods to write this in a more functional style:

\begin{verbatim}
#let moore = csv("moore.csv")

#table(
   columns: moore.first().len(),
   ..moore.map(m => m.slice(2)).flatten(),
)
\end{verbatim}

This example renders the same as the previous one, but first uses the
\texttt{\ map\ } function to change each row of the data. We pass a
function to map that gets run on each row of the CSV and returns a new
value to replace that row with. We use it to discard the first two
columns with \texttt{\ slice\ } . Then, we spread the data into the
\texttt{\ table\ } function. However, we need to pass a one-dimensional
array and \texttt{\ moore\ } \textquotesingle s value is two-dimensional
(that means that each of its row values contains an array with the cell
data). That\textquotesingle s why we call \texttt{\ flatten\ } which
converts it to a one-dimensional array. We also extract the number of
columns from the data itself.

Now that we have nice code for our table, we should try to also make the
table itself nice! The transistor counts go from millions in 1995 to
trillions in 2021 and changes are difficult to see with so many digits.
We could try to present our data logarithmically to make it more
digestible:

\begin{verbatim}
#let moore = csv("moore.csv")
#let moore-log = moore.slice(1).map(m => {
  let (.., year, count) = m
  let log = calc.log(float(count))
  let rounded = str(calc.round(log, digits: 2))
  (year, rounded)
})

#show table.cell.where(x: 0): strong

#table(
   columns: moore-log.first().len(),
   align: right,
   fill: (_, y) => if calc.odd(y) { rgb("D7D9E0") },
   stroke: none,

   table.header[Year][Transistor count ($log_10$)],
   table.hline(stroke: rgb("4D4C5B")),
   ..moore-log.flatten(),
)
\end{verbatim}

\includegraphics[width=5in,height=\textheight,keepaspectratio]{/assets/docs/ZOXJwHX6lPP_GLnz4eAxbQAAAAAAAAAA.png}

In this example, we first drop the header row from the data since we are
adding our own. Then, we discard all but the last two columns as above.
We do this by \href{/docs/reference/scripting/\#bindings}{destructuring}
the array \texttt{\ m\ } , discarding all but the two last items. We
then convert the string in \texttt{\ count\ } to a floating point
number, calculate its logarithm and store it in the variable
\texttt{\ log\ } . Finally, we round it to two digits, convert it to a
string, and store it in the variable \texttt{\ rounded\ } . Then, we
return an array with \texttt{\ year\ } and \texttt{\ rounded\ } that
replaces the original row. In our table, we have added our custom header
that tells the reader that we\textquotesingle ve applied a logarithm to
the values. Then, we spread the flattened data as above.

We also styled the table with \hyperref[fills]{stripes} , a
\hyperref[individual-lines]{horizontal line} below the first row,
\hyperref[alignment]{aligned} everything to the right, and emboldened
the first column. Click on the links to go to the relevant guide
sections and see how it\textquotesingle s done!

\subsection{What if I need the table function for something that
isn\textquotesingle t a table?}\label{table-and-grid}

Tabular layouts of content can be useful not only for matrices of
closely related data, like shown in the examples throughout this guide,
but also for presentational purposes. Typst differentiates between grids
that are for layout and presentational purposes only and tables, in
which the arrangement of the cells itself conveys information.

To make this difference clear to other software and allow templates to
heavily style tables, Typst has two functions for grid and table layout:

\begin{itemize}
\tightlist
\item
  The \href{/docs/reference/model/table/}{\texttt{\ table\ }} function
  explained throughout this guide which is intended for tabular data.
\item
  The \href{/docs/reference/layout/grid/}{\texttt{\ grid\ }} function
  which is intended for presentational purposes and page layout.
\end{itemize}

Both elements work the same way and have the same arguments. You can
apply everything you have learned about tables in this guide to grids.
There are only three differences:

\begin{itemize}
\tightlist
\item
  You\textquotesingle ll need to use the
  \href{/docs/reference/layout/grid/\#definitions-cell}{\texttt{\ grid.cell\ }}
  ,
  \href{/docs/reference/layout/grid/\#definitions-vline}{\texttt{\ grid.vline\ }}
  , and
  \href{/docs/reference/layout/grid/\#definitions-hline}{\texttt{\ grid.hline\ }}
  elements instead of
  \href{/docs/reference/model/table/\#definitions-cell}{\texttt{\ table.cell\ }}
  ,
  \href{/docs/reference/model/table/\#definitions-vline}{\texttt{\ table.vline\ }}
  , and
  \href{/docs/reference/model/table/\#definitions-hline}{\texttt{\ table.hline\ }}
  .
\item
  The grid has different defaults: It draws no strokes by default and
  has no spacing ( \texttt{\ inset\ } ) inside of its cells.
\item
  Elements like \texttt{\ figure\ } do not react to grids since they are
  supposed to have no semantical bearing on the document structure.
\end{itemize}

\href{/docs/guides/page-setup-guide/}{\pandocbounded{\includesvg[keepaspectratio]{/assets/icons/16-arrow-right.svg}}}

{ Page setup guide } { Previous page }

\href{/docs/changelog/}{\pandocbounded{\includesvg[keepaspectratio]{/assets/icons/16-arrow-right.svg}}}

{ Changelog } { Next page }

\textbf{On this page}

\begin{itemize}
\tightlist
\item
  \hyperref[basic-tables]{Basic Tables}
\item
  \hyperref[column-sizes]{Column Sizes}
\item
  \hyperref[captions-and-references]{Captions And References}
\item
  \hyperref[fills]{Fills}

  \begin{itemize}
  \tightlist
  \item
    \hyperref[fill-override]{Fill Override}
  \end{itemize}
\item
  \hyperref[strokes]{Strokes}

  \begin{itemize}
  \tightlist
  \item
    \hyperref[individual-lines]{Individual Lines}
  \item
    \hyperref[stroke-override]{Stroke Override}
  \item
    \hyperref[stroke-functions]{Stroke Functions}
  \item
    \hyperref[double-stroke]{Double Stroke}
  \end{itemize}
\item
  \hyperref[alignment]{Alignment}
\item
  \hyperref[merge-cells]{Merge Cells}
\item
  \hyperref[rotate-table]{Rotate Table}
\item
  \hyperref[table-across-pages]{Table Across Pages}
\item
  \hyperref[importing-data]{Importing Data}
\item
  \hyperref[table-and-grid]{Table And Grid}
\end{itemize}

\begin{itemize}
\tightlist
\item
  \href{/}{Home}
\item
  \href{/pricing/}{Pricing}
\item
  \href{/docs/}{Documentation}
\item
  \href{/universe/}{Universe}
\item
  \href{/about/}{About Us}
\item
  \href{/contact/}{Contact Us}
\item
  \href{/privacy/}{Privacy}
\item
  \href{https://typst.app/terms}{Terms and Conditions}
\item
  \href{/legal/}{Legal (Impressum)}
\end{itemize}

\begin{itemize}
\tightlist
\item
  \href{https://forum.typst.app}{Forum}
\item
  \href{/tools/}{Tools}
\item
  \href{/blog/}{Blog}
\item
  \href{https://github.com/typst/}{GitHub}
\item
  \href{https://discord.gg/2uDybryKPe}{Discord}
\item
  \href{https://mastodon.social/@typst}{Mastodon}
\item
  \href{https://bsky.app/profile/typst.app}{Bluesky}
\item
  \href{https://www.linkedin.com/company/typst/}{LinkedIn}
\item
  \href{https://instagram.com/typstapp/}{Instagram}
\end{itemize}

Made in Berlin
